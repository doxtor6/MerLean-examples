% In this file you should put all LaTeX macros and settings to be used both by
% the pdf version and the web version.
% This should be most of your macros.

% The theorem-like environments defined below are those that appear by default
% in the dependency graph.
\newtheorem{theorem}{Theorem}[chapter]
\newtheorem{proposition}[theorem]{Proposition}
\newtheorem{lemma}[theorem]{Lemma}
\newtheorem{corollary}[theorem]{Corollary}
\newtheorem{example}[theorem]{Example}

\theoremstyle{definition}
\newtheorem{definition}[theorem]{Definition}
\newtheorem{remark}[theorem]{Remark}
\newtheorem{axiom}[theorem]{Axiom}

% Math symbols
\newcommand{\Z}{\mathbb{Z}}
\newcommand{\F}{\mathbb{F}}
\newcommand{\C}{\mathbb{C}}
\newcommand{\N}{\mathbb{N}}
\newcommand{\R}{\mathbb{R}}
\newcommand{\Q}{\mathbb{Q}}

% Identity operator
\usepackage{dsfont}
\newcommand{\id}{\mathds{1}}

% Bra-ket notation
\usepackage{braket}

% Common operators
\newcommand{\supp}{\operatorname{supp}}
\newcommand{\im}{\operatorname{im}}
\newcommand{\tr}{\operatorname{tr}}
\newcommand{\diag}{\operatorname{diag}}
\newcommand{\rank}{\operatorname{rank}}
\newcommand{\card}[1]{\left|#1\right|}

% Graph theory
\newcommand{\boundary}{\partial}
\newcommand{\coboundary}{\delta}
\newcommand{\cyclerank}{\operatorname{cycle\_rank}}
\newcommand{\cheeger}{h}

% Pauli operators (inline notation)
\newcommand{\PauliI}{I}
\newcommand{\PauliX}{X}
\newcommand{\PauliY}{Y}
\newcommand{\PauliZ}{Z}

% Support notation
\newcommand{\xsupp}{\mathcal{S}_X}
\newcommand{\zsupp}{\mathcal{S}_Z}
