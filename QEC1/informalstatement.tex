\documentclass[11pt]{article}

\usepackage[utf8]{inputenc}
\usepackage{amsmath,amssymb,amsthm}
\usepackage{mathtools}
\usepackage{braket}
\usepackage{dsfont}
\usepackage{enumitem}
\usepackage[margin=1in]{geometry}
\usepackage{hyperref}
\usepackage{xcolor}

% Axiom warning style
\newcommand{\axiomwarning}[1]{\textcolor{red}{\textbf{[Axiom: #1]}}}

% Unnumbered theorem environments (numbering matches Lean files)
\theoremstyle{plain}
\newtheorem*{theorem}{Theorem}
\newtheorem*{lemma}{Lemma}
\newtheorem*{proposition}{Proposition}
\newtheorem*{corollary}{Corollary}

\theoremstyle{definition}
\newtheorem*{definition}{Definition}
\newtheorem*{example}{Example}
\newtheorem*{axiomenv}{\textcolor{red}{Axiom (Unproven)}}
\newenvironment{axiom}[1][]{\begin{axiomenv}[#1]\color{red}}{\end{axiomenv}}

\theoremstyle{remark}
\newtheorem*{remark}{Remark}

% Common math commands
\newcommand{\Z}{\mathbb{Z}}
\newcommand{\F}{\mathbb{F}}
\newcommand{\C}{\mathbb{C}}
\newcommand{\R}{\mathbb{R}}
\newcommand{\N}{\mathbb{N}}
\newcommand{\Q}{\mathbb{Q}}
\newcommand{\id}{\mathds{1}}

% Common operators
\DeclareMathOperator{\supp}{supp}
\DeclareMathOperator{\im}{im}
\DeclareMathOperator{\tr}{tr}
\DeclareMathOperator{\rank}{rank}
\DeclareMathOperator{\ker}{ker}
\DeclareMathOperator{\Hom}{Hom}

\title{QEC1\\[0.5em]\large A Mathematical Exposition}
\author{Auto-generated from Lean 4 Formalization}
\date{\today}

\begin{document}
\maketitle

\begin{abstract}
This document presents the mathematical content from the \texttt{QEC1} library.
The material has been translated from a verified Lean 4 formalization,
ensuring mathematical rigor while maintaining readability.

\vspace{0.5em}
\noindent\textbf{\color{red}WARNING: Unproven Axioms}

\noindent This formalization uses the following unproven axiom(s): \texttt{capturing}, \texttt{that}.

\noindent These axioms were introduced because the full proofs were not completed.
The mathematical validity of results depends on these assumptions.

\end{abstract}

\tableofcontents
\newpage

\section{Mathematical Content}

\subsection{Remark 1: StabilizerCodeConvention}

Quantum error correction relies on stabilizer codes, which provide a systematic framework for protecting quantum information against noise. In this setting, we work with an $[[n,k,d]]$ quantum low-density parity-check (qLDPC) stabilizer code that encodes $k$ logical qubits into $n$ physical qubits with minimum distance $d$. A key insight is that logical operators can be represented in a canonical form through an appropriate choice of single-qubit bases. This convention simplifies both theoretical analysis and practical implementation of quantum error correction protocols.

The mathematical foundation rests on the Pauli group and its commutation relations. By exploiting the structure of stabilizer codes and carefully choosing measurement bases, we can ensure that logical operators take a particularly simple form as products of Pauli-$X$ matrices. This representation not only clarifies the geometric structure of the code space but also facilitates efficient classical processing of syndrome information.

\begin{remark}[Remark 1: Stabilizer Code Convention]
Throughout this work, we consider an $[[n,k,d]]$ quantum low-density parity-check (qLDPC) stabilizer code on $n$ physical qubits, encoding $k$ logical qubits with distance $d$. The code is specified by a set of stabilizer checks $\{s_i\}$. A logical operator $L$ is a Pauli operator that commutes with all stabilizers but is not itself a stabilizer. By choosing an appropriate single-qubit basis for each physical qubit, we ensure that the logical operator $L$ being measured is a product of Pauli-$X$ matrices:
\[
L = \prod_{v \in \operatorname{supp}(L)} X_v,
\]
where $\operatorname{supp}(L)$ denotes the set of qubits on which $L$ acts non-trivially.
\end{remark}

This convention has profound implications for both the theoretical understanding and practical implementation of quantum error correction. The ability to express logical operators purely in terms of $X$-gates means that logical measurements reduce to computational basis measurements after appropriate basis rotations. Furthermore, this representation directly connects the support structure of logical operators to the geometric properties of the underlying code, facilitating the analysis of code distance and the design of efficient decoding algorithms.

\vspace{1em}

\subsection{Remark 2: GraphConvention}

The graph convention for gauging protocols establishes a systematic framework for implementing quantum error correction measurements using auxiliary qubits arranged on a connected graph structure. This approach enables the measurement of logical operators while maintaining the stability of quantum error-correcting codes. The key insight is that by strategically placing auxiliary qubits initialized in specific quantum states, we can perform non-destructive measurements of logical observables.

In the gauging measurement protocol, we associate a connected graph $G = (V_G, E_G)$ with the logical operator $L$ being measured. The vertices of $G$ correspond to qubits, with those in the support of $L$ representing the original code qubits, while vertices outside this support serve as "dummy" locations that do not affect the measurement outcome. Each edge receives an auxiliary gauge qubit initialized in $\ket{0}$, and dummy vertices receive auxiliary qubits initialized in $\ket{+}$. This construction ensures that dummy vertices contribute deterministically to measurements, since $X$-basis measurements on $\ket{+}$ states always yield the outcome $+1$.

\begin{remark}[Remark 2: Graph Convention]
The graph convention for gauging measurement protocols specifies how a connected graph $G = (V_G, E_G)$ relates to a logical operator $L$ being measured. Given the support $\operatorname{supp}(L)$ of the logical operator, we identify:
\begin{itemize}
\item Vertices in $\operatorname{supp}(L)$ as \textbf{support vertices} (original code qubits)
\item Vertices not in $\operatorname{supp}(L)$ as \textbf{dummy vertices} (auxiliary qubits in $\ket{+}$)
\item Each edge as receiving an \textbf{auxiliary gauge qubit} (initialized in $\ket{0}$)
\end{itemize}

The fundamental property is that dummy vertices do not affect the measurement outcome, since measuring $X$ on $\ket{+}$ deterministically returns $+1$.

We classify qubits into three types:
\begin{align}
\text{QubitType} ::= \text{LogicalSupport} \mid \text{EdgeQubit} \mid \text{DummyQubit}
\end{align}

with initial state assignments:
\begin{align}
\text{LogicalSupport} &\mapsto \text{encoded} \\
\text{EdgeQubit} &\mapsto \ket{0} \\
\text{DummyQubit} &\mapsto \ket{+}
\end{align}

For measurement outcomes in the $X$-basis, we have:
\[
X\ket{+} = \ket{+} \quad \text{(eigenvalue } +1\text{)}
\]

The gauging graph convention ensures that every vertex $v \in V_G$ is classified as either a support vertex or dummy vertex (but not both), and the total number of qubits involved is:
\[
\text{totalQubits}(G) = |\operatorname{supp}(L)| + |E_G| + |\text{dummyVertices}(G)|
\]

The effective logical support when using dummy vertices becomes the entire vertex set $V_G$, corresponding to gauging the extended operator $L \cdot \prod_{v \in \text{dummy}} X_v$.
\end{remark}

This graph convention provides the mathematical foundation for implementing fault-tolerant logical measurements in quantum error-correcting codes. By introducing auxiliary qubits with predetermined measurement outcomes, the protocol maintains code stability while enabling the extraction of logical information. The distinction between support and dummy vertices is crucial for understanding which measurements carry logical information versus which serve purely as auxiliary components in the gauging construction.

\vspace{1em}

\subsection{Remark 3: BinaryVectorNotation}

In set theory and graph theory applications, it is often convenient to represent subsets using vectors over finite fields. This encoding transforms set-theoretic operations into linear algebra, enabling the use of vector space methods to solve combinatorial problems.

The key insight is to represent each subset $S$ of a finite universe as its characteristic vector over $\mathbb{F}_2 = \mathbb{Z}/2\mathbb{Z}$, where entry $i$ equals $1$ if element $i$ belongs to $S$ and $0$ otherwise. Under this representation, the symmetric difference of sets corresponds to vector addition, making the collection of all subsets into a vector space over $\mathbb{F}_2$. This correspondence is fundamental in algebraic graph theory, coding theory, and topological data analysis.

\begin{remark}[Remark 3: Binary Vector Notation]
Throughout this work, we abuse notation by identifying a subset of vertices, edges, or cycles with its characteristic binary vector over $\mathbb{F}_2 = \mathbb{Z}/2\mathbb{Z}$. For a set $S$ of vertices/edges/cycles, the corresponding binary vector has a $1$ in position $i$ if and only if element $i$ belongs to $S$. Addition of binary vectors corresponds to symmetric difference of sets. This identification allows us to use linear algebra over $\mathbb{F}_2$ to reason about graph-theoretic properties.
\end{remark}

This notation enables powerful algebraic techniques. For example, the cycle space of a graph becomes a vector subspace, and fundamental cycles correspond to a basis. Linear independence of characteristic vectors translates to set-theoretic properties, while the rank of matrices built from these vectors yields important graph invariants such as the circuit rank.

\vspace{1em}

\subsection{Remark 4: ZTypeSupportConvention}

The concept of Z-type and X-type supports for Pauli operators is fundamental in quantum error correction, particularly for stabilizer codes. This classification allows us to decompose any Pauli operator into commuting X and Z parts, which is essential for understanding the commutation relations between Pauli operators and logical operators. The main insight is that commutation behavior is determined by the parity of overlaps between different types of supports.

\begin{remark}[Remark 4: Z-Type Support Convention]
For a Pauli operator $P$ acting on $n$ qubits, we define:
\begin{itemize}
\item The \textbf{Z-type support} $\mathcal{S}_Z(P)$ is the set of qubits on which $P$ acts via $Y$ or $Z$.
\item The \textbf{X-type support} $\mathcal{S}_X(P)$ is the set of qubits on which $P$ acts via $X$ or $Y$.
\end{itemize}

Every Pauli operator can be uniquely decomposed as
\[
P = i^{\sigma} \prod_{v \in \mathcal{S}_X(P)} X_v \prod_{v \in \mathcal{S}_Z(P)} Z_v
\]
for some phase $\sigma \in \{0,1,2,3\}$. The key commutation property is: if $P$ commutes with an X-type logical operator $L = \prod_{v \in \mathrm{supp}(L)} X_v$, then $|\mathcal{S}_Z(P) \cap \mathrm{supp}(L)| \equiv 0 \pmod{2}$.
\end{remark}

This remark establishes the foundation for analyzing commutation relations in stabilizer codes. The parity condition arises because X and Z operators anticommute, while Y anticommutes with both X and Z. Therefore, commutation between a general Pauli operator and a pure X-type operator depends on having an even number of Z-type components in the overlap region.

\vspace{1em}

\subsection{Remark 5: CheegerConstantDefinition}

The Cheeger constant, named after Jeff Cheeger, is a fundamental measure in spectral graph theory that quantifies how well-connected a graph is. It originated from Cheeger's work on Riemannian manifolds, where it measures the "bottleneck" in a manifold's geometry. In the discrete setting of graphs, the Cheeger constant captures the notion of edge expansion: roughly speaking, it measures the minimum ratio of "boundary edges" to "interior vertices" over all reasonably-sized subsets of vertices. This quantity is crucial for understanding random walks, mixing times, and the connectivity properties of networks.

Graphs with large Cheeger constants are called expanders, and they have remarkable properties that make them valuable in computer science, coding theory, and mathematics. In quantum error correction, as considered in this work, expander graphs help preserve the distance properties of codes under certain deformation procedures.

\begin{definition}[Definition: Edge Crosses Boundary]
Given a vertex set $S \subseteq V$ and an edge $e = \{u, v\} \in \mathrm{Sym}_2(V)$, we say that $e$ \emph{crosses the boundary} of $S$ if exactly one of its endpoints lies in $S$, i.e., $(u \in S \land v \notin S) \lor (u \notin S \land v \in S)$.
\end{definition}

\begin{definition}[Definition: Edge Boundary]
The \textbf{edge boundary} $\partial S$ of a vertex set $S$ in a simple graph $G = (V, E)$ is the set of edges with exactly one endpoint in $S$:
\[
\partial S = \{ e \in E(G) \mid e \text{ crosses the boundary of } S \}.
\]
\end{definition}

\begin{lemma}[Lemma: Membership in Edge Boundary]
For a simple graph $G = (V, E)$, a vertex set $S \subseteq V$, and vertices $u, v \in V$,
\[
\{u, v\} \in \partial S \iff G.\mathrm{Adj}(u,v) \land \bigl((u \in S \land v \notin S) \lor (u \notin S \land v \in S)\bigr).
\]
\end{lemma}

\begin{proof}
This follows directly by unfolding the definitions. An edge $\{u,v\}$ belongs to the edge boundary $\partial S$ if and only if it is an edge in the graph (i.e., $G.\mathrm{Adj}(u,v)$ holds) and it crosses the boundary of $S$, which by definition means exactly one endpoint lies in $S$.
\end{proof}

\begin{remark}[Remark 5: Cheeger Constant Definition]
The \textbf{Cheeger constant} of a simple graph $G = (V, E)$ is defined as
\[
h(G) = \inf_{\substack{S \subseteq V \\ S \neq \emptyset \\ 2|S| \leq |V|}} \frac{|\partial S|}{|S|},
\]
where $\partial S$ is the edge boundary of $S$. The infimum is taken over all nonempty subsets $S$ of $V$ satisfying $2|S| \leq |V|$.

A simple graph $G$ is an \textbf{expander} if there exists a constant $c > 0$ such that $h(G) \geq c$. A simple graph $G$ is a \textbf{strong expander} if $h(G) \geq 1$, which is the condition required in this work to preserve the distance of the original code under deformation.
\end{remark}

The constraint $2|S| \leq |V|$ ensures we only consider subsets containing at most half the vertices, which is necessary because the edge boundary is symmetric: $\partial S = \partial(V \setminus S)$. Without this constraint, we could always choose the larger of $S$ or its complement, making the infimum trivial. The condition $h(G) \geq 1$ for strong expanders is particularly stringent and ensures excellent connectivity properties.

\begin{lemma}[Lemma: Edge Boundary of Empty and Full Sets]
For any simple graph $G = (V, E)$:
\begin{enumerate}
\item The edge boundary of the empty set is empty: $\partial \emptyset = \emptyset$.
\item The edge boundary of the full vertex set is empty: $\partial V = \emptyset$.
\end{enumerate}
\end{lemma}

\begin{proof}
For (1): No edge can cross the boundary of the empty set since every edge has both endpoints in $V$, and neither endpoint can be in $\emptyset$.

For (2): No edge can cross the boundary of the full vertex set $V$ since every edge has both endpoints in $V$, so there is no vertex outside $V$ for the edge to connect to.
\end{proof}

\begin{lemma}[Lemma: Symmetry of Edge Boundary]
The edge boundary is symmetric under complementation: for any vertex set $S \subseteq V$,
\[
\partial S = \partial(V \setminus S).
\]
\end{lemma}

\begin{proof}
An edge $\{u,v\}$ crosses the boundary of $S$ if and only if exactly one of $u,v$ belongs to $S$. This occurs if and only if exactly one of $u,v$ belongs to $V \setminus S$, since if $u \in S$ and $v \notin S$, then $u \notin (V \setminus S)$ and $v \in (V \setminus S)$. Thus $\{u,v\}$ crosses the boundary of $V \setminus S$ as well.
\end{proof}

\begin{lemma}[Lemma: Strong Expander Implies Expander]
A strong expander is an expander. That is, if $h(G) \geq 1$, then $G$ is an expander (with witness $c = 1$).
\end{lemma}

\begin{proof}
If $h(G) \geq 1$, then taking $c = 1$ gives us $c > 0$ and $h(G) \geq c$, which is precisely the definition of an expander graph.
\end{proof}

\begin{lemma}[Lemma: Cheeger Constant is Nonnegative]
For any simple graph $G$, the Cheeger constant is nonnegative: $h(G) \geq 0$.
\end{lemma}

\begin{proof}
Since $|\partial S|$ and $|S|$ are both natural numbers, each ratio $\frac{|\partial S|}{|S|}$ in the definition of the Cheeger constant is nonnegative. Therefore, the infimum of nonnegative quantities is nonnegative, giving $h(G) \geq 0$.
\end{proof}

\vspace{1em}

\subsection{Remark 6: CircuitImplementation}

The gauging procedure described in previous sections can be implemented as a concrete quantum circuit, providing a constructive realization of the abstract gauging transformation. This circuit construction is particularly elegant because it requires no additional ancilla qubits beyond the edge qubits that naturally arise from the graph structure, making it efficient for practical quantum implementations.

The circuit operates in five distinct phases: edge initialization, entangling operations, vertex measurements, a second round of entangling operations, and final edge measurements. The two entangling phases use identical gate sequences, which simplifies both the theoretical analysis and practical implementation. This systematic approach transforms the initial product state into the desired gauged state while maintaining the essential topological properties of the original quantum code.

\begin{remark}[Remark 6: Circuit Implementation]
The gauging procedure can be implemented by a quantum circuit with no additional ancilla qubits beyond the edge qubits. After initializing the edge qubits in $\ket{0}$, one performs the entangling circuit $\prod_v \prod_{e \ni v} \mathrm{CX}_{v \to e}$, where $\mathrm{CX}_{v \to e}$ is a controlled-$X$ gate with control qubit $v$ and target qubit $e$. Next, one projectively measures $X_v$ on all vertices $v \in G$ and keeps the post-measurement state. Then one repeats the same entangling circuit $\prod_v \prod_{e \ni v} \mathrm{CX}_{v \to e}$. Finally, one measures $Z_e$ on all edge qubits and discards them.

The circuit consists of exactly five phases with the strict ordering:
\[
\texttt{InitializeEdges} < \texttt{FirstEntangling} < \texttt{MeasureXVertices} < \texttt{SecondEntangling} < \texttt{MeasureZEdges}
\]

The total number of controlled-$X$ gates required is $4|E|$, where $|E|$ is the number of edges in the graph, since each entangling circuit uses $2|E|$ gates (one for each vertex-edge incidence) and we apply the entangling circuit twice.

The ancilla qubits are exactly the edge qubits $\{\mathrm{edge}(e) \mid e \in E_G\}$, confirming that no additional auxiliary qubits beyond those naturally associated with the graph edges are needed for the implementation.
\end{remark}

This circuit implementation demonstrates the practical feasibility of the gauging procedure. The fact that only edge qubits serve as ancilla means the total qubit overhead scales linearly with the number of edges rather than requiring additional auxiliary qubits. The symmetric structure with identical entangling circuits applied before and after the vertex measurements reflects the underlying mathematical structure of the gauging transformation, where the same graph connectivity patterns appear in both the encoding and decoding phases of the procedure.

\vspace{1em}

\subsection{Definition 1: BoundaryCoboundaryMaps}

In graph theory and algebraic topology, boundary operators capture how higher-dimensional objects (edges, cycles) relate to their lower-dimensional boundaries (vertices, edges). These operators are fundamental in homological algebra and provide the foundation for studying the topological structure of graphs. The boundary and coboundary maps we define here work over the field $\mathbb{Z}_2$, which simplifies calculations by eliminating orientation concerns while preserving essential structural information.

The duality between boundary and coboundary operators reflects a deep principle in mathematics: every linear operator has a transpose that reverses the direction of mappings while preserving inner products. This duality will be crucial for understanding the relationship between cycles and their supporting edge sets.

\begin{definition}[Definition 1: Boundary and Coboundary Maps]
Let $G = (V, E)$ be a simple graph with vertex set $V$, edge set $E$, and a chosen collection $\mathcal{C}$ of cycles. We define the following $\mathbb{Z}_2$-linear maps using binary vector representations:

\textbf{Binary Vector Spaces:}
\begin{itemize}
\item $\mathbb{Z}_2^V := V \to \mathbb{Z}_2$ (binary vectors over vertices)
\item $\mathbb{Z}_2^E := E \to \mathbb{Z}_2$ (binary vectors over edges)  
\item $\mathbb{Z}_2^C := C \to \mathbb{Z}_2$ (binary vectors over cycles)
\end{itemize}

\textbf{Boundary Map $\partial : \mathbb{Z}_2^E \to \mathbb{Z}_2^V$:}
For any edge-vector $f \in \mathbb{Z}_2^E$ and vertex $v \in V$,
\[
\partial(f)(v) = \sum_{e \in \mathrm{incidentEdges}(v)} f(e),
\]
where $\mathrm{incidentEdges}(v) = \{e \in E : v \text{ is an endpoint of } e\}$.

\textbf{Coboundary Map $\delta : \mathbb{Z}_2^V \to \mathbb{Z}_2^E$:}
For any vertex-vector $g \in \mathbb{Z}_2^V$ and edge $e \in E$,
\[
\delta(g)(e) = \sum_{v \in \mathrm{edgeVertices}(e)} g(v),
\]
where $\mathrm{edgeVertices}(e)$ is the set of two endpoints of edge $e$.

\textbf{Second Boundary Map $\partial_2 : \mathbb{Z}_2^C \to \mathbb{Z}_2^E$:}
For any cycle-vector $h \in \mathbb{Z}_2^C$ and edge $e \in E$,
\[
\partial_2(h)(e) = \sum_{c \in \mathrm{cyclesContaining}(e)} h(c),
\]
where $\mathrm{cyclesContaining}(e) = \{c \in C : e \in \mathrm{cycles}(c)\}$.

\textbf{Second Coboundary Map $\delta_2 : \mathbb{Z}_2^E \to \mathbb{Z}_2^C$:}
For any edge-vector $f \in \mathbb{Z}_2^E$ and cycle $c \in C$,
\[
\delta_2(f)(c) = \sum_{e \in \mathrm{cycles}(c)} f(e).
\]

\textbf{Transpose Relations:}
The coboundary maps are transposes of their corresponding boundary maps:
\[
\delta = \partial^T \quad \text{and} \quad \delta_2 = \partial_2^T.
\]
\end{definition}

The boundary map $\partial$ sends an edge-set (represented as a binary vector) to the set of vertices that appear an odd number of times as endpoints. In $\mathbb{Z}_2$, this counts the "boundary vertices" of the edge-set. The coboundary map $\delta$ goes in the reverse direction, sending a vertex-set to the set of edges with exactly one endpoint in the vertex-set.

The second boundary map $\partial_2$ takes a collection of cycles and outputs the set of edges that appear an odd number of times across all cycles. The second coboundary $\delta_2$ sends an edge-set to the collection of cycles that have odd intersection with that edge-set. These operators form the foundation for studying the homology of the graph, where cycles that are boundaries of higher-dimensional objects are considered trivial.

\vspace{1em}

\subsection{Remark 7: ExactnessOfBoundaryCoboundary}

In the study of algebraic graph theory and homological algebra, understanding the relationship between boundary and coboundary operators is fundamental to characterizing cycles and cuts in graphs. When a graph is equipped with a generating set of cycles, these operators form chain and cochain complexes whose exactness properties reveal deep structural information about the graph's topology.

The key insight is that for connected graphs, the kernel of the coboundary map $\delta: \mathbb{F}_2^V \to \mathbb{F}_2^E$ has a remarkably simple structure: it contains only the zero vector and the all-ones vector. This follows from the fact that elements in the kernel must be constant on connected components, and over $\mathbb{F}_2$, constant functions can only take values 0 or 1.

\begin{remark}[Remark 7: Exactness of Boundary and Coboundary Maps]
For graphs with valid cycle generating sets, the boundary maps $\partial_2: \mathbb{F}_2^C \to \mathbb{F}_2^E$ and $\partial: \mathbb{F}_2^E \to \mathbb{F}_2^V$ satisfy the chain complex property $\partial \circ \partial_2 = 0$, meaning $\operatorname{im}(\partial_2) \subseteq \ker(\partial)$. Dually, the coboundary maps $\delta: \mathbb{F}_2^V \to \mathbb{F}_2^E$ and $\delta_2: \mathbb{F}_2^E \to \mathbb{F}_2^C$ satisfy the cochain complex property $\delta_2 \circ \delta = 0$, meaning $\operatorname{im}(\delta) \subseteq \ker(\delta_2)$.

For connected graphs, the kernel $\ker(\delta)$ consists precisely of the constant functions, which over $\mathbb{F}_2$ means $\ker(\delta) = \{0, \mathbf{1}\}$ where $\mathbf{1}$ denotes the all-ones vector. This characterization is crucial for establishing exactness conditions: when cycles generate all closed edge-chains and cuts generate all cocycles, we obtain exact sequences that completely characterize the homology and cohomology of the graph.
\end{remark}

The structure $\ker(\delta) = \{0, \mathbf{1}\}$ for connected graphs reflects the fundamental principle that vertex functions in the coboundary kernel must be constant on connected components. Since every edge connects exactly two vertices, the coboundary $\delta(f)(e) = f(v_1) + f(v_2)$ vanishes precisely when the function values at the endpoints are equal. In a connected graph, this forces global constancy, and over the two-element field $\mathbb{F}_2$, only two constant functions exist.

\vspace{1em}

\subsection{Remark 8: DesiderataForG}

Looking at this Lean formalization, I need to convert "Rem_8: DesiderataForG" into textbook-style LaTeX. This appears to be a remark that presents three desiderata (requirements) for choosing a graph G in quantum error correction gauging procedures.

The study of quantum error correction codes often requires auxiliary structures to implement fault-tolerant operations. In the gauging measurement procedure, a key component is the selection of an appropriate graph $G$ that facilitates the deformation of check operators while preserving essential properties of the code. The choice of this graph is not arbitrary but must satisfy certain criteria to ensure that the resulting gauged code maintains both efficiency and fault tolerance.

When deforming stabilizer checks in quantum codes, three fundamental challenges arise: controlling the weight growth of deformed operators, preserving the code distance, and ensuring that flux operators remain implementable with reasonable resources. These challenges naturally lead to three desiderata for the auxiliary graph $G$.

\begin{remark}[Remark 8: Desiderata for Graph $G$ in Gauging Measurement]
A graph $G$ used in the gauging measurement procedure should satisfy three key properties to ensure optimal performance:

\textbf{Desideratum 1 (Short Paths):} For any collection $\mathcal{Z}$ of $Z$-type support sets and bound $k \in \mathbb{N}$, the graph should satisfy $\text{ShortPaths}(G, \mathcal{Z}, k)$, meaning that for every support set $S \in \mathcal{Z}$ and vertices $u, v \in S$, there exists a walk from $u$ to $v$ of length at most $k$.

\textbf{Desideratum 2 (Sufficient Expansion):} The graph should have Cheeger constant $h(G) \geq 1$, ensuring it is a strong expander.

\textbf{Desideratum 3 (Low-Weight Cycles):} For a weight bound $w \in \mathbb{N}$, every generating cycle $c$ should satisfy $|\text{cycles}(G, c)| \leq w$.

The physical justifications are:
\begin{itemize}
\item Short paths ensure deformed check operators have bounded weight
\item Strong expansion ($h(G) \geq 1$) preserves the code distance after deformation  
\item Low-weight cycles guarantee that flux operators $B_p = \prod_{e \in p} Z_e$ remain implementable
\end{itemize}

When all three desiderata hold simultaneously, the gauging measurement achieves constant qubit overhead while maintaining fault tolerance.
\end{remark}

These desiderata represent a careful balance between topological, algebraic, and geometric constraints. The short paths requirement ensures that the deformation process does not create prohibitively heavy operators, while the expansion condition provides the spectral gap necessary for distance preservation. The cycle weight bound directly controls the complexity of implementing the resulting flux measurements, making the overall procedure practically feasible for fault-tolerant quantum computation.

\vspace{1em}

\subsection{Remark 9: WorstCaseGraphConstruction}

The construction of efficient auxiliary graphs for quantum error correction represents a fundamental challenge in fault-tolerant quantum computation. When measuring a logical operator $L$ in a quantum error-correcting code, we need to construct a graph $G$ that facilitates efficient syndrome extraction while maintaining the code's error-correcting properties. The key insight is that this graph must satisfy three crucial desiderata: short paths between certain paired vertices, sufficient expansion properties, and low-weight generating cycles.

The worst-case construction provides an explicit algorithm for building such graphs with provably optimal overhead. For a logical operator $L$ with weight $W = |\operatorname{supp}(L)|$, this construction achieves $O(W \log^2 W)$ qubit overhead through a carefully orchestrated three-step process. Each step addresses a specific structural requirement, culminating in a graph that simultaneously satisfies all necessary properties for fault-tolerant measurement.

\begin{remark}[Remark 9: Worst-Case Graph Construction]
The worst-case graph construction proceeds in three steps to build a graph $G$ satisfying the desiderata with $O(W \log^2 W)$ qubit overhead, where $W = |\operatorname{supp}(L)|$:

\textbf{Step 1 (Perfect Matching):} For each original code check overlapping the target logical $L$, construct a $\mathbb{Z}_2$-perfect matching of vertices in the $Z$-type support of that check. Add edges to $G$ for each matched pair, ensuring path length exactly 1 between matched vertices.

\textbf{Step 2 (Expansion):} Add edges to $G$ until the Cheeger constant satisfies $h(G) \geq 1$. This is achieved by randomly adding edges while preserving constant degree, or by overlaying an existing constant-degree expander graph.

\textbf{Step 3 (Cycle Sparsification):} Add $R$ additional layers that are copies of the base graph $G_0$ on dummy vertices, connected sequentially back to the original vertices. Within each layer, add edges to cellulate (triangulate) cycles and reduce the cycle-degree. The Freedman--Hastings decongestion lemma establishes that $R = O(\log^2 W)$ layers suffice to achieve constant cycle-degree.
\end{remark}

The construction's efficiency stems from the careful balance between these three requirements. Step 1 ensures that syndrome measurements can be performed with minimal depth, while Step 2 provides the expansion necessary for error correction. Step 3 addresses the topological constraints that arise from the quantum error correction requirements, using sophisticated techniques from homological algebra to control the cycle structure of the resulting graph.

\vspace{1em}

\subsection{Remark 10: Parallelization}

The ability to perform measurements in parallel is crucial for efficient quantum error correction protocols. In realistic quantum computing scenarios, sequential measurements can become a bottleneck, particularly when many logical operators need to be measured. This remark establishes the mathematical framework for when gauging measurements can be parallelized while maintaining the low-density parity-check (LDPC) structure of quantum error correcting codes.

The key insight is that parallelization requires two fundamental conditions: first, the logical operators must not interfere with each other in a way that would create measurement conflicts, and second, the number of operators acting on any single physical qubit must remain bounded to preserve the code's locality properties.

\begin{remark}[Remark 10: Parallelization of Gauging Measurements]
Gauging measurements can be applied to multiple logical operators in parallel if and only if two conditions are satisfied:

\textbf{Condition 1 (Commutativity):} No pair of logical operators acts by different non-trivial Paulis on any common qubit. Formally, for every pair of operators $L_1, L_2$ in the set $\mathcal{L}$ and every qubit position $i$, the Pauli types $(L_1)_i$ and $(L_2)_i$ are compatible.

\textbf{Condition 2 (Bounded Overlap):} At most a constant number $k$ of logical operators share support on any single qubit. This ensures the LDPC property is maintained during code deformation.

When these conditions are met, one can establish a space-time tradeoff: perform $2m-1$ measurements of equivalent logical operators in parallel for $d/m$ rounds (where $m$ divides the code distance $d$), using majority vote to determine the classical outcome.
\end{remark}

This parallelization framework is particularly valuable because it maintains the essential structural properties of quantum LDPC codes while enabling significant speedups in measurement protocols. The majority vote mechanism ensures robustness against measurement errors, while the bounded overlap condition prevents the code from losing its local structure. For codes that support many disjoint logical operators, this enables truly parallel logical gate operations with optimal resource utilization.

\vspace{1em}

\subsection{Definition 2: GaussLawOperators}

In quantum error correction, logical operators must often be measured indirectly due to their support spanning multiple qubits. The Gauss's law construction provides an elegant solution: instead of measuring the logical operator $L = \prod_{v \in V_G} X_v$ directly, we can measure a carefully chosen set of commuting operators whose product equals $L$. This approach, inspired by lattice gauge theories in physics, transforms a single complex measurement into multiple simple ones.

The key insight is that each vertex $v$ contributes an $X_v$ term to the logical operator, but we can "gauge" this by including additional edge terms that ultimately cancel out in the product. This gauging process creates local operators that are easier to measure while preserving the global information about $L$.

\begin{definition}[Definition 2: Gauss's Law Operators]
Given a connected graph $G = (V_G, E_G)$ whose vertices are identified with the qubits in the support of a logical operator $L = \prod_{v \in V_G} X_v$, the \textbf{Gauss's law operators} are the set $\mathcal{A} = \{A_v\}_{v \in V_G}$ where
\[
A_v = X_v \prod_{e \ni v} X_e.
\]
Here $X_v$ is the Pauli-$X$ operator on the vertex qubit $v$, and $X_e$ is the Pauli-$X$ operator on the edge qubit $e$. The product $\prod_{e \ni v}$ is over all edges incident to vertex $v$.

The vertex support of $A_v$ is $\operatorname{vertexSupport}(A_v) = e_v$ (the basis vector at $v$), and the edge support is $\operatorname{edgeSupport}(A_v) = \delta(v)$ (the coboundary of vertex $v$).
\end{definition}

The Gauss's law operators possess three fundamental properties that make them invaluable for quantum error correction. First, each operator is Hermitian with eigenvalues $\pm 1$, making them measurable observables. Second, all operators mutually commute, allowing simultaneous measurement. Third, their product exactly recovers the original logical operator $L$, enabling indirect measurement through the XOR of individual outcomes.

\vspace{1em}

\subsection{Definition 3: FluxOperators}

In quantum error correction schemes based on graphs, stabilizer codes emerge from studying operators that preserve quantum states. After establishing Gauss law operators that measure local constraints at vertices, we turn to a complementary set of operators that emerge from the global topological structure of the graph.

The key insight is that while individual edge operators lose their stabilizing properties after Gauss law measurements, certain products over cycles retain their commutation relations with all Gauss law operators. This leads naturally to the concept of flux operators, which encode the topological degrees of freedom of the quantum error correcting code.

\begin{definition}[Definition 3: Flux Operators]
Given a connected graph $G = (V_G, E_G)$ with a generating set of cycles $\{p\}_C$ (where $C = |E_G| - |V_G| + 1$ is the number of independent cycles by Euler's formula), the \textbf{flux operators} are the set $\mathcal{B} = \{B_p\}_{p \in C}$ where:
\[
B_p = \prod_{e \in p} Z_e
\]
Here $Z_e$ is the Pauli-$Z$ operator on the edge qubit $e$, and the product is over all edges $e$ that belong to cycle $p$.
\end{definition}

The flux operators arise naturally from the measurement process in topological quantum error correction. Initially, each edge qubit is prepared in state $|0\rangle$, making individual $Z_e$ operators stabilizers. After measuring Gauss law operators $A_v$, which involve $X_e$ terms, individual $Z_e$ operators are no longer stabilizers due to the measurement back-action. However, products $B_p = \prod_{e \in p} Z_e$ over cycles remain stabilizers because they commute with all $A_v$ operators. This commutation holds because the number of edges in any cycle $p$ incident to any vertex $v$ is always even (either 0 or 2), ensuring an even number of anticommuting $X_e$--$Z_e$ pairs.

\vspace{1em}

\subsection{Definition 4: DeformedOperator}

The construction of deformed operators is a key technique in quantum error correction for stabilizer codes on graphs. When we have a Pauli operator that commutes with the logical operator, we can systematically modify its edge support while preserving its commutation relations with the stabilizer generators (Gauss law operators). This deformation process allows us to find equivalent representations of logical operators with different edge supports, which is crucial for understanding the structure of the code space.

The fundamental insight is that any Pauli operator commuting with the logical operator $L = \prod_{v \in V} X_v$ must have an even number of $Z$-operators on vertices. This constraint corresponds to the existence of an "edge-path" whose boundary matches the $Z$-support on vertices, enabling the deformation construction.

\begin{definition}[Definition 4: Deformed Operator]
Given a deformable Pauli operator $P$ on graph $G$ and an edge-path $\gamma \subseteq E$, the \textbf{deformed operator} $\tilde{P} = P \cdot \prod_{e \in \gamma} Z_e$ is defined as the deformable Pauli operator with:
\begin{itemize}
\item $S_X^V(\tilde{P}) = S_X^V(P)$ (X-support on vertices unchanged),
\item $S_Z^V(\tilde{P}) = S_Z^V(P)$ (Z-support on vertices unchanged), 
\item $S_X^E(\tilde{P}) = S_X^E(P)$ (X-support on edges unchanged),
\item $S_Z^E(\tilde{P}) = S_Z^E(P) \triangle \gamma$ (Z-support on edges is the symmetric difference with $\gamma$),
\item $\sigma(\tilde{P}) = \sigma(P)$ (phase unchanged).
\end{itemize}

The edge-path $\gamma$ is called a \textbf{valid deforming path} for $Z$-support $S \subseteq V$ if its boundary equals the $Z$-support vector:
\[
\partial \gamma = \mathbf{s}_S,
\]
where $\partial \gamma$ is the boundary map giving $(\partial \gamma)(v) = \sum_{e \in \gamma \text{ incident to } v} 1 \pmod{2}$.

A Pauli operator is \textbf{deformable} if $|S_Z^V| \equiv 0 \pmod{2}$, which is equivalent to commuting with the logical operator $L = \prod_{v \in V} X_v$.
\end{definition}

The key structural property is that deformation preserves commutation relations with the stabilizer generators while allowing flexible manipulation of the edge support. This leads to our main result about the deformed operator's commutation properties.

\vspace{1em}

\subsection{Definition 5: DeformedCheck}

In quantum error-correcting codes, the construction of gauge-fixed stabilizer codes requires careful modification of the original stabilizer checks to ensure compatibility with gauge constraints. The deformed check construction provides a systematic way to transform stabilizer generators while preserving their essential properties and maintaining commutation with gauge operators.

The key insight is that stabilizer checks can be partitioned based on their $Z$-type support on the underlying graph structure. Checks with no $Z$-support on graph vertices require no modification, while those with nontrivial $Z$-support must be deformed by multiplying with additional $Z$-operators along carefully chosen edge paths.

\begin{definition}[Definition 5: Deformed Check]
The \textbf{deformed check} of a stabilizer check $s$ along an edge-path $\gamma$ is defined as
\[
  \tilde{s} := \operatorname{DeformedOperator}(G, s, \gamma) = s \cdot \prod_{e \in \gamma} Z_e,
\]
which modifies the original check by applying $Z$-operators on edges contained in the path $\gamma$.
\end{definition}

The deformed check construction preserves the essential structure of the original stabilizer while modifying only the $Z$-support on edges. Specifically, the deformed check maintains the same $X$-support and $Z$-support on vertices, but its $Z$-support on edges becomes the symmetric difference of the original edge support and the deforming path. This selective modification ensures that the deformed check can satisfy gauge constraints while retaining its role as a stabilizer generator.

The construction naturally leads to a partition of stabilizer checks into two disjoint sets: Set $\mathcal{C}$ contains checks with empty $Z$-support on graph vertices (requiring no deformation), while Set $\mathcal{S}$ contains checks with nontrivial $Z$-support on vertices (requiring genuine deformation through non-empty paths).

\vspace{1em}

\subsection{Definition 6: CycleSparsifiedGraph}

In quantum error correction, constructing codes with good distance properties often requires working with sparse graph structures. The cycle-sparsification construction addresses a fundamental challenge: when a graph contains many cycles that share edges, it becomes difficult to analyze the code distance. By creating multiple layers and carefully distributing cycles across these layers, we can ensure that each edge participates in at most a bounded number of cycles, leading to more manageable combinatorial structures.

The key insight is to take a graph $G$ with a generating set of cycles and create $R+1$ copies (layers) of $G$, where layer 0 contains the original graph and layers $1, \ldots, R$ contain dummy copies. We then connect these layers with inter-layer edges and add triangulation edges within designated layers to cellulate the cycles. The construction is constrained by requiring that every edge participates in at most $c$ cycles from the generating set.

\begin{definition}[Definition 6: Cycle-Sparsified Graph]
A \textbf{cycle-sparsification} of a graph $G$ with cycles $C$ is a structure consisting of:
\begin{itemize}
  \item a base graph $G$ with its chosen generating set of cycles,
  \item a number of additional layers $R$ (total layers $= R + 1$),
  \item a cycle-degree bound $c \in \mathbb{N}$,
  \item a cycle-layer assignment $\mathrm{cycleAssignment} : C \to \mathrm{Fin}(R+1)$,
  \item for each cycle, a list of vertices representing the cycle in order,
  \item the cycle-degree constraint: for every edge $e \in E$, $\mathrm{cycleDegree}(\mathrm{cycles}, e) \leq c$.
\end{itemize}

The vertex set of the sparsified graph consists of \textbf{layered vertices} $(i, v)$ where $i \in \{0, 1, \ldots, R\}$ is the layer index and $v \in V$ is the original vertex. The adjacency relation includes three types of edges:

\textbf{Intra-layer edges}: Connect vertices $(\ell_1, v_1)$ and $(\ell_2, v_2)$ when $\ell_1 = \ell_2$ and $G.\mathrm{Adj}(v_1, v_2)$.

\textbf{Inter-layer edges}: Connect a vertex $(i, v)$ to its copy $(i+1, v)$ in the next layer.

\textbf{Triangulation edges}: Diagonal edges within a layer that triangulate cycles assigned to that layer using a zigzag pattern.
\end{definition}

The construction ensures that the original graph structure is preserved in layer 0, while the additional layers provide the necessary geometric structure for quantum error correction. Each cycle is assigned to exactly one layer, where it receives triangulation edges to create a cellular structure. The inter-layer edges provide connectivity between the original and dummy layers.

A key theoretical result establishes that for graphs satisfying the Freedman-Hastings decongestion bound, the number of required layers is polylogarithmic in the number of vertices. Specifically, $R = O((\log |V|)^2 \cdot \mathrm{maxDegree})$, which ensures that the sparsified graph remains of manageable size while achieving the desired cycle-degree bound. This polylogarithmic scaling is crucial for the practical implementation of quantum LDPC codes with good distance properties.

\vspace{1em}

\subsection{Definition 7: SpaceAndTimeFaults}

In quantum error correction for the gauging measurement procedure, we must account for multiple types of faults that can occur during the physical implementation. These include Pauli errors on individual qubits, measurement errors that flip classical outcomes, and initialization errors. A comprehensive fault model must capture these diverse error mechanisms in a unified framework that respects their algebraic structure.

The foundation of our fault model begins with individual Pauli errors on qubits, which can occur at any location and time during the quantum computation. We then extend this to measurement errors and initialization faults, ultimately combining them into a single spacetime fault structure that forms a group under composition. This algebraic structure is essential for analyzing error propagation and correction capabilities.

\begin{definition}[Definition 7: Spacetime Faults]
A \textbf{spacetime fault} over qubit types $V, E$ and measurement type $M$ is a structure consisting of:
\begin{itemize}
  \item $\texttt{spaceErrors} : \texttt{QubitLoc}(V, E) \to \mathbb{N} \to \texttt{PauliType}$ --- the Pauli error at each (qubit, time) location; identity $I$ means no error.
  \item $\texttt{timeErrors} : M \to \mathbb{N} \to \texttt{Bool}$ --- whether each (measurement, time) has a flipped outcome; $\text{false}$ means no error.
\end{itemize}

The spacetime faults form a group under pointwise composition:
\begin{align}
(f \cdot g).\texttt{spaceErrors}(q, t) &= f.\texttt{spaceErrors}(q, t) \cdot g.\texttt{spaceErrors}(q, t), \\
(f \cdot g).\texttt{timeErrors}(m, t) &= f.\texttt{timeErrors}(m, t) \oplus g.\texttt{timeErrors}(m, t),
\end{align}
where $\cdot$ denotes Pauli multiplication and $\oplus$ denotes XOR.

The identity element has $\texttt{spaceErrors}(q, t) = I$ and $\texttt{timeErrors}(m, t) = \text{false}$ for all locations and times.
\end{definition}

This functional representation allows efficient pointwise operations while capturing the full complexity of fault patterns that can arise during quantum error correction protocols. The group structure ensures that fault compositions behave predictably and that every fault configuration has a well-defined inverse representing the correction operation needed to undo its effect.

\vspace{1em}

\subsection{Definition 8: Detector}

In quantum error correction, the ability to detect and localize errors is fundamental to maintaining quantum information integrity. A detector serves as a diagnostic tool that monitors specific quantum operations and measurements to identify when faults have occurred. The concept builds upon the principle that in fault-free quantum computation, certain combinations of measurement outcomes and initialization states should yield predictable results. When these expectations are violated, we can infer that errors have affected the system.

The mathematical framework for detectors relies on representing measurement outcomes in $\mathbb{Z}/2\mathbb{Z}$, where addition corresponds to multiplication of physical outcomes. This binary representation naturally captures the parity relationships that are essential for quantum error detection schemes.

\begin{definition}[Definition 8: Detector]
A \textbf{detector} for types $V$, $E$, $M$ (with decidable equality) is a structure consisting of:
\begin{itemize}
\item $\mathrm{events} : \mathrm{Finset}(\mathrm{DetectorEvent}(V, E, M))$ --- the set of events in the detector,
\item $\mathrm{expectedParity} \in \mathbb{Z}/2\mathbb{Z}$ --- the expected parity in fault-free execution ($0$ for $+1$, $1$ for $-1$).
\end{itemize}

The detector's \textbf{observed parity} given an outcome assignment $\mathrm{outcomes}$ is:
\[
\mathrm{observedParity}(D, \mathrm{outcomes}) := \mathrm{initParity}(D) + \sum_{e \in D.\mathrm{measEvents}} \mathrm{outcomes}(e.\mathrm{measurement}, e.\mathrm{time})
\]
where $\mathrm{initParity}(D) := \sum_{e \in D.\mathrm{initEvents}} e.\mathrm{expectedParity}$.

A detector $D$ is \textbf{violated} if $\mathrm{observedParity}(D, \mathrm{outcomes}) \neq D.\mathrm{expectedParity}$, and \textbf{satisfied} if $\mathrm{observedParity}(D, \mathrm{outcomes}) = D.\mathrm{expectedParity}$.
\end{definition}

This definition captures the essential property that detectors provide a binary diagnostic: they either report the expected result (indicating no detectable error) or an unexpected result (indicating that some fault has affected the measured system). The parity calculation ensures that the detector's response depends only on the total number of errors modulo 2, which is precisely the information accessible through stabilizer measurements in quantum error correction.

The binary nature of detector responses leads to powerful algebraic properties. Detectors can be combined using symmetric difference operations on their event sets, corresponding to multiplication of their associated parity checks. This enables the construction of complex error detection schemes from simpler building blocks.

\vspace{1em}

\subsection{Definition 9: Syndrome}

In quantum error correction, understanding which detectors are violated by a given fault is crucial for error detection and correction protocols. When a fault occurs in a quantum system, it affects the measurement outcomes of certain detectors, creating a distinctive pattern called the syndrome. This pattern serves as the primary diagnostic tool for identifying and correcting errors in quantum computation and communication.

The syndrome captures the essential information about how a fault manifests in the detector measurements, abstracting away the specific physical details of the error while preserving the information necessary for correction. This abstraction is particularly powerful because it allows us to work with the discrete, classical information provided by detector violations rather than the continuous quantum amplitudes of the underlying system.

\begin{definition}[Definition 9: Syndrome]
The \textbf{syndrome} of a spacetime fault $F$ over types $V$, $E$, $M$ is defined as a finite set of detectors:
\[
\mathrm{Syndrome}(V, E, M) := \mathrm{Finset}(\mathrm{Detector}\; V\; E\; M).
\]

More concretely, given a detector collection $\mathrm{DC}$, base measurement outcomes, and a spacetime fault $F$, the syndrome is the set of all violated detectors:
\[
\mathrm{syndrome}(\mathrm{DC}, \mathrm{base}, F) := \{D \in \mathrm{DC}.\mathrm{detectors} \mid D.\mathrm{isViolated}(\mathrm{applyFaultToOutcomes'}(\mathrm{base}, F))\}.
\]

The syndrome can also be viewed as a binary vector over the detector set, where entry $D$ equals 1 if detector $D$ is violated and 0 otherwise:
\[
\mathrm{syndromeVector}(\mathrm{DC}, \mathrm{base}, F)(D) =
\begin{cases}
1 & \text{if } D \in \mathrm{syndrome}(\mathrm{DC}, \mathrm{base}, F), \\
0 & \text{otherwise}.
\end{cases}
\]
\end{definition}

The syndrome inherits a natural algebraic structure from the underlying $\mathbb{Z}_2$ vector space. Syndrome addition is defined via symmetric difference, reflecting the fact that detector violations combine according to parity: if two faults each violate the same detector, their composition leaves that detector unviolated, while violations from different detectors accumulate. This $\mathbb{Z}_2$-linearity is fundamental to the mathematical structure of quantum error correction, as it allows us to decompose complex fault patterns into simpler components and understand their collective behavior through linear algebra over finite fields.

\vspace{1em}

\subsection{Definition 10: SpacetimeLogicalFault}

In quantum error correction, faults can occur during the measurement process that corrupt the syndrome information used for error detection. Understanding the structure of these faults—particularly those that evade detection while still affecting the logical information—is crucial for designing robust error correction protocols.

The key insight is that faults with empty syndrome (those that produce no detectable signature) can be partitioned into two distinct categories: those that preserve the logical information (spacetime stabilizers) and those that corrupt it (spacetime logical faults). This classification allows us to understand which undetected errors are harmless versus those that represent genuine threats to the encoded quantum information.

\begin{definition}[Definition 10: Spacetime Logical Fault]
A fault $f$ is a \textbf{spacetime logical fault} with respect to a detector collection $\mathrm{DC}$, base outcomes, and logical effect predicate $\ell$ if:
\begin{enumerate}
  \item $f$ has empty syndrome: $\mathrm{hasEmptySyndrome}(\mathrm{DC}, \mathrm{base}, f)$, and
  \item $f$ affects logical information: $\mathrm{affectsLogicalInfo}(\ell, f)$.
\end{enumerate}

Complementary to this, a fault $f$ is a \textbf{spacetime stabilizer} if it has empty syndrome but preserves logical information: $\mathrm{preservesLogicalInfo}(\ell, f)$.

The set of all spacetime logical faults is:
\[
  \mathrm{spacetimeLogicalFaultSet}(\mathrm{DC}, \mathrm{base}, \ell) = \{ f \mid \mathrm{IsSpacetimeLogicalFault}(\mathrm{DC}, \mathrm{base}, \ell, f) \}.
\]
\end{definition}

This definition captures the fundamental distinction between "good" and "bad" undetected faults. Spacetime stabilizers represent measurement errors that cancel out in a way that leaves both the syndrome and logical information unchanged, while spacetime logical faults represent the dangerous case where errors evade detection but still corrupt the encoded quantum state. The spacetime stabilizers form a group under fault composition, and two faults are considered equivalent if they differ by a spacetime stabilizer, naturally leading to a quotient group structure that classifies the essential types of logical damage that can occur.

\vspace{1em}

\subsection{Definition 11: SpacetimeFaultDistance}

The concept of fault distance is fundamental in quantum error correction, quantifying the resilience of a quantum code against errors. In the context of spacetime fault-tolerant quantum computing, we need to account not only for qubit errors but also for measurement and initialization faults that occur throughout the computation. The spacetime fault distance captures this broader notion of fault tolerance by considering all possible ways that independent faults can conspire to cause an undetectable logical error.

To formalize this concept, we first establish the mathematical framework for characterizing which fault patterns can cause logical errors without being detected by the syndrome measurement process.

\begin{definition}[Definition 11: Spacetime Fault Distance]
Let $\mathit{DC}$ denote a detector collection, and let $\mathit{times}$ be the set of relevant time steps in the computation. The \textbf{spacetime fault distance} $d_{\mathrm{ST}}$ is defined as:
\[
d_{\mathrm{ST}} = \begin{cases} 
\min W & \text{if logical faults exist}, \\ 
0 & \text{otherwise}, 
\end{cases}
\]
where $W = \{w \in \mathbb{N} \mid \exists F \text{ a spacetime fault}, F \text{ is a spacetime logical fault and } |F|_{\mathit{times}} = w\}$ is the set of logical fault weights.

Here, $|F|_{\mathit{times}}$ denotes the weight of fault $F$ restricted to the time steps in $\mathit{times}$, counting the total number of single-qubit Pauli errors, single measurement errors, and single initialization errors. A spacetime fault $F$ is called a \textbf{spacetime logical fault} if it has empty syndrome (is undetectable) and affects logical information.
\end{definition}

The spacetime fault distance represents the minimum number of independent faults required to cause a logical error without being detected. This definition naturally generalizes the classical notion of code distance to the fault-tolerant setting, where we must account for all types of faults that can occur during quantum computation.

The key insight is that the spacetime fault distance provides a sharp threshold for fault tolerance: any collection of fewer than $d_{\mathrm{ST}}$ faults can either be detected by the syndrome measurements or represents a stabilizer operation that does not affect the encoded logical information.

\vspace{1em}

\subsection{Definition 12: TimeStepConvention}

In fault-tolerant quantum error correction protocols, precise timing conventions are essential for coordinating the sequence of quantum operations, measurements, and error correction procedures. The challenge lies in distinguishing between discrete quantum states that exist at specific moments and the measurement processes that occur between these states. This temporal structure requires a mathematical framework that can capture both the discrete nature of quantum computation and the intermediate steps of measurement and error correction.

The half-integer time step convention provides an elegant solution by using integer time steps for quantum states and half-integer time steps for measurements. This approach naturally separates the evolution of quantum information from the classical measurement process, enabling precise tracking of when errors occur and when they are detected. Such precision is crucial for analyzing the performance of quantum error correction codes and designing fault-tolerant quantum circuits.

\begin{definition}[Definition 12: Time Step Convention]
A \textbf{half-integer time representation} is an inductive type with two constructors:
\begin{itemize}
  \item $\mathtt{integer}(n)$ represents the integer time step $n \in \mathbb{Z}$,
  \item $\mathtt{halfInteger}(n)$ represents the half-integer time step $n + \frac{1}{2}$ for $n \in \mathbb{Z}$.
\end{itemize}

The conversion to rational numbers is given by:
\[
\mathrm{toRat}(\mathtt{integer}(n)) = n, \qquad \mathrm{toRat}(\mathtt{halfInteger}(n)) = n + \frac{1}{2}.
\]

We define predicates to identify time types:
\[
\mathrm{isInteger}(\mathtt{integer}(n)) = \mathrm{true}, \quad \mathrm{isInteger}(\mathtt{halfInteger}(n)) = \mathrm{false},
\]
\[
\mathrm{isHalfInteger}(\mathtt{integer}(n)) = \mathrm{false}, \quad \mathrm{isHalfInteger}(\mathtt{halfInteger}(n)) = \mathrm{true}.
\]

Floor and ceiling operations are defined as:
\[
\mathrm{floor}(\mathtt{integer}(n)) = n, \quad \mathrm{floor}(\mathtt{halfInteger}(n)) = n,
\]
\[
\mathrm{ceil}(\mathtt{integer}(n)) = n, \quad \mathrm{ceil}(\mathtt{halfInteger}(n)) = n + 1.
\]

The successor and predecessor functions advance and retreat by $\frac{1}{2}$:
\[
\mathrm{succ}(\mathtt{integer}(n)) = \mathtt{halfInteger}(n), \quad \mathrm{succ}(\mathtt{halfInteger}(n)) = \mathtt{integer}(n+1),
\]
\[
\mathrm{pred}(\mathtt{integer}(n)) = \mathtt{halfInteger}(n-1), \quad \mathrm{pred}(\mathtt{halfInteger}(n)) = \mathtt{integer}(n).
\]
\end{definition}

This time step convention establishes a natural ordering where measurements occur precisely between quantum state preparations, enabling rigorous analysis of fault-tolerant quantum protocols. The half-integer framework ensures that every quantum operation can be precisely localized in time, which is essential for understanding error propagation and correction in quantum computing systems.

\vspace{1em}

\subsection{Lemma 1: DeformedCode}

In the theory of quantum error correction, the process of gauging represents a fundamental procedure for transforming stabilizer codes by introducing auxiliary qubits and modifying the stabilizer structure. When we gauge a quantum code with respect to a graph $G$, we add edge qubits $|0\rangle$ to each edge and introduce Gauss law operators that couple vertices to their incident edges. This procedure has profound implications for the code's logical structure, as it transforms certain logical operators into stabilizers through measurement.

The gauging transformation is particularly important in topological quantum error correction, where it provides a systematic method for studying the relationship between different quantum codes. The key insight is that by measuring the Gauss law operators and applying appropriate corrections, we can create a new stabilizer code whose structure is intimately related to the original code but with modified properties.

\begin{lemma}[Lemma 1: Deformed Code Stabilizer Structure]
The following operators form a generating set of stabilizer checks for the deformed (gauged) code:
\begin{enumerate}
\item \textbf{Gauss's law operators}: $A_v = X_v \prod_{e \ni v} X_e$ for all $v \in V_G$.
\item \textbf{Flux operators}: $B_p = \prod_{e \in p} Z_e$ for a generating set of cycles $\{p\}$ of $G$.
\item \textbf{Deformed checks}: $\widetilde{s}_j = s_j \prod_{e \in \gamma_j} Z_e$ for all checks $s_j$ in the original code, where $\gamma_j$ is an edge-path satisfying $\partial \gamma_j = \mathcal{S}_{Z,j} \cap V_G$.
\end{enumerate}
Moreover, the logical subspace of the deformed code has dimension $2^{k-1}$, one qubit less than the original $2^k$-dimensional logical subspace, corresponding to the measured logical $L$.
\end{lemma}

\begin{proof}
The proof proceeds in four main parts, establishing that each type of operator forms a valid stabilizer and computing the final dimension.

\textbf{Part 1: Gauss's Law Operators Become Stabilizers}

The $A_v$ operators are explicitly measured during the gauging procedure. By the measurement postulate of quantum mechanics, after measuring $A_v$ with outcome $\varepsilon_v \in \{+1, -1\}$, the state is projected into the $\varepsilon_v$-eigenspace of $A_v$. By tracking outcomes and applying conditional Pauli corrections $X_v$ when $\varepsilon_v = -1$, we ensure the code is in the $+1$ eigenspace of all $A_v$.

To verify these form valid stabilizers, we must check mutual commutativity and self-inverse properties. For any vertices $v, w \in V$, the symplectic form $\omega(A_v, A_w) = 0$ since both are X-type operators with no Z-support. Each $A_v$ is Hermitian with $A_v^2 = I$, confirming the self-inverse property.

The key constraint is that $\prod_v A_v = L = \prod_v X_v$, which means the $A_v$ operators are not all independent. However, this constraint actually benefits us: the logical operator $L$ becomes measurable through the product $\prod_v A_v$, transforming it from a logical operator into a stabilizer.

\textbf{Part 2: Flux Operators Are Stabilizers}

We establish that $B_p$ stabilizes the state in two steps:

\textit{Step 2a}: The edge qubits start in $|0\rangle^{\otimes E_G}$. Since $Z|0\rangle = (+1)|0\rangle$, we have $B_p |0\rangle^{\otimes E} = (+1)|0\rangle^{\otimes E}$.

\textit{Step 2b}: $B_p$ commutes with all $A_v$. The key observation is that for a valid cycle $p$, the number of edges in $p$ incident to any vertex $v$ is always even: either $0$ if $v \notin p$, or exactly $2$ if $v \in p$ (since cycles have no endpoints). Therefore, the symplectic form $\omega(B_p, A_v)$, which counts Z-X anticommutations, is always even modulo 2, confirming commutativity.

\textbf{Part 3: Deformed Checks Are Stabilizers}

For the deformed checks $\widetilde{s}_j = s_j \prod_{e \in \gamma_j} Z_e$, we must verify commutativity with all $A_v$ operators and confirm the eigenvalue is $+1$.

The boundary condition $\partial \gamma_j = \mathcal{S}_{Z,j} \cap V_G$ is crucial. At any vertex $v$, the anticommutation contributions from the original stabilizer $s_j$ and from the deformation factor $\prod_{e \in \gamma_j} Z_e$ are precisely equal by this boundary condition. Therefore, the total anticommutation count is $2 \times (\text{contribution}) \equiv 0 \pmod{2}$, ensuring commutativity.

For the eigenvalue, the original stabilizer $s_j$ has eigenvalue $+1$ on the code state by definition of stabilizer codes, and each $Z_e$ has eigenvalue $+1$ on the initial state $|0\rangle_e$. Therefore, $\widetilde{s}_j$ has eigenvalue $+1$ on the combined state.

\textbf{Part 4: Dimension Count}

We compute the final code dimension by counting qubits and independent stabilizers:

\textit{Qubits}: The deformed system has $n + |E_G|$ qubits (original qubits plus edge qubits).

\textit{Stabilizers}: 
\begin{itemize}
\item \textbf{Gauss law}: $|V_G|$ operators $A_v$, but with constraint $\prod_v A_v = L$. This gives $(|V_G| - 1)$ independent operators, plus $L$ itself becomes a stabilizer, for a total of $|V_G|$ stabilizers.
\item \textbf{Flux}: By Euler's formula for connected graphs, there are $|E_G| - |V_G| + 1$ independent cycles.
\item \textbf{Deformed checks}: $(n - k)$ operators from the original code.
\end{itemize}

Total independent stabilizers: $|V_G| + (|E_G| - |V_G| + 1) + (n - k) = |E_G| + n - k + 1$.

The code dimension is therefore:
\[
2^{(n + |E_G|) - (|E_G| + n - k + 1)} = 2^{k - 1}
\]

This confirms that exactly one logical qubit has been measured (the dimension drops from $2^k$ to $2^{k-1}$), corresponding to the logical operator $L$ becoming a stabilizer through the Gauss law constraint.
\end{proof}

This result demonstrates the fundamental principle of gauging in quantum error correction: by introducing auxiliary degrees of freedom and imposing local constraints (Gauss laws), we can systematically transform the logical structure of quantum codes. The reduction in logical dimension reflects the physical process of measuring and fixing certain logical operators, which is essential for implementing fault-tolerant quantum computation protocols.

\vspace{1em}

\subsection{Lemma 2: SpaceDistance}

In quantum error correction, understanding how the code distance changes under deformations is crucial for designing robust quantum codes. When we deform a quantum code by adding auxiliary qubits and modifying the stabilizer structure, a natural question arises: how does the distance of the deformed code relate to the original code distance?

This relationship depends fundamentally on the expansion properties of the underlying graph structure. The Cheeger constant, which measures the expansion of a graph, provides the key connection. Intuitively, good expansion ensures that any logical operator cannot be "localized" too much, preventing the distance from degrading significantly.

\begin{lemma}[Lemma 2: Space Distance Bound for Deformed Code]
The distance $d^\textit{$ of the deformed code satisfies
\[
d^} \geq \min(h(G), 1) \cdot d,
\]
where $h(G)$ is the Cheeger constant of the graph $G$ and $d$ is the distance of the original code.
\end{lemma}

\begin{proof}
Let $L'$ be any logical operator of the deformed code that achieves the minimum weight $d^\textit{ = |L'|$. We will show that $|L'| \geq \min(h(G), 1) \cdot d$.

\textbf{Step 1 (Structure of logical operators):} By the definition of deformed logical operators, $L'$ has the structure $(S_X^V, S_Z^V, S_X^E, S_Z^E, \phi)$ where the edge $X$-support $S_X^E$ satisfies the cocycle condition.

\textbf{Step 2 (Cocycle property):} Since $L'$ commutes with all flux operators $B_p$, the edge support $S_X^E$ forms a $1$-cocycle: for every cycle $p$, we have $|S_X^E \cap p| \equiv 0 \pmod{2}$.

\textbf{Step 3 (Exactness gives coboundary):} By the exactness condition, every $1$-cocycle is a coboundary. Therefore, there exists a vertex set $S_0$ such that $S_X^E = \partial S_0$ (the vertex cut of $S_0$).

\textbf{Step 4 (Choose smaller half):} We can choose $S$ with $\partial S = \partial S_0$ and $2|S| \leq |V|$ by taking the complement if necessary, since the vertex cut of a set equals the vertex cut of its complement.

\textbf{Step 5 (Cleaning removes edge support):} Multiplying $L'$ by the Gauss law operators $\prod_{v \in S} A_v$ gives a cleaned operator $\bar{L}$ with $\bar{L}.S_X^E = S_X^E \triangle \partial S = \emptyset$, so the cleaned operator has no edge $X$-support.

\textbf{Step 6 (Cleaned restriction is original logical):} The vertex restriction $\bar{L}|_V = (\bar{L}.S_X^V, \bar{L}.S_Z^V)$ is a nontrivial logical operator of the original code, hence has weight $|\bar{L}|_V \geq d$.

\textbf{Step 7 (Cheeger bound):} Since $S \neq \emptyset$ and $2|S| \leq |V|$, the Cheeger constant gives $|\partial S| \geq h(G) \cdot |S|$.

\textbf{Step 8 (Weight accounting):} The total weight satisfies
\[
|L'| = |S_X^V \cup S_Z^V| + |S_X^E \cup S_Z^E| \geq |\bar{L}|_V - |S| + |\partial S|,
\]
where we account for the changes in vertex support due to cleaning and the edge support that gets removed.

\textbf{Case analysis:}

\textit{Case 1: $h(G) \geq 1$.} Then $\min(h(G), 1) = 1$ and $|\partial S| \geq |S|$. Thus:
\[
|L'| \geq |\bar{L}|_V - |S| + |\partial S| \geq |\bar{L}|_V \geq d.
\]

\textit{Case 2: $h(G) < 1$.} Then $\min(h(G), 1) = h(G)$. Using $|\partial S| \geq h(G) \cdot |S|$:
\[
|L'| \geq |\bar{L}|_V - |S| + h(G) \cdot |S| = |\bar{L}|_V + (h(G) - 1) \cdot |S|.
\]
Since $h(G) - 1 < 0$ and $|S| \leq |\bar{L}|_V$ (otherwise the bound is immediate), we get:
\[
|L'| \geq |\bar{L}|_V + (h(G) - 1) \cdot |\bar{L}|_V = h(G) \cdot |\bar{L}|_V \geq h(G) \cdot d.
\]

In both cases, $|L'| \geq \min(h(G), 1) \cdot d$, completing the proof.
\end{proof}

This result has important implications for the design of quantum codes on expander graphs. When $h(G) \geq 1$ (strong expanders), the deformation preserves the code distance exactly: $d^} \geq d$. For weaker expanders, the distance may decrease, but only by a factor controlled by the Cheeger constant. This provides a quantitative trade-off between the expansion properties of the underlying graph and the robustness of the quantum error correction capabilities.

\vspace{1em}

\subsection{Lemma 3: SpacetimeCodeDetectors}

In fault-tolerant quantum error correction, the transition from one code to another through a gauging procedure requires careful tracking of measurement outcomes to detect errors that occur during the process. The gauging measurement procedure involves three distinct temporal phases: before, during, and after code deformation, each requiring different types of detectors to monitor the quantum system's evolution.

The concept of spacetime detectors emerges from the need to correlate measurement outcomes across time to identify when and where quantum errors have occurred. These detectors form the foundation of fault-tolerant error correction protocols, enabling the system to maintain coherence throughout the gauging process. The mathematical structure of these detectors reflects the underlying group-theoretic properties of quantum error correction, where measurement parities encode information about error syndromes.

\begin{lemma}[Lemma 3: Spacetime Code Detectors]
The elementary detectors form a generating set for the fault-tolerant gauging measurement procedure, satisfying both verification and completeness conditions across all temporal regions of the gauging protocol.
\end{lemma}

\begin{proof}
The proof establishes that the elementary detectors provide complete coverage and proper parity behavior across eight essential conditions, which we verify systematically.

\textbf{Part 1 --- Verification Properties:}

We first establish that all detector parities vanish in the error-free case, ensuring that the detectors only fire when genuine errors occur.

For bulk detectors, consider any measurement outcome $m \in \mathbb{Z}/2\mathbb{Z}$. In error-free projective measurement, measuring the same observable twice on the same quantum state yields identical outcomes. Therefore, the bulk detector parity satisfies:
\[
\mathrm{bulkDetectorParity}(m, m) = \mathrm{xorParity}(m, m) = m + m = 0,
\]
where the final equality follows from the characteristic property of $\mathbb{Z}/2\mathbb{Z}$ that $x + x = 0$ for all $x$.

For initial boundary conditions, edge qubits are initialized in the $|0\rangle$ state at time $t_i - \frac{1}{2}$. Since $|0\rangle$ is a $+1$ eigenstate of the Pauli-$Z$ operator, flux measurements $B_p = \prod_{e \in p} Z_e$ yield outcome $+1$ (encoded as $0 \in \mathbb{Z}/2\mathbb{Z}$) when applied to the initialized state. The initial boundary detector compares this initialization outcome with the subsequent $B_p$ measurement, giving parity $\mathrm{xorParity}(0, 0) = 0 + 0 = 0$.

For deformed stabilizer checks at the initial boundary, we measure $s_j$ at time $t_i - \frac{1}{2}$ with outcome $m_{s_j}$, then measure $\tilde{s}_j = s_j \cdot Z_\gamma$ at time $t_i + \frac{1}{2}$. Since $Z_\gamma|0\rangle = |0\rangle$ (eigenvalue $+1$), the deformed measurement outcome is $m_{\tilde{s}} = m_{s_j} + 0 = m_{s_j}$. The detector parity becomes:
\[
\mathrm{xorParity}(m_{s_j}, m_{\tilde{s}}) = \mathrm{xorParity}(m_{s_j}, m_{s_j}) = m_{s_j} + m_{s_j} = 0.
\]

For final boundary flux operators, the measurement constraint $B_p = \prod_{e \in p} Z_e$ ensures that measuring $B_p$ directly and measuring each $Z_e$ individually then computing their product (XOR in $\mathbb{Z}/2\mathbb{Z}$) yield identical results. If $m_{B_p} = m_{Z_e\text{-product}}$, then:
\[
\mathrm{xorParity}(m_{B_p}, m_{Z_e\text{-product}}) = \mathrm{xorParity}(m_{Z_e\text{-product}}, m_{Z_e\text{-product}}) = 0.
\]

For final boundary deformed checks, the relation $\tilde{s}_j = s_j \cdot Z_\gamma$ imposes a three-way constraint. If the measurement outcomes satisfy $m_{\tilde{s}} = m_{s_j} + m_{Z_\gamma}$, then:
\[
m_{\tilde{s}} + m_{s_j} + m_{Z_\gamma} = (m_{s_j} + m_{Z_\gamma}) + m_{s_j} + m_{Z_\gamma} = (m_{s_j} + m_{s_j}) + (m_{Z_\gamma} + m_{Z_\gamma}) = 0 + 0 = 0,
\]
where each self-sum vanishes in $\mathbb{Z}/2\mathbb{Z}$.

\textbf{Part 2 --- Completeness Properties:}

We now verify that detectors exist for all required measurement configurations across the entire temporal evolution.

For times $t < t_i$ (before deformation), the system operates under the original stabilizer code. For any check index $j < n_{\mathrm{checks}}$, the bulk detector $\langle \mathrm{originalCheck}(j), t, \mathrm{bulk} \rangle$ exists in the set:
\[
\bigl\{ \langle \mathrm{originalCheck}(j'), t, \mathrm{bulk} \rangle \mid j' \in \{0, \ldots, n_{\mathrm{checks}}-1\} \bigr\}.
\]
This follows by providing the witness $j$ with the constraint $j < n_{\mathrm{checks}}$.

For the initial boundary at $t = t_i$, we require detectors for both flux operators and deformed stabilizer checks. For any cycle index $p < n_C$, the detector $\langle \mathrm{flux}(p), t_i, \mathrm{initialBoundary} \rangle$ belongs to the initial boundary detector set. Similarly, for any check index $j < n_{\mathrm{checks}}$, the detector $\langle \mathrm{deformedCheck}(j), t_i, \mathrm{initialBoundary} \rangle$ exists. These detectors handle the transition from the original code to the deformed configuration.

For the final boundary at $t = t_o$, symmetric conditions apply. Detectors $\langle \mathrm{flux}(p), t_o, \mathrm{finalBoundary} \rangle$ and $\langle \mathrm{deformedCheck}(j), t_o, \mathrm{finalBoundary} \rangle$ exist for all valid indices, managing the transition back to the original code structure.

The completeness of this detector set ensures that every measurement outcome contributing to error detection is properly accounted for, while the verification properties guarantee that these detectors remain silent in the absence of errors, providing a mathematically sound foundation for fault-tolerant quantum computation.
\end{proof}

This result establishes the theoretical foundation for implementing fault-tolerant gauging procedures in quantum error correction. The detector set provides complete spatial and temporal coverage while maintaining the essential property that all detector parities vanish in error-free operation, ensuring that non-zero detector outcomes reliably indicate the presence and location of quantum errors throughout the gauging process.

\vspace{1em}

\subsection{Lemma 4: SpacetimeStabilizers}

Local stabilizers in quantum error correction are fundamental for understanding how errors can be corrected without propagating throughout the system. In the context of fault-tolerant gauging measurements, we need to characterize which fault patterns can occur during the measurement procedure while maintaining the code's error-correcting properties. These patterns must satisfy two key properties: they produce no detectable syndrome (appearing "invisible" to the error correction protocol), and they preserve the logical quantum information encoded in the system.

The following lemma establishes that a specific set of generators completely characterizes all such local spacetime stabilizers for the fault-tolerant gauging measurement procedure. This result is crucial for proving that the gauging protocol maintains the quantum error correction properties throughout the measurement process.

\begin{lemma}[Lemma 4: Spacetime Stabilizers]
The listed generators form a generating set of local spacetime stabilizers for the fault-tolerant gauging measurement procedure. Specifically:
\begin{enumerate}
\item[(a)] \textbf{Empty Syndrome}: Every generator produces no detectable syndrome
\item[(b)] \textbf{Preserves Logical}: Every generator preserves the logical quantum information
\item[(c)] \textbf{Completeness}: Every local spacetime stabilizer can be decomposed into products of these generators
\end{enumerate}

The generators are classified by time region:
\begin{itemize}
\item \textbf{Original code regions} ($t < t_i$ and $t > t_o$): space stabilizers and Pauli pairs with measurement faults
\item \textbf{Deformed code region} ($t_i < t < t_o$): vertex and edge Pauli pairs with appropriate measurement faults  
\item \textbf{Boundary regions} ($t = t_i, t_o$): initialization/finalization generators accounting for qubit preparation and measurement
\end{itemize}
\end{lemma}

\begin{proof}
The proof proceeds by establishing each part separately through detailed verification of the syndrome and logical effects.

\textbf{Part (a) - Empty Syndrome:} We verify that each generator type produces zero net effect on all relevant detectors. The key insight is that each generator is carefully constructed with compensating measurement faults.

For space stabilizers, the effect is trivially zero since stabilizers act as identity on the code space.

For Pauli pair generators (e.g., $P$ at time $t$ and $P$ at time $t+1$), we analyze the detector effects:
At detector $c^t$: The Pauli at time $t$ flips the measurement at $t + \frac{1}{2}$, but the measurement fault also flips it, so the net effect is $1 + 1 = 0$ in $\mathbb{Z}_2$
At detector $c^{t+1}$: The measurement at $t + \frac{1}{2}$ equals the base value (as above), and the physical state at $t + \frac{3}{2}$ has $P$ applied twice ($P^2 = I$), giving net effect $(1 + 1) + (1 + 1) = 0$

For boundary generators, similar calculations show that initialization faults cancel with subsequent Pauli operations, and finalization effects are properly compensated.

\textbf{Part (b) - Preserves Logical:} Each generator type has logical effect zero:
Space stabilizers: Act as identity on code space by definition
Pauli pairs: $P \cdot P = I$ cancels out any logical effect  
Boundary generators: Act on qubits being initialized/discarded, not affecting encoded information

\textbf{Part (c) - Completeness:} For any fault pattern $p$ with empty syndrome and preserved logical information, we construct a decomposition into generators.

If $p$ contains only measurement faults (no Pauli faults), it corresponds to a space stabilizer, so we take the empty list of generators.

If $p$ contains Pauli faults, we can factorize any Pauli separation $P_t \cdot P_{t+k}$ into adjacent pairs:
\[P_t \cdot P_{t+k} = \prod_{i=0}^{k-1}(P_{t+i} \cdot P_{t+i+1})\]

Each factor $(P_{t+i}, P_{t+i+1})$ corresponds to a Pauli pair generator. The intermediate Paulis cancel telescopically since each appears twice: once as the second element of pair $i-1$ and once as the first element of pair $i$.

The empty syndrome condition ensures that appropriate measurement faults are included with each Pauli pair to cancel detector effects. The logical preservation condition ensures that the net Pauli effect either cancels completely or corresponds to a valid stabilizer.

Therefore, every local spacetime stabilizer can be written as a product of the listed generators, establishing completeness.
\end{proof}

This lemma is essential for the fault-tolerance analysis of the gauging measurement protocol. It shows that any error pattern that could potentially corrupt the quantum computation can be decomposed into elementary generators, each of which is harmless (produces no syndrome and preserves logical information). This decomposition property is what allows the error correction protocol to function correctly even in the presence of faults during the gauging measurement procedure.

\vspace{1em}

\subsection{Lemma 5: TimeFaultDistance}

In quantum error correction, understanding fault patterns that preserve syndrome but affect logical outcomes is crucial for determining code distance. Time-based faults—errors occurring during measurements and state preparation—form a particularly important class because they can create error chains spanning multiple time steps without being detected by comparison operators.

The time-fault distance characterizes the minimum number of measurement and initialization errors needed to create a logical fault while maintaining an empty syndrome. This distance directly relates to the temporal span of the quantum error correction protocol, providing fundamental limits on the code's ability to detect time-correlated errors.

\begin{lemma}[Lemma 5: Time Fault Distance]
Let $I$ be a deformation interval with initial time $t_i$ and final time $t_o$, and let $\mathcal{F}$ be the set of pure time spacetime logical faults. Then the pure time fault distance equals exactly $t_o - t_i$:
\[
\min\{F.\text{weight}(\text{times}) \mid F \in \mathcal{F}\} = t_o - t_i.
\]
\end{lemma}

\begin{proof}
The proof establishes both upper and lower bounds, showing they are equal.

\textbf{Step 1 (Upper bound construction):} We construct the $A_v$ measurement fault chain, which places a measurement error on a single check $m$ at every time step $t \in [t_i, t_o)$. This fault has exactly $t_o - t_i$ time errors and no space errors, giving weight $t_o - t_i$.

For this fault to have empty syndrome, we verify that no interior comparison detectors are violated. A comparison detector $(c, t)$ compares measurement outcomes at times $t - 1/2$ and $t + 1/2$. For the $A_v$ chain on check $m$ and interior time $t_i < t < t_o$, both time steps $t-1$ and $t$ lie in the interval $[t_i, t_o)$, so both have measurement errors. The parity of errors is the same (both odd), hence the detector is satisfied.

The $A_v$ chain affects the logical outcome because it flips the measurement outcome an odd number of times (specifically, $t_o - t_i$ times) across the deformation interval, changing the gauged logical state.

\textbf{Step 2 (Lower bound):} Consider any pure time spacetime logical fault $F$ with empty syndrome and nontrivial logical effect. Since $F$ affects the logical outcome, there exists some check $m$ and time $t_0 \in [t_i, t_o)$ with an odd number of measurement errors.

Since $F$ has empty syndrome, no interior comparison detectors are violated. This means that for any check $c$ and interior time $t_i < t < t_o$, the parity of measurement errors at times $t-1$ and $t$ must be equal.

By transitivity of parity equality across the interval, if any time $t_0 \in [t_i, t_o)$ has an odd number of measurement errors on check $m$, then every time $t \in [t_i, t_o)$ must have a positive number of measurement errors on $m$.

Since $F$ is a pure time fault, its weight equals the total number of time error locations. The constraint that every time in $[t_i, t_o)$ has at least one measurement error on check $m$ implies:
\[
F.\text{weight}(\text{times}) \geq |\{(m,t) : t \in [t_i, t_o)\}| = |[t_i, t_o)| = t_o - t_i.
\]

\textbf{Step 3 (Equality):} The $A_v$ chain achieves weight exactly $t_o - t_i$ with empty syndrome and nontrivial logical effect, establishing the upper bound. Every pure time logical fault has weight at least $t_o - t_i$, establishing the lower bound. Therefore, the pure time fault distance equals $t_o - t_i$.
\end{proof}

This result has important implications for quantum error correction protocols. It shows that the temporal structure of the protocol directly determines the fault tolerance against time-correlated errors: a deformation spanning $k$ time steps can be defeated by measurement fault chains of weight exactly $k$. This provides a fundamental trade-off between protocol duration and protection against systematic measurement errors.

\vspace{1em}

\subsection{Lemma 6: SpacetimeDecoupling}

Spacetime logical faults in quantum error correction typically exhibit a complex interplay between spatial Pauli errors and temporal measurement errors distributed across multiple time steps. A natural question arises: can we decompose such faults into simpler components that are easier to analyze? This section establishes a fundamental decoupling result showing that any spacetime logical fault can be decomposed, up to stabilizer multiplication, into the product of two simpler faults: one containing only spatial errors at a single time step, and another containing only measurement errors.

This decomposition has profound implications for the analysis of fault tolerance in spacetime codes. It allows us to study space and time components separately, simplifying both theoretical analysis and practical error correction protocols. The result relies on the gauge freedom inherent in spacetime stabilizer codes, which permits us to "clean" spatial errors by moving them to a designated time step using Pauli pair stabilizers.

\begin{lemma}[Lemma 6: Spacetime Decoupling]
Let $\mathrm{DC}$ be a detector collection with group-like logical effect and group homomorphism syndrome. Let $I = [t_i, t_o]$ be a gauging interval, and let $F$ be a spacetime logical fault.

Suppose that for any such fault, there exists a cleaning stabilizer $S_{\mathrm{clean}}$ (built from Pauli pair stabilizers) such that $S_{\mathrm{clean}}$ is a spacetime stabilizer and the space errors of $F \cdot S_{\mathrm{clean}}$ are concentrated at time $t_i$.

Then there exist spacetime faults $F_{\mathrm{space}}$ and $F_{\mathrm{time}}$ such that:
\begin{enumerate}
\item $F$ is equivalent to $F_{\mathrm{space}} \cdot F_{\mathrm{time}}$ modulo stabilizers.
\item $F_{\mathrm{space}}$ is a pure space fault at the single time step $t_i$.
\item $F_{\mathrm{time}}$ is a pure time fault (only measurement errors).
\end{enumerate}
\end{lemma}

\begin{proof}
We extract the cleaning stabilizer $S_{\mathrm{clean}}$ from the hypothesis, obtaining that $S_{\mathrm{clean}}$ is a spacetime stabilizer and the space errors of $F' := F \cdot S_{\mathrm{clean}}$ are concentrated at $t_i$.

We define the two components:
\begin{itemize}
\item $F_{\mathrm{space}}$: the spacetime fault with space errors given by $F'.\mathrm{spaceErrors}(q, t)$ when $t = t_i$ and $I$ otherwise, and with no time errors.
\item $F_{\mathrm{time}}$: the spacetime fault with no space errors (all $I$) and time errors equal to $F'.\mathrm{timeErrors}$.
\end{itemize}

We verify the three claims using the witness $F_{\mathrm{space}}$, $F_{\mathrm{time}}$:

\textbf{Claim 1} ($F \sim F_{\mathrm{space}} \cdot F_{\mathrm{time}}$): We use $S = S_{\mathrm{clean}}^{-1}$ as the stabilizer witness. First, $S_{\mathrm{clean}}^{-1}$ is a stabilizer by the stabilizer inverse property: since the syndrome is a group homomorphism, it respects inverses, so the syndrome of $S_{\mathrm{clean}}^{-1}$ is empty; and since the logical effect is group-like, $S_{\mathrm{clean}}^{-1}$ preserves the logical information.

Second, we must show $F = (F_{\mathrm{space}} \cdot F_{\mathrm{time}}) \cdot S_{\mathrm{clean}}^{-1}$.

We first establish that $F' = F_{\mathrm{space}} \cdot F_{\mathrm{time}}$ by extensionality:
\begin{itemize}
\item For space errors: if $t = t_i$, then 
\[
(F_{\mathrm{space}} \cdot F_{\mathrm{time}}).\mathrm{spaceErrors}(q, t) = F'.\mathrm{spaceErrors}(q, t) \cdot I = F'.\mathrm{spaceErrors}(q, t)
\]
by the identity $P \cdot I = P$. If $t \neq t_i$, then 
\[
(F_{\mathrm{space}} \cdot F_{\mathrm{time}}).\mathrm{spaceErrors}(q, t) = I \cdot I = I,
\]
and by the concentration hypothesis, $F'.\mathrm{spaceErrors}(q, t) = I$ as well.
\item For time errors: 
\[
(F_{\mathrm{space}} \cdot F_{\mathrm{time}}).\mathrm{timeErrors}(m, t') = \mathrm{false} \oplus F'.\mathrm{timeErrors}(m, t') = F'.\mathrm{timeErrors}(m, t')
\]
by the identity $\mathrm{false} \oplus b = b$.
\end{itemize}

Then we compute:
\begin{align}
F &= F \cdot 1 \\
&= F \cdot (S_{\mathrm{clean}} \cdot S_{\mathrm{clean}}^{-1}) \\
&= (F \cdot S_{\mathrm{clean}}) \cdot S_{\mathrm{clean}}^{-1} \\
&= F' \cdot S_{\mathrm{clean}}^{-1} \\
&= (F_{\mathrm{space}} \cdot F_{\mathrm{time}}) \cdot S_{\mathrm{clean}}^{-1},
\end{align}
using the identity law, the inverse cancellation $S_{\mathrm{clean}} \cdot S_{\mathrm{clean}}^{-1} = 1$, and associativity of multiplication.

\textbf{Claim 2} ($F_{\mathrm{space}}$ is a pure space fault at $t_i$): By construction, for any qubit $q$ and time $t' \neq t_i$, we have $F_{\mathrm{space}}.\mathrm{spaceErrors}(q, t') = I$, and for all measurements $m$ and times $t'$, we have $F_{\mathrm{space}}.\mathrm{timeErrors}(m, t') = \mathrm{false}$.

\textbf{Claim 3} ($F_{\mathrm{time}}$ is a pure time fault): By construction, for all qubits $q$ and times $t$, we have $F_{\mathrm{time}}.\mathrm{spaceErrors}(q, t) = I$.
\end{proof}

The spacetime decoupling lemma has several important consequences. First, it shows that the complexity of spacetime logical faults is fundamentally limited - they can always be decomposed into spatially localized and temporally pure components. Second, it provides a systematic way to analyze fault tolerance by studying space and time errors separately. Finally, it suggests that efficient decoding algorithms might exploit this structure by treating spatial and temporal components independently.

\vspace{1em}

\subsection{Lemma 7: SpacetimeFaultDistanceLemma}

The spacetime fault-tolerant gauging measurement procedure provides a framework for performing quantum error correction in spacetime codes. A fundamental question in fault-tolerant quantum computing is whether such procedures preserve the distance properties of the underlying code. The spacetime fault-distance measures the minimum weight of any spacetime logical fault that could cause undetected logical errors during the measurement process.

This result establishes that under appropriate conditions—specifically strong expansion of the underlying graph structure and sufficient measurement rounds—the spacetime fault-distance exactly equals the distance of the original code. This preservation of distance is crucial for maintaining the error-correcting capabilities of the code throughout the fault-tolerant measurement procedure.

\begin{lemma}[Lemma 7: Spacetime Fault-Distance Lemma]
Let $d > 0$ be the distance of the original quantum code, and let $t_i < t_o$ define a deformation interval with $(t_o - t_i) \geq d$ measurement rounds. Suppose the Cheeger constant satisfies $h(G) \geq 1$. Then the spacetime fault-distance of the fault-tolerant gauging measurement procedure equals exactly $d$:
\[
d_{\mathrm{ST}} = \min\{|F| : F \text{ is a spacetime logical fault}\} = d.
\]
\end{lemma}

\begin{proof}
The proof proceeds by establishing both upper and lower bounds on the spacetime fault-distance.

\textbf{Upper Bound ($d_{\mathrm{ST}} \leq d$):} We construct an explicit spacetime logical fault of weight $d$. Consider an original code logical operator $L_{\mathrm{orig}}$ of weight $d$ applied at a time outside the deformation region (either $t < t_i$ or $t > t_o$). This operator acts on vertex qubits only, leaving edge qubits unaffected. When converted to a spacetime fault, this operator creates space errors at the specified time with no measurement errors. The resulting spacetime fault has weight exactly $d$, establishing that $d_{\mathrm{ST}} \leq d$.

\textbf{Lower Bound ($d_{\mathrm{ST}} \geq d$):} We show that every spacetime logical fault has weight at least $d$. By the Spacetime Decoupling Lemma, any spacetime logical fault $F$ admits a decomposition into two cases:

\textit{Case 1 (Time-nontrivial):} The fault decomposes as $F = F_{\mathrm{space}} \cdot F_{\mathrm{time}} \cdot S$ where $S$ is a stabilizer and $F_{\mathrm{time}}$ is a nontrivial spacetime logical fault that spans the entire interval $[t_i, t_o)$. Since $F_{\mathrm{time}}$ spans the interval, it must have at least one measurement error at each time step in the interval. This gives $|F_{\mathrm{time}}| \geq (t_o - t_i) \geq d$. Since cleaning preserves weight relationships, $|F| \geq |F_{\mathrm{time}}| \geq d$.

\textit{Case 2 (Space-only):} The fault is equivalent to a pure space fault $F_{\mathrm{space}}$ at time $t_i$ (modulo a stabilizer). Since $h(G) \geq 1$ ensures strong expansion, any logical space operator must have weight at least $d$. The cleaning process preserves this weight bound, so $|F| \geq |F_{\mathrm{space}}| \geq d$.

In both cases, we obtain $|F| \geq d$ for any spacetime logical fault $F$.

\textbf{Conclusion:} Combining the upper and lower bounds, we have $d_{\mathrm{ST}} = d$.
\end{proof}

This result has important implications for fault-tolerant quantum computation. It guarantees that the spacetime measurement procedure does not degrade the error-correcting capabilities of the original code, provided we perform sufficiently many measurement rounds and the underlying graph has good expansion properties. The theorem also establishes that any spacetime fault pattern with weight less than $d$ is either detectable (produces a non-trivial syndrome) or acts as a stabilizer (produces no logical error), which is precisely the behavior required for successful quantum error correction.

\vspace{1em}

\subsection{Theorem 1: GaugingMeasurement}

The measurement of gauge degrees of freedom in quantum error correction leads naturally to projective measurements of logical operators. In gauge theory, the physical Hilbert space is obtained by projecting onto states satisfying certain constraints (Gauss laws). When we perform measurements to enforce these constraints, we simultaneously perform a measurement of global topological degrees of freedom. This phenomenon, known as gauging by measurement, establishes a deep connection between local constraint enforcement and global quantum information processing.

The key insight is that measuring all local Gauss law operators $A_v = \prod_{e \ni v} Z_e$ with outcomes $\varepsilon_v \in \{+1, -1\}$ is equivalent to measuring the global logical operator $L = \prod_v X_v$. The measurement outcome for $L$ is determined by the parity of the number of $-1$ outcomes among the Gauss law measurements. This equivalence arises from the topological structure of the underlying graph and the algebraic relationships between local and global operators.

\begin{theorem}[Theorem 1: Gauging Measurement Equivalence]
Let $G$ be a connected graph with vertex set $V$ and edge set $E$. Let $\varepsilon: V \to \mathbb{Z}/2\mathbb{Z}$ represent Gauss law measurement outcomes, where $0$ corresponds to outcome $+1$ and $1$ corresponds to outcome $-1$. Let $z \in (\mathbb{Z}/2\mathbb{Z})^E$ be in the image of the coboundary map $\delta$, and let $c'$ be any $0$-cochain satisfying $\delta c' = z$. Define:
\begin{itemize}
\item The total outcome $\sigma = \sum_{v \in V} \varepsilon_v \in \mathbb{Z}/2\mathbb{Z}$
\item The logical operator support $L = \mathbf{1} \in (\mathbb{Z}/2\mathbb{Z})^V$ (all-ones vector)
\item For any $0$-cochain $c$, the phase factor $\varepsilon(c) = \sum_{v \in V} c_v \varepsilon_v$
\item The vertex Pauli operator $X_V(c)$ with support given by $c$
\end{itemize}

Then the following properties hold:
\begin{enumerate}
\item \textbf{Fiber structure}: $\delta c = z$ if and only if $c = c'$ or $c = c' + \mathbf{1}$
\item \textbf{Phase relation}: $\varepsilon(c' + \mathbf{1}) = \varepsilon(c') + \sigma$
\item \textbf{Operator relation}: $X_V(c' + \mathbf{1}) = X_V(c') + L$
\item \textbf{Projector properties}: $\sigma + \sigma = 0$ and $L + L = 0$ in $\mathbb{Z}/2\mathbb{Z}$
\item \textbf{Idempotence}: $\sigma \cdot \sigma = \sigma$ in $\mathbb{Z}/2\mathbb{Z}$
\end{enumerate}

These properties establish that gauging by measurement is equivalent to projective measurement of the logical operator $L$ with eigenvalue $(-1)^\sigma$, up to a byproduct operator $X_V(c')$.
\end{theorem}

\begin{proof}
We establish each property systematically.

\textbf{Part (1) - Fiber Structure}: Let $c$ be any $0$-cochain with $\delta c = z$. Since $\delta c' = z$ as well, we have $\delta(c - c') = \delta c - \delta c' = z - z = 0$. Thus $c - c' \in \ker(\delta)$.

For a connected graph $G$, the kernel of the coboundary map $\delta: (\mathbb{Z}/2\mathbb{Z})^V \to (\mathbb{Z}/2\mathbb{Z})^E$ consists precisely of the constant cochains. Since we work over $\mathbb{Z}/2\mathbb{Z}$, there are exactly two constant cochains: the zero cochain $0$ and the all-ones cochain $\mathbf{1}$. This follows from the exactness of the boundary-coboundary sequence for connected graphs.

Therefore, $c - c' = 0$ or $c - c' = \mathbf{1}$, which gives $c = c'$ or $c = c' + \mathbf{1}$.

Conversely, if $c = c'$ then clearly $\delta c = \delta c' = z$. If $c = c' + \mathbf{1}$, then $\delta c = \delta(c' + \mathbf{1}) = \delta c' + \delta \mathbf{1} = z + 0 = z$, since $\mathbf{1} \in \ker(\delta)$.

\textbf{Part (2) - Phase Relation}: By definition of $\varepsilon$:
\begin{align}
\varepsilon(c' + \mathbf{1}) &= \sum_{v \in V} (c' + \mathbf{1})_v \varepsilon_v \\
&= \sum_{v \in V} (c'_v + 1) \varepsilon_v \\
&= \sum_{v \in V} c'_v \varepsilon_v + \sum_{v \in V} \varepsilon_v \\
&= \varepsilon(c') + \sigma
\end{align}

\textbf{Part (3) - Operator Relation}: This follows directly from the definition of $X_V$ and the identification of $L$ with the all-ones support vector. The operator $X_V(c' + \mathbf{1})$ has support on vertices where $(c' + \mathbf{1})_v = 1$, which occurs when either $c'_v = 1$ and $\mathbf{1}_v = 0$ (impossible since $\mathbf{1}_v = 1$ for all $v$), or when $c'_v = 0$ and $\mathbf{1}_v = 1$, or when $c'_v = 1$ and $\mathbf{1}_v = 1$. In $\mathbb{Z}/2\mathbb{Z}$, this gives $X_V(c' + \mathbf{1}) = X_V(c') + L$.

\textbf{Part (4) - Projector Properties}: For any element $x \in \mathbb{Z}/2\mathbb{Z}$, we have $x + x = 2x = 0$ since $2 = 0$ in $\mathbb{Z}/2\mathbb{Z}$. Applying this to $\sigma$ and to each component of $L = \mathbf{1}$ gives the result.

\textbf{Part (5) - Idempotence}: In $\mathbb{Z}/2\mathbb{Z}$, we verify by cases: if $\sigma = 0$ then $0 \cdot 0 = 0$, and if $\sigma = 1$ then $1 \cdot 1 = 1$. Thus $\sigma^2 = \sigma$.

\textbf{Measurement Equivalence}: These properties combine to show that the post-measurement state after gauging is proportional to:
\[X_V(c') \left( I + (-1)^\sigma L \right) |\psi\rangle\]

The operator $\frac{1}{2}(I + (-1)^\sigma L)$ is the projector onto the $(-1)^\sigma$ eigenspace of $L$, since $L^2 = I$ and the eigenvalues of $L$ are $\pm 1$. The factor $X_V(c')$ represents a byproduct operator that depends on the specific choice of $c'$ but not on the eigenvalue measurement outcome.
\end{proof}

This theorem reveals the fundamental duality between local gauge fixing and global topological measurements in quantum error correction. The measurement of all Gauss law constraints automatically implements a projective measurement of the logical operator $L$, with the outcome determined by the parity of constraint violations. This connection provides both theoretical insight into the structure of gauge theories and practical guidance for implementing fault-tolerant quantum computation with topological codes.

\vspace{1em}

\subsection{Theorem 2: FaultTolerance}

In quantum error correction, the measurement-based approach to quantum computing requires careful analysis of how faults propagate through both space and time. The fundamental question is whether the combined spacetime system maintains the same error-correcting capabilities as the original quantum code. This chapter establishes that under appropriate conditions—sufficient expansion properties of the underlying graph and adequate measurement rounds—the spacetime fault-distance equals exactly the distance of the original code, thus preserving the fault-tolerance threshold.

The key insight is that spacetime faults can be decomposed into spatial and temporal components, each contributing independently to the overall error correction problem. When the Cheeger constant of the measurement graph is sufficiently large ($h(G) \geq 1$) and the measurement protocol runs for enough rounds ($(t_o - t_i) \geq d$), these two contributions can be bounded separately to yield optimal fault tolerance.

\begin{definition}[Definition: Fault Tolerance Configuration]
A \textbf{fault tolerance configuration} is a structure that bundles all preconditions needed to establish $d_{ST} = d$. It consists of:
\begin{itemize}
  \item A code distance $d \in \mathbb{N}$ with $d > 0$.
  \item An initial time $t_i$ and final time $t_o$ with $t_i < t_o$ (the deformation interval is nonempty).
  \item The condition $(t_o - t_i) \geq d$ (sufficient measurement rounds).
  \item A Cheeger constant $h(G) \in \mathbb{R}$ with $h(G) \geq 1$ (strong expansion).
\end{itemize}
\end{definition}

The number of measurement rounds is simply $\text{numRounds}(\text{cfg}) := t_o - t_i$, which by construction is positive and satisfies $\text{numRounds}(\text{cfg}) \geq d$.

\begin{theorem}[Theorem 2: Fault Tolerance]
Under the conditions:
\begin{enumerate}
  \item $h(G) \geq 1$ (Cheeger constant at least 1),
  \item $(t_o - t_i) \geq d$ (sufficient measurement rounds),
\end{enumerate}
the spacetime fault-distance equals exactly $d$:
\[
d_{ST} = d.
\]
More precisely, assuming that every spacetime logical fault $F$ admits a lower bound case decomposition, and there exists a weight-$d$ logical fault, then
\[
\text{spacetimeFaultDistance}(DC, \text{baseOutcomes}, \text{logicalEffect}, [t_i, t_o)) = d.
\]
\end{theorem}

\begin{proof}
The proof proceeds by establishing matching upper and lower bounds on $d_{ST}$.

\textbf{Upper bound ($d_{ST} \leq d$):} We construct an explicit spacetime logical fault of weight exactly $d$. Consider an original logical operator $L_{\text{orig}}$ from the base quantum code with support size $|\{v \mid L_{\text{orig}}.\text{vertexPaulis}(v) \neq I\}| = d$. We apply this operator at time $t < t_i$ before the measurement protocol begins. Converting this to a spacetime fault $F_d := L_{\text{orig}}.\text{toSpacetimeFault}$ yields a fault with $\text{weight}(F_d) = d$ that remains a logical operator. By the property that the spacetime fault distance is at most the weight of any logical fault, we have $d_{ST} \leq \text{weight}(F_d) = d$.

\textbf{Lower bound ($d_{ST} \geq d$):} Since $F_d$ witnesses that logical faults exist, there exists a minimum-weight logical fault $F_{\min}$ with $\text{weight}(F_{\min}) = d_{ST}$. We apply the decomposition hypothesis to $F_{\min}$ to obtain either:

\textit{Case 1 (Time-dominated):} There exists a pure time fault $F_{\text{time}}$ with $\text{weight}(F_{\text{time}}) \geq \text{numRounds}$ and $\text{weight}(F_{\min}) \geq \text{weight}(F_{\text{time}})$. Then:
\[
\text{weight}(F_{\min}) \geq \text{weight}(F_{\text{time}}) \geq \text{numRounds} \geq d.
\]

\textit{Case 2 (Space-dominated):} There exists a pure space fault $F_{\text{space}}$ at time $t_i$ with $\text{weight}(F_{\text{space}}) \geq d$ and $\text{weight}(F_{\min}) \geq \text{weight}(F_{\text{space}})$. Then:
\[
\text{weight}(F_{\min}) \geq \text{weight}(F_{\text{space}}) \geq d.
\]

In both cases, we obtain $\text{weight}(F_{\min}) \geq d$, hence $d_{ST} = \text{weight}(F_{\min}) \geq d$.

\textbf{Conclusion:} Combining $d \leq d_{ST} \leq d$, we conclude $d_{ST} = d$.
\end{proof}

This result has several important consequences. First, it guarantees that the spacetime fault distance is positive, ensuring meaningful error correction. Second, any fault with weight less than $d$ is either detectable (triggers a syndrome) or acts as a stabilizer (does not affect logical information). Finally, the code can correct up to $\lfloor(d-1)/2\rfloor$ faults, matching the classical sphere-packing bound.

The theorem's proof structure reveals the fundamental trade-off in measurement-based quantum computing: spatial expansion properties (captured by $h(G) \geq 1$) and temporal redundancy (ensured by $(t_o - t_i) \geq d$) together provide the necessary resources to maintain optimal fault tolerance. The decomposition of spacetime faults into spatial and temporal components allows each aspect to be analyzed separately, with the stronger of the two bounds determining the overall fault-tolerance threshold.

\vspace{1em}

\subsection{Corollary 1: OverheadBound}

Quantum error correcting codes require efficient measurement procedures for logical operators, but the overhead of auxiliary qubits often scales poorly with the operator weight. Traditional approaches, such as those of Cohen et al., require overhead linear in both the operator weight $W$ and the code distance $d$, leading to a scaling of $O(Wd)$ auxiliary qubits. For high-distance quantum LDPC codes where $d = \Theta(n)$, this overhead becomes prohibitive for large logical operators.

The gauging measurement technique offers a fundamentally different approach based on cycle sparsification and graph layering. By constructing a layered graph structure and using the Freedman-Hastings expansion technique, we can measure logical operators with significantly reduced overhead when the code distance exceeds the logarithmic factor $(\log W)^2$.

\begin{corollary}[Corollary 1: Overhead Bound]
For any quantum LDPC code with logical operator weight $W \geq 2$, maximum degree bound $d \geq 1$, and Freedman-Hastings constant $C_{\mathrm{FH}} > 0$, the gauging measurement procedure can be implemented using at most
\[
c \cdot W \cdot \bigl((\log_2 W)^2 + 1\bigr)
\]
auxiliary qubits, where $c = 2d + C_{\mathrm{FH}}(d+1) + C_{\mathrm{FH}}$ is a universal constant depending only on the code parameters.
\end{corollary}

\begin{proof}
The proof proceeds by constructing an explicit overhead configuration and bounding each component of the auxiliary qubit count.

\textbf{Step 1: Configuration construction.} Given parameters $W$, $d$, and $C_{\mathrm{FH}}$, we construct an overhead configuration with:
\begin{itemize}
\item Weight: $W$ (the logical operator weight)
\item Base edges: at most $Wd$ (from LDPC constraint)
\item Number of layers: $C_{\mathrm{FH}} \cdot (\log_2 W)^2 + C_{\mathrm{FH}}$ (from Freedman-Hastings construction)
\item Cellulation edges: at most $Wd$ (from triangulation)
\end{itemize}

\textbf{Step 2: Qubit overhead decomposition.} The total auxiliary qubit count consists of four components:
\begin{align}
\text{Total qubits} &= \text{Layer 0 qubits} + \text{Inter-layer qubits} + \text{Intra-layer qubits} + \text{Cellulation qubits} \\
&= Wd + R \cdot W + R \cdot Wd + Wd
\end{align}
where $R = C_{\mathrm{FH}} \cdot (\log_2 W)^2 + C_{\mathrm{FH}}$ is the number of layers.

\textbf{Step 3: Algebraic simplification.} Factoring out common terms:
\[
\text{Total qubits} = 2Wd + R \cdot W + R \cdot Wd = 2Wd + R(W + Wd) = 2Wd + RW(1 + d)
\]

\textbf{Step 4: Layer bound application.} Using the bound $R \leq C_{\mathrm{FH}} \cdot ((\log_2 W)^2 + 1)$:
\begin{align}
\text{Total qubits} &\leq 2Wd + C_{\mathrm{FH}} \cdot ((\log_2 W)^2 + 1) \cdot W(1 + d) \\
&= W \cdot \bigl[2d + C_{\mathrm{FH}}(1 + d)((\log_2 W)^2 + 1)\bigr]
\end{align}

\textbf{Step 5: Constant identification.} Expanding the coefficient:
\begin{align}
c &= 2d + C_{\mathrm{FH}}(1 + d)((\log_2 W)^2 + 1) \\
&= 2d + C_{\mathrm{FH}}((\log_2 W)^2 + 1) + C_{\mathrm{FH}}d((\log_2 W)^2 + 1) \\
&\leq 2d + C_{\mathrm{FH}}((\log_2 W)^2 + 1) + C_{\mathrm{FH}}d((\log_2 W)^2 + 1)
\end{align}

For the bound in the corollary statement, we use the fact that for any fixed values of $d$ and $C_{\mathrm{FH}}$, the expression can be bounded by $c \cdot W \cdot ((\log_2 W)^2 + 1)$ where $c = 2d + C_{\mathrm{FH}}(d+1) + C_{\mathrm{FH}}$ by absorbing the logarithmic factors into the constant through careful analysis of the dominant terms.

\textbf{Step 6: Positivity verification.} Since $d \geq 1$ and $C_{\mathrm{FH}} > 0$, we have $c = 2d + C_{\mathrm{FH}}(d+1) + C_{\mathrm{FH}} \geq 2 + C_{\mathrm{FH}} \cdot 2 + C_{\mathrm{FH}} > 0$.
\end{proof}

This result represents a significant improvement over previous methods. For quantum LDPC codes where the distance $d$ grows linearly with the code length $n$, and logical operators of weight $W = O(n)$, the condition $d > (\log_2 W)^2$ is easily satisfied for reasonable code sizes. In this regime, the gauging measurement requires only $O(W \log^2 W)$ auxiliary qubits compared to the $O(Wd)$ overhead of direct measurement approaches, yielding a logarithmic improvement that becomes substantial for high-distance codes.

\vspace{1em}

\subsection{Corollary 2: CheegerOptimality}

The Cheeger constant emerges naturally in the study of quantum error correction codes on graph states as a fundamental parameter controlling code distance preservation under gauge deformations. When constructing deformed codes, one must balance the trade-off between gauge symmetries and error correction capabilities. This leads to the fundamental question: what is the optimal choice of graph geometry to preserve the original code distance?

The answer lies in the spectral properties of the graph, specifically its Cheeger constant $h(G)$, which measures the minimum ratio of edge boundary to vertex volume over all vertex cuts. This geometric parameter directly controls the distance degradation in deformed codes through the space distance bound of Lemma 2.

\begin{corollary}[Corollary 2: Cheeger Optimality]
Let $G$ be a graph satisfying the exactness condition with Cheeger constant $h(G) > 0$. Let $d$ denote the original code distance and $d^\textit{$ the deformed code distance. Then:
\begin{enumerate}
\item \textbf{When $h(G) \geq 1$}: $d^} \geq d$ (distance is preserved).
\item \textbf{When $h(G) < 1$}: $d^\textit{ \geq h(G) \cdot d$ (distance is reduced by factor $h(G)$).
\item \textbf{In all cases}: $d^} \geq \min(h(G), 1) \cdot d$.
\end{enumerate}

Furthermore, $h(G) = 1$ is optimal in the sense that:
\begin{itemize}
\item No improvement occurs for $h(G) > 1$ (the bound remains $d^\textit{ \geq d$)
\item Distance degradation occurs for $h(G) < 1$ (the bound becomes $d^} \geq h(G) \cdot d < d$)
\item When $h(G) = 1$ and a trivially extendable logical exists, $d^\textit{ = d$ exactly
\end{itemize}
\end{corollary}

\begin{proof}
The proof follows by applying the space distance bound from Lemma 2 and analyzing the behavior of $\min(h(G), 1)$ across different regimes.

\textbf{Part 1} ($h(G) \geq 1 \implies d^} \geq d$): When $h(G) \geq 1$, we have $\min(h(G), 1) = 1$. Applying the space distance bound gives $d^\textit{ \geq 1 \cdot d = d$.

\textbf{Part 2} ($h(G) < 1 \implies d^} \geq h(G) \cdot d$): When $h(G) < 1$, we have $\min(h(G), 1) = h(G)$. The space distance bound then gives $d^\textit{ \geq h(G) \cdot d$. Since $h(G) < 1$ and $d > 0$, we obtain $h(G) \cdot d < d$, confirming distance reduction.

\textbf{Part 3} (General bound): This follows directly from the space distance bound: for any $h(G) > 0$, we have $d^} \geq \min(h(G), 1) \cdot d$.

\textbf{Optimality of $h(G) = 1$}: 
\begin{itemize}
\item For $h(G) > 1$: The factor $\min(h(G), 1) = 1$ provides no improvement over the $h(G) = 1$ case.
\item For $h(G) < 1$: The factor $\min(h(G), 1) = h(G) < 1$ gives a strictly weaker bound than the $h(G) = 1$ case.
\item For $h(G) = 1$: When a trivially extendable logical operator exists, the trivial extension provides a deformed logical of weight exactly $d$, achieving the lower bound and giving $d^* = d$.
\end{itemize}
\end{proof}

This corollary establishes $h(G) = 1$ as the critical threshold for optimal code distance preservation. Graphs with $h(G) = 1$ achieve perfect distance preservation when combined with suitable original codes, while graphs with larger Cheeger constants waste spectral resources without benefit, and graphs with smaller Cheeger constants suffer unavoidable distance degradation. This provides precise guidance for graph selection in deformed code constructions.

\vspace{1em}

\subsection{Remark 11: InitialFinalBoundaryConditions}

The perfect boundary condition assumption is a standard theoretical convention in fault-tolerant quantum computing that simplifies the analysis of error correction protocols. While actual quantum computations involve noisy initial states and imperfect final measurements, assuming perfect boundaries allows us to focus on the core mechanics of the fault-tolerant procedure without the additional complexity of boundary noise. This idealization is mathematically justified by the observation that sufficient error correction rounds before and after the main computation ensure that any error process spanning from the computation to the boundaries must have weight exceeding the code distance.

In practice, when implementing fault-tolerant protocols, the gauging measurement we study here would be embedded within a larger quantum computation that naturally provides appropriate boundary conditions through its own error correction structure.

\begin{remark}[Remark 11: Initial and Final Boundary Conditions]
In quantum error correction protocols involving gauging measurements, we adopt the \textbf{Perfect Boundary Convention}: both the initial and final rounds of stabilizer measurements are assumed to be error-free. This theoretical simplification is justified by implementing $d$ rounds of standard error correction before and after the gauging procedure, where $d$ is the code distance. Under this protection scheme, any error process that spans from the gauging measurement to either boundary must have weight greater than $d$, making such errors correctable by the underlying quantum error correction code.
\end{remark}

The mathematical framework underlying this convention involves several key concepts. We model boundary conditions through quality indicators (perfect or realistic) and track error correction rounds systematically. The central insight is that with sufficient protection rounds, low-weight errors cannot propagate between the gauging measurement and the boundaries.

Formally, an error process that involves both the gauging measurement and a boundary (initial or final) requires at least $d + 1$ fault locations when $d$ rounds of error correction protect each boundary. Since the quantum error correction code can handle up to $\lfloor d/2 \rfloor$ errors, and typical fault-tolerant protocols are designed so that $d \geq 2t + 1$ for $t$-error correction, boundary-spanning errors automatically fall outside the correctable range.

This approach allows theoretical analysis to proceed with clean mathematical statements while maintaining practical relevance. The key theorems establish that under the standard protection scheme, errors with weight at most $d$ cannot span to boundaries, making the quality of boundary measurements irrelevant for the fault-tolerance analysis of the gauging procedure itself.

The convention thus serves both analytical and practical purposes: it simplifies proofs by eliminating boundary noise from consideration, while the underlying protection mechanism ensures this simplification introduces no loss of generality for realistic implementations where the gauging measurement is one component of a larger fault-tolerant computation.

\vspace{1em}

\subsection{Remark 12: NoncommutingOperators}

In quantum error correction, not all Pauli operators can be deformed while preserving the stabilizer structure. This remark establishes a fundamental topological obstruction: operators that anticommute with the logical operator $L = \prod_{v} X_v$ cannot be deformed because their support sets have odd cardinality, while path boundaries necessarily have even cardinality. This incompatibility reveals why certain operators resist deformation through multiplication with stabilizer elements.

The obstruction arises from the interplay between algebraic commutation relations and topological properties of graph boundaries. When a Pauli operator $P$ anticommutes with $L$, its $Z$-type support $\mathcal{S}_Z \cap V_G$ has odd size, but any attempt to construct a deforming path requires this set to equal some boundary $\partial\gamma$, which must have even cardinality by the fundamental parity constraint of graph boundaries.

\begin{remark}[Remark 12: Noncommuting Operators Cannot Be Deformed]
A Pauli operator $P$ which does not commute with the logical operator $L$ cannot have a deformed version. The obstruction is topological: if $P$ anticommutes with $L = \prod_v X_v$, then $|\mathcal{S}_Z \cap V_G|$ is odd, but any path boundary $\partial\gamma$ always has even cardinality, so no edge-path $\gamma$ with $\partial\gamma = \mathcal{S}_Z \cap V_G$ can exist.

Formally, if $\text{anticommutesWithL}(\mathcal{S}_Z)$ holds (meaning $|\mathcal{S}_Z| \bmod 2 = 1$), then there exists no valid deforming path:
\[
\text{anticommutesWithL}(\mathcal{S}_Z) \implies \nexists \gamma, \; \text{IsValidDeformingPath}(G, \mathcal{S}_Z, \gamma).
\]
\end{remark}

This result has profound implications for quantum error correction protocols. It establishes a sharp dichotomy between operators that commute with the logical $L$ (which can potentially be deformed) and those that anticommute (which fundamentally cannot). The topological nature of this obstruction means it cannot be overcome by clever choices of stabilizer products—the parity mismatch between odd support sets and even boundaries represents an insurmountable barrier to deformation.

\vspace{1em}

\subsection{Remark 13: FluxCheckMeasurementFrequency}

\section{Introduction to Flux Check Measurement Frequency}

In the implementation of the deformed quantum error correcting code, three types of checks must be monitored: Gauss's law operators $A_v$ (X-type on vertices), deformed stabilizer checks $\tilde{s}_j$, and flux operators $B_p$ (Z-type on cycle edges). Traditionally, all checks are measured every round to maintain full syndrome information. However, this uniform approach may be unnecessarily resource-intensive, particularly for flux operators which can have very high weights in certain code constructions.

The key insight explored here is that flux checks $B_p$ can be measured much less frequently than the primary checks ($A_v$ and $\tilde{s}_j$), or even inferred entirely from initialization and readout data. While this modification preserves the fault distance scaling and significantly simplifies the measurement of high-weight operators, it comes with important trade-offs: detector cells become large and no fault-tolerance threshold is expected without further modifications. Nevertheless, this strategy remains valuable for small code instances where the benefits outweigh the drawbacks.

\begin{remark}[Remark 13: Flux Check Measurement Frequency Trade-offs]
Let $A_v$ denote the Gauss's law operators, $\tilde{s}_j$ the deformed stabilizer checks, and $B_p$ the flux operators in a deformed quantum error correcting code. The flux operators $B_p$ can be measured with reduced frequency or inferred entirely from initialization and readout data. This modification:
\begin{itemize}
    \item \textbf{Preserves} fault distance scaling with respect to code distance $d$
    \item \textbf{Simplifies} implementation by avoiding high-weight flux measurements
    \item \textbf{Results in} large detector cells spanning multiple rounds
    \item \textbf{Eliminates} the fault-tolerance threshold without further modifications
    \item \textbf{Remains useful} for small code instances where measurement complexity is the primary concern
\end{itemize}
\end{remark}

\textbf{Practical Implications:} This observation suggests a measurement hierarchy where primary checks ($A_v$, $\tilde{s}_j$) maintain standard frequencies to preserve error detection capabilities, while flux checks are handled through post-processing of initialization and readout data. For codes where flux operators have weight significantly exceeding the distance ($w \gg d$), this approach offers substantial practical advantages despite the theoretical limitations.

The trade-off analysis reveals that for small instances (typically $n \leq 100$ physical qubits) with high-weight flux checks, the inferred measurement strategy provides net benefits by eliminating the most challenging measurement operations while maintaining adequate error correction performance for the intended scale of operation.

\vspace{1em}

\subsection{Remark 14: Generalizations}

The development of quantum error correction has primarily focused on Pauli stabilizer codes, but the fundamental gauging measurement procedure extends far beyond this restricted setting. Understanding these generalizations reveals the broader applicability of the gauging framework and highlights important mathematical distinctions between different algebraic structures.

The gauging procedure naturally generalizes in four key directions: to arbitrary finite group representations with tensor product factorization, to non-Pauli operators that can generate magic states, to qudit systems with more than two levels per site, and to nonabelian groups. Each generalization preserves the core measurement protocol while introducing new mathematical phenomena.

\begin{remark}[Remark 14: Generalizations of the Gauging Measurement Procedure]
The gauging measurement procedure generalizes beyond Pauli stabilizer codes in four fundamental directions:
\begin{enumerate}
\item \textbf{Finite group representations}: The procedure applies to any representation of a finite group by operators with a tensor product factorization, not necessarily the logical operators of a quantum error-correcting code.
\item \textbf{Non-Pauli operators}: The gauging measurement can measure non-Pauli operators, whose measurement can produce magic states.
\item \textbf{Qudit systems}: The procedure extends to qudit systems with $d > 2$ levels per site.
\item \textbf{Nonabelian groups}: The generalization extends to nonabelian groups, but for nonabelian groups, measuring the charge locally does not fix a definite global charge.
\end{enumerate}
\end{remark}

The mathematical foundation underlying these generalizations rests on the fundamental difference between abelian and nonabelian group structures. In abelian groups, the commutativity of the group operation ensures that local measurements uniquely determine global charges, while in nonabelian groups, the order dependence of products creates ambiguity in charge determination.

\begin{theorem}[Abelian Product is Order-Independent]
For a commutative group $G$ and any function $\mathrm{charges} : \mathrm{Fin}(n) \to G$ and any permutation $\sigma$ of $\mathrm{Fin}(n)$,
\[
  \prod_{i \in \mathrm{Fin}(n)} \mathrm{charges}(i) = \prod_{i \in \mathrm{Fin}(n)} \mathrm{charges}(\sigma(i)).
\]
This is the key property that allows local measurements to determine the global charge.
\end{theorem}

\begin{proof}
By the commutativity of $G$, composing a product with any permutation $\sigma$ yields the same result. This follows directly from the fact that in a commutative group, the order of multiplication is irrelevant.
\end{proof}

\begin{theorem}[Any Two Orderings Give Equal Products]
For a commutative group $G$, any function $\mathrm{charges} : \mathrm{Fin}(n) \to G$, and any two permutations $\sigma_1, \sigma_2$ of $\mathrm{Fin}(n)$,
\[
  \prod_{i \in \mathrm{Fin}(n)} \mathrm{charges}(\sigma_1(i)) = \prod_{i \in \mathrm{Fin}(n)} \mathrm{charges}(\sigma_2(i)).
\]
\end{theorem}

\begin{proof}
By the previous theorem, both $\prod_{i} \mathrm{charges}(\sigma_1(i))$ and $\prod_{i} \mathrm{charges}(\sigma_2(i))$ equal the canonical product $\prod_i \mathrm{charges}(i)$. Therefore, they are equal to each other by transitivity.
\end{proof}

In contrast, nonabelian groups exhibit fundamentally different behavior due to the failure of commutativity.

\begin{theorem}[$S_3$ is Nonabelian]
The symmetric group $S_3$ on $3$ elements is nonabelian: there exist permutations $a, b \in S_3$ such that $a \cdot b \neq b \cdot a$.
\end{theorem}

\begin{proof}
Define $\mathrm{swap01}$ to be the transposition $(0\ 1)$ and $\mathrm{cycle012}$ to be the $3$-cycle $(0\ 1\ 2)$. We verify that these do not commute by direct computation. We have $\mathrm{swap01} \cdot \mathrm{cycle012}$ maps $0 \mapsto 0$, while $\mathrm{cycle012} \cdot \mathrm{swap01}$ maps $0 \mapsto 2$. Since $0 \neq 2$, these compositions are distinct.
\end{proof}

\begin{theorem}[Nonabelian Groups: Different Orderings Give Different Results]
For a nonabelian group $G$ (i.e., one in which there exist $a, b \in G$ with $ab \neq ba$), there exists a function $\mathrm{charges} : \mathrm{Fin}(2) \to G$ such that
\[
  \mathrm{charges}(0) \cdot \mathrm{charges}(1) \neq \mathrm{charges}(1) \cdot \mathrm{charges}(0).
\]
This is why, for nonabelian groups, local measurements do not fix a definite global charge.
\end{theorem}

\begin{proof}
From the hypothesis of nonabelianity, we obtain elements $a, b \in G$ with $ab \neq ba$. Define $\mathrm{charges}(0) = a$ and $\mathrm{charges}(1) = b$. Then $\mathrm{charges}(0) \cdot \mathrm{charges}(1) = ab$ and $\mathrm{charges}(1) \cdot \mathrm{charges}(0) = ba$, which are unequal by assumption.
\end{proof}

The extension to qudit systems demonstrates the scalability of the gauging framework beyond qubits.

\begin{definition}[Qudit Total Dimension]
A qudit system with local dimension $d$ and $n$ sites has total Hilbert space dimension
\[
  \mathrm{quditTotalDimension}(d, n) = d^n.
\]
\end{definition}

\begin{theorem}[Qudit Extension Valid]
For any $d > 2$ and $n > 0$, the qudit system has positive dimension:
\[
  \mathrm{quditTotalDimension}(d, n) > 0.
\]
This confirms the procedure extends to qudits with $d > 2$.
\end{theorem}

\begin{proof}
Since $d > 2$, we have $d \geq 2$ and therefore $d > 0$. A positive natural number raised to any positive power remains positive, so $d^n > 0$.
\end{proof}

Finally, we can systematically characterize when each generalization direction applies and which ones require special consideration.

\begin{theorem}[All Directions Applicable]
The gauging procedure applies to all four generalization directions: finite group representations, non-Pauli operators, qudit systems, and nonabelian groups.
\end{theorem}

\begin{theorem}[Only Nonabelian Has Caveat]
Among the four generalization directions, only nonabelian groups require the caveat that local measurements do not uniquely determine the global charge.
\end{theorem}

These generalizations reveal that the gauging measurement procedure is remarkably robust and extends naturally to diverse mathematical settings. The key insight is that while the procedure itself generalizes broadly, the relationship between local and global charges depends critically on the underlying algebraic structure, with nonabelian groups introducing fundamental limitations on charge determination that do not appear in the abelian case.

\vspace{1em}

\subsection{Remark 15: HypergraphGeneralization}

The traditional theory of quantum error correction focuses on measuring stabilizer generators associated with individual edges of a graph. However, many quantum codes and fault-tolerant protocols benefit from simultaneously measuring multiple operators that may not correspond to simple graph structures. This motivates a natural generalization where we replace the underlying graph with a hypergraph, allowing each "edge" to connect an arbitrary number of vertices rather than just two.

This hypergraph generalization maintains the essential algebraic structure of gauging measurements while providing greater flexibility in the types of multi-qubit operators that can be measured simultaneously. The resulting framework encompasses both traditional graph-based codes and more exotic constructions that arise in topological quantum computing and fault-tolerant protocol design.

\begin{remark}[Remark 15: Hypergraph Generalization]
The gauging measurement procedure can be generalized by replacing the graph $G$ with a \textbf{hypergraph} to measure multiple operators simultaneously. For qubits, this is equivalent to replacing the graph $G$ with a hypergraph. The generalized gauging procedure performs a code deformation by introducing a qubit for each hyperedge and measuring into new Gauss's law checks $A_v$ given by the product of $X$ on a vertex and the adjacent hyperedges.

A \textit{hypergraph} on vertices $V$ with hyperedges indexed by $H$ is a structure consisting of an incidence function
\[
\text{incidence} : H \to \text{Finset}(V),
\]
mapping each hyperedge to the set of its incident vertices. This generalizes a graph, where each edge connects exactly two vertices, to the case where each hyperedge may connect an arbitrary subset of vertices.

The \textit{parity-check map} $H_Z : \mathbb{Z}_2^V \to \mathbb{Z}_2^H$ of a hypergraph is the $\mathbb{Z}_2$-linear map defined by
\[
(H_Z(x))_h = \sum_{v \in \text{incidence}(h)} x_v \pmod{2}
\]
for each hyperedge $h \in H$. The kernel of this map characterizes the abelian group of commuting $X$-type operators.

The generalized Gauss law operators $A_v$ for each vertex $v$ consist of an $X_v$ factor on the vertex qubit and $Z$ operators on all hyperedge qubits incident to $v$. These operators commute with each other, and their product $\prod_v A_v$ has support on the logical operator $L = \prod_v X_v$ when restricted to vertex qubits.
\end{remark}

This generalization proves particularly valuable in the graph-like case where every hyperedge connects exactly two vertices, recovering the standard gauging construction. In this setting, the logical operator $L$ automatically lies in the kernel of the parity-check map since each hyperedge has even cardinality. For more general hypergraphs with higher-degree hyperedges, the logical operator belongs to the kernel if and only if every hyperedge has even cardinality, providing a clear criterion for when the generalized gauging procedure preserves the logical information of the original code.

\vspace{1em}

\subsection{Remark 16: PracticalMeasurementRounds}

In quantum error correction, the theoretical analysis of fault tolerance requires a specific number of error correction rounds before and after gauging measurements to ensure rigorous proof of correctness. However, in practical implementations, this requirement often represents a conservative upper bound rather than a strict necessity. The actual number of rounds needed depends significantly on the computational context surrounding the gauging operation.

When a gauging measurement occurs in the middle of a large quantum computation, the surrounding quantum operations naturally provide additional error correction capabilities. This environmental protection allows for a reduction in the dedicated error correction rounds without compromising the overall fault tolerance of the system. Understanding this trade-off between theoretical guarantees and practical efficiency is crucial for optimizing quantum error correction protocols in real-world implementations.

\begin{remark}[Remark 16: Practical Measurement Rounds]
The $d$ rounds of quantum error correction required before and after gauging measurements in theoretical fault tolerance proofs represent a worst-case analysis and are typically excessive in practical implementations. In full fault-tolerant computations, the optimal number of rounds depends on the surrounding operations: when gauging occurs within a larger computation, a constant number of rounds (potentially much smaller than $d$) often suffices. However, this optimization affects the effective code distance and fault tolerance threshold.
\end{remark}

The key insight is that theoretical requirements provide universal guarantees but may be overly conservative for specific computational contexts. While $d$ rounds before and after gauging ensure fault tolerance regardless of surrounding operations, practical implementations can leverage contextual information to reduce this requirement. The trade-off involves balancing computational efficiency against the robustness of fault tolerance guarantees, with the understanding that reduced rounds may lower the effective distance but maintain acceptable error rates in favorable contexts.

\vspace{1em}

\subsection{Remark 17: CircuitImplementationFaultTolerance}

In quantum error correction, the circuit implementation model introduces an interesting trade-off between fault tolerance and qubit overhead. When implementing quantum error correcting codes through circuits, vertex qubits may become coupled in ways that could compromise the fault-distance properties essential for error correction. The circuit implementation from Remark 6 demonstrates this challenge, where direct adjacencies between vertex qubits can lead to correlated errors that reduce the effective distance of the code.

To address this issue while preserving fault tolerance, we employ a classical graph-theoretic technique known as edge subdivision. By inserting a dummy vertex along each edge of the original graph, we decouple all vertex qubits, ensuring that any two original vertices are separated by at least one intermediate vertex. This construction doubles the number of edge qubits but maintains the crucial fault-distance properties needed for quantum error correction.

\begin{definition}[Definition: Subdivided Vertex Type]
Given a vertex type $V$ and an edge type $E$, the \textbf{subdivided vertex type} is the disjoint sum $V \oplus E$, written $\texttt{SubdividedVertex'}(V, E) := V \sqcup E$. Original vertices are embedded via $\mathrm{inl}$ and dummy vertices (one per edge) via $\mathrm{inr}$.
\end{definition}

\begin{definition}[Definition: Subdivision Adjacency]
Let $G$ be a simple graph on $V$ with decidable adjacency. The \textbf{subdivision adjacency} on $\texttt{SubdividedVertex'}(V, \mathrm{Sym2}(V))$ is defined by:
\[
\mathrm{subdivisionAdj}'(G)(x, y) :=
\begin{cases}
e \in E(G) \land v_1 \in e & \text{if } x = \mathrm{inl}(v_1),\; y = \mathrm{inr}(e), \\
e \in E(G) \land v_1 \in e & \text{if } x = \mathrm{inr}(e),\; y = \mathrm{inl}(v_1), \\
\mathrm{False} & \text{otherwise.}
\end{cases}
\]
That is, a vertex $\mathrm{inl}(v)$ is adjacent to $\mathrm{inr}(e)$ if and only if $v$ is an endpoint of the edge $e$. There are no $\mathrm{inl}$--$\mathrm{inl}$ or $\mathrm{inr}$--$\mathrm{inr}$ adjacencies.
\end{definition}

\begin{definition}[Definition: Subdivided Graph]
Let $G$ be a simple graph on $V$. The \textbf{subdivided graph} $\mathrm{subdivideGraph}'(G)$ is the simple graph on $\texttt{SubdividedVertex'}(V, \mathrm{Sym2}(V))$ with adjacency given by $\mathrm{subdivisionAdj}'(G)$.
\end{definition}

\begin{theorem}[Theorem: No Original--Original Adjacency]
In the subdivided graph, no two original vertices are adjacent: for all $u, v \in V$,
\[
\neg\, \mathrm{subdivisionAdj}'(G)(\mathrm{inl}(u), \mathrm{inl}(v)).
\]
This formalizes that vertex qubits are decoupled after subdivision.
\end{theorem}

\begin{proof}
By the definition of $\mathrm{subdivisionAdj}'$, the case $(\mathrm{inl}(u), \mathrm{inl}(v))$ reduces directly to $\mathrm{False}$, so the negation holds trivially.
\end{proof}

\begin{theorem}[Theorem: No Dummy--Dummy Adjacency]
In the subdivided graph, no two dummy vertices are adjacent: for all edges $e_1, e_2 \in \mathrm{Sym2}(V)$,
\[
\neg\, \mathrm{subdivisionAdj}'(G)(\mathrm{inr}(e_1), \mathrm{inr}(e_2)).
\]
\end{theorem}

\begin{proof}
By the definition of $\mathrm{subdivisionAdj}'$, the case $(\mathrm{inr}(e_1), \mathrm{inr}(e_2))$ reduces directly to $\mathrm{False}$, so the negation holds trivially.
\end{proof}

\begin{theorem}[Theorem: Subdivided Graph is Bipartite]
The subdivided graph is bipartite: every edge connects an original vertex to a dummy vertex. Formally, for all $x, y \in \texttt{SubdividedVertex'}(V, \mathrm{Sym2}(V))$ with $\mathrm{subdivisionAdj}'(G)(x,y)$, either
\[
\exists\, v,\, e,\quad x = \mathrm{inl}(v) \land y = \mathrm{inr}(e),
\]
or
\[
\exists\, e,\, v,\quad x = \mathrm{inr}(e) \land y = \mathrm{inl}(v).
\]
\end{theorem}

\begin{proof}
We proceed by cases on $x$ and $y$.

\textbf{Case 1:} $x = \mathrm{inl}(v_1)$ for some $v_1$. We further case-split on $y$.
\begin{itemize}
\item If $y = \mathrm{inl}(v_2)$ for some $v_2$, then by the definition of subdivision adjacency, we would have $\mathrm{subdivisionAdj}'(G)(\mathrm{inl}(v_1), \mathrm{inl}(v_2)) = \mathrm{False}$, contradicting our adjacency hypothesis.
\item If $y = \mathrm{inr}(e)$ for some edge $e$, then we satisfy the first disjunct with witnesses $v_1$ and $e$.
\end{itemize}

\textbf{Case 2:} $x = \mathrm{inr}(e)$ for some edge $e$. We further case-split on $y$.
\begin{itemize}
\item If $y = \mathrm{inl}(v_1)$ for some $v_1$, then we satisfy the second disjunct with witnesses $e$ and $v_1$.
\item If $y = \mathrm{inr}(e_2)$ for some edge $e_2$, then by definition, $\mathrm{subdivisionAdj}'(G)(\mathrm{inr}(e), \mathrm{inr}(e_2)) = \mathrm{False}$, contradicting our adjacency hypothesis.
\end{itemize}
\end{proof}

\begin{remark}[Remark 17: Circuit Implementation Fault Tolerance]
The edge subdivision construction provides a systematic method for preserving fault-distance in quantum error correcting codes implemented through circuits. While this construction doubles the number of edge qubits (and thus the overhead), it ensures that vertex qubits remain decoupled, which is essential for maintaining the error-correcting properties of the code. The bipartite structure of the subdivided graph guarantees that any error propagation between original vertices must pass through intermediate dummy vertices, providing natural fault isolation. This trade-off between qubit overhead and fault tolerance is fundamental in practical quantum error correction schemes.
\end{remark}

The subdivision construction demonstrates how classical graph theory can inform quantum error correction protocols. By ensuring that original vertices are never directly adjacent, we maintain the locality properties needed for effective error correction while accepting a controlled increase in resource requirements.

\vspace{1em}

\subsection{Remark 18: LatticeSurgeryAsGauging}

The relationship between lattice surgery and gauging measurement provides a unifying perspective on fault-tolerant quantum computing protocols. In surface codes, lattice surgery enables the measurement of joint logical operators across separate code patches by introducing auxiliary gauge qubits and stabilizer measurements. This procedure can be understood as a special case of the more general gauging framework, where measuring products of logical operators corresponds to gauging specific symmetries of the quantum error-correcting code.

The key insight is that lattice surgery naturally emerges when we apply gauging to surface code patches using ladder-shaped auxiliary graphs. The ladder structure provides the minimal connectivity needed to measure joint logical $X$ operators while preserving the topological properties that make surface codes practical. This connection extends beyond adjacent patches: non-adjacent regions can be connected through dummy vertex grids, and the expansion properties required for fault tolerance can be relaxed to focus only on subsets relevant to the logical operators being measured.

\begin{remark}[Remark 18: Lattice Surgery as Gauging]
Lattice surgery is a special case of gauging measurement. The gauging measurement can be interpreted as a direct generalization of lattice surgery for surface codes. Measuring $\bar{X}_1 \otimes \bar{X}_2$ on a pair of equally sized surface code blocks using a ladder graph $G$ joining the edge qubits produces a deformed code that is again a surface code on the union of the two patches. The final step of measuring out individual edge qubits matches conventional lattice surgery.

For non-adjacent patches, one uses a graph with a grid of dummy vertices between the two edges. The procedure extends to any pair of matching logical $X$ operators using two copies of graph $G$ with bridge edges. The Cheeger condition $h(G) \geq 1$ is overkill; expansion is only needed for subsets of qubits relevant to the logical operators being measured.
\end{remark}

This remark establishes the theoretical foundation showing that lattice surgery protocols can be understood through the lens of gauging theory. The ladder graph construction provides the essential connectivity pattern, while the relaxed expansion condition shows that the full Cheeger bound can be weakened when we only need to protect specific logical degrees of freedom. This perspective opens new avenues for optimizing fault-tolerant protocols by focusing expansion requirements on the operationally relevant subsets of qubits.

\vspace{1em}

\subsection{Remark 19: ShorStyleMeasurementAsGauging}

In quantum error correction, the gauging measurement framework provides a general approach to performing logical measurements on stabilizer codes. A natural question arises: how does the standard Shor-style measurement procedure, which has been widely used for fault-tolerant quantum computation, relate to this more general framework? The answer reveals that Shor-style measurements are actually a special case of gauging measurements with a particular graph structure.

The key insight is that the Shor-style procedure—involving preparation of a GHZ state on auxiliary qubits, entanglement via transversal controlled-X gates, and measurement of Pauli-X operators on each auxiliary qubit—can be reformulated using gauging theory. This reformulation uses a specific bipartite graph where each qubit in the logical operator's support is paired with a "dummy" auxiliary qubit, and all auxiliary qubits are connected in either a path or star configuration.

\begin{remark}[Remark 19: Shor-Style Measurement as Gauging]
The standard Shor-style logical measurement is equivalent to a gauging measurement on a carefully constructed graph. For a logical operator $L$ with support size $W = |\operatorname{supp}(L)|$, we construct a bipartite graph with $2W$ vertices: $W$ "support vertices" corresponding to the qubits in $\operatorname{supp}(L)$, and $W$ "dummy vertices" corresponding to auxiliary qubits. Each support vertex is connected to its corresponding dummy vertex by a "cross edge," and the dummy vertices are connected among themselves in either a path or star pattern.
\end{remark}

This correspondence provides several important insights. First, it shows that the GHZ stabilizers used in Shor-style measurements (of the form $Z_i Z_{i+1}$ for consecutive auxiliary qubits) correspond exactly to the edges connecting dummy vertices in the gauging graph. Second, the controlled-X operations that entangle each logical qubit with its auxiliary partner implement the cross edges in the graph structure. Finally, the constraint that the product of all Gauss's law operators equals the logical operator $L$ is automatically satisfied by this construction.

The equivalence between Shor-style and gauging measurements demonstrates that the gauging framework is not merely an abstract generalization, but encompasses the measurement procedures already used in practice. This connection suggests that insights from graph theory and homological algebra can be systematically applied to understand and optimize fault-tolerant measurement protocols in quantum error correction.

\vspace{1em}

\subsection{Remark 20: CohenSchemeAsGauging}

The Cohen et al. scheme for logical measurement represents a cornerstone construction in quantum error correction, demonstrating how hypergraph gauging can systematically implement logical operations while preserving the underlying quantum error-correcting properties. This scheme addresses the fundamental challenge of measuring logical operators without destroying the quantum information encoded in the error-correcting code.

The key insight is that by restricting stabilizer checks to the support of an irreducible logical operator and constructing a carefully designed layered structure, one can create a measurement protocol that extracts the logical information while maintaining the code's error-correction capabilities. This construction bridges the gap between abstract quantum error correction theory and practical measurement implementations.

\begin{remark}[Remark 20: Cohen Scheme as Gauging]
The Cohen et al. scheme for logical measurement can be recovered as a hypergraph gauging measurement through a systematic construction that restricts Z-type checks to the support of an irreducible X logical operator $L$, forming a hypergraph whose kernel is $\{0, L\}$, then building a layered structure with dummy vertices, chain connections, and hypergraph copies. The Cross et al. modification and product measurement via bridge edges are also formalized within this framework.
\end{remark}

The construction begins with the Cohen hypergraph, which captures the essential structure of the restricted stabilizer checks. Given a logical operator $L$ with support on $W$ qubits, we form a hypergraph where vertices represent qubits in $\text{supp}(L)$ and hyperedges represent the Z-type stabilizer checks restricted to this support. The crucial property is that this restriction yields a kernel spanned precisely by the logical operator $L$, reflecting its irreducibility.

The layered construction extends this base hypergraph by introducing $d$ additional dummy layers, each containing a copy of the original $W$ qubits. These layers are connected by chain edges that couple corresponding qubits across adjacent layers, creating a structured pathway for information flow. Each layer also contains a complete copy of the original hypergraph structure, ensuring that the stabilizer constraints are preserved throughout the construction.

This framework naturally accommodates the Cross et al. modification, which reduces the number of dummy layers to optimize resource requirements, and extends to product measurements that combine multiple logical operators through bridge connections. The mathematical formulation demonstrates that these constructions preserve the essential kernel properties while providing the structural flexibility needed for practical quantum error correction implementations.

\vspace{1em}

\subsection{Remark 21: CSSCodeInitializationAsGauging}

CSS (Calderbank-Shor-Steane) codes are a fundamental class of quantum error-correcting codes that exhibit a special structure allowing for efficient implementation and analysis. The standard initialization procedure for CSS codes involves preparing the state $|0\rangle^{\otimes n}$ and measuring all $X$-type stabilizer generators. This remark demonstrates that this classical initialization procedure can be reinterpreted within the unified framework of hypergraph gauging, providing new insights into the geometric structure underlying CSS codes and their measurement protocols.

The key observation is that CSS initialization can be decomposed into three steps: (1) starting with a trivial code augmented with dummy vertices corresponding to each $X$-type check, (2) performing a generalized gauging measurement using the hypergraph defined by the $Z$-type checks, and (3) ungauging by measuring $Z$ on all physical qubits. Furthermore, Steane-style syndrome measurement can be implemented by combining state preparation gauging with pairwise $XX$ gauging between data and ancilla code blocks.

\begin{remark}[Remark 21: CSS Code Initialization as Gauging]
Standard CSS code initialization and Steane-style stabilizer measurement can be implemented as instances of the hypergraph gauging framework. Specifically:

\begin{enumerate}
\item \textbf{CSS Initialization}: The $X$-type stabilizer generators of a CSS code lie in the kernel of the hypergraph operator defined by the $Z$-type checks, making them measurable via gauging.

\item \textbf{Steane Measurement}: Syndrome extraction for a stabilizer group can be performed through a three-step gauging process: ancilla state preparation via CSS gauging, pairwise $XX$ entangling operations between data and ancilla blocks, and $Z$-basis readout measurements.
\end{enumerate}

The unification under the gauging framework reveals that these apparently distinct protocols share a common mathematical structure rooted in the symplectic geometry of stabilizer codes.
\end{remark}

This reformulation provides several advantages. First, it unifies disparate quantum error correction protocols under a single mathematical framework, revealing hidden connections between different measurement schemes. Second, it suggests natural generalizations of CSS codes to hypergraph-based constructions that may yield new families of quantum codes. Finally, the gauging perspective provides a systematic way to analyze the resource requirements and fault-tolerance properties of these protocols by leveraging the rich mathematical structure of hypergraph operators and their kernels.

\vspace{1em}

\subsection{Definition 13: BivariateBicycleCode}

Quantum error correction has evolved to include structured codes that exploit algebraic symmetries to achieve efficient encoding and decoding. Among these, CSS (Calderbank-Shor-Steane) codes represent a particularly important class, where the stabilizer group decomposes into commuting X-type and Z-type generators. The bivariate bicycle codes introduced here extend the classical bicycle codes by working over a product of cyclic groups, allowing for richer algebraic structure and potentially better distance properties.

The key insight behind bivariate bicycle codes is to construct CSS codes from pairs of polynomials defined over the group algebra of $\mathbb{Z}/\ell\mathbb{Z} \times \mathbb{Z}/m\mathbb{Z}$. This approach leverages the circulant structure inherent in cyclic groups while extending to two dimensions, creating a "bicycle" pattern where the same polynomials appear in both the X and Z stabilizer generators but with transposed roles.

\begin{definition}[Definition 13: Bivariate Bicycle Code]
Let $\ell, m$ be positive integers. A \textbf{Bivariate Bicycle (BB) code} with parameters $\ell, m$ is specified by two polynomials $A, B \in \mathbb{F}_2[x,y]/(x^\ell - 1, y^m - 1)$, where this quotient ring is identified with the group algebra $\mathbb{F}_2[M]$ for $M = \mathbb{Z}/\ell\mathbb{Z} \times \mathbb{Z}/m\mathbb{Z}$.

The code structure consists of:
\begin{itemize}
\item $2\ell m$ physical qubits, partitioned into $\ell m$ \textbf{left qubits} and $\ell m$ \textbf{right qubits}
\item $\ell m$ \textbf{X-type stabilizer checks} with parity check matrix $H_X = [A \mid B]$
\item $\ell m$ \textbf{Z-type stabilizer checks} with parity check matrix $H_Z = [B^\top \mid A^\top]$
\end{itemize}

Here $A$ and $B$ are represented as $\ell m \times \ell m$ matrices via the correspondence $(a,b) \mapsto A(a-b)$ for entries, and $A^\top$ denotes the algebraic transpose $A(x^{-1}, y^{-1})$.

The code is valid as a CSS code when the \textbf{orthogonality condition} holds:
\[
A \cdot B^\top + B \cdot A^\top = 0
\]
over $\mathbb{F}_2$, ensuring that X and Z stabilizers commute.
\end{definition}

This construction elegantly unifies the classical theory of cyclic codes with the modern framework of CSS quantum error correction. The bivariate structure allows for flexible parameter choices through the dimensions $\ell$ and $m$, while the bicycle pattern ensures a symmetric relationship between X and Z stabilizers that can lead to codes with good distance properties and efficient decoders.

\vspace{1em}

\subsection{Definition 14: GrossCode}

The Gross code is one of the most elegant examples of a quantum error-correcting code, named because it operates on a ``gross'' of qubits---that is, 144 qubits, where 144 equals twelve dozen (a ``gross'' being a dozen dozens). This code belongs to the family of bivariate bicycle codes and achieves the remarkable parameters of $[[144, 12, 12]]$, meaning it encodes 12 logical qubits into 144 physical qubits with a minimum distance of 12. The construction relies on carefully chosen polynomials over finite fields and exploits the algebraic structure of group algebras to define both the stabilizer generators and logical operators.

The significance of the Gross code lies not only in its appealing numerical properties but also in its demonstration of how algebraic techniques can yield codes with good parameters. The construction uses two polynomials $A$ and $B$ that are related by a natural symmetry, and the logical operators are defined using auxiliary polynomials $f$, $g$, and $h$ that respect this underlying structure.

\begin{definition}[Definition 14: Gross Code]
The \textbf{Gross code} is a $[[144, 12, 12]]$ bivariate bicycle code defined by the following parameters and polynomials:
\begin{itemize}
\item Parameters: $\ell = 12$ and $m = 6$, giving $n = 2 \cdot \ell \cdot m = 144$ physical qubits
\item Defining polynomials: $A = x^3 + y^2 + y$ and $B = y^3 + x^2 + x$
\item The code operates on the group algebra $\mathbb{F}_2[\mathbb{Z}/12\mathbb{Z} \times \mathbb{Z}/6\mathbb{Z}]$
\end{itemize}

The logical operators are constructed using the auxiliary polynomials:
\begin{align\textit{}
f &= 1 + x + x^2 + x^3 + x^6 + x^7 + x^8 + x^9 + (x + x^5 + x^7 + x^{11})y^3 \\
g &= x + x^2 y + (1+x)y^2 + x^2 y^3 + y^4 \\
h &= 1 + (1+x)y + y^2 + (1+x)y^3
\end{align}}

For any monomial $\alpha, \beta$ in the monomial set $M = \mathbb{Z}/12\mathbb{Z} \times \mathbb{Z}/6\mathbb{Z}$, the logical operators are:
\begin{itemize}
\item $\bar{X}_\alpha = X(\alpha \cdot f, 0)$ (weight-12 $X$-type operators)
\item $\bar{X}'_\beta = X(\beta \cdot g, \beta \cdot h)$ ($X$-type operators)  
\item $\bar{Z}_\beta = Z(\beta \cdot h^T, \beta \cdot g^T)$ ($Z$-type operators)
\item $\bar{Z}'_\alpha = Z(0, \alpha \cdot f^T)$ ($Z$-type operators)
\end{itemize}
\end{definition}

The Gross code exhibits several remarkable structural properties. The number 144 indeed equals $12 \times 12$, justifying the name ``gross'' (a dozen dozens). The symmetry between $X$ and $Z$ operators is manifest in the construction: the operators $\bar{X}_\alpha$ and $\bar{Z}'_\alpha$ are related by transposition and component swapping, as are $\bar{X}'_\beta$ and $\bar{Z}_\beta$. This symmetry reflects the self-dual nature of the code construction and ensures that the $X$ and $Z$ stabilizers have equivalent geometric properties. The careful choice of the polynomials $f$, $g$, and $h$ guarantees that the resulting logical operators anticommute appropriately and that the code achieves its claimed minimum distance of 12.

\vspace{1em}

\subsection{Remark 22: GrossCodeGaugingExample}

Looking at this Lean formalization, I need to write Remark 22 about the Gross Code Gauging Example. This appears to be a comprehensive example demonstrating the efficiency of a gauging approach for quantum error correction.

\section*{The Gauging Approach to Logical Operator Measurement}

One of the central challenges in quantum error correction is efficiently measuring logical operators without introducing excessive overhead. Traditional approaches to measuring logical operators often require resources that scale as $O(Wd)$, where $W$ is the weight of the logical operator and $d$ is the code distance. For quantum codes with large distances, this overhead can become prohibitive.

The gauging approach offers a fundamentally different strategy. Instead of directly measuring the logical operator, we introduce auxiliary "gauge" qubits and construct a graph-based framework that exploits the existing stabilizer structure of the code. This method can dramatically reduce the required overhead while maintaining the same measurement accuracy.

\begin{remark}[Remark 22: Gross Code Gauging Example]
We demonstrate the efficiency of the gauging approach through a concrete analysis of measuring the logical operator $\bar{X}_\alpha$ in the Gross code. The Gross code has parameters $n = 144$, $k = 12$, and distance $d = 12$. We construct a specific gauging graph and compare its overhead to traditional measurement schemes.

\textbf{Gauging Graph Construction:} For the logical operator $\bar{X}_\alpha = X(\alpha f, 0)$ where $f$ contains exactly $W = 12$ monomials, we construct a gauging graph with:
\begin{itemize}
\item $|V| = 12$ vertices (one per monomial in $f$)
\item $18$ initial edges from existing $Z$-check connectivity  
\item $4$ additional expansion edges for connectivity
\item Total of $|E| = 22$ edges
\end{itemize}

\textbf{Overhead Analysis:} The total overhead consists of:
\begin{itemize}
\item $12$ new $X$ checks (Gauss's law operators $A_v$, one per vertex)
\item $7$ new $Z$ checks (flux operators $B_p$ after redundancy elimination)
\item $22$ new gauge qubits (one per edge)
\end{itemize}

This gives a total overhead of $12 + 7 + 22 = 41$ elements.

\textbf{Efficiency Comparison:} Traditional schemes require overhead scaling as $O(Wd) = O(12 \times 12) = 144$ elements. The gauging approach achieves the same measurement with only $41$ elements, representing a reduction of more than $70\%$ and demonstrating that the overhead is less than $29\%$ of the traditional approach.
\end{remark}

This example illustrates the practical advantages of the gauging approach. The key insight is that we can leverage the existing stabilizer structure to reduce redundancy in the measurement protocol. The $4$ redundant cycles identified in the analysis correspond to dependencies that arise from the underlying BB code structure, allowing us to eliminate $4$ of the $11$ independent cycles that would naively require flux checks.

The degree bounds are also favorable: new elements have Tanner graph degree at most $6$, while affected existing elements (the $12$ qubits in $\bar{X}_\alpha$ and the $18$ adjacent $Z$ checks) have degree at most $7$, representing only a modest increase from the original degree of $6$. This controlled degree growth ensures that the gauging construction does not significantly complicate the physical implementation of the stabilizer measurements.

\vspace{1em}

\end{document}
