%==============================================================================
% Auto-generated by autoinformalization
% Library: QEC1
% Generated: 2026-02-12 08:49:18
%==============================================================================


% COMMENTED OUT: Disconnected cluster (charVec notation, not connected to main network)
% %--- Rem_1: NotationBinaryVectors ---
% \chapter{Rem 1: Notation -- Binary Vectors}
%
% Throughout this work we use binary vectors over $\mathbb{Z}_2$ (equivalently $\mathbb{F}_2$) to indicate collections of vertices, edges, and cycles of a graph $G$. We identify the binary vector associated to a set of vertices, edges, or cycles with the set itself. Addition of binary vectors corresponds to symmetric difference of sets.
%
% \begin{definition}[Characteristic Vector]
% \label{def:charVec}
% \lean{QEC1.charVec}
% \leanok
%
% Let $\alpha$ be a type. The \emph{characteristic vector} of a set $S \subseteq \alpha$ is the function $\chi_S : \alpha \to \mathbb{Z}_2$ defined by
% \[
% \chi_S(v) = \begin{cases} 1 & \text{if } v \in S, \\ 0 & \text{otherwise,} \end{cases}
% \]
% using the set indicator function with constant value $1$.
% \end{definition}

% \begin{theorem}[Characteristic Vector of a Member]
% \label{thm:charVec_apply_of_mem}
% ... (13 theorems/corollaries omitted — all part of charVec disconnected cluster)
% \end{theorem}
% ... through charVec_union

% COMMENTED OUT: Isolated declarations (not connected to dependency graph)
% \begin{definition}[Vertex Set of a Graph]
% \label{def:vertexSetOf}
% \lean{QEC1.vertexSetOf}
% \leanok
%
% For a simple graph $G$ on vertex type $V$, the \emph{vertex set} is the universal set $\operatorname{Set.univ} : \operatorname{Set}\, V$.
% \end{definition}
%
% \begin{definition}[Edge Set of a Graph]
% \label{def:edgeSetOf}
% \lean{QEC1.edgeSetOf}
% \leanok
%
% For a simple graph $G$ on vertex type $V$ with edge type $E$, the \emph{edge set} is the universal set $\operatorname{Set.univ} : \operatorname{Set}\, E$.
% \end{definition}

%--- Rem_2: NotationPauliOperators ---
\chapter{Rem 2: Notation and Pauli Operators}

For qubits labeled by vertices $v$ of a graph or indices $i$, we denote by $X_v$ (or $X_i$)
the Pauli-$X$ operator acting on qubit $v$ (or $i$), and similarly $Z_v$ (or $Z_i$) for Pauli-$Z$.
A product of Pauli operators is written as $\prod_{v \in S} X_v$ for a set $S$ of qubit labels.
The identity operator is denoted $\mathbb{1}$. For a Pauli operator $P$, we denote by $S_X(P)$ the
$X$-type support (sites where $P$ acts via $X$ or $Y$) and $S_Z(P)$ the $Z$-type support (sites
where $P$ acts via $Y$ or $Z$).

\section{Definition of Pauli Operators}

\begin{definition}[Pauli Operator]
\label{def:PauliOp}
\lean{QEC1.PauliOp}
\leanok

A \emph{Pauli operator} on qubits labeled by a type $V$ is represented as a pair of binary vectors
$(\mathtt{xVec}, \mathtt{zVec}) \in (\mathbb{Z}/2\mathbb{Z})^V \times (\mathbb{Z}/2\mathbb{Z})^V$.
The pair $(x, z)$ represents the Pauli operator $\bigotimes_v X_v^{x_v} Z_v^{z_v}$.
At each site $v$:
\begin{itemize}
  \item $(0, 0)$ means the identity $I$,
  \item $(1, 0)$ means $X$,
  \item $(0, 1)$ means $Z$,
  \item $(1, 1)$ means $Y$ (up to phase).
\end{itemize}
\end{definition}

\begin{definition}[Identity Pauli Operator]
\label{def:PauliOp.id}
\lean{QEC1.PauliOp.id}
\leanok
\uses{def:PauliOp}
The \emph{identity Pauli operator} on qubits labeled by $V$ is defined as
$\operatorname{id}(V) := (0, 0)$,
i.e., the operator that acts as the identity on all qubits.
\end{definition}

\begin{definition}[Pauli-$X$ Operator]
\label{def:PauliOp.pauliX}
\lean{QEC1.PauliOp.pauliX}
\leanok
\uses{def:PauliOp}
For a qubit label $v \in V$, the \emph{Pauli-$X$ operator} acting on qubit $v$ is defined as
$X_v := (\delta_v, 0)$,
where $\delta_v$ is the indicator function that is $1$ at $v$ and $0$ elsewhere.
\end{definition}

\begin{definition}[Pauli-$Z$ Operator]
\label{def:PauliOp.pauliZ}
\lean{QEC1.PauliOp.pauliZ}
\leanok
\uses{def:PauliOp}
For a qubit label $v \in V$, the \emph{Pauli-$Z$ operator} acting on qubit $v$ is defined as
$Z_v := (0, \delta_v)$,
where $\delta_v$ is the indicator function that is $1$ at $v$ and $0$ elsewhere.
\end{definition}

\begin{definition}[Pauli-$Y$ Operator]
\label{def:PauliOp.pauliY}
\lean{QEC1.PauliOp.pauliY}
\leanok
\uses{def:PauliOp}
For a qubit label $v \in V$, the \emph{Pauli-$Y$ operator} acting on qubit $v$ is defined as
$Y_v := (\delta_v, \delta_v)$,
representing $X_v Z_v$ up to phase.
\end{definition}

\section{Multiplication}

\begin{definition}[Pauli Operator Multiplication]
\label{def:PauliOp.mul}
\lean{QEC1.PauliOp.mul}
\leanok
\uses{def:PauliOp}
The \emph{product} of two Pauli operators $P = (x_P, z_P)$ and $Q = (x_Q, z_Q)$ is defined by
pointwise addition in $\mathbb{Z}/2\mathbb{Z}$:
\[
  P \cdot Q := (x_P + x_Q, \; z_P + z_Q).
\]
This captures the Pauli group multiplication up to phase.
\end{definition}

\begin{lemma}[Commutativity of Pauli Multiplication]
\label{lem:PauliOp.mul_comm}
\lean{QEC1.PauliOp.mul_comm}
\leanok
\uses{def:PauliOp, def:PauliOp.mul}
For all Pauli operators $P, Q$, we have $P \cdot Q = Q \cdot P$.
\end{lemma}

\begin{proof}
\leanok
\uses{def:PauliOp.mul}
By extensionality, it suffices to show equality at each component for each $v$.
By simplification using commutativity of addition in $\mathbb{Z}/2\mathbb{Z}$, both the
$x$-component and $z$-component are equal.
\end{proof}

\begin{lemma}[Associativity of Pauli Multiplication]
\label{lem:PauliOp.mul_assoc}
\lean{QEC1.PauliOp.mul_assoc}
\leanok
\uses{def:PauliOp, def:PauliOp.mul}
For all Pauli operators $P, Q, R$, we have $(P \cdot Q) \cdot R = P \cdot (Q \cdot R)$.
\end{lemma}

\begin{proof}
\leanok
\uses{def:PauliOp.mul}
By extensionality, it suffices to show equality at each component for each $v$.
By simplification using associativity of addition in $\mathbb{Z}/2\mathbb{Z}$, both the
$x$-component and $z$-component are equal.
\end{proof}

\begin{lemma}[Left Identity]
\label{lem:PauliOp.one_mul}
\lean{QEC1.PauliOp.one_mul}
\leanok
\uses{def:PauliOp, def:PauliOp.mul}
For all Pauli operators $P$, we have $1 \cdot P = P$.
\end{lemma}

\begin{proof}
\leanok
\uses{def:PauliOp.mul}
By extensionality, it suffices to show equality at each component for each $v$.
By simplification (since $0 + a = a$ in $\mathbb{Z}/2\mathbb{Z}$), this holds.
\end{proof}

\begin{lemma}[Right Identity]
\label{lem:PauliOp.mul_one}
\lean{QEC1.PauliOp.mul_one}
\leanok
\uses{def:PauliOp, def:PauliOp.mul}
For all Pauli operators $P$, we have $P \cdot 1 = P$.
\end{lemma}

\begin{proof}
\leanok
\uses{def:PauliOp.mul}
By extensionality, it suffices to show equality at each component for each $v$.
By simplification (since $a + 0 = a$ in $\mathbb{Z}/2\mathbb{Z}$), this holds.
\end{proof}

\begin{lemma}[Self-Inverse Property]
\label{lem:PauliOp.mul_self}
\lean{QEC1.PauliOp.mul_self}
\leanok
\uses{def:PauliOp, def:PauliOp.mul}
For all Pauli operators $P$, we have $P \cdot P = 1$.
\end{lemma}

\begin{proof}
\leanok
\uses{def:PauliOp.mul}
By extensionality, it suffices to show equality at each component for each $v$.
By simplification using the characteristic-two property $a + a = 0$ in $\mathbb{Z}/2\mathbb{Z}$, both the $x$-component and $z$-component equal $0$.
\end{proof}

\section{Products over Finite Sets}

\begin{definition}[Product of Pauli-$X$ Operators]
\label{def:PauliOp.prodX}
\lean{QEC1.PauliOp.prodX}
\leanok
\uses{def:PauliOp}
For a finite set $S \subseteq V$, the \emph{product of Pauli-$X$ operators} over $S$ is
\[
  \prod_{v \in S} X_v := \Bigl(\mathbf{1}_S, \; 0\Bigr),
\]
where $\mathbf{1}_S(v) = 1$ if $v \in S$ and $0$ otherwise.
\end{definition}

\begin{definition}[Product of Pauli-$Z$ Operators]
\label{def:PauliOp.prodZ}
\lean{QEC1.PauliOp.prodZ}
\leanok
\uses{def:PauliOp}
For a finite set $S \subseteq V$, the \emph{product of Pauli-$Z$ operators} over $S$ is
\[
  \prod_{v \in S} Z_v := \Bigl(0, \; \mathbf{1}_S\Bigr),
\]
where $\mathbf{1}_S(v) = 1$ if $v \in S$ and $0$ otherwise.
\end{definition}

\section{Supports}

\begin{definition}[$X$-type Support]
\label{def:PauliOp.supportX}
\lean{QEC1.PauliOp.supportX}
\leanok
\uses{def:PauliOp}
The \emph{$X$-type support} of a Pauli operator $P = (x, z)$ is the set of sites where $P$ acts via $X$ or $Y$:
\[
  S_X(P) := \{v \in V \mid x_v \neq 0\}.
\]
\end{definition}

\begin{definition}[$Z$-type Support]
\label{def:PauliOp.supportZ}
\lean{QEC1.PauliOp.supportZ}
\leanok
\uses{def:PauliOp}
The \emph{$Z$-type support} of a Pauli operator $P = (x, z)$ is the set of sites where $P$ acts via $Y$ or $Z$:
\[
  S_Z(P) := \{v \in V \mid z_v \neq 0\}.
\]
\end{definition}

\begin{definition}[Full Support]
\label{def:PauliOp.support}
\lean{QEC1.PauliOp.support}
\leanok
\uses{def:PauliOp}
The \emph{full support} of a Pauli operator $P = (x, z)$ is the set of sites where $P$ acts non-trivially:
\[
  S(P) := \{v \in V \mid x_v \neq 0 \text{ or } z_v \neq 0\}.
\]
\end{definition}

\begin{definition}[Weight]
\label{def:PauliOp.weight}
\lean{QEC1.PauliOp.weight}
\leanok
\uses{def:PauliOp, def:PauliOp.support}
The \emph{weight} of a Pauli operator $P$ is the number of qubits on which it acts non-trivially:
\[
  \operatorname{wt}(P) := |S(P)|.
\]
\end{definition}

\section{Support Characterizations}

\begin{lemma}[$X$-Support Membership in $\mathbb{Z}/2\mathbb{Z}$]
\label{lem:PauliOp.mem_supportX_iff}
\lean{QEC1.PauliOp.mem_supportX_iff}
\leanok
\uses{def:PauliOp, def:PauliOp.supportX}
For a Pauli operator $P$ and vertex $v$,
\[
  v \in S_X(P) \iff x_v = 1.
\]
\end{lemma}

\begin{proof}
\leanok
\uses{def:PauliOp.supportX}
We rewrite using the membership characterization of $S_X$ (i.e., $v \in S_X(P) \iff x_v \neq 0$), then apply the fact that in $\mathbb{Z}/2\mathbb{Z}$, $a \neq 0 \iff a = 1$.
\end{proof}

\begin{lemma}[$Z$-Support Membership in $\mathbb{Z}/2\mathbb{Z}$]
\label{lem:PauliOp.mem_supportZ_iff}
\lean{QEC1.PauliOp.mem_supportZ_iff}
\leanok
\uses{def:PauliOp, def:PauliOp.supportZ}
For a Pauli operator $P$ and vertex $v$,
\[
  v \in S_Z(P) \iff z_v = 1.
\]
\end{lemma}

\begin{proof}
\leanok
\uses{def:PauliOp.supportZ}
We rewrite using the membership characterization of $S_Z$ (i.e., $v \in S_Z(P) \iff z_v \neq 0$), then apply the fact that in $\mathbb{Z}/2\mathbb{Z}$, $a \neq 0 \iff a = 1$.
\end{proof}

\section{Identity Support}

\begin{lemma}[$X$-Support of Identity]
\label{lem:PauliOp.supportX_one}
\lean{QEC1.PauliOp.supportX_one}
\leanok
\uses{def:PauliOp, def:PauliOp.supportX}
The $X$-type support of the identity operator is empty:
$S_X(\mathbb{1}) = \emptyset$.
\end{lemma}

\begin{proof}
\leanok
\uses{def:PauliOp.supportX}
By extensionality over $v$ and simplification: the $x$-component of the identity is $0$ everywhere, so no vertex satisfies the membership condition.
\end{proof}

\begin{lemma}[$Z$-Support of Identity]
\label{lem:PauliOp.supportZ_one}
\lean{QEC1.PauliOp.supportZ_one}
\leanok
\uses{def:PauliOp, def:PauliOp.supportZ}
The $Z$-type support of the identity operator is empty:
$S_Z(\mathbb{1}) = \emptyset$.
\end{lemma}

\begin{proof}
\leanok
\uses{def:PauliOp.supportZ}
By extensionality over $v$ and simplification: the $z$-component of the identity is $0$ everywhere, so no vertex satisfies the membership condition.
\end{proof}

\begin{lemma}[Full Support of Identity]
\label{lem:PauliOp.support_one}
\lean{QEC1.PauliOp.support_one}
\leanok
\uses{def:PauliOp, def:PauliOp.support}
The full support of the identity operator is empty:
$S(\mathbb{1}) = \emptyset$.
\end{lemma}

\begin{proof}
\leanok
\uses{def:PauliOp.support}
By extensionality over $v$ and simplification: both components of the identity are $0$ everywhere, so neither condition for membership holds.
\end{proof}

\begin{lemma}[Weight of Identity]
\label{lem:PauliOp.weight_one}
\lean{QEC1.PauliOp.weight_one}
\leanok
\uses{def:PauliOp, def:PauliOp.weight}
The weight of the identity operator is zero:
$\operatorname{wt}(\mathbb{1}) = 0$.
\end{lemma}

\begin{proof}
\leanok
\uses{def:PauliOp.weight, lem:PauliOp.support_one}
By simplification using the definition of weight and the fact that $S(\mathbb{1}) = \emptyset$, we get $|\emptyset| = 0$.
\end{proof}

\section{Single-Site Operator Supports}

\begin{lemma}[$X$-Support of $X_v$]
\label{lem:PauliOp.supportX_pauliX}
\lean{QEC1.PauliOp.supportX_pauliX}
\leanok
\uses{def:PauliOp, def:PauliOp.pauliX, def:PauliOp.supportX}
For any qubit label $v$, the $X$-type support of $X_v$ is $\{v\}$:
$S_X(X_v) = \{v\}$.
\end{lemma}

\begin{proof}
\leanok
\uses{def:PauliOp.pauliX, def:PauliOp.supportX}
By extensionality over $w$, we simplify using the definitions of $X_v$ and $S_X$, together with the behavior of the indicator function $\delta_v$: $\delta_v(w) \neq 0 \iff w = v$.
\end{proof}

\begin{lemma}[$Z$-Support of $X_v$]
\label{lem:PauliOp.supportZ_pauliX}
\lean{QEC1.PauliOp.supportZ_pauliX}
\leanok
\uses{def:PauliOp, def:PauliOp.pauliX, def:PauliOp.supportZ}
For any qubit label $v$, the $Z$-type support of $X_v$ is empty:
$S_Z(X_v) = \emptyset$.
\end{lemma}

\begin{proof}
\leanok
\uses{def:PauliOp.pauliX, def:PauliOp.supportZ}
By extensionality over $w$ and simplification: the $z$-component of$X_v$ is $0$ everywhere.
\end{proof}

\begin{lemma}[$X$-Support of $Z_v$]
\label{lem:PauliOp.supportX_pauliZ}
\lean{QEC1.PauliOp.supportX_pauliZ}
\leanok
\uses{def:PauliOp, def:PauliOp.pauliZ, def:PauliOp.supportX}
For any qubit label $v$, the $X$-type support of $Z_v$ is empty:
$S_X(Z_v) = \emptyset$.
\end{lemma}

\begin{proof}
\leanok
\uses{def:PauliOp.pauliZ, def:PauliOp.supportX}
By extensionality over $w$ and simplification: the $x$-component of $Z_v$ is $0$ everywhere.
\end{proof}

\begin{lemma}[$Z$-Support of $Z_v$]
\label{lem:PauliOp.supportZ_pauliZ}
\lean{QEC1.PauliOp.supportZ_pauliZ}
\leanok
\uses{def:PauliOp, def:PauliOp.pauliZ, def:PauliOp.supportZ}
For any qubit label $v$, the $Z$-type support of $Z_v$ is $\{v\}$:
$S_Z(Z_v) = \{v\}$.
\end{lemma}

\begin{proof}
\leanok
\uses{def:PauliOp.pauliZ, def:PauliOp.supportZ}
By extensionality over $w$, we simplify using the definitions of $Z_v$ and $S_Z$, together with the behavior of the indicator function $\delta_v$: $\delta_v(w) \neq 0 \iff w = v$.
\end{proof}

\begin{lemma}[$X$-Support of $Y_v$]
\label{lem:PauliOp.supportX_pauliY}
\lean{QEC1.PauliOp.supportX_pauliY}
\leanok
\uses{def:PauliOp, def:PauliOp.pauliY, def:PauliOp.supportX}
For any qubit label $v$, the $X$-type support of $Y_v$ is $\{v\}$:
$S_X(Y_v) = \{v\}$.
\end{lemma}

\begin{proof}
\leanok
\uses{def:PauliOp.pauliY, def:PauliOp.supportX}
By extensionality over $w$, we simplify using the definitions of $Y_v$ and $S_X$, together with the behavior of the indicator function $\delta_v$.
\end{proof}

\begin{lemma}[$Z$-Support of $Y_v$]
\label{lem:PauliOp.supportZ_pauliY}
\lean{QEC1.PauliOp.supportZ_pauliY}
\leanok
\uses{def:PauliOp, def:PauliOp.pauliY, def:PauliOp.supportZ}
For any qubit label $v$, the $Z$-type support of $Y_v$ is $\{v\}$:
$S_Z(Y_v) = \{v\}$.
\end{lemma}

\begin{proof}
\leanok
\uses{def:PauliOp.pauliY, def:PauliOp.supportZ}
By extensionality over $w$, we simplify using the definitions of $Y_v$ and $S_Z$, together with the behavior of the indicator function $\delta_v$.
\end{proof}

\section{Product Support Characterizations}

\begin{lemma}[$X$-Support of $\prod X$]
\label{lem:PauliOp.supportX_prodX}
\lean{QEC1.PauliOp.supportX_prodX}
\leanok
\uses{def:PauliOp, def:PauliOp.prodX, def:PauliOp.supportX}
For a finite set $S$, the $X$-type support of $\prod_{v \in S} X_v$ is $S$ itself:
$S_X\!\left(\prod_{v \in S} X_v\right) = S$.
\end{lemma}

\begin{proof}
\leanok
\uses{def:PauliOp.prodX, def:PauliOp.supportX}
By extensionality over $v$ and simplification using the definitions of $\operatorname{prodX}$ and $S_X$: the $x$-component of $\operatorname{prodX}(S)$ at $v$ is $1$ if and only if $v \in S$.
\end{proof}

\begin{lemma}[$Z$-Support of $\prod X$]
\label{lem:PauliOp.supportZ_prodX}
\lean{QEC1.PauliOp.supportZ_prodX}
\leanok
\uses{def:PauliOp, def:PauliOp.prodX, def:PauliOp.supportZ}
For a finite set $S$, the $Z$-type support of $\prod_{v \in S} X_v$ is empty:
$S_Z\!\left(\prod_{v \in S} X_v\right) = \emptyset$.
\end{lemma}

\begin{proof}
\leanok
\uses{def:PauliOp.prodX, def:PauliOp.supportZ}
By extensionality over $v$ and simplification: the $z$-component of $\operatorname{prodX}(S)$ is $0$ everywhere.
\end{proof}

\begin{lemma}[$X$-Support of $\prod Z$]
\label{lem:PauliOp.supportX_prodZ}
\lean{QEC1.PauliOp.supportX_prodZ}
\leanok
\uses{def:PauliOp, def:PauliOp.prodZ, def:PauliOp.supportX}
For a finite set $S$, the $X$-type support of $\prod_{v \in S} Z_v$ is empty:
$S_X\!\left(\prod_{v \in S} Z_v\right) = \emptyset$.
\end{lemma}

\begin{proof}
\leanok
\uses{def:PauliOp.prodZ, def:PauliOp.supportX}
By extensionality over $v$ and simplification: the $x$-component of $\operatorname{prodZ}(S)$ is $0$ everywhere.
\end{proof}

\begin{lemma}[$Z$-Support of $\prod Z$]
\label{lem:PauliOp.supportZ_prodZ}
\lean{QEC1.PauliOp.supportZ_prodZ}
\leanok
\uses{def:PauliOp, def:PauliOp.prodZ, def:PauliOp.supportZ}
For a finite set $S$, the $Z$-type support of $\prod_{v \in S} Z_v$ is $S$ itself:
$S_Z\!\left(\prod_{v \in S} Z_v\right) = S$.
\end{lemma}

\begin{proof}
\leanok
\uses{def:PauliOp.prodZ, def:PauliOp.supportZ}
By extensionality over $v$ and simplification using the definitions of $\operatorname{prodZ}$ and $S_Z$: the $z$-component of $\operatorname{prodZ}(S)$ at $v$ is $1$ if and only if $v \in S$.
\end{proof}

\section{Support Union Characterization}

\begin{lemma}[Support Equals Union of $X$- and $Z$-Supports]
\label{lem:PauliOp.support_eq_supportX_union_supportZ}
\lean{QEC1.PauliOp.support_eq_supportX_union_supportZ}
\leanok
\uses{def:PauliOp, def:PauliOp.support, def:PauliOp.supportX, def:PauliOp.supportZ}
For any Pauli operator $P$,
\[
  S(P) = S_X(P) \cup S_Z(P).
\]
\end{lemma}

\begin{proof}
\leanok
\uses{def:PauliOp.support, def:PauliOp.supportX, def:PauliOp.supportZ}
By extensionality over $v$ and simplification using the definitions of $S$, $S_X$, and $S_Z$: a vertex $v$ belongs to $S(P)$ if and only if $x_v \neq 0$ or $z_v \neq 0$, which is exactly the condition for membership in $S_X(P) \cup S_Z(P)$.
\end{proof}

\section{Single-Site and Product Relationships}

\begin{lemma}[$X_v$ Equals Product over Singleton]
\label{lem:PauliOp.pauliX_eq_prodX_singleton}
\lean{QEC1.PauliOp.pauliX_eq_prodX_singleton}
\leanok
\uses{def:PauliOp, def:PauliOp.pauliX, def:PauliOp.prodX}
For any qubit label $v$,
\[
  X_v = \prod_{w \in \{v\}} X_w.
\]
\end{lemma}

\begin{proof}
\leanok
\uses{def:PauliOp.pauliX, def:PauliOp.prodX}
By extensionality over $w$ for both components, we simplify using the definitions of $X_v$ and $\operatorname{prodX}(\{v\})$, together with the behavior of the indicator function: $\delta_v(w) = \mathbf{1}_{\{v\}}(w)$.
\end{proof}

\begin{lemma}[$Z_v$ Equals Product over Singleton]
\label{lem:PauliOp.pauliZ_eq_prodZ_singleton}
\lean{QEC1.PauliOp.pauliZ_eq_prodZ_singleton}
\leanok
\uses{def:PauliOp, def:PauliOp.pauliZ, def:PauliOp.prodZ}
For any qubit label $v$,
\[
  Z_v = \prod_{w \in \{v\}} Z_w.
\]
\end{lemma}

\begin{proof}
\leanok
\uses{def:PauliOp.pauliZ, def:PauliOp.prodZ}
By extensionality over $w$ for both components, we simplify using the definitions of $Z_v$ and $\operatorname{prodZ}(\{v\})$, together with the behavior of the indicator function: $\delta_v(w) = \mathbf{1}_{\{v\}}(w)$.
\end{proof}

%--- Rem_3: NotationStabilizerCodes ---
\chapter{Rem 3: Notation for Stabilizer Codes}

\begin{definition}[Symplectic Inner Product]
\label{def:PauliOp.symplecticInner}
\lean{QEC1.PauliOp.symplecticInner}
\leanok
\uses{def:PauliOp, def:PauliOp.mul}
Let $V$ be a finite type. The \emph{symplectic inner product} of two Pauli operators $P, Q : \operatorname{PauliOp}(V)$ is defined as
\[
\langle P, Q \rangle_{\mathrm{symp}} \;=\; \sum_{v \in V} \bigl(P.x_v \cdot Q.z_v + P.z_v \cdot Q.x_v\bigr) \;\in\; \mathbb{Z}/2\mathbb{Z}.
\]
This determines commutation in the full Pauli group: $P$ and $Q$ commute if and only if $\langle P,Q\rangle_{\mathrm{symp}} = 0$.
\end{definition}

\begin{definition}[Pauli Commute]
\label{def:PauliOp.PauliCommute}
\lean{QEC1.PauliOp.PauliCommute}
\leanok
\uses{def:PauliOp.symplecticInner}
Two Pauli operators $P, Q : \operatorname{PauliOp}(V)$ \emph{commute} (in the full Pauli group, including phases) if their symplectic inner product vanishes:
\[
\operatorname{PauliCommute}(P, Q) \;\iff\; \langle P, Q \rangle_{\mathrm{symp}} = 0.
\]
\end{definition}

\begin{theorem}[Symplectic Inner Product with Self]
\label{thm:PauliOp.symplecticInner_self}
\lean{QEC1.PauliOp.symplecticInner_self}
\leanok
\uses{def:PauliOp.symplecticInner}
For any Pauli operator $P : \operatorname{PauliOp}(V)$,
\[
\langle P, P \rangle_{\mathrm{symp}} = 0.
\]
\end{theorem}

\begin{proof}
\leanok
\uses{def:PauliOp.symplecticInner}
Unfolding the definition of $\langle P, P \rangle_{\mathrm{symp}}$, we have
\[
\langle P, P \rangle_{\mathrm{symp}} = \sum_{v \in V} (P.x_v \cdot P.z_v + P.z_v \cdot P.x_v).
\]
It suffices to show each summand is zero. For each $v$, by commutativity of multiplication in $\mathbb{Z}/2\mathbb{Z}$ we have $P.z_v \cdot P.x_v = P.x_v \cdot P.z_v$, so the summand equals $P.x_v \cdot P.z_v + P.x_v \cdot P.z_v = 0$ since every element of $\mathbb{Z}/2\mathbb{Z}$ satisfies $a + a = 0$. Hence the entire sum is zero.
\end{proof}

\begin{theorem}[Pauli Commute Self]
\label{thm:PauliOp.pauliCommute_self}
\lean{QEC1.PauliOp.pauliCommute_self}
\leanok
\uses{def:PauliOp.PauliCommute, thm:PauliOp.symplecticInner_self}
Every Pauli operator commutes with itself: for any $P : \operatorname{PauliOp}(V)$,
\[
\operatorname{PauliCommute}(P, P).
\]
\end{theorem}

\begin{proof}
\leanok
\uses{thm:PauliOp.symplecticInner_self}
Unfolding the definition of $\operatorname{PauliCommute}$, the goal reduces to $\langle P, P \rangle_{\mathrm{symp}} = 0$, which holds by simplification using the theorem \texttt{symplecticInner\_self}.
\end{proof}

\begin{theorem}[Pauli Commute is Symmetric]
\label{thm:PauliOp.pauliCommute_comm}
\lean{QEC1.PauliOp.pauliCommute_comm}
\leanok
\uses{def:PauliOp.PauliCommute, def:PauliOp.symplecticInner}
For any Pauli operators $P, Q : \operatorname{PauliOp}(V)$,
\[
\operatorname{PauliCommute}(P, Q) \;\iff\; \operatorname{PauliCommute}(Q, P).
\]
\end{theorem}

\begin{proof}
\leanok
\uses{def:PauliOp.PauliCommute, def:PauliOp.symplecticInner}
Unfolding the definitions, both directions amount to showing $\langle P,Q \rangle_{\mathrm{symp}} = 0 \iff \langle Q,P \rangle_{\mathrm{symp}} = 0$. For each direction, we convert the goal using the fact that at each coordinate $v$, the summand $P.x_v \cdot Q.z_v + P.z_v \cdot Q.x_v$ equals $Q.x_v \cdot P.z_v + Q.z_v \cdot P.x_v$ by ring arithmetic. Hence the two sums are equal, and the equivalence follows.
\end{proof}

\begin{theorem}[Symplectic Inner Product with Identity on Left]
\label{thm:PauliOp.symplecticInner_one_left}
\lean{QEC1.PauliOp.symplecticInner_one_left}
\leanok
\uses{def:PauliOp.symplecticInner, def:PauliOp.id}
For any Pauli operator $Q : \operatorname{PauliOp}(V)$,
\[
\langle 1, Q \rangle_{\mathrm{symp}} = 0.
\]
\end{theorem}

\begin{proof}
\leanok
\uses{def:PauliOp.symplecticInner, def:PauliOp.id}
By simplification: the identity operator has $x_v = 0$ and $z_v = 0$ for all $v$, so each summand $1.x_v \cdot Q.z_v + 1.z_v \cdot Q.x_v = 0 \cdot Q.z_v + 0 \cdot Q.x_v = 0$. The sum of zeros is zero.
\end{proof}

\begin{theorem}[Symplectic Inner Product with Identity on Right]
\label{thm:PauliOp.symplecticInner_one_right}
\lean{QEC1.PauliOp.symplecticInner_one_right}
\leanok
\uses{def:PauliOp.symplecticInner, def:PauliOp.id}
For any Pauli operator $P : \operatorname{PauliOp}(V)$,
\[
\langle P, 1 \rangle_{\mathrm{symp}} = 0.
\]
\end{theorem}

\begin{proof}
\leanok
\uses{def:PauliOp.symplecticInner, def:PauliOp.id}
By simplification: the identity operator has $x_v = 0$ and $z_v = 0$ for all $v$, so each summand $P.x_v \cdot 0 + P.z_v \cdot 0 = 0$. The sum of zeros is zero.
\end{proof}

\begin{theorem}[Identity Commutes on Left]
\label{thm:PauliOp.pauliCommute_one_left}
\lean{QEC1.PauliOp.pauliCommute_one_left}
\leanok
\uses{def:PauliOp.PauliCommute, thm:PauliOp.symplecticInner_one_left}
For any Pauli operator $Q : \operatorname{PauliOp}(V)$, $\operatorname{PauliCommute}(1, Q)$.
\end{theorem}

\begin{proof}
\leanok
\uses{thm:PauliOp.symplecticInner_one_left}
Unfolding $\operatorname{PauliCommute}$, the goal is $\langle 1, Q \rangle_{\mathrm{symp}} = 0$, which holds by simplification.
\end{proof}

\begin{theorem}[Identity Commutes on Right]
\label{thm:PauliOp.pauliCommute_one_right}
\lean{QEC1.PauliOp.pauliCommute_one_right}
\leanok
\uses{def:PauliOp.PauliCommute, thm:PauliOp.symplecticInner_one_right}
For any Pauli operator $P : \operatorname{PauliOp}(V)$, $\operatorname{PauliCommute}(P, 1)$.
\end{theorem}

\begin{proof}
\leanok
\uses{thm:PauliOp.symplecticInner_one_right}
Unfolding $\operatorname{PauliCommute}$, the goal is $\langle P, 1 \rangle_{\mathrm{symp}} = 0$, which holds by simplification.
\end{proof}

\begin{theorem}[Symplectic Inner Product is Additive on Left]
\label{thm:PauliOp.symplecticInner_mul_left}
\lean{QEC1.PauliOp.symplecticInner_mul_left}
\leanok
\uses{def:PauliOp.symplecticInner, def:PauliOp.mul}
For any Pauli operators $P, Q, R : \operatorname{PauliOp}(V)$,
\[
\langle P \cdot Q, R \rangle_{\mathrm{symp}} = \langle P, R \rangle_{\mathrm{symp}} + \langle Q, R \rangle_{\mathrm{symp}}.
\]
\end{theorem}

\begin{proof}
\leanok
\uses{def:PauliOp.symplecticInner, def:PauliOp.mul}
Expanding the definitions of the symplectic inner product and multiplication of Pauli operators (where $(P \cdot Q).x_v = P.x_v + Q.x_v$ and $(P \cdot Q).z_v = P.z_v + Q.z_v$ in $\mathbb{Z}/2\mathbb{Z}$), the left-hand side becomes
\[
\sum_v \bigl((P.x_v + Q.x_v) \cdot R.z_v + (P.z_v + Q.z_v) \cdot R.x_v\bigr).
\]
Distributing the sum and applying the distributive law of $\mathbb{Z}/2\mathbb{Z}$, this equals the sum of the right-hand side terms. The equality of the summands at each coordinate $v$ follows by ring arithmetic.
\end{proof}

\begin{theorem}[Symplectic Inner Product is Additive on Right]
\label{thm:PauliOp.symplecticInner_mul_right}
\lean{QEC1.PauliOp.symplecticInner_mul_right}
\leanok
\uses{def:PauliOp.symplecticInner, def:PauliOp.mul}
For any Pauli operators $P, Q, R : \operatorname{PauliOp}(V)$,
\[
\langle P, Q \cdot R \rangle_{\mathrm{symp}} = \langle P, Q \rangle_{\mathrm{symp}} + \langle P, R \rangle_{\mathrm{symp}}.
\]
\end{theorem}

\begin{proof}
\leanok
\uses{def:PauliOp.symplecticInner, def:PauliOp.mul}
Expanding the definitions, the left-hand side becomes
\[
\sum_v \bigl(P.x_v \cdot (Q.z_v + R.z_v) + P.z_v \cdot (Q.x_v + R.x_v)\bigr).
\]
Distributing and rearranging the sum, this equals the right-hand side. The equality of summands at each coordinate follows by ring arithmetic.
\end{proof}

\begin{definition}[Stabilizer Code]
\label{def:StabilizerCode}
\lean{QEC1.StabilizerCode}
\leanok
\uses{def:PauliOp, def:PauliOp.PauliCommute}
A \emph{stabilizer code} on qubits labeled by a finite type $V$ consists of:
\begin{itemize}
  \item A finite index type $I$ for stabilizer checks,
  \item A map $\operatorname{check} : I \to \operatorname{PauliOp}(V)$ assigning a Pauli operator to each check index,
  \item A commutativity condition: for all $i, j \in I$, $\operatorname{PauliCommute}(\operatorname{check}(i), \operatorname{check}(j))$.
\end{itemize}
\end{definition}

\begin{definition}[Stabilizer Group]
\label{def:StabilizerCode.stabilizerGroup}
\lean{QEC1.StabilizerCode.stabilizerGroup}
\leanok
\uses{def:StabilizerCode, def:PauliOp}
The \emph{stabilizer group} $S$ of a stabilizer code $C$ is the subgroup of $\operatorname{PauliOp}(V)$ generated by the set of checks $\{\operatorname{check}(i) \mid i \in I\}$:
\[
S = \langle \operatorname{check}(i) \mid i \in I \rangle \leq \operatorname{PauliOp}(V).
\]
\end{definition}

\begin{lemma}[Check in Stabilizer Group]
\label{lem:StabilizerCode.check_mem_stabilizerGroup}
\lean{QEC1.StabilizerCode.check_mem_stabilizerGroup}
\leanok
\uses{def:StabilizerCode, def:StabilizerCode.stabilizerGroup}
For any stabilizer code $C$ and check index $i \in I$, the check $\operatorname{check}(i)$ belongs to the stabilizer group:
\[
\operatorname{check}(i) \in S.
\]
\end{lemma}

\begin{proof}
\leanok
\uses{def:StabilizerCode.stabilizerGroup}
We apply the fact that the closure of a set contains the set itself. Since $\operatorname{check}(i)$ is in the range of the check map, it belongs to $\operatorname{Set.range}(\operatorname{check})$, which is contained in $\operatorname{Subgroup.closure}(\operatorname{Set.range}(\operatorname{check})) = S$.
\end{proof}

\begin{definition}[Centralizer Membership]
\label{def:StabilizerCode.inCentralizer}
\lean{QEC1.StabilizerCode.inCentralizer}
\leanok
\uses{def:StabilizerCode, def:PauliOp.PauliCommute}
A Pauli operator $P$ is \emph{in the centralizer} of a stabilizer code $C$ if it commutes (in the full Pauli group) with every stabilizer check:
\[
\operatorname{inCentralizer}(C, P) \;\iff\; \forall\, i \in I,\; \operatorname{PauliCommute}(P, \operatorname{check}(i)).
\]
\end{definition}

\begin{definition}[Centralizer Set]
\label{def:StabilizerCode.centralizerSet}
\lean{QEC1.StabilizerCode.centralizerSet}
\leanok
\uses{def:StabilizerCode, def:StabilizerCode.inCentralizer}
The \emph{centralizer set} of a stabilizer code $C$ is the set of all Pauli operators in the centralizer:
\[
C(S) = \{ P \in \operatorname{PauliOp}(V) \mid \operatorname{inCentralizer}(C, P) \}.
\]
\end{definition}

\begin{lemma}[Identity in Centralizer]
\label{lem:StabilizerCode.one_inCentralizer}
\lean{QEC1.StabilizerCode.one_inCentralizer}
\leanok
\uses{def:StabilizerCode.inCentralizer, thm:PauliOp.pauliCommute_one_left}
The identity operator belongs to the centralizer of any stabilizer code $C$:
\[
\operatorname{inCentralizer}(C, 1).
\]
\end{lemma}

\begin{proof}
\leanok
\uses{thm:PauliOp.pauliCommute_one_left}
Let $i \in I$ be arbitrary. We need to show $\operatorname{PauliCommute}(1, \operatorname{check}(i))$. This follows directly from the fact that the identity commutes with any Pauli operator on the left.
\end{proof}

\begin{lemma}[Check in Centralizer]
\label{lem:StabilizerCode.check_inCentralizer}
\lean{QEC1.StabilizerCode.check_inCentralizer}
\leanok
\uses{def:StabilizerCode.inCentralizer, def:StabilizerCode}
For any check index $i \in I$, the check $\operatorname{check}(i)$ belongs to the centralizer of $C$:
\[
\operatorname{inCentralizer}(C, \operatorname{check}(i)).
\]
\end{lemma}

\begin{proof}
\leanok
\uses{def:StabilizerCode}
Let $j \in I$ be arbitrary. We need to show $\operatorname{PauliCommute}(\operatorname{check}(i), \operatorname{check}(j))$. This follows directly from the commutativity condition of the stabilizer code $C$.
\end{proof}

\begin{definition}[Logical Operator]
\label{def:StabilizerCode.isLogicalOp}
\lean{QEC1.StabilizerCode.isLogicalOp}
\leanok
\uses{def:StabilizerCode, def:StabilizerCode.inCentralizer, def:StabilizerCode.stabilizerGroup}
A Pauli operator $P$ is a \emph{non-trivial logical operator} of a stabilizer code $C$ if:
\begin{enumerate}
  \item $P$ is in the centralizer: $\operatorname{inCentralizer}(C, P)$,
  \item $P$ is not in the stabilizer group: $P \notin S$,
  \item $P$ is not the identity: $P \neq 1$.
\end{enumerate}
\end{definition}

\begin{lemma}[Logical Operator is in Centralizer]
\label{lem:StabilizerCode.isLogicalOp_inCentralizer}
\lean{QEC1.StabilizerCode.isLogicalOp_inCentralizer}
\leanok
\uses{def:StabilizerCode.isLogicalOp, def:StabilizerCode.inCentralizer}
If $P$ is a non-trivial logical operator of $C$, then $P$ is in the centralizer.
\end{lemma}

\begin{proof}
\leanok
\uses{def:StabilizerCode.isLogicalOp}
This is the first component of the conjunction defining $\operatorname{isLogicalOp}$.
\end{proof}

\begin{lemma}[Logical Operator not in Stabilizer Group]
\label{lem:StabilizerCode.isLogicalOp_not_mem_stabilizerGroup}
\lean{QEC1.StabilizerCode.isLogicalOp_not_mem_stabilizerGroup}
\leanok
\uses{def:StabilizerCode.isLogicalOp, def:StabilizerCode.stabilizerGroup}
If $P$ is a non-trivial logical operator of $C$, then $P \notin S$.
\end{lemma}

\begin{proof}
\leanok
\uses{def:StabilizerCode.isLogicalOp}
This is the first component of the second part of the conjunction defining $\operatorname{isLogicalOp}$.
\end{proof}

\begin{lemma}[Logical Operator is Non-Identity]
\label{lem:StabilizerCode.isLogicalOp_ne_one}
\lean{QEC1.StabilizerCode.isLogicalOp_ne_one}
\leanok
\uses{def:StabilizerCode.isLogicalOp}
If $P$ is a non-trivial logical operator of $C$, then $P \neq 1$.
\end{lemma}

\begin{proof}
\leanok
\uses{def:StabilizerCode.isLogicalOp}
This is the second component of the second part of the conjunction defining $\operatorname{isLogicalOp}$.
\end{proof}

\begin{definition}[Code Distance]
\label{def:StabilizerCode.distance}
\lean{QEC1.StabilizerCode.distance}
\leanok
\uses{def:StabilizerCode, def:StabilizerCode.isLogicalOp, def:PauliOp.weight}
The \emph{code distance} of a stabilizer code $C$ is the minimum weight of a non-trivial logical operator:
\[
d(C) = \inf\{ w \in \mathbb{N} \mid \exists\, P \in \operatorname{PauliOp}(V),\; \operatorname{isLogicalOp}(C, P) \;\land\; \operatorname{weight}(P) = w \}.
\]
If no logical operators exist, this returns $0$.
\end{definition}

\begin{definition}[Number of Physical Qubits]
\label{def:StabilizerCode.numQubits}
\lean{QEC1.StabilizerCode.numQubits}
\leanok
\uses{def:StabilizerCode}
The number of \emph{physical qubits} of a stabilizer code $C$ on qubit type $V$ is
\[
n = |V| = \operatorname{Fintype.card}(V).
\]
\end{definition}

\begin{definition}[Number of Checks]
\label{def:StabilizerCode.numChecks}
\lean{QEC1.StabilizerCode.numChecks}
\leanok
\uses{def:StabilizerCode}
The number of \emph{stabilizer checks} of a stabilizer code $C$ is
\[
\operatorname{numChecks}(C) = |I| = \operatorname{Fintype.card}(I).
\]
\end{definition}

\begin{definition}[Code Parameters]
\label{def:StabilizerCode.HasParameters}
\lean{QEC1.StabilizerCode.HasParameters}
\leanok
\uses{def:StabilizerCode, def:StabilizerCode.numChecks, def:StabilizerCode.distance}
A stabilizer code $C$ has \emph{parameters} $[\![n, k, d]\!]$ if:
\begin{enumerate}
  \item $|V| = n$ (the number of physical qubits is $n$),
  \item $n - \operatorname{numChecks}(C) = k$ (the number of encoded logical qubits is $k$),
  \item $d(C) = d$ (the code distance is $d$).
\end{enumerate}
\end{definition}

\begin{definition}[Check Weight]
\label{def:StabilizerCode.checkWeight}
\lean{QEC1.StabilizerCode.checkWeight}
\leanok
\uses{def:StabilizerCode, def:PauliOp.weight}
The \emph{weight of the $i$-th stabilizer check} of a code $C$ is the weight of the corresponding Pauli operator:
\[
\operatorname{checkWeight}(C, i) = \operatorname{weight}(\operatorname{check}(i)).
\]
\end{definition}

\begin{definition}[Qubit Degree]
\label{def:StabilizerCode.qubitDegree}
\lean{QEC1.StabilizerCode.qubitDegree}
\leanok
\uses{def:StabilizerCode, def:PauliOp.support}
The \emph{qubit degree} of a qubit $v \in V$ in a stabilizer code $C$ is the number of checks that act non-trivially on $v$:
\[
\operatorname{qubitDegree}(C, v) = \bigl|\{ i \in I \mid v \in \operatorname{support}(\operatorname{check}(i)) \}\bigr|.
\]
\end{definition}

\begin{definition}[Quantum LDPC Code]
\label{def:IsQLDPC}
\lean{QEC1.IsQLDPC}
\leanok
\uses{def:StabilizerCode, def:StabilizerCode.checkWeight, def:StabilizerCode.qubitDegree}
A stabilizer code $C$ is a \emph{quantum low-density parity-check (qLDPC) code} with parameters $(w, c)$ if:
\begin{enumerate}
  \item Each check has weight bounded by $w$: $\forall\, i \in I,\; \operatorname{checkWeight}(C, i) \leq w$,
  \item Each qubit participates in at most $c$ checks: $\forall\, v \in V,\; \operatorname{qubitDegree}(C, v) \leq c$.
\end{enumerate}
\end{definition}

\begin{theorem}[Identity is not a Logical Operator]
\label{thm:StabilizerCode.not_isLogicalOp_one}
\lean{QEC1.StabilizerCode.not_isLogicalOp_one}
\leanok
\uses{def:StabilizerCode.isLogicalOp}
The identity operator is not a non-trivial logical operator of any stabilizer code $C$:
\[
\neg\, \operatorname{isLogicalOp}(C, 1).
\]
\end{theorem}

\begin{proof}
\leanok
\uses{def:StabilizerCode.isLogicalOp}
Suppose for contradiction that $\operatorname{isLogicalOp}(C, 1)$ holds. Decomposing this, we obtain $h_1$ (centralizer membership), $h_2$ (not in stabilizer group), and $h_3 : 1 \neq 1$. But $h_3$ contradicts reflexivity, so the assumption is false.
\end{proof}

\begin{theorem}[Stabilizer Group Contained in Centralizer]
\label{thm:StabilizerCode.stabilizerGroup_subset_centralizer}
\lean{QEC1.StabilizerCode.stabilizerGroup_subset_centralizer}
\leanok
\uses{def:StabilizerCode.stabilizerGroup, def:StabilizerCode.inCentralizer, thm:PauliOp.symplecticInner_mul_left}
For any stabilizer code $C$, every element of the stabilizer group is in the centralizer:
\[
\forall\, P \in S,\; \operatorname{inCentralizer}(C, P).
\]
\end{theorem}

\begin{proof}
\leanok
\uses{lem:StabilizerCode.check_inCentralizer, lem:StabilizerCode.one_inCentralizer, thm:PauliOp.symplecticInner_mul_left}
Let $P \in S$. We proceed by induction on the proof that $P$ belongs to the subgroup closure.

\textbf{Generator case:} If $P = \operatorname{check}(i)$ for some $i$ (i.e., $P$ is in the generating set $\operatorname{Set.range}(\operatorname{check})$), then we obtain $i$ such that $P = \operatorname{check}(i)$, and conclude by the lemma that each check is in the centralizer.

\textbf{Identity case:} If $P = 1$, then $\operatorname{inCentralizer}(C, 1)$ holds by the lemma that the identity is in the centralizer.

\textbf{Multiplication case:} If $P = a \cdot b$ where $a, b$ are in the closure, and we have by induction that $\operatorname{inCentralizer}(C, a)$ and $\operatorname{inCentralizer}(C, b)$, then for any check index $i$, we need to show $\operatorname{PauliCommute}(a \cdot b, \operatorname{check}(i))$, i.e., $\langle a \cdot b, \operatorname{check}(i) \rangle_{\mathrm{symp}} = 0$. By the additivity of the symplectic inner product in the first argument, this equals $\langle a, \operatorname{check}(i) \rangle_{\mathrm{symp}} + \langle b, \operatorname{check}(i) \rangle_{\mathrm{symp}}$. From the inductive hypotheses, both terms are $0$, so the sum is $0 + 0 = 0$.

\textbf{Inverse case:} If $P = a^{-1}$ and $\operatorname{inCentralizer}(C, a)$ holds by induction, then since every Pauli operator is its own inverse ($a^{-1} = a$), we have $\operatorname{inCentralizer}(C, a^{-1}) = \operatorname{inCentralizer}(C, a)$, which holds by the inductive hypothesis.
\end{proof}

\begin{theorem}[Check is not a Logical Operator]
\label{thm:StabilizerCode.not_isLogicalOp_check}
\lean{QEC1.StabilizerCode.not_isLogicalOp_check}
\leanok
\uses{def:StabilizerCode.isLogicalOp, lem:StabilizerCode.check_mem_stabilizerGroup}
No stabilizer check is a non-trivial logical operator: for any $i \in I$,
\[
\neg\, \operatorname{isLogicalOp}(C, \operatorname{check}(i)).
\]
\end{theorem}

\begin{proof}
\leanok
\uses{lem:StabilizerCode.check_mem_stabilizerGroup}
Suppose for contradiction that $\operatorname{isLogicalOp}(C, \operatorname{check}(i))$ holds. Decomposing this, we obtain the condition that $\operatorname{check}(i) \notin S$. But this contradicts the lemma that $\operatorname{check}(i) \in S$.
\end{proof}

%--- Rem_4: NotationCheegerConstant ---
\chapter{Rem 4: Cheeger Constant and Expander Graphs}

For a finite simple graph $G = (V, E)$, the Cheeger constant (also called isoperimetric number or edge expansion) is defined as
\[
  h(G) = \min_{\substack{S \subseteq V \\ 0 < |S| \le |V|/2}} \frac{|\partial S|}{|S|},
\]
where $\partial S = \{e = \{u,v\} \in E : u \in S,\, v \notin S\}$ is the edge boundary of $S$. A graph with $h(G) \ge c$ for some constant $c > 0$ is called an expander graph.

\begin{definition}[Edge Boundary]
\label{def:SimpleGraph.edgeBoundary}
\lean{QEC1.SimpleGraph.edgeBoundary}
\leanok

The \emph{edge boundary} of a vertex subset $S$ in a simple graph $G = (V, E)$ is the set of edges with exactly one endpoint in $S$. Formally,
\[
  \partial S := \{ e \in E : |e_{\mathrm{set}} \cap S| = 1 \},
\]
where $e_{\mathrm{set}}$ denotes the underlying two-element set of the edge $e$.
\end{definition}

\begin{theorem}[Membership in Edge Boundary]
\label{thm:SimpleGraph.mem_edgeBoundary_iff}
\lean{QEC1.SimpleGraph.mem_edgeBoundary_iff}
\leanok
\uses{def:SimpleGraph.edgeBoundary}
An edge $e$ belongs to the edge boundary $\partial S$ if and only if $e \in E$ and $|e_{\mathrm{set}} \cap S| = 1$.
\end{theorem}

\begin{proof}
\leanok
\uses{def:SimpleGraph.edgeBoundary}
This holds by definition, since $\partial S$ is defined as the filter of the edge set by the condition $|e_{\mathrm{set}} \cap S| = 1$. Unfolding the definition of \texttt{edgeBoundary} and simplifying gives the equivalence directly.
\end{proof}

\begin{theorem}[Edge Boundary of Empty Set]
\label{thm:SimpleGraph.edgeBoundary_empty}
\lean{QEC1.SimpleGraph.edgeBoundary_empty}
\leanok
\uses{def:SimpleGraph.edgeBoundary}
The edge boundary of the empty set is empty: $\partial \emptyset = \emptyset$.
\end{theorem}

\begin{proof}
\leanok
\uses{def:SimpleGraph.edgeBoundary}
By extensionality, it suffices to show that no edge $e$ belongs to $\partial \emptyset$. Unfolding the definition and simplifying, we note that $e_{\mathrm{set}} \cap \emptyset = \emptyset$ for any edge $e$, so the cardinality condition $|e_{\mathrm{set}} \cap \emptyset| = 1$ is never satisfied.
\end{proof}

\begin{theorem}[Edge Boundary of Universal Set]
\label{thm:SimpleGraph.edgeBoundary_univ}
\lean{QEC1.SimpleGraph.edgeBoundary_univ}
\leanok
\uses{def:SimpleGraph.edgeBoundary}
The edge boundary of the entire vertex set is empty: $\partial V = \emptyset$.
\end{theorem}

\begin{proof}
\leanok
\uses{def:SimpleGraph.edgeBoundary}
By extensionality, we show both directions. For the forward direction, suppose $e \in \partial V$. Then $e \in E$ and $|e_{\mathrm{set}} \cap V| = 1$. Since $e$ is an edge of a simple graph, $e$ is not a diagonal, so $|e_{\mathrm{set}}| = 2$. But $e_{\mathrm{set}} \cap V = e_{\mathrm{set}}$ since all vertices lie in $V$, giving $|e_{\mathrm{set}}| = 1$, a contradiction with $|e_{\mathrm{set}}| = 2$ (by integer arithmetic). For the reverse direction, no element belongs to $\emptyset$, so the implication holds vacuously.
\end{proof}

\begin{theorem}[Edge Boundary of Complement]
\label{thm:SimpleGraph.edgeBoundary_compl}
\lean{QEC1.SimpleGraph.edgeBoundary_compl}
\leanok
\uses{def:SimpleGraph.edgeBoundary}
The edge boundary of a set equals the edge boundary of its complement: $\partial S = \partial S^c$.
\end{theorem}

\begin{proof}
\leanok
\uses{def:SimpleGraph.edgeBoundary}
By extensionality, we show both inclusions using the membership characterization.

For the forward direction, suppose $e \in E$ and $|e_{\mathrm{set}} \cap S| = 1$. Since $e$ is a non-diagonal edge, $|e_{\mathrm{set}}| = 2$. Using the partition identity $|e_{\mathrm{set}} \cap S| + |e_{\mathrm{set}} \cap S^c| = |e_{\mathrm{set}}|$, we substitute $|e_{\mathrm{set}}| = 2$ and $|e_{\mathrm{set}} \cap S| = 1$ to obtain $|e_{\mathrm{set}} \cap S^c| = 1$ by integer arithmetic.

For the reverse direction, suppose $e \in E$ and $|e_{\mathrm{set}} \cap S^c| = 1$. Again $|e_{\mathrm{set}}| = 2$. Applying the partition identity for both $S$ and $S^c$, and noting that $(S^c)^c = S$, we substitute $|e_{\mathrm{set}}| = 2$ and $|e_{\mathrm{set}} \cap S^c| = 1$ into the partition for $S^c$ to obtain $|e_{\mathrm{set}} \cap S| = 1$ by integer arithmetic.
\end{proof}

\begin{definition}[Cheeger-Valid Subsets]
\label{def:cheegerValidSubsets'}
\lean{QEC1.cheegerValidSubsets'}
\leanok

The set of \emph{Cheeger-valid subsets} of a finite vertex set $V$ is the collection of all nonempty subsets $S \subseteq V$ satisfying $2|S| \le |V|$:
\[
  \mathcal{S}(V) := \{ S \subseteq V : S \neq \emptyset \text{ and } 2|S| \le |V| \}.
\]
\end{definition}

\begin{definition}[Cheeger Constant]
\label{def:cheegerConstant}
\lean{QEC1.cheegerConstant}
\leanok
\uses{def:cheegerValidSubsets', def:SimpleGraph.edgeBoundary}
The \emph{Cheeger constant} $h(G)$ of a finite simple graph $G$ is defined as
\[
  h(G) := \begin{cases}
    \displaystyle\inf_{S \in \mathcal{S}(V)} \frac{|\partial S|}{|S|} & \text{if } \mathcal{S}(V) \neq \emptyset, \\
    0 & \text{otherwise (i.e., when } |V| \le 1\text{)}.
  \end{cases}
\]
\end{definition}

\begin{definition}[Expander Graph]
\label{def:SimpleGraph.IsExpander}
\lean{QEC1.SimpleGraph.IsExpander}
\leanok
\uses{def:cheegerConstant}
A graph $G$ is a \emph{$c$-expander} if $0 < c$ and $c \le h(G)$.
\end{definition}

\begin{theorem}[Existence of Cheeger-Valid Subsets]
\label{thm:SimpleGraph.cheegerValidSubsets'_nonempty_of_card_ge_two}
\lean{QEC1.SimpleGraph.cheegerValidSubsets'_nonempty_of_card_ge_two}
\leanok
\uses{def:cheegerValidSubsets'}
If $|V| \ge 2$, then the set of Cheeger-valid subsets $\mathcal{S}(V)$ is nonempty.
\end{theorem}

\begin{proof}
\leanok
\uses{def:cheegerValidSubsets'}
Unfolding the definition, we need to find an element in the filtered set. Since $|V| \ge 2$, we have $|V| > 0$, so by the characterization of positive cardinality there exists some vertex $v \in V$. The singleton $\{v\}$ is a member of the universal finset, it is nonempty, and $2 \cdot 1 = 2 \le |V|$. Thus $\{v\} \in \mathcal{S}(V)$.
\end{proof}

\begin{theorem}[Cheeger Constant is Non-negative]
\label{thm:SimpleGraph.cheegerConstant_nonneg}
\lean{QEC1.SimpleGraph.cheegerConstant_nonneg}
\leanok
\uses{def:cheegerConstant}
For any finite simple graph $G$, we have $h(G) \ge 0$.
\end{theorem}

\begin{proof}
\leanok
\uses{def:cheegerConstant}
Unfolding the definition of $h(G)$, we split into two cases. If $\mathcal{S}(V)$ is nonempty, then $h(G)$ is the infimum of ratios $|\partial S|/|S|$. We show $0$ is a lower bound by verifying that for each valid subset $S$, the ratio $|\partial S|/|S|$ is non-negative, since both the numerator $|\partial S|$ and denominator $|S|$ are non-negative (being natural number casts). If $\mathcal{S}(V)$ is empty, then $h(G) = 0 \ge 0$ by reflexivity.
\end{proof}

\begin{theorem}[Edge Boundary Lower Bound from Cheeger Constant]
\label{thm:SimpleGraph.edgeBoundary_card_ge_of_cheeger}
\lean{QEC1.SimpleGraph.edgeBoundary_card_ge_of_cheeger}
\leanok
\uses{def:cheegerConstant, def:SimpleGraph.edgeBoundary, def:cheegerValidSubsets'}
Let $G$ be a finite simple graph. If $c \le h(G)$, $S$ is a nonempty subset of $V$, and $2|S| \le |V|$, then $c \cdot |S| \le |\partial S|$.
\end{theorem}

\begin{proof}
\leanok
\uses{def:cheegerConstant, def:SimpleGraph.edgeBoundary, def:cheegerValidSubsets'}
First, we establish that $S$ is a Cheeger-valid subset by verifying $S \in \mathcal{S}(V)$ using the nonemptiness and size conditions. This gives us that $\mathcal{S}(V)$ is nonempty.

We then show that $h(G)$ equals the infimum formula by unfolding the definition of \texttt{cheegerConstant} and simplifying with the nonemptiness witness.

Since $S$ is nonempty, we have $|S| > 0$ as a positive real number. By the definition of infimum, $h(G) \le |\partial S|/|S|$ since $S$ is one of the sets over which the infimum is taken.

We conclude by the calculation:
\[
  c \cdot |S| \le h(G) \cdot |S| \le \frac{|\partial S|}{|S|} \cdot |S| = |\partial S|,
\]
where the first inequality uses $c \le h(G)$ and $|S| \ge 0$, the second uses the infimum bound and $|S| \ge 0$, and the final equality follows by cancellation since $|S| \neq 0$.
\end{proof}

\begin{theorem}[Characterization of Expander Graphs]
\label{thm:SimpleGraph.isExpander_iff}
\lean{QEC1.SimpleGraph.isExpander_iff}
\leanok
\uses{def:SimpleGraph.IsExpander, def:cheegerConstant, def:SimpleGraph.edgeBoundary, def:cheegerValidSubsets'}
Let $G$ be a finite simple graph with $|V| \ge 2$. Then $G$ is a $c$-expander if and only if $0 < c$ and for every nonempty subset $S \subseteq V$ with $2|S| \le |V|$, we have $c \cdot |S| \le |\partial S|$.
\end{theorem}

\begin{proof}
\leanok
\uses{def:SimpleGraph.IsExpander, def:cheegerConstant, def:SimpleGraph.edgeBoundary, def:cheegerValidSubsets', thm:SimpleGraph.edgeBoundary_card_ge_of_cheeger, thm:SimpleGraph.cheegerValidSubsets'_nonempty_of_card_ge_two}
We prove both directions of the equivalence.

\textbf{Forward direction.} Suppose $G$ is a $c$-expander, i.e., $0 < c$ and $c \le h(G)$. We retain $0 < c$ and for any nonempty $S$ with $2|S| \le |V|$, the bound $c \cdot |S| \le |\partial S|$ follows from the edge boundary lower bound theorem applied to $G$, $c$, $c \le h(G)$, $S$, the nonemptiness of $S$, and the size condition.

\textbf{Reverse direction.} Suppose $0 < c$ and for all valid $S$ we have $c \cdot |S| \le |\partial S|$. We retain $0 < c$ and must show $c \le h(G)$. Since $|V| \ge 2$, the set of Cheeger-valid subsets is nonempty. Unfolding the definition of $h(G)$ and simplifying with this nonemptiness, we need to show $c$ is at most the infimum. We apply the le-inf' characterization: for each $S \in \mathcal{S}(V)$, we extract from the membership that $S$ is nonempty with $2|S| \le |V|$. Since $S$ is nonempty, $|S| > 0$ as a real number, so we rewrite the goal using the division characterization $c \le |\partial S|/|S| \iff c \cdot |S| \le |\partial S|$, which holds by hypothesis.
\end{proof}

%--- Def_1: BoundaryAndCoboundaryMaps ---
\chapter{Def 1: Boundary and Coboundary Maps}

Let $G = (V, E)$ be a finite simple graph. We define $\mathbb{Z}_2$-linear maps that form the chain and cochain maps of the graph viewed as a cell complex over $\mathbb{F}_2$.

\begin{definition}[Boundary Map]
\label{def:GraphMaps.boundaryMap}
\lean{QEC1.GraphMaps.boundaryMap}
\leanok

The \emph{boundary map} $\partial : \mathbb{Z}_2^E \to \mathbb{Z}_2^V$ is the $\mathbb{Z}_2$-linear map defined as follows. For $\gamma \in \mathbb{Z}_2^E$ and a vertex $v \in V$,
\[
(\partial\,\gamma)_v \;=\; \sum_{\substack{e \in E \\ v \in e}} \gamma_e \pmod{2}.
\]
\end{definition}

\begin{definition}[Coboundary Map]
\label{def:GraphMaps.coboundaryMap}
\lean{QEC1.GraphMaps.coboundaryMap}
\leanok
\uses{def:GraphMaps.boundaryMap}
The \emph{coboundary map} $\delta : \mathbb{Z}_2^V \to \mathbb{Z}_2^E$ is the $\mathbb{Z}_2$-linear map defined as follows. For $f \in \mathbb{Z}_2^V$ and an edge $e = \{a, b\} \in E$,
\[
(\delta\,f)_e \;=\; f(a) + f(b) \pmod{2}.
\]
\end{definition}

\begin{theorem}[Coboundary Map Is Transpose of Boundary Map]
\label{thm:GraphMaps.coboundaryMap_eq_transpose}
\lean{QEC1.GraphMaps.coboundaryMap_eq_transpose}
\leanok
\uses{def:GraphMaps.boundaryMap, def:GraphMaps.coboundaryMap}
The coboundary map $\delta$ is the transpose of the boundary map $\partial$ with respect to the standard $\mathbb{Z}_2$ inner product. That is, for all $f \in \mathbb{Z}_2^V$ and $\gamma \in \mathbb{Z}_2^E$,
\[
\sum_{e \in E} (\delta\,f)_e \cdot \gamma_e \;=\; \sum_{v \in V} f_v \cdot (\partial\,\gamma)_v.
\]
\end{theorem}

\begin{proof}
\leanok
\uses{def:GraphMaps.boundaryMap, def:GraphMaps.coboundaryMap}
We unfold the definitions of $\delta$ and $\partial$. On the left-hand side, we distribute the multiplication into the sum, obtaining
\[
\text{LHS} = \sum_{e \in E} (\delta\,f)_e \cdot \gamma_e.
\]
On the right-hand side, we similarly distribute:
\[
\text{RHS} = \sum_{v \in V} f_v \cdot \sum_{\substack{e \in E \\ v \in e}} \gamma_e = \sum_{v \in V} \sum_{e \in E} f_v \cdot \bigl(\text{if } v \in e \text{ then } \gamma_e \text{ else } 0\bigr).
\]
We exchange the order of summation on the right-hand side so that both sides sum first over edges $e$ and then over vertices. It suffices to show equality for each edge $e = \{a, b\}$. By the definition of $\delta$, we have $(\delta\,f)_e = f(a) + f(b)$. Since $G$ is loopless, $a \neq b$. We expand $(\delta\,f)_e \cdot \gamma_e = f(a)\,\gamma_e + f(b)\,\gamma_e$. On the other side, for vertex $x$, the indicator $[x \in \{a,b\}]$ is $1$ if $x = a$ or $x = b$ and $0$ otherwise. Therefore
\[
\sum_{x \in V} f(x) \cdot \bigl(\text{if } x \in e \text{ then } \gamma_e \text{ else } 0\bigr)
\]
splits as a sum of two terms: $f(a)\,\gamma_e$ (from $x = a$) and $f(b)\,\gamma_e$ (from $x = b$), with all other terms vanishing. By splitting the sum using $\sum_x (\cdots) = \sum_x (\text{if } x=a \text{ then } \cdots \text{ else } 0) + \sum_x (\text{if } x=b \text{ then } \cdots \text{ else } 0)$ and evaluating via $\sum_x [x = a] \cdot (\cdots) = (\cdots)|_{x=a}$, we obtain equality by ring arithmetic.
\end{proof}

\begin{definition}[Second Boundary Map]
\label{def:GraphMaps.secondBoundaryMap}
\lean{QEC1.GraphMaps.secondBoundaryMap}
\leanok
\uses{def:GraphMaps.boundaryMap}
Let $C$ be a finite type of ``cycles'' (or plaquettes), and let each $c \in C$ be associated with a set of edges $\operatorname{cycles}(c) \subseteq E$. The \emph{second boundary map} $\partial_2 : \mathbb{Z}_2^C \to \mathbb{Z}_2^E$ is the $\mathbb{Z}_2$-linear map defined by: for $\sigma \in \mathbb{Z}_2^C$ and an edge $e \in E$,
\[
(\partial_2\,\sigma)_e \;=\; \sum_{\substack{c \in C \\ e \in \operatorname{cycles}(c)}} \sigma_c \pmod{2}.
\]
\end{definition}

\begin{definition}[Second Coboundary Map]
\label{def:GraphMaps.secondCoboundaryMap}
\lean{QEC1.GraphMaps.secondCoboundaryMap}
\leanok
\uses{def:GraphMaps.secondBoundaryMap}
The \emph{second coboundary map} $\delta_2 : \mathbb{Z}_2^E \to \mathbb{Z}_2^C$ is the $\mathbb{Z}_2$-linear map defined by: for $\gamma \in \mathbb{Z}_2^E$ and a cycle $c \in C$,
\[
(\delta_2\,\gamma)_c \;=\; \sum_{\substack{e \in E \\ e \in \operatorname{cycles}(c)}} \gamma_e \pmod{2}.
\]
\end{definition}

\begin{theorem}[Second Coboundary Map Is Transpose of Second Boundary Map]
\label{thm:GraphMaps.secondCoboundaryMap_eq_transpose}
\lean{QEC1.GraphMaps.secondCoboundaryMap_eq_transpose}
\leanok
\uses{def:GraphMaps.secondBoundaryMap, def:GraphMaps.secondCoboundaryMap}
The second coboundary map $\delta_2$ is the transpose of the second boundary map $\partial_2$. That is, for all $\gamma \in \mathbb{Z}_2^E$ and $\sigma \in \mathbb{Z}_2^C$,
\[
\sum_{c \in C} (\delta_2\,\gamma)_c \cdot \sigma_c \;=\; \sum_{e \in E} \gamma_e \cdot (\partial_2\,\sigma)_e.
\]
\end{theorem}

\begin{proof}
\leanok
\uses{def:GraphMaps.secondBoundaryMap, def:GraphMaps.secondCoboundaryMap}
We unfold the definitions of $\delta_2$ and $\partial_2$. On the left-hand side, we distribute the multiplication of $\sigma_c$ into the inner sum, obtaining
\[
\text{LHS} = \sum_{c \in C} \Bigl(\sum_{e \in E} [\,e \in \operatorname{cycles}(c)\,]\,\gamma_e\Bigr) \cdot \sigma_c = \sum_{c \in C}\sum_{e \in E} [\,e \in \operatorname{cycles}(c)\,]\,\gamma_e \cdot \sigma_c.
\]
On the right-hand side, we distribute $\gamma_e$ into the inner sum:
\[
\text{RHS} = \sum_{e \in E} \gamma_e \cdot \Bigl(\sum_{c \in C} [\,e \in \operatorname{cycles}(c)\,]\,\sigma_c\Bigr) = \sum_{e \in E}\sum_{c \in C} [\,e \in \operatorname{cycles}(c)\,]\,\gamma_e \cdot \sigma_c.
\]
We exchange the order of summation on the left-hand side. It then suffices to show that for each pair $(e, c)$, the summands agree. If $e \in \operatorname{cycles}(c)$, both summands equal $\gamma_e \cdot \sigma_c$ by ring arithmetic. If $e \notin \operatorname{cycles}(c)$, both summands are $0$.
\end{proof}

\begin{theorem}[Boundary Map on Single Edge Indicator]
\label{thm:GraphMaps.boundaryMap_single}
\lean{QEC1.GraphMaps.boundaryMap_single}
\leanok
\uses{def:GraphMaps.boundaryMap}
For a single edge $e \in E$ and a vertex $v \in V$, the boundary of the indicator function $\mathbf{1}_e$ satisfies
\[
(\partial\,\mathbf{1}_e)(v) = \begin{cases} 1 & \text{if } v \in e, \\ 0 & \text{otherwise.}\end{cases}
\]
\end{theorem}

\begin{proof}
\leanok
\uses{def:GraphMaps.boundaryMap}
By the definition of the boundary map, $(\partial\,\mathbf{1}_e)(v) = \sum_{e'} [v \in e']\,(\mathbf{1}_e)_{e'}$. We rewrite each summand: if $v \in e'$, then $(\mathbf{1}_e)_{e'} = [e' = e]$ by the definition of the indicator function; if $v \notin e'$, the summand is $0$. Thus the sum reduces to $\sum_{e'} [e' = e] \cdot [v \in e'] = [v \in e]$, as desired.
\end{proof}

\begin{theorem}[Coboundary Map on Single Vertex Indicator]
\label{thm:GraphMaps.coboundaryMap_single}
\lean{QEC1.GraphMaps.coboundaryMap_single}
\leanok
\uses{def:GraphMaps.coboundaryMap}
For a vertex $v \in V$ and an edge $e = \{a, b\} \in E$, the coboundary of the indicator function $\mathbf{1}_v$ satisfies
\[
(\delta\,\mathbf{1}_v)(e) = \begin{cases} 1 & \text{if } v \in e, \\ 0 & \text{otherwise.}\end{cases}
\]
\end{theorem}

\begin{proof}
\leanok
\uses{def:GraphMaps.coboundaryMap}
By the definition of the coboundary map, $(\delta\,\mathbf{1}_v)(e) = \mathbf{1}_v(a) + \mathbf{1}_v(b)$ where $e = \{a, b\}$. Since $G$ is loopless, $a \neq b$. We consider three cases.
If $v = a$, then $\mathbf{1}_v(a) = 1$ and $\mathbf{1}_v(b) = 0$ (since $a \neq b$), so the result is $1$, and indeed $v \in \{a, b\}$.
If $v = b$, then $\mathbf{1}_v(a) = 0$ and $\mathbf{1}_v(b) = 1$, so the result is $1$, and indeed $v \in \{a, b\}$.
If $v \neq a$ and $v \neq b$, then $v \notin \{a, b\}$, and $\mathbf{1}_v(a) = \mathbf{1}_v(b) = 0$, so the result is $0$.
\end{proof}

\begin{theorem}[Second Boundary Map on Single Cycle Indicator]
\label{thm:GraphMaps.secondBoundaryMap_single}
\lean{QEC1.GraphMaps.secondBoundaryMap_single}
\leanok
\uses{def:GraphMaps.secondBoundaryMap}
For a cycle $c \in C$ and an edge $e \in E$, the second boundary map applied to the indicator $\mathbf{1}_c$ satisfies
\[
(\partial_2\,\mathbf{1}_c)(e) = \begin{cases} 1 & \text{if } e \in \operatorname{cycles}(c), \\ 0 & \text{otherwise.}\end{cases}
\]
\end{theorem}

\begin{proof}
\leanok
\uses{def:GraphMaps.secondBoundaryMap}
By the definition of the second boundary map, $(\partial_2\,\mathbf{1}_c)(e) = \sum_{c'} [e \in \operatorname{cycles}(c')]\,(\mathbf{1}_c)_{c'}$. We rewrite each summand: if $e \in \operatorname{cycles}(c')$, then $(\mathbf{1}_c)_{c'} = [c' = c]$; otherwise the summand is $0$. The sum thus reduces to $[e \in \operatorname{cycles}(c)]$.
\end{proof}

\begin{theorem}[Second Coboundary Map on Single Edge Indicator]
\label{thm:GraphMaps.secondCoboundaryMap_single}
\lean{QEC1.GraphMaps.secondCoboundaryMap_single}
\leanok
\uses{def:GraphMaps.secondCoboundaryMap}
For an edge $e \in E$ and a cycle $c \in C$, the second coboundary map applied to the indicator $\mathbf{1}_e$ satisfies
\[
(\delta_2\,\mathbf{1}_e)(c) = \begin{cases} 1 & \text{if } e \in \operatorname{cycles}(c), \\ 0 & \text{otherwise.}\end{cases}
\]
\end{theorem}

\begin{proof}
\leanok
\uses{def:GraphMaps.secondCoboundaryMap}
By the definition of the second coboundary map, $(\delta_2\,\mathbf{1}_e)(c) = \sum_{e'} [e' \in \operatorname{cycles}(c)]\,(\mathbf{1}_e)_{e'}$. We rewrite each summand: if $e' \in \operatorname{cycles}(c)$, then $(\mathbf{1}_e)_{e'} = [e' = e]$; otherwise the summand is $0$. The sum reduces to $[e \in \operatorname{cycles}(c)]$.
\end{proof}

%--- Rem_5: ExactnessOfSequences ---
\chapter{Rem 5: Exactness of Sequences}

For a graph $G = (V, E)$ with a collection of cycles $C$ (where each cycle $c \in C$ is specified by its edge set, and each vertex of $G$ that appears in $c$ is incident to exactly two edges of $c$), the maps $\partial_2$ and $\partial$ satisfy the chain complex property $\partial \circ \partial_2 = 0$. Similarly, $\delta_2 \circ \delta = 0$ holds by transposition. For a connected graph, $\ker(\delta) = \{0, \mathbf{1}\}$ (constant functions), so the sequences are not short exact since $\ker(\delta) \neq 0$. If $C$ generates the cycle space ($\operatorname{im}(\partial_2) = \ker(\partial)$), then $\ker(\partial) = \operatorname{im}(\partial_2)$.

\begin{theorem}[Boundary of a Cycle is Zero]
\label{thm:GraphMaps.boundary_of_cycle_eq_zero}
\lean{QEC1.GraphMaps.boundary_of_cycle_eq_zero}
\leanok
\uses{def:GraphMaps.boundaryMap}
Let $G = (V, E)$ be a graph and let $C$ be a collection of cycles specified by their edge sets. Suppose that for every cycle $c \in C$ and every vertex $v \in V$, the number of edges of $c$ incident to $v$ is even. Then for any cycle $c \in C$ and any vertex $v \in V$,
\[
\partial\!\left(\mathbf{1}_c\right)(v) = 0,
\]
where $\mathbf{1}_c$ is the indicator function of the edges of $c$.
\end{theorem}

\begin{proof}
\leanok
\uses{def:GraphMaps.boundaryMap}
By the definition of the boundary map, $\partial(\mathbf{1}_c)(v) = \sum_{e \in E} [\![v \in e]\!] \cdot \mathbf{1}_c(e)$. We rewrite each summand: the term $[\![v \in e]\!] \cdot \mathbf{1}_c(e)$ equals $[\![e \in c \land v \in e]\!]$, by splitting on cases (if $v \in e$, then the value is $\mathbf{1}_c(e)$; otherwise $0$, and if $e \notin c$, both expressions are $0$). After this rewriting, we apply the identity $\sum_{e} [\![P(e)]\!] = |\{e : P(e)\}|$ (sum of boolean indicators equals the cardinality of the filtered set). This gives $\partial(\mathbf{1}_c)(v) = |\{e \in E : e \in c \land v \in e\}| \pmod{2}$. Since by hypothesis this cardinality is even, the natural number cast to $\mathbb{Z}/2\mathbb{Z}$ is zero, as required.
\end{proof}

\begin{theorem}[Boundary of Second Boundary on a Single Cycle]
\label{thm:GraphMaps.boundaryMap_comp_secondBoundaryMap_single}
\lean{QEC1.GraphMaps.boundaryMap_comp_secondBoundaryMap_single}
\leanok
\uses{def:GraphMaps.boundaryMap, def:GraphMaps.secondBoundaryMap}
Under the same cycle evenness hypothesis, for any cycle $c \in C$ and vertex $v \in V$,
\[
\partial\!\left(\partial_2(\mathbf{e}_c)\right)(v) = 0,
\]
where $\mathbf{e}_c = \pi_c(1)$ is the standard basis vector corresponding to $c$.
\end{theorem}

\begin{proof}
\leanok
\uses{def:GraphMaps.boundaryMap, def:GraphMaps.secondBoundaryMap, thm:GraphMaps.boundary_of_cycle_eq_zero}
We expand both maps using their definitions. By the definition of $\partial_2$, for each edge $e$, $(\partial_2(\mathbf{e}_c))(e) = \sum_{c'} [\![e \in c']\!] \cdot \mathbf{e}_c(c')$. We establish the key identity: for each edge $e$, $\sum_{c'} [\![e \in c']\!] \cdot \mathbf{e}_c(c') = \mathbf{1}_c(e)$. To see this, we rewrite the sum by cases: if $c' = c$ the summand is $[\![e \in c]\!]$, and if $c' \neq c$ the summand is $0$ (since $\mathbf{e}_c(c') = 0$). Thus the sum telescopes via $\sum_{c'} [\![c' = c]\!] \cdot [\![e \in c']\!] = [\![e \in c]\!]$ by evaluating at the unique term $c' = c$.

Substituting this into the boundary map expression, for each edge $e$ we replace the inner sum by $\mathbf{1}_c(e)$ (splitting on whether $v \in e$ to handle the outer indicator). The resulting expression is exactly $\partial(\mathbf{1}_c)(v)$, which equals $0$ by the theorem \texttt{boundary\_of\_cycle\_eq\_zero}.
\end{proof}

\begin{theorem}[Chain Complex Property: $\partial \circ \partial_2 = 0$]
\label{thm:GraphMaps.boundary_comp_secondBoundary_eq_zero}
\lean{QEC1.GraphMaps.boundary_comp_secondBoundary_eq_zero}
\leanok
\uses{def:GraphMaps.boundaryMap, def:GraphMaps.secondBoundaryMap}
Under the cycle evenness hypothesis (every vertex is incident to an even number of edges of each cycle), the composition of the boundary map with the second boundary map is zero:
\[
\partial \circ \partial_2 = 0.
\]
\end{theorem}

\begin{proof}
\leanok
\uses{def:GraphMaps.boundaryMap, def:GraphMaps.secondBoundaryMap, thm:GraphMaps.boundary_of_cycle_eq_zero}
By extensionality of linear maps, it suffices to show that for every $\sigma : C \to \mathbb{Z}/2\mathbb{Z}$ and every $v \in V$, $\partial(\partial_2(\sigma))(v) = 0$.

Expanding the definitions, the left-hand side is
\[
\sum_{e \in E} [\![v \in e]\!] \cdot \left(\sum_{c \in C} [\![e \in c]\!] \cdot \sigma(c)\right).
\]
We rewrite each summand: when $v \in e$, the inner expression is $\sum_c [\![e \in c]\!] \cdot \sigma(c)$; when $v \notin e$, the summand is $0$. Combining the conditions, each summand becomes $\sum_c [\![e \in c \land v \in e]\!] \cdot \sigma(c)$.

Swapping the order of summation, we obtain $\sum_{c \in C} \sum_{e \in E} [\![e \in c \land v \in e]\!] \cdot \sigma(c)$. For each fixed $c$, we factor out $\sigma(c)$ to get $\sigma(c) \cdot \sum_{e \in E} [\![e \in c \land v \in e]\!]$. The inner sum $\sum_{e} [\![e \in c \land v \in e]\!] = |\{e \in E : e \in c \land v \in e\}|$ counts edges of $c$ incident to $v$, which is even by hypothesis. Therefore in $\mathbb{Z}/2\mathbb{Z}$ this count is $0$, and each term $\sigma(c) \cdot 0 = 0$. Summing over all $c$ gives $0$.
\end{proof}

\begin{theorem}[$\operatorname{im}(\partial_2) \leq \ker(\partial)$]
\label{thm:GraphMaps.range_secondBoundary_le_ker_boundary}
\lean{QEC1.GraphMaps.range_secondBoundary_le_ker_boundary}
\leanok
\uses{def:GraphMaps.boundaryMap, def:GraphMaps.secondBoundaryMap}
Under the cycle evenness hypothesis, the image of $\partial_2$ is contained in the kernel of $\partial$:
\[
\operatorname{im}(\partial_2) \subseteq \ker(\partial).
\]
\end{theorem}

\begin{proof}
\leanok
\uses{thm:GraphMaps.boundary_comp_secondBoundary_eq_zero}
By the characterization of range contained in kernel via composition, it suffices to show $\partial \circ \partial_2 = 0$, which is exactly the chain complex property established in the previous theorem.
\end{proof}

\begin{theorem}[Coboundary Chain Complex Property: $\delta_2 \circ \delta = 0$]
\label{thm:GraphMaps.secondCoboundary_comp_coboundary_eq_zero}
\lean{QEC1.GraphMaps.secondCoboundary_comp_coboundary_eq_zero}
\leanok
\uses{def:GraphMaps.coboundaryMap, def:GraphMaps.secondCoboundaryMap, thm:GraphMaps.coboundaryMap_eq_transpose}
Under the cycle evenness hypothesis, the composition of the second coboundary map with the coboundary map is zero:
\[
\delta_2 \circ \delta = 0.
\]
\end{theorem}

\begin{proof}
\leanok
\uses{def:GraphMaps.coboundaryMap, def:GraphMaps.secondCoboundaryMap, def:GraphMaps.boundaryMap, thm:GraphMaps.coboundaryMap_eq_transpose, thm:GraphMaps.boundary_of_cycle_eq_zero}
By extensionality, it suffices to show that for every $f : V \to \mathbb{Z}/2\mathbb{Z}$ and every $c \in C$, $\delta_2(\delta(f))(c) = 0$. Expanding the definitions, the left-hand side equals $\sum_{e \in E} [\![e \in c]\!] \cdot (\delta f)(e)$.

Let $\mathbf{1}_c : E \to \mathbb{Z}/2\mathbb{Z}$ be the indicator function of cycle $c$. We rewrite the sum as $\sum_e (\delta f)(e) \cdot \mathbf{1}_c(e)$. By the transpose identity $\sum_e (\delta f)(e) \cdot g(e) = \sum_v f(v) \cdot (\partial g)(v)$, this equals $\sum_v f(v) \cdot (\partial \mathbf{1}_c)(v)$.

Since each vertex is incident to an even number of edges of each cycle, by the theorem \texttt{boundary\_of\_cycle\_eq\_zero}, $(\partial \mathbf{1}_c)(v) = 0$ for all $v$. Therefore the right-hand side becomes $\sum_v f(v) \cdot 0 = 0$.

It remains to verify that the original goal (which after simplification involves $\operatorname{Sym2.lift}$ applied to edge terms) agrees with the coboundary map formulation. This is checked by a straightforward computation showing both expressions agree on each summand. Combining these equalities completes the proof.
\end{proof}

\begin{theorem}[$\operatorname{im}(\delta) \leq \ker(\delta_2)$]
\label{thm:GraphMaps.range_coboundary_le_ker_secondCoboundary}
\lean{QEC1.GraphMaps.range_coboundary_le_ker_secondCoboundary}
\leanok
\uses{def:GraphMaps.coboundaryMap, def:GraphMaps.secondCoboundaryMap}
Under the cycle evenness hypothesis, the image of $\delta$ is contained in the kernel of $\delta_2$:
\[
\operatorname{im}(\delta) \subseteq \ker(\delta_2).
\]
\end{theorem}

\begin{proof}
\leanok
\uses{thm:GraphMaps.secondCoboundary_comp_coboundary_eq_zero}
By the characterization of range contained in kernel, it suffices to show $\delta_2 \circ \delta = 0$, which is the coboundary chain complex property established above.
\end{proof}

\begin{definition}[All-Ones Function]
\label{def:GraphMaps.allOnes}
\lean{QEC1.GraphMaps.allOnes}
\leanok
 
The \emph{all-ones function} $\mathbf{1} : V \to \mathbb{Z}/2\mathbb{Z}$ is defined by $\mathbf{1}(v) = 1$ for all $v \in V$.
\end{definition}

\begin{theorem}[All-Ones Vector in $\ker(\delta)$]
\label{thm:GraphMaps.allOnes_mem_ker_coboundary}
\lean{QEC1.GraphMaps.allOnes_mem_ker_coboundary}
\leanok
\uses{def:GraphMaps.allOnes, def:GraphMaps.coboundaryMap}
The all-ones vector $\mathbf{1}$ lies in the kernel of the coboundary map:
\[
\mathbf{1} \in \ker(\delta).
\]
\end{theorem}

\begin{proof}
\leanok
\uses{def:GraphMaps.allOnes, def:GraphMaps.coboundaryMap}
We must show $\delta(\mathbf{1}) = 0$, i.e., for every edge $e \in E$, $(\delta \mathbf{1})(e) = 0$. By the membership criterion for the kernel and extensionality, it suffices to check each edge. For an edge $e = \{a, b\}$, we use induction on $\operatorname{Sym2}$ to reduce to the case $e = s(a,b)$. Then $(\delta \mathbf{1})(e) = \mathbf{1}(a) + \mathbf{1}(b) = 1 + 1 = 0$ in $\mathbb{Z}/2\mathbb{Z}$, since in characteristic $2$, any element added to itself is zero.
\end{proof}

\begin{theorem}[Constant Functions in $\ker(\delta)$]
\label{thm:GraphMaps.constant_mem_ker_coboundary}
\lean{QEC1.GraphMaps.constant_mem_ker_coboundary}
\leanok
\uses{def:GraphMaps.coboundaryMap}
For any $a \in \mathbb{Z}/2\mathbb{Z}$, the constant function $f(v) = a$ for all $v \in V$ lies in $\ker(\delta)$.
\end{theorem}

\begin{proof}
\leanok
\uses{def:GraphMaps.coboundaryMap}
We verify $\delta(f) = 0$ by extensionality over edges. For each edge $e = \{x, y\}$, using induction on $\operatorname{Sym2}$, we have $(\delta f)(e) = f(x) + f(y) = a + a = 0$ in $\mathbb{Z}/2\mathbb{Z}$, since $a + a = 0$ in characteristic $2$.
\end{proof}

\begin{theorem}[$\ker(\delta)$: Constant on Adjacent Vertices]
\label{thm:GraphMaps.ker_coboundary_constant_on_adj}
\lean{QEC1.GraphMaps.ker_coboundary_constant_on_adj}
\leanok
\uses{def:GraphMaps.coboundaryMap}
If $f \in \ker(\delta)$ and $a, b \in V$ are adjacent in $G$, then $f(a) = f(b)$.
\end{theorem}

\begin{proof}
\leanok
\uses{def:GraphMaps.coboundaryMap}
Since $f \in \ker(\delta)$, we have $\delta(f) = 0$. Because $a$ and $b$ are adjacent, $e = s(a,b) \in E$. Evaluating $\delta(f)$ at $e$ gives $(\delta f)(e) = f(a) + f(b) = 0$ in $\mathbb{Z}/2\mathbb{Z}$. By the characterization of addition in $\mathbb{Z}/2\mathbb{Z}$, $f(a) + f(b) = 0$ implies $f(a) = f(b)$.
\end{proof}

\begin{theorem}[$\ker(\delta)$: Constant Along Walks]
\label{thm:GraphMaps.ker_coboundary_constant_on_walk}
\lean{QEC1.GraphMaps.ker_coboundary_constant_on_walk}
\leanok
\uses{def:GraphMaps.coboundaryMap}
If $f \in \ker(\delta)$ and $p$ is a walk from $u$ to $w$ in $G$, then $f(u) = f(w)$.
\end{theorem}

\begin{proof}
\leanok
\uses{def:GraphMaps.coboundaryMap, thm:GraphMaps.ker_coboundary_constant_on_adj}
We proceed by induction on the walk $p$.

\textbf{Base case} ($p$ is the nil walk, $u = w$): The equality $f(u) = f(w)$ holds by reflexivity.

\textbf{Inductive step} ($p$ is a cons walk $u \to v \to \cdots \to w$ where $u$ is adjacent to $v$): By the previous theorem, $f(u) = f(v)$ since $u$ and $v$ are adjacent. By the induction hypothesis applied to the sub-walk from $v$ to $w$, $f(v) = f(w)$. Combining by transitivity gives $f(u) = f(w)$.
\end{proof}

\begin{theorem}[$\ker(\delta)$: Constant on Reachable Vertices]
\label{thm:GraphMaps.ker_coboundary_constant_on_reachable}
\lean{QEC1.GraphMaps.ker_coboundary_constant_on_reachable}
\leanok
\uses{def:GraphMaps.coboundaryMap}
If $f \in \ker(\delta)$ and $u, w \in V$ are reachable from each other in $G$, then $f(u) = f(w)$.
\end{theorem}

\begin{proof}
\leanok
\uses{thm:GraphMaps.ker_coboundary_constant_on_walk}
From the reachability hypothesis, we obtain a walk $p$ from $u$ to $w$. The result follows directly from the theorem that $f$ is constant along walks.
\end{proof}

\begin{theorem}[$\ker(\delta)$ for Connected Graphs: Constant Functions]
\label{thm:GraphMaps.ker_coboundary_is_constant}
\lean{QEC1.GraphMaps.ker_coboundary_is_constant}
\leanok
\uses{def:GraphMaps.coboundaryMap}
If $G$ is connected and $f \in \ker(\delta)$, then $f$ is a constant function, i.e., there exists $a \in \mathbb{Z}/2\mathbb{Z}$ such that $f = (v \mapsto a)$.
\end{theorem}

\begin{proof}
\leanok
\uses{thm:GraphMaps.ker_coboundary_constant_on_reachable}
Since $G$ is connected, the vertex set is nonempty; let $v_0 \in V$. We claim $a = f(v_0)$ works. By extensionality, for any $w \in V$, since $G$ is connected, $v_0$ and $w$ are reachable. By the theorem that $f$ is constant on reachable vertices, $f(v_0) = f(w)$. Taking the symmetric equality gives $f(w) = f(v_0) = a$ for all $w$.
\end{proof}

\begin{theorem}[$\ker(\delta)$ for Connected Graphs is $\{0, \mathbf{1}\}$]
\label{thm:GraphMaps.ker_coboundary_connected}
\lean{QEC1.GraphMaps.ker_coboundary_connected}
\leanok
\uses{def:GraphMaps.coboundaryMap, def:GraphMaps.allOnes}
For a connected graph $G$, a function $f : V \to \mathbb{Z}/2\mathbb{Z}$ lies in $\ker(\delta)$ if and only if $f = 0$ or $f = \mathbf{1}$:
\[
f \in \ker(\delta) \iff f = 0 \text{ or } f = \mathbf{1}.
\]
\end{theorem}

\begin{proof}
\leanok
\uses{thm:GraphMaps.ker_coboundary_is_constant, thm:GraphMaps.allOnes_mem_ker_coboundary, def:GraphMaps.allOnes}
We prove both directions.

$(\Rightarrow)$: Suppose $f \in \ker(\delta)$. By the theorem that kernel elements of connected graphs are constant, there exists $a \in \mathbb{Z}/2\mathbb{Z}$ such that $f(v) = a$ for all $v$. Since $\mathbb{Z}/2\mathbb{Z} = \{0, 1\}$, we have $a = 0$ or $a = 1$ (verified by case analysis on the finite type). If $a = 0$, then $f = 0$ by extensionality. If $a = 1$, then $f = \mathbf{1}$ by extensionality and the definition of $\mathbf{1}$.

$(\Leftarrow)$: If $f = 0$, then $f \in \ker(\delta)$ since the kernel is a submodule (containing $0$). If $f = \mathbf{1}$, then $f \in \ker(\delta)$ by the theorem that $\mathbf{1} \in \ker(\delta)$.
\end{proof}

\begin{theorem}[$\ker(\delta) \neq 0$: Sequences Are Not Short Exact]
\label{thm:GraphMaps.ker_coboundary_ne_bot}
\lean{QEC1.GraphMaps.ker_coboundary_ne_bot}
\leanok
\uses{def:GraphMaps.coboundaryMap, def:GraphMaps.allOnes}
Assuming $V$ is nonempty, the kernel of the coboundary map is nontrivial:
\[
\ker(\delta) \neq \{0\}.
\]
In particular, the chain complex sequences involving $\delta$ are not short exact.
\end{theorem}

\begin{proof}
\leanok
\uses{thm:GraphMaps.allOnes_mem_ker_coboundary, def:GraphMaps.allOnes}
Suppose for contradiction that $\ker(\delta) = \{0\}$ (i.e., the kernel is the bottom submodule). The all-ones vector $\mathbf{1}$ lies in $\ker(\delta)$ by the previously established theorem. By our assumption, $\mathbf{1} = 0$ as functions $V \to \mathbb{Z}/2\mathbb{Z}$. Since $V$ is nonempty, let $v \in V$. Evaluating at $v$ gives $\mathbf{1}(v) = 0(v)$, i.e., $1 = 0$ in $\mathbb{Z}/2\mathbb{Z}$. By the definition of $\mathbf{1}$, this simplifies to a contradiction.
\end{proof}

\begin{theorem}[$\partial(\partial_2(\sigma)) = 0$ Pointwise]
\label{thm:GraphMaps.boundary_secondBoundary_apply}
\lean{QEC1.GraphMaps.boundary_secondBoundary_apply}
\leanok
\uses{def:GraphMaps.boundaryMap, def:GraphMaps.secondBoundaryMap}
Under the cycle evenness hypothesis, for any $\sigma : C \to \mathbb{Z}/2\mathbb{Z}$,
\[
\partial(\partial_2(\sigma)) = 0.
\]
\end{theorem}

\begin{proof}
\leanok
\uses{thm:GraphMaps.boundary_comp_secondBoundary_eq_zero}
By the chain complex property $\partial \circ \partial_2 = 0$, we have $(\partial \circ \partial_2)(\sigma) = 0(\sigma) = 0$, which is exactly $\partial(\partial_2(\sigma)) = 0$. The proof rewrites the goal as the application of the composition and then applies the chain complex theorem.
\end{proof}

\begin{theorem}[$\delta_2(\delta(f)) = 0$ Pointwise]
\label{thm:GraphMaps.secondCoboundary_coboundary_apply}
\lean{QEC1.GraphMaps.secondCoboundary_coboundary_apply}
\leanok
\uses{def:GraphMaps.coboundaryMap, def:GraphMaps.secondCoboundaryMap}
Under the cycle evenness hypothesis, for any $f : V \to \mathbb{Z}/2\mathbb{Z}$,
\[
\delta_2(\delta(f)) = 0.
\]
\end{theorem}

\begin{proof}
\leanok
\uses{thm:GraphMaps.secondCoboundary_comp_coboundary_eq_zero}
By the coboundary chain complex property $\delta_2 \circ \delta = 0$, we have $(\delta_2 \circ \delta)(f) = 0(f) = 0$, which is exactly $\delta_2(\delta(f)) = 0$. The proof rewrites the goal as the application of the composition and then applies the coboundary chain complex theorem.
\end{proof}

%--- Def_2: GaussLawAndFluxOperators ---
\chapter{Def 2: Gauss's Law and Flux Operators}

Given a graph $G = (V, E)$ and a collection of cycles $C$, we define Pauli operators on the
extended qubit system $V \oplus E$ (vertex qubits $+$ edge qubits):
\begin{enumerate}
\item \textbf{Gauss's law operator} $A_v = X_v \prod_{e \ni v} X_e$ for each vertex $v \in V$
\item \textbf{Flux operator} $B_p = \prod_{e \in p} Z_e$ for each cycle $p \in C$
\end{enumerate}

\begin{definition}[Extended Qubit Type]
\label{def:GaussFlux.ExtQubit}
\lean{QEC1.GaussFlux.ExtQubit}
\leanok
\uses{def:PauliOp}
The \emph{extended qubit type} for a graph $G = (V,E)$ is the sum type
\[
\operatorname{ExtQubit}(G) := V \oplus G.\operatorname{edgeSet},
\]
where $\operatorname{Sum.inl}(v)$ represents a vertex qubit and $\operatorname{Sum.inr}(e)$ represents an edge qubit.
\end{definition}

\begin{definition}[Incident Edges]
\label{def:GaussFlux.incidentEdges}
\lean{QEC1.GaussFlux.incidentEdges}
\leanok
\uses{def:GaussFlux.ExtQubit}
For a vertex $v \in V$, the \emph{incident edges} of $v$ is the finset
\[
\operatorname{incidentEdges}(G, v) := \{ e \in G.\operatorname{edgeSet} \mid v \in e \}.
\]
\end{definition}

\begin{lemma}[Membership in Incident Edges]
\label{lem:GaussFlux.mem_incidentEdges}
\lean{QEC1.GaussFlux.mem_incidentEdges}
\leanok
\uses{def:GaussFlux.incidentEdges}
For a vertex $v \in V$ and an edge $e \in G.\operatorname{edgeSet}$,
\[
e \in \operatorname{incidentEdges}(G, v) \iff v \in e.
\]
\end{lemma}

\begin{proof}
\leanok
\uses{def:GaussFlux.incidentEdges}
By simplification using the definition of $\operatorname{incidentEdges}$, which is a filter on the universe by the predicate $v \in e$, this equivalence holds directly.
\end{proof}

\begin{definition}[Gauss's Law Operator]
\label{def:GaussFlux.gaussLawOp}
\lean{QEC1.GaussFlux.gaussLawOp}
\leanok
\uses{def:PauliOp, def:GaussFlux.ExtQubit}
The \emph{Gauss's law operator} $A_v$ on the extended system $V \oplus E$ is the Pauli operator defined by:
\[
A_v.\operatorname{xVec}(q) = \begin{cases} 1 & \text{if } q = \operatorname{inl}(w) \text{ and } w = v, \\ 1 & \text{if } q = \operatorname{inr}(e) \text{ and } v \in e, \\ 0 & \text{otherwise}, \end{cases}
\qquad
A_v.\operatorname{zVec} = 0.
\]
That is, $A_v = X_v \prod_{e \ni v} X_e$: it acts with $X$ on vertex qubit $v$ and all incident edge qubits.
\end{definition}

\begin{lemma}[Gauss Operator Z-Vector]
\label{lem:GaussFlux.gaussLawOp_zVec}
\lean{QEC1.GaussFlux.gaussLawOp_zVec}
\leanok
\uses{def:GaussFlux.gaussLawOp}
For all $v \in V$, $A_v.\operatorname{zVec} = 0$.
\end{lemma}

\begin{proof}
\leanok
\uses{def:GaussFlux.gaussLawOp}
This holds by reflexivity, since the $\operatorname{zVec}$ component of $\operatorname{gaussLawOp}$ is defined to be $0$.
\end{proof}

\begin{definition}[Flux Operator]
\label{def:GaussFlux.fluxOp}
\lean{QEC1.GaussFlux.fluxOp}
\leanok
\uses{def:PauliOp, def:GaussFlux.ExtQubit}
The \emph{flux operator} $B_p$ on the extended system $V \oplus E$ is the Pauli operator defined by:
\[
B_p.\operatorname{xVec} = 0,
\qquad
B_p.\operatorname{zVec}(q) = \begin{cases} 1 & \text{if } q = \operatorname{inr}(e) \text{ and } e \in \operatorname{cycles}(p), \\ 0 & \text{otherwise}. \end{cases}
\]
That is, $B_p = \prod_{e \in p} Z_e$: it acts with $Z$ on all edge qubits in cycle $p$.
\end{definition}

\begin{lemma}[Flux Operator X-Vector]
\label{lem:GaussFlux.fluxOp_xVec}
\lean{QEC1.GaussFlux.fluxOp_xVec}
\leanok
\uses{def:GaussFlux.fluxOp}
For all $p \in C$, $B_p.\operatorname{xVec} = 0$.
\end{lemma}

\begin{proof}
\leanok
\uses{def:GaussFlux.fluxOp}
This holds by reflexivity, since the $\operatorname{xVec}$ component of $\operatorname{fluxOp}$ is defined to be $0$.
\end{proof}

\begin{theorem}[Gauss Operator is Pure X-Type]
\label{thm:GaussFlux.gaussLawOp_is_pure_X}
\lean{QEC1.GaussFlux.gaussLawOp_is_pure_X}
\leanok
\uses{def:GaussFlux.gaussLawOp, def:PauliOp.supportZ}
$A_v$ is pure $X$-type: it has no $Z$ support.
\[
\operatorname{supportZ}(A_v) = \emptyset.
\]
\end{theorem}

\begin{proof}
\leanok
\uses{def:GaussFlux.gaussLawOp, def:PauliOp.supportZ}
By extensionality, for any qubit $q$, we simplify using the definitions of $\operatorname{supportZ}$ and $\operatorname{gaussLawOp}$. Since $A_v.\operatorname{zVec} = 0$, no qubit $q$ can belong to $\operatorname{supportZ}(A_v)$, so $\operatorname{supportZ}(A_v) = \emptyset$.
\end{proof}

\begin{theorem}[Flux Operator is Pure Z-Type]
\label{thm:GaussFlux.fluxOp_is_pure_Z}
\lean{QEC1.GaussFlux.fluxOp_is_pure_Z}
\leanok
\uses{def:GaussFlux.fluxOp, def:PauliOp.supportX}
$B_p$ is pure $Z$-type: it has no $X$ support.
\[
\operatorname{supportX}(B_p) = \emptyset.
\]
\end{theorem}

\begin{proof}
\leanok
\uses{def:GaussFlux.fluxOp, def:PauliOp.supportX}
By extensionality, for any qubit $q$, we simplify using the definitions of $\operatorname{supportX}$ and $\operatorname{fluxOp}$. Since $B_p.\operatorname{xVec} = 0$, no qubit $q$ can belong to $\operatorname{supportX}(B_p)$, so $\operatorname{supportX}(B_p) = \emptyset$.
\end{proof}

\begin{theorem}[Gauss Operators Mutually Commute]
\label{thm:GaussFlux.gaussLaw_commute}
\lean{QEC1.GaussFlux.gaussLaw_commute}
\leanok
\uses{def:GaussFlux.gaussLawOp, def:PauliOp.PauliCommute, def:PauliOp.symplecticInner}
All Gauss's law operators mutually commute: for all $v, w \in V$,
\[
[A_v, A_w] = 0.
\]
This holds because they are all pure $X$-type.
\end{theorem}

\begin{proof}
\leanok
\uses{def:GaussFlux.gaussLawOp, def:PauliOp.PauliCommute, def:PauliOp.symplecticInner}
We unfold the definitions of $\operatorname{PauliCommute}$ and $\operatorname{symplecticInner}$. The symplectic inner product is a sum over all qubits $q$:
\[
\langle A_v, A_w \rangle_s = \sum_q \bigl( (A_v)_x(q) \cdot (A_w)_z(q) + (A_v)_z(q) \cdot (A_w)_x(q) \bigr).
\]
We show each summand is zero. For each $q$, since both $A_v$ and $A_w$ have $\operatorname{zVec} = 0$, both terms in the sum vanish by simplification using the definition of $\operatorname{gaussLawOp}$.
\end{proof}

\begin{theorem}[Flux Operators Mutually Commute]
\label{thm:GaussFlux.flux_commute}
\lean{QEC1.GaussFlux.flux_commute}
\leanok
\uses{def:GaussFlux.fluxOp, def:PauliOp.PauliCommute, def:PauliOp.symplecticInner}
All flux operators mutually commute: for all $p, q \in C$,
\[
[B_p, B_q] = 0.
\]
This holds because they are all pure $Z$-type.
\end{theorem}

\begin{proof}
\leanok
\uses{def:GaussFlux.fluxOp, def:PauliOp.PauliCommute, def:PauliOp.symplecticInner}
We unfold the definitions of $\operatorname{PauliCommute}$ and $\operatorname{symplecticInner}$. The symplectic inner product is a sum over all qubits $r$. We show each summand is zero. For each $r$, since both $B_p$ and $B_q$ have $\operatorname{xVec} = 0$, both terms in the sum vanish by simplification using the definition of $\operatorname{fluxOp}$.
\end{proof}

\begin{definition}[Cycle Incident Count]
\label{def:GaussFlux.cycleIncidentCount}
\lean{QEC1.GaussFlux.cycleIncidentCount}
\leanok
\uses{def:GaussFlux.ExtQubit}
The \emph{cycle incident count} of a vertex $v$ and a cycle $p$ is the number of edges in $p$ that are incident to $v$:
\[
\operatorname{cycleIncidentCount}(v, p) := \bigl|\{ e \in G.\operatorname{edgeSet} \mid e \in \operatorname{cycles}(p) \land v \in e \}\bigr|.
\]
For a valid cycle, this is always even (either $0$ or $2$).
\end{definition}

\begin{theorem}[Gauss and Flux Operators Commute]
\label{thm:GaussFlux.gauss_flux_commute}
\lean{QEC1.GaussFlux.gauss_flux_commute}
\leanok
\uses{def:GaussFlux.gaussLawOp, def:GaussFlux.fluxOp, def:PauliOp.PauliCommute, def:PauliOp.symplecticInner}
Gauss's law operators commute with flux operators: assuming for every cycle $c \in C$ and every vertex $v \in V$ the number of edges in $c$ incident to $v$ is even, then for all $v \in V$ and $p \in C$,
\[
[A_v, B_p] = 0.
\]
\end{theorem}

\begin{proof}
\leanok
\uses{def:GaussFlux.gaussLawOp, def:GaussFlux.fluxOp, def:PauliOp.PauliCommute, def:PauliOp.symplecticInner}
We unfold the definitions of $\operatorname{PauliCommute}$ and $\operatorname{symplecticInner}$. The symplectic inner product is
\[
\langle A_v, B_p \rangle_s = \sum_q \bigl( (A_v)_x(q) \cdot (B_p)_z(q) + (A_v)_z(q) \cdot (B_p)_x(q) \bigr).
\]
Since $A_v$ is pure $X$-type ($\operatorname{zVec} = 0$) and $B_p$ is pure $Z$-type ($\operatorname{xVec} = 0$), the second term in each summand vanishes. We establish that
\[
\langle A_v, B_p \rangle_s = \sum_q (A_v)_x(q) \cdot (B_p)_z(q)
\]
by congruence: for each $q$, the extra terms simplify to zero using the definitions of $\operatorname{gaussLawOp}$ and $\operatorname{fluxOp}$. Rewriting with this identity, we split the sum over $V \oplus E$ into vertex and edge parts using $\operatorname{Fintype.sum\_sum\_type}$.

For the \textbf{vertex part}: $\sum_{w \in V} (A_v)_x(\operatorname{inl}(w)) \cdot (B_p)_z(\operatorname{inl}(w)) = 0$, because $B_p$ has $\operatorname{zVec} = 0$ on vertices. We show each summand is zero by simplification, then rewrite with this identity and $0 + (\cdot) = (\cdot)$.

For the \textbf{edge part}: we establish that
\[
\sum_{e \in G.\operatorname{edgeSet}} (A_v)_x(\operatorname{inr}(e)) \cdot (B_p)_z(\operatorname{inr}(e)) = \sum_e \begin{cases} 1 & \text{if } e \in \operatorname{cycles}(p) \land v \in e, \\ 0 & \text{otherwise}. \end{cases}
\]
This is shown by congruence and case analysis: for each edge $e$, we consider the cases $v \in e$ and $e \in \operatorname{cycles}(p)$ and simplify. Rewriting with this identity, we apply $\operatorname{Finset.sum\_boole}$ to convert the sum of indicators to a cardinality:
\[
\sum_e [\![e \in \operatorname{cycles}(p) \land v \in e]\!] = \bigl|\{e \mid e \in \operatorname{cycles}(p) \land v \in e\}\bigr| \pmod{2}.
\]
By $\operatorname{ZMod.natCast\_eq\_zero\_iff\_even}$, this is $0 \in \mathbb{Z}/2\mathbb{Z}$ if and only if the count is even. The result then follows directly from the hypothesis $\operatorname{hcyc}(p, v)$.
\end{proof}

\begin{theorem}[Gauss Operator X-Support on Vertices]
\label{thm:GaussFlux.gaussLawOp_supportX_vertex}
\lean{QEC1.GaussFlux.gaussLawOp_supportX_vertex}
\leanok
\uses{def:GaussFlux.gaussLawOp, def:PauliOp.supportX}
The $X$-support of $A_v$ on vertex qubits is $\{v\}$: for all $w \in V$,
\[
\operatorname{inl}(w) \in \operatorname{supportX}(A_v) \iff w = v.
\]
\end{theorem}

\begin{proof}
\leanok
\uses{def:GaussFlux.gaussLawOp, def:PauliOp.supportX}
Let $w$ be an arbitrary vertex. We simplify using the definitions of $\operatorname{supportX}$ and $\operatorname{gaussLawOp}$. The $X$-vector of $A_v$ at $\operatorname{inl}(w)$ is $1$ if $w = v$ and $0$ otherwise, so the result follows directly.
\end{proof}

\begin{theorem}[Gauss Operator X-Support on Edges]
\label{thm:GaussFlux.gaussLawOp_supportX_edge}
\lean{QEC1.GaussFlux.gaussLawOp_supportX_edge}
\leanok
\uses{def:GaussFlux.gaussLawOp, def:PauliOp.supportX}
The $X$-support of $A_v$ on edge qubits is the set of incident edges: for all $e \in G.\operatorname{edgeSet}$,
\[
\operatorname{inr}(e) \in \operatorname{supportX}(A_v) \iff v \in e.
\]
\end{theorem}

\begin{proof}
\leanok
\uses{def:GaussFlux.gaussLawOp, def:PauliOp.supportX}
Let $e$ be an arbitrary edge. We simplify using the definitionsof $\operatorname{supportX}$ and $\operatorname{gaussLawOp}$. The $X$-vector of $A_v$ at $\operatorname{inr}(e)$ is $1$ if $v \in e$ and $0$ otherwise, so the result follows directly.
\end{proof}

\begin{theorem}[Flux Operator Z-Support on Edges]
\label{thm:GaussFlux.fluxOp_supportZ_edge}
\lean{QEC1.GaussFlux.fluxOp_supportZ_edge}
\leanok
\uses{def:GaussFlux.fluxOp, def:PauliOp.supportZ}
The $Z$-support of $B_p$ on edge qubits is the set of edges in the cycle: for all $e \in G.\operatorname{edgeSet}$,
\[
\operatorname{inr}(e) \in \operatorname{supportZ}(B_p) \iff e \in \operatorname{cycles}(p).
\]
\end{theorem}

\begin{proof}
\leanok
\uses{def:GaussFlux.fluxOp, def:PauliOp.supportZ}
Let $e$ be an arbitrary edge. We simplify using the definitions of $\operatorname{supportZ}$ and $\operatorname{fluxOp}$. The $Z$-vector of $B_p$ at $\operatorname{inr}(e)$ is $1$ if $e \in \operatorname{cycles}(p)$ and $0$ otherwise, so the result follows directly.
\end{proof}

\begin{theorem}[Flux Operator Z-Support on Vertices is Empty]
\label{thm:GaussFlux.fluxOp_supportZ_vertex}
\lean{QEC1.GaussFlux.fluxOp_supportZ_vertex}
\leanok
\uses{def:GaussFlux.fluxOp, def:PauliOp.supportZ}
The $Z$-support of $B_p$ on vertex qubits is empty: for all $w \in V$,
\[
\operatorname{inl}(w) \notin \operatorname{supportZ}(B_p).
\]
\end{theorem}

\begin{proof}
\leanok
\uses{def:GaussFlux.fluxOp, def:PauliOp.supportZ}
Let $w$ be an arbitrary vertex. We simplify using the definitions of $\operatorname{supportZ}$ and $\operatorname{fluxOp}$. Since $B_p.\operatorname{zVec}(\operatorname{inl}(w)) = 0$, the vertex qubit $\operatorname{inl}(w)$ does not belong to $\operatorname{supportZ}(B_p)$.
\end{proof}

\begin{definition}[Logical Operator]
\label{def:GaussFlux.logicalOp}
\lean{QEC1.GaussFlux.logicalOp}
\leanok
\uses{def:PauliOp, def:GaussFlux.ExtQubit}
The \emph{logical operator} $L = \prod_{v \in V} X_v$ on the extended system is the Pauli operator defined by:
\[
L.\operatorname{xVec}(q) = \begin{cases} 1 & \text{if } q = \operatorname{inl}(w) \text{ for some } w \in V, \\ 0 & \text{if } q = \operatorname{inr}(e), \end{cases}
\qquad
L.\operatorname{zVec} = 0.
\]
That is, $L$ acts with $X$ on all vertex qubits and with the identity on all edge qubits.
\end{definition}

\begin{lemma}[Gauss Product X-Vector on Vertices]
\label{lem:GaussFlux.gaussLaw_product_xVec_inl}
\lean{QEC1.GaussFlux.gaussLaw_product_xVec_inl}
\leanok
\uses{def:GaussFlux.gaussLawOp}
For each vertex $v \in V$,
\[
\sum_{w \in V} (A_w)_x(\operatorname{inl}(v)) = 1.
\]
\end{lemma}

\begin{proof}
\leanok
\uses{def:GaussFlux.gaussLawOp}
By simplification using the definition of $\operatorname{gaussLawOp}$. The $X$-vector of $A_w$ at $\operatorname{inl}(v)$ is $1$ if $w = v$ and $0$ otherwise, so the sum picks out exactly the single term $w = v$, giving $1$.
\end{proof}

\begin{lemma}[Gauss Product X-Vector on Edges]
\label{lem:GaussFlux.gaussLaw_product_xVec_inr}
\lean{QEC1.GaussFlux.gaussLaw_product_xVec_inr}
\leanok
\uses{def:GaussFlux.gaussLawOp}
For each edge $e \in G.\operatorname{edgeSet}$,
\[
\sum_{v \in V} (A_v)_x(\operatorname{inr}(e)) = 0.
\]
\end{lemma}

\begin{proof}
\leanok
\uses{def:GaussFlux.gaussLawOp}
We unfold the definition of $\operatorname{gaussLawOp}$. The sum counts, modulo $2$, the number of vertices $v$ with $v \in e$. We decompose the edge: let $e = \{a, b\}$ with $a \neq b$ (since $G$ has no loops). The sum $\sum_{v \in V} [\![v \in \{a,b\}]\!]$ equals $([\![v = a]\!] + [\![v = b]\!])$ summed over $v$.

We establish for each $v$ that the indicator $[\![v \in \{a,b\}]\!] = [\![v = a]\!] + [\![v = b]\!]$ by case analysis: if $v = a$, we use $a \neq b$ and simplify via $\operatorname{Sym2.mem\_iff}$; if $v = b$, similarly; if $v \neq a$ and $v \neq b$, then $v \notin \{a,b\}$ by $\operatorname{Sym2.mem\_iff}$ and pushing negation.

Rewriting with this identity, we distribute the sum: $\sum_v ([\![v = a]\!] + [\![v = b]\!]) = \sum_v [\![v = a]\!] + \sum_v [\![v = b]\!]$. Each sum evaluates to $1$ by $\operatorname{Finset.sum\_ite\_eq'}$ over the universe. The result is $1 + 1 = 0$ in $\mathbb{Z}/2\mathbb{Z}$ by $\operatorname{CharTwo.add\_self\_eq\_zero}$.
\end{proof}

\begin{theorem}[Gauss Product Property]
\label{thm:GaussFlux.gaussLaw_product}
\lean{QEC1.GaussFlux.gaussLaw_product}
\leanok
\uses{def:GaussFlux.gaussLawOp, def:GaussFlux.logicalOp, lem:GaussFlux.gaussLaw_product_xVec_inl, lem:GaussFlux.gaussLaw_product_xVec_inr}
The product of all Gauss's law operators equals the logical operator:
\[
\prod_{v \in V} A_v = L.
\]
Each vertex gets $X$ exactly once; each edge gets $X^2 = I$.
\end{theorem}

\begin{proof}
\leanok
\uses{def:GaussFlux.gaussLawOp, def:GaussFlux.logicalOp, lem:GaussFlux.gaussLaw_product_xVec_inl, lem:GaussFlux.gaussLaw_product_xVec_inr}
By extensionality, we must show that the $\operatorname{xVec}$ and $\operatorname{zVec}$ components agree.

\textbf{For xVec:} We rewrite the $\operatorname{xVec}$ of the product as a sum over the universe, then apply the private helper lemma \texttt{prod\_pauliOp\_xVec} to express $(\prod_v A_v)_x(q) = \sum_v (A_v)_x(q)$. We case-split on $q$:
\begin{itemize}
\item If $q = \operatorname{inl}(w)$ for a vertex $w$: by simplification using $\operatorname{gaussLawOp}$ and $\operatorname{logicalOp}$, we rewrite the sum $\sum_v [\![v = w]\!]$ using $\operatorname{Finset.sum\_ite\_eq'}$ over the universe to obtain $1$, which matches $L_x(\operatorname{inl}(w)) = 1$.
\item If $q = \operatorname{inr}(e)$ for an edge $e$: by simplification using $\operatorname{gaussLawOp}$ and $\operatorname{logicalOp}$, the result follows from $\operatorname{gaussLaw\_product\_xVec\_inr}$, which gives $0 = L_x(\operatorname{inr}(e))$.
\end{itemize}

\textbf{For zVec:} We similarly rewrite using \texttt{prod\_pauliOp\_zVec}. Since every $A_v$ has $\operatorname{zVec} = 0$, the sum is $0$ for all $q$, matching $L.\operatorname{zVec} = 0$. This is verified by simplification using the definitions of $\operatorname{gaussLawOp}$ and $\operatorname{logicalOp}$.
\end{proof}

\begin{theorem}[Logical Operator is Pure X-Type]
\label{thm:GaussFlux.logicalOp_is_pure_X}
\lean{QEC1.GaussFlux.logicalOp_is_pure_X}
\leanok
\uses{def:GaussFlux.logicalOp, def:PauliOp.supportZ}
The logical operator $L$ is pure $X$-type:
\[
\operatorname{supportZ}(L) = \emptyset.
\]
\end{theorem}

\begin{proof}
\leanok
\uses{def:GaussFlux.logicalOp, def:PauliOp.supportZ}
By extensionality, for any qubit $q$, we simplify using the definitions of $\operatorname{supportZ}$ and $\operatorname{logicalOp}$. Since $L.\operatorname{zVec} = 0$, no qubit belongs to $\operatorname{supportZ}(L)$.
\end{proof}

\begin{theorem}[Logical Operator Commutes with Flux]
\label{thm:GaussFlux.logical_flux_commute}
\lean{QEC1.GaussFlux.logical_flux_commute}
\leanok
\uses{def:GaussFlux.logicalOp, def:GaussFlux.fluxOp, def:PauliOp.PauliCommute, def:PauliOp.symplecticInner}
The logical operator commutes with all flux operators: for all $p \in C$,
\[
[L, B_p] = 0.
\]
\end{theorem}

\begin{proof}
\leanok
\uses{def:GaussFlux.logicalOp, def:GaussFlux.fluxOp, def:PauliOp.PauliCommute, def:PauliOp.symplecticInner}
We unfold the definitions of $\operatorname{PauliCommute}$ and $\operatorname{symplecticInner}$. The symplectic inner product is a sum over all qubits $q$. We show each summand is zero by case-splitting on $q$:
\begin{itemize}
\item If $q = \operatorname{inl}(w)$: by simplification using the definitions of $\operatorname{logicalOp}$ and $\operatorname{fluxOp}$, the term vanishes since $B_p.\operatorname{zVec}(\operatorname{inl}(w)) = 0$ and $B_p.\operatorname{xVec} = 0$.
\item If $q = \operatorname{inr}(e)$: by simplification using the definitions of $\operatorname{logicalOp}$ and $\operatorname{fluxOp}$, the term vanishes since $L.\operatorname{zVec} = 0$ and $L.\operatorname{xVec}(\operatorname{inr}(e)) = 0$.
\end{itemize}
\end{proof}

\begin{theorem}[All Commutation Relations]
\label{thm:GaussFlux.all_commutation_relations}
\lean{QEC1.GaussFlux.all_commutation_relations}
\leanok
\uses{thm:GaussFlux.gaussLaw_commute, thm:GaussFlux.flux_commute, thm:GaussFlux.gauss_flux_commute, def:GaussFlux.gaussLawOp, def:GaussFlux.fluxOp, def:PauliOp.PauliCommute}
All three commutation relations hold simultaneously, assuming the cycle incidence condition:
\begin{enumerate}
\item[(i)] $[A_v, A_w] = 0$ for all $v, w \in V$,
\item[(ii)] $[B_p, B_q] = 0$ for all $p, q \in C$,
\item[(iii)] $[A_v, B_p] = 0$ for all $v \in V$, $p \in C$.
\end{enumerate}
\end{theorem}

\begin{proof}
\leanok
\uses{thm:GaussFlux.gaussLaw_commute, thm:GaussFlux.flux_commute, thm:GaussFlux.gauss_flux_commute}
This follows directly by combining the three individual commutation theorems: part (i) from $\operatorname{gaussLaw\_commute}$, part (ii) from $\operatorname{flux\_commute}$, and part (iii) from $\operatorname{gauss\_flux\_commute}$.
\end{proof}

\begin{theorem}[Gauss Operator X-Vector Equals Coboundary]
\label{thm:GaussFlux.gaussLawOp_xVec_edge_eq_coboundary}
\lean{QEC1.GaussFlux.gaussLawOp_xVec_edge_eq_coboundary}
\leanok
\uses{def:GaussFlux.gaussLawOp, def:GraphMaps.coboundaryMap}
The Gauss's law operator $A_v$ has $X$-support on edges related to the coboundary map: for each edge $e$,
\[
(A_v)_x(\operatorname{inr}(e)) = \delta(\mathbf{1}_v)(e),
\]
where $\delta$ is the coboundary map and $\mathbf{1}_v$ is the indicator of vertex $v$.
\end{theorem}

\begin{proof}
\leanok
\uses{def:GaussFlux.gaussLawOp, def:GraphMaps.coboundaryMap, thm:GraphMaps.coboundaryMap_single}
We unfold the definition of $\operatorname{gaussLawOp}$ and then rewrite using $\operatorname{GraphMaps.coboundaryMap\_single}$, which expresses the coboundary of a basis vector $\mathbf{1}_v$ in exactly the same indicator form as $(A_v)_x$ on edges.
\end{proof}

\begin{theorem}[Flux Operator Z-Vector Equals Second Boundary]
\label{thm:GaussFlux.fluxOp_zVec_edge_eq_secondBoundary}
\lean{QEC1.GaussFlux.fluxOp_zVec_edge_eq_secondBoundary}
\leanok
\uses{def:GaussFlux.fluxOp, def:GraphMaps.secondBoundaryMap}
The flux operator $B_p$ has $Z$-support on edges related to the second boundary map: for each edge $e$,
\[
(B_p)_z(\operatorname{inr}(e)) = \partial_2(\mathbf{1}_p)(e),
\]
where $\partial_2$ is the second boundary map and $\mathbf{1}_p$ is the indicator of cycle $p$.
\end{theorem}

\begin{proof}
\leanok
\uses{def:GaussFlux.fluxOp, def:GraphMaps.secondBoundaryMap, thm:GraphMaps.secondBoundaryMap_single}
We unfold the definition of $\operatorname{fluxOp}$ and then rewrite using $\operatorname{GraphMaps.secondBoundaryMap\_single}$, which expresses the second boundary of a basis vector $\mathbf{1}_p$ in exactly the same indicator form as $(B_p)_z$ on edges.
\end{proof}

%--- Def_3: DeformedOperator ---
\chapter{Def 3: Deformed Operator}

We define the \emph{deformed operator} $\widetilde{P} = P \cdot \prod_{e \in \gamma} Z_e$ on the extended qubit system $V \oplus E$. A Pauli operator $P$ that commutes with the logical operator $L = \prod_{v \in V} X_v$ can be lifted to the extended system using an edge-path $\gamma$ satisfying the boundary condition $\partial \gamma = S_Z(P)|_V$.

\begin{definition}[Z-Support on Vertices]
\label{def:DeformedOperator.zSupportOnVertices}
\lean{QEC1.DeformedOperator.zSupportOnVertices}
\leanok
\uses{def:PauliOp}
The \emph{Z-support on vertices} of a Pauli operator $P$ on $V$ is the binary vector $\operatorname{zSupportOnVertices}(P) \in (\mathbb{Z}/2\mathbb{Z})^V$ defined by
\[
\operatorname{zSupportOnVertices}(P)(v) = \begin{cases} 1 & \text{if } P.\operatorname{zVec}(v) \neq 0, \\ 0 & \text{otherwise.} \end{cases}
\]
This is the characteristic function of $S_Z(P) \cap V$.
\end{definition}

\begin{definition}[Commutes With Logical]
\label{def:DeformedOperator.CommutesWithLogical}
\lean{QEC1.DeformedOperator.CommutesWithLogical}
\leanok
\uses{def:DeformedOperator.zSupportOnVertices, def:PauliOp}
A Pauli operator $P$ on $V$ \emph{commutes with the logical operator} $L = \prod_{v \in V} X_v$ if the sum of its Z-support on vertices vanishes in $\mathbb{Z}/2\mathbb{Z}$:
\[
\sum_{v \in V} \operatorname{zSupportOnVertices}(P)(v) = 0.
\]
Equivalently, $P$ has an even number of vertices with $Z$-action.
\end{definition}

\begin{definition}[Boundary Condition]
\label{def:DeformedOperator.BoundaryCondition}
\lean{QEC1.DeformedOperator.BoundaryCondition}
\leanok
\uses{def:DeformedOperator.zSupportOnVertices, def:GraphMaps.boundaryMap, def:PauliOp}
The \emph{boundary condition} for a Pauli operator $P$ on $V$ and an edge-path $\gamma \in (\mathbb{Z}/2\mathbb{Z})^{E}$ asserts that
\[
\partial \gamma = \operatorname{zSupportOnVertices}(P),
\]
where $\partial$ denotes the boundary map of the graph $G$.
\end{definition}

\begin{definition}[Deformed Operator on Extended Qubits]
\label{def:DeformedOperator.deformedOpExt}
\lean{QEC1.DeformedOperator.deformedOpExt}
\leanok
\uses{def:PauliOp, def:GaussFlux.ExtQubit}
The \emph{deformed operator} $\widetilde{P}$ on the extended qubit system $V \oplus E$ is defined as follows. Given a Pauli operator $P$ on $V$ and an edge-path $\gamma \in (\mathbb{Z}/2\mathbb{Z})^E$:
\begin{itemize}
\item On vertex qubits ($v \in V$): $\widetilde{P}.\operatorname{xVec}(v) = P.\operatorname{xVec}(v)$ and $\widetilde{P}.\operatorname{zVec}(v) = P.\operatorname{zVec}(v)$.
\item On edge qubits ($e \in E$): $\widetilde{P}.\operatorname{xVec}(e) = 0$ and $\widetilde{P}.\operatorname{zVec}(e) = \gamma(e)$.
\end{itemize}
That is, $\widetilde{P}$ acts as $P$ on vertex qubits and as $Z_e$ (if $\gamma(e)=1$) or identity (if $\gamma(e)=0$) on edge qubits.
\end{definition}

\begin{theorem}[Deforming the Identity]
\label{thm:DeformedOperator.deformedOpExt_one}
\lean{QEC1.DeformedOperator.deformedOpExt_one}
\leanok
\uses{def:DeformedOperator.deformedOpExt}
Deforming the identity operator with edge-path $\gamma$ gives a pure-$Z$ edge operator:
\[
\widetilde{\mathbf{1}}(\gamma).\operatorname{xVec}(q) = 0 \quad \text{for all } q \in V \oplus E,
\]
\[
\widetilde{\mathbf{1}}(\gamma).\operatorname{zVec}(q) = \begin{cases} 0 & \text{if } q \in V, \\ \gamma(e) & \text{if } q = e \in E. \end{cases}
\]
\end{theorem}
\begin{proof}
\leanok
\uses{def:DeformedOperator.deformedOpExt}
By extensionality, it suffices to check each component. For each qubit $q \in V \oplus E$, we case-split on whether $q$ is a vertex or an edge qubit and simplify using the definition of $\operatorname{deformedOpExt}$.
\end{proof}

\begin{theorem}[Deforming with Zero Path]
\label{thm:DeformedOperator.deformedOpExt_zero_path}
\lean{QEC1.DeformedOperator.deformedOpExt_zero_path}
\leanok
\uses{def:DeformedOperator.deformedOpExt}
Deforming $P$ with the zero edge-path extends $P$ trivially to the edge qubits:
\[
\widetilde{P}(0).\operatorname{xVec}(q) = \begin{cases} P.\operatorname{xVec}(v) & \text{if } q = v \in V, \\ 0 & \text{if } q \in E, \end{cases}
\]
\[
\widetilde{P}(0).\operatorname{zVec}(q) = \begin{cases} P.\operatorname{zVec}(v) & \text{if } q = v \in V, \\ 0 & \text{if } q \in E. \end{cases}
\]
\end{theorem}
\begin{proof}
\leanok
\uses{def:DeformedOperator.deformedOpExt}
By extensionality, it suffices to check each component. For each qubit $q$, we case-split on whether $q$ is a vertex or edge qubit and simplify using the definition of $\operatorname{deformedOpExt}$.
\end{proof}

\begin{theorem}[Self-Inverse Property]
\label{thm:DeformedOperator.deformedOpExt_mul_self}
\lean{QEC1.DeformedOperator.deformedOpExt_mul_self}
\leanok
\uses{def:DeformedOperator.deformedOpExt}
The deformed operator is self-inverse: $\widetilde{P} \cdot \widetilde{P} = \mathbf{1}$.
\end{theorem}
\begin{proof}
\leanok
\uses{def:DeformedOperator.deformedOpExt}
By extensionality, it suffices to verify equality on each qubit $q \in V \oplus E$.

For the $\operatorname{xVec}$ component: if $q = v \in V$, then $(\widetilde{P} \cdot \widetilde{P}).\operatorname{xVec}(v) = P.\operatorname{xVec}(v) + P.\operatorname{xVec}(v) = 0$ by the characteristic-two identity $a + a = 0$ in $\mathbb{Z}/2\mathbb{Z}$. If $q = e \in E$, then $(\widetilde{P} \cdot \widetilde{P}).\operatorname{xVec}(e) = 0 + 0 = 0$, which equals $\mathbf{1}.\operatorname{xVec}(e)$.

For the $\operatorname{zVec}$ component: if $q = v \in V$, then $(\widetilde{P} \cdot \widetilde{P}).\operatorname{zVec}(v) = P.\operatorname{zVec}(v) + P.\operatorname{zVec}(v) = 0$ by the same characteristic-two identity. If $q = e \in E$, then $(\widetilde{P} \cdot \widetilde{P}).\operatorname{zVec}(e) = \gamma(e) + \gamma(e) = 0$.
\end{proof}

\begin{lemma}[Sum of Z-Support Equals Cardinality]
\label{lem:DeformedOperator.sum_zSupportOnVertices_eq_card}
\lean{QEC1.DeformedOperator.sum_zSupportOnVertices_eq_card}
\leanok
\uses{def:DeformedOperator.zSupportOnVertices, def:PauliOp}
For any Pauli operator $P$ on $V$,
\[
\sum_{v \in V} \operatorname{zSupportOnVertices}(P)(v) = |\{v \in V \mid P.\operatorname{zVec}(v) \neq 0\}| \pmod{2}.
\]
\end{lemma}
\begin{proof}
\leanok
\uses{def:DeformedOperator.zSupportOnVertices}
Expanding the definition of $\operatorname{zSupportOnVertices}$, the left-hand side is $\sum_{v \in V} \mathbf{1}_{P.\operatorname{zVec}(v) \neq 0}$, which by rewriting in terms of the boolean indicator function equals the cardinality of $\{v \in V \mid P.\operatorname{zVec}(v) \neq 0\}$ cast to $\mathbb{Z}/2\mathbb{Z}$.
\end{proof}

\begin{theorem}[Commutativity Iff Even Z-Support]
\label{thm:DeformedOperator.commutesWithLogical_iff_even_zSupport}
\lean{QEC1.DeformedOperator.commutesWithLogical_iff_even_zSupport}
\leanok
\uses{def:DeformedOperator.CommutesWithLogical, lem:DeformedOperator.sum_zSupportOnVertices_eq_card, def:PauliOp}
A Pauli operator $P$ commutes with the logical operator if and only if $|\{v \in V \mid P.\operatorname{zVec}(v) \neq 0\}|$ is even:
\[
\operatorname{CommutesWithLogical}(P) \iff 2 \mid |\{v \in V \mid P.\operatorname{zVec}(v) \neq 0\}|.
\]
\end{theorem}
\begin{proof}
\leanok
\uses{def:DeformedOperator.CommutesWithLogical, lem:DeformedOperator.sum_zSupportOnVertices_eq_card}
Unfolding the definition of $\operatorname{CommutesWithLogical}$, the condition becomes $\sum_v \operatorname{zSupportOnVertices}(P)(v) = 0$ in $\mathbb{Z}/2\mathbb{Z}$. By the lemma $\operatorname{sum\_zSupportOnVertices\_eq\_card}$, this sum equals the cardinality cast to $\mathbb{Z}/2\mathbb{Z}$. The result then follows from the fact that $n = 0$ in $\mathbb{Z}/2\mathbb{Z}$ if and only if $n$ is even.
\end{proof}

\begin{theorem}[Boundary Sum Equals Zero]
\label{thm:DeformedOperator.boundary_sum_eq_zero}
\lean{QEC1.DeformedOperator.boundary_sum_eq_zero}
\leanok
\uses{def:GraphMaps.boundaryMap}
For any edge-path $\gamma \in (\mathbb{Z}/2\mathbb{Z})^E$,
\[
\sum_{v \in V} (\partial \gamma)(v) = 0,
\]
since each edge contributes to exactly two vertices.
\end{theorem}
\begin{proof}
\leanok
\uses{def:GraphMaps.boundaryMap}
Expanding the definition of the boundary map, we have
\[
\sum_{v \in V} (\partial \gamma)(v) = \sum_{v \in V} \sum_{e \in E} \begin{cases} \gamma(e) & \text{if } v \in e, \\ 0 & \text{otherwise.} \end{cases}
\]
Swapping the order of summation, it suffices to show that each inner sum $\sum_{v \in V} \mathbf{1}_{v \in e} \cdot \gamma(e)$ vanishes. We factor out $\gamma(e)$ to obtain $\bigl(\sum_{v \in V} \mathbf{1}_{v \in e}\bigr) \cdot \gamma(e)$.

For each edge $e = \{a, b\}$ with $a \neq b$ (since the graph has no loops), we have $\sum_{v \in V} \mathbf{1}_{v \in \{a,b\}} = \mathbf{1}_{v=a} + \mathbf{1}_{v=b}$. After evaluating the sum over all vertices, we get the contribution $1 + 1 = 0$ in $\mathbb{Z}/2\mathbb{Z}$, so each edge's contribution is $0 \cdot \gamma(e) = 0$. The total sum is therefore $0$.
\end{proof}

\begin{theorem}[Boundary Condition Implies Commutativity]
\label{thm:DeformedOperator.boundaryCondition_implies_commutes}
\lean{QEC1.DeformedOperator.boundaryCondition_implies_commutes}
\leanok
\uses{def:DeformedOperator.BoundaryCondition, def:DeformedOperator.CommutesWithLogical, thm:DeformedOperator.boundary_sum_eq_zero, def:PauliOp}
If the boundary condition $\partial \gamma = \operatorname{zSupportOnVertices}(P)$ holds, then $P$ commutes with the logical operator.
\end{theorem}
\begin{proof}
\leanok
\uses{def:DeformedOperator.BoundaryCondition, def:DeformedOperator.CommutesWithLogical, thm:DeformedOperator.boundary_sum_eq_zero}
Unfolding the definition of $\operatorname{CommutesWithLogical}$, we need to show $\sum_v \operatorname{zSupportOnVertices}(P)(v) = 0$. Unfolding the boundary condition $\partial \gamma = \operatorname{zSupportOnVertices}(P)$, we rewrite the goal as $\sum_v (\partial \gamma)(v) = 0$. This follows directly from the theorem $\operatorname{boundary\_sum\_eq\_zero}$.
\end{proof}

\begin{theorem}[Commutation with Gauss's Law]
\label{thm:DeformedOperator.deformedOpExt_comm_gaussLaw}
\lean{QEC1.DeformedOperator.deformedOpExt_comm_gaussLaw}
\leanok
\uses{def:DeformedOperator.deformedOpExt, def:DeformedOperator.BoundaryCondition, def:GaussFlux.gaussLawOp, def:PauliOp.PauliCommute, def:PauliOp.symplecticInner, def:PauliOp}
Let $P$ be a Pauli operator on $V$, $\gamma$ an edge-path, and suppose the boundary condition $\partial \gamma = \operatorname{zSupportOnVertices}(P)$ holds. Then for every vertex $v \in V$, the deformed operator $\widetilde{P}$ commutes with the Gauss's law operator $A_v$:
\[
\operatorname{PauliCommute}(\widetilde{P}, A_v).
\]
\end{theorem}
\begin{proof}
\leanok
\uses{def:DeformedOperator.deformedOpExt, def:DeformedOperator.BoundaryCondition, def:DeformedOperator.zSupportOnVertices, def:GaussFlux.gaussLawOp, def:PauliOp.PauliCommute, def:PauliOp.symplecticInner}
Expanding the definition of $\operatorname{PauliCommute}$ and $\operatorname{symplecticInner}$, we need to show
\[
\sum_{q \in V \oplus E} \bigl(\widetilde{P}.\operatorname{xVec}(q) \cdot A_v.\operatorname{zVec}(q) + \widetilde{P}.\operatorname{zVec}(q) \cdot A_v.\operatorname{xVec}(q)\bigr) = 0.
\]
We split the sum over the type $V \oplus E$.

\textbf{Vertex contribution:} We first establish that
\[
\sum_{w \in V} \bigl(\widetilde{P}.\operatorname{xVec}(w) \cdot A_v.\operatorname{zVec}(w) + \widetilde{P}.\operatorname{zVec}(w) \cdot A_v.\operatorname{xVec}(w)\bigr) = P.\operatorname{zVec}(v).
\]
Expanding the definitions of $\widetilde{P}$ and $A_v$, the $A_v.\operatorname{zVec}$ component on vertices is $0$, so the first term vanishes. The $A_v.\operatorname{xVec}$ component on vertex $w$ is $\mathbf{1}_{w=v}$, so the sum reduces to $\sum_w P.\operatorname{zVec}(w) \cdot \mathbf{1}_{w=v} = P.\operatorname{zVec}(v)$.

\textbf{Edge contribution:} We establish that
\[
\sum_{e \in E} \bigl(\widetilde{P}.\operatorname{xVec}(e) \cdot A_v.\operatorname{zVec}(e) + \widetilde{P}.\operatorname{zVec}(e) \cdot A_v.\operatorname{xVec}(e)\bigr) = \sum_{e \in E} \begin{cases} \gamma(e) & \text{if } v \in e, \\ 0 & \text{otherwise.} \end{cases}
\]
Since $\widetilde{P}.\operatorname{xVec}(e) = 0$ on edges, the first term vanishes. The second term is $\gamma(e) \cdot \mathbf{1}_{v \in e}$, which gives the claimed expression after case-splitting on whether $v \in e$.

\textbf{Combining:} The total symplectic inner product is $P.\operatorname{zVec}(v) + \sum_e \mathbf{1}_{v \in e} \cdot \gamma(e)$. From the boundary condition, $(\partial \gamma)(v) = \operatorname{zSupportOnVertices}(P)(v)$. Expanding the boundary map, $(\partial \gamma)(v) = \sum_e \mathbf{1}_{v \in e} \cdot \gamma(e)$. Rewriting using the boundary condition, the total becomes $P.\operatorname{zVec}(v) + \operatorname{zSupportOnVertices}(P)(v)$. If $P.\operatorname{zVec}(v) = 0$, then $\operatorname{zSupportOnVertices}(P)(v) = 0$ and the sum is $0$. If $P.\operatorname{zVec}(v) \neq 0$, then $P.\operatorname{zVec}(v) = 1$ and $\operatorname{zSupportOnVertices}(P)(v) = 1$ in $\mathbb{Z}/2\mathbb{Z}$, so the sum is $1 + 1 = 0$ by the characteristic-two identity.
\end{proof}

\begin{theorem}[Compatibility with Multiplication]
\label{thm:DeformedOperator.deformedOpExt_mul}
\lean{QEC1.DeformedOperator.deformedOpExt_mul}
\leanok
\uses{def:DeformedOperator.deformedOpExt, def:PauliOp}
For Pauli operators $P, Q$ on $V$ and edge-paths $\gamma_1, \gamma_2$,
\[
\widetilde{P}(\gamma_1) \cdot \widetilde{Q}(\gamma_2) = \widetilde{P \cdot Q}(\gamma_1 + \gamma_2).
\]
\end{theorem}
\begin{proof}
\leanok
\uses{def:DeformedOperator.deformedOpExt}
By extensionality, we verify equality on each qubit $q \in V \oplus E$.

For the $\operatorname{xVec}$ component: if $q = v \in V$, then both sides equal $P.\operatorname{xVec}(v) + Q.\operatorname{xVec}(v)$ by simplification using the definition of $\operatorname{deformedOpExt}$. If $q = e \in E$, then both sides equal $0 + 0 = 0$.

For the $\operatorname{zVec}$ component: if $q = v \in V$, then both sides equal $P.\operatorname{zVec}(v) + Q.\operatorname{zVec}(v)$. If $q = e \in E$, then both sides equal $\gamma_1(e) + \gamma_2(e)$ by the pointwise addition of functions.
\end{proof}

\begin{lemma}[Additivity of Z-Support on Vertices]
\label{lem:DeformedOperator.zSupportOnVertices_mul}
\lean{QEC1.DeformedOperator.zSupportOnVertices_mul}
\leanok
\uses{def:DeformedOperator.zSupportOnVertices, def:PauliOp}
The Z-support on vertices is additive under Pauli multiplication: for all $v \in V$,
\[
\operatorname{zSupportOnVertices}(P \cdot Q)(v) = \operatorname{zSupportOnVertices}(P)(v) + \operatorname{zSupportOnVertices}(Q)(v).
\]
\end{lemma}
\begin{proof}
\leanok
\uses{def:DeformedOperator.zSupportOnVertices, def:PauliOp}
Expanding the definition, $\operatorname{zSupportOnVertices}$ is defined using the indicator $\mathbf{1}_{P.\operatorname{zVec}(v) \neq 0}$, and $P \cdot Q$ has $\operatorname{zVec}(v) = P.\operatorname{zVec}(v) + Q.\operatorname{zVec}(v)$. We use the fact that in $\mathbb{Z}/2\mathbb{Z}$, $a \neq 0 \iff a = 1$. We then case-split on whether $P.\operatorname{zVec}(v) = 0$ and $Q.\operatorname{zVec}(v) = 0$:
\begin{itemize}
\item If both are $0$: both sides are $0$.
\item If $P.\operatorname{zVec}(v) = 0$ and $Q.\operatorname{zVec}(v) \neq 0$: left side is $1$, right side is $0 + 1 = 1$.
\item If $P.\operatorname{zVec}(v) \neq 0$ and $Q.\operatorname{zVec}(v) = 0$: left side is $1$, right side is $1 + 0 = 1$.
\item If both are nonzero: $P.\operatorname{zVec}(v) + Q.\operatorname{zVec}(v) = 1 + 1 = 0$ in $\mathbb{Z}/2\mathbb{Z}$ by the characteristic-two identity, so the left side is $0$, and the right side is $1 + 1 = 0$.
\end{itemize}
\end{proof}

\begin{theorem}[Boundary Condition Compatible with Multiplication]
\label{thm:DeformedOperator.boundaryCondition_mul}
\lean{QEC1.DeformedOperator.boundaryCondition_mul}
\leanok
\uses{def:DeformedOperator.BoundaryCondition, def:DeformedOperator.zSupportOnVertices, def:GraphMaps.boundaryMap, lem:DeformedOperator.zSupportOnVertices_mul, def:PauliOp}
If $\partial \gamma_1 = \operatorname{zSupportOnVertices}(P)$ and $\partial \gamma_2 = \operatorname{zSupportOnVertices}(Q)$, then
\[
\partial(\gamma_1 + \gamma_2) = \operatorname{zSupportOnVertices}(P \cdot Q).
\]
\end{theorem}
\begin{proof}
\leanok
\uses{def:DeformedOperator.BoundaryCondition, def:GraphMaps.boundaryMap, lem:DeformedOperator.zSupportOnVertices_mul}
Unfolding the boundary condition at both hypotheses, we use the linearity of the boundary map: $\partial(\gamma_1 + \gamma_2) = \partial \gamma_1 + \partial \gamma_2$. By extensionality, for each vertex $v$ we compute
\[
\partial(\gamma_1 + \gamma_2)(v) = (\partial \gamma_1)(v) + (\partial \gamma_2)(v) = \operatorname{zSupportOnVertices}(P)(v) + \operatorname{zSupportOnVertices}(Q)(v),
\]
using the hypotheses. By the additivity lemma $\operatorname{zSupportOnVertices\_mul}$, the right-hand side equals $\operatorname{zSupportOnVertices}(P \cdot Q)(v)$.
\end{proof}

\begin{definition}[No Z-Support on V]
\label{def:DeformedOperator.HasNoZSupportOnV}
\lean{QEC1.DeformedOperator.HasNoZSupportOnV}
\leanok
\uses{def:DeformedOperator.zSupportOnVertices, def:PauliOp}
A Pauli operator $P$ on $V$ \emph{has no Z-support on $V$} if its Z-support on vertices is the zero vector:
\[
\operatorname{zSupportOnVertices}(P) = 0.
\]
\end{definition}

\begin{theorem}[No Z-Support Implies Trivial Boundary]
\label{thm:DeformedOperator.noZSupport_boundary_zero}
\lean{QEC1.DeformedOperator.noZSupport_boundary_zero}
\leanok
\uses{def:DeformedOperator.HasNoZSupportOnV, def:DeformedOperator.BoundaryCondition, def:GraphMaps.boundaryMap}
If $P$ has no Z-support on $V$, then the boundary condition is satisfied by $\gamma = 0$:
\[
\operatorname{HasNoZSupportOnV}(P) \implies \operatorname{BoundaryCondition}(P, 0).
\]
\end{theorem}
\begin{proof}
\leanok
\uses{def:DeformedOperator.HasNoZSupportOnV, def:DeformedOperator.BoundaryCondition, def:GraphMaps.boundaryMap}
Unfolding the boundary condition, we need $\partial 0 = \operatorname{zSupportOnVertices}(P)$. Since the boundary map is linear, $\partial 0 = 0$. The hypothesis gives $\operatorname{zSupportOnVertices}(P) = 0$, so the two sides are equal by symmetry.
\end{proof}

\begin{definition}[Deformed Operator]
\label{def:DeformedOperator.deformedOperator}
\lean{QEC1.DeformedOperator.deformedOperator}
\leanok
\uses{def:DeformedOperator.deformedOpExt, def:DeformedOperator.BoundaryCondition, def:PauliOp, def:GaussFlux.ExtQubit}
The \emph{deformed operator} $\widetilde{P} = P \cdot \prod_{e \in \gamma} Z_e$ on the extended qubit system $V \oplus E$ is the main construction from Definition~3. Given a Pauli operator $P$ on $V$ and an edge-path $\gamma$ satisfying the boundary condition $\partial \gamma = S_Z(P)|_V$, the deformed operator is defined as
\[
\widetilde{P} := \operatorname{deformedOpExt}(G, P, \gamma).
\]
On vertex qubits it acts as $P$, and on edge qubits it acts as $Z_e$ if $\gamma(e) = 1$ and as the identity otherwise. It commutes with all Gauss's law operators $A_v$.
\end{definition}

%--- Rem_6: NoncommutingOperatorsCannotBeDeformed ---
\chapter{Rem 6: Noncommuting Operators Cannot Be Deformed}

A Pauli operator $P$ that does not commute with $L$ cannot be deformed to commute with all Gauss's law operators $A_v$. The boundary condition $\partial \gamma = S_Z(P)|_V$ has no solution when $\operatorname{CommutesWithLogical}(P)$ fails, and multiplying $P$ by $Z_e$ operators or commuting stabilizers cannot change this.

\begin{theorem}[Noncommuting Operators Cannot Be Deformed]
\label{thm:NoncommutingOperatorsCannotBeDeformed.noncommuting_cannot_be_deformed}
\lean{QEC1.NoncommutingOperatorsCannotBeDeformed.noncommuting_cannot_be_deformed}
\leanok
\uses{def:PauliOp, def:DeformedOperator.BoundaryCondition, def:DeformedOperator.zSupportOnVertices, thm:DeformedOperator.boundaryCondition_implies_commutes}
If $\neg \operatorname{CommutesWithLogical}(P)$, then there is no edge-path $\gamma \colon E \to \mathbb{Z}/2\mathbb{Z}$ satisfying the boundary condition $\operatorname{BoundaryCondition}(G, P, \gamma)$. This is the contrapositive of the theorem that the boundary condition implies commutation.
\end{theorem}

\begin{proof}
\leanok
\uses{thm:DeformedOperator.boundaryCondition_implies_commutes}
Suppose for contradiction that there exists $\gamma$ with $\operatorname{BoundaryCondition}(G, P, \gamma)$. Then by \texttt{boundaryCondition\_implies\_commutes}, we conclude $\operatorname{CommutesWithLogical}(P)$, contradicting the hypothesis $\neg \operatorname{CommutesWithLogical}(P)$.
\end{proof}

\begin{theorem}[Z-Support Sum Additivity]
\label{thm:NoncommutingOperatorsCannotBeDeformed.zSupportOnVertices_sum_mul}
\lean{QEC1.NoncommutingOperatorsCannotBeDeformed.zSupportOnVertices_sum_mul}
\leanok
\uses{def:PauliOp, def:DeformedOperator.zSupportOnVertices, lem:DeformedOperator.zSupportOnVertices_mul}
The sum of $\operatorname{zSupportOnVertices}$ is additive under Pauli multiplication:
\[
\sum_{v \in V} \operatorname{zSupportOnVertices}(P \cdot Q)(v) = \sum_{v \in V} \operatorname{zSupportOnVertices}(P)(v) + \sum_{v \in V} \operatorname{zSupportOnVertices}(Q)(v).
\]
\end{theorem}

\begin{proof}
\leanok
\uses{lem:DeformedOperator.zSupportOnVertices_mul}
Rewriting via the distributivity of finite sums, $\sum_v f(v) + \sum_v g(v) = \sum_v (f(v) + g(v))$, it suffices to show equality pointwise. By extensionality, for each $v$, this follows directly from \texttt{zSupportOnVertices\_mul}, which gives $\operatorname{zSupportOnVertices}(P \cdot Q)(v) = \operatorname{zSupportOnVertices}(P)(v) + \operatorname{zSupportOnVertices}(Q)(v)$.
\end{proof}

\begin{theorem}[Multiplication Preserves CommutesWithLogical]
\label{thm:NoncommutingOperatorsCannotBeDeformed.mul_commuting_preserves_commutesWithLogical}
\lean{QEC1.NoncommutingOperatorsCannotBeDeformed.mul_commuting_preserves_commutesWithLogical}
\leanok
\uses{def:PauliOp, def:DeformedOperator.zSupportOnVertices}
If $\operatorname{CommutesWithLogical}(P)$ and $\operatorname{CommutesWithLogical}(Q)$, then $\operatorname{CommutesWithLogical}(P \cdot Q)$.
\end{theorem}

\begin{proof}
\leanok
\uses{def:DeformedOperator.zSupportOnVertices}
Unfolding $\operatorname{CommutesWithLogical}$ in all hypotheses and the goal, we have $\sum_v \operatorname{zSupportOnVertices}(P)(v) = 0$ and $\sum_v \operatorname{zSupportOnVertices}(Q)(v) = 0$. By the Z-support sum additivity, $\sum_v \operatorname{zSupportOnVertices}(P \cdot Q)(v) = 0 + 0 = 0$.
\end{proof}

\begin{theorem}[Contrapositive of Multiplication Preservation]
\label{thm:NoncommutingOperatorsCannotBeDeformed.mul_commuting_contrapositive}
\lean{QEC1.NoncommutingOperatorsCannotBeDeformed.mul_commuting_contrapositive}
\leanok
\uses{def:PauliOp, def:DeformedOperator.zSupportOnVertices}
If $\neg \operatorname{CommutesWithLogical}(P \cdot Q)$ and $\operatorname{CommutesWithLogical}(Q)$, then $\neg \operatorname{CommutesWithLogical}(P)$.
\end{theorem}

\begin{proof}
\leanok
\uses{def:DeformedOperator.zSupportOnVertices}
Assume for contradiction that $\operatorname{CommutesWithLogical}(P)$ holds. Then by \texttt{mul\_commuting\_preserves\_commutesWithLogical} applied to $P$ and $Q$, we obtain $\operatorname{CommutesWithLogical}(P \cdot Q)$, contradicting the hypothesis.
\end{proof}

\begin{lemma}[Z-Support Sum of pauliZ]
\label{lem:NoncommutingOperatorsCannotBeDeformed.zSupportOnVertices_singleZ_sum}
\lean{QEC1.NoncommutingOperatorsCannotBeDeformed.zSupportOnVertices_singleZ_sum}
\leanok
\uses{def:PauliOp, def:PauliOp.pauliZ, def:DeformedOperator.zSupportOnVertices}
The sum of $\operatorname{zSupportOnVertices}$ of $\operatorname{pauliZ}(v)$ equals $1$:
\[
\sum_{w \in V} \operatorname{zSupportOnVertices}(\operatorname{pauliZ}(v))(w) = 1.
\]
\end{lemma}

\begin{proof}
\leanok
\uses{def:PauliOp.pauliZ, def:DeformedOperator.zSupportOnVertices}
Unfolding $\operatorname{zSupportOnVertices}$, the goal becomes $\sum_w (\text{if } (\operatorname{pauliZ}(v)).\mathit{zVec}(w) \neq 0 \text{ then } 1 \text{ else } 0) = 1$. We first establish that for each $w$, the indicator equals $\text{if } w = v \text{ then } 1 \text{ else } 0$. For the case $w = v$: substituting, by \texttt{Pi.single\_eq\_same} the $z$-vector value at $v$ is nonzero, giving $1$. For $w \neq v$: by \texttt{Pi.single\_eq\_of\_ne}, the $z$-vector at $w$ is $0$, giving $0$. Rewriting the sum with these equalities and simplifying using the characteristic sum $\sum_w [\![w = v]\!] = 1$, we obtain $1$.
\end{proof}

\begin{theorem}[pauliZ Does Not Commute With Logical]
\label{thm:NoncommutingOperatorsCannotBeDeformed.singleZ_not_commutesWithLogical}
\lean{QEC1.NoncommutingOperatorsCannotBeDeformed.singleZ_not_commutesWithLogical}
\leanok
\uses{def:PauliOp, def:PauliOp.pauliZ, def:DeformedOperator.zSupportOnVertices}
For any vertex $v$, $\neg \operatorname{CommutesWithLogical}(\operatorname{pauliZ}(v))$.
\end{theorem}

\begin{proof}
\leanok
\uses{def:PauliOp.pauliZ, def:DeformedOperator.zSupportOnVertices}
Unfolding $\operatorname{CommutesWithLogical}$, the goal becomes $\sum_w \operatorname{zSupportOnVertices}(\operatorname{pauliZ}(v))(w) \neq 0$. By the lemma \texttt{zSupportOnVertices\_singleZ\_sum}, this sum equals $1$. Since $1 \neq 0$ in $\mathbb{Z}/2\mathbb{Z}$, the result follows.
\end{proof}

\begin{theorem}[pauliZ Flips CommutesWithLogical]
\label{thm:NoncommutingOperatorsCannotBeDeformed.singleZ_flips_commutesWithLogical}
\lean{QEC1.NoncommutingOperatorsCannotBeDeformed.singleZ_flips_commutesWithLogical}
\leanok
\uses{def:PauliOp, def:PauliOp.pauliZ, def:DeformedOperator.zSupportOnVertices}
Multiplying by $\operatorname{pauliZ}(v)$ flips the $\operatorname{CommutesWithLogical}$ condition:
\[
\operatorname{CommutesWithLogical}(P \cdot \operatorname{pauliZ}(v)) \iff \neg \operatorname{CommutesWithLogical}(P).
\]
\end{theorem}

\begin{proof}
\leanok
\uses{def:PauliOp.pauliZ, def:DeformedOperator.zSupportOnVertices}
Unfolding $\operatorname{CommutesWithLogical}$ and rewriting via the Z-support sum additivity and the fact that $\sum_w \operatorname{zSupportOnVertices}(\operatorname{pauliZ}(v))(w) = 1$, the goal reduces to showing $\sum_v \operatorname{zSupportOnVertices}(P)(v) + 1 = 0 \iff \sum_v \operatorname{zSupportOnVertices}(P)(v) \neq 0$ in $\mathbb{Z}/2\mathbb{Z}$.

For the forward direction: assume $\sum_v \operatorname{zSupportOnVertices}(P)(v) + 1 = 0$ and suppose for contradiction that $\sum_v \operatorname{zSupportOnVertices}(P)(v) = 0$. Substituting gives $0 + 1 = 0$, i.e.\ $1 = 0$, a contradiction.

For the reverse direction: assume $\sum_v \operatorname{zSupportOnVertices}(P)(v) \neq 0$. In $\mathbb{Z}/2\mathbb{Z}$, every element is $0$ or $1$. Since the sum is not $0$, it must be $1$. Then $1 + 1 = 0$ in $\mathbb{Z}/2\mathbb{Z}$ (by the characteristic-two identity), as required.
\end{proof}

\begin{definition}[Z-Support Restricted to Vertices]
\label{def:NoncommutingOperatorsCannotBeDeformed.zSupportRestricted}
\lean{QEC1.NoncommutingOperatorsCannotBeDeformed.zSupportRestricted}
\leanok
\uses{def:PauliOp, def:GaussFlux.ExtQubit}
For a Pauli operator $P$ on the extended qubit system $V \oplus E$, the \emph{Z-support restricted to vertices} is defined as
\[
\operatorname{zSupportRestricted}(P) = \sum_{v \in V} \begin{cases} 1 & \text{if } P.\mathit{zVec}(\operatorname{inl}(v)) \neq 0, \\ 0 & \text{otherwise.} \end{cases}
\]
This captures the Z-support on the vertex qubits only.
\end{definition}

\begin{definition}[CommutesWithLogical on Extended System]
\label{def:NoncommutingOperatorsCannotBeDeformed.CommutesWithLogical'}
\lean{QEC1.NoncommutingOperatorsCannotBeDeformed.CommutesWithLogical'}
\leanok
\uses{def:PauliOp, def:GaussFlux.ExtQubit}
For a Pauli operator $P$ on the extended qubit system $V \oplus E$, we define $\operatorname{CommutesWithLogical'}(P)$ as the condition that $\operatorname{zSupportRestricted}(G, P) = 0$.
\end{definition}

\begin{theorem}[Edge pauliZ Preserves CommutesWithLogical']
\label{thm:NoncommutingOperatorsCannotBeDeformed.singleZ_edge_preserves_commutesWithLogical'}
\lean{QEC1.NoncommutingOperatorsCannotBeDeformed.singleZ_edge_preserves_commutesWithLogical'}
\leanok
\uses{def:PauliOp, def:PauliOp.pauliZ, def:GaussFlux.ExtQubit}
Multiplying a Pauli operator $P$ on the extended system $V \oplus E$ by $\operatorname{pauliZ}(\operatorname{inr}(e))$ for an edge qubit $e$ does not change the $\operatorname{CommutesWithLogical'}$ condition:
\[
\operatorname{CommutesWithLogical'}(P \cdot \operatorname{pauliZ}(\operatorname{inr}(e))) \iff \operatorname{CommutesWithLogical'}(P).
\]
This is because $\operatorname{pauliZ}$ on an edge qubit $\operatorname{inr}(e)$ has zero Z-support on vertex qubits $\operatorname{inl}(v)$.
\end{theorem}

\begin{proof}
\leanok
\uses{def:PauliOp.pauliZ, def:GaussFlux.ExtQubit}
Unfolding $\operatorname{CommutesWithLogical'}$ and $\operatorname{zSupportRestricted}$, it suffices to show that for each vertex $v$, the indicator $[\![(P \cdot \operatorname{pauliZ}(\operatorname{inr}(e))).\mathit{zVec}(\operatorname{inl}(v)) \neq 0]\!]$ equals $[\![P.\mathit{zVec}(\operatorname{inl}(v)) \neq 0]\!]$. We establish that $(\operatorname{pauliZ}(\operatorname{inr}(e))).\mathit{zVec}(\operatorname{inl}(v)) = 0$ by simplifying: since $\operatorname{inl}(v) \neq \operatorname{inr}(e)$, the single function evaluates to $0$ by \texttt{Pi.single\_eq\_of\_ne}. Then the multiplication formula for $\mathit{zVec}$ gives $(P \cdot \operatorname{pauliZ}(\operatorname{inr}(e))).\mathit{zVec}(\operatorname{inl}(v)) = P.\mathit{zVec}(\operatorname{inl}(v)) + 0 = P.\mathit{zVec}(\operatorname{inl}(v))$, and the result follows by simplification.
\end{proof}

\begin{theorem}[Stabilizer Preserves Non-Commuting Status]
\label{thm:NoncommutingOperatorsCannotBeDeformed.stabilizer_preserves_noncommuting}
\lean{QEC1.NoncommutingOperatorsCannotBeDeformed.stabilizer_preserves_noncommuting}
\leanok
\uses{def:PauliOp, def:DeformedOperator.zSupportOnVertices}
If $\neg \operatorname{CommutesWithLogical}(P)$ and $\operatorname{CommutesWithLogical}(s)$, then $\neg \operatorname{CommutesWithLogical}(P \cdot s)$.
\end{theorem}

\begin{proof}
\leanok
\uses{def:DeformedOperator.zSupportOnVertices}
Assume for contradiction that $\operatorname{CommutesWithLogical}(P \cdot s)$ holds. We show $\operatorname{CommutesWithLogical}(P)$, contradicting $\neg \operatorname{CommutesWithLogical}(P)$. First note that $P \cdot s \cdot s = P$by associativity, $s \cdot s = \operatorname{id}$ (Pauli self-inverse), and $P \cdot \operatorname{id} = P$. Rewriting the goal using $P = P \cdot s \cdot s$, the result follows from \texttt{mul\_commuting\_preserves\_commutesWithLogical} applied to $P \cdot s$ and $s$.
\end{proof}

\begin{theorem}[No Deformation Exists]
\label{thm:NoncommutingOperatorsCannotBeDeformed.no_deformation_exists}
\lean{QEC1.NoncommutingOperatorsCannotBeDeformed.no_deformation_exists}
\leanok
\uses{def:PauliOp, def:DeformedOperator.BoundaryCondition, def:DeformedOperator.zSupportOnVertices, thm:DeformedOperator.boundaryCondition_implies_commutes}
For every edge-path $\gamma$, the boundary condition fails when $\neg \operatorname{CommutesWithLogical}(P)$.
\end{theorem}

\begin{proof}
\leanok
\uses{thm:DeformedOperator.boundaryCondition_implies_commutes}
Suppose $\operatorname{BoundaryCondition}(G, P, \gamma)$ holds. By \texttt{boundaryCondition\_implies\_commutes}, $\operatorname{CommutesWithLogical}(P)$, contradicting the hypothesis.
\end{proof}

\begin{theorem}[CommutesWithLogical Invariant Under Commuting Multiplication]
\label{thm:NoncommutingOperatorsCannotBeDeformed.commutesWithLogical_mul_iff}
\lean{QEC1.NoncommutingOperatorsCannotBeDeformed.commutesWithLogical_mul_iff}
\leanok
\uses{def:PauliOp, def:DeformedOperator.zSupportOnVertices}
If $\operatorname{CommutesWithLogical}(Q)$, then
\[
\operatorname{CommutesWithLogical}(P \cdot Q) \iff \operatorname{CommutesWithLogical}(P).
\]
\end{theorem}

\begin{proof}
\leanok
\uses{def:DeformedOperator.zSupportOnVertices}
For the forward direction: assume $\operatorname{CommutesWithLogical}(P \cdot Q)$. Note that $P \cdot Q \cdot Q = P$ by associativity, self-inverse ($Q \cdot Q = \operatorname{id}$), and the identity law. Rewriting, $\operatorname{CommutesWithLogical}(P)$ follows from \texttt{mul\_commuting\_preserves\_commutesWithLogical} applied to $P \cdot Q$ and $Q$.

For the reverse direction: assume $\operatorname{CommutesWithLogical}(P)$. Then $\operatorname{CommutesWithLogical}(P \cdot Q)$ follows directly from \texttt{mul\_commuting\_preserves\_commutesWithLogical} applied to $P$ and $Q$.
\end{proof}

\begin{theorem}[No Modification Helps]
\label{thm:NoncommutingOperatorsCannotBeDeformed.no_modification_helps}
\lean{QEC1.NoncommutingOperatorsCannotBeDeformed.no_modification_helps}
\leanok
\uses{def:PauliOp, def:DeformedOperator.zSupportOnVertices}
If $\neg \operatorname{CommutesWithLogical}(P)$ and $\operatorname{CommutesWithLogical}(Q)$, then $\neg \operatorname{CommutesWithLogical}(P \cdot Q)$. No product of $Z_e$ operators and commuting stabilizers can help.
\end{theorem}

\begin{proof}
\leanok
\uses{def:DeformedOperator.zSupportOnVertices}
By the equivalence $\operatorname{CommutesWithLogical}(P \cdot Q) \iff \operatorname{CommutesWithLogical}(P)$ (from \texttt{commutesWithLogical\_mul\_iff} using $\operatorname{CommutesWithLogical}(Q)$), and rewriting with the hypothesis $\neg \operatorname{CommutesWithLogical}(P)$, we obtain $\neg \operatorname{CommutesWithLogical}(P \cdot Q)$.
\end{proof}

\begin{theorem}[No Modified Deformation Exists]
\label{thm:NoncommutingOperatorsCannotBeDeformed.no_modified_deformation_exists}
\lean{QEC1.NoncommutingOperatorsCannotBeDeformed.no_modified_deformation_exists}
\leanok
\uses{def:PauliOp, def:DeformedOperator.BoundaryCondition, def:DeformedOperator.zSupportOnVertices, thm:DeformedOperator.boundaryCondition_implies_commutes}
If $\neg \operatorname{CommutesWithLogical}(P)$ and $\operatorname{CommutesWithLogical}(Q)$, then no boundary condition holds for $P \cdot Q$ either: for every $\gamma$, $\neg \operatorname{BoundaryCondition}(G, P \cdot Q, \gamma)$.
\end{theorem}

\begin{proof}
\leanok
\uses{thm:DeformedOperator.boundaryCondition_implies_commutes}
Suppose $\operatorname{BoundaryCondition}(G, P \cdot Q, \gamma)$ holds. By \texttt{boundaryCondition\_implies\_commutes}, $\operatorname{CommutesWithLogical}(P \cdot Q)$. But by \texttt{no\_modification\_helps}, $\neg \operatorname{CommutesWithLogical}(P \cdot Q)$, a contradiction.
\end{proof}

\begin{lemma}[Boundary Condition Equivalent to Range Membership]
\label{lem:NoncommutingOperatorsCannotBeDeformed.boundaryCondition_exists_iff_in_image}
\lean{QEC1.NoncommutingOperatorsCannotBeDeformed.boundaryCondition_exists_iff_in_image}
\leanok
\uses{def:PauliOp, def:DeformedOperator.BoundaryCondition, def:DeformedOperator.zSupportOnVertices, def:GraphMaps.boundaryMap}
The existence of a boundary condition for $P$ is equivalent to the Z-support on vertices being in the image of the boundary map:
\[
(\exists \gamma,\; \operatorname{BoundaryCondition}(G, P, \gamma)) \iff \operatorname{zSupportOnVertices}(P) \in \operatorname{range}(\partial).
\]
\end{lemma}

\begin{proof}
\leanok
\uses{def:DeformedOperator.BoundaryCondition, def:GraphMaps.boundaryMap}
Unfolding $\operatorname{BoundaryCondition}$, both directions follow directly. For the forward direction: given $\gamma$ and $h_\gamma$ witnessing the boundary condition, we have $\operatorname{zSupportOnVertices}(P) \in \operatorname{range}(\partial)$ by the membership criterion $\langle \gamma, h_\gamma \rangle$. For the reverse direction: given $\gamma$ and $h_\gamma$ from the range membership, we obtain the boundary condition $\langle \gamma, h_\gamma \rangle$.
\end{proof}

\begin{theorem}[Z-Support Not in Range of Boundary Map]
\label{thm:NoncommutingOperatorsCannotBeDeformed.zSupport_not_in_boundary_range}
\lean{QEC1.NoncommutingOperatorsCannotBeDeformed.zSupport_not_in_boundary_range}
\leanok
\uses{def:PauliOp, def:DeformedOperator.BoundaryCondition, def:DeformedOperator.zSupportOnVertices, def:GraphMaps.boundaryMap, thm:DeformedOperator.boundaryCondition_implies_commutes}
If $\neg \operatorname{CommutesWithLogical}(P)$, then $\operatorname{zSupportOnVertices}(P) \notin \operatorname{range}(\partial)$.
\end{theorem}

\begin{proof}
\leanok
\uses{def:DeformedOperator.BoundaryCondition, def:GraphMaps.boundaryMap, thm:DeformedOperator.boundaryCondition_implies_commutes}
Rewriting via the equivalence \texttt{boundaryCondition\_exists\_iff\_in\_image}, the goal becomes $\neg \exists \gamma,\; \operatorname{BoundaryCondition}(G, P, \gamma)$. This follows directly from \texttt{noncommuting\_cannot\_be\_deformed}.
\end{proof}

%--- Def_4: DeformedCode ---

\chapter{Def 4: Deformed Code}

\begin{definition}[Deformed Check]
\label{def:DeformedCode.deformedCheck}
\lean{QEC1.DeformedCode.deformedCheck}
\leanok
\uses{def:DeformedOperator.deformedOpExt, def:GaussFlux.ExtQubit}
Given a graph $G = (V, E)$, an original check $s \in \mathcal{P}_V$ (a Pauli operator on $V$), and an edge-path $\gamma : E(G) \to \mathbb{Z}_2$ satisfying the boundary condition, the \emph{deformed check} is the deformed operator extension of $s$ by $\gamma$:
\[
\tilde{s} := \widetilde{s}^{(\gamma)} = \operatorname{deformedOpExt}(G, s, \gamma).
\]
\end{definition}

\begin{theorem}[Deformed Check with No Z-Support]
\label{thm:DeformedCode.deformedCheck_of_noZSupport}
\lean{QEC1.DeformedCode.deformedCheck_of_noZSupport}
\leanok
\uses{def:DeformedCode.deformedCheck, def:DeformedOperator.HasNoZSupportOnV, def:DeformedOperator.deformedOpExt}
If $s$ has no $Z$-support on $V$, then the deformed check with $\gamma = 0$ satisfies
\[
\operatorname{deformedCheck}(G, s, 0) = \operatorname{deformedOpExt}(G, s, 0).
\]
That is, when $s$ has no $Z$-support, the zero edge-path suffices and the deformed check acts as $s$ on vertex qubits and as the identity on edge qubits.
\end{theorem}

\begin{proof}
\leanok
\uses{def:DeformedCode.deformedCheck, def:DeformedOperator.deformedOpExt}
This holds by definitional equality (reflexivity), since $\operatorname{deformedCheck}$ is defined as $\operatorname{deformedOpExt}$.
\end{proof}

\begin{theorem}[Deformed Check Commutes with Gauss's Law]
\label{thm:DeformedCode.deformedCheck_comm_gaussLaw}
\lean{QEC1.DeformedCode.deformedCheck_comm_gaussLaw}
\leanok
\uses{def:DeformedCode.deformedCheck, def:DeformedOperator.BoundaryCondition, def:GaussFlux.gaussLawOp, def:PauliOp.PauliCommute, thm:DeformedOperator.deformedOpExt_comm_gaussLaw}
Let $s$ be a Pauli operator on $V$, $\gamma : E(G) \to \mathbb{Z}_2$ an edge-path satisfying the boundary condition $\partial\gamma = S_Z(s)|_V$, and $v \in V$. Then the deformed check $\tilde{s}$ commutes with the Gauss's law operator $A_v$:
\[
[\tilde{s},\, A_v] = 0.
\]
\end{theorem}

\begin{proof}
\leanok
\uses{thm:DeformedOperator.deformedOpExt_comm_gaussLaw}
This follows directly from the theorem that the deformed operator extension commutes with Gauss's law when the boundary condition holds.
\end{proof}

\begin{theorem}[Deformed Check is Self-Inverse]
\label{thm:DeformedCode.deformedCheck_mul_self}
\lean{QEC1.DeformedCode.deformedCheck_mul_self}
\leanok
\uses{def:DeformedCode.deformedCheck, thm:DeformedOperator.deformedOpExt_mul_self}
For any Pauli operator $s$ on $V$ and edge-path $\gamma : E(G) \to \mathbb{Z}_2$, the deformed check is self-inverse:
\[
\tilde{s} \cdot \tilde{s} = \mathbf{1}.
\]
\end{theorem}

\begin{proof}
\leanok
\uses{thm:DeformedOperator.deformedOpExt_mul_self}
This follows directly from the self-inverse property of the deformed operator extension.
\end{proof}

\begin{definition}[Deformed Code Data]
\label{def:DeformedCode.DeformedCodeData}
\lean{QEC1.DeformedCode.DeformedCodeData}
\leanok
\uses{def:DeformedOperator.BoundaryCondition}
The \emph{deformed code data} consists of:
\begin{itemize}
\item For each original check index $j \in J$, an edge-path $\gamma_j : E(G) \to \mathbb{Z}_2$.
\item For each $j \in J$, a proof that the boundary condition holds: $\partial \gamma_j = S_Z(s_j)|_V$.
\end{itemize}
\end{definition}

\begin{definition}[Gauss's Law Checks]
\label{def:DeformedCode.gaussLawChecks}
\lean{QEC1.DeformedCode.gaussLawChecks}
\leanok
\uses{def:GaussFlux.gaussLawOp, def:GaussFlux.ExtQubit}
The \emph{Gauss's law checks} on the extended qubit system $V \oplus E(G)$ are defined by
\[
A_v := \operatorname{gaussLawOp}(G, v)
\]
for each vertex $v \in V$.
\end{definition}

\begin{definition}[Flux Checks]
\label{def:DeformedCode.fluxChecks}
\lean{QEC1.DeformedCode.fluxChecks}
\leanok
\uses{def:GaussFlux.fluxOp, def:GaussFlux.ExtQubit}
The \emph{flux checks} on the extended qubit system $V \oplus E(G)$ are defined by
\[
B_p := \operatorname{fluxOp}(G, \mathrm{cycles}, p)
\]
for each cycle $p \in C$.
\end{definition}

\begin{definition}[Deformed Original Checks]
\label{def:DeformedCode.deformedOriginalChecks}
\lean{QEC1.DeformedCode.deformedOriginalChecks}
\leanok
\uses{def:DeformedCode.deformedCheck, def:DeformedCode.DeformedCodeData}
The \emph{deformed original checks} $\tilde{s}_j$ for each $j \in J$ are defined using the edge-paths from the deformed code data:
\[
\tilde{s}_j := \operatorname{deformedCheck}(G,\, s_j,\, \gamma_j)
\]
where $\gamma_j$ is the edge-path provided by the deformed code data.
\end{definition}

\begin{definition}[Check Index Type]
\label{def:DeformedCode.CheckIndex}
\lean{QEC1.DeformedCode.CheckIndex}
\leanok
The \emph{check index type} for the deformed code is the inductive type $\operatorname{CheckIndex}(V, C, J)$ with three constructors:
\begin{itemize}
\item $\operatorname{gaussLaw}(v)$ for $v \in V$, indexing the Gauss's law checks,
\item $\operatorname{flux}(p)$ for $p \in C$, indexing the flux checks,
\item $\operatorname{deformed}(j)$ for $j \in J$, indexing the deformed original checks.
\end{itemize}
This type is equivalent to the sum type $V \oplus C \oplus J$.
\end{definition}

\begin{definition}[All Checks Map]
\label{def:DeformedCode.allChecks}
\lean{QEC1.DeformedCode.allChecks}
\leanok
\uses{def:DeformedCode.CheckIndex, def:DeformedCode.gaussLawChecks, def:DeformedCode.fluxChecks, def:DeformedCode.deformedOriginalChecks, def:DeformedCode.DeformedCodeData}
The \emph{full set of checks} for the deformed code is the function
\[
\operatorname{allChecks} : \operatorname{CheckIndex}(V, C, J) \to \mathcal{P}_{V \oplus E(G)}
\]
defined by:
\begin{align*}
\operatorname{allChecks}(\operatorname{gaussLaw}(v)) &= A_v, \\
\operatorname{allChecks}(\operatorname{flux}(p)) &= B_p, \\
\operatorname{allChecks}(\operatorname{deformed}(j)) &= \tilde{s}_j.
\end{align*}
\end{definition}

\begin{theorem}[Gauss--Gauss Commutation]
\label{thm:DeformedCode.allChecks_pairwise_commute_gaussLaw_gaussLaw}
\lean{QEC1.DeformedCode.allChecks_pairwise_commute_gaussLaw_gaussLaw}
\leanok
\uses{def:DeformedCode.gaussLawChecks, def:PauliOp.PauliCommute, thm:GaussFlux.gaussLaw_commute}
For all $v, w \in V$, the Gauss's law checks commute:
\[
[A_v,\, A_w] = 0.
\]
\end{theorem}

\begin{proof}
\leanok
\uses{thm:GaussFlux.gaussLaw_commute}
This follows directly from the commutation of Gauss's law operators (Theorem~\ref{thm:GaussFlux.gaussLaw_commute}).
\end{proof}

\begin{theorem}[Flux--Flux Commutation]
\label{thm:DeformedCode.allChecks_pairwise_commute_flux_flux}
\lean{QEC1.DeformedCode.allChecks_pairwise_commute_flux_flux}
\leanok
\uses{def:DeformedCode.fluxChecks, def:PauliOp.PauliCommute, thm:GaussFlux.flux_commute}
For all $p, q \in C$, the flux checks commute:
\[
[B_p,\, B_q] = 0.
\]
\end{theorem}

\begin{proof}
\leanok
\uses{thm:GaussFlux.flux_commute}
This follows directly from the commutation of flux operators (Theorem~\ref{thm:GaussFlux.flux_commute}).
\end{proof}

\begin{theorem}[Gauss--Flux Commutation]
\label{thm:DeformedCode.allChecks_pairwise_commute_gaussLaw_flux}
\lean{QEC1.DeformedCode.allChecks_pairwise_commute_gaussLaw_flux}
\leanok
\uses{def:DeformedCode.gaussLawChecks, def:DeformedCode.fluxChecks, def:PauliOp.PauliCommute, thm:GaussFlux.gauss_flux_commute}
Assume that for every cycle $c \in C$ and vertex $v \in V$, the number of edges in the cycle incident to $v$ is even. Then for all $v \in V$ and $p \in C$, the Gauss's law and flux checks commute:
\[
[A_v,\, B_p] = 0.
\]
\end{theorem}

\begin{proof}
\leanok
\uses{thm:GaussFlux.gauss_flux_commute}
This follows directly from the Gauss--flux commutation theorem (Theorem~\ref{thm:GaussFlux.gauss_flux_commute}).
\end{proof}

\begin{theorem}[Gauss--Deformed Commutation]
\label{thm:DeformedCode.allChecks_pairwise_commute_gaussLaw_deformed}
\lean{QEC1.DeformedCode.allChecks_pairwise_commute_gaussLaw_deformed}
\leanok
\uses{def:DeformedCode.gaussLawChecks, def:DeformedCode.deformedOriginalChecks, def:DeformedCode.DeformedCodeData, def:PauliOp.PauliCommute, thm:DeformedOperator.deformedOpExt_comm_gaussLaw}
For any deformed code data, for all $v \in V$ and $j \in J$, the Gauss's law checks commute with the deformed original checks:
\[
[A_v,\, \tilde{s}_j] = 0.
\]
\end{theorem}

\begin{proof}
\leanok
\uses{thm:DeformedOperator.deformedOpExt_comm_gaussLaw, thm:PauliOp.pauliCommute_comm}
By the commutativity of Pauli commutation (i.e., $[A, B] = 0 \Leftrightarrow [B, A] = 0$), it suffices to show that $\tilde{s}_j$ commutes with $A_v$. This follows directly from the theorem that the deformed operator extension commutes with Gauss's law when the boundary condition holds, which is guaranteed by the deformed code data.
\end{proof}

\begin{theorem}[Deformed--Flux Commutation]
\label{thm:DeformedCode.allChecks_pairwise_commute_deformed_flux}
\lean{QEC1.DeformedCode.allChecks_pairwise_commute_deformed_flux}
\leanok
\uses{def:DeformedCode.deformedOriginalChecks, def:DeformedCode.fluxChecks, def:DeformedCode.DeformedCodeData, def:PauliOp.PauliCommute, def:PauliOp.symplecticInner, def:DeformedOperator.deformedOpExt, def:GaussFlux.fluxOp}
For any deformed code data, for all $j \in J$ and $p \in C$, the deformed original checks commute with the flux checks:
\[
[\tilde{s}_j,\, B_p] = 0.
\]
This holds because flux operators are pure $Z$-type and deformed checks have no $X$-support on edges.
\end{theorem}

\begin{proof}
\leanok
\uses{def:DeformedCode.deformedOriginalChecks, def:DeformedCode.deformedCheck, def:DeformedCode.fluxChecks, def:PauliOp.PauliCommute, def:PauliOp.symplecticInner, def:DeformedOperator.deformedOpExt, def:GaussFlux.fluxOp}
We unfold the definitions of the deformed original checks, deformed check, and flux checks, and express the Pauli commutation condition in terms of the symplectic inner product. We decompose the sum over the extended qubit system $V \oplus E(G)$ into the vertex part and the edge part using $\operatorname{Fintype.sum\_sum\_type}$.

\textbf{Vertex part:} We show that
\[
\sum_{v \in V} \bigl( x_{\tilde{s}_j}(v) \cdot z_{B_p}(v) + z_{\tilde{s}_j}(v) \cdot x_{B_p}(v) \bigr) = 0.
\]
For each vertex $v$, by the definitions of $\operatorname{deformedOpExt}$ and $\operatorname{fluxOp}$, we have $z_{B_p}(v) = 0$ and $x_{B_p}(v) = 0$ on vertex qubits, so each summand vanishes. By summing zero terms, the vertex contribution is $0$.

We rewrite using $0 + (\text{edge sum}) = (\text{edge sum})$.

\textbf{Edge part:} We show that
\[
\sum_{e \in E(G)} \bigl( x_{\tilde{s}_j}(e) \cdot z_{B_p}(e) + z_{\tilde{s}_j}(e) \cdot x_{B_p}(e) \bigr) = 0.
\]
For each edge $e$, by the definitions of $\operatorname{deformedOpExt}$ and $\operatorname{fluxOp}$, the deformed check has $x_{\tilde{s}_j}(e) = 0$ on edge qubits (no $X$-support on edges), and the flux operator has $x_{B_p}(e) = 0$ (pure $Z$-type). Thus each summand vanishes, and the edge contribution is $0$.
\end{proof}

\begin{theorem}[Deformed--Deformed Commutation]
\label{thm:DeformedCode.allChecks_pairwise_commute_deformed_deformed}
\lean{QEC1.DeformedCode.allChecks_pairwise_commute_deformed_deformed}
\leanok
\uses{def:DeformedCode.deformedOriginalChecks, def:DeformedCode.DeformedCodeData, def:PauliOp.PauliCommute, def:PauliOp.symplecticInner, def:DeformedOperator.deformedOpExt}
Assume that the original checks pairwise commute: for all $i, j \in J$, $[s_i, s_j] = 0$. Then for any deformed code data and all $i, j \in J$, the deformed original checks commute:
\[
[\tilde{s}_i,\, \tilde{s}_j] = 0.
\]
On edges both deformed checks are pure $Z$-type ($Z$ commutes with $Z$); on vertices the commutation is inherited from the original checks.
\end{theorem}

\begin{proof}
\leanok
\uses{def:DeformedCode.deformedOriginalChecks, def:DeformedCode.deformedCheck, def:PauliOp.PauliCommute, def:PauliOp.symplecticInner, def:DeformedOperator.deformedOpExt}
We unfold the definitions of the deformed original checks and deformed check, and express the Pauli commutation condition via the symplectic inner product. We decompose the sum over $V \oplus E(G)$ into the vertex and edge parts.

\textbf{Edge part:} We show that
\[
\sum_{e \in E(G)} \bigl( x_{\tilde{s}_i}(e) \cdot z_{\tilde{s}_j}(e) + z_{\tilde{s}_i}(e) \cdot x_{\tilde{s}_j}(e) \bigr) = 0.
\]
For each edge $e$, by the definition of $\operatorname{deformedOpExt}$, both deformed checks have $x_{\tilde{s}_k}(e) = 0$ on edge qubits. Thus each summand vanishes and the edge sum is $0$.

\textbf{Vertex part:} We show that
\[
\sum_{v \in V} \bigl( x_{\tilde{s}_i}(v) \cdot z_{\tilde{s}_j}(v) + z_{\tilde{s}_i}(v) \cdot x_{\tilde{s}_j}(v) \bigr) = 0.
\]
From the hypothesis, the original checks $s_i$ and $s_j$ commute, i.e., $\langle s_i, s_j \rangle_{\mathrm{symp}} = 0$. Since the deformed operator extension on vertex qubits agrees with the original operator, the vertex sum equals the symplectic inner product of the original checks, which is $0$ by hypothesis. We use \texttt{convert} to match the goal with the hypothesis $h_{ij}$.

Combining both parts: the total symplectic inner product is $0 + 0 = 0$.
\end{proof}

\begin{theorem}[All Checks Commute]
\label{thm:DeformedCode.allChecks_commute}
\lean{QEC1.DeformedCode.allChecks_commute}
\leanok
\uses{def:DeformedCode.allChecks, def:DeformedCode.CheckIndex, def:DeformedCode.DeformedCodeData, def:PauliOp.PauliCommute, thm:DeformedCode.allChecks_pairwise_commute_gaussLaw_gaussLaw, thm:DeformedCode.allChecks_pairwise_commute_flux_flux, thm:DeformedCode.allChecks_pairwise_commute_gaussLaw_flux, thm:DeformedCode.allChecks_pairwise_commute_gaussLaw_deformed, thm:DeformedCode.allChecks_pairwise_commute_deformed_flux, thm:DeformedCode.allChecks_pairwise_commute_deformed_deformed}
Assume that:
\begin{enumerate}
\item For every cycle $c \in C$ and vertex $v \in V$, the number of edges in the cycle incident to $v$ is even.
\item The original checks pairwise commute: for all $i, j \in J$, $[s_i, s_j] = 0$.
\end{enumerate}
Then all checks in the deformed code pairwise commute: for all $a, b \in \operatorname{CheckIndex}(V, C, J)$,
\[
[\operatorname{allChecks}(a),\, \operatorname{allChecks}(b)] = 0.
\]
\end{theorem}

\begin{proof}
\leanok
\uses{thm:DeformedCode.allChecks_pairwise_commute_gaussLaw_gaussLaw, thm:DeformedCode.allChecks_pairwise_commute_flux_flux, thm:DeformedCode.allChecks_pairwise_commute_gaussLaw_flux, thm:DeformedCode.allChecks_pairwise_commute_gaussLaw_deformed, thm:DeformedCode.allChecks_pairwise_commute_deformed_flux, thm:DeformedCode.allChecks_pairwise_commute_deformed_deformed, thm:PauliOp.pauliCommute_comm}
We proceed by case analysis on the check indices $a$ and $b$.

\textbf{Case $a = \operatorname{gaussLaw}(v)$:}
\begin{itemize}
\item If $b = \operatorname{gaussLaw}(w)$: by Gauss--Gauss commutation.
\item If $b = \operatorname{flux}(p)$: by Gauss--flux commutation.
\item If $b = \operatorname{deformed}(j)$: by Gauss--deformed commutation.
\end{itemize}

\textbf{Case $a = \operatorname{flux}(p)$:}
\begin{itemize}
\item If $b = \operatorname{gaussLaw}(v)$: by commutativity of Pauli commutation and Gauss--flux commutation.
\item If $b = \operatorname{flux}(q)$: by flux--flux commutation.
\item If $b = \operatorname{deformed}(j)$: by commutativity of Pauli commutation and deformed--flux commutation.
\end{itemize}

\textbf{Case $a = \operatorname{deformed}(i)$:}
\begin{itemize}
\item If $b = \operatorname{gaussLaw}(v)$: by commutativity of Pauli commutation and Gauss--deformed commutation.
\item If $b = \operatorname{flux}(p)$: by deformed--flux commutation.
\item If $b = \operatorname{deformed}(j)$: by deformed--deformed commutation.
\end{itemize}
\end{proof}

\begin{theorem}[Gauss's Law Checks are Self-Inverse]
\label{thm:DeformedCode.gaussLawChecks_mul_self}
\lean{QEC1.DeformedCode.gaussLawChecks_mul_self}
\leanok
\uses{def:DeformedCode.gaussLawChecks}
For all $v \in V$, the Gauss's law check is self-inverse:
\[
A_v \cdot A_v = \mathbf{1}.
\]
\end{theorem}

\begin{proof}
\leanok
\uses{lem:PauliOp.mul_self}
This follows directly from the fact that every Pauli operator is self-inverse ($P \cdot P = \mathbf{1}$).
\end{proof}

\begin{theorem}[Flux Checks are Self-Inverse]
\label{thm:DeformedCode.fluxChecks_mul_self}
\lean{QEC1.DeformedCode.fluxChecks_mul_self}
\leanok
\uses{def:DeformedCode.fluxChecks}
For all $p \in C$, the flux check is self-inverse:
\[
B_p \cdot B_p = \mathbf{1}.
\]
\end{theorem}

\begin{proof}
\leanok
\uses{lem:PauliOp.mul_self}
This follows directly from the self-inverse property of Pauli operators.
\end{proof}

\begin{theorem}[Deformed Original Checks are Self-Inverse]
\label{thm:DeformedCode.deformedOriginalChecks_mul_self}
\lean{QEC1.DeformedCode.deformedOriginalChecks_mul_self}
\leanok
\uses{def:DeformedCode.deformedOriginalChecks, def:DeformedCode.DeformedCodeData, thm:DeformedCode.deformedCheck_mul_self}
For any deformed code data and all $j \in J$, the deformed original check is self-inverse:
\[
\tilde{s}_j \cdot \tilde{s}_j = \mathbf{1}.
\]
\end{theorem}

\begin{proof}
\leanok
\uses{thm:DeformedCode.deformedCheck_mul_self}
This follows from the self-inverse property of the deformed check applied to $s_j$ and $\gamma_j$.
\end{proof}

\begin{theorem}[All Checks are Self-Inverse]
\label{thm:DeformedCode.allChecks_self_inverse}
\lean{QEC1.DeformedCode.allChecks_self_inverse}
\leanok
\uses{def:DeformedCode.allChecks, def:DeformedCode.CheckIndex, def:DeformedCode.DeformedCodeData, thm:DeformedCode.gaussLawChecks_mul_self, thm:DeformedCode.fluxChecks_mul_self, thm:DeformedCode.deformedOriginalChecks_mul_self}
For any deformed code data and all check indices $a \in \operatorname{CheckIndex}(V, C, J)$,
\[
\operatorname{allChecks}(a) \cdot \operatorname{allChecks}(a) = \mathbf{1}.
\]
\end{theorem}

\begin{proof}
\leanok
\uses{thm:DeformedCode.gaussLawChecks_mul_self, thm:DeformedCode.fluxChecks_mul_self, thm:DeformedCode.deformedOriginalChecks_mul_self}
We proceed by case analysis on $a$:
\begin{itemize}
\item If $a = \operatorname{gaussLaw}(v)$: by self-inverse property of Gauss's law checks.
\item If $a = \operatorname{flux}(p)$: by self-inverse property of flux checks.
\item If $a = \operatorname{deformed}(j)$: by self-inverse property of deformed original checks.
\end{itemize}
\end{proof}

\begin{theorem}[Gauss's Law Checks are Pure X-Type]
\label{thm:DeformedCode.gaussLawChecks_pure_X}
\lean{QEC1.DeformedCode.gaussLawChecks_pure_X}
\leanok
\uses{def:DeformedCode.gaussLawChecks, lem:GaussFlux.gaussLawOp_zVec}
For all $v \in V$, the Gauss's law check $A_v$ is pure $X$-type, i.e., it has no $Z$-support:
\[
z_{A_v} = 0.
\]
\end{theorem}

\begin{proof}
\leanok
\uses{lem:GaussFlux.gaussLawOp_zVec}
This follows directly from the fact that the Gauss's law operator has zero $Z$-vector.
\end{proof}

\begin{theorem}[Flux Checks are Pure Z-Type]
\label{thm:DeformedCode.fluxChecks_pure_Z}
\lean{QEC1.DeformedCode.fluxChecks_pure_Z}
\leanok
\uses{def:DeformedCode.fluxChecks, lem:GaussFlux.fluxOp_xVec}
For all $p \in C$, the flux check $B_p$ is pure $Z$-type, i.e., it has no $X$-support:
\[
x_{B_p} = 0.
\]
\end{theorem}

\begin{proof}
\leanok
\uses{lem:GaussFlux.fluxOp_xVec}
This follows directly from the fact that the flux operator has zero $X$-vector.
\end{proof}

\begin{theorem}[Product of Gauss's Law Checks Equals Logical Operator]
\label{thm:DeformedCode.gaussLawChecks_product_eq_logical}
\lean{QEC1.DeformedCode.gaussLawChecks_product_eq_logical}
\leanok
\uses{def:DeformedCode.gaussLawChecks, def:GaussFlux.logicalOp, thm:GaussFlux.gaussLaw_product}
The product of all Gauss's law checks equals the logical operator $L$:
\[
\prod_{v \in V} A_v = L.
\]
\end{theorem}

\begin{proof}
\leanok
\uses{thm:GaussFlux.gaussLaw_product}
We convert the goal to the statement of the Gauss's law product theorem and apply it directly.
\end{proof}

\begin{definition}[Constructor for Deformed Code Data]
\label{def:DeformedCode.mkDeformedCodeData}
\lean{QEC1.DeformedCode.mkDeformedCodeData}
\leanok
\uses{def:DeformedCode.DeformedCodeData, def:DeformedOperator.BoundaryCondition}
Given edge-paths $\gamma_j : E(G) \to \mathbb{Z}_2$ for each $j \in J$ and proofs that the boundary condition $\partial \gamma_j = S_Z(s_j)|_V$ holds for each $j$, construct the deformed code data.
\end{definition}

\begin{definition}[Constructor for No Z-Support Case]
\label{def:DeformedCode.mkDeformedCodeData_noZSupport}
\lean{QEC1.DeformedCode.mkDeformedCodeData_noZSupport}
\leanok
\uses{def:DeformedCode.DeformedCodeData, def:DeformedOperator.HasNoZSupportOnV, thm:DeformedOperator.noZSupport_boundary_zero}
When all original checks $s_j$ have no $Z$-support on $V$, the deformed code data can be constructed by setting all edge-paths to zero: $\gamma_j = 0$ for all $j \in J$. The boundary condition is satisfied because the zero edge-path has zero boundary, which equals the (empty) $Z$-support restriction when there is no $Z$-support.
\end{definition}

\begin{theorem}[Weight of Flux Checks]
\label{thm:DeformedCode.fluxChecks_weight}
\lean{QEC1.DeformedCode.fluxChecks_weight}
\leanok
\uses{def:DeformedCode.fluxChecks, def:GaussFlux.fluxOp, def:PauliOp.weight, def:PauliOp.supportX, def:PauliOp.supportZ, thm:GaussFlux.fluxOp_is_pure_Z, lem:PauliOp.support_eq_supportX_union_supportZ}
The weight of the flux check $B_p$ equals the number of edges in the cycle $p$:
\[
\operatorname{wt}(B_p) = |\{e \in E(G) : e \in p\}|.
\]
\end{theorem}

\begin{proof}
\leanok
\uses{thm:GaussFlux.fluxOp_is_pure_Z, lem:PauliOp.support_eq_supportX_union_supportZ}
We unfold the definitions of the flux checks and Pauli operator weight.

First, we establish that $B_p$ has no $X$-support: $\operatorname{supportX}(B_p) = \emptyset$, which follows from the fact that flux operators are pure $Z$-type.

Next, we characterize the $Z$-support of $B_p$. By extensionality, for each qubit $q$ in the extended system $V \oplus E(G)$:
\begin{itemize}
\item If $q = \operatorname{inl}(v)$ for some vertex $v$: the $Z$-component of $\operatorname{fluxOp}$ is $0$ on vertices, so $q \notin \operatorname{supportZ}(B_p)$. No edge maps to a vertex under $\operatorname{inr}$, confirming both directions of the equivalence.
\item If $q = \operatorname{inr}(e)$ for some edge $e$: $q \in \operatorname{supportZ}(B_p)$ if and only if $e \in p$. In the forward direction, if $e \notin p$ then by definition of $\operatorname{fluxOp}$ the $Z$-component is $0$, a contradiction. In the reverse direction, if $e \in p$ then the $Z$-component is $1 \neq 0$.
\end{itemize}
Thus $\operatorname{supportZ}(B_p) = \operatorname{inr}(\{e \in E(G) : e \in p\})$, i.e., it is the image of the set of edges in the cycle under the embedding $\operatorname{inr}$.

Since $\operatorname{support}(B_p) = \operatorname{supportX}(B_p) \cup \operatorname{supportZ}(B_p) = \emptyset \cup \operatorname{supportZ}(B_p) = \operatorname{supportZ}(B_p)$, the weight equals $|\operatorname{supportZ}(B_p)|$. Since $\operatorname{inr}$ is injective, $|\operatorname{inr}(\{e \in E(G) : e \in p\})| = |\{e \in E(G) : e \in p\}|$.
\end{proof}

\begin{definition}[Deformed Code Checks]
\label{def:DeformedCode.deformedCodeChecks}
\lean{QEC1.DeformedCode.deformedCodeChecks}
\leanok
\uses{def:DeformedCode.allChecks, def:DeformedCode.CheckIndex, def:DeformedCode.DeformedCodeData, def:GaussFlux.ExtQubit}
The \emph{generating checks for the deformed code} on the extended qubit system $V \oplus E(G)$ is the main definition from Definition~4 of the paper. It is the function
\[
\operatorname{deformedCodeChecks} : \operatorname{CheckIndex}(V, C, J) \to \mathcal{P}_{V \oplus E(G)}
\]
defined as $\operatorname{allChecks}(G, \mathrm{cycles}, \mathrm{checks}, \mathrm{data})$, mapping:
\begin{align*}
\operatorname{gaussLaw}(v) &\mapsto A_v \quad \text{(Gauss's law operator for vertex } v\text{)}, \\
\operatorname{flux}(p) &\mapsto B_p \quad \text{(flux operator for cycle } p\text{)}, \\
\operatorname{deformed}(j) &\mapsto \tilde{s}_j \quad \text{(deformed original check for index } j\text{)}.
\end{align*}
\end{definition}

I'll start by reading the Lean file to understand its contents.This is the root import file for the QEC1 library. It simply imports all the other modules and doesn't contain any definitions, theorems, or lemmas of its own. Since there are no declarations to translate, I should not generate any LaTeX output for this file — it's just an organizational file that aggregates imports.

However, since the instructions say each file becomes its own chapter, and this file has no mathematical content, let me produce a minimal chapter that documents it as the root module:

\chapter{QEC1: Root Module}

This file serves as the root import module for the QEC1 library. It aggregates all definitions, remarks, and lemmas by importing the following modules:

\begin{itemize}
  \item \textbf{Rem 1}: Notation for Binary Vectors
  \item \textbf{Rem 2}: Notation for Pauli Operators
  \item \textbf{Rem 3}: Notation for Stabilizer Codes
  \item \textbf{Rem 4}: Notation for Cheeger Constant
  \item \textbf{Def 1}: Boundary and Coboundary Maps
  \item \textbf{Rem 5}: Exactness of Sequences
  \item \textbf{Def 2}: Gauss Law and Flux Operators
  \item \textbf{Def 3}: Deformed Operator
  \item \textbf{Rem 6}: Noncommuting Operators Cannot Be Deformed
  \item \textbf{Def 4}: Deformed Code
  \item \textbf{Lem 1}: Deformed Code Checks
\end{itemize}

No new definitions or theorems are introduced in this file.

%--- Lem_1: DeformedCodeChecks ---

\chapter{Lem 1: Deformed Code Checks}

The operators listed in Definition~4 (Deformed Code) form a valid generating set of commuting checks for a stabilizer code (or subsystem code). Specifically, all six pairwise commutation relations among the Gauss's law operators $A_v$, the flux operators $B_p$, and the deformed checks $\tilde{s}_j$ are established. The combined checks are shown to be self-inverse, and together they define a valid stabilizer code on the extended qubit system $V \oplus E(G)$.

\begin{theorem}[Gauss's Law Operators Commute]
\label{thm:DeformedCodeChecks.gaussLaw_gaussLaw_commute}
\lean{QEC1.DeformedCodeChecks.gaussLaw_gaussLaw_commute}
\leanok
\uses{def:DeformedCode.gaussLawChecks, def:PauliOp.PauliCommute, thm:DeformedCode.allChecks_pairwise_commute_gaussLaw_gaussLaw}
For any two vertices $v, w \in V$, the Gauss's law operators commute:
\[
[A_v, A_w] = 0,
\]
i.e., $\operatorname{PauliCommute}(\operatorname{gaussLawChecks}(G, v),\; \operatorname{gaussLawChecks}(G, w))$ holds. Both $A_v$ and $A_w$ are pure $X$-type operators, so they commute.
\end{theorem}

\begin{proof}
\leanok
\uses{thm:DeformedCode.allChecks_pairwise_commute_gaussLaw_gaussLaw}
This follows directly from \texttt{allChecks\_pairwise\_commute\_gaussLaw\_gaussLaw}.
\end{proof}

\begin{theorem}[Flux Operators Commute]
\label{thm:DeformedCodeChecks.flux_flux_commute}
\lean{QEC1.DeformedCodeChecks.flux_flux_commute}
\leanok
\uses{def:DeformedCode.fluxChecks, def:PauliOp.PauliCommute, thm:DeformedCode.allChecks_pairwise_commute_flux_flux}
For any two cycle indices $p, q \in C$, the flux operators commute:
\[
[B_p, B_q] = 0,
\]
i.e., $\operatorname{PauliCommute}(\operatorname{fluxChecks}(G, \mathrm{cycles}, p),\; \operatorname{fluxChecks}(G, \mathrm{cycles}, q))$ holds. Both $B_p$ and $B_q$ are pure $Z$-type operators, so they commute.
\end{theorem}

\begin{proof}
\leanok
\uses{thm:DeformedCode.allChecks_pairwise_commute_flux_flux}
This follows directly from \texttt{allChecks\_pairwise\_commute\_flux\_flux}.
\end{proof}

\begin{theorem}[Deformed Checks Commute]
\label{thm:DeformedCodeChecks.deformed_deformed_commute}
\lean{QEC1.DeformedCodeChecks.deformed_deformed_commute}
\leanok
\uses{def:DeformedCode.DeformedCodeData, def:DeformedCode.deformedOriginalChecks, def:PauliOp.PauliCommute, thm:DeformedCode.allChecks_pairwise_commute_deformed_deformed}
Let $\mathrm{data}$ be a \texttt{DeformedCodeData} for $G$ and checks, and suppose all original checks pairwise commute, i.e., $\operatorname{PauliCommute}(\mathrm{checks}(i), \mathrm{checks}(j))$ for all $i, j \in J$. Then for any $i, j \in J$, the deformed checks commute:
\[
[\tilde{s}_i, \tilde{s}_j] = 0,
\]
i.e., $\operatorname{PauliCommute}(\operatorname{deformedOriginalChecks}(G, \mathrm{checks}, \mathrm{data}, i),\; \operatorname{deformedOriginalChecks}(G, \mathrm{checks}, \mathrm{data}, j))$. On edge qubits both are pure $Z$-type; on vertex qubits commutation is inherited from the original checks.
\end{theorem}

\begin{proof}
\leanok
\uses{thm:DeformedCode.allChecks_pairwise_commute_deformed_deformed}
This follows directly from \texttt{allChecks\_pairwise\_commute\_deformed\_deformed}.
\end{proof}

\begin{theorem}[Gauss's Law and Flux Operators Commute]
\label{thm:DeformedCodeChecks.gaussLaw_flux_commute}
\lean{QEC1.DeformedCodeChecks.gaussLaw_flux_commute}
\leanok
\uses{def:DeformedCode.gaussLawChecks, def:DeformedCode.fluxChecks, def:PauliOp.PauliCommute, thm:DeformedCode.allChecks_pairwise_commute_gaussLaw_flux}
Assume for every cycle $c \in C$ and every vertex $v \in V$, the number of edges in the cycle $c$ incident to $v$ is even. Then for any $v \in V$ and $p \in C$, the Gauss's law operator and flux operator commute:
\[
[A_v, B_p] = 0.
\]
The symplectic inner product counts the overlap of the $X$-support of $A_v$ with the $Z$-support of $B_p$, which is the number of edges in $p$ incident to $v$. For a valid cycle, this is always even ($0$ or $2$).
\end{theorem}

\begin{proof}
\leanok
\uses{thm:DeformedCode.allChecks_pairwise_commute_gaussLaw_flux}
This follows directly from \texttt{allChecks\_pairwise\_commute\_gaussLaw\_flux}.
\end{proof}

\begin{theorem}[Gauss's Law and Deformed Checks Commute]
\label{thm:DeformedCodeChecks.gaussLaw_deformed_commute}
\lean{QEC1.DeformedCodeChecks.gaussLaw_deformed_commute}
\leanok
\uses{def:DeformedCode.DeformedCodeData, def:DeformedCode.gaussLawChecks, def:DeformedCode.deformedOriginalChecks, def:PauliOp.PauliCommute, thm:DeformedCode.allChecks_pairwise_commute_gaussLaw_deformed}
Let $\mathrm{data}$ be a \texttt{DeformedCodeData} for $G$ and checks. For any vertex $v \in V$ and index $j \in J$, the Gauss's law operator and deformed check commute:
\[
[A_v, \tilde{s}_j] = 0.
\]
The boundary condition ensures the anticommutation signs cancel.
\end{theorem}

\begin{proof}
\leanok
\uses{thm:DeformedCode.allChecks_pairwise_commute_gaussLaw_deformed}
This follows directly from \texttt{allChecks\_pairwise\_commute\_gaussLaw\_deformed}.
\end{proof}

\begin{theorem}[Flux and Deformed Checks Commute]
\label{thm:DeformedCodeChecks.flux_deformed_commute}
\lean{QEC1.DeformedCodeChecks.flux_deformed_commute}
\leanok
\uses{def:DeformedCode.DeformedCodeData, def:DeformedCode.fluxChecks, def:DeformedCode.deformedOriginalChecks, def:PauliOp.PauliCommute, thm:DeformedCode.allChecks_pairwise_commute_deformed_flux, thm:PauliOp.pauliCommute_comm}
Let $\mathrm{data}$ be a \texttt{DeformedCodeData} for $G$ and checks. For any cycle index $p \in C$ and check index $j \in J$, the flux operator and deformed check commute:
\[
[B_p, \tilde{s}_j] = 0.
\]
$B_p$ is pure $Z$-type and acts only on edges; $\tilde{s}_j$ has no $X$-support on edges. Since $Z$ commutes with $Z$ and $B_p$ does not act on vertex qubits, they commute.
\end{theorem}

\begin{proof}
\leanok
\uses{thm:PauliOp.pauliCommute_comm, thm:DeformedCode.allChecks_pairwise_commute_deformed_flux}
By commutativity of the Pauli commutation relation (rewriting via $\operatorname{pauliCommute\_comm}$), the goal reduces to showing $\operatorname{PauliCommute}(\operatorname{deformedOriginalChecks}(G, \mathrm{checks}, \mathrm{data}, j),\; \operatorname{fluxChecks}(G, \mathrm{cycles}, p))$. This follows directly from \texttt{allChecks\_pairwise\_commute\_deformed\_flux}.
\end{proof}

\begin{theorem}[All Deformed Code Checks Pairwise Commute]
\label{thm:DeformedCodeChecks.deformedCodeChecks_allCommute}
\lean{QEC1.DeformedCodeChecks.deformedCodeChecks_allCommute}
\leanok
\uses{def:DeformedCode.CheckIndex, def:DeformedCode.DeformedCodeData, def:DeformedCode.deformedCodeChecks, def:PauliOp.PauliCommute, thm:DeformedCode.allChecks_commute}
Let $\mathrm{data}$ be a \texttt{DeformedCodeData} for $G$ and checks. Suppose all cycles satisfy the evenness condition (for every $c \in C$ and $v \in V$, the number of edges in cycle $c$ incident to $v$ is even), and all original checks pairwise commute. Then for any two check indices $a, b \in \operatorname{CheckIndex}(V, C, J)$:
\[
\operatorname{PauliCommute}(\operatorname{deformedCodeChecks}(G, \mathrm{cycles}, \mathrm{checks}, \mathrm{data}, a),\; \operatorname{deformedCodeChecks}(G, \mathrm{cycles}, \mathrm{checks}, \mathrm{data}, b)).
\]
\end{theorem}

\begin{proof}
\leanok
\uses{thm:DeformedCode.allChecks_commute}
This follows directly from \texttt{allChecks\_commute}.
\end{proof}

\begin{theorem}[All Deformed Code Checks are Self-Inverse]
\label{thm:DeformedCodeChecks.deformedCodeChecks_allSelfInverse}
\lean{QEC1.DeformedCodeChecks.deformedCodeChecks_allSelfInverse}
\leanok
\uses{def:DeformedCode.CheckIndex, def:DeformedCode.DeformedCodeData, def:DeformedCode.deformedCodeChecks, thm:DeformedCode.allChecks_self_inverse}
Let $\mathrm{data}$ be a \texttt{DeformedCodeData} for $G$ and checks. For any check index $a \in \operatorname{CheckIndex}(V, C, J)$:
\[
\operatorname{deformedCodeChecks}(G, \mathrm{cycles}, \mathrm{checks}, \mathrm{data}, a) \cdot \operatorname{deformedCodeChecks}(G, \mathrm{cycles}, \mathrm{checks}, \mathrm{data}, a) = 1.
\]
\end{theorem}

\begin{proof}
\leanok
\uses{thm:DeformedCode.allChecks_self_inverse}
This follows directly from \texttt{allChecks\_self\_inverse}.
\end{proof}

\begin{definition}[Deformed Stabilizer Code]
\label{def:DeformedCodeChecks.deformedStabilizerCode}
\lean{QEC1.DeformedCodeChecks.deformedStabilizerCode}
\leanok
\uses{def:DeformedCode.CheckIndex, def:DeformedCode.DeformedCodeData, def:DeformedCode.deformedCodeChecks, def:PauliOp.PauliCommute, def:StabilizerCode, thm:DeformedCode.allChecks_commute}
The deformed code forms a valid stabilizer code on the extended qubit system $V \oplus E(G)$. The check index type is $\operatorname{CheckIndex}(V, C, J) = V \oplus C \oplus J$, and the check function is $\operatorname{deformedCodeChecks}$. The pairwise commutation of all checks is established by \texttt{allChecks\_commute}.

Formally, given a graph $G$, cycles, original checks, deformed code data $\mathrm{data}$, the cycle evenness hypothesis, and the hypothesis that original checks pairwise commute, we define:
\[
\operatorname{deformedStabilizerCode}(G, \mathrm{cycles}, \mathrm{checks}, \mathrm{data}, h_{\mathrm{cyc}}, h_{\mathrm{comm}}) : \operatorname{StabilizerCode}(\operatorname{ExtQubit}(G))
\]
with index type $I = \operatorname{CheckIndex}(V, C, J)$ and check map $\operatorname{deformedCodeChecks}(G, \mathrm{cycles}, \mathrm{checks}, \mathrm{data})$.
\end{definition}

\begin{theorem}[Number of Physical Qubits]
\label{thm:DeformedCodeChecks.deformedStabilizerCode_numQubits}
\lean{QEC1.DeformedCodeChecks.deformedStabilizerCode_numQubits}
\leanok
\uses{def:DeformedCodeChecks.deformedStabilizerCode, def:StabilizerCode}
The number of physical qubits in the deformed stabilizer code is $|V| + |E(G)|$:
\[
\operatorname{numQubits}(\operatorname{deformedStabilizerCode}(\ldots)) = |V| + |E(G)|.
\]
\end{theorem}

\begin{proof}
\leanok
\uses{def:DeformedCodeChecks.deformedStabilizerCode}
By simplification using the definition of $\operatorname{numQubits}$ and the fact that the cardinality of a sum type $V \oplus E(G)$ equals $|V| + |E(G)|$ (i.e., $\operatorname{Fintype.card\_sum}$), the result follows.
\end{proof}

\begin{theorem}[Number of Checks]
\label{thm:DeformedCodeChecks.deformedStabilizerCode_numChecks}
\lean{QEC1.DeformedCodeChecks.deformedStabilizerCode_numChecks}
\leanok
\uses{def:DeformedCode.CheckIndex, def:DeformedCodeChecks.deformedStabilizerCode, def:StabilizerCode}
The number of checks in the deformed stabilizer code is $|V| + |C| + |J|$:
\[
\operatorname{numChecks}(\operatorname{deformedStabilizerCode}(\ldots)) = |V| + |C| + |J|.
\]
\end{theorem}

\begin{proof}
\leanok
\uses{def:DeformedCode.CheckIndex, def:DeformedCodeChecks.deformedStabilizerCode}
By simplification using the definition of $\operatorname{numChecks}$, it suffices to show $|\operatorname{CheckIndex}(V, C, J)| = |V| + |C| + |J|$. We establish an equivalence $\operatorname{CheckIndex}(V, C, J) \simeq V \oplus C \oplus J$ by the bijection:
\begin{itemize}
  \item $\operatorname{gaussLaw}(v) \mapsto \operatorname{inl}(v)$,
  \item $\operatorname{flux}(p) \mapsto \operatorname{inr}(\operatorname{inl}(p))$,
  \item $\operatorname{deformed}(j) \mapsto \operatorname{inr}(\operatorname{inr}(j))$,
\end{itemize}
with left and right inverses verified by case analysis (each case holdsby reflexivity). Applying $\operatorname{Fintype.card\_congr}$ with this equivalence, then using $\operatorname{Fintype.card\_sum}$ twice and associativity of addition, we obtain the result.
\end{proof}

\begin{theorem}[Gauss's Law Checks in the Stabilizer Group]
\label{thm:DeformedCodeChecks.gaussLaw_mem_stabilizerGroup}
\lean{QEC1.DeformedCodeChecks.gaussLaw_mem_stabilizerGroup}
\leanok
\uses{def:DeformedCode.gaussLawChecks, def:DeformedCodeChecks.deformedStabilizerCode, def:StabilizerCode.stabilizerGroup}
For any vertex $v \in V$, the Gauss's law operator $A_v = \operatorname{gaussLawChecks}(G, v)$ is a member of the stabilizer group of the deformed stabilizer code:
\[
\operatorname{gaussLawChecks}(G, v) \in \operatorname{stabilizerGroup}(\operatorname{deformedStabilizerCode}(\ldots)).
\]
\end{theorem}

\begin{proof}
\leanok
\uses{def:DeformedCodeChecks.deformedStabilizerCode, lem:StabilizerCode.check_mem_stabilizerGroup}
We observe that $\operatorname{gaussLawChecks}(G, v) = \operatorname{check}(\operatorname{deformedStabilizerCode}(\ldots), \operatorname{gaussLaw}(v))$. This holds by reflexivity. Rewriting with this identity, the result follows from the fact that every check of a stabilizer code lies in its stabilizer group (\texttt{check\_mem\_stabilizerGroup}).
\end{proof}

\begin{theorem}[Flux Checks in the Stabilizer Group]
\label{thm:DeformedCodeChecks.flux_mem_stabilizerGroup}
\lean{QEC1.DeformedCodeChecks.flux_mem_stabilizerGroup}
\leanok
\uses{def:DeformedCode.fluxChecks, def:DeformedCodeChecks.deformedStabilizerCode, def:StabilizerCode.stabilizerGroup}
For any cycle index $p \in C$, the flux operator $B_p = \operatorname{fluxChecks}(G, \mathrm{cycles}, p)$ is a member of the stabilizer group of the deformed stabilizer code:
\[
\operatorname{fluxChecks}(G, \mathrm{cycles}, p) \in \operatorname{stabilizerGroup}(\operatorname{deformedStabilizerCode}(\ldots)).
\]
\end{theorem}

\begin{proof}
\leanok
\uses{def:DeformedCodeChecks.deformedStabilizerCode, lem:StabilizerCode.check_mem_stabilizerGroup}
We observe that $\operatorname{fluxChecks}(G, \mathrm{cycles}, p) = \operatorname{check}(\operatorname{deformedStabilizerCode}(\ldots), \operatorname{flux}(p))$. This holds by reflexivity. Rewriting with this identity, the result follows from \texttt{check\_mem\_stabilizerGroup}.
\end{proof}

\begin{theorem}[Deformed Checks in the Stabilizer Group]
\label{thm:DeformedCodeChecks.deformed_mem_stabilizerGroup}
\lean{QEC1.DeformedCodeChecks.deformed_mem_stabilizerGroup}
\leanok
\uses{def:DeformedCode.deformedOriginalChecks, def:DeformedCodeChecks.deformedStabilizerCode, def:StabilizerCode.stabilizerGroup}
For any check index $j \in J$, the deformed check $\tilde{s}_j = \operatorname{deformedOriginalChecks}(G, \mathrm{checks}, \mathrm{data}, j)$ is a member of the stabilizer group of the deformed stabilizer code:
\[
\operatorname{deformedOriginalChecks}(G, \mathrm{checks}, \mathrm{data}, j) \in \operatorname{stabilizerGroup}(\operatorname{deformedStabilizerCode}(\ldots)).
\]
\end{theorem}

\begin{proof}
\leanok
\uses{def:DeformedCodeChecks.deformedStabilizerCode, lem:StabilizerCode.check_mem_stabilizerGroup}
We observe that $\operatorname{deformedOriginalChecks}(G, \mathrm{checks}, \mathrm{data}, j) = \operatorname{check}(\operatorname{deformedStabilizerCode}(\ldots), \operatorname{deformed}(j))$. This holds by reflexivity. Rewriting with this identity, the result follows from \texttt{check\_mem\_stabilizerGroup}.
\end{proof}

\begin{theorem}[Gauss's Law Checks in the Centralizer]
\label{thm:DeformedCodeChecks.gaussLaw_inCentralizer}
\lean{QEC1.DeformedCodeChecks.gaussLaw_inCentralizer}
\leanok
\uses{def:DeformedCode.gaussLawChecks, def:DeformedCodeChecks.deformedStabilizerCode, def:StabilizerCode.inCentralizer}
For any vertex $v \in V$, the Gauss's law operator $A_v$ lies in the centralizer of the deformed stabilizer code:
\[
\operatorname{inCentralizer}(\operatorname{deformedStabilizerCode}(\ldots),\; \operatorname{gaussLawChecks}(G, v)).
\]
\end{theorem}

\begin{proof}
\leanok
\uses{def:DeformedCodeChecks.deformedStabilizerCode}
Let $a$ be an arbitrary check index. We observe that $\operatorname{gaussLawChecks}(G, v) = \operatorname{check}(\operatorname{deformedStabilizerCode}(\ldots), \operatorname{gaussLaw}(v))$ by reflexivity. Rewriting with this identity, the commutation follows from the \texttt{checks\_commute} property of the deformed stabilizer code applied to the indices $\operatorname{gaussLaw}(v)$ and $a$.
\end{proof}

\begin{theorem}[Flux Checks in the Centralizer]
\label{thm:DeformedCodeChecks.flux_inCentralizer}
\lean{QEC1.DeformedCodeChecks.flux_inCentralizer}
\leanok
\uses{def:DeformedCode.fluxChecks, def:DeformedCodeChecks.deformedStabilizerCode, def:StabilizerCode.inCentralizer}
For any cycle index $p \in C$, the flux operator $B_p$ lies in the centralizer of the deformed stabilizer code:
\[
\operatorname{inCentralizer}(\operatorname{deformedStabilizerCode}(\ldots),\; \operatorname{fluxChecks}(G, \mathrm{cycles}, p)).
\]
\end{theorem}

\begin{proof}
\leanok
\uses{def:DeformedCodeChecks.deformedStabilizerCode}
Let $a$ be an arbitrary check index. We observe that $\operatorname{fluxChecks}(G, \mathrm{cycles}, p) = \operatorname{check}(\operatorname{deformedStabilizerCode}(\ldots), \operatorname{flux}(p))$ by reflexivity. Rewriting with this identity, the commutation follows from the \texttt{checks\_commute} property of the deformed stabilizer code applied to the indices $\operatorname{flux}(p)$ and $a$.
\end{proof}

\begin{theorem}[Deformed Checks in the Centralizer]
\label{thm:DeformedCodeChecks.deformed_inCentralizer}
\lean{QEC1.DeformedCodeChecks.deformed_inCentralizer}
\leanok
\uses{def:DeformedCode.deformedOriginalChecks, def:DeformedCodeChecks.deformedStabilizerCode, def:StabilizerCode.inCentralizer}
For any check index $j \in J$, the deformed check $\tilde{s}_j$ lies in the centralizer of the deformed stabilizer code:
\[
\operatorname{inCentralizer}(\operatorname{deformedStabilizerCode}(\ldots),\; \operatorname{deformedOriginalChecks}(G, \mathrm{checks}, \mathrm{data}, j)).
\]
\end{theorem}

\begin{proof}
\leanok
\uses{def:DeformedCodeChecks.deformedStabilizerCode}
Let $a$ be an arbitrary check index. We observe that $\operatorname{deformedOriginalChecks}(G, \mathrm{checks}, \mathrm{data}, j) = \operatorname{check}(\operatorname{deformedStabilizerCode}(\ldots), \operatorname{deformed}(j))$ by reflexivity. Rewriting with this identity, the commutation follows from the \texttt{checks\_commute} property of the deformed stabilizer code applied to the indices $\operatorname{deformed}(j)$ and $a$.
\end{proof}

\begin{theorem}[Gauss's Law Checks are Pure $X$-type]
\label{thm:DeformedCodeChecks.deformedStabilizerCode_gaussLaw_pure_X}
\lean{QEC1.DeformedCodeChecks.deformedStabilizerCode_gaussLaw_pure_X}
\leanok
\uses{def:DeformedCodeChecks.deformedStabilizerCode, def:DeformedCode.gaussLawChecks, thm:DeformedCode.gaussLawChecks_pure_X}
For any vertex $v \in V$, the Gauss's law check in the deformed stabilizer code is pure $X$-type, i.e., its $Z$-vector is zero:
\[
\operatorname{zVec}\bigl(\operatorname{check}(\operatorname{deformedStabilizerCode}(\ldots), \operatorname{gaussLaw}(v))\bigr) = 0.
\]
\end{theorem}

\begin{proof}
\leanok
\uses{thm:DeformedCode.gaussLawChecks_pure_X}
This follows directly from \texttt{gaussLawChecks\_pure\_X}.
\end{proof}

\begin{theorem}[Flux Checks are Pure $Z$-type]
\label{thm:DeformedCodeChecks.deformedStabilizerCode_flux_pure_Z}
\lean{QEC1.DeformedCodeChecks.deformedStabilizerCode_flux_pure_Z}
\leanok
\uses{def:DeformedCodeChecks.deformedStabilizerCode, def:DeformedCode.fluxChecks, thm:DeformedCode.fluxChecks_pure_Z}
For any cycle index $p \in C$, the flux check in the deformed stabilizer code is pure $Z$-type, i.e., its $X$-vector is zero:
\[
\operatorname{xVec}\bigl(\operatorname{check}(\operatorname{deformedStabilizerCode}(\ldots), \operatorname{flux}(p))\bigr) = 0.
\]
\end{theorem}

\begin{proof}
\leanok
\uses{thm:DeformedCode.fluxChecks_pure_Z}
This follows directly from \texttt{fluxChecks\_pure\_Z}.
\end{proof}

\begin{theorem}[Deformed Checks Have No $X$-Support on Edges]
\label{thm:DeformedCodeChecks.deformedStabilizerCode_deformed_noXOnEdges}
\lean{QEC1.DeformedCodeChecks.deformedStabilizerCode_deformed_noXOnEdges}
\leanok
\uses{def:DeformedCode.DeformedCodeData, def:DeformedCodeChecks.deformedStabilizerCode, def:DeformedCode.deformedOriginalChecks}
For any check index $j \in J$ and any edge $e \in E(G)$, the deformed check in the deformed stabilizer code has no $X$-support on the edge qubit $e$:
\[
\operatorname{xVec}\bigl(\operatorname{check}(\operatorname{deformedStabilizerCode}(\ldots), \operatorname{deformed}(j))\bigr)(\operatorname{inr}(e)) = 0.
\]
\end{theorem}

\begin{proof}
\leanok
\uses{def:DeformedCodeChecks.deformedStabilizerCode, def:DeformedCode.deformedOriginalChecks}
This follows directly from \texttt{deformedOriginalChecks\_no\_xSupport\_on\_edges}.
\end{proof}

\begin{theorem}[Product of Gauss's Law Checks Equals Logical Operator]
\label{thm:DeformedCodeChecks.deformedStabilizerCode_gaussLaw_product}
\lean{QEC1.DeformedCodeChecks.deformedStabilizerCode_gaussLaw_product}
\leanok
\uses{def:DeformedCodeChecks.deformedStabilizerCode, def:DeformedCode.gaussLawChecks, thm:DeformedCode.gaussLawChecks_product_eq_logical}
The product of all Gauss's law checks in the deformed stabilizer code equals the logical operator $L = \prod_{v \in V} X_v$:
\[
\prod_{v \in V} \operatorname{check}\bigl(\operatorname{deformedStabilizerCode}(\ldots), \operatorname{gaussLaw}(v)\bigr) = \operatorname{logicalOp}(G).
\]
\end{theorem}

\begin{proof}
\leanok
\uses{thm:DeformedCode.gaussLawChecks_product_eq_logical}
By converting using \texttt{gaussLawChecks\_product\_eq\_logical}, the two sides are shown to be equal.
\end{proof}

%--- Rem_7: CodespaceDimensionAfterGauging ---
\chapter{Rem 7: Codespace Dimension After Gauging}

This remark analyzes how the codespace dimension changes when a stabilizer code is gauged. If the original code has parameters $[\![n, k, d]\!]$ with $J$ checks and the gauging graph $G = (V, E)$ has a cycle collection $C$ generating the full cycle space (so $|C| = |E| - |V| + 1$ for a connected graph), then the deformed code has $n + |E|$ physical qubits and $|V| + |C| + |J|$ checks, giving $k_{\text{new}} = (n + |E|) - (|V| + |C| + |J|) = k - 1$. The gauging measurement consumes exactly one logical qubit.

\begin{theorem}[Deformed Code Number of Qubits]
\label{thm:CodespaceDimension.deformed_numQubits_eq}
\lean{QEC1.CodespaceDimension.deformed_numQubits_eq}
\leanok
\uses{def:DeformedCodeChecks.deformedStabilizerCode, thm:DeformedCodeChecks.deformedStabilizerCode_numQubits}
Let $G = (V, E)$ be a simple graph, let $C$ be a collection of cycles satisfying the even-incidence condition, let $J$ be a set of pairwise-commuting Pauli checks, and let \texttt{data} be the deformed code data. Then the number of physical qubits in the deformed stabilizer code is
\[
\operatorname{numQubits}(\text{deformed code}) = |V| + |E|.
\]
\end{theorem}

\begin{proof}
\leanok
\uses{thm:DeformedCodeChecks.deformedStabilizerCode_numQubits}
This follows directly from \texttt{deformedStabilizerCode\_numQubits}.
\end{proof}

\begin{theorem}[Deformed Code Number of Checks]
\label{thm:CodespaceDimension.deformed_numChecks_eq}
\lean{QEC1.CodespaceDimension.deformed_numChecks_eq}
\leanok
\uses{def:DeformedCodeChecks.deformedStabilizerCode, thm:DeformedCodeChecks.deformedStabilizerCode_numChecks}
Under the same hypotheses as above, the number of checks in the deformed stabilizer code is
\[
\operatorname{numChecks}(\text{deformed code}) = |V| + |C| + |J|.
\]
\end{theorem}

\begin{proof}
\leanok
\uses{thm:DeformedCodeChecks.deformedStabilizerCode_numChecks}
This follows directly from \texttt{deformedStabilizerCode\_numChecks}.
\end{proof}

\begin{definition}[Cycle Rank Property]
\label{def:CodespaceDimension.CycleRankProperty}
\lean{QEC1.CodespaceDimension.CycleRankProperty}
\leanok
\uses{def:DeformedCodeChecks.deformedStabilizerCode}
The \emph{cycle rank property} for a graph $G = (V, E)$ with a cycle collection $C$ is the assertion that
\[
|C| + |V| = |E| + 1.
\]
This holds when $C$ is a complete cycle basis for a connected graph $G$, and is equivalent to the statement that the cycle rank (first Betti number) satisfies $|C| = |E| - |V| + 1$.
\end{definition}

\begin{theorem}[Codespace Dimension Change After Gauging]
\label{thm:CodespaceDimension.codespace_dimension_change_after_gauging}
\lean{QEC1.CodespaceDimension.codespace_dimension_change_after_gauging}
\leanok
\uses{def:DeformedCodeChecks.deformedStabilizerCode, def:CodespaceDimension.CycleRankProperty, thm:CodespaceDimension.deformed_numQubits_eq, thm:CodespaceDimension.deformed_numChecks_eq}
Let the original code act on $n$ qubits with $|J|$ checks, so that $k = n - |J|$ is the nominal number of logical qubits with $k \geq 1$. Suppose the gauging graph $G = (V, E)$ with cycle collection $C$ satisfies the cycle rank property $|C| + |V| = |E| + 1$. Then the deformed stabilizer code satisfies
\[
\operatorname{numQubits}(\text{deformed code}) - \operatorname{numChecks}(\text{deformed code}) = k - 1.
\]
That is, gauging consumes exactly one logical qubit.
\end{theorem}

\begin{proof}
\leanok
\uses{thm:CodespaceDimension.deformed_numQubits_eq, thm:CodespaceDimension.deformed_numChecks_eq, def:CodespaceDimension.CycleRankProperty}
We rewrite the number of qubits and checks of the deformed code using the identities $\operatorname{numQubits} = |V| + |E|$ and $\operatorname{numChecks} = |V| + |C| + |J|$. Unfolding the cycle rank property gives $|C| + |V| = |E| + 1$. We need to show $(|V| + |E|) - (|V| + |C| + |J|) = k - 1$, where $k = n - |J|$ and $n = |V|$. From the cycle rank property, $|E| = |C| + |V| - 1$, so
\[
(|V| + |E|) - (|V| + |C| + |J|) = (|V| + |C| + |V| - 1) - (|V| + |C| + |J|) = |V| - 1 - |J| = (|V| - |J|) - 1 = k - 1.
\]
This is verified by integer arithmetic (\texttt{omega}).
\end{proof}

\begin{theorem}[Additional Checks Exceed Additional Qubits by One]
\label{thm:CodespaceDimension.additional_checks_exceed_additional_qubits_by_one}
\lean{QEC1.CodespaceDimension.additional_checks_exceed_additional_qubits_by_one}
\leanok
\uses{def:CodespaceDimension.CycleRankProperty}
Let $n = |V|$ and assume the cycle rank property holds. Then the total number of checks satisfies
\[
|V| + |C| + |J| = (|V| + |E|) - n + |J| + 1.
\]
That is, the net check increase minus the net qubit increase equals $1$.
\end{theorem}

\begin{proof}
\leanok
\uses{def:CodespaceDimension.CycleRankProperty}
We unfold the cycle rank property $|C| + |V| = |E| + 1$ and verify the equality by integer arithmetic (\texttt{omega}).
\end{proof}

\begin{theorem}[Zero Logical Qubits After Gauging When $k = 0$]
\label{thm:CodespaceDimension.deformed_k_zero_when_original_k_zero}
\lean{QEC1.CodespaceDimension.deformed_k_zero_when_original_k_zero}
\leanok
\uses{def:DeformedCodeChecks.deformedStabilizerCode, def:CodespaceDimension.CycleRankProperty, thm:CodespaceDimension.deformed_numQubits_eq, thm:CodespaceDimension.deformed_numChecks_eq}
If the original code has $k = 0$ logical qubits (i.e., $|V| = |J|$) and the cycle rank property holds, then the deformed code satisfies
\[
\operatorname{numQubits}(\text{deformed code}) \leq \operatorname{numChecks}(\text{deformed code}).
\]
That is, the natural number subtraction underflows to $0$.
\end{theorem}

\begin{proof}
\leanok
\uses{thm:CodespaceDimension.deformed_numQubits_eq, thm:CodespaceDimension.deformed_numChecks_eq, def:CodespaceDimension.CycleRankProperty}
We rewrite using $\operatorname{numQubits} = |V| + |E|$ and $\operatorname{numChecks} = |V| + |C| + |J|$. Unfolding the cycle rank property $|C| + |V| = |E| + 1$ and using the hypothesis $|V| = |J|$, the inequality follows by integer arithmetic (\texttt{omega}).
\end{proof}

\begin{theorem}[Deformed Code Check Count Lower Bound]
\label{thm:CodespaceDimension.deformed_numChecks_ge}
\lean{QEC1.CodespaceDimension.deformed_numChecks_ge}
\leanok
\uses{def:DeformedCodeChecks.deformedStabilizerCode, def:CodespaceDimension.CycleRankProperty, thm:CodespaceDimension.deformed_numQubits_eq, thm:CodespaceDimension.deformed_numChecks_eq}
Under the cycle rank property, the number of checks in the deformed code satisfies
\[
\operatorname{numChecks}(\text{deformed code}) = \operatorname{numQubits}(\text{deformed code}) - |V| + |J| + 1.
\]
\end{theorem}

\begin{proof}
\leanok
\uses{thm:CodespaceDimension.deformed_numQubits_eq, thm:CodespaceDimension.deformed_numChecks_eq, def:CodespaceDimension.CycleRankProperty}
We rewrite using $\operatorname{numQubits} = |V| + |E|$ and $\operatorname{numChecks} = |V| + |C| + |J|$. Unfolding the cycle rank property and applying integer arithmetic (\texttt{omega}) yields the result.
\end{proof}

\begin{theorem}[Deformed Code Parameter $k - 1$]
\label{thm:CodespaceDimension.deformed_code_parameter_k_minus_one}
\lean{QEC1.CodespaceDimension.deformed_code_parameter_k_minus_one}
\leanok
\uses{def:DeformedCodeChecks.deformedStabilizerCode, def:CodespaceDimension.CycleRankProperty, thm:CodespaceDimension.codespace_dimension_change_after_gauging}
Let the original code have $n$ qubits with $|J|$ checks and $k = n - |J| \geq 1$ logical qubits. Under the cycle rank property, the deformed code $\mathrm{DC}$ satisfies
\[
\mathrm{DC}.\operatorname{numQubits} - \mathrm{DC}.\operatorname{numChecks} = k - 1.
\]
\end{theorem}

\begin{proof}
\leanok
\uses{thm:CodespaceDimension.codespace_dimension_change_after_gauging}
This follows directly by applying the theorem \texttt{codespace\_dimension\_change\_after\_gauging} with the given parameters.
\end{proof}

\begin{theorem}[Additional Gauging Checks Equal Qubits Plus One]
\label{thm:CodespaceDimension.additional_gauging_checks_eq_qubits_plus_one}
\lean{QEC1.CodespaceDimension.additional_gauging_checks_eq_qubits_plus_one}
\leanok
\uses{def:CodespaceDimension.CycleRankProperty}
Under the cycle rank property, the number of additional gauging checks (vertices plus cycles) exceeds the number of additional qubits (edges) by exactly one:
\[
|V| + |C| = |E| + 1.
\]
\end{theorem}

\begin{proof}
\leanok
\uses{def:CodespaceDimension.CycleRankProperty}
We unfold the cycle rank property $|C| + |V| = |E| + 1$ and the result follows by integer arithmetic (\texttt{omega}).
\end{proof}

\begin{theorem}[Edge Count Identity]
\label{thm:CodespaceDimension.edge_count_eq}
\lean{QEC1.CodespaceDimension.edge_count_eq}
\leanok
\uses{def:CodespaceDimension.CycleRankProperty}
Under the cycle rank property, the edge count satisfies
\[
|E| + 1 = |V| + |C|.
\]
\end{theorem}

\begin{proof}
\leanok
\uses{def:CodespaceDimension.CycleRankProperty}
We unfold the cycle rank property and the equality follows by integer arithmetic (\texttt{omega}).
\end{proof}

%--- Rem_8: FreedomInDeformedChecks ---
Now let me also check what the relevant definitions look like to understand the mathematical content better:Now I have a thorough understanding of the code. Let me produce the LaTeX translation:

\chapter{Rem 8: Freedom in Deformed Checks}

The deformed code is uniquely determined as a codespace, but there is freedom in choosing the generating set: the Gauss's law checks $A_v$ are fixed, the flux checks $B_p$ can use any generating set of cycles for $G$, and the deformed checks $\tilde{s}_j$ have freedom in choosing the paths $\gamma_j$. Different choices of $\gamma$ give equivalent codes: if $\gamma$ and $\gamma'$ both satisfy $\partial \gamma = \partial \gamma' = S_Z(s)|_V$, then $\tilde{s}(\gamma') = \tilde{s}(\gamma) \cdot B_c$ for some product of flux operators $B_c$. In particular, the difference $\gamma + \gamma'$ lies in $\ker(\partial)$ (the cycle space).

\section{Path Difference Lies in Cycle Space}

\begin{theorem}[Path Difference in Kernel of Boundary]
\label{thm:FreedomInDeformedChecks.diff_path_in_ker_boundary}
\lean{QEC1.FreedomInDeformedChecks.diff_path_in_ker_boundary}
\leanok
\uses{def:DeformedOperator.BoundaryCondition, def:GraphMaps.boundaryMap}
Let $G$ be a simple graph on vertices $V$, and let $s$ be a Pauli operator on $V$. If $\gamma, \gamma' : E(G) \to \mathbb{Z}/2\mathbb{Z}$ both satisfy the boundary condition $\partial \gamma = S_Z(s)|_V$ and $\partial \gamma' = S_Z(s)|_V$, then
\[
\partial(\gamma + \gamma') = 0.
\]
\end{theorem}

\begin{proof}
\leanok
\uses{def:DeformedOperator.BoundaryCondition, def:GraphMaps.boundaryMap}
Since the boundary map $\partial$ is linear, we have $\partial(\gamma + \gamma') = \partial \gamma + \partial \gamma'$. Unfolding the boundary condition, $\partial \gamma = S_Z(s)|_V$ and $\partial \gamma' = S_Z(s)|_V$. Thus $\partial(\gamma + \gamma') = S_Z(s)|_V + S_Z(s)|_V$. By extensionality, for each vertex $v$, this equals $S_Z(s)|_V(v) + S_Z(s)|_V(v) = 0$ since we work over $\mathbb{Z}/2\mathbb{Z}$ where $a + a = 0$.
\end{proof}

\begin{theorem}[Path Difference in Kernel (LinearMap.ker Version)]
\label{thm:FreedomInDeformedChecks.diff_path_mem_ker_boundary}
\lean{QEC1.FreedomInDeformedChecks.diff_path_mem_ker_boundary}
\leanok
\uses{def:DeformedOperator.BoundaryCondition, def:GraphMaps.boundaryMap}
Under the same hypotheses as above, $\gamma + \gamma' \in \ker(\partial_G)$.
\end{theorem}

\begin{proof}
\leanok
\uses{thm:FreedomInDeformedChecks.diff_path_in_ker_boundary}
By the characterization of the kernel as $\{x \mid \partial x = 0\}$, this follows directly from the fact that $\partial(\gamma + \gamma') = 0$.
\end{proof}

\section{Pure-Z Edge Operator}

\begin{definition}[Pure-Z Edge Operator]
\label{def:FreedomInDeformedChecks.pureZEdgeOp}
\lean{QEC1.FreedomInDeformedChecks.pureZEdgeOp}
\leanok
\uses{def:DeformedOperator.deformedOpExt}
Given an edge vector $\delta : E(G) \to \mathbb{Z}/2\mathbb{Z}$, the \emph{pure-Z edge operator} is defined as
\[
\operatorname{pureZEdgeOp}(\delta) := \widetilde{\operatorname{ext}}(\mathbf{1}, \delta),
\]
i.e., the deformed operator extension of the identity operator $\mathbf{1}$ with edge-path $\delta$. This operator acts as the identity on all vertex qubits and applies $Z_e$ on edge qubit $e$ whenever $\delta(e) = 1$.
\end{definition}

\begin{theorem}[Pure-Z Edge Operator Is Pure Z-Type]
\label{thm:FreedomInDeformedChecks.pureZEdgeOp_pure_Z}
\lean{QEC1.FreedomInDeformedChecks.pureZEdgeOp_pure_Z}
\leanok
\uses{def:FreedomInDeformedChecks.pureZEdgeOp, def:DeformedOperator.deformedOpExt}
For any edge vector $\delta$, the pure-Z edge operator has no X-support:
\[
\operatorname{xVec}(\operatorname{pureZEdgeOp}(\delta)) = 0.
\]
\end{theorem}

\begin{proof}
\leanok
\uses{def:FreedomInDeformedChecks.pureZEdgeOp, def:DeformedOperator.deformedOpExt}
By extensionality, we verify for each qubit $q$ in the extended system $V \oplus E(G)$. For vertex qubits $q = \operatorname{inl}(v)$, the X-component is $0$ by definition of $\operatorname{deformedOpExt}$. For edge qubits $q = \operatorname{inr}(e)$, the X-component is also $0$ by definition. Hence $\operatorname{xVec} = 0$.
\end{proof}

\begin{theorem}[Pure-Z Edge Operator Is Self-Inverse]
\label{thm:FreedomInDeformedChecks.pureZEdgeOp_mul_self}
\lean{QEC1.FreedomInDeformedChecks.pureZEdgeOp_mul_self}
\leanok
\uses{def:FreedomInDeformedChecks.pureZEdgeOp, thm:DeformedOperator.deformedOpExt_mul_self}
For any edge vector $\delta$,
\[
\operatorname{pureZEdgeOp}(\delta) \cdot \operatorname{pureZEdgeOp}(\delta) = \mathbf{1}.
\]
\end{theorem}

\begin{proof}
\leanok
\uses{thm:DeformedOperator.deformedOpExt_mul_self}
This follows directly from the fact that $\widetilde{\operatorname{ext}}(P, \gamma) \cdot \widetilde{\operatorname{ext}}(P, \gamma) = \mathbf{1}$ applied with $P = \mathbf{1}$ and $\gamma = \delta$.
\end{proof}

\section{Deformed Checks Differ by a Pure-Z Edge Operator}

\begin{theorem}[Deformed Check Difference Is Pure-Z on Edges]
\label{thm:FreedomInDeformedChecks.deformedCheck_diff_is_pure_Z_edge}
\lean{QEC1.FreedomInDeformedChecks.deformedCheck_diff_is_pure_Z_edge}
\leanok
\uses{def:DeformedCode.deformedCheck, def:FreedomInDeformedChecks.pureZEdgeOp, thm:DeformedOperator.deformedOpExt_mul}
For any Pauli operator $s$ on $V$ and edge-paths $\gamma, \gamma' : E(G) \to \mathbb{Z}/2\mathbb{Z}$,
\[
\tilde{s}(\gamma') = \tilde{s}(\gamma) \cdot \operatorname{pureZEdgeOp}(\gamma + \gamma').
\]
\end{theorem}

\begin{proof}
\leanok
\uses{def:DeformedCode.deformedCheck, def:FreedomInDeformedChecks.pureZEdgeOp, thm:DeformedOperator.deformedOpExt_mul}
Unfolding the definitions of $\tilde{s}$ and $\operatorname{pureZEdgeOp}$, we apply the multiplication law for deformed operator extensions: $\widetilde{\operatorname{ext}}(s, \gamma) \cdot \widetilde{\operatorname{ext}}(\mathbf{1}, \gamma + \gamma') = \widetilde{\operatorname{ext}}(s \cdot \mathbf{1}, \gamma + (\gamma + \gamma'))$. Since $s \cdot \mathbf{1} = s$ and $\gamma + (\gamma + \gamma') = \gamma'$ (the latter because $\gamma + \gamma = 0$ in $\mathbb{Z}/2\mathbb{Z}$, so $\gamma + (\gamma + \gamma') = (\gamma + \gamma) + \gamma' = 0 + \gamma' = \gamma'$), we obtain $\widetilde{\operatorname{ext}}(s, \gamma')$. The result follows by symmetry.
\end{proof}

\begin{theorem}[Deformed Check Difference -- Alternate Form]
\label{thm:FreedomInDeformedChecks.deformedCheck_diff_is_pure_Z_edge'}
\lean{QEC1.FreedomInDeformedChecks.deformedCheck_diff_is_pure_Z_edge'}
\leanok
\uses{def:DeformedCode.deformedCheck, def:FreedomInDeformedChecks.pureZEdgeOp, thm:FreedomInDeformedChecks.deformedCheck_diff_is_pure_Z_edge}
Equivalently, $\tilde{s}(\gamma') = \tilde{s}(\gamma) \cdot \operatorname{pureZEdgeOp}(\gamma' + \gamma)$.
\end{theorem}

\begin{proof}
\leanok
\uses{thm:FreedomInDeformedChecks.deformedCheck_diff_is_pure_Z_edge}
Since addition in $\mathbb{Z}/2\mathbb{Z}$ is commutative, $\gamma' + \gamma = \gamma + \gamma'$. The result then follows from the previous theorem.
\end{proof}

\section{The Difference Operator Commutes with All Checks}

\begin{theorem}[Pure-Z Edge Operator Commutes with Gauss's Law]
\label{thm:FreedomInDeformedChecks.pureZEdgeOp_comm_gaussLaw}
\lean{QEC1.FreedomInDeformedChecks.pureZEdgeOp_comm_gaussLaw}
\leanok
\uses{def:FreedomInDeformedChecks.pureZEdgeOp, def:GaussFlux.gaussLawOp, def:DeformedOperator.BoundaryCondition, thm:DeformedOperator.deformedOpExt_comm_gaussLaw}
If $\delta : E(G) \to \mathbb{Z}/2\mathbb{Z}$ satisfies $\partial \delta = 0$, then for every vertex $v \in V$,
\[
[\operatorname{pureZEdgeOp}(\delta),\; A_v] = 0,
\]
i.e., the pure-Z edge operator commutes with the Gauss's law operator at $v$.
\end{theorem}

\begin{proof}
\leanok
\uses{def:DeformedOperator.BoundaryCondition, def:DeformedOperator.zSupportOnVertices, thm:DeformedOperator.deformedOpExt_comm_gaussLaw}
We first show that $\delta$ satisfies the boundary condition for the identity operator: since $\partial \delta = 0$ and $S_Z(\mathbf{1})|_V = 0$ (the identity has no Z-support on vertices), we have $\partial \delta = S_Z(\mathbf{1})|_V$. The result then follows from the theorem that $\widetilde{\operatorname{ext}}(P, \gamma)$ commutes with $A_v$ whenever the boundary condition holds.
\end{proof}

\begin{theorem}[Pure-Z Edge Operator Commutes with Flux]
\label{thm:FreedomInDeformedChecks.pureZEdgeOp_comm_flux}
\lean{QEC1.FreedomInDeformedChecks.pureZEdgeOp_comm_flux}
\leanok
\uses{def:FreedomInDeformedChecks.pureZEdgeOp, def:GaussFlux.fluxOp, def:PauliOp.PauliCommute, def:PauliOp.symplecticInner}
For any edge vector $\delta$ and any cycle index $p$,
\[
[\operatorname{pureZEdgeOp}(\delta),\; B_p] = 0,
\]
i.e., the pure-Z edge operator commutes with the flux operator $B_p$.
\end{theorem}

\begin{proof}
\leanok
\uses{def:FreedomInDeformedChecks.pureZEdgeOp, def:GaussFlux.fluxOp, def:DeformedOperator.deformedOpExt}
Expanding the symplectic inner product and splitting the sum over vertices and edges: for the vertex sum, each term vanishes because both operators have zero X-component and zero Z-component on vertices (the pure-Z edge operator has identity on vertices, and the flux operator is pure Z on edges with zero X on vertices). For the edge sum, each term vanishes because both operators have X-component equal to zero on edges (the pure-Z edge operator has $\operatorname{xVec} = 0$ everywhere, and the flux operator is pure Z-type). Hence the symplectic inner product is zero.
\end{proof}

\begin{theorem}[Pure-Z Edge Operator Commutes with Deformed Checks]
\label{thm:FreedomInDeformedChecks.pureZEdgeOp_comm_deformed}
\lean{QEC1.FreedomInDeformedChecks.pureZEdgeOp_comm_deformed}
\leanok
\uses{def:FreedomInDeformedChecks.pureZEdgeOp, def:DeformedCode.deformedOriginalChecks, def:DeformedCode.deformedCheck, def:DeformedCode.DeformedCodeData, def:DeformedOperator.deformedOpExt, def:PauliOp.PauliCommute, def:PauliOp.symplecticInner}
For any edge vector $\delta$, any deformed code data, and any check index $j$,
\[
[\operatorname{pureZEdgeOp}(\delta),\; \tilde{s}_j] = 0,
\]
i.e., the pure-Z edge operator commutes with every deformed original check.
\end{theorem}

\begin{proof}
\leanok
\uses{def:FreedomInDeformedChecks.pureZEdgeOp, def:DeformedCode.deformedOriginalChecks, def:DeformedCode.deformedCheck, def:DeformedOperator.deformedOpExt}
Unfolding the definitions of $\operatorname{deformedOriginalChecks}$ and $\operatorname{deformedCheck}$, we expand the symplectic inner product and split the sum over vertices and edges. For the vertex sum: the pure-Z edge operator has $\operatorname{xVec} = 0$ and $\operatorname{zVec} = 0$ on vertices, so each vertex term vanishes. For the edge sum: the deformed operator extension has $\operatorname{xVec} = 0$ on edges, and the pure-Z edge operator also has $\operatorname{xVec} = 0$ on edges, so each edge term vanishes. Hence the total symplectic inner product is zero.
\end{proof}

\section{Different Paths Yield the Same Stabilizer Group}

\begin{theorem}[Alternative Deformed Check Decomposition]
\label{thm:FreedomInDeformedChecks.deformedOriginalCheck_alternative_in_group}
\lean{QEC1.FreedomInDeformedChecks.deformedOriginalCheck_alternative_in_group}
\leanok
\uses{def:DeformedCode.deformedOriginalChecks, def:DeformedCode.DeformedCodeData, def:FreedomInDeformedChecks.pureZEdgeOp, def:GraphMaps.boundaryMap, thm:FreedomInDeformedChecks.deformedCheck_diff_is_pure_Z_edge, thm:FreedomInDeformedChecks.pureZEdgeOp_pure_Z, thm:FreedomInDeformedChecks.diff_path_in_ker_boundary}
Let $\operatorname{data}$ and $\operatorname{data}'$ be two instances of $\operatorname{DeformedCodeData}$ (with potentially different edge-paths). For each check index $j$, there exists a Pauli operator $B$ on the extended system such that:
\begin{enumerate}
\item $\tilde{s}_j(\gamma'_j) = \tilde{s}_j(\gamma_j) \cdot B$,
\item $\operatorname{xVec}(B) = 0$ (i.e., $B$ is pure Z-type),
\item $\partial(\gamma_j + \gamma'_j) = 0$ (i.e., the path difference lies in the cycle space).
\end{enumerate}
\end{theorem}

\begin{proof}
\leanok
\uses{thm:FreedomInDeformedChecks.deformedCheck_diff_is_pure_Z_edge, thm:FreedomInDeformedChecks.pureZEdgeOp_pure_Z, thm:FreedomInDeformedChecks.diff_path_in_ker_boundary}
We take $B = \operatorname{pureZEdgeOp}(\gamma_j + \gamma'_j)$. Then:
\begin{enumerate}
\item The equality $\tilde{s}_j(\gamma'_j) = \tilde{s}_j(\gamma_j) \cdot B$ follows from the deformed check difference theorem.
\item The pure Z-type property $\operatorname{xVec}(B) = 0$ follows from the fact that pure-Z edge operators have no X-support.
\item The boundary condition $\partial(\gamma_j + \gamma'_j) = 0$ follows from the path difference kernel theorem, using the boundary conditions $\partial \gamma_j = S_Z(s_j)|_V$ and $\partial \gamma'_j = S_Z(s_j)|_V$ stored in both data and data$'$.
\end{enumerate}
\end{proof}

\begin{theorem}[Different Paths Produce the Same Code]
\label{thm:FreedomInDeformedChecks.different_paths_same_code}
\lean{QEC1.FreedomInDeformedChecks.different_paths_same_code}
\leanok
\uses{def:DeformedCode.deformedOriginalChecks, def:DeformedCode.DeformedCodeData, def:FreedomInDeformedChecks.pureZEdgeOp, thm:FreedomInDeformedChecks.deformedCheck_diff_is_pure_Z_edge}
For any two instances $\operatorname{data}$ and $\operatorname{data}'$ of $\operatorname{DeformedCodeData}$ and any check index $j$:
\[
\tilde{s}_j^{\operatorname{data}'} = \tilde{s}_j^{\operatorname{data}} \cdot \operatorname{pureZEdgeOp}(\gamma_j + \gamma'_j).
\]
\end{theorem}

\begin{proof}
\leanok
\uses{thm:FreedomInDeformedChecks.deformedCheck_diff_is_pure_Z_edge}
This follows directly from the deformed check difference theorem applied to the original check $s_j$ with edge-paths $\gamma_j$ (from data) and $\gamma'_j$ (from data$'$).
\end{proof}

\begin{theorem}[Path Differences Lie in the Cycle Space]
\label{thm:FreedomInDeformedChecks.different_paths_diff_in_cycle_space}
\lean{QEC1.FreedomInDeformedChecks.different_paths_diff_in_cycle_space}
\leanok
\uses{def:DeformedCode.DeformedCodeData, def:GraphMaps.boundaryMap, thm:FreedomInDeformedChecks.diff_path_mem_ker_boundary}
For any two instances $\operatorname{data}$ and $\operatorname{data}'$ of $\operatorname{DeformedCodeData}$ and any check index $j$:
\[
\gamma_j + \gamma'_j \in \ker(\partial_G).
\]
\end{theorem}

\begin{proof}
\leanok
\uses{thm:FreedomInDeformedChecks.diff_path_mem_ker_boundary}
This follows from the path difference kernel membership theorem, using the boundary conditions stored in data and data$'$.
\end{proof}

\section{Gauss's Law and Flux Checks Are Fixed}

\begin{theorem}[Gauss's Law Checks Are Independent of Path Choice]
\label{thm:FreedomInDeformedChecks.gaussLaw_checks_fixed}
\lean{QEC1.FreedomInDeformedChecks.gaussLaw_checks_fixed}
\leanok
\uses{def:DeformedCode.deformedCodeChecks, def:DeformedCode.DeformedCodeData, def:DeformedCode.CheckIndex}
For any two instances $\operatorname{data}$ and $\operatorname{data}'$ of $\operatorname{DeformedCodeData}$ and any vertex $v \in V$:
\[
\operatorname{deformedCodeChecks}(\operatorname{data}, \operatorname{gaussLaw}(v)) = \operatorname{deformedCodeChecks}(\operatorname{data}', \operatorname{gaussLaw}(v)).
\]
\end{theorem}

\begin{proof}
\leanok
\uses{def:DeformedCode.deformedCodeChecks, def:DeformedCode.CheckIndex}
This holds by reflexivity: the Gauss's law checks $A_v$ do not depend on the edge-path data at all, so both sides are definitionally equal.
\end{proof}

\begin{theorem}[Flux Checks Are Independent of Path Choice]
\label{thm:FreedomInDeformedChecks.flux_checks_fixed}
\lean{QEC1.FreedomInDeformedChecks.flux_checks_fixed}
\leanok
\uses{def:DeformedCode.deformedCodeChecks, def:DeformedCode.DeformedCodeData, def:DeformedCode.CheckIndex}
For any two instances $\operatorname{data}$ and $\operatorname{data}'$ of $\operatorname{DeformedCodeData}$ and any cycle index $p$:
\[
\operatorname{deformedCodeChecks}(\operatorname{data}, \operatorname{flux}(p)) = \operatorname{deformedCodeChecks}(\operatorname{data}', \operatorname{flux}(p)).
\]
\end{theorem}

\begin{proof}
\leanok
\uses{def:DeformedCode.deformedCodeChecks, def:DeformedCode.CheckIndex}
This holds by reflexivity: the flux checks $B_p$ do not depend on the edge-path data at all, so both sides are definitionally equal.
\end{proof}

\section{Pure-Z Edge Operator Commutes with All Checks}

\begin{theorem}[Pure-Z Edge Operator Commutes with All Deformed Code Checks]
\label{thm:FreedomInDeformedChecks.pureZEdgeOp_comm_allChecks}
\lean{QEC1.FreedomInDeformedChecks.pureZEdgeOp_comm_allChecks}
\leanok
\uses{def:FreedomInDeformedChecks.pureZEdgeOp, def:DeformedCode.allChecks, def:DeformedCode.CheckIndex, def:DeformedCode.DeformedCodeData, thm:FreedomInDeformedChecks.pureZEdgeOp_comm_gaussLaw, thm:FreedomInDeformedChecks.pureZEdgeOp_comm_flux, thm:FreedomInDeformedChecks.pureZEdgeOp_comm_deformed}
Let $\delta : E(G) \to \mathbb{Z}/2\mathbb{Z}$ satisfy $\partial \delta = 0$, and let $\operatorname{data}$ be any deformed code data. Assume that for every cycle $c$ and vertex $v$, the number of edges in $c$ incident to $v$ is even. Then for every check index $a \in V \sqcup C \sqcup J$:
\[
[\operatorname{pureZEdgeOp}(\delta),\; \operatorname{allChecks}(a)] = 0.
\]
\end{theorem}

\begin{proof}
\leanok
\uses{thm:FreedomInDeformedChecks.pureZEdgeOp_comm_gaussLaw, thm:FreedomInDeformedChecks.pureZEdgeOp_comm_flux, thm:FreedomInDeformedChecks.pureZEdgeOp_comm_deformed}
We proceed by cases on the check index $a$:
\begin{itemize}
\item \textbf{Case} $a = \operatorname{gaussLaw}(v)$: The result follows from the theorem that the pure-Z edge operator commutes with Gauss's law operators when $\partial \delta = 0$.
\item \textbf{Case} $a = \operatorname{flux}(p)$: The result follows from the theorem that the pure-Z edge operator commutes with flux operators.
\item \textbf{Case} $a = \operatorname{deformed}(j)$: The result follows from the theorem that the pure-Z edge operator commutes with deformed original checks.
\end{itemize}
\end{proof}

\section{Corollaries}

\begin{theorem}[Deformed Checks from Different Paths Always Commute]
\label{thm:FreedomInDeformedChecks.deformedChecks_different_paths_commute}
\lean{QEC1.FreedomInDeformedChecks.deformedChecks_different_paths_commute}
\leanok
\uses{def:DeformedCode.deformedCheck, def:DeformedOperator.deformedOpExt, def:DeformedOperator.BoundaryCondition, def:PauliOp.PauliCommute, def:PauliOp.symplecticInner}
Let $s$ be a Pauli operator on $V$, and let $\gamma, \gamma'$ be edge-paths both satisfying the boundary condition for $s$. Then
\[
[\tilde{s}(\gamma),\; \tilde{s}(\gamma')] = 0,
\]
i.e., the two deformed checks commute.
\end{theorem}

\begin{proof}
\leanok
\uses{def:DeformedCode.deformedCheck, def:DeformedOperator.deformedOpExt, def:PauliOp.symplecticInner}
Unfolding the definition of deformed check, we compute the symplectic inner product by splitting the sum over vertices and edges. For the vertex sum: both deformed checks have the same X-vector and Z-vector on vertices (both coming from $s$), so the contribution is $\sum_{v \in V} (s^X_v \cdot s^Z_v + s^Z_v \cdot s^X_v)$. Since multiplication in $\mathbb{Z}/2\mathbb{Z}$ is commutative, each term is $s^X_v \cdot s^Z_v + s^X_v \cdot s^Z_v = 0$ (as $a + a = 0$ in characteristic 2). For the edge sum: both deformed operator extensions have X-component equal to $0$ on edges, so each term $0 \cdot z + z' \cdot 0 = 0$. Hence the total symplectic inner product is zero.
\end{proof}

\begin{theorem}[Recovering a Deformed Check from Another]
\label{thm:FreedomInDeformedChecks.deformedCheck_recover}
\lean{QEC1.FreedomInDeformedChecks.deformedCheck_recover}
\leanok
\uses{def:DeformedCode.deformedCheck, def:FreedomInDeformedChecks.pureZEdgeOp, thm:FreedomInDeformedChecks.deformedCheck_diff_is_pure_Z_edge}
For any Pauli operator $s$ and edge-paths $\gamma, \gamma'$:
\[
\tilde{s}(\gamma) = \tilde{s}(\gamma') \cdot \operatorname{pureZEdgeOp}(\gamma' + \gamma).
\]
\end{theorem}

\begin{proof}
\leanok
\uses{thm:FreedomInDeformedChecks.deformedCheck_diff_is_pure_Z_edge}
This is the deformed check difference theorem applied with the roles of $\gamma$ and $\gamma'$ swapped: $\tilde{s}(\gamma) = \tilde{s}(\gamma') \cdot \operatorname{pureZEdgeOp}(\gamma' + \gamma)$.
\end{proof}

%--- Def_5: GaugingMeasurementAlgorithm ---
\chapter{Def 5: Gauging Measurement Algorithm}

The gauging measurement procedure measures a logical operator $L = \prod_{v \in V} X_v$ by introducing auxiliary edge qubits, measuring Gauss's law operators $A_v$ and edge qubits $Z_e$, and applying byproduct corrections. We formalize the classical data and processing of the algorithm: measurement outcomes $(\varepsilon_v, \omega_e)$, the sign $\sigma = \sum_v \varepsilon_v$ (in $\mathbb{Z}/2\mathbb{Z}$), the byproduct correction rule based on walk parities, and the commutativity of all measured operators.

We use $\mathbb{Z}/2\mathbb{Z}$ for measurement outcomes: $0 \leftrightarrow +1$, $1 \leftrightarrow -1$. Products of $\pm 1$ values correspond to sums in $\mathbb{Z}/2\mathbb{Z}$.

\begin{definition}[Gauging Input]
\label{def:GaugingMeasurement.GaugingInput}
\lean{QEC1.GaugingMeasurement.GaugingInput}
\leanok
\uses{def:GaussFlux.ExtQubit}
The input data for the gauging measurement procedure consists of:
\begin{itemize}
  \item A connected graph $G = (V, E)$ with $V$ the support of the logical operator $L = \prod_{v \in V} X_v$.
  \item A distinguished base vertex $v_0 \in V$ for byproduct correction.
  \item A proof that $G$ is connected.
\end{itemize}
\end{definition}

\begin{definition}[Gauging Outcomes]
\label{def:GaugingMeasurement.GaugingOutcomes}
\lean{QEC1.GaugingMeasurement.GaugingOutcomes}
\leanok
\uses{def:GaussFlux.ExtQubit}
The measurement outcomes from the gauging procedure. We use $\mathbb{Z}/2\mathbb{Z}$ to represent $\pm 1$ outcomes ($0 \leftrightarrow +1$, $1 \leftrightarrow -1$):
\begin{itemize}
  \item $\varepsilon_v \in \mathbb{Z}/2\mathbb{Z}$: the outcome of measuring Gauss's law operator $A_v$ for each $v \in V$.
  \item $\omega_e \in \mathbb{Z}/2\mathbb{Z}$: the outcome of measuring $Z_e$ for each $e \in E$.
\end{itemize}
\end{definition}

\begin{definition}[Measurement Sign]
\label{def:GaugingMeasurement.measurementSign}
\lean{QEC1.GaugingMeasurement.measurementSign}
\leanok
\uses{def:GaugingMeasurement.GaugingOutcomes}
The measurement sign $\sigma = \sum_{v \in V} \varepsilon_v \in \mathbb{Z}/2\mathbb{Z}$, encoding the product of all Gauss's law outcomes. This is the measurement result of the logical operator $L = \prod_{v \in V} X_v$: $\sigma = 0$ means the $+1$ eigenvalue, $\sigma = 1$ means $-1$.
\end{definition}

\begin{definition}[Walk Parity]
\label{def:GaugingMeasurement.walkParity}
\lean{QEC1.GaugingMeasurement.walkParity}
\leanok
\uses{def:GaugingMeasurement.GaugingOutcomes}
Given edge measurement outcomes $\omega$ and a walk $w$ in $G$ from $u$ to $v$, the walk parity is the $\mathbb{Z}/2\mathbb{Z}$-sum of $\omega(e)$ over all edges $e$ traversed by the walk:
\[
  \operatorname{walkParity}(\omega, w) = \sum_{d \in w.\mathrm{darts}} \omega(d.\mathrm{edge}).
\]
\end{definition}

\begin{lemma}[Walk Parity of Nil]
\label{lem:GaugingMeasurement.walkParity_nil}
\lean{QEC1.GaugingMeasurement.walkParity_nil}
\leanok
\uses{def:GaugingMeasurement.walkParity}
The walk parity of the trivial (nil) walk at any vertex $v$ is $0$.
\end{lemma}
\begin{proof}
\leanok
\uses{def:GaugingMeasurement.walkParity}
By simplification using the definition of walk parity and the fact that the nil walk has no darts, the sum is empty and equals $0$.
\end{proof}

\begin{lemma}[Walk Parity of Cons]
\label{lem:GaugingMeasurement.walkParity_cons}
\lean{QEC1.GaugingMeasurement.walkParity_cons}
\leanok
\uses{def:GaugingMeasurement.walkParity}
For an adjacency $h : G.\mathrm{Adj}(u, v)$ and a walk $p$ from $v$ to $w$, the walk parity of the extended walk $\operatorname{cons}(h, p)$ satisfies:
\[
  \operatorname{walkParity}(\omega, \operatorname{cons}(h, p)) = \omega(\{u, v\}) + \operatorname{walkParity}(\omega, p).
\]
\end{lemma}
\begin{proof}
\leanok
\uses{def:GaugingMeasurement.walkParity}
By simplification using the definition of walk parity and the fact that $\operatorname{darts}(\operatorname{cons}(h, p))$ prepends the dart for edge $\{u,v\}$ to $\operatorname{darts}(p)$, the map and sum decompose as the first term plus the remaining sum.
\end{proof}

\begin{theorem}[Walk Parity is Additive Under Concatenation]
\label{thm:GaugingMeasurement.walkParity_append}
\lean{QEC1.GaugingMeasurement.walkParity_append}
\leanok
\uses{def:GaugingMeasurement.walkParity}
For walks $p : u \to v$ and $q : v \to w$, the walk parity is additive:
\[
  \operatorname{walkParity}(\omega, p \cdot q) = \operatorname{walkParity}(\omega, p) + \operatorname{walkParity}(\omega, q).
\]
\end{theorem}
\begin{proof}
\leanok
\uses{lem:GaugingMeasurement.walkParity_nil, lem:GaugingMeasurement.walkParity_cons}
We proceed by induction on $p$.

\textbf{Base case} ($p = \mathrm{nil}$): The concatenation $\mathrm{nil} \cdot q = q$, and $\operatorname{walkParity}(\omega, \mathrm{nil}) = 0$, so the result follows by simplification.

\textbf{Inductive step} ($p = \operatorname{cons}(h, p')$): We have $\operatorname{cons}(h, p') \cdot q = \operatorname{cons}(h, p' \cdot q)$. Rewriting using the cons formula twice and the inductive hypothesis, we get
\[
  \omega(\{u,v\}) + (\operatorname{walkParity}(\omega, p') + \operatorname{walkParity}(\omega, q)) = (\omega(\{u,v\}) + \operatorname{walkParity}(\omega, p')) + \operatorname{walkParity}(\omega, q),
\]
which follows by ring arithmetic.
\end{proof}

\begin{theorem}[Walk Parity is Invariant Under Reversal]
\label{thm:GaugingMeasurement.walkParity_reverse}
\lean{QEC1.GaugingMeasurement.walkParity_reverse}
\leanok
\uses{def:GaugingMeasurement.walkParity, lem:GaugingMeasurement.walkParity_nil, lem:GaugingMeasurement.walkParity_cons, thm:GaugingMeasurement.walkParity_append}
For any walk $p : u \to v$, reversing the walk does not change its parity:
\[
  \operatorname{walkParity}(\omega, p^{\mathrm{rev}}) = \operatorname{walkParity}(\omega, p).
\]
\end{theorem}
\begin{proof}
\leanok
\uses{lem:GaugingMeasurement.walkParity_nil, lem:GaugingMeasurement.walkParity_cons, thm:GaugingMeasurement.walkParity_append}
We proceed by induction on $p$.

\textbf{Base case} ($p = \mathrm{nil}$): Both sides are $0$ by simplification.

\textbf{Inductive step} ($p = \operatorname{cons}(h, p')$ where $h : G.\mathrm{Adj}(a, b)$ and $p' : b \to c$): The reverse is $\operatorname{reverse}(\operatorname{cons}(h, p')) = \operatorname{reverse}(p') \cdot \operatorname{cons}(\bar{h}, \mathrm{nil})$. Rewriting using the append formula, the cons formula, the nil formula, and the inductive hypothesis, we obtain
\[
  \operatorname{walkParity}(\omega, p'^{\mathrm{rev}}) + \omega(\{b,a\}) + 0 = \omega(\{a,b\}) + \operatorname{walkParity}(\omega, p').
\]
Since $\{b,a\} = \{a,b\}$ as $\operatorname{Sym2}$ elements (by $\operatorname{Sym2.eq\_swap}$), the terms match and the result follows by ring arithmetic.
\end{proof}

\begin{lemma}[Walk Parity of Single Edge]
\label{lem:GaugingMeasurement.walkParity_single}
\lean{QEC1.GaugingMeasurement.walkParity_single}
\leanok
\uses{def:GaugingMeasurement.walkParity, lem:GaugingMeasurement.walkParity_cons, lem:GaugingMeasurement.walkParity_nil}
For a single-edge walk along adjacency $h : G.\mathrm{Adj}(u, v)$,
\[
  \operatorname{walkParity}(\omega, \operatorname{cons}(h, \mathrm{nil})) = \omega(\{u, v\}).
\]
\end{lemma}
\begin{proof}
\leanok
\uses{lem:GaugingMeasurement.walkParity_cons, lem:GaugingMeasurement.walkParity_nil}
Rewriting using the cons formula and the nil formula, the result reduces to $\omega(\{u,v\}) + 0 = \omega(\{u,v\})$, which holds by $\operatorname{add\_zero}$.
\end{proof}

\begin{theorem}[Walk Parity Difference Equals Closed Walk Parity]
\label{thm:GaugingMeasurement.walkParity_diff_eq_closed}
\lean{QEC1.GaugingMeasurement.walkParity_diff_eq_closed}
\leanok
\uses{def:GaugingMeasurement.walkParity, thm:GaugingMeasurement.walkParity_append, thm:GaugingMeasurement.walkParity_reverse}
For two walks $p, q : u \to v$, the sum of their parities equals the parity of the closed walk $p \cdot q^{\mathrm{rev}}$:
\[
  \operatorname{walkParity}(\omega, p) + \operatorname{walkParity}(\omega, q) = \operatorname{walkParity}(\omega, p \cdot q^{\mathrm{rev}}).
\]
\end{theorem}
\begin{proof}
\leanok
\uses{thm:GaugingMeasurement.walkParity_append, thm:GaugingMeasurement.walkParity_reverse}
Rewriting the right-hand side using the append additivity formula and the reversal invariance, this is immediate.
\end{proof}

\begin{theorem}[Walk Parity is Well-Defined]
\label{thm:GaugingMeasurement.walkParity_well_defined}
\lean{QEC1.GaugingMeasurement.walkParity_well_defined}
\leanok
\uses{def:GaugingMeasurement.walkParity, thm:GaugingMeasurement.walkParity_append, thm:GaugingMeasurement.walkParity_reverse}
If all closed walks have zero parity, i.e., $\operatorname{walkParity}(\omega, c) = 0$ for every closed walk $c : v \to v$, then for any two walks $p, q : u \to v$,
\[
  \operatorname{walkParity}(\omega, p) = \operatorname{walkParity}(\omega, q).
\]
\end{theorem}
\begin{proof}
\leanok
\uses{thm:GaugingMeasurement.walkParity_append, thm:GaugingMeasurement.walkParity_reverse}
Let $h$ be the hypothesis applied to the closed walk $p \cdot q^{\mathrm{rev}}$ at vertex $u$. Rewriting $h$ using the append additivity and reversal invariance, we obtain $\operatorname{walkParity}(\omega, p) + \operatorname{walkParity}(\omega, q) = 0$. Since we work in $\mathbb{Z}/2\mathbb{Z}$, the characteristic-two property ($a + b = 0 \Rightarrow a = b$) gives the result.
\end{proof}

\begin{definition}[Byproduct Correction at a Vertex]
\label{def:GaugingMeasurement.byproductCorrectionAt}
\lean{QEC1.GaugingMeasurement.byproductCorrectionAt}
\leanok
\uses{def:GaugingMeasurement.walkParity}
The byproduct correction at vertex $v$, given a specific walk $\gamma$ from $v_0$ to $v$, is defined as the walk parity:
\[
  c(v) = \operatorname{walkParity}(\omega, \gamma) \in \mathbb{Z}/2\mathbb{Z}.
\]
Here $c(v) = 1$ means ``apply $X_v$'' (corresponding to $\prod_{e \in \gamma} \omega_e = -1$ in $\pm 1$ notation).
\end{definition}

\begin{lemma}[Byproduct Correction at Base Vertex]
\label{lem:GaugingMeasurement.byproductCorrectionAt_self}
\lean{QEC1.GaugingMeasurement.byproductCorrectionAt_self}
\leanok
\uses{def:GaugingMeasurement.byproductCorrectionAt}
The byproduct correction at the base vertex $v_0$ using the trivial walk is $0$:
\[
  c(v_0) = 0.
\]
\end{lemma}
\begin{proof}
\leanok
\uses{def:GaugingMeasurement.byproductCorrectionAt}
By simplification using the definition of byproduct correction, this reduces to the walk parity of the nil walk, which is $0$.
\end{proof}

\begin{theorem}[Byproduct Correction is Well-Defined]
\label{thm:GaugingMeasurement.byproductCorrectionAt_well_defined}
\lean{QEC1.GaugingMeasurement.byproductCorrectionAt_well_defined}
\leanok
\uses{def:GaugingMeasurement.byproductCorrectionAt, thm:GaugingMeasurement.walkParity_well_defined}
If all closed walks have zero parity, then the byproduct correction at vertex $v$ is independent of the choice of walk from $v_0$ to $v$: for any two walks $\gamma_1, \gamma_2 : v_0 \to v$,
\[
  c_{\gamma_1}(v) = c_{\gamma_2}(v).
\]
\end{theorem}
\begin{proof}
\leanok
\uses{thm:GaugingMeasurement.walkParity_well_defined}
This follows directly from the well-definedness of walk parity (Theorem~\ref{thm:GaugingMeasurement.walkParity_well_defined}) applied to $\gamma_1$ and $\gamma_2$.
\end{proof}

\begin{definition}[Byproduct Correction Operator]
\label{def:GaugingMeasurement.byproductCorrectionOp}
\lean{QEC1.GaugingMeasurement.byproductCorrectionOp}
\leanok
\uses{def:PauliOp, def:GaugingMeasurement.byproductCorrectionAt}
The byproduct correction Pauli operator for a given walk assignment: for each vertex $v$, apply $X_v$ if and only if $c(v) = 1$. Formally, this is the Pauli operator with
\[
  \mathrm{xVec}(v) = c(v), \qquad \mathrm{zVec} = 0.
\]
\end{definition}

\begin{lemma}[Byproduct Correction Operator is Pure X-Type]
\label{lem:GaugingMeasurement.byproductCorrectionOp_pure_X}
\lean{QEC1.GaugingMeasurement.byproductCorrectionOp_pure_X}
\leanok
\uses{def:GaugingMeasurement.byproductCorrectionOp}
The byproduct correction operator has no $Z$-component:
\[
  (\operatorname{byproductCorrectionOp}).\mathrm{zVec} = 0.
\]
\end{lemma}
\begin{proof}
\leanok
\uses{def:GaugingMeasurement.byproductCorrectionOp}
This holds by definition (reflexivity), since the $\mathrm{zVec}$ field is defined to be $0$.
\end{proof}

\begin{theorem}[Byproduct Correction Operator is Self-Inverse]
\label{thm:GaugingMeasurement.byproductCorrectionOp_mul_self}
\lean{QEC1.GaugingMeasurement.byproductCorrectionOp_mul_self}
\leanok
\uses{def:GaugingMeasurement.byproductCorrectionOp, def:PauliOp}
The byproduct correction operator is self-inverse:
\[
  \operatorname{byproductCorrectionOp} \cdot \operatorname{byproductCorrectionOp} = \mathbf{1}.
\]
\end{theorem}
\begin{proof}
\leanok
\uses{def:GaugingMeasurement.byproductCorrectionOp}
By extensionality, it suffices to show equality for both the $x$-vector and $z$-vector components for each vertex $v$.
\begin{itemize}
  \item For the $x$-vector: by simplification using the multiplication formula and the identity $x$-vector, this reduces to $c(v) + c(v) = 0$, which holds by the characteristic-two property $a + a = 0$ in $\mathbb{Z}/2\mathbb{Z}$.
  \item For the $z$-vector: by simplification using the multiplication formula and the fact that $\mathrm{zVec} = 0$ by definition, this reduces to $0 + 0 = 0$.
\end{itemize}
\end{proof}

\begin{definition}[Gauging Result]
\label{def:GaugingMeasurement.GaugingResult}
\lean{QEC1.GaugingMeasurement.GaugingResult}
\leanok
\uses{def:GaugingMeasurement.measurementSign, def:GaugingMeasurement.byproductCorrectionAt}
The classical output of the gauging measurement algorithm, consisting of:
\begin{itemize}
  \item The measurement sign $\sigma \in \mathbb{Z}/2\mathbb{Z}$: $0 \leftrightarrow +1$ eigenvalue of $L$, $1 \leftrightarrow -1$ eigenvalue.
  \item The byproduct correction vector $c : V \to \mathbb{Z}/2\mathbb{Z}$, where $c(v) = 1$ means apply $X_v$.
\end{itemize}
\end{definition}

\begin{definition}[Gauging Algorithm]
\label{def:GaugingMeasurement.gaugingAlgorithm}
\lean{QEC1.GaugingMeasurement.gaugingAlgorithm}
\leanok
\uses{def:GaugingMeasurement.GaugingInput, def:GaugingMeasurement.GaugingOutcomes, def:GaugingMeasurement.GaugingResult, def:GaugingMeasurement.measurementSign, def:GaugingMeasurement.byproductCorrectionAt}
The gauging measurement algorithm: given input data, measurement outcomes, and a choice of walks from $v_0$ to each vertex, computes the classical output:
\begin{itemize}
  \item $\operatorname{sign} = \sum_{v \in V} \varepsilon_v$ (the sum of all Gauss outcomes in $\mathbb{Z}/2\mathbb{Z}$).
  \item $\operatorname{correction}(v) = \operatorname{walkParity}(\gamma_v, \omega)$ (the parity of edge outcomes along the walk from $v_0$ to $v$).
\end{itemize}
\end{definition}

\begin{theorem}[Gauging Algorithm Correction is Independent of Walk Choice]
\label{thm:GaugingMeasurement.gaugingAlgorithm_correction_independent}
\lean{QEC1.GaugingMeasurement.gaugingAlgorithm_correction_independent}
\leanok
\uses{def:GaugingMeasurement.gaugingAlgorithm, thm:GaugingMeasurement.byproductCorrectionAt_well_defined}
If all closed walks have zero parity, then two different walk choices produce the same correction vector:
\[
  (\operatorname{gaugingAlgorithm}(\mathrm{input}, \mathrm{outcomes}, \mathrm{walks}_1)).\mathrm{correction} = (\operatorname{gaugingAlgorithm}(\mathrm{input}, \mathrm{outcomes}, \mathrm{walks}_2)).\mathrm{correction}.
\]
\end{theorem}
\begin{proof}
\leanok
\uses{thm:GaugingMeasurement.byproductCorrectionAt_well_defined}
By extensionality, it suffices to show equality at each vertex $v$. This follows directly from the well-definedness of the byproduct correction (Theorem~\ref{thm:GaugingMeasurement.byproductCorrectionAt_well_defined}) applied to $\mathrm{walks}_1(v)$ and $\mathrm{walks}_2(v)$.
\end{proof}

\begin{definition}[Sign is Plus]
\label{def:GaugingMeasurement.signIsPlus}
\lean{QEC1.GaugingMeasurement.signIsPlus}
\leanok
\uses{def:GaugingMeasurement.measurementSign}
The proposition that $\sigma = 0 \pmod{2}$, meaning the $+1$ eigenvalue of $L$ was measured.
\end{definition}

\begin{definition}[Sign is Minus]
\label{def:GaugingMeasurement.signIsMinus}
\lean{QEC1.GaugingMeasurement.signIsMinus}
\leanok
\uses{def:GaugingMeasurement.measurementSign}
The proposition that $\sigma = 1 \pmod{2}$, meaning the $-1$ eigenvalue of $L$ was measured.
\end{definition}

\begin{theorem}[Sign Dichotomy]
\label{thm:GaugingMeasurement.sign_dichotomy}
\lean{QEC1.GaugingMeasurement.sign_dichotomy}
\leanok
\uses{def:GaugingMeasurement.signIsPlus, def:GaugingMeasurement.signIsMinus}
The measurement sign is always $0$ or $1$ in $\mathbb{Z}/2\mathbb{Z}$:
\[
  \operatorname{signIsPlus}(\mathrm{outcomes}) \lor \operatorname{signIsMinus}(\mathrm{outcomes}).
\]
\end{theorem}
\begin{proof}
\leanok
\uses{def:GaugingMeasurement.signIsPlus, def:GaugingMeasurement.signIsMinus, def:GaugingMeasurement.measurementSign}
This follows from the fact that every element of $\mathbb{Z}/2\mathbb{Z}$ is either $0$ or $1$, which is verified by case analysis ($\operatorname{fin\_cases}$) and simplification.
\end{proof}

\begin{definition}[Gauss's Law Operator Measured]
\label{def:GaugingMeasurement.gaussLawOperatorMeasured}
\lean{QEC1.GaugingMeasurement.gaussLawOperatorMeasured}
\leanok
\uses{def:GaussFlux.gaussLawOp, def:GaussFlux.ExtQubit}
The Gauss's law operator measured at vertex $v$ in step 3 of the gauging procedure:
\[
  A_v = X_v \prod_{e \ni v} X_e
\]
on the extended qubit system $V \oplus E$. This is the abbreviation for $\operatorname{gaussLawOp}(G, v)$.
\end{definition}

\begin{definition}[Edge Z Operator Measured]
\label{def:GaugingMeasurement.edgeZOperatorMeasured}
\lean{QEC1.GaugingMeasurement.edgeZOperatorMeasured}
\leanok
\uses{def:PauliOp.pauliZ, def:GaussFlux.ExtQubit}
The edge $Z$ operator measured at edge $e$ in step 4 of the gauging procedure:
\[
  Z_e \text{ on } V \oplus E,
\]
defined as $\operatorname{pauliZ}(\operatorname{inr}(e))$.
\end{definition}

\begin{theorem}[Gauss's Law Measurements Commute]
\label{thm:GaugingMeasurement.gaussLaw_measured_commute}
\lean{QEC1.GaugingMeasurement.gaussLaw_measured_commute}
\leanok
\uses{def:GaugingMeasurement.gaussLawOperatorMeasured, thm:GaussFlux.gaussLaw_commute, def:PauliOp.PauliCommute}
All Gauss's law operators commute: for any vertices $v, w \in V$,
\[
  [A_v, A_w] = 0.
\]
They can be measured in any order in step 3.
\end{theorem}
\begin{proof}
\leanok
\uses{thm:GaussFlux.gaussLaw_commute}
This follows directly from the commutativity of Gauss's law operators (Theorem~\ref{thm:GaussFlux.gaussLaw_commute}).
\end{proof}

\begin{theorem}[Edge Z Measurements Commute]
\label{thm:GaugingMeasurement.edgeZ_measured_commute}
\lean{QEC1.GaugingMeasurement.edgeZ_measured_commute}
\leanok
\uses{def:GaugingMeasurement.edgeZOperatorMeasured, def:PauliOp.PauliCommute, def:PauliOp.pauliZ}
All edge $Z$ operators commute: for any edges $e_1, e_2 \in E$,
\[
  [Z_{e_1}, Z_{e_2}] = 0.
\]
They can be measured in any order in step 4.
\end{theorem}
\begin{proof}
\leanok
\uses{def:GaugingMeasurement.edgeZOperatorMeasured, def:PauliOp.pauliZ, def:PauliOp.PauliCommute}
By simplification using the definitions of edge $Z$ operators, Pauli commutation, and the symplectic inner product, the goal reduces to showing a Finset sum equals zero. For each qubit $q$, the summand involves $\operatorname{pauliZ}$ which has zero $x$-vector, so each term is $0 \cdot z + x \cdot 0 = 0$. The sum of zeros is zero.
\end{proof}

\begin{theorem}[Gauss's Law Product Gives Logical Operator]
\label{thm:GaugingMeasurement.gaussLaw_product_gives_logical}
\lean{QEC1.GaugingMeasurement.gaussLaw_product_gives_logical}
\leanok
\uses{def:GaugingMeasurement.gaussLawOperatorMeasured, def:GaussFlux.logicalOp, thm:GaussFlux.gaussLaw_product}
The product of all Gauss's law operators equals the logical operator $L = \prod_{v \in V} X_v$:
\[
  \prod_{v \in V} A_v = L.
\]
This is why $\sigma = \sum_v \varepsilon_v$ gives the measurement result of $L$.
\end{theorem}
\begin{proof}
\leanok
\uses{thm:GaussFlux.gaussLaw_product}
This follows directly from the Gauss's law product theorem (Theorem~\ref{thm:GaussFlux.gaussLaw_product}).
\end{proof}

\begin{lemma}[Edge Z Operator is Pure Z-Type]
\label{lem:GaugingMeasurement.edgeZOperatorMeasured_xVec}
\lean{QEC1.GaugingMeasurement.edgeZOperatorMeasured_xVec}
\leanok
\uses{def:GaugingMeasurement.edgeZOperatorMeasured, def:PauliOp.pauliZ}
The edge $Z$ operator $Z_e$ has no $X$-support:
\[
  (Z_e).\mathrm{xVec} = 0.
\]
\end{lemma}
\begin{proof}
\leanok
\uses{def:GaugingMeasurement.edgeZOperatorMeasured, def:PauliOp.pauliZ}
By simplification using the definitions of the edge $Z$ operator and $\operatorname{pauliZ}$, the $x$-vector is identically zero.
\end{proof}

\begin{lemma}[Gauss's Law Operator is Pure X-Type]
\label{lem:GaugingMeasurement.gaussLawOperatorMeasured_zVec}
\lean{QEC1.GaugingMeasurement.gaussLawOperatorMeasured_zVec}
\leanok
\uses{def:GaugingMeasurement.gaussLawOperatorMeasured, def:GaussFlux.gaussLawOp}
The Gauss's law operator $A_v$ has no $Z$-support:
\[
  (A_v).\mathrm{zVec} = 0.
\]
\end{lemma}
\begin{proof}
\leanok
\uses{def:GaugingMeasurement.gaussLawOperatorMeasured}
By simplification using the definition of the Gauss's law operator measured, the $z$-vector is identically zero.
\end{proof}

\begin{theorem}[Gauss's Law Operator is Self-Inverse]
\label{thm:GaugingMeasurement.gaussLawOperatorMeasured_mul_self}
\lean{QEC1.GaugingMeasurement.gaussLawOperatorMeasured_mul_self}
\leanok
\uses{def:GaugingMeasurement.gaussLawOperatorMeasured, lem:PauliOp.mul_self}
Each Gauss's law operator is self-inverse:
\[
  A_v \cdot A_v = \mathbf{1}.
\]
\end{theorem}
\begin{proof}
\leanok
\uses{lem:PauliOp.mul_self}
This follows directly from the fact that every Pauli operator is self-inverse ($P \cdot P = \mathbf{1}$).
\end{proof}

\begin{theorem}[Edge Z Operator is Self-Inverse]
\label{thm:GaugingMeasurement.edgeZOperatorMeasured_mul_self}
\lean{QEC1.GaugingMeasurement.edgeZOperatorMeasured_mul_self}
\leanok
\uses{def:GaugingMeasurement.edgeZOperatorMeasured, lem:PauliOp.mul_self}
Each edge $Z$ operator is self-inverse:
\[
  Z_e \cdot Z_e = \mathbf{1}.
\]
\end{theorem}
\begin{proof}
\leanok
\uses{lem:PauliOp.mul_self}
This follows directly from the fact that every Pauli operator is self-inverse ($P \cdot P = \mathbf{1}$).
\end{proof}

%--- Thm_1: GaugingMeasurementCorrectness ---

I'll start by reading the Lean file to understand its contents.Now I have a thorough understanding of the file. Let me produce the complete LaTeX translation.

\chapter{Thm 1: Gauging Measurement Correctness}

\begin{definition}[Gauss Subset Product]
\label{def:GaugingMeasurementCorrectness.gaussSubsetProduct}
\lean{QEC1.GaugingMeasurementCorrectness.gaussSubsetProduct}
\leanok
\uses{def:PauliOp, def:GaussFlux.ExtQubit, def:GraphMaps.coboundaryMap}
Let $G$ be a simple graph on vertex set $V$ with edge set $E(G)$. Given a binary vector $c : V \to \mathbb{Z}/2\mathbb{Z}$, define the Pauli operator $\operatorname{gaussSubsetProduct}(c)$ on the extended qubit system $V \oplus E(G)$ by:
\[
  \operatorname{xVec}(q) = \begin{cases} c(v) & \text{if } q = \operatorname{inl}(v) \text{ for } v \in V, \\ (\delta c)(e) & \text{if } q = \operatorname{inr}(e) \text{ for } e \in E(G), \end{cases}
  \qquad \operatorname{zVec} = 0,
\]
where $\delta$ denotes the coboundary map. This represents the product $\prod_{v \in \operatorname{supp}(c)} A_v$ of Gauss's law operators, yielding $X_V(c) \cdot X_E(\delta c)$ on the extended system.
\end{definition}

\begin{theorem}[Gauss Subset Product is Pure X]
\label{thm:GaugingMeasurementCorrectness.gaussSubsetProduct_pure_X}
\lean{QEC1.GaugingMeasurementCorrectness.gaussSubsetProduct_pure_X}
\leanok
\uses{def:GaugingMeasurementCorrectness.gaussSubsetProduct}
For any binary vector $c : V \to \mathbb{Z}/2\mathbb{Z}$, the operator $\operatorname{gaussSubsetProduct}(c)$ is pure $X$-type, i.e., its $Z$-vector is identically zero.
\end{theorem}

\begin{proof}
\leanok
\uses{def:GaugingMeasurementCorrectness.gaussSubsetProduct}
This holds by definition, since the $Z$-vector of $\operatorname{gaussSubsetProduct}(c)$ is set to $0$.
\end{proof}

\begin{theorem}[Gauss Subset Product of Zero]
\label{thm:GaugingMeasurementCorrectness.gaussSubsetProduct_zero}
\lean{QEC1.GaugingMeasurementCorrectness.gaussSubsetProduct_zero}
\leanok
\uses{def:GaugingMeasurementCorrectness.gaussSubsetProduct, def:GraphMaps.coboundaryMap}
The Gauss subset product for the zero vector is the identity operator:
\[
  \operatorname{gaussSubsetProduct}(0) = \mathbf{1}.
\]
\end{theorem}

\begin{proof}
\leanok
\uses{def:GaugingMeasurementCorrectness.gaussSubsetProduct, def:GraphMaps.coboundaryMap}
By extensionality, it suffices to show equality of both $X$- and $Z$-vectors for each qubit. For the $X$-vector: if $q = \operatorname{inl}(v)$, then $c(v) = 0$ by definition. If $q = \operatorname{inr}(e)$, we use the fact that $\delta(0) = 0$ (linearity of the coboundary map), so $(\delta 0)(e) = 0$. For the $Z$-vector, both sides are zero by definition.
\end{proof}

\begin{theorem}[Gauss Subset Product of All-Ones Equals Logical Operator]
\label{thm:GaugingMeasurementCorrectness.gaussSubsetProduct_allOnes}
\lean{QEC1.GaugingMeasurementCorrectness.gaussSubsetProduct_allOnes}
\leanok
\uses{def:GaugingMeasurementCorrectness.gaussSubsetProduct, def:GaussFlux.logicalOp, def:GraphMaps.allOnes, def:GraphMaps.coboundaryMap, thm:GraphMaps.allOnes_mem_ker_coboundary}
The Gauss subset product for the all-ones vector equals the logical operator $L$:
\[
  \operatorname{gaussSubsetProduct}(\mathbf{1}) = L.
\]
\end{theorem}

\begin{proof}
\leanok
\uses{def:GaugingMeasurementCorrectness.gaussSubsetProduct, def:GaussFlux.logicalOp, def:GraphMaps.allOnes, def:GraphMaps.coboundaryMap, thm:GraphMaps.allOnes_mem_ker_coboundary}
By extensionality, we check each component. For the $X$-vector at $q = \operatorname{inl}(v)$: the Gauss subset product gives $1$ and the logical operator gives $1$, so they agree. For $q = \operatorname{inr}(e)$: we use the fact that the all-ones vector lies in $\ker(\delta)$ (i.e., $\delta(\mathbf{1}) = 0$), which follows from \texttt{allOnes\_mem\_ker\_coboundary}. Rewriting via the kernel membership, we obtain $(\delta \mathbf{1})(e) = 0$, matching the logical operator's edge component. For the $Z$-vector, both are zero.
\end{proof}

\begin{theorem}[Gauss Subset Product is Additive]
\label{thm:GaugingMeasurementCorrectness.gaussSubsetProduct_add}
\lean{QEC1.GaugingMeasurementCorrectness.gaussSubsetProduct_add}
\leanok
\uses{def:GaugingMeasurementCorrectness.gaussSubsetProduct, def:GraphMaps.coboundaryMap}
The Gauss subset product is a group homomorphism: for binary vectors $c_1, c_2 : V \to \mathbb{Z}/2\mathbb{Z}$,
\[
  \operatorname{gaussSubsetProduct}(c_1 + c_2) = \operatorname{gaussSubsetProduct}(c_1) \cdot \operatorname{gaussSubsetProduct}(c_2).
\]
\end{theorem}

\begin{proof}
\leanok
\uses{def:GaugingMeasurementCorrectness.gaussSubsetProduct, def:GraphMaps.coboundaryMap}
By extensionality on both components. For the $X$-vector at $q = \operatorname{inl}(v)$: both sides give $(c_1 + c_2)(v) = c_1(v) + c_2(v)$. For $q = \operatorname{inr}(e)$: by linearity of the coboundary map, $\delta(c_1 + c_2) = \delta c_1 + \delta c_2$, so the edge components agree. For the $Z$-vector: both sides give $0 + 0 = 0$.
\end{proof}

\begin{definition}[Finset Indicator Function]
\label{def:GaugingMeasurementCorrectness.finsetIndicator}
\lean{QEC1.GaugingMeasurementCorrectness.finsetIndicator}
\leanok
\uses{def:PauliOp}
The indicator function of a finset $S \subseteq V$ as a binary vector:
\[
  \operatorname{finsetIndicator}(S)(v) = \begin{cases} 1 & \text{if } v \in S, \\ 0 & \text{if } v \notin S. \end{cases}
\]
\end{definition}

\begin{theorem}[Product of Gauss Operators Equals Gauss Subset Product]
\label{thm:GaugingMeasurementCorrectness.prod_gaussLawOp_eq_gaussSubsetProduct}
\lean{QEC1.GaugingMeasurementCorrectness.prod_gaussLawOp_eq_gaussSubsetProduct}
\leanok
\uses{def:GaugingMeasurementCorrectness.gaussSubsetProduct, def:GaugingMeasurementCorrectness.finsetIndicator, def:GaussFlux.gaussLawOp, def:GraphMaps.coboundaryMap, thm:GraphMaps.coboundaryMap_single}
For any finset $S \subseteq V$, the product of Gauss's law operators over $S$ equals the Gauss subset product of the indicator:
\[
  \prod_{v \in S} A_v = \operatorname{gaussSubsetProduct}\bigl(\operatorname{finsetIndicator}(S)\bigr).
\]
\end{theorem}

\begin{proof}
\leanok
\uses{def:GaugingMeasurementCorrectness.gaussSubsetProduct, def:GaugingMeasurementCorrectness.finsetIndicator, def:GaussFlux.gaussLawOp, def:GraphMaps.coboundaryMap, thm:GraphMaps.coboundaryMap_single}
We proceed by induction on $S$ using finset insertion.

\textbf{Base case} ($S = \emptyset$): The empty product is the identity. We show that $\operatorname{finsetIndicator}(\emptyset) = 0$ by extensionality (every vertex gives $0$), and then $\operatorname{gaussSubsetProduct}(0) = \mathbf{1}$.

\textbf{Inductive step}: Suppose $S' = \{a\} \cup S$ with $a \notin S$. By the induction hypothesis, $\prod_{v \in S} A_v = \operatorname{gaussSubsetProduct}(\operatorname{finsetIndicator}(S))$. We rewrite $\prod_{v \in S'} A_v = A_a \cdot \prod_{v \in S} A_v$ and check equality by extensionality. For the $X$-vector at $\operatorname{inl}(v)$: both sides reduce to $\operatorname{finsetIndicator}(\{a\} \cup S)(v)$ by case analysis on whether $v = a$. For $\operatorname{inr}(e)$: we decompose the indicator as $\operatorname{finsetIndicator}(\{a\} \cup S) = \mathbf{1}_{\{a\}} + \operatorname{finsetIndicator}(S)$, apply linearity of $\delta$, and use $\delta(\mathbf{1}_{\{a\}}) = \operatorname{coboundaryMap\_single}$ to match the Gauss operator's edge component. For the $Z$-vector, both sides are zero.
\end{proof}

\begin{theorem}[Full Gauss Product Equals Logical Operator]
\label{thm:GaugingMeasurementCorrectness.prod_gaussLawOp_eq_logical}
\lean{QEC1.GaugingMeasurementCorrectness.prod_gaussLawOp_eq_logical}
\leanok
\uses{def:GaussFlux.gaussLawOp, def:GaussFlux.logicalOp, thm:GaussFlux.gaussLaw_product}
The product of all Gauss's law operators equals the logical operator:
\[
  \prod_{v \in V} A_v = L.
\]
\end{theorem}

\begin{proof}
\leanok
\uses{thm:GaussFlux.gaussLaw_product}
This follows directly from the Gauss law product theorem (\texttt{gaussLaw\_product}).
\end{proof}

\begin{theorem}[Coboundary Equality via Kernel]
\label{thm:GaugingMeasurementCorrectness.coboundary_eq_iff_diff_in_ker}
\lean{QEC1.GaugingMeasurementCorrectness.coboundary_eq_iff_diff_in_ker}
\leanok
\uses{def:GraphMaps.coboundaryMap}
Two binary vectors $c, c' : V \to \mathbb{Z}/2\mathbb{Z}$ have the same coboundary if and only if their sum lies in the kernel of $\delta$:
\[
  \delta c = \delta c' \iff c + c' \in \ker(\delta).
\]
\end{theorem}

\begin{proof}
\leanok
\uses{def:GraphMaps.coboundaryMap}
We prove both directions.

$(\Rightarrow)$: Assume $\delta c = \delta c'$. By linearity, $\delta(c + c') = \delta c + \delta c'$. Rewriting with the hypothesis, each component satisfies $\delta c(e) + \delta c(e) = 0$ by the characteristic-$2$ identity $a + a = 0$. Hence $\delta(c + c') = 0$, so $c + c' \in \ker(\delta)$.

$(\Leftarrow)$: Assume $\delta(c + c') = 0$. By linearity, $\delta c + \delta c' = 0$. For each edge $e$, we have $\delta c(e) + \delta c'(e) = 0$, and by the characteristic-$2$ identity $a + b = 0 \Rightarrow a = b$, we conclude $\delta c(e) = \delta c'(e)$. By extensionality, $\delta c = \delta c'$.
\end{proof}

\begin{theorem}[Coboundary Preimage Shift]
\label{thm:GaugingMeasurementCorrectness.coboundary_preimage_shift}
\lean{QEC1.GaugingMeasurementCorrectness.coboundary_preimage_shift}
\leanok
\uses{def:GraphMaps.coboundaryMap}
If $\delta c' = \omega$, then $\delta c = \omega$ if and only if $c + c' \in \ker(\delta)$.
\end{theorem}

\begin{proof}
\leanok
\uses{def:GraphMaps.coboundaryMap}
Rewriting $\omega$ as $\delta c'$ via the hypothesis, this reduces directly to the equivalence $\delta c = \delta c' \iff c + c' \in \ker(\delta)$.
\end{proof}

\begin{theorem}[Coboundary Preimage for Connected Graphs]
\label{thm:GaugingMeasurementCorrectness.coboundary_preimage_connected}
\lean{QEC1.GaugingMeasurementCorrectness.coboundary_preimage_connected}
\leanok
\uses{def:GraphMaps.coboundaryMap, def:GraphMaps.allOnes, thm:GraphMaps.ker_coboundary_connected}
For a connected graph $G$, if $\delta c' = \omega$ and $\delta c = \omega$, then
\[
  c = c' \quad \text{or} \quad c = c' + \mathbf{1},
\]
where $\mathbf{1}$ denotes the all-ones vector.
\end{theorem}

\begin{proof}
\leanok
\uses{def:GraphMaps.coboundaryMap, def:GraphMaps.allOnes, thm:GraphMaps.ker_coboundary_connected}
From $\delta c = \omega = \delta c'$, we obtain $c + c' \in \ker(\delta)$ by the coboundary preimage shift. Since $G$ is connected, the kernel classification theorem (\texttt{ker\_coboundary\_connected}) gives $c + c' = 0$ or $c + c' = \mathbf{1}$.

In the first case, for each $v$ we have $c(v) + c'(v) = 0$, so $c(v) = c'(v)$ by the characteristic-$2$ identity, giving $c = c'$.

In the second case, $c(v) + c'(v) = 1$ for all $v$. Rearranging via $c(v) = 1 + c'(v) = c'(v) + 1 = (c' + \mathbf{1})(v)$, we get $c = c' + \mathbf{1}$.
\end{proof}

\begin{definition}[Signed Outcome Function]
\label{def:GaugingMeasurementCorrectness.signedOutcome}
\lean{QEC1.GaugingMeasurementCorrectness.signedOutcome}
\leanok
\uses{def:PauliOp}
The signed outcome function $\varepsilon(c)$ for a binary outcome vector $\varepsilon : V \to \mathbb{Z}/2\mathbb{Z}$ and subset vector $c : V \to \mathbb{Z}/2\mathbb{Z}$ is:
\[
  \varepsilon(c) = \sum_{v \in V} c(v) \cdot \varepsilon(v) \in \mathbb{Z}/2\mathbb{Z}.
\]
This encodes the product $\prod_{v \in \operatorname{supp}(c)} \varepsilon_v$ of $\pm 1$ outcomes as an additive quantity in $\mathbb{Z}/2\mathbb{Z}$.
\end{definition}

\begin{theorem}[Signed Outcome is Additive]
\label{thm:GaugingMeasurementCorrectness.signedOutcome_add}
\lean{QEC1.GaugingMeasurementCorrectness.signedOutcome_add}
\leanok
\uses{def:GaugingMeasurementCorrectness.signedOutcome}
The signed outcome function is a group homomorphism:
\[
  \varepsilon(c_1 + c_2) = \varepsilon(c_1) + \varepsilon(c_2).
\]
\end{theorem}

\begin{proof}
\leanok
\uses{def:GaugingMeasurementCorrectness.signedOutcome}
Expanding the definition, $\varepsilon(c_1 + c_2) = \sum_v (c_1(v) + c_2(v)) \cdot \varepsilon(v)$. Distributing the sum gives $\sum_v c_1(v) \cdot \varepsilon(v) + \sum_v c_2(v) \cdot \varepsilon(v) = \varepsilon(c_1) + \varepsilon(c_2)$, using linearity of addition over the finite sum and the ring identity $(a + b) \cdot c = a \cdot c + b \cdot c$.
\end{proof}

\begin{theorem}[Signed Outcome of All-Ones]
\label{thm:GaugingMeasurementCorrectness.signedOutcome_allOnes}
\lean{QEC1.GaugingMeasurementCorrectness.signedOutcome_allOnes}
\leanok
\uses{def:GaugingMeasurementCorrectness.signedOutcome, def:GraphMaps.allOnes}
The signed outcome for the all-ones vector equals the measurement sign $\sigma$:
\[
  \varepsilon(\mathbf{1}) = \sum_{v \in V} \varepsilon(v) = \sigma.
\]
\end{theorem}

\begin{proof}
\leanok
\uses{def:GaugingMeasurementCorrectness.signedOutcome, def:GraphMaps.allOnes}
Expanding the definition with $c = \mathbf{1}$: $\varepsilon(\mathbf{1}) = \sum_v 1 \cdot \varepsilon(v) = \sum_v \varepsilon(v)$.
\end{proof}

\begin{theorem}[Signed Outcome Shift Identity]
\label{thm:GaugingMeasurementCorrectness.signedOutcome_shift}
\lean{QEC1.GaugingMeasurementCorrectness.signedOutcome_shift}
\leanok
\uses{def:GaugingMeasurementCorrectness.signedOutcome, def:GraphMaps.allOnes}
For any binary vectors $\varepsilon$ and $c'$,
\[
  \varepsilon(c' + \mathbf{1}) = \varepsilon(c') + \sigma,
\]
where $\sigma = \sum_{v \in V} \varepsilon(v)$.
\end{theorem}

\begin{proof}
\leanok
\uses{def:GaugingMeasurementCorrectness.signedOutcome, def:GraphMaps.allOnes}
By the additivity of the signed outcome, $\varepsilon(c' + \mathbf{1}) = \varepsilon(c') + \varepsilon(\mathbf{1})$. Then $\varepsilon(\mathbf{1}) = \sum_v \varepsilon(v)$ by the all-ones identity.
\end{proof}

\begin{theorem}[Gauss Subset Product Shift Equals Product with Logical]
\label{thm:GaugingMeasurementCorrectness.gaussSubsetProduct_shift_eq_mul_logical}
\lean{QEC1.GaugingMeasurementCorrectness.gaussSubsetProduct_shift_eq_mul_logical}
\leanok
\uses{def:GaugingMeasurementCorrectness.gaussSubsetProduct, def:GaussFlux.logicalOp, def:GraphMaps.allOnes}
On vertex qubits, the Pauli operator for subset $c' + \mathbf{1}$ is the product of the operator for $c'$ and the logical operator $L$:
\[
  \operatorname{gaussSubsetProduct}(c' + \mathbf{1}) = \operatorname{gaussSubsetProduct}(c') \cdot L.
\]
\end{theorem}

\begin{proof}
\leanok
\uses{def:GaugingMeasurementCorrectness.gaussSubsetProduct, def:GaussFlux.logicalOp, def:GraphMaps.allOnes}
Rewriting via $\operatorname{gaussSubsetProduct}(\mathbf{1}) = L$ (the all-ones identity), the left-hand side becomes $\operatorname{gaussSubsetProduct}(c' + \mathbf{1})$, which equals $\operatorname{gaussSubsetProduct}(c') \cdot \operatorname{gaussSubsetProduct}(\mathbf{1}) = \operatorname{gaussSubsetProduct}(c') \cdot L$ by the additivity of the Gauss subset product.
\end{proof}

\begin{definition}[Walk Parity Vector]
\label{def:GaugingMeasurementCorrectness.walkParityVector}
\lean{QEC1.GaugingMeasurementCorrectness.walkParityVector}
\leanok
\uses{def:GaugingMeasurement.walkParity}
Given a base vertex $v_0 \in V$, a family of walks $\gamma_v : v_0 \leadsto v$ for each $v \in V$, and an edge outcome vector $\omega : E(G) \to \mathbb{Z}/2\mathbb{Z}$, the \emph{walk parity vector} is:
\[
  c'(v) = \operatorname{walkParity}(\omega, \gamma_v) \in \mathbb{Z}/2\mathbb{Z}.
\]
This gives the byproduct correction vector from walk parities.
\end{definition}

\begin{theorem}[Walk Parity Vector at Base Vertex]
\label{thm:GaugingMeasurementCorrectness.walkParityVector_base}
\lean{QEC1.GaugingMeasurementCorrectness.walkParityVector_base}
\leanok
\uses{def:GaugingMeasurementCorrectness.walkParityVector, def:GaugingMeasurement.walkParity}
The walk parity vector at the base vertex $v_0$ is $0$, provided $\gamma_{v_0}$ is the nil walk:
\[
  c'(v_0) = 0.
\]
\end{theorem}

\begin{proof}
\leanok
\uses{def:GaugingMeasurementCorrectness.walkParityVector, def:GaugingMeasurement.walkParity}
Unfolding the definition of the walk parity vector at $v_0$, we get $\operatorname{walkParity}(\omega, \gamma_{v_0})$. Since $\gamma_{v_0}$ is the nil walk, the walk parity is $0$.
\end{proof}

\begin{theorem}[Walk Parity Vector Gives Coboundary Preimage]
\label{thm:GaugingMeasurementCorrectness.walkParityVector_coboundary_eq}
\lean{QEC1.GaugingMeasurementCorrectness.walkParityVector_coboundary_eq}
\leanok
\uses{def:GaugingMeasurementCorrectness.walkParityVector, def:GaugingMeasurement.walkParity, def:GraphMaps.coboundaryMap, thm:GaugingMeasurement.walkParity_append, thm:GaugingMeasurement.walkParity_reverse}
If all closed walks have zero parity (i.e., $\operatorname{walkParity}(\omega, c) = 0$ for every closed walk $c$ at every vertex $v$), then for each edge $e = \{a, b\} \in E(G)$:
\[
  (\delta c')(e) = \omega(e),
\]
where $c'$ is the walk parity vector.
\end{theorem}

\begin{proof}
\leanok
\uses{def:GaugingMeasurementCorrectness.walkParityVector, def:GaugingMeasurement.walkParity, def:GraphMaps.coboundaryMap, thm:GaugingMeasurement.walkParity_append, thm:GaugingMeasurement.walkParity_reverse}
Let $e = \{a, b\}$ with $a \sim b$ in $G$. We construct the closed walk $p = \gamma_a \cdot (a \to b) \cdot \gamma_b^{-1}$ at $v_0$. By hypothesis, $\operatorname{walkParity}(\omega, p) = 0$.

Applying the walk parity append theorem: $\operatorname{walkParity}(\omega, p) = \operatorname{walkParity}(\omega, \gamma_a) + \operatorname{walkParity}(\omega, (a \to b) \cdot \gamma_b^{-1})$.

Applying the walk parity cons rule: $\operatorname{walkParity}(\omega, (a \to b) \cdot \gamma_b^{-1}) = \omega(e) + \operatorname{walkParity}(\omega, \gamma_b^{-1})$.

Applying the walk parity reverse rule: $\operatorname{walkParity}(\omega, \gamma_b^{-1}) = \operatorname{walkParity}(\omega, \gamma_b)$.

Combining: $c'(a) + (\omega(e) + c'(b)) = 0$. Rearranging in $\mathbb{Z}/2\mathbb{Z}$: $c'(a) + c'(b) + \omega(e) = 0$, so $c'(a) + c'(b) = \omega(e)$. Since $(\delta c')(e) = c'(a) + c'(b)$, we obtain $(\delta c')(e) = \omega(e)$.
\end{proof}

\begin{theorem}[Byproduct Correction Equals Walk Parity Product]
\label{thm:GaugingMeasurementCorrectness.byproductCorrectionOp_eq_prodX_walkParity}
\lean{QEC1.GaugingMeasurementCorrectness.byproductCorrectionOp_eq_prodX_walkParity}
\leanok
\uses{def:GaugingMeasurement.byproductCorrectionOp, def:GaugingMeasurement.byproductCorrectionAt, def:GaugingMeasurementCorrectness.walkParityVector, def:GaugingMeasurement.walkParity}
The byproduct correction operator equals the pure-$X$ Pauli operator with $X$-vector given by the walk parity vector:
\[
  \operatorname{byproductCorrectionOp}(\gamma, \omega) = \bigl(\operatorname{walkParityVector}(\gamma, \omega),\; 0\bigr).
\]
\end{theorem}

\begin{proof}
\leanok
\uses{def:GaugingMeasurement.byproductCorrectionOp, def:GaugingMeasurement.byproductCorrectionAt, def:GaugingMeasurementCorrectness.walkParityVector}
By extensionality on both components. For the $X$-vector at vertex $v$: unfolding the definitions of byproduct correction and walk parity vector, both yield $\operatorname{walkParity}(\omega, \gamma_v)$. For the $Z$-vector: both sides are $0$.
\end{proof}

\begin{theorem}[Walk Parity is Coboundary Preimage]
\label{thm:GaugingMeasurementCorrectness.walkParity_is_coboundary_preimage}
\lean{QEC1.GaugingMeasurementCorrectness.walkParity_is_coboundary_preimage}
\leanok
\uses{def:GaugingMeasurementCorrectness.walkParityVector, def:GraphMaps.coboundaryMap, def:GaugingMeasurement.walkParity}
When all closed walks have zero parity, the walk parity vector $c'$ satisfies $\delta c' = \omega$:
\[
  \delta\bigl(\operatorname{walkParityVector}(\gamma, \omega)\bigr) = \omega.
\]
\end{theorem}

\begin{proof}
\leanok
\uses{def:GaugingMeasurementCorrectness.walkParityVector, def:GraphMaps.coboundaryMap}
By extensionality: for each edge $e$, the result follows from the per-edge identity established in \texttt{walkParityVector\_coboundary\_eq}.
\end{proof}

\begin{definition}[Eigenspace Projector Pair]
\label{def:GaugingMeasurementCorrectness.eigenspaceProjectorPair}
\lean{QEC1.GaugingMeasurementCorrectness.eigenspaceProjectorPair}
\leanok
\uses{def:PauliOp}
The $\sigma$-dependent projector pair on vertex qubits, representing the Pauli decomposition of $\mathbf{1} + \sigma L$:
\[
  \operatorname{eigenspaceProjectorPair}(\sigma) = \bigl(\mathbf{1},\; (\sigma \cdot \mathbf{1}_V, 0)\bigr),
\]
where the first component is the identity operator and the second has $X$-vector $v \mapsto \sigma$ and $Z$-vector $0$. When $\sigma = 0$, both components are the identity; when $\sigma = 1$, the second component is $L|_V$.
\end{definition}

\begin{theorem}[Eigenspace Projector Pair at Zero]
\label{thm:GaugingMeasurementCorrectness.eigenspaceProjectorPair_zero}
\lean{QEC1.GaugingMeasurementCorrectness.eigenspaceProjectorPair_zero}
\leanok
\uses{def:GaugingMeasurementCorrectness.eigenspaceProjectorPair}
When $\sigma = 0$, the projector pair is $(\mathbf{1}, \mathbf{1})$, corresponding to $(\mathbf{1} + L)/2$ projecting onto the $+1$ eigenspace.
\end{theorem}

\begin{proof}
\leanok
\uses{def:GaugingMeasurementCorrectness.eigenspaceProjectorPair}
Unfolding the definition with $\sigma = 0$: the first component is $\mathbf{1}$ by construction. For the second, by extensionality on both $X$- and $Z$-vectors, the $X$-vector gives $v \mapsto 0$ and the $Z$-vector is $0$, which equals $\mathbf{1}$.
\end{proof}

\begin{theorem}[Eigenspace Projector Pair at One]
\label{thm:GaugingMeasurementCorrectness.eigenspaceProjectorPair_one_snd_eq_L_on_V}
\lean{QEC1.GaugingMeasurementCorrectness.eigenspaceProjectorPair_one_snd_eq_L_on_V}
\leanok
\uses{def:GaugingMeasurementCorrectness.eigenspaceProjectorPair}
When $\sigma = 1$, the second component has $X$-vector equal to the all-ones vector, matching $L$ restricted to vertices:
\[
  \operatorname{eigenspaceProjectorPair}(1)_2 = (v \mapsto 1,\; 0).
\]
\end{theorem}

\begin{proof}
\leanok
\uses{def:GaugingMeasurementCorrectness.eigenspaceProjectorPair}
This holds by definitional equality (reflexivity).
\end{proof}

\begin{definition}[Logical Operator on Vertices]
\label{def:GaugingMeasurementCorrectness.logicalOpV}
\lean{QEC1.GaugingMeasurementCorrectness.logicalOpV}
\leanok
\uses{def:PauliOp}
The logical operator $L$ restricted to vertex qubits only:
\[
  L|_V = (v \mapsto 1,\; 0),
\]
i.e., the Pauli operator with $X$-vector identically $1$ and $Z$-vector identically $0$.
\end{definition}

\begin{theorem}[Vertex Logical is Self-Inverse]
\label{thm:GaugingMeasurementCorrectness.logicalOpV_self_inverse}
\lean{QEC1.GaugingMeasurementCorrectness.logicalOpV_self_inverse}
\leanok
\uses{def:GaugingMeasurementCorrectness.logicalOpV, def:PauliOp}
The vertex restriction of $L$ satisfies $L|_V \cdot L|_V = \mathbf{1}$.
\end{theorem}

\begin{proof}
\leanok
\uses{def:GaugingMeasurementCorrectness.logicalOpV}
This follows from the general self-inverse property of Pauli operators: $P \cdot P = \mathbf{1}$ for any Pauli operator $P$ (since in $\mathbb{Z}/2\mathbb{Z}$, $a + a = 0$).
\end{proof}

\begin{theorem}[Correction Commutes with Vertex Logical]
\label{thm:GaugingMeasurementCorrectness.correction_commutes_with_logicalV}
\lean{QEC1.GaugingMeasurementCorrectness.correction_commutes_with_logicalV}
\leanok
\uses{def:GaugingMeasurement.byproductCorrectionOp, def:GaugingMeasurementCorrectness.logicalOpV, def:PauliOp.PauliCommute, def:PauliOp.symplecticInner}
The byproduct correction operator commutes with $L|_V$, since both are pure $X$-type operators.
\end{theorem}

\begin{proof}
\leanok
\uses{def:GaugingMeasurement.byproductCorrectionOp, def:GaugingMeasurementCorrectness.logicalOpV, def:PauliOp.symplecticInner}
The symplectic inner product $\langle \operatorname{correction}, L|_V \rangle = \sum_v (\operatorname{xVec}_{\mathrm{corr}}(v) \cdot \operatorname{zVec}_{L|_V}(v) + \operatorname{zVec}_{\mathrm{corr}}(v) \cdot \operatorname{xVec}_{L|_V}(v))$. Since $L|_V$ has $Z$-vector $0$ and the correction has $Z$-vector $0$, every term in the sum is $0$. Hence the symplectic inner product is $0$, proving commutativity.
\end{proof}

\begin{definition}[Projective Measurement of $L$]
\label{def:GaugingMeasurementCorrectness.ProjectiveMeasurementOfL}
\lean{QEC1.GaugingMeasurementCorrectness.ProjectiveMeasurementOfL}
\leanok
\uses{def:GaugingMeasurement.GaugingOutcomes, def:GaugingMeasurement.measurementSign, def:GaussFlux.gaussLawOp, def:GaussFlux.logicalOp, def:GraphMaps.coboundaryMap, def:GraphMaps.allOnes, def:GaugingMeasurementCorrectness.gaussSubsetProduct, def:GaugingMeasurementCorrectness.signedOutcome, def:GaugingMeasurementCorrectness.walkParityVector, def:GaugingMeasurement.byproductCorrectionOp, def:GaugingMeasurement.walkParity, def:PauliOp.PauliCommute, def:GaugingMeasurementCorrectness.logicalOpV}
A gauging procedure implements a \emph{projective measurement of $L$} if the following conditions hold:
\begin{enumerate}
  \item \textbf{Sign determination}: The measurement sign $\sigma$ equals $\sum_{v \in V} \varepsilon_v$, the sum of all Gauss outcomes.
  \item \textbf{Gauss product}: $\prod_{v \in V} A_v = L$.
  \item \textbf{Preimage structure}: For any $c'$ with $\delta c' = \omega$, every $c$ with $\delta c = \omega$ satisfies $c = c'$ or $c = c' + \mathbf{1}$.
  \item \textbf{Two terms}: $\operatorname{gaussSubsetProduct}(c' + \mathbf{1}) = \operatorname{gaussSubsetProduct}(c') \cdot L$.
  \item \textbf{Signed outcome additivity}: $\varepsilon(c' + \mathbf{1}) = \varepsilon(c') + \sigma$.
  \item \textbf{Walk parity preimage}: Under the closed-walk-zero-parity condition, $\delta(\operatorname{walkParityVector}(\gamma, \omega)) = \omega$.
  \item \textbf{Correction pure $X$}: The byproduct correction operator has $Z$-vector $0$.
  \item \textbf{Correction commutes with $L|_V$}: The correction is a pure $X$-type operator, hence commutes with $L|_V$.
  \item \textbf{Correction self-inverse}: Applying the correction twice gives the identity.
  \item \textbf{Walk independence}: Under the closed-walk-zero-parity condition, the walk parity vector is independent of the choice of walks.
\end{enumerate}
\end{definition}

\begin{theorem}[Walk Parity Vector is Well-Defined]
\label{thm:GaugingMeasurementCorrectness.walkParityVector_well_defined}
\lean{QEC1.GaugingMeasurementCorrectness.walkParityVector_well_defined}
\leanok
\uses{def:GaugingMeasurementCorrectness.walkParityVector, def:GaugingMeasurement.walkParity, thm:GaugingMeasurement.walkParity_well_defined}
If all closed walks have zero parity, then the walk parity vector is independent of the choice of walks: for any two families of walks $\gamma^{(1)}_v, \gamma^{(2)}_v$ from $v_0$ to $v$,
\[
  \operatorname{walkParityVector}(\gamma^{(1)}, \omega) = \operatorname{walkParityVector}(\gamma^{(2)}, \omega).
\]
\end{theorem}

\begin{proof}
\leanok
\uses{def:GaugingMeasurementCorrectness.walkParityVector, thm:GaugingMeasurement.walkParity_well_defined}
By extensionality: for each vertex $v$, we apply the walk parity well-definedness theorem (\texttt{walkParity\_well\_defined}), which shows that under the closed-walk-zero-parity hypothesis, $\operatorname{walkParity}(\omega, \gamma^{(1)}_v) = \operatorname{walkParity}(\omega, \gamma^{(2)}_v)$.
\end{proof}

\begin{theorem}[Gauging Implements Projective Measurement]
\label{thm:GaugingMeasurementCorrectness.gauging_implements_projective_measurement}
\lean{QEC1.GaugingMeasurementCorrectness.gauging_implements_projective_measurement}
\leanok
\uses{def:GaugingMeasurementCorrectness.ProjectiveMeasurementOfL, def:GaugingMeasurement.GaugingOutcomes, def:GaugingMeasurement.measurementSign, def:GaugingMeasurement.byproductCorrectionOp, def:GaugingMeasurement.walkParity, def:GaussFlux.gaussLawOp, def:GaussFlux.logicalOp, def:GraphMaps.coboundaryMap, def:GraphMaps.allOnes, def:PauliOp.PauliCommute, thm:GaussFlux.gaussLaw_product, thm:GaugingMeasurement.byproductCorrectionOp_mul_self, thm:GaugingMeasurement.walkParity_well_defined}
\textbf{(Theorem 1: Gauging Measurement Correctness.)}
For a connected graph $G$, the gauging measurement procedure implements a projective measurement of the logical operator $L = \prod_{v \in V} X_v$.

Given an initial code state $|\psi\rangle$, the procedure outputs $(\sigma, |\Psi\rangle)$ where:
\begin{itemize}
  \item $\sigma = \sum_v \varepsilon_v \in \mathbb{Z}/2\mathbb{Z}$ encodes the measurement outcome ($+1$ or $-1$),
  \item The output state satisfies $|\Psi\rangle \propto (\mathbf{1} + \sigma L)|\psi\rangle$, the projection onto the $\sigma$-eigenspace of $L$.
\end{itemize}
\end{theorem}

\begin{proof}
\leanok
\uses{def:GaugingMeasurementCorrectness.ProjectiveMeasurementOfL, thm:GaussFlux.gaussLaw_product, thm:GaugingMeasurement.byproductCorrectionOp_mul_self, thm:GaugingMeasurement.walkParity_well_defined}
We verify each condition of the \texttt{ProjectiveMeasurementOfL} structure:

\textbf{Sign determination}: The measurement sign $\sigma$ equals $\sum_v \varepsilon_v$ by definition (reflexivity).

\textbf{Gauss product}: $\prod_{v \in V} A_v = L$ follows directly from the Gauss law product theorem.

\textbf{Preimage structure}: For connected $G$, if $\delta c' = \omega$ and $\delta c = \omega$, then $c = c'$ or $c = c' + \mathbf{1}$, as established in the coboundary preimage theorem for connected graphs.

\textbf{Two terms}: $\operatorname{gaussSubsetProduct}(c' + \mathbf{1}) = \operatorname{gaussSubsetProduct}(c') \cdot L$ follows from the shift identity.

\textbf{Signed outcome additivity}: $\varepsilon(c' + \mathbf{1}) = \varepsilon(c') + \sigma$ follows from the signed outcome shift.

\textbf{Walk parity preimage}: Under the closed-walk-zero-parity condition, $\delta c' = \omega$ follows from the walk parity coboundary preimage theorem.

\textbf{Correction pure $X$}: The $Z$-vector of the correction is $0$ by definition (reflexivity).

\textbf{Correction commutes with $L|_V$}: Both operators are pure $X$-type, so their symplectic inner product vanishes.

\textbf{Correction self-inverse}: This follows from \texttt{byproductCorrectionOp\_mul\_self}.

\textbf{Walk independence}: Under the closed-walk-zero-parity condition, the walk parity vector is well-defined by \texttt{walkParityVector\_well\_defined}, which itself relies on \texttt{walkParity\_well\_defined}.
\end{proof}

\begin{theorem}[Effective Vertex Operator Equals Projector]
\label{thm:GaugingMeasurementCorrectness.effective_vertex_operator_eq_projector}
\lean{QEC1.GaugingMeasurementCorrectness.effective_vertex_operator_eq_projector}
\leanok
\uses{def:GaugingMeasurementCorrectness.gaussSubsetProduct, def:GaugingMeasurementCorrectness.signedOutcome, def:GraphMaps.allOnes}
The effective vertex operator after the gauging procedure equals the projector $\mathbf{1} + \sigma L$. Specifically, for any binary vectors $c'$ (the coboundary preimage) and $\varepsilon$ (the Gauss outcomes), the following hold simultaneously:
\begin{enumerate}
  \item $\operatorname{gaussSubsetProduct}(c')$ restricted to vertex $X$-vectors equals $c'$.
  \item $\operatorname{gaussSubsetProduct}(c' + \mathbf{1})$ restricted to vertex $X$-vectors equals $c' + \mathbf{1}$.
  \item After byproduct correction (adding $c'$), the identity term gives $c'(v) + c'(v) = 0$ for all $v$.
  \item After byproduct correction, the $L$ term gives $(c' + \mathbf{1})(v) + c'(v) = 1$ for all $v$.
  \item The signed outcomes satisfy $\varepsilon(c' + \mathbf{1}) = \varepsilon(c') + \sigma$.
  \item Both terms are pure $X$-type (have $Z$-vector $0$).
\end{enumerate}
\end{theorem}

\begin{proof}
\leanok
\uses{def:GaugingMeasurementCorrectness.gaussSubsetProduct, def:GaugingMeasurementCorrectness.signedOutcome, def:GraphMaps.allOnes}
We verify each component:

\textbf{(1)} By extensionality and simplification of the Gauss subset product definition, the vertex component at $v$ is $c'(v)$.

\textbf{(2)} By extensionality and simplification, the vertex component at $v$ is $(c' + \mathbf{1})(v) = c'(v) + 1$.

\textbf{(3)} For each $v$, $c'(v) + c'(v) = 0$ by the characteristic-$2$ identity $a + a = 0$.

\textbf{(4)} For each $v$, $(c'(v) + 1) + c'(v) = 1 + c'(v) + c'(v) = 1 + 0 = 1$, using commutativity of addition, associativity, and the characteristic-$2$ identity.

\textbf{(5)} This is the signed outcome shift identity: $\varepsilon(c' + \mathbf{1}) = \varepsilon(c') + \sum_v \varepsilon(v)$.

\textbf{(6)} Both $Z$-vectors are $0$ by definition of the Gauss subset product.
\end{proof}

\begin{theorem}[Logical Operator is Self-Inverse]
\label{thm:GaugingMeasurementCorrectness.logical_self_inverse}
\lean{QEC1.GaugingMeasurementCorrectness.logical_self_inverse}
\leanok
\uses{def:GaussFlux.logicalOp, def:GaussFlux.ExtQubit, def:PauliOp}
The logical operator is self-inverse: $L \cdot L = \mathbf{1}$.
\end{theorem}

\begin{proof}
\leanok
\uses{def:GaussFlux.logicalOp, def:PauliOp}
This follows from the general Pauli self-inverse property: any Pauli operator $P$ satisfies $P \cdot P = \mathbf{1}$ in $\mathbb{Z}/2\mathbb{Z}$ arithmetic.
\end{proof}

\begin{theorem}[Projector Idempotency Components]
\label{thm:GaugingMeasurementCorrectness.projector_idempotent_components}
\lean{QEC1.GaugingMeasurementCorrectness.projector_idempotent_components}
\leanok
\uses{def:GaussFlux.logicalOp, def:GaussFlux.ExtQubit, def:PauliOp}
The projector $(\mathbf{1} + \sigma L)/2$ is idempotent. At the level of Pauli algebra components:
\[
  L \cdot L = \mathbf{1}, \quad \mathbf{1} \cdot L = L, \quad L \cdot \mathbf{1} = L.
\]
Hence $(\mathbf{1} + \sigma L) \cdot (\mathbf{1} + \sigma L) = 2(\mathbf{1} + \sigma L)$ since $\sigma^2 = 1$ in $\mathbb{Z}/2\mathbb{Z}$.
\end{theorem}

\begin{proof}
\leanok
\uses{def:GaussFlux.logicalOp, def:PauliOp}
The three identities follow from the Pauli self-inverse property ($L \cdot L = \mathbf{1}$), the left identity law ($\mathbf{1} \cdot L = L$), and the right identity law ($L \cdot \mathbf{1} = L$).
\end{proof}

\begin{theorem}[Sign Dichotomy]
\label{thm:GaugingMeasurementCorrectness.sign_dichotomy'}
\lean{QEC1.GaugingMeasurementCorrectness.sign_dichotomy'}
\leanok
\uses{def:GaugingMeasurement.GaugingOutcomes, def:GaugingMeasurement.signIsPlus, def:GaugingMeasurement.signIsMinus, def:GaugingMeasurement.measurementSign}
The measurement sign is always either $+1$ or $-1$: for any gauging outcomes,
\[
  \operatorname{signIsPlus}(\varepsilon) \;\lor\; \operatorname{signIsMinus}(\varepsilon).
\]
\end{theorem}

\begin{proof}
\leanok
\uses{def:GaugingMeasurement.signIsPlus, def:GaugingMeasurement.signIsMinus, thm:GaugingMeasurement.sign_dichotomy}
This follows directly from the sign dichotomy theorem (\texttt{sign\_dichotomy}).
\end{proof}

\begin{theorem}[Gauss Law Measurements All Commute]
\label{thm:GaugingMeasurementCorrectness.gaussLaw_all_commute}
\lean{QEC1.GaugingMeasurementCorrectness.gaussLaw_all_commute}
\leanok
\uses{def:GaugingMeasurement.gaussLawOperatorMeasured, def:PauliOp.PauliCommute, thm:GaugingMeasurement.gaussLaw_measured_commute}
All Gauss's law measurements are mutually compatible: for any vertices $v, w \in V$,
\[
  [A_v^{\mathrm{meas}},\; A_w^{\mathrm{meas}}] = 0.
\]
\end{theorem}

\begin{proof}
\leanok
\uses{thm:GaugingMeasurement.gaussLaw_measured_commute}
This follows directly from \texttt{gaussLaw\_measured\_commute}.
\end{proof}

\begin{theorem}[Edge $Z$ Measurements All Commute]
\label{thm:GaugingMeasurementCorrectness.edgeZ_all_commute}
\lean{QEC1.GaugingMeasurementCorrectness.edgeZ_all_commute}
\leanok
\uses{def:GaugingMeasurement.edgeZOperatorMeasured, def:PauliOp.PauliCommute, thm:GaugingMeasurement.edgeZ_measured_commute}
All edge $Z$ measurements are mutually compatible: for any edges $e_1, e_2 \in E(G)$,
\[
  [Z_{e_1}^{\mathrm{meas}},\; Z_{e_2}^{\mathrm{meas}}] = 0.
\]
\end{theorem}

\begin{proof}
\leanok
\uses{thm:GaugingMeasurement.edgeZ_measured_commute}
This follows directly from \texttt{edgeZ\_measured\_commute}.
\end{proof}

\begin{theorem}[Correction is Walk-Independent]
\label{thm:GaugingMeasurementCorrectness.correction_walk_independent}
\lean{QEC1.GaugingMeasurementCorrectness.correction_walk_independent}
\leanok
\uses{def:GaugingMeasurement.GaugingInput, def:GaugingMeasurement.GaugingOutcomes, def:GaugingMeasurement.gaugingAlgorithm, def:GaugingMeasurement.walkParity, thm:GaugingMeasurement.gaugingAlgorithm_correction_independent}
The gauging algorithm is well-defined: the correction operator is independent of the choice of walks. For any two families of walks $\gamma^{(1)}, \gamma^{(2)}$ from the base vertex, under the closed-walk-zero-parity condition:
\[
  \operatorname{gaugingAlgorithm}(\gamma^{(1)}).\mathrm{correction} = \operatorname{gaugingAlgorithm}(\gamma^{(2)}).\mathrm{correction}.
\]
\end{theorem}

\begin{proof}
\leanok
\uses{thm:GaugingMeasurement.gaugingAlgorithm_correction_independent}
This follows directly from \texttt{gaugingAlgorithm\_correction\_independent}.
\end{proof}

%--- Rem_9: CircuitImplementation ---

I'll start by reading the Lean file to understand its contents.Now I have the full file. Let me produce the LaTeX translation.

\chapter{Rem 9: Circuit Implementation}

The gauging measurement procedure can be implemented by a quantum circuit with no additional qubits beyond the edge qubits. The protocol proceeds as follows:
\begin{enumerate}
\item Initialize each edge qubit $e$ in state $|0\rangle_e$.
\item Apply the entangling circuit $\prod_v \prod_{e \ni v} \mathrm{CX}_{v \to e}$.
\item Measure $X_v$ on all vertex qubits $v \in V$.
\item Apply the same entangling circuit again.
\item Measure $Z_e$ on all edge qubits and discard them.
\item Apply byproduct corrections as in Definition~5.
\end{enumerate}

The key insight is that the entangling circuit transforms $A_v = X_v \prod_{e \ni v} X_e$ into $X_v$, since $\mathrm{CX}_{v \to e}$ conjugates $X_v \mapsto X_v X_e$ and $X_e \mapsto X_e$. Thus measuring $X_v$ after the circuit is equivalent to measuring $A_v$ before it.

\section{CX Gate Conjugation on Pauli Operators}

\begin{definition}[CX Conjugation]
\label{def:CircuitImplementation.cxConjugate}
\lean{QEC1.CircuitImplementation.cxConjugate}
\leanok
\uses{def:PauliOp}
Given a qubit space $W$, a control qubit $c \in W$, a target qubit $t \in W$ with $c \neq t$, and a Pauli operator $P \in \operatorname{PauliOp}(W)$, the \emph{CX conjugation} $\mathrm{CX}_{c \to t}(P)$ is the Pauli operator defined by:
\[
\mathrm{CX}_{c \to t}(P).\mathrm{xVec}(q) = \begin{cases} P.\mathrm{xVec}(q) + P.\mathrm{xVec}(c) & \text{if } q = t, \\ P.\mathrm{xVec}(q) & \text{otherwise,} \end{cases}
\]
\[
\mathrm{CX}_{c \to t}(P).\mathrm{zVec}(q) = \begin{cases} P.\mathrm{zVec}(q) + P.\mathrm{zVec}(t) & \text{if } q = c, \\ P.\mathrm{zVec}(q) & \text{otherwise.} \end{cases}
\]
\end{definition}

\begin{theorem}[CX Conjugation is Involutive]
\label{thm:CircuitImplementation.cxConjugate_involutive}
\lean{QEC1.CircuitImplementation.cxConjugate_involutive}
\leanok
\uses{def:CircuitImplementation.cxConjugate, def:PauliOp}
For any qubit space $W$, control $c$, target $t$ with $c \neq t$, and Pauli operator $P \in \operatorname{PauliOp}(W)$,
\[
\mathrm{CX}_{c \to t}(\mathrm{CX}_{c \to t}(P)) = P.
\]
\end{theorem}

\begin{proof}
\leanok
\uses{def:CircuitImplementation.cxConjugate}
By extensionality, it suffices to show equality of $\mathrm{xVec}$ and $\mathrm{zVec}$ for each qubit $q$. For the $\mathrm{xVec}$ component: if $q = t$, then by the definition of CX conjugation,
\[
\mathrm{CX}(\mathrm{CX}(P)).\mathrm{xVec}(t) = (P.\mathrm{xVec}(t) + P.\mathrm{xVec}(c)) + P.\mathrm{xVec}(c) = P.\mathrm{xVec}(t)
\]
since $P.\mathrm{xVec}(c) + P.\mathrm{xVec}(c) = 0$ in $\mathbb{Z}/2\mathbb{Z}$ by the characteristic-two property. For $q \neq t$, $\mathrm{CX}$ acts as the identity, so the result holds by reflexivity.

For the $\mathrm{zVec}$ component: if $q = c$, then
\[
\mathrm{CX}(\mathrm{CX}(P)).\mathrm{zVec}(c) = (P.\mathrm{zVec}(c) + P.\mathrm{zVec}(t)) + P.\mathrm{zVec}(t) = P.\mathrm{zVec}(c)
\]
by the same characteristic-two cancellation. For $q \neq c$, the result holds by reflexivity.
\end{proof}

\begin{theorem}[CX Conjugation Distributes over Multiplication]
\label{thm:CircuitImplementation.cxConjugate_mul}
\lean{QEC1.CircuitImplementation.cxConjugate_mul}
\leanok
\uses{def:CircuitImplementation.cxConjugate, def:PauliOp, def:PauliOp.mul}
For any Pauli operators $P, Q \in \operatorname{PauliOp}(W)$,
\[
\mathrm{CX}_{c \to t}(P \cdot Q) = \mathrm{CX}_{c \to t}(P) \cdot \mathrm{CX}_{c \to t}(Q).
\]
\end{theorem}

\begin{proof}
\leanok
\uses{def:CircuitImplementation.cxConjugate, def:PauliOp.mul}
By extensionality on each qubit $q$. For the $\mathrm{xVec}$ component, we unfold the definitions of CX conjugation and Pauli multiplication. In each case (whether $q = t$ or $q \neq t$), the equality reduces to a ring identity in $\mathbb{Z}/2\mathbb{Z}$. The $\mathrm{zVec}$ component follows by the analogous argument, splitting on whether $q = c$ or $q \neq c$ and verifying the ring identity in each case.
\end{proof}

\begin{theorem}[CX Maps $X_c$ to $X_c \cdot X_t$]
\label{thm:CircuitImplementation.cxConjugate_pauliX_control}
\lean{QEC1.CircuitImplementation.cxConjugate_pauliX_control}
\leanok
\uses{def:CircuitImplementation.cxConjugate, def:PauliOp.pauliX, def:PauliOp.mul}
For $c \neq t$,
\[
\mathrm{CX}_{c \to t}(X_c) = X_c \cdot X_t.
\]
\end{theorem}

\begin{proof}
\leanok
\uses{def:CircuitImplementation.cxConjugate, def:PauliOp.pauliX, def:PauliOp.mul}
By extensionality on each qubit $q$. For the $\mathrm{xVec}$ component: if $q = t$, then the CX conjugation adds $(\operatorname{pauliX} c).\mathrm{xVec}(c) = 1$ to $(\operatorname{pauliX} c).\mathrm{xVec}(t) = 0$, giving $1$. The product $X_c \cdot X_t$ has $\mathrm{xVec}(t) = 0 + 1 = 1$, so they agree. For $q \neq t$, CX acts as the identity and the values agree by simplification using $c \neq t$. For the $\mathrm{zVec}$ component, $\operatorname{pauliX}$ has zero $\mathrm{zVec}$ everywhere, so both sides reduce to $0$ by simplification.
\end{proof}

\begin{theorem}[CX Fixes $X_t$]
\label{thm:CircuitImplementation.cxConjugate_pauliX_target}
\lean{QEC1.CircuitImplementation.cxConjugate_pauliX_target}
\leanok
\uses{def:CircuitImplementation.cxConjugate, def:PauliOp.pauliX}
For $c \neq t$,
\[
\mathrm{CX}_{c \to t}(X_t) = X_t.
\]
\end{theorem}

\begin{proof}
\leanok
\uses{def:CircuitImplementation.cxConjugate, def:PauliOp.pauliX}
By extensionality on each qubit $q$. For $\mathrm{xVec}$: if $q = t$, then $\mathrm{CX}_{c \to t}(X_t).\mathrm{xVec}(t) = (\operatorname{pauliX} t).\mathrm{xVec}(t) + (\operatorname{pauliX} t).\mathrm{xVec}(c)$. Since $c \neq t$, we have $(\operatorname{pauliX} t).\mathrm{xVec}(c) = 0$, so this equals $(\operatorname{pauliX} t).\mathrm{xVec}(t)$. For $q \neq t$, CX acts as the identity. For $\mathrm{zVec}$: $\operatorname{pauliX}$ has zero $\mathrm{zVec}$, so both sides are $0$.
\end{proof}

\begin{theorem}[CX Fixes $Z_c$]
\label{thm:CircuitImplementation.cxConjugate_pauliZ_control}
\lean{QEC1.CircuitImplementation.cxConjugate_pauliZ_control}
\leanok
\uses{def:CircuitImplementation.cxConjugate, def:PauliOp.pauliZ}
For $c \neq t$,
\[
\mathrm{CX}_{c \to t}(Z_c) = Z_c.
\]
\end{theorem}

\begin{proof}
\leanok
\uses{def:CircuitImplementation.cxConjugate, def:PauliOp.pauliZ}
By extensionality on each qubit $q$. For $\mathrm{xVec}$: $\operatorname{pauliZ}$ has zero $\mathrm{xVec}$, so both sides are $0$. For $\mathrm{zVec}$: if $q = c$, then $\mathrm{CX}_{c \to t}(Z_c).\mathrm{zVec}(c) = (\operatorname{pauliZ} c).\mathrm{zVec}(c) + (\operatorname{pauliZ} c).\mathrm{zVec}(t)$. Since $c \neq t$, we have $(\operatorname{pauliZ} c).\mathrm{zVec}(t) = 0$, and the result follows by simplification. For $q \neq c$, CX acts as the identity.
\end{proof}

\begin{theorem}[CX Maps $Z_t$ to $Z_c \cdot Z_t$]
\label{thm:CircuitImplementation.cxConjugate_pauliZ_target}
\lean{QEC1.CircuitImplementation.cxConjugate_pauliZ_target}
\leanok
\uses{def:CircuitImplementation.cxConjugate, def:PauliOp.pauliZ, def:PauliOp.mul}
For $c \neq t$,
\[
\mathrm{CX}_{c \to t}(Z_t) = Z_c \cdot Z_t.
\]
\end{theorem}

\begin{proof}
\leanok
\uses{def:CircuitImplementation.cxConjugate, def:PauliOp.pauliZ, def:PauliOp.mul}
By extensionality on each qubit $q$. For $\mathrm{xVec}$: $\operatorname{pauliZ}$ has zero $\mathrm{xVec}$, so both sides are $0$. For $\mathrm{zVec}$: if $q = c$, then $\mathrm{CX}_{c \to t}(Z_t).\mathrm{zVec}(c) = (\operatorname{pauliZ} t).\mathrm{zVec}(c) + (\operatorname{pauliZ} t).\mathrm{zVec}(t)$. Since $c \neq t$, this is $0 + 1 = 1$, and $(Z_c \cdot Z_t).\mathrm{zVec}(c) = 1 + 0 = 1$. For $q \neq c$, we simplify using the condition $q \neq c$.
\end{proof}

\section{The Full Entangling Circuit Action}

\begin{definition}[Entangling Circuit Action]
\label{def:CircuitImplementation.entanglingCircuitAction}
\lean{QEC1.CircuitImplementation.entanglingCircuitAction}
\leanok
\uses{def:PauliOp, def:GaussFlux.ExtQubit, def:GaussFlux.incidentEdges}
Given a graph $G$ on vertices $V$ and a Pauli operator $P \in \operatorname{PauliOp}(\operatorname{ExtQubit}(G))$, the \emph{entangling circuit action} is the Pauli operator defined by:
\[
\mathrm{EC}(P).\mathrm{xVec}(q) = \begin{cases} P.\mathrm{xVec}(v) & \text{if } q = \operatorname{inl}(v) \text{ for } v \in V, \\ P.\mathrm{xVec}(e) + \sum_{v \in V} [\![v \in e]\!] \cdot P.\mathrm{xVec}(v) & \text{if } q = \operatorname{inr}(e) \text{ for } e \in E, \end{cases}
\]
\[
\mathrm{EC}(P).\mathrm{zVec}(q) = \begin{cases} P.\mathrm{zVec}(v) + \sum_{e \in \operatorname{inc}(v)} P.\mathrm{zVec}(e) & \text{if } q = \operatorname{inl}(v) \text{ for } v \in V, \\ P.\mathrm{zVec}(e) & \text{if } q = \operatorname{inr}(e) \text{ for } e \in E, \end{cases}
\]
where $[\![v \in e]\!]$ denotes the indicator for vertex $v$ being an endpoint of edge $e$, and $\operatorname{inc}(v)$ is the set of edges incident to $v$.
\end{definition}

\begin{theorem}[Entangling Circuit is Involutive]
\label{thm:CircuitImplementation.entanglingCircuitAction_involutive}
\lean{QEC1.CircuitImplementation.entanglingCircuitAction_involutive}
\leanok
\uses{def:CircuitImplementation.entanglingCircuitAction, def:PauliOp, def:GaussFlux.ExtQubit}
For any Pauli operator $P \in \operatorname{PauliOp}(\operatorname{ExtQubit}(G))$,
\[
\mathrm{EC}(\mathrm{EC}(P)) = P.
\]
That is, applying the entangling circuit twice gives the identity.
\end{theorem}

\begin{proof}
\leanok
\uses{def:CircuitImplementation.entanglingCircuitAction}
By extensionality on each qubit $q$. For the $\mathrm{xVec}$ component, we consider two cases. If $q = \operatorname{inl}(w)$ for a vertex $w$, then $\mathrm{EC}(\mathrm{EC}(P)).\mathrm{xVec}(\operatorname{inl}(w)) = \mathrm{EC}(P).\mathrm{xVec}(\operatorname{inl}(w)) = P.\mathrm{xVec}(\operatorname{inl}(w))$ by the vertex rule applied twice, and the result follows by simplification. If $q = \operatorname{inr}(e)$ for an edge $e$, then we use the edge rule twice: the inner application adds a vertex sum, and the outer application adds the same vertex sum again. Since the vertex rule preserves vertex values, we get $P.\mathrm{xVec}(\operatorname{inr}(e)) + S + S$ where $S$ is the vertex sum. By the characteristic-two property $S + S = 0$ in $\mathbb{Z}/2\mathbb{Z}$, so this equals $P.\mathrm{xVec}(\operatorname{inr}(e))$.

For the $\mathrm{zVec}$ component: if $q = \operatorname{inl}(w)$, the vertex rule applies twice, adding the sum over incident edges in each application. Since the edge rule preserves edge values, we get $P.\mathrm{zVec}(\operatorname{inl}(w)) + T + T$ where $T = \sum_{e \in \operatorname{inc}(w)} P.\mathrm{zVec}(\operatorname{inr}(e))$. Again $T + T = 0$, so the result follows. If $q = \operatorname{inr}(e)$, the edge rule gives the identity directly.
\end{proof}

\begin{theorem}[Entangling Circuit Distributes over Multiplication]
\label{thm:CircuitImplementation.entanglingCircuitAction_mul}
\lean{QEC1.CircuitImplementation.entanglingCircuitAction_mul}
\leanok
\uses{def:CircuitImplementation.entanglingCircuitAction, def:PauliOp, def:PauliOp.mul, def:GaussFlux.ExtQubit}
For any Pauli operators $P, Q \in \operatorname{PauliOp}(\operatorname{ExtQubit}(G))$,
\[
\mathrm{EC}(P \cdot Q) = \mathrm{EC}(P) \cdot \mathrm{EC}(Q).
\]
\end{theorem}

\begin{proof}
\leanok
\uses{def:CircuitImplementation.entanglingCircuitAction, def:PauliOp.mul}
By extensionality on each qubit $q$. For the $\mathrm{xVec}$ component with $q = \operatorname{inl}(w)$: both sides reduce to $P.\mathrm{xVec}(\operatorname{inl}(w)) + Q.\mathrm{xVec}(\operatorname{inl}(w))$ by the vertex rule. For $q = \operatorname{inr}(e)$: the left-hand side is $(P.\mathrm{xVec}(e) + Q.\mathrm{xVec}(e)) + \sum_v [\![v \in e]\!] (P.\mathrm{xVec}(v) + Q.\mathrm{xVec}(v))$, and the right-hand side is $(P.\mathrm{xVec}(e) + \sum_v [\![v \in e]\!] P.\mathrm{xVec}(v)) + (Q.\mathrm{xVec}(e) + \sum_v [\![v \in e]\!] Q.\mathrm{xVec}(v))$. Using the ring identity $(a + b) + (c + d) = (a + c) + (b + d)$ and distributing the sum, the equality follows.

For the $\mathrm{zVec}$ component with $q = \operatorname{inl}(w)$: both sides expand using the vertex rule, and the equality follows by the same rearrangement and distributivity of the sum over incident edges. For $q = \operatorname{inr}(e)$: both sides reduce to $P.\mathrm{zVec}(\operatorname{inr}(e)) + Q.\mathrm{zVec}(\operatorname{inr}(e))$ by the edge rule.
\end{proof}

\section{Symplectic Inner Product Preservation}

\begin{theorem}[Entangling Circuit Preserves Symplectic Inner Product]
\label{thm:CircuitImplementation.entanglingCircuit_preserves_symplecticInner}
\lean{QEC1.CircuitImplementation.entanglingCircuit_preserves_symplecticInner}
\leanok
\uses{def:CircuitImplementation.entanglingCircuitAction, def:PauliOp, def:PauliOp.symplecticInner, def:GaussFlux.ExtQubit, def:GaussFlux.incidentEdges}
For any Pauli operators $P, Q \in \operatorname{PauliOp}(\operatorname{ExtQubit}(G))$,
\[
\langle \mathrm{EC}(P), \mathrm{EC}(Q) \rangle_{\mathrm{symp}} = \langle P, Q \rangle_{\mathrm{symp}}.
\]
\end{theorem}

\begin{proof}
\leanok
\uses{def:CircuitImplementation.entanglingCircuitAction, def:PauliOp.symplecticInner, def:GaussFlux.incidentEdges}
We expand the symplectic inner product $\langle \mathrm{EC}(P), \mathrm{EC}(Q) \rangle_{\mathrm{symp}} = \sum_q \mathrm{EC}(P).\mathrm{xVec}(q) \cdot \mathrm{EC}(Q).\mathrm{zVec}(q) + \mathrm{EC}(P).\mathrm{zVec}(q) \cdot \mathrm{EC}(Q).\mathrm{xVec}(q)$ and split the sum over qubit type (vertices and edges).

For the vertex contribution, each term expands to $P.\mathrm{xVec}(v) \cdot (Q.\mathrm{zVec}(v) + \sum_{e \in \operatorname{inc}(v)} Q.\mathrm{zVec}(e)) + (P.\mathrm{zVec}(v) + \sum_{e \in \operatorname{inc}(v)} P.\mathrm{zVec}(e)) \cdot Q.\mathrm{xVec}(v)$. We rearrange this as the ``main'' term $P.\mathrm{xVec}(v) \cdot Q.\mathrm{zVec}(v) + P.\mathrm{zVec}(v) \cdot Q.\mathrm{xVec}(v)$ plus a ``cross'' term involving the sums over incident edges.

For the edge contribution, each term expands to $(P.\mathrm{xVec}(e) + \sum_v [\![v \in e]\!] P.\mathrm{xVec}(v)) \cdot Q.\mathrm{zVec}(e) + P.\mathrm{zVec}(e) \cdot (Q.\mathrm{xVec}(e) + \sum_v [\![v \in e]\!] Q.\mathrm{xVec}(v))$. We similarly separate this into a main term $P.\mathrm{xVec}(e) \cdot Q.\mathrm{zVec}(e) + P.\mathrm{zVec}(e) \cdot Q.\mathrm{xVec}(e)$ and a cross term.

The main terms sum to $\langle P, Q \rangle_{\mathrm{symp}}$, so it suffices to show that the total cross terms vanish.

We expand the vertex cross terms: $\sum_v \sum_{e \in \operatorname{inc}(v)} (P.\mathrm{xVec}(v) \cdot Q.\mathrm{zVec}(e) + P.\mathrm{zVec}(e) \cdot Q.\mathrm{xVec}(v))$ by distributing the multiplication over the sums. Similarly, the edge cross terms expand to $\sum_e \sum_v [\![v \in e]\!] (P.\mathrm{xVec}(v) \cdot Q.\mathrm{zVec}(e) + P.\mathrm{zVec}(e) \cdot Q.\mathrm{xVec}(v))$.

We rewrite the vertex cross sum by converting the sum over $e \in \operatorname{inc}(v)$ into a sum over all edges with an indicator function $[\![v \in e]\!]$, then swap the summation order. After the swap, the vertex cross sum equals the edge cross sum. Since both are the same element of $\mathbb{Z}/2\mathbb{Z}$, their sum is $X + X = 0$ by the characteristic-two property.
\end{proof}

\begin{theorem}[Entangling Circuit Preserves Commutation]
\label{thm:CircuitImplementation.entanglingCircuit_preserves_commutation}
\lean{QEC1.CircuitImplementation.entanglingCircuit_preserves_commutation}
\leanok
\uses{def:CircuitImplementation.entanglingCircuitAction, def:PauliOp, def:PauliOp.PauliCommute, def:PauliOp.symplecticInner}
For any Pauli operators $P, Q \in \operatorname{PauliOp}(\operatorname{ExtQubit}(G))$,
\[
\operatorname{PauliCommute}(\mathrm{EC}(P), \mathrm{EC}(Q)) \iff \operatorname{PauliCommute}(P, Q).
\]
\end{theorem}

\begin{proof}
\leanok
\uses{thm:CircuitImplementation.entanglingCircuit_preserves_symplecticInner, def:PauliOp.PauliCommute}
We unfold the definition of $\operatorname{PauliCommute}$ in terms of the symplectic inner product, and rewrite using the fact that the entangling circuit preserves the symplectic inner product.
\end{proof}

\section{Main Theorem: Entangling Circuit Transforms $A_v$ to $X_v$}

\begin{theorem}[Entangling Circuit Transforms $A_v$ to $X_v$]
\label{thm:CircuitImplementation.entanglingCircuit_transforms_gaussLaw}
\lean{QEC1.CircuitImplementation.entanglingCircuit_transforms_gaussLaw}
\leanok
\uses{def:CircuitImplementation.entanglingCircuitAction, def:GaussFlux.gaussLawOp, def:PauliOp.pauliX, def:GaussFlux.ExtQubit}
For every vertex $v \in V$,
\[
\mathrm{EC}(A_v) = X_v,
\]
where $A_v = \operatorname{gaussLawOp}(G, v)$ is the Gauss law operator and $X_v = \operatorname{pauliX}(\operatorname{inl}(v))$.
\end{theorem}

\begin{proof}
\leanok
\uses{def:CircuitImplementation.entanglingCircuitAction, def:GaussFlux.gaussLawOp, def:PauliOp.pauliX}
By extensionality on each qubit $q$.

For the $\mathrm{xVec}$ component: if $q = \operatorname{inl}(w)$ for a vertex $w$, then $\mathrm{EC}(A_v).\mathrm{xVec}(\operatorname{inl}(w))$ equals $A_v.\mathrm{xVec}(\operatorname{inl}(w))$ by the vertex rule, which equals $[\![w = v]\!]$ by the definition of $A_v$, matching $(\operatorname{pauliX}(\operatorname{inl}(v))).\mathrm{xVec}(\operatorname{inl}(w))$.

If $q = \operatorname{inr}(e)$ for an edge $e$, then $\mathrm{EC}(A_v).\mathrm{xVec}(\operatorname{inr}(e))$ equals $A_v.\mathrm{xVec}(\operatorname{inr}(e)) + \sum_w [\![w \in e]\!] \cdot A_v.\mathrm{xVec}(\operatorname{inl}(w))$. By the definition of $A_v$, the first term is $[\![v \in e]\!]$ and the sum $\sum_w [\![w \in e]\!] \cdot [\![w = v]\!]$ simplifies to $[\![v \in e]\!]$ (by evaluating the sum with the Kronecker delta). Thus the total is $[\![v \in e]\!] + [\![v \in e]\!] = 0$ in $\mathbb{Z}/2\mathbb{Z}$, matching the right-hand side $(\operatorname{pauliX}(\operatorname{inl}(v))).\mathrm{xVec}(\operatorname{inr}(e)) = 0$.

For the $\mathrm{zVec}$ component: if $q = \operatorname{inl}(w)$, then since $A_v$ has zero $\mathrm{zVec}$, we get $0 + \sum_{e \in \operatorname{inc}(w)} 0 = 0$, matching $(\operatorname{pauliX}).\mathrm{zVec} = 0$. If $q = \operatorname{inr}(e)$, then $\mathrm{EC}(A_v).\mathrm{zVec}(\operatorname{inr}(e)) = A_v.\mathrm{zVec}(\operatorname{inr}(e)) = 0$.
\end{proof}

\begin{theorem}[Entangling Circuit Transforms $X_v$ to $A_v$]
\label{thm:CircuitImplementation.entanglingCircuit_transforms_pauliX_to_gaussLaw}
\lean{QEC1.CircuitImplementation.entanglingCircuit_transforms_pauliX_to_gaussLaw}
\leanok
\uses{def:CircuitImplementation.entanglingCircuitAction, def:GaussFlux.gaussLawOp, def:PauliOp.pauliX, def:GaussFlux.ExtQubit}
For every vertex $v \in V$,
\[
\mathrm{EC}(X_v) = A_v.
\]
\end{theorem}

\begin{proof}
\leanok
\uses{thm:CircuitImplementation.entanglingCircuit_transforms_gaussLaw, thm:CircuitImplementation.entanglingCircuitAction_involutive}
Rewriting the goal using the fact that $\mathrm{EC}(A_v) = X_v$, it suffices to show $\mathrm{EC}(\mathrm{EC}(A_v)) = A_v$. This follows directly from the involutivity of the entangling circuit action.
\end{proof}

\begin{theorem}[Measuring $X_v$ After Circuit Equals Measuring $A_v$]
\label{thm:CircuitImplementation.measuring_Xv_after_circuit_eq_Av}
\lean{QEC1.CircuitImplementation.measuring_Xv_after_circuit_eq_Av}
\leanok
\uses{def:CircuitImplementation.entanglingCircuitAction, def:GaussFlux.gaussLawOp, def:PauliOp.pauliX, def:GaussFlux.ExtQubit}
For every vertex $v \in V$,
\[
\mathrm{EC}(A_v) = X_v.
\]
That is, measuring $X_v$ in the entangled frame is equivalent to measuring the Gauss law operator $A_v$ in the original frame.
\end{theorem}

\begin{proof}
\leanok
\uses{thm:CircuitImplementation.entanglingCircuit_transforms_gaussLaw}
This follows directly from the theorem that the entangling circuit transforms $A_v$ to $X_v$.
\end{proof}

\section{Effect on Edge $Z$ Operators}

\begin{theorem}[Entangling Circuit Transforms Edge $Z$]
\label{thm:CircuitImplementation.entanglingCircuit_transforms_edgeZ}
\lean{QEC1.CircuitImplementation.entanglingCircuit_transforms_edgeZ}
\leanok
\uses{def:CircuitImplementation.entanglingCircuitAction, def:PauliOp, def:PauliOp.pauliZ, def:GaussFlux.ExtQubit, def:GaussFlux.incidentEdges}
For an edge $e \in E(G)$, the entangling circuit transforms $Z_e$ into the operator with $\mathrm{xVec} = 0$ and
\[
\mathrm{zVec}(q) = \begin{cases} [\![w \in e]\!] & \text{if } q = \operatorname{inl}(w), \\ [\![e' = e]\!] & \text{if } q = \operatorname{inr}(e'). \end{cases}
\]
That is, $\mathrm{EC}(Z_e)$ acts as $Z$ on the edge $e$ and on both its endpoint vertices.
\end{theorem}

\begin{proof}
\leanok
\uses{def:CircuitImplementation.entanglingCircuitAction, def:PauliOp.pauliZ, def:GaussFlux.incidentEdges, lem:GaussFlux.mem_incidentEdges}
By extensionality on each qubit $q$.

For the $\mathrm{xVec}$ component: both vertex and edge cases give $0$, since $\operatorname{pauliZ}$ has zero $\mathrm{xVec}$ and the vertex sums involve zero terms.

For the $\mathrm{zVec}$ component with $q = \operatorname{inl}(w)$: we compute $\mathrm{EC}(Z_e).\mathrm{zVec}(\operatorname{inl}(w)) = (\operatorname{pauliZ}(\operatorname{inr}(e))).\mathrm{zVec}(\operatorname{inl}(w)) + \sum_{e' \in \operatorname{inc}(w)} (\operatorname{pauliZ}(\operatorname{inr}(e))).\mathrm{zVec}(\operatorname{inr}(e'))$. The first term is $0$ (since $\operatorname{inl}(w) \neq \operatorname{inr}(e)$), and the sum $\sum_{e' \in \operatorname{inc}(w)} [\![e' = e]\!]$ equals $[\![e \in \operatorname{inc}(w)]\!]$. By the characterization of incident edges, $e \in \operatorname{inc}(w)$ if and only if $w \in e$, so this gives $[\![w \in e]\!]$ as required.

For the $\mathrm{zVec}$ component with $q = \operatorname{inr}(e')$: by the edge rule, $\mathrm{EC}(Z_e).\mathrm{zVec}(\operatorname{inr}(e')) = (\operatorname{pauliZ}(\operatorname{inr}(e))).\mathrm{zVec}(\operatorname{inr}(e')) = [\![e' = e]\!]$.
\end{proof}

\section{Circuit Protocol Steps}

\begin{theorem}[Step 2: Circuit Transforms $A_v$ to $X_v$]
\label{thm:CircuitImplementation.circuit_step2_transforms_Av_to_Xv}
\lean{QEC1.CircuitImplementation.circuit_step2_transforms_Av_to_Xv}
\leanok
\uses{def:CircuitImplementation.entanglingCircuitAction, def:GaussFlux.gaussLawOp, def:PauliOp.pauliX, def:GaussFlux.ExtQubit}
For every vertex $v \in V$,
\[
\mathrm{EC}(A_v) = X_v.
\]
\end{theorem}

\begin{proof}
\leanok
\uses{thm:CircuitImplementation.entanglingCircuit_transforms_gaussLaw}
This follows directly from the key result that the entangling circuit transforms $A_v$ to $X_v$.
\end{proof}

\begin{theorem}[Step 4: Applying Circuit Twice is Identity]
\label{thm:CircuitImplementation.circuit_step4_undoes_step2}
\lean{QEC1.CircuitImplementation.circuit_step4_undoes_step2}
\leanok
\uses{def:CircuitImplementation.entanglingCircuitAction, def:PauliOp, def:GaussFlux.ExtQubit}
For any Pauli operator $P$,
\[
\mathrm{EC}(\mathrm{EC}(P)) = P.
\]
\end{theorem}

\begin{proof}
\leanok
\uses{thm:CircuitImplementation.entanglingCircuitAction_involutive}
This follows directly from the involutivity of the entangling circuit action.
\end{proof}

\begin{theorem}[Steps 2+3+4: Measuring $X_v$ in CX Frame Equals Measuring $A_v$]
\label{thm:CircuitImplementation.steps234_equivalent_to_measuring_Av}
\lean{QEC1.CircuitImplementation.steps234_equivalent_to_measuring_Av}
\leanok
\uses{def:CircuitImplementation.entanglingCircuitAction, def:GaussFlux.gaussLawOp, def:PauliOp.pauliX, def:GaussFlux.ExtQubit}
For every vertex $v \in V$,
\[
\mathrm{EC}(X_v) = A_v.
\]
\end{theorem}

\begin{proof}
\leanok
\uses{thm:CircuitImplementation.entanglingCircuit_transforms_pauliX_to_gaussLaw}
This follows directly from the inverse direction of the circuit transformation.
\end{proof}

\begin{theorem}[Edge Measurements Return to Original Frame]
\label{thm:CircuitImplementation.circuit_edge_measurement_in_original_frame}
\lean{QEC1.CircuitImplementation.circuit_edge_measurement_in_original_frame}
\leanok
\uses{def:CircuitImplementation.entanglingCircuitAction, def:PauliOp.pauliZ, def:GaussFlux.ExtQubit}
For any edge $e \in E(G)$,
\[
\mathrm{EC}(\mathrm{EC}(Z_e)) = Z_e.
\]
After applying the entangling circuit a second time (step 4), edge $Z$ measurements are back in the original frame.
\end{theorem}

\begin{proof}
\leanok
\uses{thm:CircuitImplementation.entanglingCircuitAction_involutive}
This follows directly from the involutivity of the entangling circuit action applied to $Z_e$.
\end{proof}

\section{Simultaneous Transformation and Consistency}

\begin{theorem}[All Gauss Operators Simultaneously Transform to $X_v$]
\label{thm:CircuitImplementation.entanglingCircuit_transforms_all_gaussLaw}
\lean{QEC1.CircuitImplementation.entanglingCircuit_transforms_all_gaussLaw}
\leanok
\uses{def:CircuitImplementation.entanglingCircuitAction, def:GaussFlux.gaussLawOp, def:PauliOp.pauliX, def:GaussFlux.ExtQubit}
For all vertices $v \in V$ simultaneously,
\[
\mathrm{EC}(A_v) = X_v.
\]
\end{theorem}

\begin{proof}
\leanok
\uses{thm:CircuitImplementation.entanglingCircuit_transforms_gaussLaw}
Let $v$ be an arbitrary vertex. The result follows directly from the key transformation theorem applied to each $v$.
\end{proof}

\begin{theorem}[Product of Transformed Operators Equals Logical Operator]
\label{thm:CircuitImplementation.transformed_gaussLaw_product}
\lean{QEC1.CircuitImplementation.transformed_gaussLaw_product}
\leanok
\uses{def:PauliOp.pauliX, def:GaussFlux.logicalOp, def:GaussFlux.ExtQubit}
The product of all vertex $X$ operators equals the logical operator:
\[
\prod_{v \in V} X_v = L,
\]
where $L = \operatorname{logicalOp}(G)$.
\end{theorem}

\begin{proof}
\leanok
\uses{def:PauliOp.pauliX, def:GaussFlux.logicalOp}
By extensionality on each qubit $q$.

For the $\mathrm{xVec}$ component: we rewrite the product's $\mathrm{xVec}$ as a sum over all vertices using the product formula for Pauli operators. If $q = \operatorname{inl}(w)$ for a vertex $w$, then $\sum_v (\operatorname{pauliX}(\operatorname{inl}(v))).\mathrm{xVec}(\operatorname{inl}(w)) = \sum_v [\![w = v]\!]$. Evaluating using the indicator sum identity $\sum_{v \in V} [\![w = v]\!] = 1$ (since $w \in V$), this gives $1$, which matches $\operatorname{logicalOp}(G).\mathrm{xVec}(\operatorname{inl}(w)) = 1$. If $q = \operatorname{inr}(e)$ for an edge $e$, then $\sum_v (\operatorname{pauliX}(\operatorname{inl}(v))).\mathrm{xVec}(\operatorname{inr}(e)) = 0$ since $\operatorname{inl}(v) \neq \operatorname{inr}(e)$ for all $v$, matching $\operatorname{logicalOp}(G).\mathrm{xVec}(\operatorname{inr}(e)) = 0$.

For the $\mathrm{zVec}$ component: we rewrite the product's $\mathrm{zVec}$ as a sum. Since $\operatorname{pauliX}$ has zero $\mathrm{zVec}$, the sum is $0$ for all qubits, matching $\operatorname{logicalOp}(G).\mathrm{zVec} = 0$.
\end{proof}

\begin{theorem}[Entangling Circuit Preserves Logical Operator]
\label{thm:CircuitImplementation.entanglingCircuit_preserves_logical}
\lean{QEC1.CircuitImplementation.entanglingCircuit_preserves_logical}
\leanok
\uses{def:CircuitImplementation.entanglingCircuitAction, def:GaussFlux.logicalOp, def:GaussFlux.ExtQubit}
The entangling circuit fixes the logical operator:
\[
\mathrm{EC}(L) = L.
\]
\end{theorem}

\begin{proof}
\leanok
\uses{def:CircuitImplementation.entanglingCircuitAction, def:GaussFlux.logicalOp}
By extensionality on each qubit $q$.

For the $\mathrm{xVec}$ component: if $q = \operatorname{inl}(w)$, then $\mathrm{EC}(L).\mathrm{xVec}(\operatorname{inl}(w)) = L.\mathrm{xVec}(\operatorname{inl}(w))$ by the vertex rule and the result follows by simplification. If $q = \operatorname{inr}(e)$ for an edge $e$, then $\mathrm{EC}(L).\mathrm{xVec}(\operatorname{inr}(e)) = L.\mathrm{xVec}(\operatorname{inr}(e)) + \sum_v [\![v \in e]\!] \cdot L.\mathrm{xVec}(\operatorname{inl}(v))$. Since $L.\mathrm{xVec}(\operatorname{inr}(e)) = 0$ and $L.\mathrm{xVec}(\operatorname{inl}(v)) = 1$ for all $v$, this equals $\sum_v [\![v \in e]\!]$. Writing $e = \{a, b\}$ with $a \neq b$ (since $G$ has no loops), we compute $\sum_v [\![v \in e]\!] = [\![a \in e]\!] + [\![b \in e]\!] = 1 + 1 = 0$ in $\mathbb{Z}/2\mathbb{Z}$, matching $L.\mathrm{xVec}(\operatorname{inr}(e)) = 0$.

For the $\mathrm{zVec}$ component: since $L.\mathrm{zVec} = 0$ everywhere, both vertex and edge cases give $0$.
\end{proof}

%--- Rem_10: FlexibilityOfGraphG ---
\chapter{Rem 10: Flexibility of Graph G}

The gauging measurement procedure allows arbitrary choice of the connected graph $G$ with vertex set equal to the support of the logical operator $L$. The properties of the deformed code depend strongly on this choice. In particular:
\begin{enumerate}
\item \textbf{Dummy vertices:} $G$ can have additional vertices beyond the support of $L$, achieved by gauging $L' = L \cdot \prod_{\text{dummy}} X_v$.
\item \textbf{Graph properties:} The choice of $G$ determines $|E|$ (the edge qubit count), deformed check weights (via path lengths), and code distance (via expansion).
\end{enumerate}

%--- Dummy Vertices ---

\begin{definition}[Original Logical Operator]
\label{def:FlexibilityOfGraphG.originalLogicalOp}
\lean{QEC1.FlexibilityOfGraphG.originalLogicalOp}
\leanok
\uses{def:PauliOp}
The original logical operator $L$ on vertex set $V$ is defined as
\[
L = \prod_{v \in V} X_v,
\]
i.e., the Pauli operator with $\mathrm{xVec}(v) = 1$ for all $v \in V$ and $\mathrm{zVec} = 0$.
\end{definition}

\begin{definition}[Extended Logical Operator]
\label{def:FlexibilityOfGraphG.extendedLogicalOp}
\lean{QEC1.FlexibilityOfGraphG.extendedLogicalOp}
\leanok
\uses{def:PauliOp}
The extended logical operator $L'$ on the extended vertex type $V \oplus D$ is defined as
\[
L' = \prod_{q \in V \oplus D} X_q,
\]
i.e., the Pauli operator with $\mathrm{xVec}(q) = 1$ for all $q \in V \oplus D$ and $\mathrm{zVec} = 0$.
\end{definition}

\begin{definition}[Dummy X Product]
\label{def:FlexibilityOfGraphG.dummyXProduct}
\lean{QEC1.FlexibilityOfGraphG.dummyXProduct}
\leanok
\uses{def:PauliOp}
The dummy $X$ product $\prod_{d \in D} X_d$ acting on $V \oplus D$ is the Pauli operator with
\[
\mathrm{xVec}(q) = \begin{cases} 0 & \text{if } q \in V, \\ 1 & \text{if } q \in D, \end{cases}
\qquad \mathrm{zVec} = 0.
\]
\end{definition}

\begin{definition}[Lifted Logical Operator]
\label{def:FlexibilityOfGraphG.liftedLogicalOp}
\lean{QEC1.FlexibilityOfGraphG.liftedLogicalOp}
\leanok
\uses{def:PauliOp}
The lifted logical operator $L_{\mathrm{lift}}$ on $V \oplus D$ acts as $X$ on all $V$-qubits and as the identity on $D$-qubits:
\[
\mathrm{xVec}(q) = \begin{cases} 1 & \text{if } q \in V, \\ 0 & \text{if } q \in D, \end{cases}
\qquad \mathrm{zVec} = 0.
\]
\end{definition}

\begin{theorem}[Key Factorization of Extended Logical]
\label{thm:FlexibilityOfGraphG.extendedLogicalOp_eq_mul}
\lean{QEC1.FlexibilityOfGraphG.extendedLogicalOp_eq_mul}
\leanok
\uses{def:FlexibilityOfGraphG.extendedLogicalOp, def:FlexibilityOfGraphG.liftedLogicalOp, def:FlexibilityOfGraphG.dummyXProduct}
The extended logical operator satisfies
\[
L' = L_{\mathrm{lift}} \cdot \prod_{d \in D} X_d
\]
on $V \oplus D$.
\end{theorem}

\begin{proof}
\leanok
\uses{def:FlexibilityOfGraphG.extendedLogicalOp, def:FlexibilityOfGraphG.liftedLogicalOp, def:FlexibilityOfGraphG.dummyXProduct}
By extensionality, it suffices to show equality for each component for an arbitrary qubit $q$. We consider both the $\mathrm{xVec}$ and $\mathrm{zVec}$ components, and case-split on whether $q \in V$ or $q \in D$. In each case, the equality follows by simplification using the definitions of $L'$, $L_{\mathrm{lift}}$, and the dummy $X$ product, together with the formulas for multiplication of Pauli operators.
\end{proof}

\begin{theorem}[Extended Logical is Self-Inverse]
\label{thm:FlexibilityOfGraphG.extendedLogicalOp_self_inverse}
\lean{QEC1.FlexibilityOfGraphG.extendedLogicalOp_self_inverse}
\leanok
\uses{def:FlexibilityOfGraphG.extendedLogicalOp}
The extended logical operator is self-inverse: $L' \cdot L' = \mathbf{1}$.
\end{theorem}

\begin{proof}
\leanok
\uses{def:FlexibilityOfGraphG.extendedLogicalOp}
This follows directly from the fact that every Pauli operator satisfies $P \cdot P = \mathbf{1}$ (the \texttt{mul\_self} lemma).
\end{proof}

\begin{theorem}[Extended Logical is Pure X]
\label{thm:FlexibilityOfGraphG.extendedLogicalOp_pure_X}
\lean{QEC1.FlexibilityOfGraphG.extendedLogicalOp_pure_X}
\leanok
\uses{def:FlexibilityOfGraphG.extendedLogicalOp}
The extended logical operator $L'$ is pure $X$-type: it has no $Z$-support, i.e., $\mathrm{zVec}(L') = 0$.
\end{theorem}

\begin{proof}
\leanok
\uses{def:FlexibilityOfGraphG.extendedLogicalOp}
By extensionality over qubits $q$, we simplify using the definition of $L'$, which has $\mathrm{zVec} = 0$.
\end{proof}

\begin{theorem}[X-Support of Extended Logical is Universal]
\label{thm:FlexibilityOfGraphG.extendedLogicalOp_supportX_eq_univ}
\lean{QEC1.FlexibilityOfGraphG.extendedLogicalOp_supportX_eq_univ}
\leanok
\uses{def:FlexibilityOfGraphG.extendedLogicalOp, def:PauliOp.supportX}
The $X$-support of $L'$ is all of $V \oplus D$: $\operatorname{supportX}(L') = V \oplus D$.
\end{theorem}

\begin{proof}
\leanok
\uses{def:FlexibilityOfGraphG.extendedLogicalOp, def:PauliOp.supportX}
By extensionality, for each qubit $q$, membership in $\operatorname{supportX}(L')$ is equivalent to $\mathrm{xVec}(q) \ne 0$. By the definition of $L'$, $\mathrm{xVec}(q) = 1$ for all $q$, so every qubit is in the support, yielding $\operatorname{supportX}(L') = \operatorname{univ}$.
\end{proof}

\begin{theorem}[Extended Logical Equals $\operatorname{prodX}(\operatorname{univ})$]
\label{thm:FlexibilityOfGraphG.extendedLogicalOp_eq_prodX_univ}
\lean{QEC1.FlexibilityOfGraphG.extendedLogicalOp_eq_prodX_univ}
\leanok
\uses{def:FlexibilityOfGraphG.extendedLogicalOp, def:PauliOp.prodX}
The extended logical operator equals $\operatorname{prodX}(\operatorname{univ})$ on $V \oplus D$.
\end{theorem}

\begin{proof}
\leanok
\uses{def:FlexibilityOfGraphG.extendedLogicalOp, def:PauliOp.prodX}
By extensionality over both the $\mathrm{xVec}$ and $\mathrm{zVec}$ components, this follows by simplification using the definitions of $L'$ and $\operatorname{prodX}$.
\end{proof}

\begin{theorem}[Measurement Sign with Dummies]
\label{thm:FlexibilityOfGraphG.measurement_sign_with_dummies}
\lean{QEC1.FlexibilityOfGraphG.measurement_sign_with_dummies}
\leanok
\uses{def:FlexibilityOfGraphG.extendedLogicalOp}
Let $\sigma_V : V \to \mathbb{Z}/2\mathbb{Z}$ be the measurement outcomes on original vertices and $\sigma_D : D \to \mathbb{Z}/2\mathbb{Z}$ the outcomes on dummy vertices. If all dummy vertices give $+1$ outcomes (i.e., $\sigma_D(d) = 0$ for all $d \in D$), then
\[
\sum_{q \in V \oplus D} \sigma(q) = \sum_{v \in V} \sigma_V(v).
\]
That is, dummy vertices do not affect the logical measurement sign.
\end{theorem}

\begin{proof}
\leanok
\uses{def:FlexibilityOfGraphG.extendedLogicalOp}
We rewrite the sum over $V \oplus D$ as $\sum_{v \in V} \sigma_V(v) + \sum_{d \in D} \sigma_D(d)$ using the decomposition of a sum over a sum type. Since $\sigma_D(d) = 0$ for all $d \in D$ by hypothesis, the second sum vanishes by simplification, yielding the result.
\end{proof}

\begin{theorem}[Weight of Extended Logical]
\label{thm:FlexibilityOfGraphG.extendedLogicalOp_weight}
\lean{QEC1.FlexibilityOfGraphG.extendedLogicalOp_weight}
\leanok
\uses{def:FlexibilityOfGraphG.extendedLogicalOp, def:PauliOp.weight}
The weight of the extended logical operator is $|V| + |D|$:
\[
\operatorname{weight}(L') = |V| + |D|.
\]
\end{theorem}

\begin{proof}
\leanok
\uses{def:FlexibilityOfGraphG.extendedLogicalOp, def:PauliOp.weight, def:PauliOp.support}
Unfolding the definitions of weight and support, and simplifying using the definition of $L'$ (which has $\mathrm{xVec}(q) = 1$ for all $q$ and $\mathrm{zVec} = 0$), the support is all of $V \oplus D$. The cardinality of $V \oplus D$ equals $|V| + |D|$.
\end{proof}

\begin{theorem}[No Dummies Implies Original Weight]
\label{thm:FlexibilityOfGraphG.extendedLogicalOp_no_dummies}
\lean{QEC1.FlexibilityOfGraphG.extendedLogicalOp_no_dummies}
\leanok
\uses{def:FlexibilityOfGraphG.extendedLogicalOp, thm:FlexibilityOfGraphG.extendedLogicalOp_weight}
When $D$ is empty, the weight of $L'$ restricted to $V$ equals $|V|$:
\[
\operatorname{weight}(L') = |V| \quad \text{when } D = \emptyset.
\]
\end{theorem}

\begin{proof}
\leanok
\uses{thm:FlexibilityOfGraphG.extendedLogicalOp_weight}
Rewriting using the weight formula $\operatorname{weight}(L') = |V| + |D|$, and noting that $|D| = 0$ since $D$ is empty, we obtain $|V| + 0 = |V|$.
\end{proof}

\begin{theorem}[Dummy X Product Has Zero Z-Support]
\label{thm:FlexibilityOfGraphG.dummyXProduct_zVec_zero}
\lean{QEC1.FlexibilityOfGraphG.dummyXProduct_zVec_zero}
\leanok
\uses{def:FlexibilityOfGraphG.dummyXProduct}
For every qubit $q \in V \oplus D$, $\mathrm{zVec}(\prod_{d \in D} X_d)(q) = 0$. This means the dummy $X$ product commutes with everything on the $Z$ side, and adding dummy $X$ operators never changes $Z$-support parity.
\end{theorem}

\begin{proof}
\leanok
\uses{def:FlexibilityOfGraphG.dummyXProduct}
By simplification using the definition of the dummy $X$ product, which has $\mathrm{zVec} = 0$.
\end{proof}

\begin{theorem}[Lifted Logical Has Zero Z-Support]
\label{thm:FlexibilityOfGraphG.liftedLogicalOp_zVec_zero}
\lean{QEC1.FlexibilityOfGraphG.liftedLogicalOp_zVec_zero}
\leanok
\uses{def:FlexibilityOfGraphG.liftedLogicalOp}
For every qubit $q \in V \oplus D$, $\mathrm{zVec}(L_{\mathrm{lift}})(q) = 0$. The lifted logical is pure $X$-type.
\end{theorem}

\begin{proof}
\leanok
\uses{def:FlexibilityOfGraphG.liftedLogicalOp}
By simplification using the definition of $L_{\mathrm{lift}}$, which has $\mathrm{zVec} = 0$.
\end{proof}

\begin{theorem}[Dummy X Product is Self-Inverse]
\label{thm:FlexibilityOfGraphG.dummyXProduct_self_inverse}
\lean{QEC1.FlexibilityOfGraphG.dummyXProduct_self_inverse}
\leanok
\uses{def:FlexibilityOfGraphG.dummyXProduct}
The dummy $X$ product is self-inverse: $\bigl(\prod_{d \in D} X_d\bigr)^2 = \mathbf{1}$.
\end{theorem}

\begin{proof}
\leanok
\uses{def:FlexibilityOfGraphG.dummyXProduct}
This follows directly from \texttt{mul\_self}, since every Pauli operator squares to the identity.
\end{proof}

\begin{theorem}[Lifted Logical is Self-Inverse]
\label{thm:FlexibilityOfGraphG.liftedLogicalOp_self_inverse}
\lean{QEC1.FlexibilityOfGraphG.liftedLogicalOp_self_inverse}
\leanok
\uses{def:FlexibilityOfGraphG.liftedLogicalOp}
The lifted logical operator is self-inverse: $L_{\mathrm{lift}}^2 = \mathbf{1}$.
\end{theorem}

\begin{proof}
\leanok
\uses{def:FlexibilityOfGraphG.liftedLogicalOp}
This follows directly from \texttt{mul\_self}, since every Pauli operator squares to the identity.
\end{proof}

\begin{theorem}[Lifted Logical Commutes with Dummy X Product]
\label{thm:FlexibilityOfGraphG.liftedLogicalOp_comm_dummyXProduct}
\lean{QEC1.FlexibilityOfGraphG.liftedLogicalOp_comm_dummyXProduct}
\leanok
\uses{def:FlexibilityOfGraphG.liftedLogicalOp, def:FlexibilityOfGraphG.dummyXProduct}
The lifted logical and dummy $X$ product commute:
\[
L_{\mathrm{lift}} \cdot \prod_{d \in D} X_d = \prod_{d \in D} X_d \cdot L_{\mathrm{lift}}.
\]
\end{theorem}

\begin{proof}
\leanok
\uses{def:FlexibilityOfGraphG.liftedLogicalOp, def:FlexibilityOfGraphG.dummyXProduct, lem:PauliOp.mul_comm}
This follows directly from commutativity of Pauli operator multiplication (\texttt{PauliOp.mul\_comm}).
\end{proof}

%--- Graph Overhead ---

\begin{theorem}[Deformed Code Edge Overhead]
\label{thm:FlexibilityOfGraphG.deformed_code_edge_overhead}
\lean{QEC1.FlexibilityOfGraphG.deformed_code_edge_overhead}
\leanok
\uses{def:DeformedCodeChecks.deformedStabilizerCode, thm:DeformedCodeChecks.deformedStabilizerCode_numQubits, def:StabilizerCode.numQubits, def:DeformedCode.DeformedCodeData}
The deformed code's qubit count is $|V| + |E|$, so the edge qubit overhead (the additional qubits beyond the original $|V|$) is exactly $|E(G)|$. Different graphs $G$ give different overhead via their edge count:
\[
\operatorname{numQubits}(\widetilde{\mathcal{C}}) = |V| + |E(G)|.
\]
\end{theorem}

\begin{proof}
\leanok
\uses{thm:DeformedCodeChecks.deformedStabilizerCode_numQubits}
This follows directly from \texttt{deformedStabilizerCode\_numQubits}.
\end{proof}

\begin{theorem}[Deformed Code Check Overhead]
\label{thm:FlexibilityOfGraphG.deformed_code_check_overhead}
\lean{QEC1.FlexibilityOfGraphG.deformed_code_check_overhead}
\leanok
\uses{def:DeformedCodeChecks.deformedStabilizerCode, thm:DeformedCodeChecks.deformedStabilizerCode_numChecks, def:StabilizerCode.numChecks, def:DeformedCode.DeformedCodeData}
The deformed code's check count is $|V| + |C| + |J|$: $|V|$ Gauss law checks, $|C|$ flux checks, and $|J|$ deformed original checks. The overhead in checks beyond the original $|J|$ is $|V| + |C|$:
\[
\operatorname{numChecks}(\widetilde{\mathcal{C}}) = |V| + |C| + |J|.
\]
\end{theorem}

\begin{proof}
\leanok
\uses{thm:DeformedCodeChecks.deformedStabilizerCode_numChecks}
This follows directly from \texttt{deformedStabilizerCode\_numChecks}.
\end{proof}

\begin{theorem}[Different Graphs Give Different Qubit Counts]
\label{thm:FlexibilityOfGraphG.different_graphs_different_qubit_count}
\lean{QEC1.FlexibilityOfGraphG.different_graphs_different_qubit_count}
\leanok
\uses{def:DeformedCodeChecks.deformedStabilizerCode, thm:DeformedCodeChecks.deformedStabilizerCode_numQubits}
Two graphs $G_1, G_2$ on the same vertex set $V$ with different edge counts ($|E(G_1)| \ne |E(G_2)|$) produce deformed codes with different qubit counts:
\[
|V| + |E(G_1)| \ne |V| + |E(G_2)|.
\]
\end{theorem}

\begin{proof}
\leanok
\uses{thm:DeformedCodeChecks.deformedStabilizerCode_numQubits}
This follows by integer arithmetic (\texttt{omega}): if $|E(G_1)| \ne |E(G_2)|$, then $|V| + |E(G_1)| \ne |V| + |E(G_2)|$.
\end{proof}

%--- Expansion and Distance ---

\begin{theorem}[Edge Expansion Lower Bounds Boundary]
\label{thm:FlexibilityOfGraphG.edge_expansion_lower_bounds_boundary}
\lean{QEC1.FlexibilityOfGraphG.edge_expansion_lower_bounds_boundary}
\leanok
\uses{def:SimpleGraph.IsExpander, def:SimpleGraph.edgeBoundary}
For an expander graph $G$ with expansion constant $c$, every nonempty subset $S \subseteq V$ with $2|S| \le |V|$ satisfies
\[
c \cdot |S| \le |\partial S|,
\]
where $\partial S$ denotes the edge boundary. This is the mechanism by which expansion of $G$ controls the distance of the deformed code: logical operators supported on small subsets necessarily have large edge boundary.
\end{theorem}

\begin{proof}
\leanok
\uses{def:SimpleGraph.IsExpander, def:SimpleGraph.edgeBoundary, thm:SimpleGraph.edgeBoundary_card_ge_of_cheeger}
This follows directly from \texttt{edgeBoundary\_card\_ge\_of\_cheeger}, applied to the graph $G$, expansion constant $c$, the expander hypothesis, subset $S$, its nonemptiness, and the size bound $2|S| \le |V|$.
\end{proof}

\begin{theorem}[Connected Graph Has Edges]
\label{thm:FlexibilityOfGraphG.connected_has_edge}
\lean{QEC1.FlexibilityOfGraphG.connected_has_edge}
\leanok
\uses{thm:SimpleGraph.cheegerValidSubsets'_nonempty_of_card_ge_two}
For a connected graph $G$ on $|V| \ge 2$ vertices, the edge set is nonempty ($0 < |E(G)|$). This means gauging always introduces at least one auxiliary qubit.
\end{theorem}

\begin{proof}
\leanok
\uses{thm:SimpleGraph.cheegerValidSubsets'_nonempty_of_card_ge_two}
We rewrite the goal using the characterization that $|E(G)| > 0$ iff the edge set is nonempty. From the hypothesis $|V| \ge 2$, we obtain the existence of a nonempty valid subset $S$ via \texttt{cheegerValidSubsets'\_nonempty\_of\_card\_ge\_two}. Since $|V| > 1$, the type $V$ is nontrivial, so there exist distinct vertices $a, b \in V$ with $a \ne b$. By connectedness of $G$, there exists a walk from $a$ to $b$. We proceed by induction on the walk. In the base case (\texttt{nil}), the walk from $a$ to $a$ contradicts $a \ne b$. In the inductive case (\texttt{cons}), the walk starts with an edge from $u$ to $v'$ via an adjacency $\mathrm{hadj}$, and we exhibit the edge $\{u, v'\}$ as an element of $E(G)$ (using $\mathrm{hadj}$ to verify membership in the edge set), proving the edge set is nonempty.
\end{proof}

%--- Deformed Check Weight ---

\begin{theorem}[Deformed Check Z-Support on Edges]
\label{thm:FlexibilityOfGraphG.deformedCheck_zSupport_on_edges}
\lean{QEC1.FlexibilityOfGraphG.deformedCheck_zSupport_on_edges}
\leanok
\uses{def:DeformedCode.deformedCheck}
The edge $Z$-support of a deformed check equals the support of the edge-path $\gamma$. For any original check $s$, edge-path $\gamma : E(G) \to \mathbb{Z}/2\mathbb{Z}$, and edge $e \in E(G)$:
\[
\mathrm{zVec}(\tilde{s})(e) = \gamma(e).
\]
More edges in $\gamma$ means more $Z$-support on edge qubits, hence higher weight.
\end{theorem}

\begin{proof}
\leanok
\uses{def:DeformedCode.deformedCheck}
This follows directly from \texttt{deformedOpExt\_zVec\_edge}, which states that the $Z$-component of the deformed operator extension on edge qubits equals $\gamma$.
\end{proof}

\begin{theorem}[Deformed Check Has No X on Edges]
\label{thm:FlexibilityOfGraphG.deformedCheck_xSupport_on_edges}
\lean{QEC1.FlexibilityOfGraphG.deformedCheck_xSupport_on_edges}
\leanok
\uses{def:DeformedCode.deformedCheck}
The edge $X$-support of a deformed check is empty: deformed checks have no $X$-action on edge qubits. For any original check $s$, edge-path $\gamma$, and edge $e$:
\[
\mathrm{xVec}(\tilde{s})(e) = 0.
\]
\end{theorem}

\begin{proof}
\leanok
\uses{def:DeformedCode.deformedCheck}
This follows directly from \texttt{deformedOpExt\_xVec\_edge}, which states that the $X$-component of the deformed operator extension on edge qubits is zero.
\end{proof}

\begin{theorem}[Deformed Check Edge Action Count]
\label{thm:FlexibilityOfGraphG.deformedCheck_edge_action_count}
\lean{QEC1.FlexibilityOfGraphG.deformedCheck_edge_action_count}
\leanok
\uses{def:DeformedCode.deformedCheck}
The number of edge qubits where the deformed check acts non-trivially (via $Z$) equals the number of edges in $\gamma$ with $\gamma(e) \ne 0$:
\[
\bigl|\{e \in E(G) : \mathrm{zVec}(\tilde{s})(e) \ne 0\}\bigr| = \bigl|\{e \in E(G) : \gamma(e) \ne 0\}\bigr|.
\]
This is the ``edge contribution'' to the deformed check weight.
\end{theorem}

\begin{proof}
\leanok
\uses{def:DeformedCode.deformedCheck}
The two finite sets are equal by congruence: the filter predicates are definitionally equal since $\mathrm{zVec}(\tilde{s})(e) = \gamma(e)$ for all edges $e$.
\end{proof}

\begin{theorem}[Zero Path Adds No Edge Weight]
\label{thm:FlexibilityOfGraphG.deformedCheck_zero_path_edge_count}
\lean{QEC1.FlexibilityOfGraphG.deformedCheck_zero_path_edge_count}
\leanok
\uses{def:DeformedCode.deformedCheck, thm:FlexibilityOfGraphG.deformedCheck_zSupport_on_edges}
A zero edge-path adds no edge weight: when $\gamma = 0$, the deformed check acts only on vertices:
\[
\bigl|\{e \in E(G) : \mathrm{zVec}(\tilde{s})(e) \ne 0\}\bigr| = 0.
\]
\end{theorem}

\begin{proof}
\leanok
\uses{thm:FlexibilityOfGraphG.deformedCheck_zSupport_on_edges}
By simplification using the fact that $\mathrm{zVec}(\tilde{s})(e) = \gamma(e) = 0$ for all edges $e$ when $\gamma = 0$, the filter set is empty and has cardinality $0$.
\end{proof}

%--- Rem_11: DesiderataForGraphG ---
\chapter{Rem 11: Desiderata for Graph $G$}

When choosing a constant-degree graph $G$ for the gauging measurement, three desiderata must be satisfied:
\begin{enumerate}
\item \textbf{Short paths for deformation:} $G$ should contain a constant-length edge-path between any pair of vertices in the $Z$-type support of the same original check, ensuring deformed checks have bounded weight.
\item \textbf{Sufficient expansion:} The Cheeger constant should satisfy $h(G) \geq 1$, ensuring the deformed code has distance at least $d$.
\item \textbf{Low-weight cycle basis:} There should exist a generating set of cycles where each cycle has weight at most some constant, ensuring the flux operators $B_p$ have bounded weight.
\end{enumerate}
When all three conditions are satisfied with constants independent of $n$, the gauging measurement procedure has constant overhead per qubit, the LDPC property is preserved, and there is no distance reduction.

\section{Desideratum 1: Short Paths for Deformation}

\begin{definition}[Short Paths for Deformation]
\label{def:DesiderataForGraphG.ShortPathsForDeformation}
\lean{QEC1.DesiderataForGraphG.ShortPathsForDeformation}
\leanok
\uses{def:PauliOp.supportZ}
Let $G$ be a simple graph on vertices $V$, let $\{s_j\}_{j \in J}$ be a family of Pauli operators, and let $D \in \mathbb{N}$. We say $G$ satisfies \emph{short paths for deformation} with parameter $D$ if for every check $s_j$ and every pair of vertices $u, v$ in the $Z$-support of $s_j$, the graph distance satisfies $\operatorname{dist}_G(u,v) \leq D$.
\end{definition}

\begin{theorem}[Short Paths Bound Distance]
\label{thm:DesiderataForGraphG.short_paths_bound_distance}
\lean{QEC1.DesiderataForGraphG.short_paths_bound_distance}
\leanok
\uses{def:DesiderataForGraphG.ShortPathsForDeformation, def:PauliOp.supportZ}
If $G$ satisfies short paths for deformation with parameter $D$, then for any check $s_j$ and any $u, v \in \operatorname{supportZ}(s_j)$, we have $\operatorname{dist}_G(u,v) \leq D$.
\end{theorem}
\begin{proof}
\leanok
\uses{def:DesiderataForGraphG.ShortPathsForDeformation}
This follows directly from the definition of short paths for deformation applied to $j$, $u$, $v$, and the membership hypotheses.
\end{proof}

\begin{theorem}[Short Paths Imply Reachability]
\label{thm:DesiderataForGraphG.short_paths_imply_reachable}
\lean{QEC1.DesiderataForGraphG.short_paths_imply_reachable}
\leanok
\uses{def:PauliOp.supportZ}
If $G$ is connected, then for any check $s_j$ and any $u, v \in \operatorname{supportZ}(s_j)$, the vertex $v$ is reachable from $u$ in $G$.
\end{theorem}
\begin{proof}
\leanok
This follows directly from the connectedness of $G$: since $G$ is connected, any two vertices $u$ and $v$ are reachable from each other.
\end{proof}

\begin{theorem}[$Z$-Support Cardinality Bounded by Weight]
\label{thm:DesiderataForGraphG.zSupport_card_le_weight}
\lean{QEC1.DesiderataForGraphG.zSupport_card_le_weight}
\leanok
\uses{def:PauliOp.supportZ, def:PauliOp.weight, def:PauliOp.support}
For any check $s_j$, we have $|\operatorname{supportZ}(s_j)| \leq \operatorname{weight}(s_j)$.
\end{theorem}
\begin{proof}
\leanok
\uses{def:PauliOp.weight, def:PauliOp.supportZ, def:PauliOp.support}
We unfold the definition of weight as $|\operatorname{support}(s_j)|$. We show $\operatorname{supportZ}(s_j) \subseteq \operatorname{support}(s_j)$ by taking any $v \in \operatorname{supportZ}(s_j)$, rewriting via the membership characterization of $\operatorname{supportZ}$, then showing $v \in \operatorname{support}(s_j)$ via the right disjunct of the support membership condition. The result then follows by monotonicity of cardinality.
\end{proof}

\begin{theorem}[Deformed Check Edge Bound]
\label{thm:DesiderataForGraphG.deformed_check_edge_bound}
\lean{QEC1.DesiderataForGraphG.deformed_check_edge_bound}
\leanok
\uses{def:DeformedCode.deformedCheck}
Let $\gamma : E(G) \to \mathbb{Z}/2\mathbb{Z}$ be an edge-path with at most $B$ nonzero entries. Then the number of edges where the deformed check $\tilde{s}_j(\gamma)$ acts nontrivially in the $Z$-component is at most $B$.
\end{theorem}
\begin{proof}
\leanok
\uses{thm:FlexibilityOfGraphG.deformedCheck_edge_action_count}
Rewriting using the edge action count characterization from the flexibility analysis, the number of edges where the deformed check has nonzero $Z$-action equals the number of nonzero entries of $\gamma$, which is at most $B$ by hypothesis.
\end{proof}

\begin{theorem}[Zero Path Adds No Edge Weight]
\label{thm:DesiderataForGraphG.deformed_check_zero_path_no_edges}
\lean{QEC1.DesiderataForGraphG.deformed_check_zero_path_no_edges}
\leanok
\uses{def:DeformedCode.deformedCheck}
When $\gamma = 0$ (the zero edge-path), the deformed check $\tilde{s}_j(0)$ has no nonzero $Z$-action on any edge. That is, the number of edges with nontrivial $Z$-action is $0$.
\end{theorem}
\begin{proof}
\leanok
\uses{thm:FlexibilityOfGraphG.deformedCheck_zero_path_edge_count}
This follows directly from the zero-path edge count result.
\end{proof}

\section{Desideratum 2: Sufficient Expansion}

\begin{definition}[Sufficient Expansion]
\label{def:DesiderataForGraphG.SufficientExpansion}
\lean{QEC1.DesiderataForGraphG.SufficientExpansion}
\leanok
\uses{def:cheegerConstant}
A simple graph $G$ has \emph{sufficient expansion} if $1 \leq h(G)$, where $h(G)$ is the Cheeger constant of $G$.
\end{definition}

\begin{theorem}[Sufficient Expansion Implies Expander]
\label{thm:DesiderataForGraphG.sufficient_expansion_implies_expander}
\lean{QEC1.DesiderataForGraphG.sufficient_expansion_implies_expander}
\leanok
\uses{def:DesiderataForGraphG.SufficientExpansion, def:SimpleGraph.IsExpander}
If $G$ has sufficient expansion, then $G$ is a $1$-expander.
\end{theorem}
\begin{proof}
\leanok
\uses{def:DesiderataForGraphG.SufficientExpansion, def:SimpleGraph.IsExpander}
We construct the pair: the positivity condition $0 < 1$ holds by \texttt{norm\_num}, and the Cheeger bound $1 \leq h(G)$ is exactly the sufficient expansion hypothesis.
\end{proof}

\begin{theorem}[Expansion Gives Boundary Bound]
\label{thm:DesiderataForGraphG.expansion_gives_boundary_bound}
\lean{QEC1.DesiderataForGraphG.expansion_gives_boundary_bound}
\leanok
\uses{def:DesiderataForGraphG.SufficientExpansion, def:SimpleGraph.edgeBoundary}
If $G$ has sufficient expansion, then for every nonempty $S \subseteq V$ with $2|S| \leq |V|$, we have $|S| \leq |\partial S|$, where $\partial S$ is the edge boundary.
\end{theorem}
\begin{proof}
\leanok
\uses{thm:SimpleGraph.edgeBoundary_card_ge_of_cheeger, def:DesiderataForGraphG.SufficientExpansion}
From the Cheeger constant bound theorem, we have $h(G) \cdot |S| \leq |\partial S|$ as real numbers. Since $h(G) \geq 1$ by the sufficient expansion hypothesis, we obtain $|S| \leq |\partial S|$ by linear arithmetic.
\end{proof}

\begin{theorem}[Sufficient Expansion is Positive]
\label{thm:DesiderataForGraphG.sufficient_expansion_pos}
\lean{QEC1.DesiderataForGraphG.sufficient_expansion_pos}
\leanok
\uses{def:DesiderataForGraphG.SufficientExpansion, def:cheegerConstant}
If $G$ has sufficient expansion, then $h(G) > 0$.
\end{theorem}
\begin{proof}
\leanok
\uses{def:DesiderataForGraphG.SufficientExpansion}
Since $0 < 1$ and $1 \leq h(G)$, we have $0 < h(G)$ by transitivity.
\end{proof}

\section{Desideratum 3: Low-Weight Cycle Basis}

\begin{definition}[Low-Weight Cycle Basis]
\label{def:DesiderataForGraphG.LowWeightCycleBasis}
\lean{QEC1.DesiderataForGraphG.LowWeightCycleBasis}
\leanok

Let $G$ be a simple graph, let $\{\text{cycle}_c\}_{c \in C}$ be a family of subsets of edges, and let $W \in \mathbb{N}$. The cycle basis is \emph{low-weight} with parameter $W$ if for every $c \in C$, the number of edges in $\text{cycle}_c$ is at most $W$.
\end{definition}

\begin{theorem}[Low-Weight Cycles Bound Flux Weight]
\label{thm:DesiderataForGraphG.low_weight_cycles_bound_flux_weight}
\lean{QEC1.DesiderataForGraphG.low_weight_cycles_bound_flux_weight}
\leanok
\uses{def:DesiderataForGraphG.LowWeightCycleBasis, def:DeformedCode.fluxChecks, def:PauliOp.weight}
If the cycle basis has low weight with parameter $W$, then for every plaquette $p$, the weight of the flux check $B_p$ is at most $W$.
\end{theorem}
\begin{proof}
\leanok
\uses{thm:DeformedCode.fluxChecks_weight, def:DesiderataForGraphG.LowWeightCycleBasis}
Rewriting the weight of the flux check using the flux check weight characterization, the weight equals the number of edges in cycle $p$, which is at most $W$ by the low-weight hypothesis.
\end{proof}

\begin{theorem}[All Flux Checks Bounded]
\label{thm:DesiderataForGraphG.all_flux_checks_bounded}
\lean{QEC1.DesiderataForGraphG.all_flux_checks_bounded}
\leanok
\uses{def:DesiderataForGraphG.LowWeightCycleBasis, def:DeformedCode.fluxChecks}
If the cycle basis has low weight with parameter $W$, then for all plaquettes $p$, $\operatorname{weight}(B_p) \leq W$.
\end{theorem}
\begin{proof}
\leanok
\uses{thm:DesiderataForGraphG.low_weight_cycles_bound_flux_weight}
For each $p$, this follows directly from the low-weight cycles bound on flux weight.
\end{proof}

\section{Constant Degree and Edge Overhead}

\begin{definition}[Constant Degree]
\label{def:DesiderataForGraphG.ConstantDegree}
\lean{QEC1.DesiderataForGraphG.ConstantDegree}
\leanok

A simple graph $G$ has \emph{constant degree} with parameter $\Delta$ if for every vertex $v$, $\deg(v) \leq \Delta$.
\end{definition}

\begin{theorem}[Constant Degree Bounds Edges]
\label{thm:DesiderataForGraphG.constant_degree_bounds_edges}
\lean{QEC1.DesiderataForGraphG.constant_degree_bounds_edges}
\leanok
\uses{def:DesiderataForGraphG.ConstantDegree}
If $G$ has constant degree $\Delta$, then $2|E| \leq \Delta \cdot |V|$.
\end{theorem}
\begin{proof}
\leanok
\uses{def:DesiderataForGraphG.ConstantDegree}
We first establish that $\sum_{v \in V} \deg(v) \leq \Delta \cdot |V|$. By the degree bound, $\sum_v \deg(v) \leq \sum_v \Delta = \Delta \cdot |V|$, where the first inequality uses the constant degree hypothesis applied to each vertex, and the second equality follows by simplification with the sum of a constant. By the handshaking lemma, $\sum_v \deg(v) = 2|E|$. We obtain this from the graph's sum-of-degrees-equals-twice-edges result, adjusting for the fintype edge set instance by subsingleton elimination. The conclusion $2|E| \leq \Delta \cdot |V|$ then follows by integer arithmetic.
\end{proof}

\begin{theorem}[Edge Overhead Per Vertex]
\label{thm:DesiderataForGraphG.edge_overhead_per_vertex}
\lean{QEC1.DesiderataForGraphG.edge_overhead_per_vertex}
\leanok
\uses{def:DesiderataForGraphG.ConstantDegree}
If $G$ has constant degree $\Delta$ and $|V| > 0$, then $\frac{|E|}{|V|} \leq \frac{\Delta}{2}$.
\end{theorem}
\begin{proof}
\leanok
\uses{thm:DesiderataForGraphG.constant_degree_bounds_edges, def:DesiderataForGraphG.ConstantDegree}
From the constant degree bound $2|E| \leq \Delta \cdot |V|$, we rewrite the division inequality using the characterization $\frac{a}{b} \leq \frac{c}{d}$ iff $a \cdot d \leq c \cdot b$ (for positive denominators). The positivity of $|V|$ and $2$ are verified. Casting the natural number inequality $2|E| \leq \Delta \cdot |V|$ to $\mathbb{R}$ and rearranging by linear arithmetic yields the result.
\end{proof}

\begin{theorem}[Total Qubits Bounded]
\label{thm:DesiderataForGraphG.total_qubits_bounded}
\lean{QEC1.DesiderataForGraphG.total_qubits_bounded}
\leanok
\uses{def:DesiderataForGraphG.ConstantDegree}
If $G$ has constant degree $\Delta$, then $2(|V| + |E|) \leq (2 + \Delta) \cdot |V|$.
\end{theorem}
\begin{proof}
\leanok
\uses{thm:DesiderataForGraphG.constant_degree_bounds_edges, def:DesiderataForGraphG.ConstantDegree}
From the bound $2|E| \leq \Delta \cdot |V|$, the result $2(|V| + |E|) \leq (2 + \Delta) \cdot |V|$ follows by nonlinear arithmetic.
\end{proof}

\section{Gauss Check Weight Characterization}

\begin{lemma}[Gauss Law $X$-Vector at Vertex]
\label{lem:DesiderataForGraphG.gaussLawOp_xVec_inl}
\lean{QEC1.DesiderataForGraphG.gaussLawOp_xVec_inl}
\leanok
\uses{def:GaussFlux.gaussLawOp}
For any vertices $v, w$, the $X$-component of $A_v$ at vertex $w$ is given by $(A_v)^X_{w} = \begin{cases} 1 & \text{if } w = v \\ 0 & \text{otherwise} \end{cases}$.
\end{lemma}
\begin{proof}
\leanok
\uses{def:GaussFlux.gaussLawOp}
This follows by simplification from the definition of the Gauss law operator.
\end{proof}

\begin{lemma}[Gauss Law $X$-Vector at Edge]
\label{lem:DesiderataForGraphG.gaussLawOp_xVec_inr}
\lean{QEC1.DesiderataForGraphG.gaussLawOp_xVec_inr}
\leanok
\uses{def:GaussFlux.gaussLawOp}
For any vertex $v$ and edge $e$, the $X$-component of $A_v$ at edge $e$ is $(A_v)^X_e = \begin{cases} 1 & \text{if } v \in e \\ 0 & \text{otherwise} \end{cases}$.
\end{lemma}
\begin{proof}
\leanok
\uses{def:GaussFlux.gaussLawOp}
This follows by simplification from the definition of the Gauss law operator.
\end{proof}

\begin{lemma}[Gauss Law Support at Vertex]
\label{lem:DesiderataForGraphG.gaussLaw_support_inl}
\lean{QEC1.DesiderataForGraphG.gaussLaw_support_inl}
\leanok
\uses{def:GaussFlux.gaussLawOp, def:PauliOp.support}
$A_v$ acts nontrivially at vertex $w$ if and only if $w = v$.
\end{lemma}
\begin{proof}
\leanok
\uses{lem:DesiderataForGraphG.gaussLawOp_xVec_inl, def:GaussFlux.gaussLawOp}
We simplify using the support membership characterization, the $X$-vector formula at vertices, and the fact that the $Z$-vector of $A_v$ is zero. For the forward direction, given that $w$ is in the support, we case split: if the $X$-component is nonzero, then by contradiction if $w \neq v$ the component would be zero; if the $Z$-component is nonzero, this contradicts it being zero. For the reverse direction, assuming $w = v$, the left disjunct holds by simplification.
\end{proof}

\begin{lemma}[Gauss Law Support at Edge]
\label{lem:DesiderataForGraphG.gaussLaw_support_inr}
\lean{QEC1.DesiderataForGraphG.gaussLaw_support_inr}
\leanok
\uses{def:GaussFlux.gaussLawOp, def:PauliOp.support}
$A_v$ acts nontrivially at edge $e$ if and only if $v \in e$.
\end{lemma}
\begin{proof}
\leanok
\uses{lem:DesiderataForGraphG.gaussLawOp_xVec_inr, def:GaussFlux.gaussLawOp}
We simplify using the support membership, the $X$-vector formula at edges, and the zero $Z$-vector. For the forward direction, case splitting: if the $X$-component is nonzero, by contradiction if $v \notin e$ the component would be zero; the $Z$-component case gives a contradiction. For the reverse, assuming $v \in e$, the left disjunct holds by simplification.
\end{proof}

\begin{theorem}[Gauss Law Support Characterization]
\label{thm:DesiderataForGraphG.gaussLaw_support_eq}
\lean{QEC1.DesiderataForGraphG.gaussLaw_support_eq}
\leanok
\uses{def:GaussFlux.gaussLawOp, def:PauliOp.support}
The support of $A_v$ is
\[
\operatorname{support}(A_v) = \{\operatorname{inl}(v)\} \cup \{\operatorname{inr}(e) \mid v \in e\}.
\]
\end{theorem}
\begin{proof}
\leanok
\uses{lem:DesiderataForGraphG.gaussLaw_support_inl, lem:DesiderataForGraphG.gaussLaw_support_inr}
By extensionality, we take an arbitrary element $q$ and case split. If $q = \operatorname{inl}(w)$: membership in the left-hand side is equivalent to $w = v$ by the vertex support lemma; membership in the right-hand side reduces to the singleton condition or being in the image of the edge filter, where the latter is impossible for a vertex element. If $q = \operatorname{inr}(e)$: membership in the left-hand side is equivalent to $v \in e$ by the edge support lemma; for the right-hand side, the forward direction constructs the witness $e$ with its membership proof, and the reverse direction extracts the edge from the existential and applies injectivity of $\operatorname{inr}$.
\end{proof}

\begin{theorem}[Gauss Law Weight]
\label{thm:DesiderataForGraphG.gaussLaw_weight_eq}
\lean{QEC1.DesiderataForGraphG.gaussLaw_weight_eq}
\leanok
\uses{def:GaussFlux.gaussLawOp, def:PauliOp.weight}
The weight of $A_v$ equals $1 + |\{e \in E(G) \mid v \in e\}|$.
\end{theorem}
\begin{proof}
\leanok
\uses{thm:DesiderataForGraphG.gaussLaw_support_eq}
Unfolding the weight as the cardinality of the support and rewriting with the support characterization, we compute the cardinality of the disjoint union. Disjointness holds because $\operatorname{inl}(v)$ cannot equal any $\operatorname{inr}(e)$: assuming membership in both the singleton and the mapped filter set, we obtain a contradiction from the impossibility of $\operatorname{inl}(v) = \operatorname{inr}(e)$. The cardinality of the union of disjoint sets is the sum of cardinalities, giving $1 + |\{e \mid v \in e\}|$ since the map preserves cardinality.
\end{proof}

\begin{theorem}[Incident Edges Count]
\label{thm:DesiderataForGraphG.incident_edges_eq}
\lean{QEC1.DesiderataForGraphG.incident_edges_eq}
\leanok
\uses{def:GaussFlux.incidentEdges}
The number of edges $e$ with $v \in e$ equals the cardinality of $\operatorname{incidentEdges}(G, v)$.
\end{theorem}
\begin{proof}
\leanok
\uses{def:GaussFlux.incidentEdges}
The two finsets have equal underlying sets, so their cardinalities agree by congruence.
\end{proof}

\begin{theorem}[Gauss Law Weight via Incident Edges]
\label{thm:DesiderataForGraphG.gaussLaw_weight_eq_incident}
\lean{QEC1.DesiderataForGraphG.gaussLaw_weight_eq_incident}
\leanok
\uses{def:GaussFlux.gaussLawOp, def:GaussFlux.incidentEdges, def:PauliOp.weight}
The weight of $A_v$ equals $1 + |\operatorname{incidentEdges}(G, v)|$.
\end{theorem}
\begin{proof}
\leanok
\uses{thm:DesiderataForGraphG.gaussLaw_weight_eq, thm:DesiderataForGraphG.incident_edges_eq}
Rewriting with the weight formula and the incident edges characterization, we obtain the result directly.
\end{proof}

\begin{theorem}[Incident Edges Equals Degree]
\label{thm:DesiderataForGraphG.incidentEdges_card_eq_degree}
\lean{QEC1.DesiderataForGraphG.incidentEdges_card_eq_degree}
\leanok
\uses{def:GaussFlux.incidentEdges}
The cardinality of $\operatorname{incidentEdges}(G, v)$ equals $\deg_G(v)$.
\end{theorem}
\begin{proof}
\leanok
\uses{def:GaussFlux.incidentEdges}
We first rewrite $\deg_G(v)$ as $|\operatorname{incidenceFinset}(v)|$ using the standard characterization. We then show the images under the subtype coercion agree: for the forward direction, given an edge in $\operatorname{incidentEdges}(G,v)$, its underlying $\operatorname{Sym2}(V)$ element satisfies both the edge membership and incidence conditions; for the reverse, given an element of the incidence finset, we construct the corresponding subtype element. The cardinality equality follows from injectivity of the subtype coercion.
\end{proof}

\begin{theorem}[Gauss Law Weight Bounded by Degree]
\label{thm:DesiderataForGraphG.gaussLaw_weight_le}
\lean{QEC1.DesiderataForGraphG.gaussLaw_weight_le}
\leanok
\uses{def:GaussFlux.gaussLawOp, def:DesiderataForGraphG.ConstantDegree, def:PauliOp.weight}
On a constant-degree graph with parameter $\Delta$, the weight of $A_v$ is at most $1 + \Delta$.
\end{theorem}
\begin{proof}
\leanok
\uses{thm:DesiderataForGraphG.gaussLaw_weight_eq_incident, thm:DesiderataForGraphG.incidentEdges_card_eq_degree, def:DesiderataForGraphG.ConstantDegree}
Rewriting the weight as $1 + |\operatorname{incidentEdges}(G,v)|$ and then as $1 + \deg_G(v)$, the bound $\deg_G(v) \leq \Delta$ from the constant degree hypothesis yields $\operatorname{weight}(A_v) \leq 1 + \Delta$ by linear arithmetic.
\end{proof}

\section{All Desiderata Satisfied}

\begin{definition}[All Desiderata Satisfied]
\label{def:DesiderataForGraphG.AllDesiderataSatisfied}
\lean{QEC1.DesiderataForGraphG.AllDesiderataSatisfied}
\leanok
\uses{def:DesiderataForGraphG.ShortPathsForDeformation, def:DesiderataForGraphG.SufficientExpansion, def:DesiderataForGraphG.LowWeightCycleBasis}
A structure bundling all three desiderata for graph $G$ with parameters $D$ and $W$:
\begin{enumerate}
\item \texttt{short\_paths}: $G$ satisfies short paths for deformation with parameter $D$.
\item \texttt{expansion}: $G$ has sufficient expansion ($h(G) \geq 1$).
\item \texttt{low\_weight\_cycles}: The cycle basis of $G$ has low weight with parameter $W$.
\end{enumerate}
\end{definition}

\begin{theorem}[Desiderata Imply Bounded Flux Weight]
\label{thm:DesiderataForGraphG.desiderata_imply_bounded_flux_weight}
\lean{QEC1.DesiderataForGraphG.desiderata_imply_bounded_flux_weight}
\leanok
\uses{def:DesiderataForGraphG.AllDesiderataSatisfied, def:DeformedCode.fluxChecks}
When all desiderata are satisfied with parameters $D, W$, the flux checks have weight at most $W$.
\end{theorem}
\begin{proof}
\leanok
\uses{thm:DesiderataForGraphG.low_weight_cycles_bound_flux_weight, def:DesiderataForGraphG.AllDesiderataSatisfied}
This follows directly from the low-weight cycles bound on flux weight, using the low-weight cycle component of the all-desiderata structure.
\end{proof}

\begin{theorem}[Desiderata Imply Expander]
\label{thm:DesiderataForGraphG.desiderata_imply_expander}
\lean{QEC1.DesiderataForGraphG.desiderata_imply_expander}
\leanok
\uses{def:DesiderataForGraphG.AllDesiderataSatisfied, def:SimpleGraph.IsExpander}
When all desiderata are satisfied, $G$ is a $1$-expander.
\end{theorem}
\begin{proof}
\leanok
\uses{thm:DesiderataForGraphG.sufficient_expansion_implies_expander, def:DesiderataForGraphG.AllDesiderataSatisfied}
This follows by applying the sufficient expansion implies expander result to the expansion component of the all-desiderata structure.
\end{proof}

\begin{theorem}[Desiderata Imply No Distance Loss]
\label{thm:DesiderataForGraphG.desiderata_imply_no_distance_loss}
\lean{QEC1.DesiderataForGraphG.desiderata_imply_no_distance_loss}
\leanok
\uses{def:DesiderataForGraphG.AllDesiderataSatisfied, def:SimpleGraph.edgeBoundary}
When all desiderata are satisfied, every nonempty $S \subseteq V$ with $2|S| \leq |V|$ satisfies $|S| \leq |\partial S|$.
\end{theorem}
\begin{proof}
\leanok
\uses{thm:DesiderataForGraphG.expansion_gives_boundary_bound, def:DesiderataForGraphG.AllDesiderataSatisfied}
This follows by applying the expansion boundary bound to the expansion component of the all-desiderata structure.
\end{proof}

\section{LDPC Preservation}

\begin{definition}[Deformed Code Is LDPC]
\label{def:DesiderataForGraphG.DeformedCodeIsLDPC}
\lean{QEC1.DesiderataForGraphG.DeformedCodeIsLDPC}
\leanok
\uses{def:DeformedCodeChecks.deformedStabilizerCode, def:IsQLDPC}
The deformed code is LDPC with weight bound $w$ and check bound $c$ if the deformed stabilizer code satisfies the $\operatorname{IsQLDPC}$ predicate with parameters $w$ and $c$.
\end{definition}

\begin{theorem}[Gauss Law Checks Weight Bounded]
\label{thm:DesiderataForGraphG.gaussLaw_checks_weight_bounded}
\lean{QEC1.DesiderataForGraphG.gaussLaw_checks_weight_bounded}
\leanok
\uses{def:DesiderataForGraphG.ConstantDegree, def:DeformedCode.gaussLawChecks}
On a constant-degree graph with parameter $\Delta$, the Gauss law check $A_v$ has weight at most $1 + \Delta$.
\end{theorem}
\begin{proof}
\leanok
\uses{thm:DesiderataForGraphG.gaussLaw_weight_le, def:DesiderataForGraphG.ConstantDegree}
This follows directly from the Gauss law weight bound.
\end{proof}

\begin{theorem}[Flux Checks Weight Bounded]
\label{thm:DesiderataForGraphG.flux_checks_weight_bounded}
\lean{QEC1.DesiderataForGraphG.flux_checks_weight_bounded}
\leanok
\uses{def:DesiderataForGraphG.LowWeightCycleBasis, def:DeformedCode.fluxChecks}
If the cycle basis has low weight $W$, then the flux check $B_p$ has weight at most $W$.
\end{theorem}
\begin{proof}
\leanok
\uses{thm:DesiderataForGraphG.low_weight_cycles_bound_flux_weight}
This follows directly from the low-weight cycles bound on flux weight.
\end{proof}

\section{Summary: Combined Consequences}

\begin{theorem}[Desiderata Consequences]
\label{thm:DesiderataForGraphG.desiderata_consequences}
\lean{QEC1.DesiderataForGraphG.desiderata_consequences}
\leanok
\uses{def:DesiderataForGraphG.ShortPathsForDeformation, def:DesiderataForGraphG.SufficientExpansion, def:DesiderataForGraphG.LowWeightCycleBasis, def:DesiderataForGraphG.ConstantDegree, def:SimpleGraph.IsExpander, def:DeformedCode.fluxChecks, def:DeformedCode.gaussLawChecks}
When all three desiderata and constant degree $\Delta$ are satisfied, the following hold simultaneously:
\begin{enumerate}
\item Flux checks bounded: for all plaquettes $p$, $\operatorname{weight}(B_p) \leq W$.
\item $G$ is a $1$-expander.
\item Edge overhead is linear: $2|E| \leq \Delta \cdot |V|$.
\item Gauss checks bounded: for all vertices $v$, $\operatorname{weight}(A_v) \leq 1 + \Delta$.
\end{enumerate}
\end{theorem}
\begin{proof}
\leanok
\uses{thm:DesiderataForGraphG.low_weight_cycles_bound_flux_weight, thm:DesiderataForGraphG.sufficient_expansion_implies_expander, thm:DesiderataForGraphG.constant_degree_bounds_edges, thm:DesiderataForGraphG.gaussLaw_weight_le}
We construct the four-tuple directly: (1) for each plaquette $p$, apply the low-weight cycles bound; (2) apply the sufficient expansion implies expander result; (3) apply the constant degree bounds edges result; (4) for each vertex $v$, apply the Gauss law weight bound.
\end{proof}

\begin{theorem}[Overhead Per Vertex Bounded]
\label{thm:DesiderataForGraphG.overhead_per_vertex_bounded}
\lean{QEC1.DesiderataForGraphG.overhead_per_vertex_bounded}
\leanok
\uses{def:DesiderataForGraphG.ShortPathsForDeformation, def:DesiderataForGraphG.SufficientExpansion, def:DesiderataForGraphG.LowWeightCycleBasis, def:DesiderataForGraphG.ConstantDegree}
When all three desiderata and constant degree $\Delta$ are satisfied with $|V| > 0$, the additional qubit overhead per vertex satisfies $\frac{|E|}{|V|} \leq \frac{\Delta}{2}$.
\end{theorem}
\begin{proof}
\leanok
\uses{thm:DesiderataForGraphG.edge_overhead_per_vertex, def:DesiderataForGraphG.ConstantDegree}
This follows directly from the edge overhead per vertex result applied to the constant degree hypothesis.
\end{proof}

%--- Def_6: CycleSparsifiedGraph ---
\chapter{Def 6: Cycle-Sparsified Graph}

The cycle-sparsified graph (decongested graph) $\overline{\overline{G}}$ with cycle-degree bound $c$ is constructed from a connected graph $G$ with a chosen generating set of cycles by creating multiple layers and distributing cycles across them, then adding cellulation (triangulation) edges via a zigzag pattern to decompose high-degree cycles into triangles.

\begin{definition}[Sparsified Vertex]
\label{def:CycleSparsification.SparsifiedVertex}
\lean{QEC1.CycleSparsification.SparsifiedVertex}
\leanok

The vertex type of the sparsified graph is $V \times \operatorname{Fin}(R+1)$, where a pair $(v, i)$ represents vertex $v$ in layer $i$. Layer $0$ is the original graph layer.
\end{definition}

\begin{definition}[Original Layer]
\label{def:CycleSparsification.originalLayer}
\lean{QEC1.CycleSparsification.originalLayer}
\leanok
\uses{def:CycleSparsification.SparsifiedVertex}
The original layer is layer $0$, i.e., $\operatorname{originalLayer} = \langle 0, \ldots \rangle : \operatorname{Fin}(R+1)$.
\end{definition}

\begin{theorem}[Cardinality of Sparsified Vertices]
\label{thm:CycleSparsification.card_sparsifiedVertex}
\lean{QEC1.CycleSparsification.card_sparsifiedVertex}
\leanok
\uses{def:CycleSparsification.SparsifiedVertex}
The number of vertices in the sparsified graph satisfies
\[
|\operatorname{SparsifiedVertex}(V, R)| = (R+1) \cdot |V|.
\]
\end{theorem}

\begin{proof}
\leanok

By simplification using the cardinality of the product type $V \times \operatorname{Fin}(R+1)$, we have $|V \times \operatorname{Fin}(R+1)| = |\operatorname{Fin}(R+1)| \cdot |V|$, and $|\operatorname{Fin}(R+1)| = R+1$. The result then follows by commutativity of multiplication.
\end{proof}

\begin{definition}[Layer Vertices]
\label{def:CycleSparsification.layerVertices}
\lean{QEC1.CycleSparsification.layerVertices}
\leanok
\uses{def:CycleSparsification.SparsifiedVertex}
The set of vertices in layer $i$ is defined as
\[
\operatorname{layerVertices}(i) = \{ p \in \operatorname{SparsifiedVertex}(V, R) \mid p_2 = i \}.
\]
\end{definition}

\begin{theorem}[Cardinality of Layer Vertices]
\label{thm:CycleSparsification.card_layerVertices}
\lean{QEC1.CycleSparsification.card_layerVertices}
\leanok
\uses{def:CycleSparsification.layerVertices}
For each layer $i \in \operatorname{Fin}(R+1)$, the number of vertices in that layer equals the number of original vertices:
\[
|\operatorname{layerVertices}(i)| = |V|.
\]
\end{theorem}

\begin{proof}
\leanok
\uses{def:CycleSparsification.layerVertices}
We rewrite the filter $\{p \in V \times \operatorname{Fin}(R+1) \mid p_2 = i\}$ as the product $V \times \{i\}$, using extensionality and simplification. The cardinality of this product is $|V| \cdot |\{i\}| = |V| \cdot 1 = |V|$.
\end{proof}

\begin{definition}[Embed Original]
\label{def:CycleSparsification.embedOriginal}
\lean{QEC1.CycleSparsification.embedOriginal}
\leanok
\uses{def:CycleSparsification.SparsifiedVertex, def:CycleSparsification.originalLayer}
The embedding of an original vertex $v \in V$ into the sparsified vertex type is
\[
\operatorname{embedOriginal}(v) = (v, \operatorname{originalLayer}).
\]
\end{definition}

\begin{theorem}[Embed Original is Injective]
\label{thm:CycleSparsification.embedOriginal_injective}
\lean{QEC1.CycleSparsification.embedOriginal_injective}
\leanok
\uses{def:CycleSparsification.embedOriginal}
The function $\operatorname{embedOriginal} : V \to \operatorname{SparsifiedVertex}(V, R)$ is injective.
\end{theorem}

\begin{proof}
\leanok
\uses{def:CycleSparsification.embedOriginal}
Let $v, w \in V$ and suppose $\operatorname{embedOriginal}(v) = \operatorname{embedOriginal}(w)$. Then $(v, \operatorname{originalLayer}) = (w, \operatorname{originalLayer})$, so by the extensionality of pairs, the first components are equal: $v = w$.
\end{proof}

\begin{theorem}[Vertex Partition]
\label{thm:CycleSparsification.vertex_partition}
\lean{QEC1.CycleSparsification.vertex_partition}
\leanok
\uses{def:CycleSparsification.layerVertices, def:CycleSparsification.SparsifiedVertex}
Every sparsified vertex $p \in \operatorname{SparsifiedVertex}(V, R)$ belongs to a unique layer: there exists a unique $i \in \operatorname{Fin}(R+1)$ such that $p \in \operatorname{layerVertices}(i)$.
\end{theorem}

\begin{proof}
\leanok
\uses{def:CycleSparsification.layerVertices}
For any $p = (v, j)$, we take $i = p_2 = j$. Then $p \in \operatorname{layerVertices}(j)$ since $p_2 = j$. For uniqueness, if $p \in \operatorname{layerVertices}(k)$, then $p_2 = k$ by the filter condition, so $k = j$.
\end{proof}

\begin{theorem}[Vertex Partition Disjoint]
\label{thm:CycleSparsification.vertex_partition_disjoint}
\lean{QEC1.CycleSparsification.vertex_partition_disjoint}
\leanok
\uses{def:CycleSparsification.layerVertices}
For distinct layers $i \neq j$, the layer vertex sets are disjoint:
\[
\operatorname{layerVertices}(i) \cap \operatorname{layerVertices}(j) = \emptyset.
\]
\end{theorem}

\begin{proof}
\leanok
\uses{def:CycleSparsification.layerVertices}
We prove disjointness via $\operatorname{Finset.disjoint\_left}$. Suppose $x \in \operatorname{layerVertices}(i)$ and $x \in \operatorname{layerVertices}(j)$. By the filter conditions, $x_2 = i$ and $x_2 = j$, hence $i = j$, contradicting $i \neq j$.
\end{proof}

\begin{definition}[Intra-Layer Edge]
\label{def:CycleSparsification.isIntraLayerEdge}
\lean{QEC1.CycleSparsification.isIntraLayerEdge}
\leanok
\uses{def:CycleSparsification.SparsifiedVertex}
An \emph{intra-layer edge} between sparsified vertices $p$ and $q$ is defined by
\[
\operatorname{isIntraLayerEdge}_G(p, q) \iff p_2 = q_2 \;\land\; G.\operatorname{Adj}(p_1, q_1),
\]
i.e., $p$ and $q$ are in the same layer and the corresponding original vertices are adjacent in $G$. The original graph edges are replicated in every layer.
\end{definition}

\begin{definition}[Inter-Layer Edge]
\label{def:CycleSparsification.isInterLayerEdge}
\lean{QEC1.CycleSparsification.isInterLayerEdge}
\leanok
\uses{def:CycleSparsification.SparsifiedVertex}
An \emph{inter-layer edge} between sparsified vertices $p$ and $q$ is defined by
\[
\operatorname{isInterLayerEdge}(p, q) \iff p_1 = q_1 \;\land\; (p_2 + 1 = q_2 \;\lor\; q_2 + 1 = p_2),
\]
i.e., they represent the same original vertex in consecutive layers.
\end{definition}

\begin{definition}[Zigzag Go (Auxiliary)]
\label{def:CycleSparsification.zigzagGo}
\lean{QEC1.CycleSparsification.zigzagGo}
\leanok

The auxiliary recursive function $\operatorname{zigzagGo}$ takes a vertex array $\mathit{vs} : \operatorname{Fin}(n) \to V$, left and right indices, a Boolean alternation flag $\mathit{fromLeft}$, and a fuel parameter. It produces zigzag diagonal pairs by alternating between:
\begin{itemize}
\item If $\mathit{fromLeft}$ is true: emit $(\mathit{vs}[\mathit{left}],\; \mathit{vs}[\mathit{right}-1])$ and recurse with $\mathit{right}' = \mathit{right} - 1$ and flag false.
\item If $\mathit{fromLeft}$ is false: emit $(\mathit{vs}[\mathit{right}],\; \mathit{vs}[\mathit{left}+1])$ and recurse with $\mathit{left}' = \mathit{left} + 1$ and flag true.
\end{itemize}
The recursion terminates when $\mathit{right} \le \mathit{left} + 1$ or fuel is exhausted.
\end{definition}

\begin{definition}[Zigzag Diagonals]
\label{def:CycleSparsification.zigzagDiagonals}
\lean{QEC1.CycleSparsification.zigzagDiagonals}
\leanok
\uses{def:CycleSparsification.zigzagGo}
Given a list of cycle vertices $[v_1, v_2, \ldots, v_m]$, the \emph{zigzag diagonals} are the cellulation diagonal pairs produced by the zigzag pattern:
\[
(v_1, v_{m-1}),\; (v_{m-1}, v_2),\; (v_2, v_{m-2}),\; (v_{m-2}, v_3),\; \ldots
\]
For $m < 4$, no diagonals are needed (the polygon is a triangle or simpler). For $m = 4$ (square), one diagonal $(v_1, v_3)$. For $m = 5$ (pentagon), two diagonals: $(v_1, v_4)$ and $(v_4, v_2)$.

Formally, if $|\mathit{vs}| \ge 4$, define $\mathit{arr}(i) = \mathit{vs}[i]$ and return $\operatorname{zigzagGo}(\mathit{arr}, 0, |\mathit{vs}|-1, \ldots, \operatorname{true}, |\mathit{vs}|)$; otherwise return the empty list.
\end{definition}

\begin{definition}[Triangulation Edge]
\label{def:CycleSparsification.isTriangulationEdge}
\lean{QEC1.CycleSparsification.isTriangulationEdge}
\leanok
\uses{def:CycleSparsification.SparsifiedVertex, def:CycleSparsification.originalLayer, def:CycleSparsification.zigzagDiagonals}
A \emph{triangulation edge} between sparsified vertices $p$ and $q$ exists when there is a cycle $c$ such that:
\begin{enumerate}
\item The cycle $c$ is assigned to a non-original layer: $\operatorname{cycleAssignment}(c) \neq \operatorname{originalLayer}$.
\item Both $p$ and $q$ are in the layer assigned to $c$: $p_2 = q_2 = \operatorname{cycleAssignment}(c)$.
\item The pair $(p_1, q_1)$ or $(q_1, p_1)$ is a zigzag diagonal of $c$'s vertex list.
\end{enumerate}
\end{definition}

\begin{theorem}[Intra-Layer Edge Symmetry]
\label{thm:CycleSparsification.isIntraLayerEdge_symm}
\lean{QEC1.CycleSparsification.isIntraLayerEdge_symm}
\leanok
\uses{def:CycleSparsification.isIntraLayerEdge}
If $(p, q)$ is an intra-layer edge, then so is $(q, p)$.
\end{theorem}

\begin{proof}
\leanok
\uses{def:CycleSparsification.isIntraLayerEdge}
From $p_2 = q_2$ we get $q_2 = p_2$ by symmetry, and from $G.\operatorname{Adj}(p_1, q_1)$ we get $G.\operatorname{Adj}(q_1, p_1)$ by the symmetry of the simple graph adjacency relation.
\end{proof}

\begin{theorem}[Inter-Layer Edge Symmetry]
\label{thm:CycleSparsification.isInterLayerEdge_symm}
\lean{QEC1.CycleSparsification.isInterLayerEdge_symm}
\leanok
\uses{def:CycleSparsification.isInterLayerEdge}
If $(p, q)$ is an inter-layer edge, then so is $(q, p)$.
\end{theorem}

\begin{proof}
\leanok
\uses{def:CycleSparsification.isInterLayerEdge}
From $p_1 = q_1$ we get $q_1 = p_1$, and the disjunction $p_2 + 1 = q_2 \lor q_2 + 1 = p_2$ is symmetric under swapping $p$ and $q$.
\end{proof}

\begin{theorem}[Triangulation Edge Symmetry]
\label{thm:CycleSparsification.isTriangulationEdge_symm}
\lean{QEC1.CycleSparsification.isTriangulationEdge_symm}
\leanok
\uses{def:CycleSparsification.isTriangulationEdge}
If $(p, q)$ is a triangulation edge, then so is $(q, p)$.
\end{theorem}

\begin{proof}
\leanok
\uses{def:CycleSparsification.isTriangulationEdge}
Decomposing the hypothesis, we obtain a cycle $c$ with $\operatorname{cycleAssignment}(c) \neq \operatorname{originalLayer}$, $p_2 = \operatorname{cycleAssignment}(c)$, $q_2 = \operatorname{cycleAssignment}(c)$, and $(p_1, q_1)$ or $(q_1, p_1)$ is a zigzag diagonal. We use the same cycle $c$, swap the layer equalities, and note that the disjunction on diagonal membership is symmetric.
\end{proof}

\begin{theorem}[Intra-Layer Edge Irreflexivity]
\label{thm:CycleSparsification.isIntraLayerEdge_irrefl}
\lean{QEC1.CycleSparsification.isIntraLayerEdge_irrefl}
\leanok
\uses{def:CycleSparsification.isIntraLayerEdge}
No vertex has an intra-layer edge to itself: $\neg\operatorname{isIntraLayerEdge}_G(p, p)$.
\end{theorem}

\begin{proof}
\leanok
\uses{def:CycleSparsification.isIntraLayerEdge}
Suppose $\operatorname{isIntraLayerEdge}_G(p, p)$ holds. Then $G.\operatorname{Adj}(p_1, p_1)$, which contradicts the loopless property of simple graphs.
\end{proof}

\begin{theorem}[Inter-Layer Edge Irreflexivity]
\label{thm:CycleSparsification.isInterLayerEdge_irrefl}
\lean{QEC1.CycleSparsification.isInterLayerEdge_irrefl}
\leanok
\uses{def:CycleSparsification.isInterLayerEdge}
No vertex has an inter-layer edge to itself: $\neg\operatorname{isInterLayerEdge}(p, p)$.
\end{theorem}

\begin{proof}
\leanok
\uses{def:CycleSparsification.isInterLayerEdge}
Suppose $\operatorname{isInterLayerEdge}(p, p)$ holds. Then the layer condition gives $p_2 + 1 = p_2$ or $p_2 + 1 = p_2$, which is impossible by integer arithmetic (omega).
\end{proof}

\begin{definition}[Sparsified Adjacency]
\label{def:CycleSparsification.sparsifiedAdj}
\lean{QEC1.CycleSparsification.sparsifiedAdj}
\leanok
\uses{def:CycleSparsification.isIntraLayerEdge, def:CycleSparsification.isInterLayerEdge, def:CycleSparsification.isTriangulationEdge}
The adjacency relation of the sparsified graph combines all three edge types:
\[
\operatorname{sparsifiedAdj}_G(p, q) \iff p \neq q \;\land\; \bigl(\operatorname{isIntraLayerEdge}_G(p,q) \;\lor\; \operatorname{isInterLayerEdge}(p,q) \;\lor\; \operatorname{isTriangulationEdge}(p,q)\bigr).
\]
\end{definition}

\begin{theorem}[Sparsified Adjacency Symmetry]
\label{thm:CycleSparsification.sparsifiedAdj_symm}
\lean{QEC1.CycleSparsification.sparsifiedAdj_symm}
\leanok
\uses{def:CycleSparsification.sparsifiedAdj, thm:CycleSparsification.isIntraLayerEdge_symm, thm:CycleSparsification.isInterLayerEdge_symm, thm:CycleSparsification.isTriangulationEdge_symm}
The sparsified adjacency relation is symmetric: if $\operatorname{sparsifiedAdj}_G(p, q)$, then $\operatorname{sparsifiedAdj}_G(q, p)$.
\end{theorem}

\begin{proof}
\leanok
\uses{thm:CycleSparsification.isIntraLayerEdge_symm, thm:CycleSparsification.isInterLayerEdge_symm, thm:CycleSparsification.isTriangulationEdge_symm}
From $p \neq q$ we get $q \neq p$. We then case split on the three edge types and apply the respective symmetry theorems: $\operatorname{isIntraLayerEdge\_symm}$, $\operatorname{isInterLayerEdge\_symm}$, or $\operatorname{isTriangulationEdge\_symm}$.
\end{proof}

\begin{theorem}[Sparsified Adjacency Irreflexivity]
\label{thm:CycleSparsification.sparsifiedAdj_irrefl}
\lean{QEC1.CycleSparsification.sparsifiedAdj_irrefl}
\leanok
\uses{def:CycleSparsification.sparsifiedAdj}
The sparsified adjacency relation is irreflexive: $\neg\operatorname{sparsifiedAdj}_G(p, p)$.
\end{theorem}

\begin{proof}
\leanok
\uses{def:CycleSparsification.sparsifiedAdj}
Suppose $\operatorname{sparsifiedAdj}_G(p, p)$ holds. The first condition requires $p \neq p$, which is a contradiction.
\end{proof}

\begin{definition}[Sparsified Graph]
\label{def:CycleSparsification.sparsifiedGraph}
\lean{QEC1.CycleSparsification.sparsifiedGraph}
\leanok
\uses{def:CycleSparsification.sparsifiedAdj, thm:CycleSparsification.sparsifiedAdj_symm, thm:CycleSparsification.sparsifiedAdj_irrefl}
The \emph{cycle-sparsified graph} $\overline{\overline{G}}$ is defined as a $\operatorname{SimpleGraph}$ on $\operatorname{SparsifiedVertex}(V, R)$ with adjacency given by $\operatorname{sparsifiedAdj}$, symmetry given by $\operatorname{sparsifiedAdj\_symm}$, and the loopless property given by $\operatorname{sparsifiedAdj\_irrefl}$.
\end{definition}

\begin{theorem}[Layer Zero Preserves Original Graph]
\label{thm:CycleSparsification.layer_zero_adj_of_original}
\lean{QEC1.CycleSparsification.layer_zero_adj_of_original}
\leanok
\uses{def:CycleSparsification.sparsifiedGraph, def:CycleSparsification.originalLayer}
If $v$ and $w$ are adjacent in the original graph $G$, then their embeddings in layer $0$ are adjacent in the sparsified graph:
\[
G.\operatorname{Adj}(v, w) \implies \overline{\overline{G}}.\operatorname{Adj}\bigl((v, \operatorname{originalLayer}),\; (w, \operatorname{originalLayer})\bigr).
\]
\end{theorem}

\begin{proof}
\leanok
\uses{def:CycleSparsification.sparsifiedGraph, def:CycleSparsification.isIntraLayerEdge}
We need to show $p \neq q$ and that an edge type holds. For distinctness, if $(v, \operatorname{originalLayer}) = (w, \operatorname{originalLayer})$ then $v = w$ by the first component, contradicting the loopless property $\neg G.\operatorname{Adj}(v, v)$. For the edge type, we use the intra-layer edge: both are in the same layer $0$ and $G.\operatorname{Adj}(v, w)$.
\end{proof}

\begin{theorem}[Inter-Layer Edges Exist]
\label{thm:CycleSparsification.interLayerEdge_exists}
\lean{QEC1.CycleSparsification.interLayerEdge_exists}
\leanok
\uses{def:CycleSparsification.sparsifiedGraph}
For any vertex $v \in V$ and layer $i < R$, the vertices $(v, i)$ and $(v, i+1)$ are adjacent in the sparsified graph.
\end{theorem}

\begin{proof}
\leanok
\uses{def:CycleSparsification.sparsifiedGraph, def:CycleSparsification.isInterLayerEdge}
Let $i_0 = \langle i, \ldots \rangle$ and $i_1 = \langle i+1, \ldots \rangle$ be the consecutive layer indices. For distinctness, $(v, i_0) = (v, i_1)$ would imply $i = i+1$ by the second component, a contradiction. For the edge type, we use the inter-layer edge: $p_1 = q_1 = v$ and $i_0 + 1 = i_1$.
\end{proof}

\begin{theorem}[Triangulation Edges are Same-Layer]
\label{thm:CycleSparsification.triangulationEdge_same_layer}
\lean{QEC1.CycleSparsification.triangulationEdge_same_layer}
\leanok
\uses{def:CycleSparsification.isTriangulationEdge}
If $(p, q)$ is a triangulation edge, then $p$ and $q$ are in the same layer: $p_2 = q_2$.
\end{theorem}

\begin{proof}
\leanok
\uses{def:CycleSparsification.isTriangulationEdge}
Decomposing the hypothesis, we get $p_2 = \operatorname{cycleAssignment}(c)$ and $q_2 = \operatorname{cycleAssignment}(c)$ for some cycle $c$. Hence $p_2 = q_2$.
\end{proof}

\begin{theorem}[Triangulation Edges are Not in Layer Zero]
\label{thm:CycleSparsification.triangulationEdge_not_layer_zero}
\lean{QEC1.CycleSparsification.triangulationEdge_not_layer_zero}
\leanok
\uses{def:CycleSparsification.isTriangulationEdge, def:CycleSparsification.originalLayer}
If $(p, q)$ is a triangulation edge, then $p_2 \neq \operatorname{originalLayer}$.
\end{theorem}

\begin{proof}
\leanok
\uses{def:CycleSparsification.isTriangulationEdge}
From the hypothesis, $p_2 = \operatorname{cycleAssignment}(c)$ and $\operatorname{cycleAssignment}(c) \neq \operatorname{originalLayer}$, so $p_2 \neq \operatorname{originalLayer}$.
\end{proof}

\begin{theorem}[Edge Trichotomy]
\label{thm:CycleSparsification.edge_trichotomy}
\lean{QEC1.CycleSparsification.edge_trichotomy}
\leanok
\uses{def:CycleSparsification.sparsifiedGraph, def:CycleSparsification.isIntraLayerEdge, def:CycleSparsification.isInterLayerEdge, def:CycleSparsification.isTriangulationEdge}
Every edge in the sparsified graph is either an intra-layer edge, an inter-layer edge, or a triangulation edge:
\[
\overline{\overline{G}}.\operatorname{Adj}(p, q) \implies \operatorname{isIntraLayerEdge}_G(p,q) \;\lor\; \operatorname{isInterLayerEdge}(p,q) \;\lor\; \operatorname{isTriangulationEdge}(p,q).
\]
\end{theorem}

\begin{proof}
\leanok
\uses{def:CycleSparsification.sparsifiedAdj}
This follows directly from the second component of $\operatorname{sparsifiedAdj}$, which is the disjunction of the three edge types.
\end{proof}

\begin{theorem}[Different Layers Not Intra-Layer]
\label{thm:CycleSparsification.diff_layer_not_intra}
\lean{QEC1.CycleSparsification.diff_layer_not_intra}
\leanok
\uses{def:CycleSparsification.isIntraLayerEdge}
If $p_2 \neq q_2$, then $(p, q)$ is not an intra-layer edge.
\end{theorem}

\begin{proof}
\leanok
\uses{def:CycleSparsification.isIntraLayerEdge}
Suppose $\operatorname{isIntraLayerEdge}_G(p, q)$ holds. Then $p_2 = q_2$, contradicting the hypothesis $p_2 \neq q_2$.
\end{proof}

\begin{theorem}[Different Vertices Not Inter-Layer]
\label{thm:CycleSparsification.diff_vertex_not_inter}
\lean{QEC1.CycleSparsification.diff_vertex_not_inter}
\leanok
\uses{def:CycleSparsification.isInterLayerEdge}
If $p_1 \neq q_1$, then $(p, q)$ is not an inter-layer edge.
\end{theorem}

\begin{proof}
\leanok
\uses{def:CycleSparsification.isInterLayerEdge}
Suppose $\operatorname{isInterLayerEdge}(p, q)$ holds. Then $p_1 = q_1$, contradicting the hypothesis $p_1 \neq q_1$.
\end{proof}

\begin{theorem}[Squares are 4-Cycles]
\label{thm:CycleSparsification.square_is_cycle}
\lean{QEC1.CycleSparsification.square_is_cycle}
\leanok
\uses{def:CycleSparsification.sparsifiedGraph}
Given an edge $(v, u) \in E(G)$ and a layer index $i < R$, the four vertices $(v, i)$, $(v, i{+}1)$, $(u, i{+}1)$, $(u, i)$ form a 4-cycle in the sparsified graph:
\begin{align*}
&\overline{\overline{G}}.\operatorname{Adj}\bigl((v, i),\; (v, i{+}1)\bigr) \;\land\; \overline{\overline{G}}.\operatorname{Adj}\bigl((v, i{+}1),\; (u, i{+}1)\bigr) \\
\land\; &\overline{\overline{G}}.\operatorname{Adj}\bigl((u, i{+}1),\; (u, i)\bigr) \;\land\; \overline{\overline{G}}.\operatorname{Adj}\bigl((u, i),\; (v, i)\bigr).
\end{align*}
\end{theorem}

\begin{proof}
\leanok
\uses{def:CycleSparsification.sparsifiedAdj, def:CycleSparsification.isIntraLayerEdge, def:CycleSparsification.isInterLayerEdge}
First note $v \neq u$, since $G.\operatorname{Adj}(v, u)$ and $G$ is loopless. We verify each of the four edges:
\begin{enumerate}
\item $(v, i) \to (v, i{+}1)$: distinctness holds since $i \neq i+1$; this is an inter-layer edge with $p_1 = q_1 = v$ and $p_2 + 1 = q_2$.
\item $(v, i{+}1) \to (u, i{+}1)$: distinctness holds since $v \neq u$; this is an intra-layer edge with same layer $i+1$ and $G.\operatorname{Adj}(v, u)$.
\item $(u, i{+}1) \to (u, i)$: distinctness holds since $i+1 \neq i$; this is an inter-layer edge with $p_1 = q_1 = u$ and $q_2 + 1 = p_2$.
\item $(u, i) \to (v, i)$: distinctness holds since $u \neq v$; this is an intra-layer edge with same layer $i$ and $G.\operatorname{Adj}(u, v)$ (by symmetry of $G$).
\end{enumerate}
\end{proof}

\begin{theorem}[Zigzag Go Produces Cycle Members]
\label{thm:CycleSparsification.zigzagGo_mem_range}
\lean{QEC1.CycleSparsification.zigzagGo_mem_range}
\leanok
\uses{def:CycleSparsification.zigzagGo}
Every pair $(a, b)$ in the output of $\operatorname{zigzagGo}(\mathit{vs}, \mathit{left}, \mathit{right}, \ldots)$ satisfies: $a = \mathit{vs}(i)$ and $b = \mathit{vs}(j)$ for some indices $i, j \in \operatorname{Fin}(n)$.
\end{theorem}

\begin{proof}
\leanok
\uses{def:CycleSparsification.zigzagGo}
We proceed by induction on the fuel parameter. In the base case (fuel $= 0$), $\operatorname{zigzagGo}$ returns the empty list, so there is nothing to prove. In the inductive step (fuel $= k + 1$), we first check if $\mathit{right} \le \mathit{left} + 1$; if so, the result is empty. Otherwise, we split on the $\mathit{fromLeft}$ flag. In the fromLeft case, the emitted pair is $(\mathit{vs}[\mathit{left}], \mathit{vs}[\mathit{right}-1])$, which clearly consists of values of $\mathit{vs}$; for the remaining pairs we apply the induction hypothesis. The fromRight case is symmetric: the emitted pair is $(\mathit{vs}[\mathit{right}], \mathit{vs}[\mathit{left}+1])$, and the tail follows by the induction hypothesis.
\end{proof}

\begin{theorem}[Zigzag Diagonals are Cycle Members]
\label{thm:CycleSparsification.zigzagDiagonals_mem}
\lean{QEC1.CycleSparsification.zigzagDiagonals_mem}
\leanok
\uses{def:CycleSparsification.zigzagDiagonals, thm:CycleSparsification.zigzagGo_mem_range}
Every vertex appearing in a zigzag diagonal pair belongs to the original cycle list: if $(a, b) \in \operatorname{zigzagDiagonals}(\mathit{vs})$, then $a \in \mathit{vs}$ and $b \in \mathit{vs}$.
\end{theorem}

\begin{proof}
\leanok
\uses{def:CycleSparsification.zigzagDiagonals, thm:CycleSparsification.zigzagGo_mem_range}
We unfold $\operatorname{zigzagDiagonals}$. If $|\mathit{vs}| < 4$, the result is empty and there is nothing to prove. Otherwise, we apply $\operatorname{zigzagGo\_mem\_range}$ to obtain indices $i, j$ with $\mathit{vs}(i) = a$ and $\mathit{vs}(j) = b$. Since $\mathit{vs}(i)$ is the $i$-th element of $\mathit{vs}$, it is a member of the list, and similarly for $j$.
\end{proof}

\begin{theorem}[Zigzag Go Nil When Gap Closed]
\label{thm:CycleSparsification.zigzagGo_nil}
\lean{QEC1.CycleSparsification.zigzagGo_nil}
\leanok
\uses{def:CycleSparsification.zigzagGo}
When the gap is closed ($\mathit{right} \le \mathit{left} + 1$), $\operatorname{zigzagGo}$ returns the empty list.
\end{theorem}

\begin{proof}
\leanok
\uses{def:CycleSparsification.zigzagGo}
We case split on fuel. If fuel $= 0$, $\operatorname{zigzagGo}$ returns $[]$ by definition. If fuel $= k + 1$, the first branch of the match checks $\mathit{right} \le \mathit{left} + 1$, which holds by hypothesis, so it returns $[]$.
\end{proof}

\begin{theorem}[Exact Length of Zigzag Go]
\label{thm:CycleSparsification.zigzagGo_exact_length}
\lean{QEC1.CycleSparsification.zigzagGo_exact_length}
\leanok
\uses{def:CycleSparsification.zigzagGo, thm:CycleSparsification.zigzagGo_nil}
When $\mathit{left} + 1 < \mathit{right}$ and fuel $\ge \mathit{right} - \mathit{left} - 1$, the output of $\operatorname{zigzagGo}$ has exactly $\mathit{right} - \mathit{left} - 1$ pairs.
\end{theorem}

\begin{proof}
\leanok
\uses{def:CycleSparsification.zigzagGo, thm:CycleSparsification.zigzagGo_nil}
We proceed by induction on fuel. The base case (fuel $= 0$) is impossible since $\mathit{right} - \mathit{left} - 1 \le 0$ contradicts $\mathit{left} + 1 < \mathit{right}$ (by omega). For the inductive step (fuel $= k + 1$), we unfold $\operatorname{zigzagGo}$ and note the condition $\mathit{right} \le \mathit{left} + 1$ is false. We split on the $\mathit{fromLeft}$ flag:
\begin{itemize}
\item If fromLeft: one pair is emitted and we recurse with $\mathit{right}' = \mathit{right} - 1$. If $\mathit{left} + 1 < \mathit{right} - 1$, the induction hypothesis gives length $\mathit{right} - 1 - \mathit{left} - 1$, so total length is $1 + (\mathit{right} - \mathit{left} - 2) = \mathit{right} - \mathit{left} - 1$. If $\mathit{left} + 1 \ge \mathit{right} - 1$, then $\operatorname{zigzagGo\_nil}$ gives length $0$, so total length is $1$, and $\mathit{right} - \mathit{left} - 1 = 1$ by arithmetic.
\item If not fromLeft: symmetric argument with $\mathit{left}' = \mathit{left} + 1$.
\end{itemize}
\end{proof}

\begin{theorem}[Zigzag Diagonals Length]
\label{thm:CycleSparsification.zigzagDiagonals_length}
\lean{QEC1.CycleSparsification.zigzagDiagonals_length}
\leanok
\uses{def:CycleSparsification.zigzagDiagonals, thm:CycleSparsification.zigzagGo_exact_length}
For a cycle of length $m \ge 4$, the zigzag cellulation produces exactly $m - 2$ diagonals:
\[
|\operatorname{zigzagDiagonals}(\mathit{vs})| = |\mathit{vs}| - 2.
\]
This corresponds to decomposing an $m$-gon into $m - 2$ triangles.
\end{theorem}

\begin{proof}
\leanok
\uses{def:CycleSparsification.zigzagDiagonals, thm:CycleSparsification.zigzagGo_exact_length}
We unfold $\operatorname{zigzagDiagonals}$ and note $|\mathit{vs}| \ge 4$, so the branch selecting the non-empty case is taken. We apply $\operatorname{zigzagGo\_exact\_length}$ with $\mathit{left} = 0$, $\mathit{right} = |\mathit{vs}| - 1$, and fuel $= |\mathit{vs}|$. The conditions $0 + 1 < |\mathit{vs}| - 1$ and $|\mathit{vs}| - 1 - 0 - 1 \le |\mathit{vs}|$ are verified by omega. The result is $(|\mathit{vs}| - 1) - 0 - 1 = |\mathit{vs}| - 2$.
\end{proof}

\begin{theorem}[Cellulation Adjacency from Diagonal]
\label{thm:CycleSparsification.cellulation_adj_of_diagonal}
\lean{QEC1.CycleSparsification.cellulation_adj_of_diagonal}
\leanok
\uses{def:CycleSparsification.sparsifiedGraph, def:CycleSparsification.zigzagDiagonals, def:CycleSparsification.isTriangulationEdge}
For a cycle $c$ assigned to layer $i > 0$, if $(a, b)$ is a zigzag diagonal of $c$ with $a \neq b$, then $(a, i)$ and $(b, i)$ are adjacent in the sparsified graph.
\end{theorem}

\begin{proof}
\leanok
\uses{def:CycleSparsification.sparsifiedAdj, def:CycleSparsification.isTriangulationEdge}
Distinctness $(a, i) \neq (b, i)$ follows from $a \neq b$ via the first component. The edge is a triangulation edge witnessed by the cycle $c$: $\operatorname{cycleAssignment}(c) \neq \operatorname{originalLayer}$, both vertices are in layer $i = \operatorname{cycleAssignment}(c)$, and $(a, b)$ is a zigzag diagonal of $c$.
\end{proof}

\begin{theorem}[Cellulation Adjacency from Cycle Edge]
\label{thm:CycleSparsification.cellulation_adj_of_cycle_edge}
\lean{QEC1.CycleSparsification.cellulation_adj_of_cycle_edge}
\leanok
\uses{def:CycleSparsification.sparsifiedGraph, def:CycleSparsification.isIntraLayerEdge}
If consecutive vertices $u, v$ of a cycle are adjacent in $G$ with $u \neq v$, then $(u, i)$ and $(v, i)$ are adjacent in the sparsified graph for any layer $i$.
\end{theorem}

\begin{proof}
\leanok
\uses{def:CycleSparsification.sparsifiedAdj, def:CycleSparsification.isIntraLayerEdge}
Distinctness $(u, i) \neq (v, i)$ follows from $u \neq v$. This is an intra-layer edge: both vertices are in layer $i$ and $G.\operatorname{Adj}(u, v)$.
\end{proof}

\begin{theorem}[Cellulation Triangle Existence]
\label{thm:CycleSparsification.cellulation_triangle_exists}
\lean{QEC1.CycleSparsification.cellulation_triangle_exists}
\leanok
\uses{def:CycleSparsification.sparsifiedGraph, thm:CycleSparsification.cellulation_adj_of_diagonal, thm:CycleSparsification.cellulation_adj_of_cycle_edge}
For a cycle $c$ assigned to layer $i > 0$, if $(a, b)$ is a zigzag diagonal of $c$ and $x$ is a common neighbor with $G.\operatorname{Adj}(a, x)$ and $G.\operatorname{Adj}(b, x)$, then the three vertices $(a, i)$, $(b, i)$, $(x, i)$ form a triangle in the sparsified graph:
\[
\overline{\overline{G}}.\operatorname{Adj}\bigl((a,i), (b,i)\bigr) \;\land\; \overline{\overline{G}}.\operatorname{Adj}\bigl((a,i), (x,i)\bigr) \;\land\; \overline{\overline{G}}.\operatorname{Adj}\bigl((b,i), (x,i)\bigr).
\]
This establishes the paper's claim that cellulation decomposes cycles into triangles of weight $3$.
\end{theorem}

\begin{proof}
\leanok
\uses{thm:CycleSparsification.cellulation_adj_of_diagonal, thm:CycleSparsification.cellulation_adj_of_cycle_edge}
The three edges are verified by applying the previous results:
\begin{enumerate}
\item $(a, i) \sim (b, i)$: by $\operatorname{cellulation\_adj\_of\_diagonal}$ applied to the diagonal $(a, b)$ with $a \neq b$ and $\operatorname{cycleAssignment}(c) \neq \operatorname{originalLayer}$.
\item $(a, i) \sim (x, i)$: by $\operatorname{cellulation\_adj\_of\_cycle\_edge}$ applied to $a \neq x$ and $G.\operatorname{Adj}(a, x)$.
\item $(b, i) \sim (x, i)$: by $\operatorname{cellulation\_adj\_of\_cycle\_edge}$ applied to $b \neq x$ and $G.\operatorname{Adj}(b, x)$.
\end{enumerate}
\end{proof}

\begin{definition}[Edge Cycle Degree in Layer]
\label{def:CycleSparsification.edgeCycleDegreeInLayer}
\lean{QEC1.CycleSparsification.edgeCycleDegreeInLayer}
\leanok

The \emph{edge cycle degree in layer $i$} counts how many original cycles assigned to layer $i$ contain a given edge $e$:
\[
\operatorname{edgeCycleDegreeInLayer}(\mathit{cycles}, f, e, i) = \bigl|\{c \in C \mid f(c) = i \;\land\; e \in \mathit{cycles}(c)\}\bigr|.
\]
This is the per-layer edge participation count; the purpose of distributing cycles across layers is to make this quantitysmall.
\end{definition}

\begin{definition}[Layer Cycle Degree Bound]
\label{def:CycleSparsification.LayerCycleDegreeBound}
\lean{QEC1.CycleSparsification.LayerCycleDegreeBound}
\leanok
\uses{def:CycleSparsification.edgeCycleDegreeInLayer}
The \emph{per-layer cycle-degree bound} states that for every edge $e$ and every layer $i$, the number of cycles assigned to layer $i$ containing edge $e$ is at most $\mathit{bound}$:
\[
\operatorname{LayerCycleDegreeBound}(\mathit{cycles}, f, \mathit{bound}) \iff \forall e,\, \forall i,\;\; \operatorname{edgeCycleDegreeInLayer}(\mathit{cycles}, f, e, i) \le \mathit{bound}.
\]
\end{definition}

\begin{definition}[Edge Cycle Degree (Global)]
\label{def:CycleSparsification.edgeCycleDegree}
\lean{QEC1.CycleSparsification.edgeCycleDegree}
\leanok

The \emph{global edge-cycle degree} is the total number of generating cycles containing edge $e$ across all layers:
\[
\operatorname{edgeCycleDegree}(\mathit{cycles}, e) = \bigl|\{c \in C \mid e \in \mathit{cycles}(c)\}\bigr|.
\]
\end{definition}

\begin{theorem}[Edge Cycle Degree Bounded by Number of Cycles]
\label{thm:CycleSparsification.edgeCycleDegree_le_card}
\lean{QEC1.CycleSparsification.edgeCycleDegree_le_card}
\leanok
\uses{def:CycleSparsification.edgeCycleDegree}
The global edge-cycle degree is bounded by the total number of cycles:
\[
\operatorname{edgeCycleDegree}(\mathit{cycles}, e) \le |C|.
\]
\end{theorem}

\begin{proof}
\leanok
\uses{def:CycleSparsification.edgeCycleDegree}
This follows from $\operatorname{Finset.card\_filter\_le}$: the cardinality of a filtered subset of $\operatorname{univ}$ is at most $|\operatorname{univ}| = |C|$.
\end{proof}

\begin{theorem}[Layer Bound Implies Global Bound]
\label{thm:CycleSparsification.layerBound_implies_global_bound}
\lean{QEC1.CycleSparsification.layerBound_implies_global_bound}
\leanok
\uses{def:CycleSparsification.LayerCycleDegreeBound, def:CycleSparsification.edgeCycleDegree, def:CycleSparsification.edgeCycleDegreeInLayer}
If the per-layer cycle-degree bound is $\mathit{bound}$, then the global edge-cycle degree satisfies:
\[
\operatorname{edgeCycleDegree}(\mathit{cycles}, e) \le (R' + 1) \cdot \mathit{bound}.
\]
\end{theorem}

\begin{proof}
\leanok
\uses{def:CycleSparsification.edgeCycleDegree, def:CycleSparsification.edgeCycleDegreeInLayer, def:CycleSparsification.LayerCycleDegreeBound}
We unfold $\operatorname{edgeCycleDegree}$ and rewrite the set $\{c \mid e \in \mathit{cycles}(c)\}$ as the union over all layers $i$ of $\{c \mid f(c) = i \land e \in \mathit{cycles}(c)\}$. By extensionality: forward, given $e \in \mathit{cycles}(c)$, take $i = f(c)$; backward, the condition $e \in \mathit{cycles}(c)$ is preserved.

After rewriting, we compute:
\begin{align*}
\bigl|\bigcup_{i \in \operatorname{Fin}(R'+1)} \{c \mid f(c) = i \land e \in \mathit{cycles}(c)\}\bigr|
&\le \sum_{i \in \operatorname{Fin}(R'+1)} \bigl|\{c \mid f(c) = i \land e \in \mathit{cycles}(c)\}\bigr| \\
&\le \sum_{i \in \operatorname{Fin}(R'+1)} \mathit{bound} \\
&= (R'+1) \cdot \mathit{bound},
\end{align*}
where the first inequality is $\operatorname{Finset.card\_biUnion\_le}$, the second uses the per-layer bound hypothesis $\operatorname{hbound}(e, i)$, and the final equality follows from the sum of a constant over $\operatorname{Fin}(R'+1)$.
\end{proof}

\begin{definition}[Sparsification Data]
\label{def:CycleSparsification.SparsificationData}
\lean{QEC1.CycleSparsification.SparsificationData}
\leanok
\uses{def:CycleSparsification.LayerCycleDegreeBound}
The complete data for a cycle-sparsified graph construction is a structure consisting of:
\begin{itemize}
\item $\operatorname{numLayers} : \mathbb{N}$ --- the number of additional layers $R$.
\item $\operatorname{cycleAssignment} : C \to \operatorname{Fin}(R+1)$ --- assigns each cycle to a layer.
\item $\operatorname{cycleVerts} : C \to \operatorname{List}(V)$ --- the vertex list of each cycle.
\item $\operatorname{cycleBound} : \mathbb{N}$ --- the per-layer cycle-degree bound.
\item $\operatorname{bound\_holds}$ --- proof that $\operatorname{LayerCycleDegreeBound}(\mathit{cycles}, \operatorname{cycleAssignment}, \operatorname{cycleBound})$ holds.
\end{itemize}
\end{definition}

\begin{definition}[Sparsification Data Graph]
\label{def:CycleSparsification.SparsificationData.graph}
\lean{QEC1.CycleSparsification.SparsificationData.graph}
\leanok
\uses{def:CycleSparsification.SparsificationData, def:CycleSparsification.sparsifiedGraph}
Given sparsification data, the associated sparsified graph is
\[
\operatorname{data.graph} = \operatorname{sparsifiedGraph}(G, \operatorname{data.cycleAssignment}, \operatorname{data.cycleVerts}).
\]
\end{definition}

\begin{definition}[Minimum Number of Layers]
\label{def:CycleSparsification.minLayers}
\lean{QEC1.CycleSparsification.minLayers}
\leanok
\uses{def:CycleSparsification.LayerCycleDegreeBound}
The \emph{minimum number of layers} $R_G^c$ for cycle-sparsification with per-layer cycle-degree bound $\mathit{bound}$ is the smallest $R$ such that there exists a cycle assignment $f : C \to \operatorname{Fin}(R+1)$ achieving $\operatorname{LayerCycleDegreeBound}(\mathit{cycles}, f, \mathit{bound})$. Formally, it is defined using $\operatorname{Nat.find}$ applied to the existence hypothesis.
\end{definition}

\begin{theorem}[Minimum Layers Specification]
\label{thm:CycleSparsification.minLayers_spec}
\lean{QEC1.CycleSparsification.minLayers_spec}
\leanok
\uses{def:CycleSparsification.minLayers, def:CycleSparsification.LayerCycleDegreeBound}
The minimum layers value achieves the bound: there exists an assignment $f : C \to \operatorname{Fin}(\operatorname{minLayers} + 1)$ such that $\operatorname{LayerCycleDegreeBound}(\mathit{cycles}, f, \mathit{bound})$.
\end{theorem}

\begin{proof}
\leanok
\uses{def:CycleSparsification.minLayers}
We unfold $\operatorname{minLayers}$ and apply $\operatorname{Nat.find\_spec}$ to the existence hypothesis.
\end{proof}

\begin{theorem}[Minimum Layers Minimality]
\label{thm:CycleSparsification.minLayers_minimal}
\lean{QEC1.CycleSparsification.minLayers_minimal}
\leanok
\uses{def:CycleSparsification.minLayers, def:CycleSparsification.LayerCycleDegreeBound}
No smaller number of layers suffices: for any $R' < \operatorname{minLayers}$,
\[
\neg \exists f : C \to \operatorname{Fin}(R'+1),\; \operatorname{LayerCycleDegreeBound}(\mathit{cycles}, f, \mathit{bound}).
\]
\end{theorem}

\begin{proof}
\leanok
\uses{def:CycleSparsification.minLayers}
We unfold $\operatorname{minLayers}$ and apply $\operatorname{Nat.find\_min}$ to $R' < \operatorname{minLayers}$.
\end{proof}

\begin{theorem}[Zigzag Diagonals Short]
\label{thm:CycleSparsification.zigzagDiagonals_short}
\lean{QEC1.CycleSparsification.zigzagDiagonals_short}
\leanok
\uses{def:CycleSparsification.zigzagDiagonals}
For lists of length less than $4$, no diagonals are produced: if $|\mathit{vs}| < 4$, then $\operatorname{zigzagDiagonals}(\mathit{vs}) = []$.
\end{theorem}

\begin{proof}
\leanok
\uses{def:CycleSparsification.zigzagDiagonals}
By simplification: the definition of $\operatorname{zigzagDiagonals}$ checks $|\mathit{vs}| < 4$ and returns $[]$ in that case.
\end{proof}

\begin{theorem}[Zigzag Go Length Bound]
\label{thm:CycleSparsification.zigzagGo_length_le}
\lean{QEC1.CycleSparsification.zigzagGo_length_le}
\leanok
\uses{def:CycleSparsification.zigzagGo}
The $\operatorname{zigzagGo}$ function produces at most $\mathit{fuel}$ diagonal pairs:
\[
|\operatorname{zigzagGo}(\mathit{vs}, \mathit{left}, \mathit{right}, \ldots, \mathit{fuel})| \le \mathit{fuel}.
\]
\end{theorem}

\begin{proof}
\leanok
\uses{def:CycleSparsification.zigzagGo}
We proceed by induction on fuel, generalizing over left, right, and the Boolean flag.
\begin{itemize}
\item Base case (fuel $= 0$): $\operatorname{zigzagGo}$ returns $[]$, which has length $0 \le 0$.
\item Inductive step (fuel $= k + 1$): We unfold $\operatorname{zigzagGo}$. If $\mathit{right} \le \mathit{left} + 1$, the result is $[]$ with length $0 \le k + 1$. Otherwise, whether fromLeft or not, one pair is emitted and the tail has length $\le k$ by the induction hypothesis, so the total length is $\le k + 1$.
\end{itemize}
\end{proof}

%--- Lem_2: DecongestionLemmaBound ---
\chapter{Lem 2: Decongestion Lemma Bound}

This chapter establishes the Decongestion Lemma bound from Freedman--Hastings. For any constant-degree expander graph $G$ with $W$ vertices and degree bound $\Delta$, the number of layers $R_G^c$ needed for cycle-sparsification with cycle-degree bound $c$ satisfies $R_G^c = O(\log^2 W)$, where the implicit constant depends on $\Delta$ and $c$ but not on $W$.

The proof proceeds through: (1) edge count and cycle space bounds for $\Delta$-bounded graphs, (2) analysis of the maximum edge-cycle degree $M$, (3) greedy packing from an $M$-coloring via K\"onig's theorem, and (4) the Freedman--Hastings bound $M = O(\log^2 W)$ for expanders.

\section{Edge Count Bounds for Constant-Degree Graphs}

\begin{theorem}[Edge Count Upper Bound]
\label{thm:DecongestionLemma.edge_count_upper_bound}
\lean{QEC1.DecongestionLemma.edge_count_upper_bound}
\leanok
\uses{def:DesiderataForGraphG.ConstantDegree, thm:DesiderataForGraphG.constant_degree_bounds_edges}
For a constant-degree graph $G$ with degree bound $\Delta$, we have
\[
  2 |E| \le \Delta \cdot |V|.
\]
\end{theorem}

\begin{proof}
\leanok
\uses{thm:DesiderataForGraphG.constant_degree_bounds_edges}
This follows directly from the constant degree edge bound (Theorem~\ref{thm:DesiderataForGraphG.constant_degree_bounds_edges}).
\end{proof}

\begin{theorem}[Edge Count Half Bound]
\label{thm:DecongestionLemma.edge_count_half_bound}
\lean{QEC1.DecongestionLemma.edge_count_half_bound}
\leanok
\uses{def:DesiderataForGraphG.ConstantDegree, thm:DesiderataForGraphG.constant_degree_bounds_edges}
For a constant-degree graph $G$ with degree bound $\Delta$, we have
\[
  |E| \le \frac{\Delta \cdot |V|}{2}.
\]
\end{theorem}

\begin{proof}
\leanok
\uses{thm:DesiderataForGraphG.constant_degree_bounds_edges}
From the edge count upper bound we have $2|E| \le \Delta \cdot |V|$. The result follows by integer arithmetic (dividing both sides by $2$).
\end{proof}

\section{Cycle Space Dimension Bounds}

\begin{theorem}[Cycle Count Bounded by Edges]
\label{thm:DecongestionLemma.cycle_count_le_edges}
\lean{QEC1.DecongestionLemma.cycle_count_le_edges}
\leanok
\uses{def:CycleSparsification.edgeCycleDegree}
Let $C$ be a type of cycles for a graph $G$ satisfying the cycle rank property $|C| + |V| = |E| + 1$. If $|V| > 0$, then $|C| \le |E|$.
\end{theorem}

\begin{proof}
\leanok

From the hypothesis $|C| + |V| = |E| + 1$ and $|V| > 0$, the result follows by integer arithmetic ($|C| = |E| + 1 - |V| \le |E|$).
\end{proof}

\begin{theorem}[Cycle Count Bound for Constant-Degree Graphs]
\label{thm:DecongestionLemma.cycle_count_bound}
\lean{QEC1.DecongestionLemma.cycle_count_bound}
\leanok
\uses{def:DesiderataForGraphG.ConstantDegree, thm:DesiderataForGraphG.constant_degree_bounds_edges}
For a $\Delta$-bounded graph $G$ satisfying the cycle rank property $|C| + |V| = |E| + 1$ with $|V| > 0$, we have
\[
  |C| \le \frac{\Delta \cdot |V|}{2}.
\]
\end{theorem}

\begin{proof}
\leanok
\uses{thm:DesiderataForGraphG.constant_degree_bounds_edges}
We first establish $|C| \le |E|$ from the cycle rank property. Then we apply the edge count half bound $|E| \le \Delta \cdot |V| / 2$. The result follows by integer arithmetic.
\end{proof}

\section{Maximum Edge-Cycle Degree}

\begin{definition}[Maximum Edge-Cycle Degree]
\label{def:DecongestionLemma.maxEdgeCycleDegree}
\lean{QEC1.DecongestionLemma.maxEdgeCycleDegree}
\leanok
\uses{def:CycleSparsification.edgeCycleDegree}
Given a collection of cycles $\{\gamma_c\}_{c \in C}$ over edges $\alpha$, the \emph{maximum edge-cycle degree} is
\[
  M = \max_{e \in \alpha}\, d_e,
\]
where $d_e = |\{c \in C : e \in \gamma_c\}|$ is the edge-cycle degree of $e$.
\end{definition}

\begin{theorem}[Edge-Cycle Degree Bounded by Maximum]
\label{thm:DecongestionLemma.le_maxEdgeCycleDegree}
\lean{QEC1.DecongestionLemma.le_maxEdgeCycleDegree}
\leanok
\uses{def:CycleSparsification.edgeCycleDegree, def:DecongestionLemma.maxEdgeCycleDegree}
For each edge $e$, the edge-cycle degree satisfies $d_e \le M$, where $M$ is the maximum edge-cycle degree.
\end{theorem}

\begin{proof}
\leanok
\uses{def:DecongestionLemma.maxEdgeCycleDegree}
This follows from the definition of $M$ as the supremum of $d_e$ over all edges $e$, applied at the element $e \in \operatorname{univ}$.
\end{proof}

\begin{theorem}[Maximum Edge-Cycle Degree Bounded by $|C|$]
\label{thm:DecongestionLemma.maxEdgeCycleDegree_le_card}
\lean{QEC1.DecongestionLemma.maxEdgeCycleDegree_le_card}
\leanok
\uses{def:CycleSparsification.edgeCycleDegree, def:DecongestionLemma.maxEdgeCycleDegree, thm:CycleSparsification.edgeCycleDegree_le_card}
The maximum edge-cycle degree satisfies $M \le |C|$.
\end{theorem}

\begin{proof}
\leanok
\uses{thm:CycleSparsification.edgeCycleDegree_le_card, def:DecongestionLemma.maxEdgeCycleDegree}
We apply $\operatorname{sup\_le}$: for each edge $e$, we have $d_e \le |C|$ by Theorem~\ref{thm:CycleSparsification.edgeCycleDegree_le_card}.
\end{proof}

\section{Layer Assignment and Greedy Packing}

\begin{theorem}[Greedy Layer Assignment Exists]
\label{thm:DecongestionLemma.greedy_layer_assignment_exists}
\lean{QEC1.DecongestionLemma.greedy_layer_assignment_exists}
\leanok
\uses{def:CycleSparsification.LayerCycleDegreeBound, def:CycleSparsification.edgeCycleDegreeInLayer}
For any collection of cycles and any positive per-layer bound $b > 0$, there exists $R \le |C|$ and a layer assignment $f : C \to \operatorname{Fin}(R+1)$ satisfying the layer-cycle-degree bound $b$. (One cycle per layer.)
\end{theorem}

\begin{proof}
\leanok
\uses{def:CycleSparsification.LayerCycleDegreeBound}
We use the one-per-layer construction: let $\operatorname{equiv} : C \simeq \operatorname{Fin}(|C|)$ be the canonical equivalence and assign each cycle $c$ to layer $\operatorname{equiv}(c)$. Then $R = |C|$ and $R \le |C|$ trivially. For the layer-cycle-degree bound, observe that each layer contains at most one cycle (by injectivity of $\operatorname{equiv}$), so the per-layer edge-cycle degree is at most $1 \le b$.
\end{proof}

\begin{theorem}[Layer Packing Lemma]
\label{thm:DecongestionLemma.layer_packing}
\lean{QEC1.DecongestionLemma.layer_packing}
\leanok
\uses{def:CycleSparsification.LayerCycleDegreeBound, def:CycleSparsification.edgeCycleDegreeInLayer}
Given a layer assignment $f_0 : C \to \operatorname{Fin}(M+1)$ with per-layer bound $1$ and a positive integer $c > 0$, there exists a layer assignment $f : C \to \operatorname{Fin}(M/c + 1)$ with per-layer bound $c$.
\end{theorem}

\begin{proof}
\leanok
\uses{def:CycleSparsification.LayerCycleDegreeBound, def:CycleSparsification.edgeCycleDegreeInLayer}
Define the packed assignment $f(\gamma) = \lfloor f_0(\gamma) / c \rfloor$. This maps into $\operatorname{Fin}(M/c + 1)$ since $f_0(\gamma) / c \le M / c$.

To verify the per-layer bound $c$, fix an edge $e$ and a packed layer $i$. Let $S$ be the set of cycles $\gamma$ with $f(\gamma) = i$ and $e \in \gamma$. We construct an injection $\varphi : S \to \operatorname{Fin}(c)$ by $\varphi(\gamma) = f_0(\gamma) \bmod c$.

To show $\varphi$ is injective on $S$: suppose $\gamma_1, \gamma_2 \in S$ with $f_0(\gamma_1) \bmod c = f_0(\gamma_2) \bmod c$. Since both are in layer $i$, we have $f_0(\gamma_1) / c = f_0(\gamma_2) / c$. By the division algorithm (combining the equal quotients and equal remainders), $f_0(\gamma_1) = f_0(\gamma_2)$, so $f_0(\gamma_1) = f_0(\gamma_2)$ as elements of $\operatorname{Fin}(M+1)$. Since the original assignment $f_0$ has per-layer bound $1$, the set of cycles in the original layer $f_0(\gamma_1)$ passing through $e$ has cardinality at most $1$. Both $\gamma_1$ and $\gamma_2$ belong to this set, so $\gamma_1 = \gamma_2$.

Therefore $|S| \le |\operatorname{Fin}(c)| = c$.
\end{proof}

\begin{theorem}[Greedy Packing Bound]
\label{thm:DecongestionLemma.greedy_packing_bound}
\lean{QEC1.DecongestionLemma.greedy_packing_bound}
\leanok
\uses{def:CycleSparsification.LayerCycleDegreeBound, def:CycleSparsification.edgeCycleDegreeInLayer}
Given an initial assignment $f_0 : C \to \operatorname{Fin}(M+1)$ with per-layer bound $1$ (from K\"onig's theorem on bipartite edge coloring) and a positive integer $c > 0$, packing $c$ consecutive layers gives
\[
  \exists R,\, \exists f : C \to \operatorname{Fin}(R+1),\quad \text{LayerCycleDegreeBound}(f, c) \;\wedge\; R \le M / c.
\]
\end{theorem}

\begin{proof}
\leanok
\uses{def:CycleSparsification.LayerCycleDegreeBound}
We obtain $f$ from the layer packing lemma (Theorem~\ref{thm:DecongestionLemma.layer_packing}) applied to $f_0$, $M$, and $c$. This gives a layer assignment with per-layer bound $c$ using $M/c + 1$ layers, hence $R = M/c \le M/c$.
\end{proof}

\begin{theorem}[Decongestion Layers Bounded by $|C|$]
\label{thm:DecongestionLemma.decongestion_layers_le_card}
\lean{QEC1.DecongestionLemma.decongestion_layers_le_card}
\leanok
\uses{def:CycleSparsification.LayerCycleDegreeBound, def:CycleSparsification.minLayers}
If a layer assignment with per-layer bound $b > 0$ exists, then the minimum number of layers $R_G^b = \operatorname{minLayers}(\text{cycles}, b)$ satisfies $R_G^b \le |C|$.
\end{theorem}

\begin{proof}
\leanok
\uses{def:CycleSparsification.minLayers, def:CycleSparsification.LayerCycleDegreeBound}
We unfold $\operatorname{minLayers}$ as $\operatorname{Nat.find}$. By the one-per-layer construction, there exists a layer assignment with $|C|$ layers achieving any positive per-layer bound. Applying $\operatorname{Nat.find\_le}$ at $|C|$ gives the result.
\end{proof}

\section{Coarse Bound}

\begin{theorem}[Coarse Decongestion Bound]
\label{thm:DecongestionLemma.decongestion_coarse_bound}
\lean{QEC1.DecongestionLemma.decongestion_coarse_bound}
\leanok
\uses{def:CycleSparsification.LayerCycleDegreeBound, def:CycleSparsification.minLayers, def:DesiderataForGraphG.ConstantDegree, thm:DesiderataForGraphG.constant_degree_bounds_edges}
For a $\Delta$-bounded graph $G$ with cycle rank property $|C| + |V| = |E| + 1$ and $|V| > 0$,
\[
  R_G^b \le \frac{\Delta \cdot |V|}{2}.
\]
\end{theorem}

\begin{proof}
\leanok
\uses{def:CycleSparsification.minLayers, thm:DesiderataForGraphG.constant_degree_bounds_edges}
We have $R_G^b \le |C|$ from the decongestion layers bound, and $|C| \le \Delta \cdot |V| / 2$ from the cycle count bound. The result follows by integer arithmetic.
\end{proof}

% COMMENTED OUT: Isolated declarations (not connected to dependency graph)
% \section{Diameter Bounds}
%
% \begin{theorem}[Connected Graph Has Finite Extended Diameter]
% \label{thm:DecongestionLemma.connected_ediam_ne_top}
% \lean{QEC1.DecongestionLemma.connected_ediam_ne_top}
% \leanok
%
% For any connected finite graph $G$ on at least $1$ vertex, the extended diameter $\operatorname{ediam}(G) \ne \top$.
% \end{theorem}
%
% \begin{proof}
% \leanok
%
% Since $|V| > 0$, the type $V$ is nonempty. The result follows from the Mathlib equivalence $G.\text{Connected} \iff G.\text{ediam} \ne \top$.
% \end{proof}
%
% \begin{theorem}[Connected Graph Distance Bound]
% \label{thm:DecongestionLemma.connected_edist_le}
% \lean{QEC1.DecongestionLemma.connected_edist_le}
% \leanok
%
% For any connected finite graph $G$ and any vertices $u, v$,
% \[
%   \operatorname{edist}_G(u, v) \le |V| - 1.
% \]
% \end{theorem}
%
% \begin{proof}
% \leanok
%
% By connectedness, there exists a walk from $u$ to $v$. Converting to a path $p$, we have $|p| < |V|$ (since a path visits distinct vertices). Then $\operatorname{edist}_G(u,v) \le |p| \le |V| - 1$.
% \end{proof}
%
% \begin{theorem}[Connected Graph Diameter Bound]
% \label{thm:DecongestionLemma.connected_diam_le_card_sub_one}
% \lean{QEC1.DecongestionLemma.connected_diam_le_card_sub_one}
% \leanok
%
% For any connected finite graph $G$ with $|V| > 0$,
% \[
%   \operatorname{diam}(G) \le |V| - 1.
% \]
% \end{theorem}
%
% \begin{proof}
% \leanok
%
% We apply $\operatorname{ENat.toNat\_le\_of\_le\_coe}$ and $\operatorname{ediam\_le\_iff}$. For all vertices $u, v$, the connected distance bound gives $\operatorname{edist}_G(u,v) \le |V| - 1$.
% \end{proof}
%
% \begin{theorem}[Connected Graph $\operatorname{dist}$ Bound]
% \label{thm:DecongestionLemma.connected_dist_le}
% \lean{QEC1.DecongestionLemma.connected_dist_le}
% \leanok
%
% For any connected finite graph $G$ and vertices $u, v$,
% \[
%   \operatorname{dist}_G(u, v) \le |V| - 1.
% \]
% \end{theorem}
%
% \begin{proof}
% \leanok
%
% Since $V$ is nonempty (witnessed by $u$) and $G$ is connected, $\operatorname{ediam}(G) \ne \top$. Then $\operatorname{dist}_G(u,v) \le \operatorname{diam}(G) \le |V| - 1$.
% \end{proof}

\section{BFS Ball Growth and Expander Diameter}

\begin{definition}[BFS Ball]
\label{def:DecongestionLemma.bfsBall}
\lean{QEC1.DecongestionLemma.bfsBall}
\leanok

The \emph{BFS ball} of radius $r$ around vertex $v$ in a graph $G$ is
\[
  B(v, r) = \{u \in V : \operatorname{dist}_G(v, u) \le r\}.
\]
\end{definition}

\begin{theorem}[BFS Ball Growth from Expansion]
\label{thm:DecongestionLemma.bfsBall_growth_from_expansion}
\lean{QEC1.DecongestionLemma.bfsBall_growth_from_expansion}
\leanok
\uses{def:DecongestionLemma.bfsBall, def:cheegerConstant, def:SimpleGraph.edgeBoundary}
For a graph $G$ with Cheeger constant $h(G) > 0$, if $|B(v,r)| \le |V|/2$ and $B(v,r) \ne \emptyset$, then
\[
  h(G) \cdot |B(v,r)| \le |\partial B(v,r)|,
\]
where $\partial B(v,r)$ denotes the edge boundary of $B(v,r)$.
\end{theorem}

\begin{proof}
\leanok
\uses{def:DecongestionLemma.bfsBall, def:cheegerConstant, def:SimpleGraph.edgeBoundary}
This follows directly from the Cheeger inequality for edge boundaries (Theorem~\ref{thm:SimpleGraph.edgeBoundary_card_ge_of_cheeger}), applied to the set $B(v,r)$ with $h = h(G)$.
\end{proof}

\begin{theorem}[BFS Ball Covers Graph]
\label{thm:DecongestionLemma.bfsBall_full}
\lean{QEC1.DecongestionLemma.bfsBall_full}
\leanok
\uses{def:DecongestionLemma.bfsBall}
For a connected graph $G$ with $|V| > 0$ and any vertex $v$,
\[
  B(v, |V| - 1) = V.
\]
\end{theorem}

\begin{proof}
\leanok
\uses{def:DecongestionLemma.bfsBall}
By extensionality, for any $u$ we have $u \in B(v, |V|-1)$ if and only if $\operatorname{dist}_G(v, u) \le |V| - 1$, which holds by the connected distance bound.
\end{proof}

\begin{theorem}[Expander Diameter Bound from BFS]
\label{thm:DecongestionLemma.expander_diameter_bound_from_bfs}
\lean{QEC1.DecongestionLemma.expander_diameter_bound_from_bfs}
\leanok
\uses{def:DecongestionLemma.bfsBall, def:cheegerConstant, def:SimpleGraph.edgeBoundary}
For a connected expander graph $G$ with $|V| > 0$ and Cheeger constant $h(G) > 0$:
\begin{enumerate}
  \item $\operatorname{diam}(G) \le |V| - 1$, and
  \item for all $v \in V$ and $r \in \mathbb{N}$, if $2|B(v,r)| \le |V|$, then $h(G) \cdot |B(v,r)| \le |\partial B(v,r)|$.
\end{enumerate}
\end{theorem}

\begin{proof}
\leanok
\uses{def:DecongestionLemma.bfsBall, def:cheegerConstant, def:SimpleGraph.edgeBoundary}
Part (1) follows from the connected diameter bound (Theorem~\ref{thm:DecongestionLemma.connected_diam_le_card_sub_one}). Part (2) follows from the BFS ball growth from expansion (Theorem~\ref{thm:DecongestionLemma.bfsBall_growth_from_expansion}), noting that $B(v,r)$ is nonempty since it always contains $v$.
\end{proof}

\section{Constructive Decongestion Lemma Bound}

\begin{theorem}[Decongestion Lemma Bound (Constructive Coarse Version)]
\label{thm:DecongestionLemma.decongestion_lemma_bound}
\lean{QEC1.DecongestionLemma.decongestion_lemma_bound}
\leanok
\uses{def:CycleSparsification.LayerCycleDegreeBound, def:DesiderataForGraphG.ConstantDegree, thm:DesiderataForGraphG.constant_degree_bounds_edges}
For a $\Delta$-bounded graph $G$ on $|V| \ge 2$ vertices, with a cycle collection satisfying the cycle rank property $|C| + |V| = |E| + 1$ and any positive per-layer bound $b > 0$, there exist $R$ and a layer assignment $f : C \to \operatorname{Fin}(R+1)$ such that
\[
  \operatorname{LayerCycleDegreeBound}(f, b) \quad\text{and}\quad R \le \frac{\Delta \cdot |V|}{2}.
\]
\end{theorem}

\begin{proof}
\leanok
\uses{def:CycleSparsification.LayerCycleDegreeBound, thm:DesiderataForGraphG.constant_degree_bounds_edges}
We obtain a layer assignment $f$ from the one-per-layer construction with $R = |C|$ layers. The bound $R \le \Delta \cdot |V| / 2$ follows from the cycle count bound (since $|V| \ge 2 > 0$).
\end{proof}

\section{Universal Linear Bound}

\begin{theorem}[Decongestion Lemma: Universal Linear Bound]
\label{thm:DecongestionLemma.decongestion_lemma_linear_exists}
\lean{QEC1.DecongestionLemma.decongestion_lemma_linear_exists}
\leanok
\uses{def:CycleSparsification.LayerCycleDegreeBound, def:DesiderataForGraphG.ConstantDegree, thm:DesiderataForGraphG.constant_degree_bounds_edges}
For any $\Delta > 0$ and $c > 0$, there exists $K > 0$ (depending only on $\Delta$) such that for any $\Delta$-bounded connected graph $G$ on $W \ge 2$ vertices with cycle rank property, there exist $R$ and $f$ with
\[
  \operatorname{LayerCycleDegreeBound}(f, c) \quad\text{and}\quad R \le K \cdot W.
\]
In fact, $K = \Delta$ suffices.
\end{theorem}

\begin{proof}
\leanok
\uses{def:CycleSparsification.LayerCycleDegreeBound, thm:DesiderataForGraphG.constant_degree_bounds_edges}
We take $K = \Delta$, which is positive by hypothesis. For any graph $G'$ satisfying the conditions, we apply the constructive coarse bound to obtain $R$ and $f$ with $R \le \Delta \cdot W / 2$. Since $\Delta \cdot W / 2 \le \Delta \cdot W$ (by $\operatorname{Nat.div\_le\_self}$), the result follows.
\end{proof}

\section{Main Theorem: Decongestion Lemma (Freedman--Hastings)}

\begin{theorem}[Decongestion Lemma: $O(\log^2 W)$ Bound]
\label{thm:DecongestionLemma.decongestion_log_squared}
\lean{QEC1.DecongestionLemma.decongestion_log_squared}
\leanok
\uses{def:CycleSparsification.LayerCycleDegreeBound, def:DesiderataForGraphG.ConstantDegree, def:CycleSparsification.edgeCycleDegreeInLayer, def:CycleSparsification.edgeCycleDegree, def:CycleSparsification.minLayers, thm:CycleSparsification.edgeCycleDegree_le_card, thm:DesiderataForGraphG.constant_degree_bounds_edges}
For a $\Delta$-bounded connected expander graph $G$ on $W \ge 2$ vertices with Cheeger constant $h(G) > 0$, a cycle collection with cycle rank property, and a positive integer $c > 0$: given an $M$-coloring $f_0 : C \to \operatorname{Fin}(M+1)$ with per-layer bound $1$ (from K\"onig's theorem) where $M \le K \cdot \log_2^2(W)$ (from the Freedman--Hastings analysis), there exist $R$ and $f$ with
\[
  \operatorname{LayerCycleDegreeBound}(f, c) \quad\text{and}\quad R \le \frac{K \cdot \log_2^2(W)}{c}.
\]
\end{theorem}

\begin{proof}
\leanok
\uses{def:CycleSparsification.LayerCycleDegreeBound}
We apply the greedy packing bound (Theorem~\ref{thm:DecongestionLemma.greedy_packing_bound}) to $f_0$, $M$, and $c$, obtaining $R$ and $f$ with $R \le M/c$. Since $M \le K \cdot \log_2^2(W)$, we have $R \le M/c \le K \cdot \log_2^2(W) / c$ by monotonicity of integer division.
\end{proof}

\begin{theorem}[Decongestion Lemma: $O(\log^2 W)$ Layer Bound (Weaker Form)]
\label{thm:DecongestionLemma.decongestion_log_squared_bound}
\lean{QEC1.DecongestionLemma.decongestion_log_squared_bound}
\leanok
\uses{def:CycleSparsification.LayerCycleDegreeBound, def:DesiderataForGraphG.ConstantDegree, def:CycleSparsification.edgeCycleDegreeInLayer, def:CycleSparsification.edgeCycleDegree, def:CycleSparsification.minLayers, thm:CycleSparsification.edgeCycleDegree_le_card, thm:DesiderataForGraphG.constant_degree_bounds_edges}
Under the same hypotheses as the main decongestion lemma, we have the weaker bound
\[
  R \le K \cdot \log_2^2(W).
\]
\end{theorem}

\begin{proof}
\leanok
\uses{def:CycleSparsification.LayerCycleDegreeBound}
We apply the main decongestion lemma (Theorem~\ref{thm:DecongestionLemma.decongestion_log_squared}) to obtain $R \le K \cdot \log_2^2(W) / c$. Since $K \cdot \log_2^2(W) / c \le K \cdot \log_2^2(W)$ by $\operatorname{Nat.div\_le\_self}$, the result follows.
\end{proof}

\section{Consequences for the Sparsified Graph}

\begin{theorem}[Sparsified Vertex Count Bound]
\label{thm:DecongestionLemma.sparsified_vertex_count_bound}
\lean{QEC1.DecongestionLemma.sparsified_vertex_count_bound}
\leanok
\uses{def:CycleSparsification.minLayers}
With $R \le K \cdot \log_2^2(W)$ layers, the sparsified graph has at most $(K \cdot \log_2^2(W) + 1) \cdot W$ vertices:
\[
  (R + 1) \cdot W \le (K \cdot \log_2^2(W) + 1) \cdot W.
\]
\end{theorem}

\begin{proof}
\leanok

This follows from $R \le K \cdot \log_2^2(W)$ by monotonicity of multiplication on the right: $(R+1) \cdot W \le (K \cdot \log_2^2(W) + 1) \cdot W$.
\end{proof}

\begin{theorem}[Sparsified Edge Overhead Bound]
\label{thm:DecongestionLemma.sparsified_edge_overhead_bound}
\lean{QEC1.DecongestionLemma.sparsified_edge_overhead_bound}
\leanok
\uses{def:CycleSparsification.minLayers}
With $R \le K \cdot \log_2^2(W)$ layers and degree bound $\Delta > 0$, the edge overhead satisfies
\[
  (R+1) \cdot \frac{\Delta \cdot W}{2} \le (K \cdot \log_2^2(W) + 1) \cdot \frac{\Delta \cdot W}{2}.
\]
\end{theorem}

\begin{proof}
\leanok

This follows from $R \le K \cdot \log_2^2(W)$ by monotonicity of multiplication on the right.
\end{proof}

\section{Summary}

\begin{theorem}[Decongestion Lemma Summary]
\label{thm:DecongestionLemma.decongestion_summary}
\lean{QEC1.DecongestionLemma.decongestion_summary}
\leanok
\uses{def:CycleSparsification.LayerCycleDegreeBound, def:CycleSparsification.edgeCycleDegreeInLayer, def:CycleSparsification.edgeCycleDegree, def:CycleSparsification.minLayers, def:DesiderataForGraphG.ConstantDegree, def:DecongestionLemma.bfsBall, def:cheegerConstant, def:SimpleGraph.edgeBoundary, thm:CycleSparsification.edgeCycleDegree_le_card, thm:DesiderataForGraphG.constant_degree_bounds_edges}
For a $\Delta$-bounded connected expander graph $G$ on $W \ge 2$ vertices with Cheeger constant $h(G) > 0$, a cycle collection satisfying the cycle rank property, and a positive per-layer bound $b > 0$:
\begin{enumerate}
  \item \textbf{Coarse bound:} There exist $R$ and $f$ with $\operatorname{LayerCycleDegreeBound}(f, b)$ and $R \le \Delta \cdot W / 2$.
  \item \textbf{Greedy packing:} Given any $M$-coloring $f_0$ with per-layer bound $1$, there exist $R$ and $f$ with $\operatorname{LayerCycleDegreeBound}(f, b)$ and $R \le M / b$.
  \item \textbf{Cycle count bound:} $|C| \le \Delta \cdot W / 2$.
  \item \textbf{Diameter bound and Cheeger inequality:} $\operatorname{diam}(G) \le W - 1$, and for all $v \in V$, $r \in \mathbb{N}$, if $2|B(v,r)| \le W$ then $h(G) \cdot |B(v,r)| \le |\partial B(v,r)|$.
\end{enumerate}
\end{theorem}

\begin{proof}
\leanok
\uses{def:CycleSparsification.LayerCycleDegreeBound, def:DecongestionLemma.bfsBall, def:cheegerConstant, def:SimpleGraph.edgeBoundary, thm:DesiderataForGraphG.constant_degree_bounds_edges}
Part (1) follows from the constructive coarse bound (Theorem~\ref{thm:DecongestionLemma.decongestion_lemma_bound}). Part (2) follows from the greedy packing bound (Theorem~\ref{thm:DecongestionLemma.greedy_packing_bound}). Part (3) follows from the cycle count bound (Theorem~\ref{thm:DecongestionLemma.cycle_count_bound}), noting $W \ge 2 > 0$. Part (4) follows from the expander diameter bound from BFS (Theorem~\ref{thm:DecongestionLemma.expander_diameter_bound_from_bfs}), noting $W \ge 2 > 0$.
\end{proof}

%--- Rem_12: WorstCaseGraphConstruction ---

\chapter{Rem 12: Worst-Case Graph Construction}

We outline a construction that produces a graph $G$ with qubit overhead $O(W \log^2 W)$ satisfying all desiderata from Remark~11, for a logical operator $L$ of weight $W$.

Let $L$ denote the support of $L$ (so $|L| = W$), and let $V = L$ be the vertex set.

\textbf{Step 1: $Z$-type support matching.}
For each check $s_j$ with $S_Z(s_j) \cap L \neq \emptyset$, the intersection has even cardinality (since $s_j$ commutes with $L = \prod_v X_v$). Pick a perfect matching and add edges, achieving Desideratum~1 with $D = 1$.

\textbf{Step 2: Achieve expansion.}
Add edges until $h(G) \geq 1$ (Cheeger constant), satisfying Desideratum~2. The resulting graph $G_0$ is constant-degree.

\textbf{Step 3: Cycle sparsification.}
Apply Definition~6 to $G_0$ with $R = O(\log^2 W)$ layers (Lemma~2). All generating cycles have weight $\leq 4$, satisfying Desideratum~3.

\textbf{Result:}
\begin{itemize}
  \item Total qubits: $O(W) \cdot O(\log^2 W) = O(W \log^2 W)$.
  \item All checks have bounded weight.
  \item Distance is preserved: $d^* \geq d$.
\end{itemize}

%——— Step 1: Even Z-support from commutation with L ———

\begin{theorem}[Symplectic Inner Product with $\prod X$]
\label{thm:WorstCaseGraphConstruction.symplecticInner_prodX_univ}
\lean{QEC1.WorstCaseGraphConstruction.symplecticInner_prodX_univ}
\leanok
\uses{def:PauliOp.symplecticInner, def:PauliOp.prodX}
Let $V$ be a finite type and let $P$ be a Pauli operator on $V$. Then
\[
  \langle P, \operatorname{prodX}(V) \rangle_{\mathrm{symp}} \;=\; \sum_{v \in V} P.\operatorname{zVec}(v).
\]
\end{theorem}

\begin{proof}
\leanok
\uses{def:PauliOp.symplecticInner, def:PauliOp.prodX}
Expanding the definition of the symplectic inner product and $\operatorname{prodX}$, the $X$-vector of $\operatorname{prodX}(\operatorname{univ})$ is the all-ones function and the $Z$-vector is identically zero. Thus the sum reduces to $\sum_v P.\operatorname{zVec}(v) \cdot 1 + P.\operatorname{xVec}(v) \cdot 0 = \sum_v P.\operatorname{zVec}(v)$. By simplification and functional extensionality (using commutativity of the ring), the two sides are equal.
\end{proof}

\begin{theorem}[Commutation with $\prod X \Leftrightarrow$ CommutesWithLogical]
\label{thm:WorstCaseGraphConstruction.commute_prodX_iff_commutesWithLogical}
\lean{QEC1.WorstCaseGraphConstruction.commute_prodX_iff_commutesWithLogical}
\leanok
\uses{def:PauliOp.PauliCommute, def:PauliOp.prodX, def:DeformedOperator.CommutesWithLogical}
Let $P$ be a Pauli operator on a finite type $V$. Then
\[
  \operatorname{PauliCommute}(P, \operatorname{prodX}(V)) \;\Longleftrightarrow\; \operatorname{CommutesWithLogical}(P).
\]
\end{theorem}

\begin{proof}
\leanok
\uses{thm:WorstCaseGraphConstruction.symplecticInner_prodX_univ, def:PauliOp.PauliCommute, def:DeformedOperator.CommutesWithLogical}
Unfolding the definitions of $\operatorname{PauliCommute}$ and $\operatorname{CommutesWithLogical}$, we rewrite using the identity $\langle P, \operatorname{prodX}(V) \rangle_{\mathrm{symp}} = \sum_v P.\operatorname{zVec}(v)$ (from the previous theorem), and then use the equality $\sum_v P.\operatorname{zVec}(v) = \sum_v \operatorname{zSupportOnVertices}(P)(v)$ to conclude.
\end{proof}

\begin{theorem}[Commuting Check Has Even $Z$-Support]
\label{thm:WorstCaseGraphConstruction.check_commutes_implies_even_zSupport}
\lean{QEC1.WorstCaseGraphConstruction.check_commutes_implies_even_zSupport}
\leanok
\uses{def:PauliOp.PauliCommute, def:PauliOp.prodX, def:DeformedOperator.CommutesWithLogical}
Let $P$ be a Pauli operator on a finite type $V$. If $\operatorname{PauliCommute}(P, \operatorname{prodX}(V))$, then
\[
  \bigl|\{v \in V \mid P.\operatorname{zVec}(v) \neq 0\}\bigr| \text{ is even.}
\]
\end{theorem}

\begin{proof}
\leanok
\uses{thm:WorstCaseGraphConstruction.commute_prodX_iff_commutesWithLogical, thm:DeformedOperator.commutesWithLogical_iff_even_zSupport}
Rewriting the hypothesis using the equivalence $\operatorname{PauliCommute}(P, \operatorname{prodX}(V)) \Leftrightarrow \operatorname{CommutesWithLogical}(P)$, the result follows from the implication $\operatorname{CommutesWithLogical}(P) \Rightarrow \text{Even}(|\{v : P.\operatorname{zVec}(v) \neq 0\}|)$.
\end{proof}

\begin{theorem}[Even $Z$-Support Cardinality Modulo 2]
\label{thm:WorstCaseGraphConstruction.even_zSupport_card}
\lean{QEC1.WorstCaseGraphConstruction.even_zSupport_card}
\leanok
\uses{def:PauliOp.PauliCommute, def:PauliOp.prodX}
Let $P$ be a Pauli operator on a finite type $V$. If $\operatorname{PauliCommute}(P, \operatorname{prodX}(V))$, then
\[
  \bigl|\{v \in V \mid P.\operatorname{zVec}(v) \neq 0\}\bigr| \bmod 2 = 0.
\]
\end{theorem}

\begin{proof}
\leanok
\uses{thm:WorstCaseGraphConstruction.check_commutes_implies_even_zSupport}
This follows directly from the previous theorem: since the cardinality is even, it is congruent to $0$ modulo $2$.
\end{proof}

%——— Step 1: Z-support matching gives Desideratum 1 ———

\begin{theorem}[Step 1 Achieves Desideratum 1]
\label{thm:WorstCaseGraphConstruction.step1_achieves_desideratum1}
\lean{QEC1.WorstCaseGraphConstruction.step1_achieves_desideratum1}
\leanok
\uses{def:DesiderataForGraphG.ShortPathsForDeformation, def:PauliOp.supportZ}
Let $G$ be a simple graph on a finite type $V$, and let $\{\operatorname{checks}(j)\}_{j \in J}$ be a family of Pauli operators. Suppose that for every check $j$ and every pair of distinct vertices $u, v$ in $\operatorname{supportZ}(\operatorname{checks}(j))$, $u$ and $v$ are adjacent in $G$. Then $\operatorname{ShortPathsForDeformation}(G, \operatorname{checks}, 1)$ holds.
\end{theorem}

\begin{proof}
\leanok
\uses{def:DesiderataForGraphG.ShortPathsForDeformation}
Let $j$, $u$, $v$ be given with $u \in \operatorname{supportZ}(\operatorname{checks}(j))$ and $v \in \operatorname{supportZ}(\operatorname{checks}(j))$. We consider two cases. If $u = v$, then $\operatorname{dist}(u, v) = 0 \leq 1$ by reflexivity of the distance. If $u \neq v$, then by hypothesis $G.\operatorname{Adj}(u, v)$, so $\operatorname{dist}(u, v) = 1 \leq 1$ by the characterization of graph distance equal to one.
\end{proof}

\begin{theorem}[Step 1 Edge-Path Bound]
\label{thm:WorstCaseGraphConstruction.step1_edge_path_bound}
\lean{QEC1.WorstCaseGraphConstruction.step1_edge_path_bound}
\leanok
\uses{def:PauliOp.supportZ, def:DesiderataForGraphG.ShortPathsForDeformation}
Under the same hypotheses as the previous theorem (every check's $Z$-support forms a clique in $G$), for any check $j$ and any $u, v \in \operatorname{supportZ}(\operatorname{checks}(j))$:
\[
  \operatorname{dist}_G(u, v) \leq 1.
\]
\end{theorem}

\begin{proof}
\leanok
\uses{thm:WorstCaseGraphConstruction.step1_achieves_desideratum1}
This follows directly from applying the theorem $\operatorname{step1\_achieves\_desideratum1}$.
\end{proof}

%——— Step 2: Expansion ———

\begin{theorem}[Steps 1 and 2 Satisfy Desiderata 1 and 2]
\label{thm:WorstCaseGraphConstruction.steps12_satisfy_desiderata12}
\lean{QEC1.WorstCaseGraphConstruction.steps12_satisfy_desiderata12}
\leanok
\uses{def:DesiderataForGraphG.ShortPathsForDeformation, def:DesiderataForGraphG.SufficientExpansion, def:PauliOp.supportZ}
Let $G$ be a simple graph on $V$ with checks $\{\operatorname{checks}(j)\}_{j \in J}$. If every pair of distinct vertices in the $Z$-support of each check are adjacent in $G$, and $G$ has sufficient expansion, then both $\operatorname{ShortPathsForDeformation}(G, \operatorname{checks}, 1)$ and $\operatorname{SufficientExpansion}(G)$ hold.
\end{theorem}

\begin{proof}
\leanok
\uses{thm:WorstCaseGraphConstruction.step1_achieves_desideratum1, def:DesiderataForGraphG.SufficientExpansion}
The pair is constructed from $\operatorname{step1\_achieves\_desideratum1}$ (giving Desideratum~1) and the hypothesis $\operatorname{hexp}$ (giving Desideratum~2).
\end{proof}

%——— Step 3: Cycle Sparsification ———

\begin{theorem}[Sparsified Vertex Count]
\label{thm:WorstCaseGraphConstruction.step3_vertex_count}
\lean{QEC1.WorstCaseGraphConstruction.step3_vertex_count}
\leanok
\uses{def:CycleSparsification.SparsifiedVertex, thm:CycleSparsification.card_sparsifiedVertex}
For natural numbers $R$ and $W$, the sparsified graph's vertex count satisfies
\[
  (R+1) \cdot W = |\operatorname{SparsifiedVertex}(\operatorname{Fin}(W), R)|.
\]
\end{theorem}

\begin{proof}
\leanok
\uses{thm:CycleSparsification.card_sparsifiedVertex}
Rewriting using the identity $|\operatorname{SparsifiedVertex}(\operatorname{Fin}(W), R)| = (R+1) \cdot |\operatorname{Fin}(W)|$ and the fact that $|\operatorname{Fin}(W)| = W$, the equality holds.
\end{proof}

\begin{theorem}[Sparsified Vertex Bound]
\label{thm:WorstCaseGraphConstruction.step3_vertex_bound}
\lean{QEC1.WorstCaseGraphConstruction.step3_vertex_bound}
\leanok
\uses{def:CycleSparsification.SparsifiedVertex, thm:DecongestionLemma.sparsified_vertex_count_bound}
For natural numbers $R$, $K$, $W$ with $R \leq K \cdot (\log_2 W)^2$:
\[
  (R+1) \cdot W \leq (K \cdot (\log_2 W)^2 + 1) \cdot W.
\]
\end{theorem}

\begin{proof}
\leanok
\uses{thm:DecongestionLemma.sparsified_vertex_count_bound}
This follows directly from the sparsified vertex count bound established in Lemma~2.
\end{proof}

\begin{theorem}[Step 3 Satisfies Desideratum 3]
\label{thm:WorstCaseGraphConstruction.step3_satisfies_desideratum3}
\lean{QEC1.WorstCaseGraphConstruction.step3_satisfies_desideratum3}
\leanok
\uses{def:DesiderataForGraphG.LowWeightCycleBasis}
Let $G$ be a simple graph with a finite edge set, and let $\{\operatorname{cycles}(c)\}_{c \in C}$ be a collection of subsets of $G.\operatorname{edgeSet}$. If every cycle $c$ satisfies $|\{e \in G.\operatorname{edgeSet} \mid e \in \operatorname{cycles}(c)\}| \leq 4$, then $\operatorname{LowWeightCycleBasis}(G, \operatorname{cycles}, 4)$ holds.
\end{theorem}

\begin{proof}
\leanok
\uses{def:DesiderataForGraphG.LowWeightCycleBasis}
By definition, $\operatorname{LowWeightCycleBasis}(G, \operatorname{cycles}, 4)$ requires exactly that every cycle has at most $4$ edges, which is the hypothesis.
\end{proof}

%——— Total Overhead ———

\begin{theorem}[Total Qubit Bound Arithmetic]
\label{thm:WorstCaseGraphConstruction.total_qubit_bound_arithmetic}
\lean{QEC1.WorstCaseGraphConstruction.total_qubit_bound_arithmetic}
\leanok
\uses{def:CycleSparsification.SparsifiedVertex}
For natural numbers $W$, $R$, $\Delta$, $K$ with $R \leq K \cdot (\log_2 W)^2$:
\[
  (R+1) \cdot W \leq (K \cdot (\log_2 W)^2 + 1) \cdot W.
\]
\end{theorem}

\begin{proof}
\leanok

By monotonicity of multiplication on the right, it suffices to show $(R+1) \leq K \cdot (\log_2 W)^2 + 1$, which follows from $R \leq K \cdot (\log_2 W)^2$ by integer arithmetic.
\end{proof}

\begin{theorem}[Total Qubit Overhead from Degree]
\label{thm:WorstCaseGraphConstruction.total_qubit_overhead_from_degree}
\lean{QEC1.WorstCaseGraphConstruction.total_qubit_overhead_from_degree}
\leanok
\uses{def:DesiderataForGraphG.ConstantDegree, thm:DesiderataForGraphG.total_qubits_bounded}
Let $G$ be a simple graph on $V$ with $\operatorname{ConstantDegree}(G, \Delta)$. Then
\[
  2 \cdot (|V| + |E(G)|) \leq (2 + \Delta) \cdot |V|.
\]
\end{theorem}

\begin{proof}
\leanok
\uses{thm:DesiderataForGraphG.total_qubits_bounded, def:DesiderataForGraphG.ConstantDegree}
This follows directly from the total qubits bounded theorem.
\end{proof}

\begin{theorem}[Total Overhead is $O(W \log^2 W)$]
\label{thm:WorstCaseGraphConstruction.total_overhead_is_wlog2w}
\lean{QEC1.WorstCaseGraphConstruction.total_overhead_is_wlog2w}
\leanok
\uses{def:CycleSparsification.SparsifiedVertex}
For natural numbers $W$, $R$, $K$ with $R \leq K \cdot (\log_2 W)^2$, $K > 0$, and $W \geq 2$:
\[
  (R+1) \cdot W \leq (K \cdot (\log_2 W)^2 + 1) \cdot W.
\]
\end{theorem}

\begin{proof}
\leanok

By monotonicity of multiplication on the right, it suffices to show $(R+1) \leq K \cdot (\log_2 W)^2 + 1$, which follows from the hypothesis $R \leq K \cdot (\log_2 W)^2$ by integer arithmetic.
\end{proof}

%——— Bounded Weight ———

\begin{theorem}[Gauss Check Weight from Degree]
\label{thm:WorstCaseGraphConstruction.gauss_weight_from_degree}
\lean{QEC1.WorstCaseGraphConstruction.gauss_weight_from_degree}
\leanok
\uses{def:DesiderataForGraphG.ConstantDegree, def:DeformedCode.gaussLawChecks, thm:DesiderataForGraphG.gaussLaw_checks_weight_bounded}
Let $G$ be a simple graph on $V$ with $\operatorname{ConstantDegree}(G, \Delta)$. For every vertex $v \in V$:
\[
  \operatorname{weight}(\operatorname{gaussLawChecks}(G, v)) \leq 1 + \Delta.
\]
\end{theorem}

\begin{proof}
\leanok
\uses{thm:DesiderataForGraphG.gaussLaw_checks_weight_bounded}
This follows directly from the Gauss law checks weight bound.
\end{proof}

\begin{theorem}[Flux Check Weight from Cycles]
\label{thm:WorstCaseGraphConstruction.flux_weight_from_cycles}
\lean{QEC1.WorstCaseGraphConstruction.flux_weight_from_cycles}
\leanok
\uses{def:DesiderataForGraphG.LowWeightCycleBasis, def:DeformedCode.fluxChecks, thm:DesiderataForGraphG.flux_checks_weight_bounded}
Let $G$ be a simple graph with cycles $\{\operatorname{cycles}(c)\}_{c \in C}$ satisfying $\operatorname{LowWeightCycleBasis}(G, \operatorname{cycles}, W)$. For every plaquette $p \in C$:
\[
  \operatorname{weight}(\operatorname{fluxChecks}(G, \operatorname{cycles}, p)) \leq W.
\]
\end{theorem}

\begin{proof}
\leanok
\uses{thm:DesiderataForGraphG.flux_checks_weight_bounded}
This follows directly from the flux checks weight bound.
\end{proof}

\begin{theorem}[Deformed Check Weight Bounded by Path]
\label{thm:WorstCaseGraphConstruction.deformed_check_weight_bounded_by_path}
\lean{QEC1.WorstCaseGraphConstruction.deformed_check_weight_bounded_by_path}
\leanok
\uses{def:DeformedCode.deformedCheck, thm:DesiderataForGraphG.deformed_check_edge_bound}
Let $G$ be a simple graph on $V$ with checks $\{\operatorname{checks}(j)\}_{j \in J}$. Given a check index $j$, an edge path $\gamma : G.\operatorname{edgeSet} \to \mathbb{Z}/2\mathbb{Z}$, and a bound $B$ such that $|\{e : \gamma(e) \neq 0\}| \leq B$, then
\[
  \bigl|\{e \in G.\operatorname{edgeSet} \mid \operatorname{deformedCheck}(G, \operatorname{checks}(j), \gamma).\operatorname{zVec}(\operatorname{inr}(e)) \neq 0\}\bigr| \leq B.
\]
\end{theorem}

\begin{proof}
\leanok
\uses{thm:DesiderataForGraphG.deformed_check_edge_bound}
This follows directly from the deformed check edge bound.
\end{proof}

%——— Summary ———

\begin{theorem}[Distance Preserved by Expansion]
\label{thm:WorstCaseGraphConstruction.distance_preserved_by_expansion}
\lean{QEC1.WorstCaseGraphConstruction.distance_preserved_by_expansion}
\leanok
\uses{def:DesiderataForGraphG.SufficientExpansion, def:SimpleGraph.edgeBoundary, thm:DesiderataForGraphG.expansion_gives_boundary_bound}
Let $G$ be a simple graph on $V$ with $\operatorname{SufficientExpansion}(G)$. For every nonempty subset $S \subseteq V$ with $2|S| \leq |V|$:
\[
  |S| \leq |\partial_G S|,
\]
where $\partial_G S$ is the edge boundary of $S$ in $G$.
\end{theorem}

\begin{proof}
\leanok
\uses{thm:DesiderataForGraphG.expansion_gives_boundary_bound}
This follows directly from the expansion gives boundary bound theorem.
\end{proof}

\begin{theorem}[Construction Satisfies All Desiderata]
\label{thm:WorstCaseGraphConstruction.construction_satisfies_all_desiderata}
\lean{QEC1.WorstCaseGraphConstruction.construction_satisfies_all_desiderata}
\leanok
\uses{def:DesiderataForGraphG.AllDesiderataSatisfied, def:DesiderataForGraphG.ShortPathsForDeformation, def:DesiderataForGraphG.SufficientExpansion, def:DesiderataForGraphG.LowWeightCycleBasis, def:DesiderataForGraphG.ConstantDegree}
Let $G$ be a simple graph on $V$ with checks $\{\operatorname{checks}(j)\}_{j \in J}$ and cycles $\{\operatorname{cycles}(c)\}_{c \in C}$. Suppose:
\begin{enumerate}
  \item Every pair of distinct vertices in the $Z$-support of each check are adjacent in $G$.
  \item $\operatorname{SufficientExpansion}(G)$ holds.
  \item $\operatorname{LowWeightCycleBasis}(G, \operatorname{cycles}, 4)$ holds.
  \item $\operatorname{ConstantDegree}(G, \Delta)$ holds.
\end{enumerate}
Then $\operatorname{AllDesiderataSatisfied}(G, \operatorname{cycles}, \operatorname{checks}, 1, 4)$.
\end{theorem}

\begin{proof}
\leanok
\uses{thm:WorstCaseGraphConstruction.step1_achieves_desideratum1, def:DesiderataForGraphG.AllDesiderataSatisfied}
The three components of $\operatorname{AllDesiderataSatisfied}$ are constructed as a triple: $\operatorname{step1\_achieves\_desideratum1}$ gives $\operatorname{ShortPathsForDeformation}(G, \operatorname{checks}, 1)$, the hypothesis gives $\operatorname{SufficientExpansion}(G)$, and the low-weight cycle basis hypothesis gives $\operatorname{LowWeightCycleBasis}(G, \operatorname{cycles}, 4)$.
\end{proof}

\begin{theorem}[Construction Gives Bounded Weights]
\label{thm:WorstCaseGraphConstruction.construction_gives_bounded_weights}
\lean{QEC1.WorstCaseGraphConstruction.construction_gives_bounded_weights}
\leanok
\uses{def:DesiderataForGraphG.ConstantDegree, def:DesiderataForGraphG.LowWeightCycleBasis, def:DesiderataForGraphG.SufficientExpansion, def:SimpleGraph.IsExpander, def:DeformedCode.gaussLawChecks, def:DeformedCode.fluxChecks}
Let $G$ be a simple graph on $V$ with $\operatorname{ConstantDegree}(G, \Delta)$, $\operatorname{LowWeightCycleBasis}(G, \operatorname{cycles}, 4)$, and $\operatorname{SufficientExpansion}(G)$. Then:
\begin{enumerate}
  \item For all $v \in V$: $\operatorname{weight}(\operatorname{gaussLawChecks}(G, v)) \leq 1 + \Delta$.
  \item For all $p \in C$: $\operatorname{weight}(\operatorname{fluxChecks}(G, \operatorname{cycles}, p)) \leq 4$.
  \item $G$ is a $1$-expander: $\operatorname{IsExpander}(G, 1)$.
\end{enumerate}
\end{theorem}

\begin{proof}
\leanok
\uses{thm:DesiderataForGraphG.gaussLaw_checks_weight_bounded, thm:DesiderataForGraphG.flux_checks_weight_bounded, thm:DesiderataForGraphG.sufficient_expansion_implies_expander}
The three components are constructed as a triple:
\begin{enumerate}
  \item For each vertex $v$, the Gauss law check weight bound gives $\operatorname{weight}(\operatorname{gaussLawChecks}(G, v)) \leq 1 + \Delta$.
  \item For each plaquette $p$, the flux check weight bound gives $\operatorname{weight}(\operatorname{fluxChecks}(G, \operatorname{cycles}, p)) \leq 4$.
  \item The sufficient expansion implies expander theorem gives $\operatorname{IsExpander}(G, 1)$.
\end{enumerate}
\end{proof}

\begin{theorem}[Worst-Case Construction Summary]
\label{thm:WorstCaseGraphConstruction.worst_case_construction_summary}
\lean{QEC1.WorstCaseGraphConstruction.worst_case_construction_summary}
\leanok
\uses{def:DesiderataForGraphG.AllDesiderataSatisfied, def:DesiderataForGraphG.ConstantDegree, def:DesiderataForGraphG.SufficientExpansion, def:DesiderataForGraphG.LowWeightCycleBasis, def:CycleSparsification.LayerCycleDegreeBound, def:CodespaceDimension.CycleRankProperty, def:SimpleGraph.edgeBoundary}
Let $G$ be a connected simple graph on $V$ with $\operatorname{ConstantDegree}(G, \Delta)$, $|V| \geq 2$, $\operatorname{SufficientExpansion}(G)$, clique adjacency on each check's $Z$-support, $\operatorname{LowWeightCycleBasis}(G, \operatorname{cycles}, 4)$, and the cycle rank property $|C| + |V| = |E(G)| + 1$. Given a positive bound parameter, the following all hold simultaneously:
\begin{enumerate}
  \item $\operatorname{AllDesiderataSatisfied}(G, \operatorname{cycles}, \operatorname{checks}, 1, 4)$.
  \item Edge overhead is linear: $2|E(G)| \leq \Delta \cdot |V|$.
  \item Gauss checks bounded: $\forall v,\; \operatorname{weight}(\operatorname{gaussLawChecks}(G, v)) \leq 1 + \Delta$.
  \item Flux checks bounded: $\forall p,\; \operatorname{weight}(\operatorname{fluxChecks}(G, \operatorname{cycles}, p)) \leq 4$.
  \item Distance preserved: for all nonempty $S \subseteq V$ with $2|S| \leq |V|$, $|S| \leq |\partial_G S|$.
  \item Decongestion: there exist $R$ and a layer assignment $f : C \to \operatorname{Fin}(R+1)$ with $\operatorname{LayerCycleDegreeBound}(\operatorname{cycles}, f, \operatorname{bound})$ and $R \leq \Delta \cdot |V| / 2$.
\end{enumerate}
\end{theorem}

\begin{proof}
\leanok
\uses{thm:WorstCaseGraphConstruction.construction_satisfies_all_desiderata, thm:DesiderataForGraphG.constant_degree_bounds_edges, thm:DesiderataForGraphG.gaussLaw_checks_weight_bounded, thm:DesiderataForGraphG.flux_checks_weight_bounded, thm:DesiderataForGraphG.expansion_gives_boundary_bound, thm:DecongestionLemma.decongestion_lemma_bound}
The six-tuple is constructed from:
\begin{enumerate}
  \item The construction satisfies all desiderata theorem gives the first component.
  \item The constant degree bounds edges theorem gives $2|E(G)| \leq \Delta \cdot |V|$.
  \item For each vertex $v$, the Gauss law checks weight bound applies.
  \item For each plaquette $p$, the flux checks weight bound applies.
  \item For each valid subset $S$, the expansion gives boundary bound applies.
  \item The decongestion lemma bound provides the layer assignment with the required properties.
\end{enumerate}
\end{proof}

\begin{theorem}[Worst-Case Construction with $O(\log^2 W)$ Layers]
\label{thm:WorstCaseGraphConstruction.worst_case_construction_log_squared}
\lean{QEC1.WorstCaseGraphConstruction.worst_case_construction_log_squared}
\leanok
\uses{def:DesiderataForGraphG.AllDesiderataSatisfied, def:DesiderataForGraphG.ConstantDegree, def:DesiderataForGraphG.SufficientExpansion, def:DesiderataForGraphG.LowWeightCycleBasis, def:CycleSparsification.LayerCycleDegreeBound, def:CodespaceDimension.CycleRankProperty, def:SimpleGraph.edgeBoundary}
Let $G$ be a connected simple graph on $V$ with $\operatorname{ConstantDegree}(G, \Delta)$, $|V| \geq 2$, $\operatorname{SufficientExpansion}(G)$, clique adjacency on each check's $Z$-support, $\operatorname{LowWeightCycleBasis}(G, \operatorname{cycles}, 4)$, the cycle rank property, and a K\"{o}nig layer assignment $f_0 : C \to \operatorname{Fin}(M+1)$ with $\operatorname{LayerCycleDegreeBound}(\operatorname{cycles}, f_0, 1)$ and $M \leq K \cdot (\log_2 |V|)^2$. Then for any positive $c$:
\begin{enumerate}
  \item $\operatorname{AllDesiderataSatisfied}(G, \operatorname{cycles}, \operatorname{checks}, 1, 4)$.
  \item There exist $R$ and $f : C \to \operatorname{Fin}(R+1)$ with $\operatorname{LayerCycleDegreeBound}(\operatorname{cycles}, f, c)$ and $R \leq K \cdot (\log_2 |V|)^2 / c$.
  \item Distance preserved: for all nonempty $S \subseteq V$ with $2|S| \leq |V|$, $|S| \leq |\partial_G S|$.
  \item Vertex count bound: for all $R \leq K \cdot (\log_2 |V|)^2 / c$, $(R+1) \cdot |V| \leq (K \cdot (\log_2 |V|)^2 / c + 1) \cdot |V|$.
\end{enumerate}
\end{theorem}

\begin{proof}
\leanok
\uses{thm:WorstCaseGraphConstruction.construction_satisfies_all_desiderata, thm:DecongestionLemma.decongestion_log_squared, thm:DesiderataForGraphG.expansion_gives_boundary_bound, thm:DesiderataForGraphG.sufficient_expansion_pos}
We construct the four-tuple using a refinement:
\begin{enumerate}
  \item The construction satisfies all desiderata theorem gives the first component.
  \item The decongestion log-squared theorem, applied with $\Delta$, the constant degree hypothesis, connectivity, $|V| \geq 2$, the positivity of expansion (from $\operatorname{sufficient\_expansion\_pos}$), the cycles, $c > 0$, the cycle rank property, and the K\"{o}nig layer bound, provides $R$ and $f$ with the required properties.
  \item For each valid subset $S$, the expansion gives boundary bound applies.
  \item For the vertex count bound, by monotonicity of multiplication on the right, it suffices to show $R + 1 \leq K \cdot (\log_2 |V|)^2 / c + 1$, which follows from $R \leq K \cdot (\log_2 |V|)^2 / c$ by integer arithmetic.
\end{enumerate}
\end{proof}

%——— Overhead ———

\begin{theorem}[Qubit Overhead Bound]
\label{thm:WorstCaseGraphConstruction.qubit_overhead_bound}
\lean{QEC1.WorstCaseGraphConstruction.qubit_overhead_bound}
\leanok
\uses{def:CycleSparsification.SparsifiedVertex}
For natural numbers $W$, $R$, $\Delta_{\mathrm{sp}}$ with $R \leq (\log_2 W)^2$ and $W > 0$:
\[
  2 \cdot (R+1) \cdot W \leq 2 \cdot ((\log_2 W)^2 + 1) \cdot W.
\]
\end{theorem}

\begin{proof}
\leanok

By monotonicity of multiplication on the left (factor $2$) and on the right (factor $W$), it suffices to show $R + 1 \leq (\log_2 W)^2 + 1$, which follows from $R \leq (\log_2 W)^2$ by integer arithmetic.
\end{proof}

\begin{theorem}[Overhead Ratio]
\label{thm:WorstCaseGraphConstruction.overhead_ratio}
\lean{QEC1.WorstCaseGraphConstruction.overhead_ratio}
\leanok
\uses{def:CycleSparsification.SparsifiedVertex}
For natural numbers $W$, $R$, $K$ with $R \leq K \cdot (\log_2 W)^2$ and $W > 0$:
\[
  R + 1 \leq K \cdot (\log_2 W)^2 + 1.
\]
\end{theorem}

\begin{proof}
\leanok

This follows immediately from the hypothesis $R \leq K \cdot (\log_2 W)^2$ by integer arithmetic.
\end{proof}

%——— Code Parameters ———

\begin{theorem}[Deformed Code Loses One Logical Qubit]
\label{thm:WorstCaseGraphConstruction.deformed_code_loses_one_logical}
\lean{QEC1.WorstCaseGraphConstruction.deformed_code_loses_one_logical}
\leanok
\uses{def:DeformedCodeChecks.deformedStabilizerCode, def:DeformedCode.DeformedCodeData, def:CodespaceDimension.CycleRankProperty, def:PauliOp.PauliCommute, thm:CodespaceDimension.codespace_dimension_change_after_gauging}
Let $G$ be a simple graph on $V$ with checks $\{\operatorname{checks}(j)\}_{j \in J}$, cycles $\{\operatorname{cycles}(c)\}_{c \in C}$, and deformed code data. Suppose: the cycles satisfy the even incidence property at every vertex, the checks pairwise commute, the original stabilizer code has parameters $[\![n, k, d]\!]$ with $n = |V|$, $n - |J| = k$, $k \geq 1$, and the cycle rank property holds. Then the deformed stabilizer code satisfies
\[
  n' - m' = k - 1,
\]
where $n'$ and $m'$ are the number of qubits and checks of the deformed code, respectively.
\end{theorem}

\begin{proof}
\leanok
\uses{thm:CodespaceDimension.codespace_dimension_change_after_gauging}
This follows directly from the codespace dimension change after gauging theorem.
\end{proof}

% COMMENTED OUT: Isolated declaration (not connected to dependency graph)
% \begin{theorem}[Additional Qubits Equal Edge Count]
% \label{thm:WorstCaseGraphConstruction.additional_qubits_eq_edges}
% \lean{QEC1.WorstCaseGraphConstruction.additional_qubits_eq_edges}
% \leanok
%
% Let $G$ be a simple graph on $V$ with $n = |V|$. Then
% \[
%   |V| + |E(G)| - n = |E(G)|.
% \]
% \end{theorem}
%
% \begin{proof}
% \leanok
%
% Since $|V| = n$, we have $|V| + |E(G)| - n = n + |E(G)| - n = |E(G)|$ by integer arithmetic.
% \end{proof}

%--- Lem_3: SpaceDistance ---

I'll start by reading the Lean file to understand its contents.Now I have the complete file. Let me produce the LaTeX translation.

\chapter{Lem 3: Space Distance}

\begin{definition}[Restriction to Vertex Qubits]
\label{def:SpaceDistance.restrictToV}
\lean{QEC1.SpaceDistance.restrictToV}
\leanok
\uses{def:GaussFlux.ExtQubit, def:PauliOp}
Given a Pauli operator $P$ on the extended qubit space $V \oplus E$, the \emph{restriction to vertex qubits} $\operatorname{restrictToV}(P)$ is the Pauli operator on $V$ defined by
\[
\operatorname{restrictToV}(P).\mathbf{x}(v) = P.\mathbf{x}(\operatorname{inl}(v)), \qquad
\operatorname{restrictToV}(P).\mathbf{z}(v) = P.\mathbf{z}(\operatorname{inl}(v)).
\]
\end{definition}

\begin{definition}[Edge X-Support]
\label{def:SpaceDistance.edgeXSupport}
\lean{QEC1.SpaceDistance.edgeXSupport}
\leanok
\uses{def:GaussFlux.ExtQubit, def:PauliOp}
Given a Pauli operator $P$ on $V \oplus E$, the \emph{edge X-support} is the function $\operatorname{edgeXSupport}(P) : E(G) \to \mathbb{Z}/2\mathbb{Z}$ defined by
\[
\operatorname{edgeXSupport}(P)(e) = P.\mathbf{x}(\operatorname{inr}(e)).
\]
\end{definition}

\begin{definition}[Cleaned Operator]
\label{def:SpaceDistance.cleanedOp}
\lean{QEC1.SpaceDistance.cleanedOp}
\leanok
\uses{def:GaussFlux.ExtQubit, def:PauliOp, def:GaugingMeasurementCorrectness.gaussSubsetProduct}
Given a Pauli operator $L'$ on $V \oplus E$ and a function $c : V \to \mathbb{Z}/2\mathbb{Z}$, the \emph{cleaned operator} is defined by
\[
\operatorname{cleanedOp}(L', c) = L' \cdot \operatorname{gaussSubsetProduct}(G, c).
\]
\end{definition}

\begin{lemma}[Cleaned Operator X-Vector on Edges]
\label{lem:SpaceDistance.cleanedOp_xVec_edge}
\lean{QEC1.SpaceDistance.cleanedOp_xVec_edge}
\leanok
\uses{def:SpaceDistance.cleanedOp, def:GaugingMeasurementCorrectness.gaussSubsetProduct, def:GraphMaps.coboundaryMap}
For any Pauli operator $L'$ on $V \oplus E$, function $c : V \to \mathbb{Z}/2\mathbb{Z}$, and edge $e \in E(G)$:
\[
\operatorname{cleanedOp}(L', c).\mathbf{x}(\operatorname{inr}(e)) = L'.\mathbf{x}(\operatorname{inr}(e)) + \delta(c)(e),
\]
where $\delta$ denotes the coboundary map.
\end{lemma}

\begin{proof}
\leanok
\uses{def:SpaceDistance.cleanedOp, def:GaugingMeasurementCorrectness.gaussSubsetProduct}
By definition of $\operatorname{cleanedOp}$ and the multiplication rule for Pauli operators, we simplify using the $\mathbf{x}$-vector multiplication formula and pointwise addition. The result follows directly.
\end{proof}

\begin{lemma}[Cleaned Operator X-Vector on Vertices]
\label{lem:SpaceDistance.cleanedOp_xVec_vertex}
\lean{QEC1.SpaceDistance.cleanedOp_xVec_vertex}
\leanok
\uses{def:SpaceDistance.cleanedOp, def:GaugingMeasurementCorrectness.gaussSubsetProduct}
For any $L'$, $c$, and vertex $v \in V$:
\[
\operatorname{cleanedOp}(L', c).\mathbf{x}(\operatorname{inl}(v)) = L'.\mathbf{x}(\operatorname{inl}(v)) + c(v).
\]
\end{lemma}

\begin{proof}
\leanok
\uses{def:SpaceDistance.cleanedOp, def:GaugingMeasurementCorrectness.gaussSubsetProduct}
This follows by simplification using the definition of $\operatorname{cleanedOp}$, the Pauli multiplication rule for $\mathbf{x}$-vectors, and pointwise addition.
\end{proof}

\begin{lemma}[Cleaned Operator Preserves Z-Vector]
\label{lem:SpaceDistance.cleanedOp_zVec}
\lean{QEC1.SpaceDistance.cleanedOp_zVec}
\leanok
\uses{def:SpaceDistance.cleanedOp, def:GaugingMeasurementCorrectness.gaussSubsetProduct}
For any $L'$ and $c$:
\[
\operatorname{cleanedOp}(L', c).\mathbf{z} = L'.\mathbf{z}.
\]
\end{lemma}

\begin{proof}
\leanok
\uses{def:SpaceDistance.cleanedOp, def:GaugingMeasurementCorrectness.gaussSubsetProduct}
By extensionality, for an arbitrary qubit $q$, we expand the definition of $\operatorname{cleanedOp}$ and use the Pauli multiplication rule for $\mathbf{z}$-vectors. Since $\operatorname{gaussSubsetProduct}(G, c)$ is pure $X$, its $\mathbf{z}$-component is zero, and the result follows by simplification using pointwise addition.
\end{proof}

\begin{lemma}[Cleaned Operator Has No X-Support on Edges]
\label{lem:SpaceDistance.cleaned_has_no_xSupport_on_edges}
\lean{QEC1.SpaceDistance.cleaned_has_no_xSupport_on_edges}
\leanok
\uses{def:SpaceDistance.cleanedOp, def:SpaceDistance.edgeXSupport, def:GraphMaps.coboundaryMap}
If $\delta(c)(e) = L'.\mathbf{x}(\operatorname{inr}(e))$ for all edges $e$, then
\[
\operatorname{cleanedOp}(L', c).\mathbf{x}(\operatorname{inr}(e)) = 0 \quad \text{for all } e \in E(G).
\]
\end{lemma}

\begin{proof}
\leanok
\uses{lem:SpaceDistance.cleanedOp_xVec_edge}
Let $e$ be an arbitrary edge. Rewriting using the cleaned operator X-vector formula on edges, we get $L'.\mathbf{x}(\operatorname{inr}(e)) + \delta(c)(e)$. By the hypothesis $\delta(c)(e) = L'.\mathbf{x}(\operatorname{inr}(e))$, this equals $L'.\mathbf{x}(\operatorname{inr}(e)) + L'.\mathbf{x}(\operatorname{inr}(e)) = 0$ over $\mathbb{Z}/2\mathbb{Z}$.
\end{proof}

\begin{lemma}[Symplectic Inner Product with Deformed Check, No X on Edges]
\label{lem:SpaceDistance.symplecticInner_noXEdge_deformed}
\lean{QEC1.SpaceDistance.symplecticInner_noXEdge_deformed}
\leanok
\uses{def:SpaceDistance.restrictToV, def:PauliOp.symplecticInner, def:DeformedCode.deformedOriginalChecks}
Let $\bar{L}$ be a Pauli operator on $V \oplus E$ with $\bar{L}.\mathbf{x}(\operatorname{inr}(e)) = 0$ for all edges $e$. Then for any original check index $j$:
\[
\langle \bar{L}, \tilde{s}_j \rangle = \langle \operatorname{restrictToV}(\bar{L}), s_j \rangle,
\]
where $\tilde{s}_j$ denotes the deformed original check and $s_j$ the original check.
\end{lemma}

\begin{proof}
\leanok
\uses{def:SpaceDistance.restrictToV, def:PauliOp.symplecticInner, def:DeformedCode.deformedOriginalChecks}
We unfold the symplectic inner product and decompose the sum over $V \oplus E$ into vertex and edge contributions using $\sum_{V \oplus E} = \sum_V + \sum_E$.

For the edge contribution: each summand involves $\bar{L}.\mathbf{x}(\operatorname{inr}(e))$, which is zero by hypothesis, and the $\mathbf{x}$-component of the deformed check on edges, which can be simplified using the definitions. The entire edge sum equals zero.

For the vertex contribution: by the definitions of $\operatorname{restrictToV}$ and $\operatorname{deformedOriginalChecks}$, each vertex summand for $\bar{L}$ agrees with the corresponding summand for $\operatorname{restrictToV}(\bar{L})$ and $s_j$.

Combining, we rewrite with the vertex equality and the zero edge sum, and the result follows by adding zero.
\end{proof}

\begin{lemma}[Cleaned Operator Commutes with Original Check]
\label{lem:SpaceDistance.cleaned_commutes_with_original_check}
\lean{QEC1.SpaceDistance.cleaned_commutes_with_original_check}
\leanok
\uses{def:SpaceDistance.restrictToV, def:PauliOp.PauliCommute, def:DeformedCode.deformedOriginalChecks}
If $\bar{L}$ commutes with the deformed check $\tilde{s}_j$ and has no X-support on edges, then $\operatorname{restrictToV}(\bar{L})$ commutes with the original check $s_j$.
\end{lemma}

\begin{proof}
\leanok
\uses{lem:SpaceDistance.symplecticInner_noXEdge_deformed}
Unfolding the definition of Pauli commutation, we rewrite the symplectic inner product $\langle \operatorname{restrictToV}(\bar{L}), s_j \rangle$ as $\langle \bar{L}, \tilde{s}_j \rangle$ using the symplectic inner product lemma for operators with no X on edges. The result then follows directly from the commutativity hypothesis $\langle \bar{L}, \tilde{s}_j \rangle = 0$.
\end{proof}

\begin{lemma}[Cleaned Operator Commutes with All Original Checks]
\label{lem:SpaceDistance.cleaned_commutes_with_all_original_checks}
\lean{QEC1.SpaceDistance.cleaned_commutes_with_all_original_checks}
\leanok
\uses{def:SpaceDistance.restrictToV, def:PauliOp.PauliCommute, def:DeformedCode.deformedOriginalChecks}
If $\bar{L}$ commutes with all deformed checks and has no X-support on edges, then $\operatorname{restrictToV}(\bar{L})$ commutes with all original checks.
\end{lemma}

\begin{proof}
\leanok
\uses{lem:SpaceDistance.cleaned_commutes_with_original_check}
Let $j$ be an arbitrary check index. We apply the single-check commutation lemma to the hypothesis that $\bar{L}$ commutes with $\tilde{s}_j$.
\end{proof}

\begin{theorem}[gaussSubsetProduct Commutes with Deformed Checks]
\label{thm:SpaceDistance.gaussSubsetProduct_comm_deformedCheck}
\lean{QEC1.SpaceDistance.gaussSubsetProduct_comm_deformedCheck}
\leanok
\uses{def:GaugingMeasurementCorrectness.gaussSubsetProduct, def:GaugingMeasurementCorrectness.finsetIndicator, def:DeformedCode.deformedOriginalChecks, def:PauliOp.PauliCommute, thm:GaugingMeasurementCorrectness.prod_gaussLawOp_eq_gaussSubsetProduct}
For any deformed code data and cleaning vector $c : V \to \mathbb{Z}/2\mathbb{Z}$, the operator $\operatorname{gaussSubsetProduct}(G, c)$ commutes with every deformed original check $\tilde{s}_j$.
\end{theorem}

\begin{proof}
\leanok
\uses{def:GaugingMeasurementCorrectness.finsetIndicator, thm:GaugingMeasurementCorrectness.prod_gaussLawOp_eq_gaussSubsetProduct, thm:PauliOp.pauliCommute_comm}
We first rewrite $\operatorname{gaussSubsetProduct}(G, c)$ as $\operatorname{gaussSubsetProduct}(G, \mathbf{1}_S)$ where $S = \{v \mid c(v) = 1\}$, using the fact that any $\mathbb{Z}/2\mathbb{Z}$-valued function equals the indicator of its support. We then rewrite using the product formula $\operatorname{gaussSubsetProduct}(G, \mathbf{1}_S) = \prod_{v \in S} A_v$. Each $A_v$ commutes with $\tilde{s}_j$ by the boundary condition (since each Gauss law operator commutes with each deformed check). Since a product of operators that each commute with $R$ also commutes with $R$, the result follows.
\end{proof}

\begin{theorem}[gaussSubsetProduct Commutes with Gauss Law Checks]
\label{thm:SpaceDistance.gaussSubsetProduct_comm_gaussLaw}
\lean{QEC1.SpaceDistance.gaussSubsetProduct_comm_gaussLaw}
\leanok
\uses{def:GaugingMeasurementCorrectness.gaussSubsetProduct, def:DeformedCode.gaussLawChecks, def:PauliOp.PauliCommute}
For any cleaning vector $c$ and vertex $v$, the operator $\operatorname{gaussSubsetProduct}(G, c)$ commutes with the Gauss law check $A_v$.
\end{theorem}

\begin{proof}
\leanok
\uses{def:GaugingMeasurementCorrectness.gaussSubsetProduct, def:DeformedCode.gaussLawChecks, def:GaussFlux.gaussLawOp}
Unfolding Pauli commutation and the symplectic inner product, we decompose the sum over $V \oplus E$. For the edge contribution: the $\mathbf{z}$-components of both $\operatorname{gaussSubsetProduct}(G, c)$ and the Gauss law check vanish on edges (since both are pure $X$ operators on their respective supports). The edge sum is zero. For the vertex contribution: similarly, both $\mathbf{z}$-components vanish on vertices. The vertex sum is also zero. Combining, the total symplectic inner product is $0 + 0 = 0$.
\end{proof}

\begin{theorem}[gaussSubsetProduct Commutes with Flux Checks]
\label{thm:SpaceDistance.gaussSubsetProduct_comm_flux}
\lean{QEC1.SpaceDistance.gaussSubsetProduct_comm_flux}
\leanok
\uses{def:GaugingMeasurementCorrectness.gaussSubsetProduct, def:GaugingMeasurementCorrectness.finsetIndicator, def:DeformedCode.fluxChecks, def:PauliOp.PauliCommute, thm:GaugingMeasurementCorrectness.prod_gaussLawOp_eq_gaussSubsetProduct}
Under the cycle parity condition, for any cleaning vector $c$ and cycle $p$, the operator $\operatorname{gaussSubsetProduct}(G, c)$ commutes with the flux check $B_p$.
\end{theorem}

\begin{proof}
\leanok
\uses{def:GaugingMeasurementCorrectness.finsetIndicator, thm:GaugingMeasurementCorrectness.prod_gaussLawOp_eq_gaussSubsetProduct}
We rewrite $\operatorname{gaussSubsetProduct}(G, c)$ as a product of Gauss law operators $\prod_{v \in S} A_v$ using the indicator and product formula. Each Gauss law operator $A_v$ commutes with each flux check $B_p$ by the pairwise commutation relations between Gauss law and flux checks. Since a product of operators commuting with $B_p$ also commutes with $B_p$, the result follows.
\end{proof}

\begin{theorem}[gaussSubsetProduct Commutes with All Deformed Code Checks]
\label{thm:SpaceDistance.gaussSubsetProduct_comm_allChecks}
\lean{QEC1.SpaceDistance.gaussSubsetProduct_comm_allChecks}
\leanok
\uses{def:GaugingMeasurementCorrectness.gaussSubsetProduct, def:DeformedCode.deformedCodeChecks, def:DeformedCode.CheckIndex, def:PauliOp.PauliCommute}
Under the cycle parity condition, for any cleaning vector $c$ and any check index $a$ of the deformed code, $\operatorname{gaussSubsetProduct}(G, c)$ commutes with the deformed code check at index $a$.
\end{theorem}

\begin{proof}
\leanok
\uses{thm:SpaceDistance.gaussSubsetProduct_comm_gaussLaw, thm:SpaceDistance.gaussSubsetProduct_comm_flux, thm:SpaceDistance.gaussSubsetProduct_comm_deformedCheck}
We case-split on the check index $a$:
\begin{itemize}
\item If $a = \operatorname{gaussLaw}(v)$: apply the gaussSubsetProduct-commutes-with-Gauss-law result.
\item If $a = \operatorname{flux}(p)$: apply the gaussSubsetProduct-commutes-with-flux result.
\item If $a = \operatorname{deformed}(j)$: apply the gaussSubsetProduct-commutes-with-deformed-check result.
\end{itemize}
\end{proof}

\begin{theorem}[gaussSubsetProduct Is in the Centralizer]
\label{thm:SpaceDistance.gaussSubsetProduct_inCentralizer}
\lean{QEC1.SpaceDistance.gaussSubsetProduct_inCentralizer}
\leanok
\uses{def:GaugingMeasurementCorrectness.gaussSubsetProduct, def:DeformedCodeChecks.deformedStabilizerCode, def:StabilizerCode.inCentralizer}
Under the cycle parity condition and pairwise commutation of original checks, for any cleaning vector $c$, the operator $\operatorname{gaussSubsetProduct}(G, c)$ is in the centralizer of the deformed stabilizer code.
\end{theorem}

\begin{proof}
\leanok
\uses{thm:SpaceDistance.gaussSubsetProduct_comm_allChecks}
Let $a$ be an arbitrary check index. The result follows directly from the fact that $\operatorname{gaussSubsetProduct}(G, c)$ commutes with all deformed code checks.
\end{proof}

\begin{theorem}[gaussSubsetProduct Is in the Stabilizer Group]
\label{thm:SpaceDistance.gaussSubsetProduct_mem_stabilizerGroup}
\lean{QEC1.SpaceDistance.gaussSubsetProduct_mem_stabilizerGroup}
\leanok
\uses{def:GaugingMeasurementCorrectness.gaussSubsetProduct, def:GaugingMeasurementCorrectness.finsetIndicator, def:DeformedCodeChecks.deformedStabilizerCode, def:StabilizerCode.stabilizerGroup, thm:GaugingMeasurementCorrectness.prod_gaussLawOp_eq_gaussSubsetProduct, thm:DeformedCodeChecks.gaussLaw_mem_stabilizerGroup}
Under the cycle parity and commutation conditions, for any cleaning vector $c$, the operator $\operatorname{gaussSubsetProduct}(G, c)$ is in the stabilizer group of the deformed code.
\end{theorem}

\begin{proof}
\leanok
\uses{def:GaugingMeasurementCorrectness.finsetIndicator, thm:GaugingMeasurementCorrectness.prod_gaussLawOp_eq_gaussSubsetProduct, thm:DeformedCodeChecks.gaussLaw_mem_stabilizerGroup}
We rewrite $\operatorname{gaussSubsetProduct}(G, c)$ as $\operatorname{gaussSubsetProduct}(G, \mathbf{1}_S)$ where $S = \{v \mid c(v) = 1\}$. Then we apply the product formula to express it as $\prod_{v \in S} A_v$. Since the stabilizer group is a subgroup and each Gauss law operator $A_v$ is in the stabilizer group of the deformed code, the product is also in the stabilizer group.
\end{proof}

\begin{lemma}[Weight Decomposition into Vertex and Edge Contributions]
\label{lem:SpaceDistance.weight_eq_vertex_plus_edge}
\lean{QEC1.SpaceDistance.weight_eq_vertex_plus_edge}
\leanok
\uses{def:PauliOp.weight, def:GaussFlux.ExtQubit}
For a Pauli operator $P$ on $V \oplus E$:
\[
\operatorname{wt}(P) = |\{v \in V \mid P.\mathbf{x}(\operatorname{inl}(v)) \neq 0 \lor P.\mathbf{z}(\operatorname{inl}(v)) \neq 0\}| + |\{e \in E \mid P.\mathbf{x}(\operatorname{inr}(e)) \neq 0 \lor P.\mathbf{z}(\operatorname{inr}(e)) \neq 0\}|.
\]
\end{lemma}

\begin{proof}
\leanok
\uses{def:PauliOp.weight, def:PauliOp.support}
We unfold the weight as the cardinality of the support. The support over $V \oplus E$ decomposes as the disjoint union of the vertex support (mapped via $\operatorname{inl}$) and the edge support (mapped via $\operatorname{inr}$). We establish this decomposition explicitly by showing that every element of $V \oplus E$ in the support is either $\operatorname{inl}(v)$ or $\operatorname{inr}(e)$. The disjointness follows because $\operatorname{inl}$ and $\operatorname{inr}$ have disjoint images. Taking cardinalities of a disjoint union gives the sum of cardinalities.
\end{proof}

\begin{lemma}[Weight Decomposition via Restriction]
\label{lem:SpaceDistance.weight_decomposition}
\lean{QEC1.SpaceDistance.weight_decomposition}
\leanok
\uses{def:SpaceDistance.restrictToV, def:PauliOp.weight, def:GaussFlux.ExtQubit}
For a Pauli operator $P$ on $V \oplus E$:
\[
\operatorname{wt}(P) = \operatorname{wt}(\operatorname{restrictToV}(P)) + |\{e \in E \mid P.\mathbf{x}(\operatorname{inr}(e)) \neq 0 \lor P.\mathbf{z}(\operatorname{inr}(e)) \neq 0\}|.
\]
\end{lemma}

\begin{proof}
\leanok
\uses{lem:SpaceDistance.weight_eq_vertex_plus_edge}
Rewriting using the vertex-plus-edge weight decomposition, we observe that the vertex contribution equals the weight of $\operatorname{restrictToV}(P)$ by congruence.
\end{proof}

\begin{lemma}[Restriction of Logical Operator]
\label{lem:SpaceDistance.restrictToV_logicalOp}
\lean{QEC1.SpaceDistance.restrictToV_logicalOp}
\leanok
\uses{def:SpaceDistance.restrictToV, def:GaussFlux.logicalOp, def:GaugingMeasurementCorrectness.logicalOpV}
The restriction of the logical operator $L = \prod_q X_q$ on $V \oplus E$ to vertex qubits equals $\operatorname{logicalOpV}$:
\[
\operatorname{restrictToV}(\operatorname{logicalOp}(G)) = \operatorname{logicalOpV}.
\]
\end{lemma}

\begin{proof}
\leanok
\uses{def:SpaceDistance.restrictToV, def:GaussFlux.logicalOp, def:GaugingMeasurementCorrectness.logicalOpV}
By extensionality on both the $\mathbf{x}$ and $\mathbf{z}$ components, this follows by simplification using the definitions of $\operatorname{restrictToV}$, $\operatorname{logicalOp}$, and $\operatorname{logicalOpV}$.
\end{proof}

\begin{lemma}[Cleaned Operators with Complementary Vectors Differ by Logical]
\label{lem:SpaceDistance.cleanedOp_complement_eq_mul_logical}
\lean{QEC1.SpaceDistance.cleanedOp_complement_eq_mul_logical}
\leanok
\uses{def:SpaceDistance.cleanedOp, def:GaugingMeasurementCorrectness.gaussSubsetProduct, def:GaussFlux.logicalOp, thm:GaugingMeasurementCorrectness.gaussSubsetProduct_add, thm:GaugingMeasurementCorrectness.gaussSubsetProduct_allOnes}
For any $L'$ and $c$:
\[
\operatorname{cleanedOp}(L', c + \mathbf{1}) = \operatorname{cleanedOp}(L', c) \cdot \operatorname{logicalOp}(G).
\]
\end{lemma}

\begin{proof}
\leanok
\uses{def:SpaceDistance.cleanedOp, thm:GaugingMeasurementCorrectness.gaussSubsetProduct_add, thm:GaugingMeasurementCorrectness.gaussSubsetProduct_allOnes}
Expanding $\operatorname{cleanedOp}$, we have $L' \cdot \operatorname{gaussSubsetProduct}(G, c + \mathbf{1})$. Using the additivity $\operatorname{gaussSubsetProduct}(G, c + \mathbf{1}) = \operatorname{gaussSubsetProduct}(G, c) \cdot \operatorname{gaussSubsetProduct}(G, \mathbf{1})$ and the identity $\operatorname{gaussSubsetProduct}(G, \mathbf{1}) = \operatorname{logicalOp}(G)$, together with associativity of multiplication, the result follows.
\end{proof}

\begin{lemma}[Restrictions of Complementary Cleaned Operators Differ by logicalOpV]
\label{lem:SpaceDistance.restrictToV_cleanedOp_complement}
\lean{QEC1.SpaceDistance.restrictToV_cleanedOp_complement}
\leanok
\uses{def:SpaceDistance.restrictToV, def:SpaceDistance.cleanedOp, def:GaugingMeasurementCorrectness.logicalOpV}
For any $L'$ and $c$:
\[
\operatorname{restrictToV}(\operatorname{cleanedOp}(L', c + \mathbf{1})) = \operatorname{restrictToV}(\operatorname{cleanedOp}(L', c)) \cdot \operatorname{logicalOpV}.
\]
\end{lemma}

\begin{proof}
\leanok
\uses{lem:SpaceDistance.cleanedOp_complement_eq_mul_logical, lem:SpaceDistance.restrictToV_logicalOp}
Rewriting the left-hand side using the complement-equals-multiply-by-logical lemma, then distributing $\operatorname{restrictToV}$ over multiplication, and replacing $\operatorname{restrictToV}(\operatorname{logicalOp}(G))$ by $\operatorname{logicalOpV}$, the result follows.
\end{proof}

\begin{theorem}[Cleaned Operator Not in Stabilizer Group]
\label{thm:SpaceDistance.cleanedOp_not_in_stabilizerGroup}
\lean{QEC1.SpaceDistance.cleanedOp_not_in_stabilizerGroup}
\leanok
\uses{def:SpaceDistance.cleanedOp, def:DeformedCodeChecks.deformedStabilizerCode, def:StabilizerCode.stabilizerGroup, def:StabilizerCode.isLogicalOp, def:GaugingMeasurementCorrectness.gaussSubsetProduct}
If $L'$ is a logical operator of the deformed code, then for any cleaning vector $c$, the cleaned operator $\operatorname{cleanedOp}(L', c) \notin \mathcal{S}^*$, where $\mathcal{S}^*$ is the stabilizer group of the deformed code.
\end{theorem}

\begin{proof}
\leanok
\uses{thm:SpaceDistance.gaussSubsetProduct_mem_stabilizerGroup}
Suppose for contradiction that $\operatorname{cleanedOp}(L', c) = L' \cdot \operatorname{gaussSubsetProduct}(G, c) \in \mathcal{S}^*$. Since $\operatorname{gaussSubsetProduct}(G, c) \in \mathcal{S}^*$, we can recover $L' = \operatorname{cleanedOp}(L', c) \cdot (\operatorname{gaussSubsetProduct}(G, c))^{-1}$. Since $\mathcal{S}^*$ is a subgroup, this product is in $\mathcal{S}^*$, so $L' \in \mathcal{S}^*$, contradicting $L'$ being a logical operator (which requires $L' \notin \mathcal{S}^*$).
\end{proof}

\begin{theorem}[Cleaned Operator Is Not Identity]
\label{thm:SpaceDistance.cleanedOp_ne_one}
\lean{QEC1.SpaceDistance.cleanedOp_ne_one}
\leanok
\uses{def:SpaceDistance.cleanedOp, def:DeformedCodeChecks.deformedStabilizerCode, def:StabilizerCode.isLogicalOp, def:GaugingMeasurementCorrectness.gaussSubsetProduct}
If $L'$ is a logical operator of the deformed code, then for any cleaning vector $c$, $\operatorname{cleanedOp}(L', c) \neq 1$.
\end{theorem}

\begin{proof}
\leanok
\uses{thm:SpaceDistance.gaussSubsetProduct_mem_stabilizerGroup}
Suppose $\operatorname{cleanedOp}(L', c) = L' \cdot \operatorname{gaussSubsetProduct}(G, c) = 1$. Then multiplying both sides on the right by $\operatorname{gaussSubsetProduct}(G, c)$ and using self-inverseness (in $\mathbb{Z}/2\mathbb{Z}$ Pauli algebra, $P^2 = I$), we obtain $L' = \operatorname{gaussSubsetProduct}(G, c)$. Since $\operatorname{gaussSubsetProduct}(G, c) \in \mathcal{S}^*$, we get $L' \in \mathcal{S}^*$, contradicting $L'$ being a logical operator.
\end{proof}

\begin{theorem}[Cleaned Operator Is Logical]
\label{thm:SpaceDistance.cleanedOp_isLogical}
\lean{QEC1.SpaceDistance.cleanedOp_isLogical}
\leanok
\uses{def:SpaceDistance.cleanedOp, def:DeformedCodeChecks.deformedStabilizerCode, def:StabilizerCode.isLogicalOp}
If $L'$ is a logical operator of the deformed code, then for any cleaning vector $c$, $\operatorname{cleanedOp}(L', c)$ is also a logical operator of the deformed code.
\end{theorem}

\begin{proof}
\leanok
\uses{thm:SpaceDistance.gaussSubsetProduct_inCentralizer, thm:SpaceDistance.cleanedOp_not_in_stabilizerGroup, thm:SpaceDistance.cleanedOp_ne_one, thm:PauliOp.symplecticInner_mul_left}
We verify the three conditions for being a logical operator:
\begin{enumerate}
\item \textbf{Centralizer membership:} For any check index $a$, unfolding Pauli commutation and using the symplectic inner product multiplicativity $\langle P \cdot Q, R \rangle = \langle P, R \rangle + \langle Q, R \rangle$, the commutation of $L'$ with all checks and the centralizer membership of $\operatorname{gaussSubsetProduct}(G, c)$ give $\langle L', \text{check}_a \rangle + \langle \operatorname{gaussSubsetProduct}(G, c), \text{check}_a \rangle = 0 + 0 = 0$.
\item \textbf{Not in stabilizer group:} This follows from the cleaned-operator-not-in-stabilizer-group theorem.
\item \textbf{Not identity:} This follows from the cleaned-operator-not-identity theorem.
\end{enumerate}
\end{proof}

\begin{theorem}[Key Theorem: At Least One Cleaned Restriction Is a Logical]
\label{thm:SpaceDistance.cleaned_restriction_logical_disjunction}
\lean{QEC1.SpaceDistance.cleaned_restriction_logical_disjunction}
\leanok
\uses{def:SpaceDistance.restrictToV, def:SpaceDistance.cleanedOp, def:SpaceDistance.edgeXSupport, def:DeformedCodeChecks.deformedStabilizerCode, def:StabilizerCode.isLogicalOp, def:GaugingMeasurementCorrectness.logicalOpV, def:GraphMaps.coboundaryMap}
Let $L'$ be a logical operator of the deformed code and $c$ a cleaning vector with $\delta(c) = \operatorname{edgeXSupport}(L')$. Suppose $\operatorname{logicalOpV}$ is a logical of the original code. Then at least one of the following holds:
\begin{itemize}
\item $\operatorname{restrictToV}(\operatorname{cleanedOp}(L', c))$ is a logical of the original code, or
\item $\operatorname{restrictToV}(\operatorname{cleanedOp}(L', c + \mathbf{1}))$ is a logical of the original code.
\end{itemize}
\end{theorem}

\begin{proof}
\leanok
\uses{lem:SpaceDistance.cleaned_has_no_xSupport_on_edges, lem:SpaceDistance.cleaned_commutes_with_original_check, lem:SpaceDistance.restrictToV_cleanedOp_complement, thm:SpaceDistance.cleanedOp_isLogical, thm:GraphMaps.allOnes_mem_ker_coboundary}
Both cleaned operators $\bar{L}_0 = \operatorname{cleanedOp}(L', c)$ and $\bar{L}_1 = \operatorname{cleanedOp}(L', c + \mathbf{1})$ have no X-support on edges. For $\bar{L}_0$, this follows from $\delta(c) = \operatorname{edgeXSupport}(L')$. For $\bar{L}_1$, since $\delta(\mathbf{1}) = 0$ (the all-ones vector is in the kernel of the coboundary map), $\delta(c + \mathbf{1}) = \delta(c) + 0 = \delta(c) = \operatorname{edgeXSupport}(L')$.

Both cleaned operators are logicals of the deformed code, so they commute with all deformed checks. Since they have no X on edges, their restrictions commute with all original checks, placing both restrictions in the centralizer of the original code.

The two restrictions differ by $\operatorname{logicalOpV}$: $\operatorname{restrictToV}(\bar{L}_1) = \operatorname{restrictToV}(\bar{L}_0) \cdot \operatorname{logicalOpV}$.

Suppose for contradiction that neither restriction is a logical. Each is in the centralizer but not a logical, so each is either in the original stabilizer group or equals $1$. We consider four cases:
\begin{itemize}
\item Both in the stabilizer group: their quotient $\operatorname{logicalOpV}$ would be in the stabilizer group, contradicting $\operatorname{logicalOpV}$ being a logical.
\item First in stabilizer, second is $1$: then $\operatorname{restrictToV}(\bar{L}_0) \cdot \operatorname{logicalOpV} = 1$, so $\operatorname{logicalOpV} = \operatorname{restrictToV}(\bar{L}_0) \in \mathcal{S}$, a contradiction.
\item First is $1$, second in stabilizer: then $\operatorname{logicalOpV} = \operatorname{restrictToV}(\bar{L}_1) \in \mathcal{S}$, a contradiction.
\item Both equal $1$: then $\operatorname{logicalOpV} = 1$, contradicting $\operatorname{logicalOpV}$ being a logical.
\end{itemize}
All cases lead to contradiction.
\end{proof}

\begin{theorem}[Symplectic Inner Product with Flux Check Equals Second Coboundary]
\label{thm:SpaceDistance.symplecticInner_flux_eq_secondCoboundary}
\lean{QEC1.SpaceDistance.symplecticInner_flux_eq_secondCoboundary}
\leanok
\uses{def:SpaceDistance.edgeXSupport, def:DeformedCode.fluxChecks, def:PauliOp.symplecticInner, def:GraphMaps.secondCoboundaryMap}
For any Pauli operator $L'$ on $V \oplus E$ and cycle $p$:
\[
\langle L', B_p \rangle = \delta_2(\operatorname{edgeXSupport}(L'))(p),
\]
where $\delta_2$ is the second coboundary map.
\end{theorem}

\begin{proof}
\leanok
\uses{def:SpaceDistance.edgeXSupport, def:DeformedCode.fluxChecks, def:GaussFlux.fluxOp, def:GraphMaps.secondCoboundaryMap}
Unfolding the symplectic inner product and decomposing the sum over $V \oplus E$: the vertex contribution is zero because the flux check $B_p$ has both $\mathbf{x}$ and $\mathbf{z}$ components equal to zero on vertices (as $B_p$ is supported only on edges). After adding zero, the edge sum is rewritten using the definitions of $\operatorname{fluxChecks}$ and $\operatorname{secondCoboundaryMap}$. Each edge summand matches by case analysis on whether the edge belongs to the cycle, giving the desired equality.
\end{proof}

\begin{theorem}[Step 2: Edge X-Support Is a Cocycle]
\label{thm:SpaceDistance.edgeXSupport_in_ker_secondCoboundary}
\lean{QEC1.SpaceDistance.edgeXSupport_in_ker_secondCoboundary}
\leanok
\uses{def:SpaceDistance.edgeXSupport, def:DeformedCodeChecks.deformedStabilizerCode, def:StabilizerCode.isLogicalOp, def:GraphMaps.secondCoboundaryMap}
If $L'$ is a logical operator of the deformed code, then $\operatorname{edgeXSupport}(L') \in \ker(\delta_2)$, i.e., the edge X-support is a $1$-cocycle.
\end{theorem}

\begin{proof}
\leanok
\uses{thm:SpaceDistance.symplecticInner_flux_eq_secondCoboundary}
We must show $\delta_2(\operatorname{edgeXSupport}(L')) = 0$. By extensionality, let $p$ be an arbitrary cycle. Since $L'$ is a logical of the deformed code, it commutes with the flux check $B_p$, meaning $\langle L', B_p \rangle = 0$. By the symplectic-inner-product-equals-second-coboundary theorem, $\delta_2(\operatorname{edgeXSupport}(L'))(p) = \langle L', B_p \rangle = 0$.
\end{proof}

\begin{theorem}[Coboundary Support Equals Edge Boundary]
\label{thm:SpaceDistance.coboundary_support_card_eq_edgeBoundary_card}
\lean{QEC1.SpaceDistance.coboundary_support_card_eq_edgeBoundary_card}
\leanok
\uses{def:GraphMaps.coboundaryMap, def:SimpleGraph.edgeBoundary}
For any $c : V \to \mathbb{Z}/2\mathbb{Z}$:
\[
|\{e \in E(G) \mid \delta(c)(e) \neq 0\}| = |\partial_E(\operatorname{supp}(c))|,
\]
where $\operatorname{supp}(c) = \{v \in V \mid c(v) \neq 0\}$ and $\partial_E$ denotes the edge boundary.
\end{theorem}

\begin{proof}
\leanok
\uses{def:GraphMaps.coboundaryMap, def:SimpleGraph.edgeBoundary}
Let $S = \{v \mid c(v) \neq 0\}$. We show that the edge boundary $\partial_E(S)$ equals the set of coboundary-support edges (mapped via the subtype embedding) by establishing a bijection.

For the forward direction: given an edge $e = s(a,b) \in \partial_E(S)$ (meaning $|s(a,b).\operatorname{toFinset} \cap S| = 1$), we show $\delta(c)(e) \neq 0$. If $\delta(c)(e) = 0$, then $c(a) + c(b) = 0$ in $\mathbb{Z}/2\mathbb{Z}$, so $c(a) = c(b)$. If both are nonzero, the intersection has cardinality $2$, contradicting $|s(a,b).\operatorname{toFinset} \cap S| = 1$. If both are zero, the intersection is empty, also a contradiction.

For the backward direction: given $e = s(a,b)$ with $\delta(c)(e) = c(a) + c(b) \neq 0$, we have $c(a) \neq c(b)$. In $\mathbb{Z}/2\mathbb{Z}$, this means exactly one is $0$ and the other is $1$, so $|s(a,b).\operatorname{toFinset} \cap S| = 1$, placing $e$ in $\partial_E(S)$.
\end{proof}

\begin{lemma}[Edge Support at Least Coboundary Support]
\label{lem:SpaceDistance.edge_support_ge_coboundary_support}
\lean{QEC1.SpaceDistance.edge_support_ge_coboundary_support}
\leanok
\uses{def:SpaceDistance.edgeXSupport, def:GraphMaps.coboundaryMap}
If $\delta(c) = \operatorname{edgeXSupport}(L')$, then
\[
|\{e \mid \delta(c)(e) \neq 0\}| \leq |\{e \mid L'.\mathbf{x}(\operatorname{inr}(e)) \neq 0 \lor L'.\mathbf{z}(\operatorname{inr}(e)) \neq 0\}|.
\]
\end{lemma}

\begin{proof}
\leanok
\uses{def:SpaceDistance.edgeXSupport, def:GraphMaps.coboundaryMap}
We apply monotonicity of cardinality. For any edge $e$ with $\delta(c)(e) \neq 0$, substituting $\delta(c) = \operatorname{edgeXSupport}(L')$ gives $L'.\mathbf{x}(\operatorname{inr}(e)) \neq 0$, which implies the disjunction $L'.\mathbf{x}(\operatorname{inr}(e)) \neq 0 \lor L'.\mathbf{z}(\operatorname{inr}(e)) \neq 0$.
\end{proof}

\begin{lemma}[Support Partition]
\label{lem:SpaceDistance.support_add_complement_card}
\lean{QEC1.SpaceDistance.support_add_complement_card}
\leanok
\uses{def:GaugingMeasurementCorrectness.finsetIndicator}
For any $c : V \to \mathbb{Z}/2\mathbb{Z}$:
\[
|\operatorname{supp}(c)| + |\operatorname{supp}(c + \mathbf{1})| = |V|.
\]
\end{lemma}

\begin{proof}
\leanok

The support of $c + \mathbf{1}$ is the complement of the support of $c$: for any $v$, $c(v) + 1 \neq 0$ if and only if $c(v) = 0$ (since in $\mathbb{Z}/2\mathbb{Z}$, if $c(v) = 1$ then $c(v) + 1 = 0$, and if $c(v) = 0$ then $c(v) + 1 = 1 \neq 0$). The result follows from $|S| + |S^c| = |V|$ for any $S \subseteq V$.
\end{proof}

\begin{lemma}[Coboundary of Complement Equals Coboundary]
\label{lem:SpaceDistance.coboundary_add_ones_eq}
\lean{QEC1.SpaceDistance.coboundary_add_ones_eq}
\leanok
\uses{def:GraphMaps.coboundaryMap, def:GraphMaps.allOnes, thm:GraphMaps.allOnes_mem_ker_coboundary}
For any $c : V \to \mathbb{Z}/2\mathbb{Z}$:
\[
\delta(c + \mathbf{1}) = \delta(c).
\]
\end{lemma}

\begin{proof}
\leanok
\uses{thm:GraphMaps.allOnes_mem_ker_coboundary}
By linearity of $\delta$, $\delta(c + \mathbf{1}) = \delta(c) + \delta(\mathbf{1})$. The all-ones vector $\mathbf{1}$ is in the kernel of the coboundary map, so $\delta(\mathbf{1}) = 0$. Thus $\delta(c + \mathbf{1}) = \delta(c) + 0 = \delta(c)$.
\end{proof}

\begin{theorem}[Restriction of Deformed Original Check]
\label{thm:SpaceDistance.restrictToV_deformedOriginalCheck}
\lean{QEC1.SpaceDistance.restrictToV_deformedOriginalCheck}
\leanok
\uses{def:SpaceDistance.restrictToV, def:DeformedCode.deformedOriginalChecks}
For any deformed code data and check index $j$:
\[
\operatorname{restrictToV}(\tilde{s}_j) = s_j.
\]
\end{theorem}

\begin{proof}
\leanok
\uses{def:SpaceDistance.restrictToV, def:DeformedCode.deformedOriginalChecks, def:DeformedCode.deformedCheck, def:DeformedOperator.deformedOpExt}
By extensionality on both $\mathbf{x}$ and $\mathbf{z}$ components, this follows by simplification using the definitions of $\operatorname{restrictToV}$, $\operatorname{deformedOriginalChecks}$, $\operatorname{deformedCheck}$, and $\operatorname{deformedOpExt}$.
\end{proof}

\begin{definition}[Pure-Z Edge Operator Multiplication]
\label{def:SpaceDistance.pureZEdgeOp_mul}
\lean{QEC1.SpaceDistance.pureZEdgeOp_mul}
\leanok
\uses{def:FreedomInDeformedChecks.pureZEdgeOp}
Multiplication of pure-Z edge operators satisfies:
\[
\operatorname{pureZEdgeOp}(\delta_1) \cdot \operatorname{pureZEdgeOp}(\delta_2) = \operatorname{pureZEdgeOp}(\delta_1 + \delta_2).
\]
\end{definition}

\begin{theorem}[pureZEdgeOp in Stabilizer Group from Range of Second Boundary]
\label{thm:SpaceDistance.pureZEdgeOp_mem_stabilizerGroup_of_range}
\lean{QEC1.SpaceDistance.pureZEdgeOp_mem_stabilizerGroup_of_range}
\leanok
\uses{def:FreedomInDeformedChecks.pureZEdgeOp, def:DeformedCodeChecks.deformedStabilizerCode, def:StabilizerCode.stabilizerGroup, def:GraphMaps.secondBoundaryMap, def:DeformedCode.fluxChecks, thm:DeformedCodeChecks.flux_mem_stabilizerGroup}
If $\delta \in \operatorname{im}(\partial_2)$, then $\operatorname{pureZEdgeOp}(\delta)$ is in the stabilizer group of the deformed code.
\end{theorem}

\begin{proof}
\leanok
\uses{def:FreedomInDeformedChecks.pureZEdgeOp, def:DeformedCode.fluxChecks, def:GraphMaps.secondBoundaryMap, thm:DeformedCodeChecks.flux_mem_stabilizerGroup}
Since $\delta \in \operatorname{im}(\partial_2)$, there exists $\sigma : C \to \mathbb{Z}/2\mathbb{Z}$ with $\delta = \partial_2(\sigma)$. We decompose $\sigma = \sum_c \sigma(c) \cdot \mathbf{e}_c$ and apply the linearity of $\partial_2$ to get $\partial_2(\sigma) = \sum_c \sigma(c) \cdot \partial_2(\mathbf{e}_c)$. Over $\mathbb{Z}/2\mathbb{Z}$, $\sigma(c) \cdot x$ equals $x$ if $\sigma(c) = 1$ and $0$ otherwise, so the sum reduces to $\sum_{c : \sigma(c) \neq 0} \mathbf{1}_{\text{cycle}(c)}$, where $\mathbf{1}_{\text{cycle}(c)}$ is the indicator of cycle $c$.

We show that $\operatorname{pureZEdgeOp}$ of a sum equals the product of individual $\operatorname{pureZEdgeOp}$'s (by induction using the multiplication formula). Each $\operatorname{pureZEdgeOp}(\mathbf{1}_{\text{cycle}(c)})$ equals the flux check $B_c$, which is in the stabilizer group. Since the stabilizer group is closed under finite products, the result follows.
\end{proof}

\begin{theorem}[Cleaned Operator Cannot Be Identity on V]
\label{thm:SpaceDistance.cleaned_not_identity_on_V}
\lean{QEC1.SpaceDistance.cleaned_not_identity_on_V}
\leanok
\uses{def:SpaceDistance.restrictToV, def:SpaceDistance.cleanedOp, def:SpaceDistance.edgeXSupport, def:FreedomInDeformedChecks.pureZEdgeOp, def:DeformedCodeChecks.deformedStabilizerCode, def:StabilizerCode.isLogicalOp, def:GraphMaps.coboundaryMap, def:GraphMaps.boundaryMap, def:GraphMaps.secondBoundaryMap}
If $L'$ is a logical of the deformed code, $\delta(c) = \operatorname{edgeXSupport}(L')$, and $\operatorname{restrictToV}(\operatorname{cleanedOp}(L', c)) = 1$, then assuming boundary exactness ($\ker(\partial) \subseteq \operatorname{im}(\partial_2)$), we reach a contradiction.
\end{theorem}

\begin{proof}
\leanok
\uses{lem:SpaceDistance.cleaned_has_no_xSupport_on_edges, thm:SpaceDistance.cleanedOp_isLogical, thm:SpaceDistance.pureZEdgeOp_mem_stabilizerGroup_of_range}
Let $\bar{L} = \operatorname{cleanedOp}(L', c)$. Since $\delta(c) = \operatorname{edgeXSupport}(L')$, $\bar{L}$ has no X-support on edges. Since $\operatorname{restrictToV}(\bar{L}) = 1$, we have $\bar{L}.\mathbf{x}(\operatorname{inl}(v)) = 0$ and $\bar{L}.\mathbf{z}(\operatorname{inl}(v)) = 0$ for all $v$. Combined with zero X on edges, $\bar{L} = \operatorname{pureZEdgeOp}(\delta)$ where $\delta(e) = \bar{L}.\mathbf{z}(\operatorname{inr}(e))$.

Since $\bar{L}$ is a logical of the deformed code, it commutes with all Gauss law checks. Computing the symplectic inner product $\langle \bar{L}, A_v \rangle$ explicitly, the vertex contribution is zero (both components vanish on vertices) and the edge contribution reduces to $\partial(\delta)(v)$. Thus $\partial(\delta) = 0$, i.e., $\delta \in \ker(\partial)$.

By boundary exactness, $\delta \in \operatorname{im}(\partial_2)$, so $\operatorname{pureZEdgeOp}(\delta) \in \mathcal{S}^*$. But $\bar{L} = \operatorname{pureZEdgeOp}(\delta)$ is a logical and cannot be in $\mathcal{S}^*$. Contradiction.
\end{proof}

\begin{theorem}[Lifting Original Stabilizers to Deformed Code]
\label{thm:SpaceDistance.lift_originalStabilizer_to_deformed}
\lean{QEC1.SpaceDistance.lift_originalStabilizer_to_deformed}
\leanok
\uses{def:SpaceDistance.restrictToV, def:DeformedCodeChecks.deformedStabilizerCode, def:StabilizerCode.stabilizerGroup, def:DeformedCode.deformedOriginalChecks, thm:DeformedCodeChecks.deformed_mem_stabilizerGroup}
For any element $S$ of the original stabilizer group, there exists $\tilde{S}$ in the deformed stabilizer group such that $\operatorname{restrictToV}(\tilde{S}) = S$ and $\tilde{S}$ has no X-support on edges.
\end{theorem}

\begin{proof}
\leanok
\uses{def:SpaceDistance.restrictToV, def:DeformedCode.deformedOriginalChecks, thm:DeformedCodeChecks.deformed_mem_stabilizerGroup, thm:SpaceDistance.restrictToV_deformedOriginalCheck}
We proceed by induction on the structure of the stabilizer group (closure induction):
\begin{itemize}
\item \textbf{Generator case:} If $s = s_j$ is an original check, we take $\tilde{S} = \tilde{s}_j$ (the deformed original check). Then $\tilde{s}_j \in \mathcal{S}^*$, $\operatorname{restrictToV}(\tilde{s}_j) = s_j$, and $\tilde{s}_j$ has no X on edges (by the definition of $\operatorname{deformedOpExt}$).
\item \textbf{Identity:} Take $\tilde{S} = 1$, which trivially satisfies all conditions.
\item \textbf{Product:} Given $\tilde{S}_1$ and $\tilde{S}_2$ lifting $S_1$ and $S_2$ respectively, take $\tilde{S} = \tilde{S}_1 \cdot \tilde{S}_2$. The product is in $\mathcal{S}^*$ (subgroup closure), the restriction distributes over multiplication, and the no-X-on-edges condition is preserved under multiplication (since $0 + 0 = 0$ in $\mathbb{Z}/2\mathbb{Z}$).
\item \textbf{Inverse:} Given $\tilde{S}$ lifting $S$, take $\tilde{S}^{-1}$. In the $\mathbb{Z}/2\mathbb{Z}$ Pauli algebra, $P^{-1} = P$, so all conditions are inherited.
\end{itemize}
\end{proof}

\begin{theorem}[Cleaned Operator Cannot Be a Stabilizer on V]
\label{thm:SpaceDistance.cleaned_not_stabilizer_on_V}
\lean{QEC1.SpaceDistance.cleaned_not_stabilizer_on_V}
\leanok
\uses{def:SpaceDistance.restrictToV, def:SpaceDistance.cleanedOp, def:SpaceDistance.edgeXSupport, def:FreedomInDeformedChecks.pureZEdgeOp, def:DeformedCodeChecks.deformedStabilizerCode, def:StabilizerCode.isLogicalOp, def:StabilizerCode.stabilizerGroup, def:GraphMaps.coboundaryMap, def:GraphMaps.boundaryMap, def:GraphMaps.secondBoundaryMap}
If $L'$ is a logical of the deformed code, $\delta(c) = \operatorname{edgeXSupport}(L')$, and $\operatorname{restrictToV}(\operatorname{cleanedOp}(L', c)) \in \mathcal{S}$ (the original stabilizer group), then assuming boundary exactness, we reach a contradiction.
\end{theorem}

\begin{proof}
\leanok
\uses{lem:SpaceDistance.cleaned_has_no_xSupport_on_edges, thm:SpaceDistance.cleanedOp_isLogical, thm:SpaceDistance.lift_originalStabilizer_to_deformed, thm:SpaceDistance.pureZEdgeOp_mem_stabilizerGroup_of_range}
Let $\bar{L} = \operatorname{cleanedOp}(L', c)$, which is a logical of the deformed code with no X on edges. We lift $\operatorname{restrictToV}(\bar{L})$ to a deformed stabilizer $\tilde{S}$ with $\operatorname{restrictToV}(\tilde{S}) = \operatorname{restrictToV}(\bar{L})$ and no X on edges.

Consider $Q = \bar{L} \cdot \tilde{S}$. Since $\operatorname{restrictToV}(\bar{L}) = \operatorname{restrictToV}(\tilde{S})$ and Pauli operators are self-inverse, $\operatorname{restrictToV}(Q) = 1$. Also $Q$ has no X on edges.

We case-split:
\begin{itemize}
\item If $Q = 1$: then $\bar{L} = \tilde{S} \in \mathcal{S}^*$, contradicting $\bar{L}$ being a logical.
\item If $Q \in \mathcal{S}^*$: then $\bar{L} = Q \cdot \tilde{S} \in \mathcal{S}^*$, contradiction.
\item Otherwise: $Q$ is in the centralizer of the deformed code (since both $\bar{L}$ and $\tilde{S}$ commute with all checks), is not in $\mathcal{S}^*$, and is not $1$. So $Q$ is a logical of the deformed code. Since $Q$ has $\operatorname{restrictToV}(Q) = 1$ and no X on edges, $Q = \operatorname{pureZEdgeOp}(\delta_Q)$. By the same argument as in the identity case (commutation with Gauss law checks gives $\partial(\delta_Q) = 0$, boundary exactness gives $\delta_Q \in \operatorname{im}(\partial_2)$), $Q \in \mathcal{S}^*$. Contradiction.
\end{itemize}
\end{proof}

\begin{theorem}[Any Cleaning Gives a Logical of the Original Code]
\label{thm:SpaceDistance.any_cleaning_gives_logical}
\lean{QEC1.SpaceDistance.any_cleaning_gives_logical}
\leanok
\uses{def:SpaceDistance.restrictToV, def:SpaceDistance.cleanedOp, def:SpaceDistance.edgeXSupport, def:DeformedCodeChecks.deformedStabilizerCode, def:StabilizerCode.isLogicalOp, def:GaugingMeasurementCorrectness.logicalOpV, def:GraphMaps.coboundaryMap, def:GraphMaps.boundaryMap, def:GraphMaps.secondBoundaryMap}
Given boundary exactness ($\ker(\partial) \subseteq \operatorname{im}(\partial_2)$) and a logical $L'$ of the deformed code, for any $c$ with $\delta(c) = \operatorname{edgeXSupport}(L')$, the restriction $\operatorname{restrictToV}(\operatorname{cleanedOp}(L', c))$ is a logical of the original code.
\end{theorem}

\begin{proof}
\leanok
\uses{lem:SpaceDistance.cleaned_has_no_xSupport_on_edges, lem:SpaceDistance.cleaned_commutes_with_original_check, thm:SpaceDistance.cleanedOp_isLogical, thm:SpaceDistance.cleaned_not_identity_on_V, thm:SpaceDistance.cleaned_not_stabilizer_on_V}
Let $\bar{L} = \operatorname{cleanedOp}(L', c)$, which is a logical of the deformed code with no X on edges. Since $\bar{L}$ commutes with all deformed checks and has no X on edges, its restriction $\operatorname{restrictToV}(\bar{L})$ is in the centralizer of the original code.

Suppose for contradiction that $\operatorname{restrictToV}(\bar{L})$ is not a logical. Being in the centralizer, it must be either in the original stabilizer group or equal to $1$:
\begin{itemize}
\item If it is in the stabilizer group: this contradicts the cleaned-not-stabilizer-on-V theorem.
\item If it equals $1$: this contradicts the cleaned-not-identity-on-V theorem.
\end{itemize}
Both cases lead to contradiction, so $\operatorname{restrictToV}(\bar{L})$ must be a logical of the original code.
\end{proof}

\begin{theorem}[Construction of WLOG Cleaning Vector]
\label{thm:SpaceDistance.construct_wlog}
\lean{QEC1.SpaceDistance.construct_wlog}
\leanok
\uses{def:SpaceDistance.restrictToV, def:SpaceDistance.cleanedOp, def:SpaceDistance.edgeXSupport, def:DeformedCodeChecks.deformedStabilizerCode, def:StabilizerCode.isLogicalOp, def:GaugingMeasurementCorrectness.logicalOpV, def:GraphMaps.coboundaryMap, def:GraphMaps.secondCoboundaryMap, def:GraphMaps.boundaryMap, def:GraphMaps.secondBoundaryMap}
Given exactness of both chain sequences and a logical $L'$ of the deformed code, there exists a cleaning vector $c : V \to \mathbb{Z}/2\mathbb{Z}$ such that:
\begin{enumerate}
\item $\delta(c) = \operatorname{edgeXSupport}(L')$,
\item $\operatorname{restrictToV}(\operatorname{cleanedOp}(L', c))$ is a logical of the original code, and
\item $2 \cdot |\operatorname{supp}(c)| \leq |V|$.
\end{enumerate}
\end{theorem}

\begin{proof}
\leanok
\uses{thm:SpaceDistance.edgeXSupport_in_ker_secondCoboundary, thm:SpaceDistance.any_cleaning_gives_logical, lem:SpaceDistance.support_add_complement_card, lem:SpaceDistance.coboundary_add_ones_eq}
By Step~2, $\operatorname{edgeXSupport}(L') \in \ker(\delta_2)$. By exactness of the first sequence ($\ker(\delta_2) \subseteq \operatorname{im}(\delta)$), there exists $c_0$ with $\delta(c_0) = \operatorname{edgeXSupport}(L')$. By the any-cleaning-gives-logical theorem, $\operatorname{restrictToV}(\operatorname{cleanedOp}(L', c_0))$ is a logical.

If $2 \cdot |\operatorname{supp}(c_0)| \leq |V|$, take $c = c_0$. Otherwise, let $c_1 = c_0 + \mathbf{1}$. Then $\delta(c_1) = \delta(c_0)$ (since $\delta(\mathbf{1}) = 0$), and any cleaning gives a logical. The support partition $|\operatorname{supp}(c_0)| + |\operatorname{supp}(c_1)| = |V|$ together with $2 \cdot |\operatorname{supp}(c_0)| > |V|$ gives $2 \cdot |\operatorname{supp}(c_1)| \leq |V|$. Take $c = c_1$.
\end{proof}

\begin{lemma}[Core Arithmetic for Distance Bound]
\label{lem:SpaceDistance.weight_lower_bound_arithmetic}
\lean{QEC1.SpaceDistance.weight_lower_bound_arithmetic}
\leanok

Let $w, d, |S|, |\partial| \in \mathbb{N}$ and $h \geq 0$ a real number. If:
\begin{enumerate}
\item $w + |S| \geq d + |\partial|$,
\item $h \cdot |S| \leq |\partial|$, and
\item $|\partial| \leq w$,
\end{enumerate}
then $w \geq \min(h, 1) \cdot d$.
\end{lemma}

\begin{proof}
\leanok

We case-split on whether $h \geq 1$.

If $h \geq 1$: then $\min(h, 1) = 1$. From condition (2), $|S| \leq h \cdot |S| \leq |\partial|$. Combined with conditions (1) and (3), $w + |S| \geq d + |\partial|$ and $|\partial| \leq w$, so $d \leq w$.

If $h < 1$: then $\min(h, 1) = h$. We sub-case on $|S| \leq d$ vs.\ $|S| > d$.
\begin{itemize}
\item If $|S| \leq d$: From condition (1), $w \geq d + |\partial| - |S| \geq d + h \cdot |S| - |S| = d + (h-1) \cdot |S|$. Since $h - 1 < 0$ and $|S| \leq d$, $(h-1) \cdot |S| \geq (h-1) \cdot d$, so $w \geq d + (h-1) \cdot d = h \cdot d$.
\item If $|S| > d$: From condition (3), $w \geq |\partial| \geq h \cdot |S| > h \cdot d$.
\end{itemize}
\end{proof}

\begin{lemma}[Vertex Weight Plus Support Bounds Cleaned Weight]
\label{lem:SpaceDistance.vertex_weight_plus_c_ge_cleaned_weight}
\lean{QEC1.SpaceDistance.vertex_weight_plus_c_ge_cleaned_weight}
\leanok
\uses{def:SpaceDistance.restrictToV, def:SpaceDistance.cleanedOp, def:PauliOp.weight}
For any $L'$ and $c$:
\[
\operatorname{wt}(\operatorname{restrictToV}(L')) + |\operatorname{supp}(c)| \geq \operatorname{wt}(\operatorname{restrictToV}(\operatorname{cleanedOp}(L', c))).
\]
\end{lemma}

\begin{proof}
\leanok
\uses{def:SpaceDistance.restrictToV, def:SpaceDistance.cleanedOp, def:PauliOp.weight, def:PauliOp.support}
We show that the support of the cleaned restriction is contained in the union of the support of $\operatorname{restrictToV}(L')$ and the support of $c$, then apply monotonicity and the union cardinality bound.

For any vertex $v$ in the support of $\operatorname{restrictToV}(\operatorname{cleanedOp}(L', c))$: either its $\mathbf{x}$-component is nonzero or its $\mathbf{z}$-component is nonzero. If the $\mathbf{x}$-component is nonzero and $c(v) = 0$, the cleaning does not affect it, so $v$ is in the support of $\operatorname{restrictToV}(L')$. If $c(v) \neq 0$, then $v$ is in the support of $c$. If the $\mathbf{z}$-component is nonzero, since cleaning preserves $\mathbf{z}$-vectors, $v$ is in the support of $\operatorname{restrictToV}(L')$.
\end{proof}

\begin{theorem}[Lemma 3: Per-Operator Distance Bound]
\label{thm:SpaceDistance.distance_ge_min_cheeger_one_mul_d}
\lean{QEC1.SpaceDistance.distance_ge_min_cheeger_one_mul_d}
\leanok
\uses{def:SpaceDistance.restrictToV, def:SpaceDistance.cleanedOp, def:SpaceDistance.edgeXSupport, def:DeformedCodeChecks.deformedStabilizerCode, def:StabilizerCode.isLogicalOp, def:StabilizerCode.distance, def:GaugingMeasurementCorrectness.logicalOpV, def:PauliOp.weight, def:cheegerConstant, def:GraphMaps.coboundaryMap, def:GraphMaps.secondCoboundaryMap, def:GraphMaps.boundaryMap, def:GraphMaps.secondBoundaryMap}
Let $G$ be the graph used in gauging, $L'$ a logical operator of the deformed code, and $d$ the distance of the original code. Assuming connectivity, $|V| \geq 2$, exactness of both chain sequences, and that $\operatorname{logicalOpV}$ is a logical of the original code:
\[
\operatorname{wt}(L') \geq \min(h(G), 1) \cdot d.
\]
\end{theorem}

\begin{proof}
\leanok
\uses{thm:SpaceDistance.construct_wlog, lem:SpaceDistance.weight_decomposition, lem:SpaceDistance.vertex_weight_plus_c_ge_cleaned_weight, lem:SpaceDistance.weight_lower_bound_arithmetic, thm:SimpleGraph.cheegerConstant_nonneg, thm:SimpleGraph.edgeBoundary_card_ge_of_cheeger, thm:SpaceDistance.coboundary_support_card_eq_edgeBoundary_card, lem:SpaceDistance.edge_support_ge_coboundary_support}
By the WLOG construction, we obtain $c$ with $\delta(c) = \operatorname{edgeXSupport}(L')$, such that $\operatorname{restrictToV}(\operatorname{cleanedOp}(L', c))$ is a logical of the original code and $2|\operatorname{supp}(c)| \leq |V|$.

Let $S = |\operatorname{supp}(c)|$ and $E_{\text{supp}} = |\{e \mid L'.\mathbf{x}(\operatorname{inr}(e)) \neq 0 \lor L'.\mathbf{z}(\operatorname{inr}(e)) \neq 0\}|$.

We verify the hypotheses of the core arithmetic lemma:
\begin{enumerate}
\item $\operatorname{wt}(L') + S \geq d + E_{\text{supp}}$: The cleaned restriction is a logical, so its weight $\geq d$. By the vertex-weight-plus-support bound, $\operatorname{wt}(\operatorname{restrictToV}(L')) + S \geq \operatorname{wt}(\text{cleaned restriction}) \geq d$. By the weight decomposition, $\operatorname{wt}(L') = \operatorname{wt}(\operatorname{restrictToV}(L')) + E_{\text{supp}}$.
\item $h(G) \cdot S \leq E_{\text{supp}}$: If $c = 0$ then $S = 0$ and this holds trivially. Otherwise, by the Cheeger constant definition, $h(G) \cdot S \leq |\partial_E(\operatorname{supp}(c))|$, which equals the coboundary support size (by the coboundary-equals-edge-boundary theorem), which is $\leq E_{\text{supp}}$.
\item $E_{\text{supp}} \leq \operatorname{wt}(L')$: This follows from the weight decomposition.
\end{enumerate}

Applying the arithmetic lemma yields $\operatorname{wt}(L') \geq \min(h(G), 1) \cdot d$.
\end{proof}

\begin{theorem}[Sufficient Expansion: Per-Operator Distance Preservation]
\label{thm:SpaceDistance.distance_ge_d_of_sufficient_expansion}
\lean{QEC1.SpaceDistance.distance_ge_d_of_sufficient_expansion}
\leanok
\uses{def:DeformedCodeChecks.deformedStabilizerCode, def:StabilizerCode.isLogicalOp, def:StabilizerCode.distance, def:DesiderataForGraphG.SufficientExpansion, def:GaugingMeasurementCorrectness.logicalOpV, def:PauliOp.weight, def:cheegerConstant}
If $h(G) \geq 1$ (sufficient expansion), then for every logical operator $L'$ of the deformed code:
\[
d \leq \operatorname{wt}(L').
\]
\end{theorem}

\begin{proof}
\leanok
\uses{thm:SpaceDistance.distance_ge_min_cheeger_one_mul_d, def:DesiderataForGraphG.SufficientExpansion}
By the per-operator bound, $\operatorname{wt}(L') \geq \min(h(G), 1) \cdot d$. Since $h(G) \geq 1$, $\min(h(G), 1) = 1$, so $\operatorname{wt}(L') \geq 1 \cdot d = d$.
\end{proof}

\begin{theorem}[Deformed Code Distance Bound]
\label{thm:SpaceDistance.deformed_distance_ge_min_cheeger_d}
\lean{QEC1.SpaceDistance.deformed_distance_ge_min_cheeger_d}
\leanok
\uses{def:DeformedCodeChecks.deformedStabilizerCode, def:StabilizerCode.isLogicalOp, def:StabilizerCode.distance, def:GaugingMeasurementCorrectness.logicalOpV, def:PauliOp.weight, def:cheegerConstant}
Assuming the deformed code has at least one logical operator:
\[
d^* \geq \min(h(G), 1) \cdot d,
\]
where $d^*$ denotes the distance of the deformed code and $d$ the distance of the original code.
\end{theorem}

\begin{proof}
\leanok
\uses{thm:SpaceDistance.distance_ge_min_cheeger_one_mul_d, thm:SimpleGraph.cheegerConstant_nonneg}
The deformed code distance is $d^* = \inf\{\operatorname{wt}(P) \mid P \text{ is a logical of the deformed code}\}$. The set of weights is nonempty by hypothesis. For every weight $w$ in this set with witnessing logical $P$, the per-operator bound gives $w = \operatorname{wt}(P) \geq \min(h(G), 1) \cdot d$.

Taking the ceiling of $\min(h(G), 1) \cdot d$ and using the property that $x \leq \lceil x \rceil$ for reals, together with $\lceil x \rceil \leq w$ for all $w$ in the set, we obtain $\lceil \min(h(G), 1) \cdot d \rceil \leq \inf\{w\} = d^*$. Thus $\min(h(G), 1) \cdot d \leq d^*$.
\end{proof}

\begin{theorem}[Sufficient Expansion Preserves Distance]
\label{thm:SpaceDistance.deformed_distance_ge_d_sufficient_expansion}
\lean{QEC1.SpaceDistance.deformed_distance_ge_d_sufficient_expansion}
\leanok
\uses{def:DeformedCodeChecks.deformedStabilizerCode, def:StabilizerCode.isLogicalOp, def:StabilizerCode.distance, def:DesiderataForGraphG.SufficientExpansion, def:GaugingMeasurementCorrectness.logicalOpV, def:cheegerConstant}
If $h(G) \geq 1$ (sufficient expansion) and the deformed code has at least one logical operator, then
\[
d \leq d^*.
\]
\end{theorem}

\begin{proof}
\leanok
\uses{thm:SpaceDistance.deformed_distance_ge_min_cheeger_d, def:DesiderataForGraphG.SufficientExpansion}
By the deformed code distance bound, $\min(h(G), 1) \cdot d \leq d^*$. Since $h(G) \geq 1$, $\min(h(G), 1) = 1$, so $1 \cdot d = d \leq d^*$.
\end{proof}

%--- Rem_13: OptimalCheegerConstant ---
\chapter{Rem 13: Optimal Cheeger Constant}

This remark establishes that $h(G) = 1$ is the optimal Cheeger constant for distance preservation in the deformed code construction. Three points are shown: (1) $h(G) \geq 1$ is sufficient for $d^* \geq d$; (2) $h(G) > 1$ does not help further since $d^* \leq d$, giving $d^* = d$; (3) $h(G) < 1$ causes distance loss with $d^* \geq h(G) \cdot d < d$.

\section*{Lifting original operators to the extended system}

\begin{definition}[Lift to Extended System]
\label{def:OptimalCheegerConstant.liftToExtended}
\lean{QEC1.OptimalCheegerConstant.liftToExtended}
\leanok
\uses{def:DeformedOperator.deformedOpExt}
Given a Pauli operator $P$ on $V$, define its \emph{lift} to the extended qubit space $V \oplus E(G)$ by
\[
\operatorname{liftToExtended}(P) := \operatorname{deformedOpExt}(P, 0),
\]
i.e., $P$ acts on vertex qubits as before and acts as the identity on all edge qubits.
\end{definition}

\begin{lemma}[Lifted operator has same weight]
\label{lem:OptimalCheegerConstant.liftToExtended_weight}
\lean{QEC1.OptimalCheegerConstant.liftToExtended_weight}
\leanok
\uses{def:OptimalCheegerConstant.liftToExtended, def:PauliOp.weight}
For any Pauli operator $P$ on $V$,
\[
\operatorname{weight}(\operatorname{liftToExtended}(P)) = \operatorname{weight}(P).
\]
\end{lemma}
\begin{proof}
\leanok
\uses{lem:SpaceDistance.weight_eq_vertex_plus_edge, def:OptimalCheegerConstant.liftToExtended}
We use the vertex--edge weight decomposition. The edge contribution is zero: for every edge qubit $e \in E(G)$, the lifted operator satisfies $\operatorname{xVec}(\operatorname{liftToExtended}(P))(\operatorname{inr}(e)) = 0$ and $\operatorname{zVec}(\operatorname{liftToExtended}(P))(\operatorname{inr}(e)) = 0$, so the filter over edges with non-trivial support is empty and has cardinality $0$. Adding zero to the vertex weight, the vertex contribution equals $\operatorname{weight}(P)$ by unfolding the definitions of weight and support and applying congruence.
\end{proof}

\begin{lemma}[Restriction of lift recovers original]
\label{lem:OptimalCheegerConstant.restrictToV_liftToExtended}
\lean{QEC1.OptimalCheegerConstant.restrictToV_liftToExtended}
\leanok
\uses{def:SpaceDistance.restrictToV, def:OptimalCheegerConstant.liftToExtended}
For any Pauli operator $P$ on $V$,
\[
\operatorname{restrictToV}(\operatorname{liftToExtended}(P)) = P.
\]
\end{lemma}
\begin{proof}
\leanok
\uses{def:SpaceDistance.restrictToV, def:OptimalCheegerConstant.liftToExtended}
By extensionality on each vertex $v$, both the $x$-vector and $z$-vector components agree by simplification using the definitions of $\operatorname{restrictToV}$, $\operatorname{liftToExtended}$, and $\operatorname{deformedOpExt}$.
\end{proof}

\begin{lemma}[Lift preserves multiplication]
\label{lem:OptimalCheegerConstant.liftToExtended_mul}
\lean{QEC1.OptimalCheegerConstant.liftToExtended_mul}
\leanok
\uses{def:OptimalCheegerConstant.liftToExtended}
For any Pauli operators $P, Q$ on $V$,
\[
\operatorname{liftToExtended}(P \cdot Q) = \operatorname{liftToExtended}(P) \cdot \operatorname{liftToExtended}(Q).
\]
\end{lemma}
\begin{proof}
\leanok
\uses{def:OptimalCheegerConstant.liftToExtended}
By extensionality on each qubit $q \in V \oplus E(G)$, case-splitting on whether $q$ is a vertex or edge qubit, then simplifying using the definitions of $\operatorname{liftToExtended}$, $\operatorname{deformedOpExt}$, and the multiplication formulas for $x$-vectors and $z$-vectors.
\end{proof}

\begin{lemma}[Lift sends identity to identity]
\label{lem:OptimalCheegerConstant.liftToExtended_one}
\lean{QEC1.OptimalCheegerConstant.liftToExtended_one}
\leanok
\uses{def:OptimalCheegerConstant.liftToExtended}
We have $\operatorname{liftToExtended}(\mathbf{1}) = \mathbf{1}$.
\end{lemma}
\begin{proof}
\leanok
\uses{def:OptimalCheegerConstant.liftToExtended}
By extensionality on each qubit $q$, case-splitting on vertex or edge, and simplifying using the definitions and the formulas for $\operatorname{xVec}$ and $\operatorname{zVec}$ of the identity.
\end{proof}

\begin{lemma}[Lift is injective]
\label{lem:OptimalCheegerConstant.liftToExtended_ne_one}
\lean{QEC1.OptimalCheegerConstant.liftToExtended_ne_one}
\leanok
\uses{def:OptimalCheegerConstant.liftToExtended, lem:OptimalCheegerConstant.restrictToV_liftToExtended}
If $P \neq \mathbf{1}$, then $\operatorname{liftToExtended}(P) \neq \mathbf{1}$.
\end{lemma}
\begin{proof}
\leanok
\uses{lem:OptimalCheegerConstant.restrictToV_liftToExtended}
Suppose for contradiction that $\operatorname{liftToExtended}(P) = \mathbf{1}$. Applying $\operatorname{restrictToV}$ to both sides and using the fact that $\operatorname{restrictToV}(\operatorname{liftToExtended}(P)) = P$ and $\operatorname{restrictToV}(\mathbf{1}) = \mathbf{1}$, we obtain $P = \mathbf{1}$, contradicting $P \neq \mathbf{1}$.
\end{proof}

\section*{Commutation of lifted operators with deformed code checks}

\begin{lemma}[Lift commutes with flux checks]
\label{lem:OptimalCheegerConstant.liftToExtended_comm_flux}
\lean{QEC1.OptimalCheegerConstant.liftToExtended_comm_flux}
\leanok
\uses{def:OptimalCheegerConstant.liftToExtended, def:DeformedCode.fluxChecks, def:PauliOp.PauliCommute, def:PauliOp.symplecticInner}
For any Pauli operator $P$ on $V$ and any cycle index $p$, the lifted operator commutes with the flux check:
\[
[\operatorname{liftToExtended}(P),\, \operatorname{fluxChecks}(p)] = 0.
\]
\end{lemma}
\begin{proof}
\leanok
\uses{def:OptimalCheegerConstant.liftToExtended, def:DeformedCode.fluxChecks, def:GaussFlux.fluxOp}
We unfold $\operatorname{PauliCommute}$ and $\operatorname{symplecticInner}$, then decompose the sum over the type $V \oplus E(G)$ into vertex and edge parts. For the vertex sum: since flux checks are pure $Z$ on edges and have trivial $x$-vector on vertices, each summand $\operatorname{xVec}(\operatorname{lift}(P))(\operatorname{inl}(v)) \cdot \operatorname{zVec}(\operatorname{flux}(p))(\operatorname{inl}(v)) + \operatorname{zVec}(\operatorname{lift}(P))(\operatorname{inl}(v)) \cdot \operatorname{xVec}(\operatorname{flux}(p))(\operatorname{inl}(v))$ equals $0$ by simplification. For the edge sum: since the lift has zero support on edges, each summand equals $0$. Adding the two zero contributions gives $0$.
\end{proof}

\begin{lemma}[Lift commutes with deformed original checks]
\label{lem:OptimalCheegerConstant.liftToExtended_comm_deformedCheck}
\lean{QEC1.OptimalCheegerConstant.liftToExtended_comm_deformedCheck}
\leanok
\uses{def:OptimalCheegerConstant.liftToExtended, def:DeformedCode.deformedOriginalChecks, def:PauliOp.PauliCommute, lem:SpaceDistance.symplecticInner_noXEdge_deformed, lem:OptimalCheegerConstant.restrictToV_liftToExtended}
Let $P$ be a Pauli operator on $V$ and $j$ a check index such that $P$ commutes with the original check $\operatorname{checks}(j)$. Then $\operatorname{liftToExtended}(P)$ commutes with the deformed original check $\operatorname{deformedOriginalChecks}(j)$.
\end{lemma}
\begin{proof}
\leanok
\uses{lem:SpaceDistance.symplecticInner_noXEdge_deformed, lem:OptimalCheegerConstant.restrictToV_liftToExtended}
We rewrite the commutation condition as $\operatorname{symplecticInner}(\operatorname{liftToExtended}(P), \operatorname{deformedOriginalChecks}(j)) = 0$. Since the lift has no $X$-support on edges, we apply the lemma that symplectic inner products with deformed checks reduce to the restriction on $V$. Rewriting using $\operatorname{restrictToV}(\operatorname{liftToExtended}(P)) = P$, the result follows from the hypothesis that $P$ commutes with $\operatorname{checks}(j)$.
\end{proof}

\begin{lemma}[Symplectic inner product with Gauss law]
\label{lem:OptimalCheegerConstant.liftToExtended_symplecticInner_gaussLaw}
\lean{QEC1.OptimalCheegerConstant.liftToExtended_symplecticInner_gaussLaw}
\leanok
\uses{def:OptimalCheegerConstant.liftToExtended, def:GaussFlux.gaussLawOp, def:PauliOp.symplecticInner}
For any Pauli operator $P$ on $V$ and vertex $v \in V$,
\[
\langle \operatorname{liftToExtended}(P),\, A_v \rangle_{\mathrm{symp}} = P.\operatorname{zVec}(v).
\]
\end{lemma}
\begin{proof}
\leanok
\uses{def:OptimalCheegerConstant.liftToExtended, def:GaussFlux.gaussLawOp}
We unfold $\operatorname{symplecticInner}$ and decompose the sum over $V \oplus E(G)$. For the vertex sum: since $A_v$ has $\operatorname{zVec} = 0$ on vertices and $\operatorname{xVec}(\operatorname{inl}(w)) = \delta_{v,w}$, we simplify each term and use single-element summation to extract the $v$-th term, obtaining $P.\operatorname{zVec}(v)$. For the edge sum: since the lift has zero components on edges, each summand is zero. Adding the vertex sum and the zero edge sum gives the result.
\end{proof}

\begin{lemma}[Lift commutes with Gauss law iff no $Z$-support]
\label{lem:OptimalCheegerConstant.liftToExtended_comm_gaussLaw_iff}
\lean{QEC1.OptimalCheegerConstant.liftToExtended_comm_gaussLaw_iff}
\leanok
\uses{def:OptimalCheegerConstant.liftToExtended, def:GaussFlux.gaussLawOp, def:PauliOp.PauliCommute, lem:OptimalCheegerConstant.liftToExtended_symplecticInner_gaussLaw}
For any Pauli operator $P$ on $V$ and vertex $v$,
\[
\operatorname{PauliCommute}(\operatorname{liftToExtended}(P), A_v) \iff P.\operatorname{zVec}(v) = 0.
\]
\end{lemma}
\begin{proof}
\leanok
\uses{lem:OptimalCheegerConstant.liftToExtended_symplecticInner_gaussLaw}
Unfolding $\operatorname{PauliCommute}$, the condition becomes $\langle \operatorname{liftToExtended}(P), A_v \rangle_{\mathrm{symp}} = 0$. By the previous lemma, this equals $P.\operatorname{zVec}(v) = 0$.
\end{proof}

\begin{lemma}[Pure-$X$ lift commutes with all Gauss law checks]
\label{lem:OptimalCheegerConstant.liftToExtended_comm_gaussLaw_of_pureX}
\lean{QEC1.OptimalCheegerConstant.liftToExtended_comm_gaussLaw_of_pureX}
\leanok
\uses{def:OptimalCheegerConstant.liftToExtended, def:GaussFlux.gaussLawOp, lem:OptimalCheegerConstant.liftToExtended_comm_gaussLaw_iff}
If $P$ is a pure-$X$ operator (i.e., $P.\operatorname{zVec} = 0$), then $\operatorname{liftToExtended}(P)$ commutes with $A_v$ for every vertex $v \in V$.
\end{lemma}
\begin{proof}
\leanok
\uses{lem:OptimalCheegerConstant.liftToExtended_comm_gaussLaw_iff}
By the commutation criterion, it suffices to show $P.\operatorname{zVec}(v) = 0$, which follows from evaluating the hypothesis $P.\operatorname{zVec} = 0$ at $v$.
\end{proof}

\section*{Pure-$X$ logicals lift to deformed code centralizer}

\begin{theorem}[Pure-$X$ centralizer element lifts to deformed centralizer]
\label{thm:OptimalCheegerConstant.liftToExtended_inCentralizer_of_pureX}
\lean{QEC1.OptimalCheegerConstant.liftToExtended_inCentralizer_of_pureX}
\leanok
\uses{def:OptimalCheegerConstant.liftToExtended, def:DeformedCodeChecks.deformedStabilizerCode, def:StabilizerCode.inCentralizer, lem:OptimalCheegerConstant.liftToExtended_comm_gaussLaw_of_pureX, lem:OptimalCheegerConstant.liftToExtended_comm_flux, lem:OptimalCheegerConstant.liftToExtended_comm_deformedCheck}
Let $P$ be a pure-$X$ Pauli operator on $V$ that commutes with all original checks. Then $\operatorname{liftToExtended}(P)$ lies in the centralizer of the deformed stabilizer code.
\end{theorem}
\begin{proof}
\leanok
\uses{lem:OptimalCheegerConstant.liftToExtended_comm_gaussLaw_of_pureX, lem:OptimalCheegerConstant.liftToExtended_comm_flux, lem:OptimalCheegerConstant.liftToExtended_comm_deformedCheck}
Let $a$ be an arbitrary check of the deformed code.We simplify using the structure of the deformed stabilizer code and case-split on the type of check $a$:
\begin{itemize}
\item \textbf{Gauss law check} ($a = \operatorname{gaussLaw}(v)$): commutation follows from the pure-$X$ Gauss law commutation lemma.
\item \textbf{Flux check} ($a = \operatorname{flux}(p)$): commutation follows from the flux commutation lemma.
\item \textbf{Deformed original check} ($a = \operatorname{deformed}(j)$): commutation follows from the deformed check commutation lemma, using the hypothesis that $P$ commutes with the original check $\operatorname{checks}(j)$.
\end{itemize}
\end{proof}

\section*{Point 1: $h(G) \geq 1$ is sufficient for $d^* \geq d$}

\begin{theorem}[Sufficient expansion gives $d^* \geq d$]
\label{thm:OptimalCheegerConstant.sufficient_expansion_gives_dstar_ge_d}
\lean{QEC1.OptimalCheegerConstant.sufficient_expansion_gives_dstar_ge_d}
\leanok
\uses{def:StabilizerCode, def:StabilizerCode.distance, def:DeformedCodeChecks.deformedStabilizerCode, def:DesiderataForGraphG.SufficientExpansion, thm:SpaceDistance.deformed_distance_ge_d_sufficient_expansion}
Let $C$ be a stabilizer code with checks $\{\operatorname{checks}(j)\}_{j \in J}$, and let $G$ be a connected graph on $V$ with $|V| \geq 2$ satisfying the sufficient expansion condition $h(G) \geq 1$. Assume the exactness conditions on the chain complex and that the original code has a logical operator. Then the deformed code distance satisfies $d^* \geq d$.
\end{theorem}
\begin{proof}
\leanok
\uses{thm:SpaceDistance.deformed_distance_ge_d_sufficient_expansion}
This is a direct application of the theorem $\texttt{deformed\_distance\_ge\_d\_sufficient\_expansion}$.
\end{proof}

\section*{Point 2: $d^* \leq d$ via lifting}

\begin{lemma}[Deformed distance is at most the weight of any logical]
\label{lem:OptimalCheegerConstant.deformed_distance_le_weight}
\lean{QEC1.OptimalCheegerConstant.deformed_distance_le_weight}
\leanok
\uses{def:DeformedCodeChecks.deformedStabilizerCode, def:StabilizerCode.distance, def:StabilizerCode.isLogicalOp, def:PauliOp.weight}
For any logical operator $L'$ of the deformed code,
\[
d^* \leq \operatorname{weight}(L').
\]
\end{lemma}
\begin{proof}
\leanok
\uses{def:StabilizerCode.distance}
Unfolding the definition of distance as an infimum, we apply $\operatorname{Nat.sInf\_le}$ using the witness $(L', \text{hlog}, \operatorname{rfl})$.
\end{proof}

\begin{theorem}[Lifted logical is a deformed code logical]
\label{thm:OptimalCheegerConstant.liftToExtended_isLogical}
\lean{QEC1.OptimalCheegerConstant.liftToExtended_isLogical}
\leanok
\uses{def:OptimalCheegerConstant.liftToExtended, def:StabilizerCode.isLogicalOp, def:DeformedCodeChecks.deformedStabilizerCode, thm:OptimalCheegerConstant.liftToExtended_inCentralizer_of_pureX, lem:OptimalCheegerConstant.liftToExtended_ne_one}
Let $P$ be a pure-$X$ logical operator of the original code whose lift is not in the deformed stabilizer group. Then $\operatorname{liftToExtended}(P)$ is a logical operator of the deformed code.
\end{theorem}
\begin{proof}
\leanok
\uses{thm:OptimalCheegerConstant.liftToExtended_inCentralizer_of_pureX, lem:OptimalCheegerConstant.liftToExtended_ne_one}
From the hypothesis that $P$ is a logical operator, we extract that $P$ is in the centralizer of the original code, $P$ is not in the stabilizer group, and $P \neq \mathbf{1}$. We verify the three conditions for $\operatorname{liftToExtended}(P)$ being a logical of the deformed code:
\begin{enumerate}
\item \emph{Centralizer membership:} Since $P$ is pure-$X$ and commutes with all original checks, we apply the centralizer lifting theorem.
\item \emph{Not in deformed stabilizer:} This is given by hypothesis.
\item \emph{Non-identity:} Since $P \neq \mathbf{1}$, the lift is non-identity by injectivity of the lift.
\end{enumerate}
\end{proof}

\begin{theorem}[Point 2: $d^* \leq d$]
\label{thm:OptimalCheegerConstant.deformed_distance_le_original}
\lean{QEC1.OptimalCheegerConstant.deformed_distance_le_original}
\leanok
\uses{def:StabilizerCode.distance, def:DeformedCodeChecks.deformedStabilizerCode, def:OptimalCheegerConstant.liftToExtended, thm:OptimalCheegerConstant.liftToExtended_isLogical, lem:OptimalCheegerConstant.deformed_distance_le_weight, lem:OptimalCheegerConstant.liftToExtended_weight}
If there exists a pure-$X$ logical $P$ of the original code with $\operatorname{weight}(P) = d$ whose lift is not a deformed stabilizer, then $d^* \leq d$.
\end{theorem}
\begin{proof}
\leanok
\uses{thm:OptimalCheegerConstant.liftToExtended_isLogical, lem:OptimalCheegerConstant.deformed_distance_le_weight, lem:OptimalCheegerConstant.liftToExtended_weight}
We first establish that $\operatorname{liftToExtended}(P)$ is a logical of the deformed code. Then:
\[
d^* \leq \operatorname{weight}(\operatorname{liftToExtended}(P)) = \operatorname{weight}(P) = d,
\]
where the first inequality uses the deformed distance bound, the first equality uses weight preservation of the lift, and the second equality uses the hypothesis $\operatorname{weight}(P) = d$.
\end{proof}

\section*{Point 2 combined with Point 1: $d^* = d$ when $h(G) \geq 1$}

\begin{theorem}[$d^* = d$ when $h(G) \geq 1$]
\label{thm:OptimalCheegerConstant.deformed_distance_eq}
\lean{QEC1.OptimalCheegerConstant.deformed_distance_eq}
\leanok
\uses{def:StabilizerCode.distance, def:DeformedCodeChecks.deformedStabilizerCode, def:DesiderataForGraphG.SufficientExpansion, thm:OptimalCheegerConstant.deformed_distance_le_original, thm:OptimalCheegerConstant.sufficient_expansion_gives_dstar_ge_d}
Under the hypotheses of both Point 1 and Point 2 (sufficient expansion $h(G) \geq 1$, existence of a pure-$X$ minimum-weight logical whose lift is not a deformed stabilizer), the deformed code distance equals the original code distance:
\[
d^* = d.
\]
\end{theorem}
\begin{proof}
\leanok
\uses{thm:OptimalCheegerConstant.deformed_distance_le_original, thm:OptimalCheegerConstant.sufficient_expansion_gives_dstar_ge_d}
We apply antisymmetry of $\leq$. The inequality $d^* \leq d$ follows from Point~2 (deformed distance at most original). The inequality $d \leq d^*$ follows from Point~1 (sufficient expansion gives $d^* \geq d$).
\end{proof}

\section*{Point 3: $h(G) < 1$ causes distance loss}

\begin{theorem}[Distance loss when $h(G) < 1$]
\label{thm:OptimalCheegerConstant.distance_loss_when_cheeger_lt_one}
\lean{QEC1.OptimalCheegerConstant.distance_loss_when_cheeger_lt_one}
\leanok
\uses{def:StabilizerCode.distance, def:DeformedCodeChecks.deformedStabilizerCode, def:cheegerConstant, thm:SpaceDistance.deformed_distance_ge_min_cheeger_d}
If $h(G) < 1$, then the deformed code distance satisfies
\[
h(G) \cdot d \leq d^*.
\]
Since $h(G) < 1$, this lower bound is strictly less than $d$, representing distance loss.
\end{theorem}
\begin{proof}
\leanok
\uses{thm:SpaceDistance.deformed_distance_ge_min_cheeger_d}
By the general bound from Lemma~3, $d^* \geq \min(h(G), 1) \cdot d$. Since $h(G) < 1$, we have $\min(h(G), 1) = h(G)$, so rewriting gives $d^* \geq h(G) \cdot d$. The result follows by casting to the appropriate type.
\end{proof}

\begin{lemma}[$h(G) < 1$ bound is strictly less than $d$]
\label{lem:OptimalCheegerConstant.cheeger_lt_one_bound_lt_d}
\lean{QEC1.OptimalCheegerConstant.cheeger_lt_one_bound_lt_d}
\leanok
\uses{def:StabilizerCode.distance, def:cheegerConstant}
If $d > 0$, $h(G) > 0$, and $h(G) < 1$, then
\[
h(G) \cdot d < d.
\]
\end{lemma}
\begin{proof}
\leanok
\uses{def:cheegerConstant}
We compute:
\[
h(G) \cdot d < 1 \cdot d = d,
\]
where the strict inequality follows from $h(G) < 1$ and $d > 0$ (using multiplication by a positive right factor), and the equality is by the identity $1 \cdot d = d$.
\end{proof}

\section*{Optimality of $h(G) = 1$}

\begin{lemma}[Bound saturates at $h(G) = 1$]
\label{lem:OptimalCheegerConstant.bound_saturates_at_cheeger_one}
\lean{QEC1.OptimalCheegerConstant.bound_saturates_at_cheeger_one}
\leanok
\uses{def:cheegerConstant}
If $h(G) \geq 1$, then for any $d \in \mathbb{N}$,
\[
\min(h(G), 1) \cdot d = d.
\]
\end{lemma}
\begin{proof}
\leanok
\uses{def:cheegerConstant}
Since $1 \leq h(G)$, we have $\min(h(G), 1) = 1$, so the expression simplifies to $1 \cdot d = d$.
\end{proof}

\begin{theorem}[$h(G) > 1$ does not help further]
\label{thm:OptimalCheegerConstant.cheeger_gt_one_bound_eq_d}
\lean{QEC1.OptimalCheegerConstant.cheeger_gt_one_bound_eq_d}
\leanok
\uses{def:StabilizerCode.distance, def:DeformedCodeChecks.deformedStabilizerCode, def:cheegerConstant, thm:OptimalCheegerConstant.sufficient_expansion_gives_dstar_ge_d}
If $h(G) > 1$, then under the same hypotheses as Point~1,
\[
d \leq d^*.
\]
That is, the lower bound does not improve beyond what $h(G) = 1$ already provides.
\end{theorem}
\begin{proof}
\leanok
\uses{thm:OptimalCheegerConstant.sufficient_expansion_gives_dstar_ge_d}
Since $h(G) > 1$ implies $h(G) \geq 1$, this is a direct application of the sufficient expansion theorem.
\end{proof}

\section*{Summary: Distance bound trichotomy}

\begin{theorem}[Distance trichotomy]
\label{thm:OptimalCheegerConstant.distance_trichotomy}
\lean{QEC1.OptimalCheegerConstant.distance_trichotomy}
\leanok
\uses{def:StabilizerCode.distance, def:cheegerConstant, lem:OptimalCheegerConstant.cheeger_lt_one_bound_lt_d}
Let $d > 0$ and $h(G) > 0$. The distance bound exhibits the following dichotomy:
\begin{enumerate}
\item If $h(G) \geq 1$, then $\min(h(G), 1) \cdot d = d$ (the bound saturates).
\item If $h(G) < 1$, then $\min(h(G), 1) \cdot d < d$ (strict distance loss).
\end{enumerate}
\end{theorem}
\begin{proof}
\leanok
\uses{lem:OptimalCheegerConstant.cheeger_lt_one_bound_lt_d}
We prove both conjuncts separately.
\begin{enumerate}
\item Assume $h(G) \geq 1$. Then $\min(h(G), 1) = 1$, so $\min(h(G), 1) \cdot d = 1 \cdot d = d$.
\item Assume $h(G) < 1$. Then $\min(h(G), 1) = h(G)$ (since $h(G) \leq h(G) < 1$). We apply the lemma that $h(G) \cdot d < d$ when $d > 0$, $h(G) > 0$, and $h(G) < 1$.
\end{enumerate}
\end{proof}

%--- Rem_14: ParallelGaugingMeasurement ---

I'll start by reading the Lean file to understand its contents.Now I have the full file. Let me produce the LaTeX translation.

\chapter{Rem 14: Parallel Gauging Measurement}

The gauging measurement can be applied to multiple logical operators in
parallel, subject to compatibility conditions:
(1) \emph{Non-overlapping support:} If logical operators $L_1, \ldots, L_m$ have disjoint supports,
they can be gauged simultaneously with independent graphs $G_1, \ldots, G_m$.
(2) \emph{Same-type overlapping support:} If $L_i$ and $L_j$ share support where both act by the
same Pauli type on every shared qubit, they can be gauged in parallel.
(3) \emph{LDPC constraint:} At most a constant number of logical operators should share
support on any single qubit to maintain the LDPC property.
(4) \emph{Time-space tradeoff:} Measuring $2m-1$ copies in parallel with $\lceil d/m \rceil$ rounds and
majority vote trades time for space.

\section*{Part 1: Non-overlapping support}

\begin{definition}[Disjoint Supports]
\label{def:ParallelGaugingMeasurement.DisjointSupports}
\lean{QEC1.ParallelGaugingMeasurement.DisjointSupports}
\leanok
\uses{def:PauliOp, def:PauliOp.support}
Two Pauli operators $P, Q \in \mathcal{P}_V$ have \emph{disjoint supports} if
$\operatorname{supp}(P) \cap \operatorname{supp}(Q) = \emptyset$,
i.e., $\operatorname{supp}(P)$ and $\operatorname{supp}(Q)$ are disjoint as finsets.
\end{definition}

\begin{definition}[Pairwise Disjoint Supports]
\label{def:ParallelGaugingMeasurement.PairwiseDisjointSupports}
\lean{QEC1.ParallelGaugingMeasurement.PairwiseDisjointSupports}
\leanok
\uses{def:PauliOp, def:ParallelGaugingMeasurement.DisjointSupports}
A family of Pauli operators $(P_i)_{i \in \operatorname{Fin}(m)}$ has \emph{pairwise disjoint supports} if for all $i \neq j$, we have $\operatorname{DisjointSupports}(P_i, P_j)$.
\end{definition}

\begin{theorem}[Disjoint Supports Imply Commutation]
\label{thm:ParallelGaugingMeasurement.disjoint_supports_imply_commutation}
\lean{QEC1.ParallelGaugingMeasurement.disjoint_supports_imply_commutation}
\leanok
\uses{def:ParallelGaugingMeasurement.DisjointSupports, def:PauliOp.PauliCommute, def:PauliOp.symplecticInner, def:PauliOp, def:PauliOp.support}
If two Pauli operators $P, Q \in \mathcal{P}_V$ have disjoint supports, then they commute:
$\operatorname{DisjointSupports}(P, Q) \Rightarrow \operatorname{PauliCommute}(P, Q)$.
\end{theorem}

\begin{proof}
\leanok
\uses{def:ParallelGaugingMeasurement.DisjointSupports, def:PauliOp.PauliCommute, def:PauliOp.symplecticInner}
We unfold the definitions of $\operatorname{PauliCommute}$ and $\operatorname{symplecticInner}$ and show the sum $\sum_{v} (P.\mathrm{xVec}(v) \cdot Q.\mathrm{zVec}(v) + P.\mathrm{zVec}(v) \cdot Q.\mathrm{xVec}(v)) = 0$. We apply $\operatorname{Finset.sum\_eq\_zero}$ and consider an arbitrary vertex $v$. Unfolding the disjointness hypothesis via $\operatorname{Finset.disjoint\_left}$, we case-split on whether $v \in \operatorname{supp}(P)$. If $v \in \operatorname{supp}(P)$, then by disjointness $v \notin \operatorname{supp}(Q)$, so $Q.\mathrm{xVec}(v) = 0$ and $Q.\mathrm{zVec}(v) = 0$, making the summand zero by simplification. If $v \notin \operatorname{supp}(P)$, then $P.\mathrm{xVec}(v) = 0$ and $P.\mathrm{zVec}(v) = 0$, again making the summand zero.
\end{proof}

\begin{theorem}[Pairwise Disjoint Implies Pairwise Commute]
\label{thm:ParallelGaugingMeasurement.pairwise_disjoint_implies_pairwise_commute}
\lean{QEC1.ParallelGaugingMeasurement.pairwise_disjoint_implies_pairwise_commute}
\leanok
\uses{def:ParallelGaugingMeasurement.PairwiseDisjointSupports, def:PauliOp.PauliCommute, def:PauliOp}
If a family of Pauli operators $(P_i)_{i \in \operatorname{Fin}(m)}$ has pairwise disjoint supports, then all pairs commute: for all $i, j$, $\operatorname{PauliCommute}(P_i, P_j)$.
\end{theorem}

\begin{proof}
\leanok
\uses{thm:ParallelGaugingMeasurement.disjoint_supports_imply_commutation, thm:PauliOp.pauliCommute_self}
Let $i, j \in \operatorname{Fin}(m)$. We case-split on whether $i = j$. If $i = j$, then we substitute and apply $\operatorname{pauliCommute\_self}$. If $i \neq j$, then by the pairwise disjointness hypothesis we have $\operatorname{DisjointSupports}(P_i, P_j)$, and we apply the theorem $\operatorname{disjoint\_supports\_imply\_commutation}$.
\end{proof}

\begin{theorem}[Disjoint Supports Are Symmetric]
\label{thm:ParallelGaugingMeasurement.disjointSupports_comm}
\lean{QEC1.ParallelGaugingMeasurement.disjointSupports_comm}
\leanok
\uses{def:ParallelGaugingMeasurement.DisjointSupports, def:PauliOp}
For Pauli operators $P, Q$, $\operatorname{DisjointSupports}(P, Q) \Leftrightarrow \operatorname{DisjointSupports}(Q, P)$.
\end{theorem}

\begin{proof}
\leanok
\uses{def:ParallelGaugingMeasurement.DisjointSupports}
We unfold the definition of $\operatorname{DisjointSupports}$ and apply the commutativity of the $\operatorname{Disjoint}$ predicate ($\operatorname{disjoint\_comm}$).
\end{proof}

\begin{theorem}[Disjoint Supports with Identity]
\label{thm:ParallelGaugingMeasurement.disjointSupports_one}
\lean{QEC1.ParallelGaugingMeasurement.disjointSupports_one}
\leanok
\uses{def:ParallelGaugingMeasurement.DisjointSupports, def:PauliOp, lem:PauliOp.support_one}
For any Pauli operator $P$, $\operatorname{DisjointSupports}(P, \mathbf{1})$ holds.
\end{theorem}

\begin{proof}
\leanok
\uses{def:ParallelGaugingMeasurement.DisjointSupports, lem:PauliOp.support_one}
We unfold the definition and simplify using the fact that the support of the identity operator is empty ($\operatorname{support\_one}$).
\end{proof}

\section*{Part 2: Same-type overlapping support}

\begin{definition}[Same-Type X Overlap]
\label{def:ParallelGaugingMeasurement.SameTypeXOverlap}
\lean{QEC1.ParallelGaugingMeasurement.SameTypeXOverlap}
\leanok
\uses{def:PauliOp, def:PauliOp.support}
Two Pauli operators $P, Q$ have \emph{same-type X overlap} if on every shared support qubit $v \in \operatorname{supp}(P) \cap \operatorname{supp}(Q)$, both have vanishing $Z$-component: $P.\mathrm{zVec}(v) = 0$ and $Q.\mathrm{zVec}(v) = 0$.
\end{definition}

\begin{definition}[Same-Type Z Overlap]
\label{def:ParallelGaugingMeasurement.SameTypeZOverlap}
\lean{QEC1.ParallelGaugingMeasurement.SameTypeZOverlap}
\leanok
\uses{def:PauliOp, def:PauliOp.support}
Two Pauli operators $P, Q$ have \emph{same-type Z overlap} if on every shared support qubit $v \in \operatorname{supp}(P) \cap \operatorname{supp}(Q)$, both have vanishing $X$-component: $P.\mathrm{xVec}(v) = 0$ and $Q.\mathrm{xVec}(v) = 0$.
\end{definition}

\begin{definition}[Same-Type Overlap]
\label{def:ParallelGaugingMeasurement.SameTypeOverlap}
\lean{QEC1.ParallelGaugingMeasurement.SameTypeOverlap}
\leanok
\uses{def:PauliOp, def:PauliOp.support}
Two Pauli operators $P, Q$ have \emph{same-type overlap} if on every shared support qubit $v \in \operatorname{supp}(P) \cap \operatorname{supp}(Q)$, either both have vanishing $Z$-component ($P.\mathrm{zVec}(v) = 0 \land Q.\mathrm{zVec}(v) = 0$) or both have vanishing $X$-component ($P.\mathrm{xVec}(v) = 0 \land Q.\mathrm{xVec}(v) = 0$).
\end{definition}

\begin{theorem}[Same-Type X Overlap Implies Same-Type Overlap]
\label{thm:ParallelGaugingMeasurement.sameTypeXOverlap_implies_sameTypeOverlap}
\lean{QEC1.ParallelGaugingMeasurement.sameTypeXOverlap_implies_sameTypeOverlap}
\leanok
\uses{def:ParallelGaugingMeasurement.SameTypeXOverlap, def:ParallelGaugingMeasurement.SameTypeOverlap, def:PauliOp}
If $P, Q$ have same-type X overlap, then they have same-type overlap.
\end{theorem}

\begin{proof}
\leanok
\uses{def:ParallelGaugingMeasurement.SameTypeXOverlap, def:ParallelGaugingMeasurement.SameTypeOverlap}
Let $v$ be a qubit with $v \in \operatorname{supp}(P)$ and $v \in \operatorname{supp}(Q)$. We take the left disjunct: by the same-type X overlap hypothesis, $P.\mathrm{zVec}(v) = 0$ and $Q.\mathrm{zVec}(v) = 0$.
\end{proof}

\begin{theorem}[Same-Type Z Overlap Implies Same-Type Overlap]
\label{thm:ParallelGaugingMeasurement.sameTypeZOverlap_implies_sameTypeOverlap}
\lean{QEC1.ParallelGaugingMeasurement.sameTypeZOverlap_implies_sameTypeOverlap}
\leanok
\uses{def:ParallelGaugingMeasurement.SameTypeZOverlap, def:ParallelGaugingMeasurement.SameTypeOverlap, def:PauliOp}
If $P, Q$ have same-type Z overlap, then they have same-type overlap.
\end{theorem}

\begin{proof}
\leanok
\uses{def:ParallelGaugingMeasurement.SameTypeZOverlap, def:ParallelGaugingMeasurement.SameTypeOverlap}
Let $v$ be a qubit with $v \in \operatorname{supp}(P)$ and $v \in \operatorname{supp}(Q)$. We take the right disjunct: by the same-type Z overlap hypothesis, $P.\mathrm{xVec}(v) = 0$ and $Q.\mathrm{xVec}(v) = 0$.
\end{proof}

\begin{theorem}[Disjoint Supports Imply Same-Type Overlap]
\label{thm:ParallelGaugingMeasurement.disjointSupports_implies_sameTypeOverlap}
\lean{QEC1.ParallelGaugingMeasurement.disjointSupports_implies_sameTypeOverlap}
\leanok
\uses{def:ParallelGaugingMeasurement.DisjointSupports, def:ParallelGaugingMeasurement.SameTypeOverlap, def:PauliOp}
Disjoint supports imply same-type overlap vacuously: if $\operatorname{supp}(P)$ and $\operatorname{supp}(Q)$ are disjoint, the condition on shared qubits holds vacuously since there are no shared qubits.
\end{theorem}

\begin{proof}
\leanok
\uses{def:ParallelGaugingMeasurement.DisjointSupports, def:ParallelGaugingMeasurement.SameTypeOverlap}
Let $v$ be a qubit with $v \in \operatorname{supp}(P)$ and $v \in \operatorname{supp}(Q)$. We derive a contradiction: by $\operatorname{Finset.disjoint\_left}$ applied to the disjointness hypothesis, $v \in \operatorname{supp}(P)$ and $v \in \operatorname{supp}(Q)$ cannot both hold.
\end{proof}

\begin{theorem}[Same-Type Overlap Implies Commutation]
\label{thm:ParallelGaugingMeasurement.sameType_overlap_implies_commutation}
\lean{QEC1.ParallelGaugingMeasurement.sameType_overlap_implies_commutation}
\leanok
\uses{def:ParallelGaugingMeasurement.SameTypeOverlap, def:PauliOp.PauliCommute, def:PauliOp.symplecticInner, def:PauliOp, def:PauliOp.support}
If $P, Q$ have same-type overlap, then they commute: $\operatorname{SameTypeOverlap}(P, Q) \Rightarrow \operatorname{PauliCommute}(P, Q)$.
\end{theorem}

\begin{proof}
\leanok
\uses{def:ParallelGaugingMeasurement.SameTypeOverlap, def:PauliOp.PauliCommute, def:PauliOp.symplecticInner}
We unfold $\operatorname{PauliCommute}$ and $\operatorname{symplecticInner}$ and apply $\operatorname{Finset.sum\_eq\_zero}$. For each vertex $v$, we case-split on whether $v \in \operatorname{supp}(P)$. If $v \in \operatorname{supp}(P)$, we further case-split on whether $v \in \operatorname{supp}(Q)$. If both hold, we obtain from the same-type overlap hypothesis that either ($P.\mathrm{zVec}(v) = 0$ and $Q.\mathrm{zVec}(v) = 0$) or ($P.\mathrm{xVec}(v) = 0$ and $Q.\mathrm{xVec}(v) = 0$). In either case, the cross terms in the symplectic inner product vanish by simplification. If $v \notin \operatorname{supp}(Q)$, then $Q.\mathrm{xVec}(v) = 0$ and $Q.\mathrm{zVec}(v) = 0$, giving zero. If $v \notin \operatorname{supp}(P)$, then $P.\mathrm{xVec}(v) = 0$ and $P.\mathrm{zVec}(v) = 0$, giving zero.
\end{proof}

\begin{theorem}[Gauss's Law Operator Is Pure X-Type]
\label{thm:ParallelGaugingMeasurement.gaussLaw_pure_X_type}
\lean{QEC1.ParallelGaugingMeasurement.gaussLaw_pure_X_type}
\leanok
\uses{def:GaussFlux.gaussLawOp}
For any graph $G$ on vertex set $V$ and any vertex $v \in V$, the Gauss's law operator $A_v$satisfies $(A_v).\mathrm{zVec} = 0$, i.e., it is a pure $X$-type operator.
\end{theorem}

\begin{proof}
\leanok
\uses{lem:GaussFlux.gaussLawOp_zVec}
This follows directly from $\operatorname{gaussLawOp\_zVec}$.
\end{proof}

\begin{theorem}[Pure X-Type Operators Have Same-Type X Overlap]
\label{thm:ParallelGaugingMeasurement.pure_X_sameType_overlap}
\lean{QEC1.ParallelGaugingMeasurement.pure_X_sameType_overlap}
\leanok
\uses{def:ParallelGaugingMeasurement.SameTypeXOverlap, def:PauliOp}
If $P$ and $Q$ are both pure $X$-type operators (i.e., $P.\mathrm{zVec} = 0$ and $Q.\mathrm{zVec} = 0$), then they have same-type X overlap.
\end{theorem}

\begin{proof}
\leanok
\uses{def:ParallelGaugingMeasurement.SameTypeXOverlap}
Let $v$ be any qubit in both supports. Since $P.\mathrm{zVec} = 0$ and $Q.\mathrm{zVec} = 0$ globally, evaluating at $v$ gives $P.\mathrm{zVec}(v) = 0$ and $Q.\mathrm{zVec}(v) = 0$ by congruence.
\end{proof}

\begin{theorem}[Pure X-Type Operators Commute]
\label{thm:ParallelGaugingMeasurement.pure_X_commute}
\lean{QEC1.ParallelGaugingMeasurement.pure_X_commute}
\leanok
\uses{def:PauliOp.PauliCommute, def:PauliOp}
If $P$ and $Q$ are both pure $X$-type operators ($P.\mathrm{zVec} = 0$ and $Q.\mathrm{zVec} = 0$), then $\operatorname{PauliCommute}(P, Q)$.
\end{theorem}

\begin{proof}
\leanok
\uses{thm:ParallelGaugingMeasurement.sameType_overlap_implies_commutation, thm:ParallelGaugingMeasurement.sameTypeXOverlap_implies_sameTypeOverlap, thm:ParallelGaugingMeasurement.pure_X_sameType_overlap}
By the pure X same-type overlap lemma, $P$ and $Q$ have same-type X overlap. By the implication from X overlap to same-type overlap, they have same-type overlap. By the theorem that same-type overlap implies commutation, they commute.
\end{proof}

\begin{theorem}[Pure Z-Type Operators Commute]
\label{thm:ParallelGaugingMeasurement.pure_Z_commute}
\lean{QEC1.ParallelGaugingMeasurement.pure_Z_commute}
\leanok
\uses{def:PauliOp.PauliCommute, def:PauliOp, def:ParallelGaugingMeasurement.SameTypeZOverlap}
If $P$ and $Q$ are both pure $Z$-type operators ($P.\mathrm{xVec} = 0$ and $Q.\mathrm{xVec} = 0$), then $\operatorname{PauliCommute}(P, Q)$.
\end{theorem}

\begin{proof}
\leanok
\uses{thm:ParallelGaugingMeasurement.sameType_overlap_implies_commutation, thm:ParallelGaugingMeasurement.sameTypeZOverlap_implies_sameTypeOverlap}
We construct a proof that $P$ and $Q$ have same-type Z overlap: for any qubit $v$ in both supports, since $P.\mathrm{xVec} = 0$ and $Q.\mathrm{xVec} = 0$ globally, evaluating at $v$ gives $P.\mathrm{xVec}(v) = 0$ and $Q.\mathrm{xVec}(v) = 0$. By the implication from Z overlap to same-type overlap, they have same-type overlap. By the theorem that same-type overlap implies commutation, they commute.
\end{proof}

\section*{Part 3: LDPC constraint for parallel gauging}

\begin{definition}[Qubit Participation Count]
\label{def:ParallelGaugingMeasurement.qubitParticipationCount}
\lean{QEC1.ParallelGaugingMeasurement.qubitParticipationCount}
\leanok
\uses{def:PauliOp, def:PauliOp.support}
For a family of Pauli operators $(P_i)_{i \in \operatorname{Fin}(m)}$ and a qubit $v \in V$, the \emph{qubit participation count} is the number of operators whose support contains $v$:
\[
\operatorname{qubitParticipationCount}(\{P_i\}, v) = \bigl|\{i \in \operatorname{Fin}(m) : v \in \operatorname{supp}(P_i)\}\bigr|.
\]
\end{definition}

\begin{definition}[Parallel LDPC Bound]
\label{def:ParallelGaugingMeasurement.ParallelLDPCBound}
\lean{QEC1.ParallelGaugingMeasurement.ParallelLDPCBound}
\leanok
\uses{def:ParallelGaugingMeasurement.qubitParticipationCount, def:PauliOp}
A family of Pauli operators $(P_i)_{i \in \operatorname{Fin}(m)}$ satisfies the \emph{parallel LDPC bound} with constant $c$ if every qubit participates in at most $c$ of the operators' supports:
\[
\forall v \in V, \quad \operatorname{qubitParticipationCount}(\{P_i\}, v) \le c.
\]
\end{definition}

\begin{theorem}[Disjoint Participation Is At Most 1]
\label{thm:ParallelGaugingMeasurement.disjoint_participation_le_one}
\lean{QEC1.ParallelGaugingMeasurement.disjoint_participation_le_one}
\leanok
\uses{def:ParallelGaugingMeasurement.PairwiseDisjointSupports, def:ParallelGaugingMeasurement.qubitParticipationCount, def:PauliOp}
If a family of operators has pairwise disjoint supports, then for every qubit $v$, $\operatorname{qubitParticipationCount}(\{P_i\}, v) \le 1$.
\end{theorem}

\begin{proof}
\leanok
\uses{def:ParallelGaugingMeasurement.qubitParticipationCount, def:ParallelGaugingMeasurement.PairwiseDisjointSupports}
We unfold the definition of $\operatorname{qubitParticipationCount}$ and apply $\operatorname{Finset.card\_le\_one}$. We must show that if $i$ and $j$ are both in the filter (i.e., $v \in \operatorname{supp}(P_i)$ and $v \in \operatorname{supp}(P_j)$), then $i = j$. Suppose for contradiction that $i \neq j$. Then by pairwise disjointness, $\operatorname{supp}(P_i)$ and $\operatorname{supp}(P_j)$ are disjoint, so by $\operatorname{Finset.disjoint\_left}$, $v$ cannot belong to both---contradiction.
\end{proof}

\begin{theorem}[Disjoint Implies LDPC Bound]
\label{thm:ParallelGaugingMeasurement.disjoint_implies_LDPC_bound}
\lean{QEC1.ParallelGaugingMeasurement.disjoint_implies_LDPC_bound}
\leanok
\uses{def:ParallelGaugingMeasurement.PairwiseDisjointSupports, def:ParallelGaugingMeasurement.ParallelLDPCBound, def:PauliOp}
If a family of operators has pairwise disjoint supports, then the parallel LDPC bound holds with $c = 1$.
\end{theorem}

\begin{proof}
\leanok
\uses{thm:ParallelGaugingMeasurement.disjoint_participation_le_one}
This follows directly from the theorem that disjoint participation is at most 1, applied to each qubit $v$.
\end{proof}

\begin{theorem}[Participation Bounded by Total Operators]
\label{thm:ParallelGaugingMeasurement.participation_le_total}
\lean{QEC1.ParallelGaugingMeasurement.participation_le_total}
\leanok
\uses{def:ParallelGaugingMeasurement.qubitParticipationCount, def:PauliOp}
For any family of $m$ operators and any qubit $v$, $\operatorname{qubitParticipationCount}(\{P_i\}, v) \le m$.
\end{theorem}

\begin{proof}
\leanok
\uses{def:ParallelGaugingMeasurement.qubitParticipationCount}
We unfold the definition. The filter of $\operatorname{Fin}(m)$ has cardinality at most $|\operatorname{Fin}(m)| = m$ by $\operatorname{Finset.card\_filter\_le}$ and $\operatorname{Finset.card\_fin}$.
\end{proof}

\begin{theorem}[Sum of Participation Equals Sum of Support Sizes]
\label{thm:ParallelGaugingMeasurement.sum_participation_eq_sum_support}
\lean{QEC1.ParallelGaugingMeasurement.sum_participation_eq_sum_support}
\leanok
\uses{def:ParallelGaugingMeasurement.qubitParticipationCount, def:PauliOp, def:PauliOp.support}
For a family of $m$ operators $(P_i)$,
\[
\sum_{v \in V} \operatorname{qubitParticipationCount}(\{P_i\}, v) = \sum_{i \in \operatorname{Fin}(m)} |\operatorname{supp}(P_i)|.
\]
\end{theorem}

\begin{proof}
\leanok
\uses{def:ParallelGaugingMeasurement.qubitParticipationCount}
We unfold the definition and rewrite the cardinalities of the filter sets using $\operatorname{Finset.card\_filter}$ as sums of indicator functions ($\operatorname{Finset.sum\_boole}$). We then interchange the order of summation using $\operatorname{Finset.sum\_comm}$. The resulting inner sums are shown equal by extensionality and simplification.
\end{proof}

\begin{theorem}[Parallel Qubit Degree Bound]
\label{thm:ParallelGaugingMeasurement.parallel_qubit_degree_bound}
\lean{QEC1.ParallelGaugingMeasurement.parallel_qubit_degree_bound}
\leanok
\uses{def:ParallelGaugingMeasurement.ParallelLDPCBound, def:ParallelGaugingMeasurement.qubitParticipationCount, def:PauliOp}
When the parallel LDPC bound holds with constant $c$, the participation count at each qubit $v$ is at most $c$:
$\operatorname{ParallelLDPCBound}(\{P_i\}, c) \Rightarrow \operatorname{qubitParticipationCount}(\{P_i\}, v) \le c$.
\end{theorem}

\begin{proof}
\leanok
\uses{def:ParallelGaugingMeasurement.ParallelLDPCBound}
This follows directly from the definition of $\operatorname{ParallelLDPCBound}$ applied to $v$.
\end{proof}

% COMMENTED OUT: Disconnected cluster (ParallelGaugingMeasurement time-space tradeoff, not connected to main network)
% \section*{Part 4: Time-space tradeoff}
%
% \begin{definition}[Number of Copies]
% \label{def:ParallelGaugingMeasurement.numCopies}
% \lean{QEC1.ParallelGaugingMeasurement.numCopies}
% \leanok
%
% The number of parallel copies in the time-space tradeoff: $\operatorname{numCopies}(m) = 2m - 1$ for $m \ge 1$. Each copy is an equivalent logical representative with its own gauging graph.
% \end{definition}
%
% \begin{definition}[Number of Rounds]
% \label{def:ParallelGaugingMeasurement.numRounds}
% \lean{QEC1.ParallelGaugingMeasurement.numRounds}
% \leanok
%
% The number of rounds in the time-space tradeoff: $\operatorname{numRounds}(d, m) = \lceil d/m \rceil$, computed as $(d + m - 1) / m$ (integer ceiling division).
% \end{definition}

% (19 declarations from numCopies_pos through minority_vote_bound omitted
%  — all part of ParallelGaugingMeasurement disconnected cluster)

\section*{Combined properties}

\begin{theorem}[Disjoint Parallel Gauging Compatible]
\label{thm:ParallelGaugingMeasurement.disjoint_parallel_gauging_compatible}
\lean{QEC1.ParallelGaugingMeasurement.disjoint_parallel_gauging_compatible}
\leanok
\uses{def:ParallelGaugingMeasurement.PairwiseDisjointSupports, def:PauliOp.PauliCommute, def:ParallelGaugingMeasurement.ParallelLDPCBound, def:PauliOp}
For pairwise disjoint supports, all conditions for parallel gauging are automatically satisfied: (1) all pairs commute, and (2) the parallel LDPC bound holds with $c = 1$.
\end{theorem}

\begin{proof}
\leanok
\uses{thm:ParallelGaugingMeasurement.pairwise_disjoint_implies_pairwise_commute, thm:ParallelGaugingMeasurement.disjoint_implies_LDPC_bound}
Both components follow directly: pairwise commutation from $\operatorname{pairwise\_disjoint\_implies\_pairwise\_commute}$, and the LDPC bound from $\operatorname{disjoint\_implies\_LDPC\_bound}$.
\end{proof}

% COMMENTED OUT: Remaining ParallelGaugingMeasurement disconnected cluster
% (time_space_tradeoff_summary, no_parallelism_case, max_parallelism_case, overhead_additional_copies)
% — 4 theorems referencing disconnected numCopies/numRounds definitions

%--- Def_7: SpaceAndTimeFaults ---
\chapter{Def 7: Space and Time Faults}

In the context of fault-tolerant quantum error correction, we formalize the notions of space-faults (Pauli errors on qubits), time-faults (measurement errors), and their combined spacetime-fault model.

\begin{definition}[Space Fault]
\label{def:SpaceFault}
\lean{QEC1.SpaceFault}
\leanok
\uses{def:PauliOp}
A \emph{space-fault} (or space error) on a qubit set $Q$ at times $T$ is a structure consisting of:
\begin{itemize}
  \item a qubit $q \in Q$ where the error occurs,
  \item a time $t \in T$ at which the error occurs,
  \item an $X$-component $x \in \mathbb{Z}/2\mathbb{Z}$ (equal to $1$ if the error has an $X$ or $Y$ component, $0$ otherwise),
  \item a $Z$-component $z \in \mathbb{Z}/2\mathbb{Z}$ (equal to $1$ if the error has a $Z$ or $Y$ component, $0$ otherwise),
  \item a nontriviality condition: $x \neq 0$ or $z \neq 0$.
\end{itemize}
This encodes a single-qubit Pauli error ($X$, $Y$, or $Z$) at a specific qubit and time.
\end{definition}

\begin{definition}[Time Fault]
\label{def:TimeFault}
\lean{QEC1.TimeFault}
\leanok
\uses{def:PauliOp}
A \emph{time-fault} (or measurement error) on a measurement set $M$ is a structure consisting of a single field:
\begin{itemize}
  \item a measurement $m \in M$ whose outcome is reported incorrectly (i.e., $+1$ is reported as $-1$ or vice versa).
\end{itemize}
By convention, state mis-initialization faults are also modeled as time-faults: initializing $|0\rangle$ but obtaining $|1\rangle$ is equivalent to a perfect initialization followed by a Pauli $X$ error.
\end{definition}

\begin{definition}[Spacetime Fault]
\label{def:SpacetimeFault}
\lean{QEC1.SpacetimeFault}
\leanok
\uses{def:SpaceFault, def:TimeFault}
A \emph{spacetime fault} on qubit set $Q$, time set $T$, and measurement set $M$ is a structure consisting of:
\begin{itemize}
  \item a finite set $\texttt{spaceFaults} \subseteq \operatorname{Finset}(\operatorname{SpaceFault}(Q, T))$ of space-faults,
  \item a finite set $\texttt{timeFaults} \subseteq \operatorname{Finset}(\operatorname{TimeFault}(M))$ of time-faults.
\end{itemize}
This represents a collection of Pauli errors on qubits at specific times together with measurement errors.
\end{definition}

\begin{definition}[Weight of a Spacetime Fault]
\label{def:SpacetimeFault.weight}
\lean{QEC1.SpacetimeFault.weight}
\leanok
\uses{def:SpacetimeFault}
The \emph{weight} of a spacetime fault $F$ is the total number of individual faults:
\[
  \operatorname{weight}(F) = |\texttt{spaceFaults}(F)| + |\texttt{timeFaults}(F)|.
\]
\end{definition}

\begin{definition}[Empty Spacetime Fault]
\label{def:SpacetimeFault.empty}
\lean{QEC1.SpacetimeFault.empty}
\leanok
\uses{def:SpacetimeFault}
The \emph{empty} spacetime fault is the fault-free configuration $(\emptyset, \emptyset)$, consisting of no space-faults and no time-faults.
\end{definition}

\begin{definition}[Spacetime Fault from a Single Space Fault]
\label{def:SpacetimeFault.ofSpaceFault}
\lean{QEC1.SpacetimeFault.ofSpaceFault}
\leanok
\uses{def:SpacetimeFault, def:SpaceFault}
Given a single space-fault $f$, the spacetime fault $\operatorname{ofSpaceFault}(f) = (\{f\}, \emptyset)$ consists of that single space-fault and no time-faults.
\end{definition}

\begin{definition}[Spacetime Fault from a Single Time Fault]
\label{def:SpacetimeFault.ofTimeFault}
\lean{QEC1.SpacetimeFault.ofTimeFault}
\leanok
\uses{def:SpacetimeFault, def:TimeFault}
Given a single time-fault $f$, the spacetime fault $\operatorname{ofTimeFault}(f) = (\emptyset, \{f\})$ consists of no space-faults and that single time-fault.
\end{definition}

\begin{definition}[Composition of Spacetime Faults]
\label{def:SpacetimeFault.compose}
\lean{QEC1.SpacetimeFault.compose}
\leanok
\uses{def:SpacetimeFault}
The \emph{composition} of two spacetime faults $F_1$ and $F_2$ is defined via the symmetric difference:
\[
  \operatorname{compose}(F_1, F_2) = \bigl(\texttt{spaceFaults}(F_1) \mathbin{\triangle} \texttt{spaceFaults}(F_2),\; \texttt{timeFaults}(F_1) \mathbin{\triangle} \texttt{timeFaults}(F_2)\bigr).
\]
This captures the fact that applying the same error twice cancels it, and flipping an outcome twice restores it.
\end{definition}

\begin{definition}[Pure-Space Fault]
\label{def:SpacetimeFault.isPureSpace}
\lean{QEC1.SpacetimeFault.isPureSpace}
\leanok
\uses{def:SpacetimeFault}
A spacetime fault $F$ is \emph{pure-space} if it has no time-faults: $\texttt{timeFaults}(F) = \emptyset$.
\end{definition}

\begin{definition}[Pure-Time Fault]
\label{def:SpacetimeFault.isPureTime}
\lean{QEC1.SpacetimeFault.isPureTime}
\leanok
\uses{def:SpacetimeFault}
A spacetime fault $F$ is \emph{pure-time} if it has no space-faults: $\texttt{spaceFaults}(F) = \emptyset$.
\end{definition}

\begin{definition}[Space Weight]
\label{def:SpacetimeFault.spaceWeight}
\lean{QEC1.SpacetimeFault.spaceWeight}
\leanok
\uses{def:SpacetimeFault}
The \emph{space weight} of a spacetime fault $F$ is $\operatorname{spaceWeight}(F) = |\texttt{spaceFaults}(F)|$.
\end{definition}

\begin{definition}[Time Weight]
\label{def:SpacetimeFault.timeWeight}
\lean{QEC1.SpacetimeFault.timeWeight}
\leanok
\uses{def:SpacetimeFault}
The \emph{time weight} of a spacetime fault $F$ is $\operatorname{timeWeight}(F) = |\texttt{timeFaults}(F)|$.
\end{definition}

\begin{lemma}[Weight of Empty Fault]
\label{lem:SpacetimeFault.weight_empty}
\lean{QEC1.SpacetimeFault.weight_empty}
\leanok
\uses{def:SpacetimeFault.weight, def:SpacetimeFault.empty}
The empty spacetime fault has weight $0$:
\[
  \operatorname{weight}(\operatorname{empty}) = 0.
\]
\end{lemma}

\begin{proof}
\leanok
\uses{def:SpacetimeFault.weight, def:SpacetimeFault.empty}
By simplification using the definitions of weight and empty, we have $|\emptyset| + |\emptyset| = 0$.
\end{proof}

\begin{lemma}[Weight Decomposition]
\label{lem:SpacetimeFault.weight_eq_space_plus_time}
\lean{QEC1.SpacetimeFault.weight_eq_space_plus_time}
\leanok
\uses{def:SpacetimeFault.weight, def:SpacetimeFault.spaceWeight, def:SpacetimeFault.timeWeight}
For any spacetime fault $F$:
\[
  \operatorname{weight}(F) = \operatorname{spaceWeight}(F) + \operatorname{timeWeight}(F).
\]
\end{lemma}

\begin{proof}
\leanok
\uses{def:SpacetimeFault.weight, def:SpacetimeFault.spaceWeight, def:SpacetimeFault.timeWeight}
By simplification using the definitions of weight, spaceWeight, and timeWeight, all three expand to the same sum of cardinalities.
\end{proof}

\begin{lemma}[Space Weight of Empty Fault]
\label{lem:SpacetimeFault.spaceWeight_empty}
\lean{QEC1.SpacetimeFault.spaceWeight_empty}
\leanok
\uses{def:SpacetimeFault.spaceWeight, def:SpacetimeFault.empty}
$\operatorname{spaceWeight}(\operatorname{empty}) = 0$.
\end{lemma}

\begin{proof}
\leanok
\uses{def:SpacetimeFault.spaceWeight, def:SpacetimeFault.empty}
By simplification using the definitions of spaceWeight and empty, the cardinality of the empty set is $0$.
\end{proof}

\begin{lemma}[Time Weight of Empty Fault]
\label{lem:SpacetimeFault.timeWeight_empty}
\lean{QEC1.SpacetimeFault.timeWeight_empty}
\leanok
\uses{def:SpacetimeFault.timeWeight, def:SpacetimeFault.empty}
$\operatorname{timeWeight}(\operatorname{empty}) = 0$.
\end{lemma}

\begin{proof}
\leanok
\uses{def:SpacetimeFault.timeWeight, def:SpacetimeFault.empty}
By simplification using the definitions of timeWeight and empty, the cardinality of the empty set is $0$.
\end{proof}

\begin{lemma}[Weight of a Single Space Fault]
\label{lem:SpacetimeFault.weight_ofSpaceFault}
\lean{QEC1.SpacetimeFault.weight_ofSpaceFault}
\leanok
\uses{def:SpacetimeFault.weight, def:SpacetimeFault.ofSpaceFault}
For any space-fault $f$:
\[
  \operatorname{weight}(\operatorname{ofSpaceFault}(f)) = 1.
\]
\end{lemma}

\begin{proof}
\leanok
\uses{def:SpacetimeFault.weight, def:SpacetimeFault.ofSpaceFault}
By simplification using the definitions of weight and ofSpaceFault, we have $|\{f\}| + |\emptyset| = 1 + 0 = 1$.
\end{proof}

\begin{lemma}[Weight of a Single Time Fault]
\label{lem:SpacetimeFault.weight_ofTimeFault}
\lean{QEC1.SpacetimeFault.weight_ofTimeFault}
\leanok
\uses{def:SpacetimeFault.weight, def:SpacetimeFault.ofTimeFault}
For any time-fault $f$:
\[
  \operatorname{weight}(\operatorname{ofTimeFault}(f)) = 1.
\]
\end{lemma}

\begin{proof}
\leanok
\uses{def:SpacetimeFault.weight, def:SpacetimeFault.ofTimeFault}
By simplification using the definitions of weight and ofTimeFault, we have $|\emptyset| + |\{f\}| = 0 + 1 = 1$.
\end{proof}

\begin{lemma}[Empty Fault is Pure-Space]
\label{lem:SpacetimeFault.isPureSpace_empty}
\lean{QEC1.SpacetimeFault.isPureSpace_empty}
\leanok
\uses{def:SpacetimeFault.isPureSpace, def:SpacetimeFault.empty}
The empty spacetime fault is pure-space.
\end{lemma}

\begin{proof}
\leanok
\uses{def:SpacetimeFault.isPureSpace, def:SpacetimeFault.empty}
By simplification using the definitions of isPureSpace and empty, the time-faults of the empty fault form the empty set.
\end{proof}

\begin{lemma}[Empty Fault is Pure-Time]
\label{lem:SpacetimeFault.isPureTime_empty}
\lean{QEC1.SpacetimeFault.isPureTime_empty}
\leanok
\uses{def:SpacetimeFault.isPureTime, def:SpacetimeFault.empty}
The empty spacetime fault is pure-time.
\end{lemma}

\begin{proof}
\leanok
\uses{def:SpacetimeFault.isPureTime, def:SpacetimeFault.empty}
By simplification using the definitions of isPureTime and empty, the space-faults of the empty fault form the empty set.
\end{proof}

\begin{lemma}[Single Space Fault is Pure-Space]
\label{lem:SpacetimeFault.isPureSpace_ofSpaceFault}
\lean{QEC1.SpacetimeFault.isPureSpace_ofSpaceFault}
\leanok
\uses{def:SpacetimeFault.isPureSpace, def:SpacetimeFault.ofSpaceFault}
For any space-fault $f$, $\operatorname{ofSpaceFault}(f)$ is pure-space.
\end{lemma}

\begin{proof}
\leanok
\uses{def:SpacetimeFault.isPureSpace, def:SpacetimeFault.ofSpaceFault}
By simplification using the definitions of isPureSpace and ofSpaceFault, the time-faults component is $\emptyset$.
\end{proof}

\begin{lemma}[Single Time Fault is Pure-Time]
\label{lem:SpacetimeFault.isPureTime_ofTimeFault}
\lean{QEC1.SpacetimeFault.isPureTime_ofTimeFault}
\leanok
\uses{def:SpacetimeFault.isPureTime, def:SpacetimeFault.ofTimeFault}
For any time-fault $f$, $\operatorname{ofTimeFault}(f)$ is pure-time.
\end{lemma}

\begin{proof}
\leanok
\uses{def:SpacetimeFault.isPureTime, def:SpacetimeFault.ofTimeFault}
By simplification using the definitions of isPureTime and ofTimeFault, the space-faults component is $\emptyset$.
\end{proof}

\begin{lemma}[Positive Weight from Nonempty Space Faults]
\label{lem:SpacetimeFault.weight_pos_of_nonempty_space}
\lean{QEC1.SpacetimeFault.weight_pos_of_nonempty_space}
\leanok
\uses{def:SpacetimeFault.weight, def:SpacetimeFault}
If $\texttt{spaceFaults}(F)$ is nonempty, then $\operatorname{weight}(F) > 0$.
\end{lemma}

\begin{proof}
\leanok
\uses{def:SpacetimeFault.weight}
Unfolding the definition of weight, since $\texttt{spaceFaults}(F)$ is nonempty we have $|\texttt{spaceFaults}(F)| > 0$ by \texttt{Finset.card\_pos}. The result follows by integer arithmetic since $\operatorname{weight}(F) = |\texttt{spaceFaults}(F)| + |\texttt{timeFaults}(F)| \geq |\texttt{spaceFaults}(F)| > 0$.
\end{proof}

\begin{lemma}[Positive Weight from Nonempty Time Faults]
\label{lem:SpacetimeFault.weight_pos_of_nonempty_time}
\lean{QEC1.SpacetimeFault.weight_pos_of_nonempty_time}
\leanok
\uses{def:SpacetimeFault.weight, def:SpacetimeFault}
If $\texttt{timeFaults}(F)$ is nonempty, then $\operatorname{weight}(F) > 0$.
\end{lemma}

\begin{proof}
\leanok
\uses{def:SpacetimeFault.weight}
Unfolding the definition of weight, since $\texttt{timeFaults}(F)$ is nonempty we have $|\texttt{timeFaults}(F)| > 0$ by \texttt{Finset.card\_pos}. The result follows by integer arithmetic since $\operatorname{weight}(F) = |\texttt{spaceFaults}(F)| + |\texttt{timeFaults}(F)| \geq |\texttt{timeFaults}(F)| > 0$.
\end{proof}

\begin{lemma}[Composition is Commutative]
\label{lem:SpacetimeFault.compose_comm}
\lean{QEC1.SpacetimeFault.compose_comm}
\leanok
\uses{def:SpacetimeFault.compose}
For any spacetime faults $F_1, F_2$:
\[
  \operatorname{compose}(F_1, F_2) = \operatorname{compose}(F_2, F_1).
\]
\end{lemma}

\begin{proof}
\leanok
\uses{def:SpacetimeFault.compose}
By extensionality, it suffices to show equality for each component. For the space-faults component, by the definition of compose and membership in the symmetric difference, $a \in S_1 \mathbin{\triangle} S_2$ if and only if $a \in S_2 \mathbin{\triangle} S_1$, which follows by propositional logic (tauto). The same argument applies to the time-faults component.
\end{proof}

\begin{lemma}[Composition is Involutive]
\label{lem:SpacetimeFault.compose_self}
\lean{QEC1.SpacetimeFault.compose_self}
\leanok
\uses{def:SpacetimeFault.compose, def:SpacetimeFault.empty}
For any spacetime fault $F$:
\[
  \operatorname{compose}(F, F) = \operatorname{empty}.
\]
\end{lemma}

\begin{proof}
\leanok
\uses{def:SpacetimeFault.compose, def:SpacetimeFault.empty}
By extensionality, it suffices to check each component. For the space-faults, $a \in S \mathbin{\triangle} S$ if and only if $a \in \emptyset$, since $S \mathbin{\triangle} S = \emptyset$ (an element is in $S$ and not in $S$, or vice versa, which is always false). This follows by simplification using the definition of compose, empty, and membership in the symmetric difference. The same holds for the time-faults.
\end{proof}

\begin{lemma}[Left Identity of Composition]
\label{lem:SpacetimeFault.compose_empty_left}
\lean{QEC1.SpacetimeFault.compose_empty_left}
\leanok
\uses{def:SpacetimeFault.compose, def:SpacetimeFault.empty}
For any spacetime fault $F$:
\[
  \operatorname{compose}(\operatorname{empty}, F) = F.
\]
\end{lemma}

\begin{proof}
\leanok
\uses{def:SpacetimeFault.compose, def:SpacetimeFault.empty}
By extensionality, for each component we have $\emptyset \mathbin{\triangle} S = S$. This follows by simplification using the definitions of compose and empty and membership in the symmetric difference.
\end{proof}

\begin{lemma}[Right Identity of Composition]
\label{lem:SpacetimeFault.compose_empty_right}
\lean{QEC1.SpacetimeFault.compose_empty_right}
\leanok
\uses{def:SpacetimeFault.compose, def:SpacetimeFault.empty}
For any spacetime fault $F$:
\[
  \operatorname{compose}(F, \operatorname{empty}) = F.
\]
\end{lemma}

\begin{proof}
\leanok
\uses{def:SpacetimeFault.compose, def:SpacetimeFault.empty}
By extensionality, for each component we have $S \mathbin{\triangle} \emptyset = S$. This follows by simplification using the definitions of compose and empty and membership in the symmetric difference.
\end{proof}

\begin{lemma}[Composition is Associative]
\label{lem:SpacetimeFault.compose_assoc}
\lean{QEC1.SpacetimeFault.compose_assoc}
\leanok
\uses{def:SpacetimeFault.compose}
For any spacetime faults $F_1, F_2, F_3$:
\[
  \operatorname{compose}(\operatorname{compose}(F_1, F_2), F_3) = \operatorname{compose}(F_1, \operatorname{compose}(F_2, F_3)).
\]
\end{lemma}

\begin{proof}
\leanok
\uses{def:SpacetimeFault.compose}
By extensionality, it suffices to check each component. For the space-faults, we use the definition of compose and membership in the symmetric difference: $a \in (S_1 \mathbin{\triangle} S_2) \mathbin{\triangle} S_3$ if and only if $a \in S_1 \mathbin{\triangle} (S_2 \mathbin{\triangle} S_3)$, which follows by propositional logic (tauto). The same argument applies to the time-faults.
\end{proof}

\begin{definition}[Space Component]
\label{def:SpacetimeFault.spaceComponent}
\lean{QEC1.SpacetimeFault.spaceComponent}
\leanok
\uses{def:SpacetimeFault}
The \emph{space component} of a spacetime fault $F$ is the spacetime fault $(\texttt{spaceFaults}(F), \emptyset)$, keeping only the space-faults.
\end{definition}

\begin{definition}[Time Component]
\label{def:SpacetimeFault.timeComponent}
\lean{QEC1.SpacetimeFault.timeComponent}
\leanok
\uses{def:SpacetimeFault}
The \emph{time component} of a spacetime fault $F$ is the spacetime fault $(\emptyset, \texttt{timeFaults}(F))$, keeping only the time-faults.
\end{definition}

\begin{lemma}[Space Component is Pure-Space]
\label{lem:SpacetimeFault.spaceComponent_isPureSpace}
\lean{QEC1.SpacetimeFault.spaceComponent_isPureSpace}
\leanok
\uses{def:SpacetimeFault.spaceComponent, def:SpacetimeFault.isPureSpace}
For any spacetime fault $F$, the space component $F.\operatorname{spaceComponent}$ is pure-space.
\end{lemma}

\begin{proof}
\leanok
\uses{def:SpacetimeFault.spaceComponent, def:SpacetimeFault.isPureSpace}
By simplification using the definitions of spaceComponent and isPureSpace, the time-faults of the space component are $\emptyset$.
\end{proof}

\begin{lemma}[Time Component is Pure-Time]
\label{lem:SpacetimeFault.timeComponent_isPureTime}
\lean{QEC1.SpacetimeFault.timeComponent_isPureTime}
\leanok
\uses{def:SpacetimeFault.timeComponent, def:SpacetimeFault.isPureTime}
For any spacetime fault $F$, the time component $F.\operatorname{timeComponent}$ is pure-time.
\end{lemma}

\begin{proof}
\leanok
\uses{def:SpacetimeFault.timeComponent, def:SpacetimeFault.isPureTime}
By simplification using the definitions of timeComponent and isPureTime, the space-faults of the time component are $\emptyset$.
\end{proof}

\begin{lemma}[Weight of Space Component]
\label{lem:SpacetimeFault.weight_spaceComponent}
\lean{QEC1.SpacetimeFault.weight_spaceComponent}
\leanok
\uses{def:SpacetimeFault.spaceComponent, def:SpacetimeFault.weight, def:SpacetimeFault.spaceWeight}
For any spacetime fault $F$:
\[
  \operatorname{weight}(F.\operatorname{spaceComponent}) = \operatorname{spaceWeight}(F).
\]
\end{lemma}

\begin{proof}
\leanok
\uses{def:SpacetimeFault.spaceComponent, def:SpacetimeFault.weight, def:SpacetimeFault.spaceWeight}
By simplification using the definitions of spaceComponent, weight, and spaceWeight, the weight of $(\texttt{spaceFaults}(F), \emptyset)$ equals $|\texttt{spaceFaults}(F)| + 0 = |\texttt{spaceFaults}(F)|$.
\end{proof}

\begin{lemma}[Weight of Time Component]
\label{lem:SpacetimeFault.weight_timeComponent}
\lean{QEC1.SpacetimeFault.weight_timeComponent}
\leanok
\uses{def:SpacetimeFault.timeComponent, def:SpacetimeFault.weight, def:SpacetimeFault.timeWeight}
For any spacetime fault $F$:
\[
  \operatorname{weight}(F.\operatorname{timeComponent}) = \operatorname{timeWeight}(F).
\]
\end{lemma}

\begin{proof}
\leanok
\uses{def:SpacetimeFault.timeComponent, def:SpacetimeFault.weight, def:SpacetimeFault.timeWeight}
By simplification using the definitions of timeComponent, weight, and timeWeight, the weight of $(\emptyset, \texttt{timeFaults}(F))$ equals $0 + |\texttt{timeFaults}(F)| = |\texttt{timeFaults}(F)|$.
\end{proof}

\begin{lemma}[Decomposition into Space and Time Components]
\label{lem:SpacetimeFault.decompose_space_time}
\lean{QEC1.SpacetimeFault.decompose_space_time}
\leanok
\uses{def:SpacetimeFault.compose, def:SpacetimeFault.spaceComponent, def:SpacetimeFault.timeComponent}
Any spacetime fault $F$ decomposes into the composition of its space and time components:
\[
  \operatorname{compose}(F.\operatorname{spaceComponent}, F.\operatorname{timeComponent}) = F.
\]
\end{lemma}

\begin{proof}
\leanok
\uses{def:SpacetimeFault.compose, def:SpacetimeFault.spaceComponent, def:SpacetimeFault.timeComponent}
By extensionality, for the space-faults component we have $\texttt{spaceFaults}(F) \mathbin{\triangle} \emptyset = \texttt{spaceFaults}(F)$, and for the time-faults component we have $\emptyset \mathbin{\triangle} \texttt{timeFaults}(F) = \texttt{timeFaults}(F)$. Both follow by simplification using the definitions of compose, spaceComponent, timeComponent, and membership in the symmetric difference.
\end{proof}

\begin{definition}[Space Faults at a Given Time]
\label{def:SpacetimeFault.spaceFaultsAt}
\lean{QEC1.SpacetimeFault.spaceFaultsAt}
\leanok
\uses{def:SpacetimeFault, def:SpaceFault}
The \emph{space-faults at time $t$} of a spacetime fault $F$ is the subset of space-faults whose time field equals $t$:
\[
  \operatorname{spaceFaultsAt}(F, t) = \{f \in \texttt{spaceFaults}(F) \mid f.\operatorname{time} = t\}.
\]
\end{definition}

\begin{definition}[Pauli Error at a Given Time]
\label{def:SpacetimeFault.pauliErrorAt}
\lean{QEC1.SpacetimeFault.pauliErrorAt}
\leanok
\uses{def:SpacetimeFault.spaceFaultsAt, def:PauliOp}
The \emph{composite Pauli error at time $t$} of a spacetime fault $F$ on the qubit system $Q$ is the Pauli operator $P \in \operatorname{PauliOp}(Q)$ defined by:
\[
  P.\operatorname{xVec}(q) = \sum_{f \in \operatorname{spaceFaultsAt}(F, t)} \begin{cases} f.\operatorname{xComponent} & \text{if } f.\operatorname{qubit} = q, \\ 0 & \text{otherwise,}\end{cases}
\]
\[
  P.\operatorname{zVec}(q) = \sum_{f \in \operatorname{spaceFaultsAt}(F, t)} \begin{cases} f.\operatorname{zComponent} & \text{if } f.\operatorname{qubit} = q, \\ 0 & \text{otherwise.}\end{cases}
\]
The sums are taken in $\mathbb{Z}/2\mathbb{Z}$.
\end{definition}

\begin{lemma}[Pauli Error at Empty Fault is Identity]
\label{lem:SpacetimeFault.pauliErrorAt_empty}
\lean{QEC1.SpacetimeFault.pauliErrorAt_empty}
\leanok
\uses{def:SpacetimeFault.pauliErrorAt, def:SpacetimeFault.empty}
For any time $t$, the Pauli error of the empty fault at time $t$ is the identity:
\[
  \operatorname{pauliErrorAt}(\operatorname{empty}, t) = \mathbf{1}.
\]
\end{lemma}

\begin{proof}
\leanok
\uses{def:SpacetimeFault.pauliErrorAt, def:SpacetimeFault.empty, def:SpacetimeFault.spaceFaultsAt}
By extensionality over all qubits $q$, for both the $x$- and $z$-components: the sum is over $\operatorname{spaceFaultsAt}(\operatorname{empty}, t)$, which by the definitions of spaceFaultsAt and empty is a filter of $\emptyset$, hence empty. The empty sum in $\mathbb{Z}/2\mathbb{Z}$ equals $0$, which matches the identity Pauli operator.
\end{proof}

\begin{definition}[Initialization Fault]
\label{def:SpacetimeFault.initializationFault}
\lean{QEC1.SpacetimeFault.initializationFault}
\leanok
\uses{def:SpacetimeFault.ofTimeFault, def:TimeFault}
An \emph{initialization fault} for initialization measurement $m$ is the spacetime fault $\operatorname{ofTimeFault}(\langle m \rangle)$, modeling the initialization error as a time-fault on the measurement associated with the initialization.
\end{definition}

\begin{definition}[Initialization as Space Fault]
\label{def:SpacetimeFault.initializationAsSpaceFault}
\lean{QEC1.SpacetimeFault.initializationAsSpaceFault}
\leanok
\uses{def:SpacetimeFault.ofSpaceFault, def:SpaceFault}
The equivalent space-fault view of an initialization error on qubit $q$ at time $t$: this is $\operatorname{ofSpaceFault}(\langle q, t, 1, 0, \cdot \rangle)$, representing a perfect initialization followed by a Pauli $X$ error (since $x$-component is $1$ and $z$-component is $0$).
\end{definition}

\begin{lemma}[Weight of Initialization Fault]
\label{lem:SpacetimeFault.initializationFault_weight}
\lean{QEC1.SpacetimeFault.initializationFault_weight}
\leanok
\uses{def:SpacetimeFault.initializationFault, def:SpacetimeFault.weight}
For any measurement $m$:
\[
  \operatorname{weight}(\operatorname{initializationFault}(m)) = 1.
\]
\end{lemma}

\begin{proof}
\leanok
\uses{def:SpacetimeFault.initializationFault, def:SpacetimeFault.ofTimeFault, def:SpacetimeFault.weight}
By simplification using the definitions of initializationFault, ofTimeFault, and weight, we have $|\emptyset| + |\{m\}| = 0 + 1 = 1$.
\end{proof}

\begin{lemma}[Weight of Initialization as Space Fault]
\label{lem:SpacetimeFault.initializationAsSpaceFault_weight}
\lean{QEC1.SpacetimeFault.initializationAsSpaceFault_weight}
\leanok
\uses{def:SpacetimeFault.initializationAsSpaceFault, def:SpacetimeFault.weight}
For any qubit $q$ and time $t$:
\[
  \operatorname{weight}(\operatorname{initializationAsSpaceFault}(q, t)) = 1.
\]
\end{lemma}

\begin{proof}
\leanok
\uses{def:SpacetimeFault.initializationAsSpaceFault, def:SpacetimeFault.ofSpaceFault, def:SpacetimeFault.weight}
By simplification using the definitions of initializationAsSpaceFault, ofSpaceFault, and weight, we have $|\{\langle q, t, 1, 0, \cdot\rangle\}| + |\emptyset| = 1 + 0 = 1$.
\end{proof}

\begin{lemma}[Space Weight Bounded by Total Weight]
\label{lem:SpacetimeFault.spaceWeight_le_weight}
\lean{QEC1.SpacetimeFault.spaceWeight_le_weight}
\leanok
\uses{def:SpacetimeFault.spaceWeight, def:SpacetimeFault.weight}
For any spacetime fault $F$:
\[
  \operatorname{spaceWeight}(F) \leq \operatorname{weight}(F).
\]
\end{lemma}

\begin{proof}
\leanok
\uses{def:SpacetimeFault.spaceWeight, def:SpacetimeFault.weight}
Unfolding the definitions of weight and spaceWeight, the claim becomes $|\texttt{spaceFaults}(F)| \leq |\texttt{spaceFaults}(F)| + |\texttt{timeFaults}(F)|$, which follows by integer arithmetic (omega).
\end{proof}

\begin{lemma}[Time Weight Bounded by Total Weight]
\label{lem:SpacetimeFault.timeWeight_le_weight}
\lean{QEC1.SpacetimeFault.timeWeight_le_weight}
\leanok
\uses{def:SpacetimeFault.timeWeight, def:SpacetimeFault.weight}
For any spacetime fault $F$:
\[
  \operatorname{timeWeight}(F) \leq \operatorname{weight}(F).
\]
\end{lemma}

\begin{proof}
\leanok
\uses{def:SpacetimeFault.timeWeight, def:SpacetimeFault.weight}
Unfolding the definitions of weight and timeWeight, the claim becomes $|\texttt{timeFaults}(F)| \leq |\texttt{spaceFaults}(F)| + |\texttt{timeFaults}(F)|$, which follows by integer arithmetic (omega).
\end{proof}

\begin{lemma}[Weight Zero Characterization]
\label{lem:SpacetimeFault.weight_zero_iff}
\lean{QEC1.SpacetimeFault.weight_zero_iff}
\leanok
\uses{def:SpacetimeFault.weight, def:SpacetimeFault}
For any spacetime fault $F$:
\[
  \operatorname{weight}(F) = 0 \iff \texttt{spaceFaults}(F) = \emptyset \;\land\; \texttt{timeFaults}(F) = \emptyset.
\]
\end{lemma}

\begin{proof}
\leanok
\uses{def:SpacetimeFault.weight}
By simplification using the definition of weight, $\operatorname{weight}(F) = 0$ is equivalent to $|\texttt{spaceFaults}(F)| + |\texttt{timeFaults}(F)| = 0$, which by the fact that a sum of natural numbers is zero if and only if both summands are zero (\texttt{Nat.add\_eq\_zero\_iff}), and the fact that a finset has cardinality zero if and only if it is empty (\texttt{Finset.card\_eq\_zero}), gives the desired equivalence.
\end{proof}

\begin{lemma}[Weight Zero iff Empty]
\label{lem:SpacetimeFault.weight_zero_iff_empty}
\lean{QEC1.SpacetimeFault.weight_zero_iff_empty}
\leanok
\uses{def:SpacetimeFault.weight, def:SpacetimeFault.empty}
For any spacetime fault $F$:
\[
  \operatorname{weight}(F) = 0 \iff F = \operatorname{empty}.
\]
\end{lemma}

\begin{proof}
\leanok
\uses{def:SpacetimeFault.weight, def:SpacetimeFault.empty, lem:SpacetimeFault.weight_zero_iff}
We prove both directions. For the forward direction, assume $\operatorname{weight}(F) = 0$. From the characterization \texttt{weight\_zero\_iff}, we obtain that $\texttt{spaceFaults}(F) = \emptyset$ and $\texttt{timeFaults}(F) = \emptyset$. By extensionality, and simplifying using the definition of empty, $F = \operatorname{empty}$. For the reverse direction, assume $F = \operatorname{empty}$. Substituting, we simplify to obtain $\operatorname{weight}(\operatorname{empty}) = 0$.
\end{proof}

\begin{definition}[Trivial Fault]
\label{def:SpacetimeFault.isTrivial}
\lean{QEC1.SpacetimeFault.isTrivial}
\leanok
\uses{def:SpacetimeFault, def:SpacetimeFault.empty}
A spacetime fault $F$ is \emph{trivial} if $F = \operatorname{empty}$, i.e., it contains no faults at all.
\end{definition}

\begin{definition}[Affected Qubits]
\label{def:SpacetimeFault.affectedQubits}
\lean{QEC1.SpacetimeFault.affectedQubits}
\leanok
\uses{def:SpacetimeFault, def:SpaceFault}
The set of \emph{affected qubits} of a spacetime fault $F$ is the image of the space-faults under the qubit projection:
\[
  \operatorname{affectedQubits}(F) = \{f.\operatorname{qubit} \mid f \in \texttt{spaceFaults}(F)\}.
\]
\end{definition}

\begin{definition}[Active Times]
\label{def:SpacetimeFault.activeTimes}
\lean{QEC1.SpacetimeFault.activeTimes}
\leanok
\uses{def:SpacetimeFault, def:SpaceFault}
The set of \emph{active times} of a spacetime fault $F$ is the image of the space-faults under the time projection:
\[
  \operatorname{activeTimes}(F) = \{f.\operatorname{time} \mid f \in \texttt{spaceFaults}(F)\}.
\]
\end{definition}

\begin{definition}[Affected Measurements]
\label{def:SpacetimeFault.affectedMeasurements}
\lean{QEC1.SpacetimeFault.affectedMeasurements}
\leanok
\uses{def:SpacetimeFault, def:TimeFault}
The set of \emph{affected measurements} of a spacetime fault $F$ is the image of the time-faults under the measurement projection:
\[
  \operatorname{affectedMeasurements}(F) = \{f.\operatorname{measurement} \mid f \in \texttt{timeFaults}(F)\}.
\]
\end{definition}

\begin{definition}[Observed Outcome]
\label{def:observedOutcome}
\lean{QEC1.observedOutcome}
\leanok
\uses{def:TimeFault}
Given an ideal measurement outcome function $\operatorname{ideal} : M \to \mathbb{Z}/2\mathbb{Z}$ and a set of time-faults, the \emph{observed outcome} for measurement $m$ is:
\[
  \operatorname{observedOutcome}(\operatorname{ideal}, \texttt{faults}, m) = \operatorname{ideal}(m) + \begin{cases} 1 & \text{if } \langle m \rangle \in \texttt{faults}, \\ 0 & \text{otherwise.}\end{cases}
\]
A time-fault on measurement $m$ flips the observed outcome.
\end{definition}

\begin{lemma}[Observed Outcome Without Fault]
\label{lem:observedOutcome_no_fault}
\lean{QEC1.observedOutcome_no_fault}
\leanok
\uses{def:observedOutcome}
If measurement $m$ is not affected by any time-fault (i.e., $\langle m \rangle \notin \texttt{faults}$), then the observed outcome equals the ideal outcome:
\[
  \operatorname{observedOutcome}(\operatorname{ideal}, \texttt{faults}, m) = \operatorname{ideal}(m).
\]
\end{lemma}

\begin{proof}
\leanok
\uses{def:observedOutcome}
By simplification using the definition of observedOutcome and the hypothesis $\langle m \rangle \notin \texttt{faults}$, the conditional evaluates to $0$, giving $\operatorname{ideal}(m) + 0 = \operatorname{ideal}(m)$.
\end{proof}

\begin{lemma}[Observed Outcome With Fault]
\label{lem:observedOutcome_with_fault}
\lean{QEC1.observedOutcome_with_fault}
\leanok
\uses{def:observedOutcome}
If measurement $m$ is affected by a time-fault (i.e., $\langle m \rangle \in \texttt{faults}$), thenthe observed outcome is the ideal outcome plus $1$:
\[
  \operatorname{observedOutcome}(\operatorname{ideal}, \texttt{faults}, m) = \operatorname{ideal}(m) + 1.
\]
\end{lemma}

\begin{proof}
\leanok
\uses{def:observedOutcome}
By simplification using the definition of observedOutcome and the hypothesis $\langle m \rangle \in \texttt{faults}$, the conditional evaluates to $1$, giving $\operatorname{ideal}(m) + 1$.
\end{proof}

\begin{lemma}[Fault Flips Outcome]
\label{lem:observedOutcome_flip}
\lean{QEC1.observedOutcome_flip}
\leanok
\uses{def:observedOutcome, lem:observedOutcome_with_fault}
If measurement $m$ is affected by a time-fault (i.e., $\langle m \rangle \in \texttt{faults}$), then the observed outcome differs from the ideal outcome:
\[
  \operatorname{observedOutcome}(\operatorname{ideal}, \texttt{faults}, m) \neq \operatorname{ideal}(m).
\]
\end{lemma}

\begin{proof}
\leanok
\uses{lem:observedOutcome_with_fault}
We first rewrite using the lemma \texttt{observedOutcome\_with\_fault}, so the goal becomes $\operatorname{ideal}(m) + 1 \neq \operatorname{ideal}(m)$. Assume for contradiction that $\operatorname{ideal}(m) + 1 = \operatorname{ideal}(m)$. Then subtracting $\operatorname{ideal}(m)$ from both sides, we obtain $0 = 1$ in $\mathbb{Z}/2\mathbb{Z}$, which is simplified to a contradiction since $0 \neq 1$.
\end{proof}

%--- Def_8: Detectors ---
\chapter{Def 8: Detectors}

\begin{definition}[Detector]
\label{def:Detector}
\lean{QEC1.Detector}
\leanok
\uses{def:observedOutcome}
A \textbf{detector} is a structure $(D, f, c)$ where:
\begin{itemize}
  \item $D \subseteq M$ is a finite set of measurement labels (a $\operatorname{Finset} M$),
  \item $f : M \to \mathbb{Z}/2\mathbb{Z}$ assigns to each measurement its ideal outcome ($0 \leftrightarrow +1$, $1 \leftrightarrow -1$),
  \item $c$ is a proof of the \emph{detector constraint}: $\sum_{m \in D} f(m) = 0$ in $\mathbb{Z}/2\mathbb{Z}$.
\end{itemize}
In the $\{+1, -1\}$ encoding, the constraint states that the product of ideal outcomes over the detector's measurements equals $+1$.
\end{definition}

\begin{definition}[Observed Parity]
\label{def:Detector.observedParity}
\lean{QEC1.Detector.observedParity}
\leanok
\uses{def:Detector, def:observedOutcome, def:SpacetimeFault}
Given a detector $D$ and a set of time-faults $F$, the \textbf{observed parity} is defined as
\[
  \operatorname{observedParity}(D, F) = \sum_{m \in D.\mathrm{measurements}} \operatorname{observedOutcome}(D.\mathrm{idealOutcome}, F, m) \in \mathbb{Z}/2\mathbb{Z}.
\]
In $\{+1,-1\}$ encoding, $0$ means the product of observed outcomes is $+1$, and $1$ means it is $-1$.
\end{definition}

\begin{definition}[Detector Violation]
\label{def:Detector.isViolated}
\lean{QEC1.Detector.isViolated}
\leanok
\uses{def:Detector, def:Detector.observedParity}
A detector $D$ is \textbf{violated} by a set of time-faults $F$ if the observed parity equals $1$:
\[
  \operatorname{isViolated}(D, F) \iff \operatorname{observedParity}(D, F) = 1.
\]
\end{definition}

\begin{definition}[Flip Parity]
\label{def:Detector.flipParity}
\lean{QEC1.Detector.flipParity}
\leanok
\uses{def:Detector, def:SpacetimeFault}
The \textbf{flip parity} of a detector $D$ with respect to faults $F$ is the sum in $\mathbb{Z}/2\mathbb{Z}$ of the indicator of faulted measurements in $D$:
\[
  \operatorname{flipParity}(D, F) = \sum_{m \in D.\mathrm{measurements}} \begin{cases} 1 & \text{if } \langle m \rangle \in F \\ 0 & \text{otherwise} \end{cases}.
\]
\end{definition}

\begin{theorem}[No-Fault Observed Parity]
\label{thm:Detector.observedParity_no_faults}
\lean{QEC1.Detector.observedParity_no_faults}
\leanok
\uses{def:Detector, def:Detector.observedParity}
In the absence of faults, the observed parity of any detector equals $0$:
\[
  \operatorname{observedParity}(D, \emptyset) = 0.
\]
\end{theorem}

\begin{proof}
\leanok
\uses{def:Detector.observedParity, def:observedOutcome}
Unfolding the definitions of observed parity and observed outcome, for every $m \in D.\mathrm{measurements}$ we have $\langle m \rangle \notin \emptyset$, so the observed outcome reduces to the ideal outcome $D.\mathrm{idealOutcome}(m)$. Rewriting the sum using this simplification, the result is $\sum_{m \in D.\mathrm{measurements}} D.\mathrm{idealOutcome}(m)$, which equals $0$ by the detector constraint.
\end{proof}

\begin{theorem}[No Violation Without Faults]
\label{thm:Detector.not_isViolated_no_faults}
\lean{QEC1.Detector.not_isViolated_no_faults}
\leanok
\uses{def:Detector, def:Detector.isViolated}
In the absence of faults, no detector is violated:
\[
  \neg\, \operatorname{isViolated}(D, \emptyset).
\]
\end{theorem}

\begin{proof}
\leanok
\uses{thm:Detector.observedParity_no_faults, def:Detector.isViolated}
Unfolding the definition of violation, we need $\operatorname{observedParity}(D, \emptyset) \neq 1$. By the no-fault observed parity theorem, the observed parity is $0$, and $0 \neq 1$.
\end{proof}

\begin{theorem}[Observed Parity Equals Flip Parity]
\label{thm:Detector.observedParity_eq_flipParity}
\lean{QEC1.Detector.observedParity_eq_flipParity}
\leanok
\uses{def:Detector.observedParity, def:Detector.flipParity}
The observed parity equals the flip parity:
\[
  \operatorname{observedParity}(D, F) = \operatorname{flipParity}(D, F).
\]
That is, the observed parity reduces to counting (mod 2) how many of the detector's measurements are faulted, because the ideal outcomes cancel by the detector constraint.
\end{theorem}

\begin{proof}
\leanok
\uses{def:Detector.observedParity, def:Detector.flipParity, def:observedOutcome}
Unfolding the definitions of observed parity, flip parity, and observed outcome, the sum $\sum_{m \in D.\mathrm{measurements}} (D.\mathrm{idealOutcome}(m) + \mathbf{1}_{\langle m \rangle \in F})$ splits by the distributivity of summation as
\[
  \sum_{m \in D.\mathrm{measurements}} D.\mathrm{idealOutcome}(m) \;+\; \sum_{m \in D.\mathrm{measurements}} \mathbf{1}_{\langle m \rangle \in F}.
\]
By the detector constraint, the first sum is $0$, so the result equals the flip parity.
\end{proof}

\begin{theorem}[Violation Iff Flip Parity One]
\label{thm:Detector.isViolated_iff_flipParity}
\lean{QEC1.Detector.isViolated_iff_flipParity}
\leanok
\uses{def:Detector.isViolated, def:Detector.flipParity}
A detector is violated if and only if its flip parity equals $1$:
\[
  \operatorname{isViolated}(D, F) \iff \operatorname{flipParity}(D, F) = 1.
\]
\end{theorem}

\begin{proof}
\leanok
\uses{thm:Detector.observedParity_eq_flipParity, def:Detector.isViolated}
Unfolding the definition of violation, we rewrite the observed parity as the flip parity using the equality $\operatorname{observedParity} = \operatorname{flipParity}$.
\end{proof}

\begin{theorem}[Flip Parity Without Faults]
\label{thm:Detector.flipParity_no_faults}
\lean{QEC1.Detector.flipParity_no_faults}
\leanok
\uses{def:Detector.flipParity}
The flip parity with no faults is $0$:
\[
  \operatorname{flipParity}(D, \emptyset) = 0.
\]
\end{theorem}

\begin{proof}
\leanok
\uses{def:Detector.flipParity}
Unfolding the definition of flip parity, since no measurement label belongs to the empty fault set, every summand is $0$, and the result follows by simplification.
\end{proof}

\begin{theorem}[Violation Depends on Intersection]
\label{thm:Detector.isViolated_depends_on_intersection}
\lean{QEC1.Detector.isViolated_depends_on_intersection}
\leanok
\uses{def:Detector.isViolated, def:Detector.flipParity}
The violation of a detector depends only on which of the detector's measurements appear in the fault set. If two fault sets $F_1, F_2$ satisfy $\langle m \rangle \in F_1 \iff \langle m \rangle \in F_2$ for all $m \in D.\mathrm{measurements}$, then
\[
  \operatorname{isViolated}(D, F_1) \iff \operatorname{isViolated}(D, F_2).
\]
\end{theorem}

\begin{proof}
\leanok
\uses{thm:Detector.isViolated_iff_flipParity, def:Detector.flipParity}
By the characterization of violation in terms of flip parity, it suffices to show the flip parities are equal. Both directions follow by converting the goal and applying the sum congruence: for each $m \in D.\mathrm{measurements}$, the indicator $\mathbf{1}_{\langle m \rangle \in F_1}$ equals $\mathbf{1}_{\langle m \rangle \in F_2}$ by hypothesis $h$.
\end{proof}

\begin{theorem}[Violation Invariant Under Disjoint Faults]
\label{thm:Detector.isViolated_of_superset_disjoint}
\lean{QEC1.Detector.isViolated_of_superset_disjoint}
\leanok
\uses{def:Detector.isViolated}
Violation is invariant if we add faults outside the detector. If no measurement of $D$ appears in $\mathrm{extra}$, then
\[
  \operatorname{isViolated}(D, F \cup \mathrm{extra}) \iff \operatorname{isViolated}(D, F).
\]
\end{theorem}

\begin{proof}
\leanok
\uses{thm:Detector.isViolated_depends_on_intersection}
We apply the intersection-dependence theorem. For each $m \in D.\mathrm{measurements}$, we show $\langle m \rangle \in F \cup \mathrm{extra} \iff \langle m \rangle \in F$. The forward direction: if $\langle m \rangle \in F \cup \mathrm{extra}$, then either $\langle m \rangle \in F$ (done) or $\langle m \rangle \in \mathrm{extra}$, which contradicts the disjointness hypothesis. The backward direction: if $\langle m \rangle \in F$, then $\langle m \rangle \in F \cup \mathrm{extra}$ by left inclusion.
\end{proof}

\begin{definition}[Repeated Measurement Detector]
\label{def:repeatedMeasurementDetector}
\lean{QEC1.repeatedMeasurementDetector}
\leanok
\uses{def:Detector}
A \textbf{repeated measurement detector} is formed from two consecutive measurements $m_1 \neq m_2$ of the same stabilizer check with common ideal outcome $o \in \mathbb{Z}/2\mathbb{Z}$. The detector has:
\begin{itemize}
  \item measurements $= \{m_1, m_2\}$,
  \item ideal outcome $f(m) = o$ if $m = m_1$ or $m = m_2$, and $0$ otherwise.
\end{itemize}
The detector constraint holds because $o + o = 0$ in $\mathbb{Z}/2\mathbb{Z}$.
\end{definition}

\begin{definition}[Initialization-Measurement Detector]
\label{def:initAndMeasureDetector}
\lean{QEC1.initAndMeasureDetector}
\leanok
\uses{def:Detector}
An \textbf{initialization-measurement detector} is formed from an initialization event $m_{\mathrm{init}}$ and a later measurement $m_{\mathrm{meas}}$ with $m_{\mathrm{init}} \neq m_{\mathrm{meas}}$, both having the same ideal outcome $o$. Per Definition~7, initializations are treated as measurements, so this forms a valid detector with measurements $\{m_{\mathrm{init}}, m_{\mathrm{meas}}\}$ and ideal outcome $o$ on both, satisfying the constraint $o + o = 0$ in $\mathbb{Z}/2\mathbb{Z}$.
\end{definition}

\begin{theorem}[Repeated Measurement Violation Characterization]
\label{thm:Detector.repeatedMeasurement_violated_iff}
\lean{QEC1.Detector.repeatedMeasurement_violated_iff}
\leanok
\uses{def:repeatedMeasurementDetector, def:Detector.isViolated}
A repeated measurement detector on $m_1, m_2$ with $m_1 \neq m_2$ is violated by faults $F$ if and only if exactly one of the two measurements is faulted:
\[
  \operatorname{isViolated}(\operatorname{repeatedMeasurementDetector}(m_1, m_2), F) \iff (\langle m_1 \rangle \in F) \oplus (\langle m_2 \rangle \in F).
\]
\end{theorem}

\begin{proof}
\leanok
\uses{thm:Detector.isViolated_iff_flipParity, def:Detector.flipParity, def:repeatedMeasurementDetector}
We rewrite using the flip parity characterization and unfold the definitions. The flip parity over $\{m_1, m_2\}$ equals $\mathbf{1}_{\langle m_1 \rangle \in F} + \mathbf{1}_{\langle m_2 \rangle \in F}$ in $\mathbb{Z}/2\mathbb{Z}$. For the forward direction, assuming this sum equals $1$, we case-split on whether $\langle m_1 \rangle \in F$ and $\langle m_2 \rangle \in F$; in each case we simplify to conclude exactly one holds, giving exclusive or. For the backward direction, given the exclusive or, we similarly case-split and simplify to show the sum equals $1$.
\end{proof}

\begin{definition}[Syndrome]
\label{def:Detector.syndrome}
\lean{QEC1.Detector.syndrome}
\leanok
\uses{def:Detector, def:Detector.observedParity}
The \textbf{syndrome} of a set of time-faults $F$ with respect to a family of detectors $(D_i)_{i \in I}$ is the set of detector indices whose detectors are violated:
\[
  \operatorname{syndrome}(\{D_i\}, F) = \{i \in I \mid \operatorname{observedParity}(D_i, F) = 1\}.
\]
\end{definition}

\begin{theorem}[Membership in Syndrome]
\label{thm:Detector.mem_syndrome_iff}
\lean{QEC1.Detector.mem_syndrome_iff}
\leanok
\uses{def:Detector.syndrome, def:Detector.isViolated}
A detector index $i$ is in the syndrome if and only if the detector $D_i$ is violated:
\[
  i \in \operatorname{syndrome}(\{D_i\}, F) \iff \operatorname{isViolated}(D_i, F).
\]
\end{theorem}

\begin{proof}
\leanok
\uses{def:Detector.syndrome, def:Detector.isViolated}
This follows by simplification, unfolding the definitions of syndrome and violation.
\end{proof}

\begin{theorem}[Empty Syndrome Without Faults]
\label{thm:Detector.syndrome_empty_no_faults}
\lean{QEC1.Detector.syndrome_empty_no_faults}
\leanok
\uses{def:Detector.syndrome}
The syndrome is empty when there are no faults:
\[
  \operatorname{syndrome}(\{D_i\}, \emptyset) = \emptyset.
\]
\end{theorem}

\begin{proof}
\leanok
\uses{def:Detector.syndrome, thm:Detector.observedParity_no_faults}
We show the filter is empty by checking that for every $i$, the observed parity $\operatorname{observedParity}(D_i, \emptyset) = 0 \neq 1$, using the no-fault observed parity theorem.
\end{proof}

\begin{definition}[Disjoint Union of Detectors]
\label{def:Detector.disjointUnion}
\lean{QEC1.Detector.disjointUnion}
\leanok
\uses{def:Detector}
Given two detectors $D_1, D_2$ with disjoint measurement sets and identical ideal outcomes, their \textbf{disjoint union} is the detector with:
\begin{itemize}
  \item measurements $= D_1.\mathrm{measurements} \cup D_2.\mathrm{measurements}$,
  \item ideal outcome $= D_1.\mathrm{idealOutcome}$.
\end{itemize}
The detector constraint holds because the sum over the union splits into $\sum_{D_1} + \sum_{D_2}$, both of which are $0$.
\end{definition}

\begin{definition}[Empty Detector]
\label{def:Detector.emptyDetector}
\lean{QEC1.Detector.emptyDetector}
\leanok
\uses{def:Detector}
The \textbf{empty detector} has no measurements ($D.\mathrm{measurements} = \emptyset$). The detector constraint is trivially satisfied since the empty sum is $0$.
\end{definition}

\begin{theorem}[Empty Detector Not Violated]
\label{thm:Detector.emptyDetector_not_violated}
\lean{QEC1.Detector.emptyDetector_not_violated}
\leanok
\uses{def:Detector.emptyDetector, def:Detector.isViolated}
The empty detector is never violated:
\[
  \neg\, \operatorname{isViolated}(\operatorname{emptyDetector}, F)
\]
for any fault set $F$.
\end{theorem}

\begin{proof}
\leanok
\uses{def:Detector.isViolated, def:Detector.observedParity, def:Detector.emptyDetector}
Unfolding the definitions of violation, observed parity, and empty detector, the sum over the empty measurement set is $0$, and $0 \neq 1$.
\end{proof}

\begin{definition}[Single-Measurement Detector]
\label{def:Detector.singleMeasurementDetector}
\lean{QEC1.Detector.singleMeasurementDetector}
\leanok
\uses{def:Detector}
A \textbf{single-measurement detector} for measurement $m$ has measurements $= \{m\}$ and ideal outcome constantly $0$. The constraint $\sum_{m' \in \{m\}} 0 = 0$ holds trivially. This detector fires when measurement $m$ is faulted.
\end{definition}

\begin{theorem}[Single-Measurement Violation Characterization]
\label{thm:Detector.singleMeasurement_violated_iff}
\lean{QEC1.Detector.singleMeasurement_violated_iff}
\leanok
\uses{def:Detector.singleMeasurementDetector, def:Detector.isViolated}
A single-measurement detector on $m$ is violated if and only if the measurement $m$ is faulted:
\[
  \operatorname{isViolated}(\operatorname{singleMeasurementDetector}(m), F) \iff \langle m \rangle \in F.
\]
\end{theorem}

\begin{proof}
\leanok
\uses{thm:Detector.isViolated_iff_flipParity, def:Detector.flipParity, def:Detector.singleMeasurementDetector}
We rewrite using the flip parity characterization and unfold the definitions. The flip parity over the singleton $\{m\}$ is $\mathbf{1}_{\langle m \rangle \in F}$. For the forward direction, if this equals $1$ and $\langle m \rangle \notin F$, we obtain a contradiction since the indicator would be $0$. For the backward direction, if $\langle m \rangle \in F$ then the indicator is $1$.
\end{proof}

\begin{theorem}[Violation Under Spacetime Faults]
\label{thm:Detector.isViolated_spacetimeFault}
\lean{QEC1.Detector.isViolated_spacetimeFault}
\leanok
\uses{def:Detector.isViolated, def:SpacetimeFault, def:SpaceFault, def:Detector.observedParity}
The violation status of a detector under a spacetime fault $F$ is determined by the time-fault component. Space-faults change the quantum state but the detector constraint is purely about measurement outcome flips:
\[
  \operatorname{isViolated}(D, F.\mathrm{timeFaults}) \iff \operatorname{observedParity}(D, F.\mathrm{timeFaults}) = 1.
\]
\end{theorem}

\begin{proof}
\leanok
\uses{def:Detector.isViolated}
This holds by reflexivity, as it is the definition of $\operatorname{isViolated}$.
\end{proof}

\begin{definition}[Detector Weight]
\label{def:Detector.detectorWeight}
\lean{QEC1.Detector.detectorWeight}
\leanok
\uses{def:Detector}
The \textbf{weight} of a detector $D$ is the number of measurements it contains:
\[
  \operatorname{detectorWeight}(D) = |D.\mathrm{measurements}|.
\]
\end{definition}

\begin{theorem}[Repeated Measurement Weight]
\label{thm:Detector.repeatedMeasurement_weight}
\lean{QEC1.Detector.repeatedMeasurement_weight}
\leanok
\uses{def:repeatedMeasurementDetector, def:Detector.detectorWeight}
The repeated measurement detector has weight $2$:
\[
  \operatorname{detectorWeight}(\operatorname{repeatedMeasurementDetector}(m_1, m_2)) = 2.
\]
\end{theorem}

\begin{proof}
\leanok
\uses{def:Detector.detectorWeight, def:repeatedMeasurementDetector}
Unfolding the definitions of detector weight and repeated measurement detector, the measurement set is $\{m_1, m_2\}$ with $m_1 \neq m_2$, so its cardinality is $2$.
\end{proof}

\begin{theorem}[Initialization-Measurement Weight]
\label{thm:Detector.initAndMeasure_weight}
\lean{QEC1.Detector.initAndMeasure_weight}
\leanok
\uses{def:initAndMeasureDetector, def:Detector.detectorWeight}
The initialization-measurement detector has weight $2$:
\[
  \operatorname{detectorWeight}(\operatorname{initAndMeasureDetector}(m_{\mathrm{init}}, m_{\mathrm{meas}})) = 2.
\]
\end{theorem}

\begin{proof}
\leanok
\uses{def:Detector.detectorWeight, def:initAndMeasureDetector}
Unfolding the definitions, the measurement set is $\{m_{\mathrm{init}}, m_{\mathrm{meas}}\}$ with $m_{\mathrm{init}} \neq m_{\mathrm{meas}}$, so its cardinality is $2$.
\end{proof}

\begin{theorem}[Empty Detector Weight]
\label{thm:Detector.emptyDetector_weight}
\lean{QEC1.Detector.emptyDetector_weight}
\leanok
\uses{def:Detector.emptyDetector, def:Detector.detectorWeight}
The empty detector has weight $0$:
\[
  \operatorname{detectorWeight}(\operatorname{emptyDetector}) = 0.
\]
\end{theorem}

\begin{proof}
\leanok
\uses{def:Detector.detectorWeight, def:Detector.emptyDetector}
Unfolding the definitions, the measurement set is $\emptyset$, which has cardinality $0$.
\end{proof}

\begin{theorem}[Single-Measurement Weight]
\label{thm:Detector.singleMeasurement_weight}
\lean{QEC1.Detector.singleMeasurement_weight}
\leanok
\uses{def:Detector.singleMeasurementDetector, def:Detector.detectorWeight}
The single-measurement detector has weight $1$:
\[
  \operatorname{detectorWeight}(\operatorname{singleMeasurementDetector}(m)) = 1.
\]
\end{theorem}

\begin{proof}
\leanok
\uses{def:Detector.detectorWeight, def:Detector.singleMeasurementDetector}
Unfolding the definitions, the measurement set is $\{m\}$, which has cardinality $1$.
\end{proof}

%--- Def_9: Syndrome ---
\chapter{Def 9: Syndrome}

The \textbf{syndrome} of a spacetime fault (Def~7) is the set of detectors (Def~8) that are violated by the fault. A detector is violated if the product of its measurement outcomes is $-1$ instead of the expected $+1$. Equivalently, working over $\mathbb{Z}_2$, if detectors are indexed as $D_1, \ldots, D_m$, the syndrome can be identified with a binary vector $s \in \mathbb{Z}_2^m$ where $s_j = 1$ if and only if detector $D_j$ is violated.

\begin{definition}[Syndrome Fault Set]
\label{def:syndromeFault}
\lean{QEC1.syndromeFault}
\leanok
\uses{def:SpacetimeFault, def:Detector.isViolated}
Let $I$ be a finite type indexing a collection of detectors $\{D_i\}_{i \in I}$, and let $F$ be a spacetime fault. The \emph{syndrome fault set} of $F$ with respect to the detectors is the finset of detector indices that are violated by $F$:
\[
\operatorname{syndromeFault}(\{D_i\}, F) \;=\; \{ i \in I \mid D_i \text{ is violated by } F.\mathrm{timeFaults} \}.
\]
Since detectors depend only on measurement outcomes and only time-faults flip outcomes, the syndrome is determined by the time-fault component of~$F$.
\end{definition}

\begin{definition}[Syndrome Vector]
\label{def:syndromeVec}
\lean{QEC1.syndromeVec}
\leanok
\uses{def:SpacetimeFault, def:Detector.isViolated}
The \emph{syndrome vector} of a spacetime fault $F$ with respect to detectors $\{D_i\}_{i \in I}$ is the binary vector $s \colon I \to \mathbb{Z}_2$ defined by
\[
s(i) \;=\; \begin{cases} 1 & \text{if } D_i \text{ is violated by } F.\mathrm{timeFaults}, \\ 0 & \text{otherwise.} \end{cases}
\]
This is the $\mathbb{Z}_2^m$ representation of the syndrome.
\end{definition}

\begin{theorem}[Syndrome Membership Characterization]
\label{thm:mem_syndromeFault_iff}
\lean{QEC1.mem_syndromeFault_iff}
\leanok
\uses{def:syndromeFault, def:SpacetimeFault, def:Detector.isViolated}
A detector index $i$ is in the syndrome if and only if detector $D_i$ is violated by the spacetime fault:
\[
i \in \operatorname{syndromeFault}(\{D_k\}, F) \;\Leftrightarrow\; D_i.\mathrm{isViolated}(F.\mathrm{timeFaults}).
\]
\end{theorem}

\begin{proof}
\leanok
\uses{def:syndromeFault}
By the definition of $\operatorname{syndromeFault}$ as a filter over the universal finset, the result follows by simplification.
\end{proof}

\begin{theorem}[Syndrome Vector Equals One Iff Violated]
\label{thm:syndromeVec_eq_one_iff}
\lean{QEC1.syndromeVec_eq_one_iff}
\leanok
\uses{def:syndromeVec, def:SpacetimeFault, def:Detector.isViolated}
The syndrome vector satisfies $s(i) = 1$ if and only if detector $D_i$ is violated:
\[
\operatorname{syndromeVec}(\{D_k\}, F)(i) = 1 \;\Leftrightarrow\; D_i.\mathrm{isViolated}(F.\mathrm{timeFaults}).
\]
\end{theorem}

\begin{proof}
\leanok
\uses{def:syndromeVec}
Unfolding the definition of $\operatorname{syndromeVec}$, we split on the conditional. If the detector is violated, then $s(i) = 1$ by definition, and the forward direction is trivially satisfied. If the detector is not violated, then $s(i) = 0$, so $s(i) = 1$ is contradicted by $0 \neq 1$, and the backward direction is contradicted by the assumption that the detector is not violated.
\end{proof}

\begin{theorem}[Syndrome Vector Equals Zero Iff Not Violated]
\label{thm:syndromeVec_eq_zero_iff}
\lean{QEC1.syndromeVec_eq_zero_iff}
\leanok
\uses{def:syndromeVec, def:SpacetimeFault, def:Detector.isViolated}
The syndrome vector satisfies $s(i) = 0$ if and only if detector $D_i$ is not violated:
\[
\operatorname{syndromeVec}(\{D_k\}, F)(i) = 0 \;\Leftrightarrow\; \neg\, D_i.\mathrm{isViolated}(F.\mathrm{timeFaults}).
\]
\end{theorem}

\begin{proof}
\leanok
\uses{def:syndromeVec}
Unfolding the definition of $\operatorname{syndromeVec}$ and splitting on the conditional, in the case where the detector is violated we have $s(i) = 1$ and must show $1 = 0 \Leftrightarrow \bot$, which follows by simplification. In the case where the detector is not violated, $s(i) = 0$ and the equivalence follows by simplification.
\end{proof}

\begin{theorem}[Spacetime Syndrome Equals Detector Syndrome]
\label{thm:syndromeFault_eq_syndrome}
\lean{QEC1.syndromeFault_eq_syndrome}
\leanok
\uses{def:syndromeFault, def:Detector.syndrome, def:SpacetimeFault}
The spacetime syndrome equals the Def~8 syndrome applied to the time-fault component. This captures the key fact that detectors depend only on measurement outcomes, so only time-faults affect the syndrome:
\[
\operatorname{syndromeFault}(\{D_k\}, F) \;=\; \operatorname{Detector.syndrome}(\{D_k\}, F.\mathrm{timeFaults}).
\]
\end{theorem}

\begin{proof}
\leanok
\uses{def:syndromeFault, def:Detector.syndrome}
By extensionality on the index $i$, we simplify using the definitions of $\operatorname{syndromeFault}$, $\operatorname{Detector.syndrome}$, and $\operatorname{Detector.isViolated}$. Both sides reduce to filtering the universal set by the same predicate.
\end{proof}

\begin{theorem}[Empty Fault Has Empty Syndrome]
\label{thm:syndromeFault_empty}
\lean{QEC1.syndromeFault_empty}
\leanok
\uses{def:syndromeFault, def:SpacetimeFault.empty, def:SpaceFault}
The syndrome is empty for the fault-free spacetime configuration:
\[
\operatorname{syndromeFault}(\{D_k\}, \mathrm{empty}) = \emptyset.
\]
\end{theorem}

\begin{proof}
\leanok
\uses{def:syndromeFault, def:SpacetimeFault.empty, thm:Detector.not_isViolated_no_faults}
Simplifying using the definition of $\operatorname{syndromeFault}$, the filter is empty if no detector is violated. For each detector index $i$, the result follows from the fact that no detector is violated in the absence of faults (Detector.not\_isViolated\_no\_faults).
\end{proof}

\begin{theorem}[Zero Fault Has Zero Syndrome Vector]
\label{thm:syndromeVec_zero}
\lean{QEC1.syndromeVec_zero}
\leanok
\uses{def:syndromeVec, def:SpacetimeFault.empty, def:SpaceFault}
The syndrome vector is zero for the fault-free configuration:
\[
\operatorname{syndromeVec}(\{D_k\}, \mathrm{empty}) = 0.
\]
\end{theorem}

\begin{proof}
\leanok
\uses{def:syndromeVec, def:SpacetimeFault.empty, thm:Detector.not_isViolated_no_faults}
By extensionality on the index $i$, we simplify the definition of $\operatorname{syndromeVec}$ at the empty fault. We establish that no detector is violated in the absence of faults using Detector.not\_isViolated\_no\_faults. Since the detector is not violated, the conditional evaluates to $0$, which equals the zero function applied at $i$.
\end{proof}

\begin{theorem}[Syndrome Finset Equals Support of Syndrome Vector]
\label{thm:syndromeFault_eq_support}
\lean{QEC1.syndromeFault_eq_support}
\leanok
\uses{def:syndromeFault, def:syndromeVec, def:SpacetimeFault}
The syndrome finset is exactly the support of the syndrome vector:
\[
\operatorname{syndromeFault}(\{D_k\}, F) = \{ i \in I \mid \operatorname{syndromeVec}(\{D_k\}, F)(i) = 1 \}.
\]
\end{theorem}

\begin{proof}
\leanok
\uses{thm:mem_syndromeFault_iff, thm:syndromeVec_eq_one_iff}
By extensionality on the index $i$, we simplify using the membership characterization of the syndrome finset and the characterization of the syndrome vector being $1$, which both reduce to the detector being violated.
\end{proof}

\begin{theorem}[Syndrome Membership Via Vector]
\label{thm:mem_syndromeFault_iff_vec}
\lean{QEC1.mem_syndromeFault_iff_vec}
\leanok
\uses{def:syndromeFault, def:syndromeVec, def:SpacetimeFault}
Membership in the syndrome finset is equivalent to the syndrome vector entry being $1$:
\[
i \in \operatorname{syndromeFault}(\{D_k\}, F) \;\Leftrightarrow\; \operatorname{syndromeVec}(\{D_k\}, F)(i) = 1.
\]
\end{theorem}

\begin{proof}
\leanok
\uses{thm:mem_syndromeFault_iff, thm:syndromeVec_eq_one_iff}
Rewriting using the membership characterization $\operatorname{mem\_syndromeFault\_iff}$ and the syndrome vector characterization $\operatorname{syndromeVec\_eq\_one\_iff}$, the result follows directly.
\end{proof}

\begin{theorem}[Non-Membership Via Vector]
\label{thm:not_mem_syndromeFault_iff_vec}
\lean{QEC1.not_mem_syndromeFault_iff_vec}
\leanok
\uses{def:syndromeFault, def:syndromeVec, def:SpacetimeFault}
Non-membership in the syndrome finset is equivalent to the syndrome vector entry being $0$:
\[
i \notin \operatorname{syndromeFault}(\{D_k\}, F) \;\Leftrightarrow\; \operatorname{syndromeVec}(\{D_k\}, F)(i) = 0.
\]
\end{theorem}

\begin{proof}
\leanok
\uses{thm:mem_syndromeFault_iff, thm:syndromeVec_eq_zero_iff}
Rewriting using the membership characterization and the zero characterization of the syndrome vector, the result follows directly.
\end{proof}

\begin{theorem}[Syndrome Depends Only on Time-Faults]
\label{thm:syndromeFault_depends_on_timeFaults}
\lean{QEC1.syndromeFault_depends_on_timeFaults}
\leanok
\uses{def:syndromeFault, def:SpacetimeFault}
Space-faults do not affect the syndrome: two spacetime faults with the same time-fault component produce the same syndrome. If $F_1.\mathrm{timeFaults} = F_2.\mathrm{timeFaults}$, then
\[
\operatorname{syndromeFault}(\{D_k\}, F_1) = \operatorname{syndromeFault}(\{D_k\}, F_2).
\]
\end{theorem}

\begin{proof}
\leanok
\uses{def:syndromeFault}
By extensionality on the index $i$, we simplify using the definitions of $\operatorname{syndromeFault}$, $\operatorname{Detector.isViolated}$, and $\operatorname{Detector.observedParity}$, together with the hypothesis $h$ that the time-fault components are equal. Both sides reduce to the same expression.
\end{proof}

\begin{theorem}[Syndrome Vector Depends Only on Time-Faults]
\label{thm:syndromeVec_depends_on_timeFaults}
\lean{QEC1.syndromeVec_depends_on_timeFaults}
\leanok
\uses{def:syndromeVec, def:SpacetimeFault}
The syndrome vector is the same for faults with identical time-fault components. If $F_1.\mathrm{timeFaults} = F_2.\mathrm{timeFaults}$, then
\[
\operatorname{syndromeVec}(\{D_k\}, F_1) = \operatorname{syndromeVec}(\{D_k\}, F_2).
\]
\end{theorem}

\begin{proof}
\leanok
\uses{def:syndromeVec}
By extensionality on the index $i$, we simplify using the definitions of $\operatorname{syndromeVec}$, $\operatorname{Detector.isViolated}$, and $\operatorname{Detector.observedParity}$, together with the hypothesis that the time-fault components are equal.
\end{proof}

\begin{theorem}[Pure Space-Fault Has Empty Syndrome]
\label{thm:syndromeFault_pureSpace}
\lean{QEC1.syndromeFault_pureSpace}
\leanok
\uses{def:syndromeFault, def:SpacetimeFault.isPureSpace, def:SpaceFault}
A pure space-fault (no time-faults) has empty syndrome:
\[
F.\mathrm{isPureSpace} \;\Rightarrow\; \operatorname{syndromeFault}(\{D_k\}, F) = \emptyset.
\]
\end{theorem}

\begin{proof}
\leanok
\uses{def:syndromeFault, def:SpacetimeFault.isPureSpace, lem:observedOutcome_no_fault}
From the hypothesis that $F$ is a pure space-fault, we obtain that $F.\mathrm{timeFaults} = \emptyset$. We then show the filter is empty: for each detector index $i$, unfolding $\operatorname{Detector.isViolated}$ and rewriting the time-faults as empty, the observed parity with no faults is $0$ by Detector.observedParity\_no\_faults. Since $0 \neq 1$, the detector is not violated.
\end{proof}

\begin{theorem}[Pure Space-Fault Has Zero Syndrome Vector]
\label{thm:syndromeVec_pureSpace}
\lean{QEC1.syndromeVec_pureSpace}
\leanok
\uses{def:syndromeVec, def:SpacetimeFault.isPureSpace, def:SpaceFault}
A pure space-fault has zero syndrome vector:
\[
F.\mathrm{isPureSpace} \;\Rightarrow\; \operatorname{syndromeVec}(\{D_k\}, F) = 0.
\]
\end{theorem}

\begin{proof}
\leanok
\uses{def:syndromeVec, def:SpacetimeFault.isPureSpace, lem:observedOutcome_no_fault}
From the hypothesis that $F$ is a pure space-fault, we obtain $F.\mathrm{timeFaults} = \emptyset$. By extensionality on $i$, we simplify the definition of $\operatorname{syndromeVec}$. We establish that the detector is not violated by unfolding $\operatorname{Detector.isViolated}$, rewriting with the empty time-faults, and using Detector.observedParity\_no\_faults to show the parity is $0 \neq 1$. Since the detector is not violated, the conditional evaluates to $0$.
\end{proof}

\begin{theorem}[Syndrome Cardinality]
\label{thm:syndromeFault_card}
\lean{QEC1.syndromeFault_card}
\leanok
\uses{def:syndromeFault, def:SpacetimeFault}
The number of violated detectors equals the cardinality of the syndrome:
\[
|\operatorname{syndromeFault}(\{D_k\}, F)| = |\{ i \in I \mid D_i.\mathrm{isViolated}(F.\mathrm{timeFaults}) \}|.
\]
\end{theorem}

\begin{proof}
\leanok
\uses{def:syndromeFault}
This holds by reflexivity, as both sides are definitionally equal.
\end{proof}

\begin{theorem}[Syndrome Vector Sum]
\label{thm:syndromeVec_sum}
\lean{QEC1.syndromeVec_sum}
\leanok
\uses{def:syndromeVec, def:syndromeFault, def:SpacetimeFault}
The sum of the syndrome vector counts the number of violated detectors modulo $2$:
\[
\sum_{i \in I} \operatorname{syndromeVec}(\{D_k\}, F)(i) = |\operatorname{syndromeFault}(\{D_k\}, F)| \pmod{2}.
\]
\end{theorem}

\begin{proof}
\leanok
\uses{def:syndromeVec, def:syndromeFault}
Simplifying the definitions of $\operatorname{syndromeVec}$ and $\operatorname{syndromeFault}$, we rewrite the sum using the identity for summing an if-then-else function: the sum splits into a sum of $1$'s over the violated detectors (which equals the cardinality of the filtered set) plus a sum of $0$'s over the non-violated detectors. Adding zero and casting the cardinality to $\mathbb{Z}_2$ yields the result.
\end{proof}

\begin{theorem}[Syndrome Via Flip Parity]
\label{thm:syndromeFault_eq_flipParity_filter}
\lean{QEC1.syndromeFault_eq_flipParity_filter}
\leanok
\uses{def:syndromeFault, def:Detector.flipParity, def:SpacetimeFault}
The syndrome is equivalently described using flip parities of detectors:
\[
\operatorname{syndromeFault}(\{D_k\}, F) = \{ i \in I \mid D_i.\mathrm{flipParity}(F.\mathrm{timeFaults}) = 1 \}.
\]
\end{theorem}

\begin{proof}
\leanok
\uses{def:syndromeFault, thm:Detector.observedParity_eq_flipParity}
By extensionality on $i$, we simplify using the definition of $\operatorname{syndromeFault}$, $\operatorname{Detector.isViolated}$, and the equivalence between observed parity and flip parity (Detector.observedParity\_eq\_flipParity).
\end{proof}

\begin{theorem}[Syndrome Vector Equals Flip Parity]
\label{thm:syndromeVec_eq_flipParity}
\lean{QEC1.syndromeVec_eq_flipParity}
\leanok
\uses{def:syndromeVec, def:Detector.flipParity, def:SpacetimeFault}
The syndrome vector can be computed via flip parities:
\[
\operatorname{syndromeVec}(\{D_k\}, F)(i) = D_i.\mathrm{flipParity}(F.\mathrm{timeFaults}).
\]
\end{theorem}

\begin{proof}
\leanok
\uses{def:syndromeVec, thm:Detector.observedParity_eq_flipParity}
Simplifying the definition of $\operatorname{syndromeVec}$ using $\operatorname{Detector.isViolated}$ and the equivalence $\operatorname{Detector.observedParity\_eq\_flipParity}$, we split on the conditional. If the flip parity equals $1$, then by symmetry the result is $1$. If not, we push the negation and case-split on whether the flip parity is $0$ or $1$ (since every element of $\mathbb{Z}_2$ is $0$ or $1$). In the $0$ case the result follows by symmetry; the $1$ case leads to a contradiction.
\end{proof}

\begin{theorem}[Space-Fault Embedding Has Empty Syndrome]
\label{thm:syndromeFault_ofSpaceFault}
\lean{QEC1.syndromeFault_ofSpaceFault}
\leanok
\uses{def:syndromeFault, def:SpacetimeFault.ofSpaceFault, def:SpaceFault}
A pure space-fault embedded as a spacetime fault has empty syndrome, since space-faults do not violate detectors:
\[
\operatorname{syndromeFault}(\{D_k\}, \operatorname{ofSpaceFault}(f)) = \emptyset.
\]
\end{theorem}

\begin{proof}
\leanok
\uses{thm:syndromeFault_pureSpace, lem:SpacetimeFault.isPureSpace_ofSpaceFault}
We apply the theorem that pure space-faults have empty syndrome ($\operatorname{syndromeFault\_pureSpace}$). The hypothesis that $\operatorname{ofSpaceFault}(f)$ is a pure space-fault follows from $\operatorname{SpacetimeFault.isPureSpace\_ofSpaceFault}$.
\end{proof}

\begin{theorem}[Disjoint Detector Not Violated]
\label{thm:not_isViolated_disjoint}
\lean{QEC1.not_isViolated_disjoint}
\leanok
\uses{def:Detector.isViolated, def:Detector.flipParity}
A detector not containing any faulted measurement is not violated. If for every measurement $m$ in $D.\mathrm{measurements}$, the time-fault $\langle m \rangle$ is not in the fault set, then $D$ is not violated:
\[
(\forall m \in D.\mathrm{measurements},\; \langle m \rangle \notin \mathrm{faults}) \;\Rightarrow\; \neg\, D.\mathrm{isViolated}(\mathrm{faults}).
\]
\end{theorem}

\begin{proof}
\leanok
\uses{thm:Detector.isViolated_iff_flipParity}
Rewriting $\operatorname{isViolated}$ in terms of $\operatorname{flipParity}$ using $\operatorname{Detector.isViolated\_iff\_flipParity}$, we show the flip parity is $0$. Unfolding $\operatorname{Detector.flipParity}$, the sum over measurements $m$ in $D.\mathrm{measurements}$ reduces to $0$ because each term is $0$: for each $m$, since $\langle m \rangle \notin \mathrm{faults}$, the indicator is $0$ by simplification. Since the flip parity is $0$ and $0 \neq 1$, the detector is not violated.
\end{proof}

\begin{theorem}[Syndrome Equality Characterization]
\label{thm:syndromeFault_eq_iff}
\lean{QEC1.syndromeFault_eq_iff}
\leanok
\uses{def:syndromeFault, def:SpacetimeFault, def:Detector.isViolated}
The syndrome captures the error information revealed by detectors: two faults produce the same syndrome if and only if they violate the same detectors:
\[
\operatorname{syndromeFault}(\{D_k\}, F_1) = \operatorname{syndromeFault}(\{D_k\}, F_2) \;\Leftrightarrow\; \forall i,\; D_i.\mathrm{isViolated}(F_1.\mathrm{timeFaults}) \leftrightarrow D_i.\mathrm{isViolated}(F_2.\mathrm{timeFaults}).
\]
\end{theorem}

\begin{proof}
\leanok
\uses{thm:mem_syndromeFault_iff, def:syndromeFault}
We prove both directions. For the forward direction, assume the syndrome sets are equal. For each $i$, we use the membership characterization for both $F_1$ and $F_2$. Rewriting the hypothesis of equal sets into the membership for $F_1$, we obtain the biconditional: if $D_i$ is violated by $F_1$ then the membership in the (rewritten) syndrome for $F_2$ gives violation by $F_2$, and vice versa. For the reverse direction, assume the pointwise biconditional. By extensionality on $i$, we simplify the definition of $\operatorname{syndromeFault}$ and apply the hypothesis for index $i$.
\end{proof}

\begin{theorem}[Syndrome Vector Equality Characterization]
\label{thm:syndromeVec_eq_iff}
\lean{QEC1.syndromeVec_eq_iff}
\leanok
\uses{def:syndromeVec, def:syndromeFault, def:SpacetimeFault}
Two faults have the same syndrome vector if and only if they have the same syndrome finset:
\[
\operatorname{syndromeVec}(\{D_k\}, F_1) = \operatorname{syndromeVec}(\{D_k\}, F_2) \;\Leftrightarrow\; \operatorname{syndromeFault}(\{D_k\}, F_1) = \operatorname{syndromeFault}(\{D_k\}, F_2).
\]
\end{theorem}

\begin{proof}
\leanok
\uses{thm:mem_syndromeFault_iff, thm:syndromeVec_eq_one_iff}
We prove both directions. For the forward direction, assume the syndrome vectors are equal. By extensionality on the syndrome finsets, for each $i$ we extract the pointwise equality $\operatorname{syndromeVec}(\{D_k\}, F_1)(i) = \operatorname{syndromeVec}(\{D_k\}, F_2)(i)$ using $\operatorname{congr\_fun}$. Then we simplify using the membership characterization and the syndrome vector characterization. For the reverse direction, assume the syndrome finsets are equal. By extensionality on $i$, we establish that membership in the two syndromes is equivalent by rewriting with the hypothesis. We then rewrite membership in terms of the syndrome vector being $1$. Case-splitting on whether each syndrome vector entry is $0$ or $1$ (using the fact that every element of $\mathbb{Z}_2$ is $0$ or $1$), the matching cases are immediate ($0 = 0$ or $1 = 1$), and the mismatched cases lead to contradictions via $0 \neq 1$.
\end{proof}

%--- Def_10: FaultTolerantGaugingProcedure ---
\chapter{Def 10: Fault-Tolerant Gauging Procedure}

\begin{definition}[Gauging Phase]
\label{def:GaugingPhase}
\lean{QEC1.GaugingPhase}
\leanok

The three phases of the fault-tolerant gauging procedure form an inductive type with three constructors:
\begin{enumerate}
\item \texttt{preDeformation}: Phase~1, in which the original stabilizer checks are measured.
\item \texttt{deformedCode}: Phase~2, in which the deformed code checks (Gauss's law, flux, and deformed original checks) are measured.
\item \texttt{postDeformation}: Phase~3, in which the edge qubits are ungauged and the original checks are resumed.
\end{enumerate}
The type has decidable equality and is a finite type with exactly three elements.
\end{definition}

\begin{definition}[Pre-Deformation Measurement]
\label{def:PreDeformationMeasurement}
\lean{QEC1.PreDeformationMeasurement}
\leanok

A \emph{Phase~1 measurement} label for a code with check index set $J$ and code distance $d$ is a pair
\[
(\mathtt{checkIdx} : J,\; \mathtt{round} : \operatorname{Fin}(d)),
\]
recording which original stabilizer check $j \in J$ is measured in which round $r \in \{0, \ldots, d-1\}$.
\end{definition}

\begin{definition}[Deformed Code Measurement]
\label{def:DeformedCodeMeasurement}
\lean{QEC1.DeformedCodeMeasurement}
\leanok

A \emph{Phase~2 measurement} label for vertex set $V$, cycle set $C$, check set $J$, and code distance $d$ is one of:
\begin{enumerate}
\item $\mathtt{gaussLaw}(v, r)$: Gauss's law measurement at vertex $v \in V$ in round $r \in \operatorname{Fin}(d)$.
\item $\mathtt{flux}(p, r)$: Flux measurement for cycle $p \in C$ in round $r \in \operatorname{Fin}(d)$.
\item $\mathtt{deformed}(j, r)$: Deformed original check $j \in J$ in round $r \in \operatorname{Fin}(d)$.
\end{enumerate}
\end{definition}

\begin{definition}[Post-Deformation Measurement]
\label{def:PostDeformationMeasurement}
\lean{QEC1.PostDeformationMeasurement}
\leanok

A \emph{Phase~3 measurement} label for check set $J$, edge set $E$, and code distance $d$ is one of:
\begin{enumerate}
\item $\mathtt{edgeZ}(e)$: The $Z_e$ measurement on edge qubit $e \in E$ to ungauge.
\item $\mathtt{originalCheck}(j, r)$: Original check $j \in J$ measured in round $r \in \operatorname{Fin}(d)$.
\end{enumerate}
\end{definition}

\begin{definition}[Edge Initialization]
\label{def:EdgeInitialization}
\lean{QEC1.EdgeInitialization}
\leanok

An \emph{edge initialization} label for edge set $E$ is a structure recording a single edge $e \in E$, representing initialization of the edge qubit in the $|0\rangle$ state (treated as a measurement per Definition~7).
\end{definition}

\begin{definition}[FT Gauging Measurement Labels]
\label{def:FTGaugingMeasurement}
\lean{QEC1.FTGaugingMeasurement}
\leanok
\uses{def:PreDeformationMeasurement, def:DeformedCodeMeasurement, def:PostDeformationMeasurement, def:EdgeInitialization}
The type of \emph{all measurement labels} across the entire fault-tolerant gauging procedure is the inductive type with four constructors:
\begin{enumerate}
\item $\mathtt{phase1}(m)$: A Phase~1 measurement $m$.
\item $\mathtt{edgeInit}(\mathit{init})$: An edge initialization event.
\item $\mathtt{phase2}(m)$: A Phase~2 measurement $m$.
\item $\mathtt{phase3}(m)$: A Phase~3 measurement $m$.
\end{enumerate}
\end{definition}

\begin{definition}[Phase Assignment Function]
\label{def:phaseOf}
\lean{QEC1.phaseOf}
\leanok
\uses{def:GaugingPhase}
Given code distance $d$ and a time step $t \in \mathbb{N}$, the function $\operatorname{phaseOf}(d, t)$ determines which phase $t$ belongs to:
\[
\operatorname{phaseOf}(d, t) = \begin{cases}
\text{preDeformation} & \text{if } t < d, \\
\text{deformedCode} & \text{if } d \le t < 2d, \\
\text{postDeformation} & \text{if } 2d \le t.
\end{cases}
\]
\end{definition}

\begin{lemma}[Phase Of -- Pre-Deformation]
\label{lem:phaseOf_preDeformation}
\lean{QEC1.phaseOf_preDeformation}
\leanok
\uses{def:phaseOf, def:GaugingPhase}
If $t < d$, then $\operatorname{phaseOf}(d, t) = \text{preDeformation}$.
\end{lemma}
\begin{proof}
\leanok
\uses{def:phaseOf}
Unfolding the definition of $\operatorname{phaseOf}$ and simplifying with the hypothesis $t < d$ yields the result directly.
\end{proof}

\begin{lemma}[Phase Of -- Deformed Code]
\label{lem:phaseOf_deformedCode}
\lean{QEC1.phaseOf_deformedCode}
\leanok
\uses{def:phaseOf, def:GaugingPhase}
If $d \le t$ and $t < 2d$, then $\operatorname{phaseOf}(d, t) = \text{deformedCode}$.
\end{lemma}
\begin{proof}
\leanok
\uses{def:phaseOf}
We simplify using the definition of $\operatorname{phaseOf}$. Since $d \le t$, we have $\neg(t < d)$ by integer arithmetic, so the first branch is false. Since $t < 2d$, the second branch is true, giving the result.
\end{proof}

\begin{lemma}[Phase Of -- Post-Deformation]
\label{lem:phaseOf_postDeformation}
\lean{QEC1.phaseOf_postDeformation}
\leanok
\uses{def:phaseOf, def:GaugingPhase}
If $2d \le t$, then $\operatorname{phaseOf}(d, t) = \text{postDeformation}$.
\end{lemma}
\begin{proof}
\leanok
\uses{def:phaseOf}
We simplify using the definition of $\operatorname{phaseOf}$. Since $2d \le t$, we have $\neg(t < d)$ and $\neg(t < 2d)$ by integer arithmetic, so both conditional branches are false and the result is $\text{postDeformation}$.
\end{proof}

\begin{definition}[Fault-Tolerant Gauging Procedure]
\label{def:FaultTolerantGaugingProcedure}
\lean{QEC1.FaultTolerantGaugingProcedure}
\leanok
\uses{def:PauliOp.PauliCommute, def:DeformedCode.DeformedCodeData}
The \emph{fault-tolerant gauging measurement procedure} for measuring a logical operator $L$ in an $[\![n,k,d]\!]$ stabilizer code using a connected graph $G = (V, E)$ with cycle set $C$ and check set $J$ is a structure consisting of:
\begin{enumerate}
\item $d : \mathbb{N}$, the code distance (number of rounds per phase), with $d \ge 1$.
\item A proof that the original stabilizer code checks pairwise commute: for all $i, j \in J$, $\operatorname{PauliCommute}(\mathit{checks}(i), \mathit{checks}(j))$.
\item A gauging input specifying the base vertex and connectivity data.
\item Deformed code data: edge-paths satisfying boundary conditions.
\item A cycle parity condition: for each cycle $c \in C$ and vertex $v \in V$, the number of edges in the cycle incident to $v$ is even.
\end{enumerate}
\end{definition}

\begin{definition}[Phase 2 Start Time]
\label{def:FaultTolerantGaugingProcedure.phase2Start}
\lean{QEC1.FaultTolerantGaugingProcedure.phase2Start}
\leanok
\uses{def:FaultTolerantGaugingProcedure}
Phase~2 begins at time $d$.
\end{definition}

\begin{definition}[Phase 3 Start Time]
\label{def:FaultTolerantGaugingProcedure.phase3Start}
\lean{QEC1.FaultTolerantGaugingProcedure.phase3Start}
\leanok
\uses{def:FaultTolerantGaugingProcedure}
Phase~3 begins at time $2d$.
\end{definition}

\begin{definition}[Procedure End Time]
\label{def:FaultTolerantGaugingProcedure.procedureEnd}
\lean{QEC1.FaultTolerantGaugingProcedure.procedureEnd}
\leanok
\uses{def:FaultTolerantGaugingProcedure}
The procedure ends at time $3d$.
\end{definition}

\begin{lemma}[Phase 1 Duration]
\label{lem:FaultTolerantGaugingProcedure.phase1_duration}
\lean{QEC1.FaultTolerantGaugingProcedure.phase1_duration}
\leanok
\uses{def:FaultTolerantGaugingProcedure.phase2Start}
The duration of Phase~1 equals $d$: $\mathtt{phase2Start} = d$.
\end{lemma}
\begin{proof}
\leanok
\uses{def:FaultTolerantGaugingProcedure.phase2Start}
This holds by definition.
\end{proof}

\begin{lemma}[Phase 2 Duration]
\label{lem:FaultTolerantGaugingProcedure.phase2_duration}
\lean{QEC1.FaultTolerantGaugingProcedure.phase2_duration}
\leanok
\uses{def:FaultTolerantGaugingProcedure.phase3Start, def:FaultTolerantGaugingProcedure.phase2Start}
The duration of Phase~2 equals $d$: $\mathtt{phase3Start} - \mathtt{phase2Start} = d$.
\end{lemma}
\begin{proof}
\leanok
\uses{def:FaultTolerantGaugingProcedure.phase3Start, def:FaultTolerantGaugingProcedure.phase2Start}
Unfolding the definitions of $\mathtt{phase3Start} = 2d$ and $\mathtt{phase2Start} = d$, the result $2d - d = d$ follows by integer arithmetic.
\end{proof}

\begin{lemma}[Phase 3 Duration]
\label{lem:FaultTolerantGaugingProcedure.phase3_duration}
\lean{QEC1.FaultTolerantGaugingProcedure.phase3_duration}
\leanok
\uses{def:FaultTolerantGaugingProcedure.procedureEnd, def:FaultTolerantGaugingProcedure.phase3Start}
The duration of Phase~3 equals $d$: $\mathtt{procedureEnd} - \mathtt{phase3Start} = d$.
\end{lemma}
\begin{proof}
\leanok
\uses{def:FaultTolerantGaugingProcedure.procedureEnd, def:FaultTolerantGaugingProcedure.phase3Start}
Unfolding the definitions of $\mathtt{procedureEnd} = 3d$ and $\mathtt{phase3Start} = 2d$, the result $3d - 2d = d$ follows by integer arithmetic.
\end{proof}

\begin{lemma}[Total Duration]
\label{lem:FaultTolerantGaugingProcedure.total_duration}
\lean{QEC1.FaultTolerantGaugingProcedure.total_duration}
\leanok
\uses{def:FaultTolerantGaugingProcedure.procedureEnd}
The total duration of the procedure is $3d$: $\mathtt{procedureEnd} = 3d$.
\end{lemma}
\begin{proof}
\leanok
\uses{def:FaultTolerantGaugingProcedure.procedureEnd}
This holds by definition.
\end{proof}

\begin{lemma}[Phase 2 Start $\le$ Phase 3 Start]
\label{lem:FaultTolerantGaugingProcedure.phase2Start_le_phase3Start}
\lean{QEC1.FaultTolerantGaugingProcedure.phase2Start_le_phase3Start}
\leanok
\uses{def:FaultTolerantGaugingProcedure.phase2Start, def:FaultTolerantGaugingProcedure.phase3Start}
We have $\mathtt{phase2Start} \le \mathtt{phase3Start}$, i.e., $d \le 2d$.
\end{lemma}
\begin{proof}
\leanok
\uses{def:FaultTolerantGaugingProcedure.phase2Start, def:FaultTolerantGaugingProcedure.phase3Start}
Unfolding definitions, $d \le 2d$ follows by integer arithmetic.
\end{proof}

\begin{lemma}[Phase 3 Start $\le$ Procedure End]
\label{lem:FaultTolerantGaugingProcedure.phase3Start_le_procedureEnd}
\lean{QEC1.FaultTolerantGaugingProcedure.phase3Start_le_procedureEnd}
\leanok
\uses{def:FaultTolerantGaugingProcedure.phase3Start, def:FaultTolerantGaugingProcedure.procedureEnd}
We have $\mathtt{phase3Start} \le \mathtt{procedureEnd}$, i.e., $2d \le 3d$.
\end{lemma}
\begin{proof}
\leanok
\uses{def:FaultTolerantGaugingProcedure.phase3Start, def:FaultTolerantGaugingProcedure.procedureEnd}
Unfolding definitions, $2d \le 3d$ follows by integer arithmetic.
\end{proof}

\begin{definition}[Measurement Phase Assignment]
\label{def:FaultTolerantGaugingProcedure.measurementPhase}
\lean{QEC1.FaultTolerantGaugingProcedure.measurementPhase}
\leanok
\uses{def:FaultTolerantGaugingProcedure, def:GaugingPhase, def:FTGaugingMeasurement}
The function $\mathtt{measurementPhase}$ assigns a phase to each measurement label:
\begin{itemize}
\item $\mathtt{phase1}(\_) \mapsto \text{preDeformation}$,
\item $\mathtt{edgeInit}(\_) \mapsto \text{deformedCode}$,
\item $\mathtt{phase2}(\_) \mapsto \text{deformedCode}$,
\item $\mathtt{phase3}(\_) \mapsto \text{postDeformation}$.
\end{itemize}
\end{definition}

\begin{definition}[Measurement Time Assignment]
\label{def:FaultTolerantGaugingProcedure.measurementTime}
\lean{QEC1.FaultTolerantGaugingProcedure.measurementTime}
\leanok
\uses{def:FaultTolerantGaugingProcedure, def:FTGaugingMeasurement}
The function $\mathtt{measurementTime}$ assigns an integer time step to each measurement label:
\begin{itemize}
\item $\mathtt{phase1}(\langle j, r \rangle) \mapsto r$,
\item $\mathtt{edgeInit}(\_) \mapsto d$,
\item $\mathtt{phase2}(\mathtt{gaussLaw}\;v\;r) \mapsto d + r$; similarly for flux and deformed measurements,
\item $\mathtt{phase3}(\mathtt{edgeZ}\;e) \mapsto 2d$; $\mathtt{phase3}(\mathtt{originalCheck}\;j\;r) \mapsto 2d + r$.
\end{itemize}
\end{definition}

\begin{lemma}[Phase 1 Measurement Time Bound]
\label{lem:FaultTolerantGaugingProcedure.measurementTime_phase1_lt}
\lean{QEC1.FaultTolerantGaugingProcedure.measurementTime_phase1_lt}
\leanok
\uses{def:FaultTolerantGaugingProcedure.measurementTime, def:PreDeformationMeasurement}
For any Phase~1 measurement $\mathit{pm}$, $\mathtt{measurementTime}(\mathtt{phase1}(\mathit{pm})) < d$.
\end{lemma}
\begin{proof}
\leanok
\uses{def:FaultTolerantGaugingProcedure.measurementTime}
Unfolding the definition of $\mathtt{measurementTime}$, this reduces to $\mathit{pm}.\mathit{round}.\mathit{val} < d$, which holds because $\mathit{pm}.\mathit{round}$ is an element of $\operatorname{Fin}(d)$.
\end{proof}

\begin{lemma}[Phase 2 Measurement Time Lower Bound]
\label{lem:FaultTolerantGaugingProcedure.measurementTime_phase2_ge}
\lean{QEC1.FaultTolerantGaugingProcedure.measurementTime_phase2_ge}
\leanok
\uses{def:FaultTolerantGaugingProcedure.measurementTime, def:DeformedCodeMeasurement}
For any Phase~2 measurement $\mathit{dm}$, $d \le \mathtt{measurementTime}(\mathtt{phase2}(\mathit{dm}))$.
\end{lemma}
\begin{proof}
\leanok
\uses{def:FaultTolerantGaugingProcedure.measurementTime}
We case-split on the type of Phase~2 measurement (gaussLaw, flux, or deformed). In each case, the measurement time has the form $d + r$ for some $r \ge 0$, so $d \le d + r$ follows by simplification.
\end{proof}

\begin{lemma}[Phase 2 Measurement Time Upper Bound]
\label{lem:FaultTolerantGaugingProcedure.measurementTime_phase2_lt}
\lean{QEC1.FaultTolerantGaugingProcedure.measurementTime_phase2_lt}
\leanok
\uses{def:FaultTolerantGaugingProcedure.measurementTime, def:DeformedCodeMeasurement}
For any Phase~2 measurement $\mathit{dm}$, $\mathtt{measurementTime}(\mathtt{phase2}(\mathit{dm})) < 2d$.
\end{lemma}
\begin{proof}
\leanok
\uses{def:FaultTolerantGaugingProcedure.measurementTime}
We case-split on the type of Phase~2 measurement. In each case, the measurement time is $d + r$ where $r < d$ (since $r \in \operatorname{Fin}(d)$), so $d + r < 2d$ follows by simplification and integer arithmetic.
\end{proof}

\begin{lemma}[Phase 3 Measurement Time Lower Bound]
\label{lem:FaultTolerantGaugingProcedure.measurementTime_phase3_ge}
\lean{QEC1.FaultTolerantGaugingProcedure.measurementTime_phase3_ge}
\leanok
\uses{def:FaultTolerantGaugingProcedure.measurementTime, def:PostDeformationMeasurement}
For any Phase~3 measurement $\mathit{pm}$, $2d \le \mathtt{measurementTime}(\mathtt{phase3}(\mathit{pm}))$.
\end{lemma}
\begin{proof}
\leanok
\uses{def:FaultTolerantGaugingProcedure.measurementTime}
We case-split on the type of Phase~3 measurement (edgeZ or originalCheck). In each case, the measurement time is $2d + r$ for some $r \ge 0$, so $2d \le 2d + r$ follows by simplification.
\end{proof}

\begin{lemma}[Phase 3 Measurement Time Upper Bound]
\label{lem:FaultTolerantGaugingProcedure.measurementTime_phase3_lt}
\lean{QEC1.FaultTolerantGaugingProcedure.measurementTime_phase3_lt}
\leanok
\uses{def:FaultTolerantGaugingProcedure.measurementTime, def:PostDeformationMeasurement}
For any Phase~3 measurement $\mathit{pm}$, $\mathtt{measurementTime}(\mathtt{phase3}(\mathit{pm})) < 3d$.
\end{lemma}
\begin{proof}
\leanok
\uses{def:FaultTolerantGaugingProcedure.measurementTime}
We case-split on the type of Phase~3 measurement. For $\mathtt{edgeZ}(e)$, the time is $2d$ and $2d < 3d$ follows from $d \ge 1$ by integer arithmetic. For $\mathtt{originalCheck}(j, r)$, the time is $2d + r$ where $r < d$, so $2d + r < 3d$ by integer arithmetic.
\end{proof}

\begin{lemma}[Edge Initialization Time]
\label{lem:FaultTolerantGaugingProcedure.measurementTime_edgeInit}
\lean{QEC1.FaultTolerantGaugingProcedure.measurementTime_edgeInit}
\leanok
\uses{def:FaultTolerantGaugingProcedure.measurementTime, def:EdgeInitialization}
For any edge initialization $\mathit{ei}$, $\mathtt{measurementTime}(\mathtt{edgeInit}(\mathit{ei})) = d$.
\end{lemma}
\begin{proof}
\leanok
\uses{def:FaultTolerantGaugingProcedure.measurementTime}
Unfolding the definition of $\mathtt{measurementTime}$, this holds by reflexivity.
\end{proof}

\begin{theorem}[Measurement Time Consistent with Phase]
\label{thm:FaultTolerantGaugingProcedure.measurementTime_consistent_with_phase}
\lean{QEC1.FaultTolerantGaugingProcedure.measurementTime_consistent_with_phase}
\leanok
\uses{def:phaseOf, def:FaultTolerantGaugingProcedure.measurementTime, def:FaultTolerantGaugingProcedure.measurementPhase}
For every measurement label $m$,
\[
\operatorname{phaseOf}(d, \mathtt{measurementTime}(m)) = \mathtt{measurementPhase}(m).
\]
\end{theorem}
\begin{proof}
\leanok
\uses{lem:phaseOf_preDeformation, lem:phaseOf_deformedCode, lem:phaseOf_postDeformation, lem:FaultTolerantGaugingProcedure.measurementTime_phase1_lt, lem:FaultTolerantGaugingProcedure.measurementTime_phase2_ge, lem:FaultTolerantGaugingProcedure.measurementTime_phase2_lt, lem:FaultTolerantGaugingProcedure.measurementTime_phase3_ge}
We case-split on the measurement label $m$:
\begin{itemize}
\item \textbf{Case} $m = \mathtt{phase1}(\mathit{pm})$: We simplify the measurement phase to $\text{preDeformation}$. Since $\mathtt{measurementTime}(\mathtt{phase1}(\mathit{pm})) < d$, we apply the lemma $\mathtt{phaseOf\_preDeformation}$.
\item \textbf{Case} $m = \mathtt{edgeInit}(\mathit{ei})$: We simplify the measurement phase to $\text{deformedCode}$. The measurement time is $d$. We apply $\mathtt{phaseOf\_deformedCode}$ with $d \le d$ (by reflexivity) and $d < 2d$ (since $d \ge 1$, by integer arithmetic).
\item \textbf{Case} $m = \mathtt{phase2}(\mathit{dm})$: We simplify the measurement phase to $\text{deformedCode}$. We apply $\mathtt{phaseOf\_deformedCode}$ using the bounds $d \le \mathtt{measurementTime}$ and $\mathtt{measurementTime} < 2d$.
\item \textbf{Case} $m = \mathtt{phase3}(\mathit{pm})$: We simplify the measurement phase to $\text{postDeformation}$. We apply $\mathtt{phaseOf\_postDeformation}$ using $2d \le \mathtt{measurementTime}$.
\end{itemize}
\end{proof}

\begin{definition}[Phase Predicates]
\label{def:FaultTolerantGaugingProcedure.isPhase1}
\lean{QEC1.FaultTolerantGaugingProcedure.isPhase1}
\leanok
\uses{def:FaultTolerantGaugingProcedure, def:FTGaugingMeasurement}
The Boolean predicate $\mathtt{isPhase1}$ returns true if and only if the measurement label is of the form $\mathtt{phase1}(\_)$.
\end{definition}

\begin{definition}[Phase 2 Predicate]
\label{def:FaultTolerantGaugingProcedure.isPhase2}
\lean{QEC1.FaultTolerantGaugingProcedure.isPhase2}
\leanok
\uses{def:FaultTolerantGaugingProcedure, def:FTGaugingMeasurement}
The Boolean predicate $\mathtt{isPhase2}$ returns true if and only if the measurement label is of the form $\mathtt{phase2}(\_)$ or $\mathtt{edgeInit}(\_)$.
\end{definition}

\begin{definition}[Phase 3 Predicate]
\label{def:FaultTolerantGaugingProcedure.isPhase3}
\lean{QEC1.FaultTolerantGaugingProcedure.isPhase3}
\leanok
\uses{def:FaultTolerantGaugingProcedure, def:FTGaugingMeasurement}
The Boolean predicate $\mathtt{isPhase3}$ returns true if and only if the measurement label is of the form $\mathtt{phase3}(\_)$.
\end{definition}

\begin{definition}[Phase 2 Operator]
\label{def:FaultTolerantGaugingProcedure.phase2Operator}
\lean{QEC1.FaultTolerantGaugingProcedure.phase2Operator}
\leanok
\uses{def:FaultTolerantGaugingProcedure, def:DeformedCodeMeasurement, def:DeformedCode.gaussLawChecks, def:DeformedCode.fluxChecks, def:DeformedCode.deformedOriginalChecks}
The Pauli operator measured for a Phase~2 measurement $\mathit{dm}$ is:
\begin{itemize}
\item $\mathtt{gaussLaw}(v, r) \mapsto \text{gaussLawChecks}(G, v)$,
\item $\mathtt{flux}(p, r) \mapsto \text{fluxChecks}(G, \mathit{cycles}, p)$,
\item $\mathtt{deformed}(j, r) \mapsto \text{deformedOriginalChecks}(G, \mathit{checks}, \mathit{deformedData}, j)$.
\end{itemize}
\end{definition}

\begin{theorem}[Phase 2 Operator is a Deformed Code Check]
\label{thm:FaultTolerantGaugingProcedure.phase2Operator_is_allChecks}
\lean{QEC1.FaultTolerantGaugingProcedure.phase2Operator_is_allChecks}
\leanok
\uses{def:FaultTolerantGaugingProcedure.phase2Operator, def:DeformedCode.CheckIndex, def:DeformedCode.allChecks}
For every Phase~2 measurement $\mathit{dm}$, there exists a check index $\mathit{ci} \in \mathtt{CheckIndex}(V, C, J)$ such that
\[
\mathtt{phase2Operator}(\mathit{dm}) = \mathtt{allChecks}(\mathit{ci}).
\]
\end{theorem}
\begin{proof}
\leanok
\uses{def:FaultTolerantGaugingProcedure.phase2Operator, def:DeformedCode.allChecks}
We match on $\mathit{dm}$:
\begin{itemize}
\item For $\mathtt{gaussLaw}(v, \_)$, we take $\mathit{ci} = \mathtt{gaussLaw}(v)$ and the equality holds by reflexivity.
\item For $\mathtt{flux}(p, \_)$, we take $\mathit{ci} = \mathtt{flux}(p)$ and the equality holds by reflexivity.
\item For $\mathtt{deformed}(j, \_)$, we take $\mathit{ci} = \mathtt{deformed}(j)$ and the equality holds by reflexivity.
\end{itemize}
\end{proof}

\begin{theorem}[Phase 2 Operators Pairwise Commute]
\label{thm:FaultTolerantGaugingProcedure.phase2Operators_commute}
\lean{QEC1.FaultTolerantGaugingProcedure.phase2Operators_commute}
\leanok
\uses{def:FaultTolerantGaugingProcedure.phase2Operator, def:PauliOp.PauliCommute, thm:DeformedCode.allChecks_commute}
For any two Phase~2 measurements $\mathit{dm}_1$ and $\mathit{dm}_2$,
\[
\operatorname{PauliCommute}(\mathtt{phase2Operator}(\mathit{dm}_1), \mathtt{phase2Operator}(\mathit{dm}_2)).
\]
\end{theorem}
\begin{proof}
\leanok
\uses{thm:FaultTolerantGaugingProcedure.phase2Operator_is_allChecks, thm:DeformedCode.allChecks_commute}
From the previous theorem, we obtain check indices $\mathit{ci}_1$ and $\mathit{ci}_2$ such that $\mathtt{phase2Operator}(\mathit{dm}_i) = \mathtt{allChecks}(\mathit{ci}_i)$. Rewriting with these equalities, the result follows from $\mathtt{allChecks\_commute}$, which guarantees that all checks of the deformed code pairwise commute (using the cycle parity condition and the original checks' commutativity).
\end{proof}

\begin{theorem}[Phase 2 Operators Self-Inverse]
\label{thm:FaultTolerantGaugingProcedure.phase2Operators_selfInverse}
\lean{QEC1.FaultTolerantGaugingProcedure.phase2Operators_selfInverse}
\leanok
\uses{def:FaultTolerantGaugingProcedure.phase2Operator, thm:DeformedCode.allChecks_self_inverse}
For any Phase~2 measurement $\mathit{dm}$,
\[
\mathtt{phase2Operator}(\mathit{dm}) \cdot \mathtt{phase2Operator}(\mathit{dm}) = 1.
\]
\end{theorem}
\begin{proof}
\leanok
\uses{thm:FaultTolerantGaugingProcedure.phase2Operator_is_allChecks, thm:DeformedCode.allChecks_self_inverse}
From $\mathtt{phase2Operator\_is\_allChecks}$, we obtain a check index $\mathit{ci}$ with $\mathtt{phase2Operator}(\mathit{dm}) = \mathtt{allChecks}(\mathit{ci})$. Rewriting, the result follows directly from $\mathtt{allChecks\_self\_inverse}$.
\end{proof}

\begin{lemma}[Phase 2 Gauss's Law Identification]
\label{lem:FaultTolerantGaugingProcedure.phase2_gaussLaw_eq}
\lean{QEC1.FaultTolerantGaugingProcedure.phase2_gaussLaw_eq}
\leanok
\uses{def:FaultTolerantGaugingProcedure.phase2Operator, def:DeformedCode.gaussLawChecks}
For any $v \in V$ and round $r$, $\mathtt{phase2Operator}(\mathtt{gaussLaw}(v, r)) = \text{gaussLawChecks}(G, v)$.
\end{lemma}
\begin{proof}
\leanok
\uses{def:FaultTolerantGaugingProcedure.phase2Operator}
This holds by definition.
\end{proof}

\begin{lemma}[Phase 2 Flux Identification]
\label{lem:FaultTolerantGaugingProcedure.phase2_flux_eq}
\lean{QEC1.FaultTolerantGaugingProcedure.phase2_flux_eq}
\leanok
\uses{def:FaultTolerantGaugingProcedure.phase2Operator, def:DeformedCode.fluxChecks}
For any $p \in C$ and round $r$, $\mathtt{phase2Operator}(\mathtt{flux}(p, r)) = \text{fluxChecks}(G, \mathit{cycles}, p)$.
\end{lemma}
\begin{proof}
\leanok
\uses{def:FaultTolerantGaugingProcedure.phase2Operator}
This holds by definition.
\end{proof}

\begin{lemma}[Phase 2 Deformed Check Identification]
\label{lem:FaultTolerantGaugingProcedure.phase2_deformed_eq}
\lean{QEC1.FaultTolerantGaugingProcedure.phase2_deformed_eq}
\leanok
\uses{def:FaultTolerantGaugingProcedure.phase2Operator, def:DeformedCode.deformedOriginalChecks}
For any $j \in J$ and round $r$, $\mathtt{phase2Operator}(\mathtt{deformed}(j, r)) = \text{deformedOriginalChecks}(G, \mathit{checks}, \mathit{deformedData}, j)$.
\end{lemma}
\begin{proof}
\leanok
\uses{def:FaultTolerantGaugingProcedure.phase2Operator}
This holds by definition.
\end{proof}

\begin{theorem}[Gauss's Law Product Equals Logical Operator]
\label{thm:FaultTolerantGaugingProcedure.phase2_gauss_product_eq_logical}
\lean{QEC1.FaultTolerantGaugingProcedure.phase2_gauss_product_eq_logical}
\leanok
\uses{def:FaultTolerantGaugingProcedure.phase2Operator, def:DeformedCode.gaussLawChecks}
The product of all Gauss's law operators over $V$ equals the logical operator $L$:
\[
\prod_{v \in V} \mathtt{phase2Operator}(\mathtt{gaussLaw}(v, 0)) = \text{logicalOp}(G).
\]
\end{theorem}
\begin{proof}
\leanok
\uses{def:FaultTolerantGaugingProcedure.phase2Operator, def:DeformedCode.gaussLawChecks}
Simplifying $\mathtt{phase2Operator}$ on Gauss's law measurements reduces each factor to $\text{gaussLawChecks}(G, v)$. The result then follows directly from $\mathtt{gaussLaw\_product}$, which establishes that the product of all Gauss's law operators is the logical operator.
\end{proof}

\begin{definition}[Phase 1 Repeated Detector]
\label{def:FaultTolerantGaugingProcedure.phase1RepeatedDetector}
\lean{QEC1.FaultTolerantGaugingProcedure.phase1RepeatedDetector}
\leanok
\uses{def:FaultTolerantGaugingProcedure, def:Detector, def:FTGaugingMeasurement, def:PreDeformationMeasurement}
For consecutive Phase~1 rounds $r$ and $r'$ (with $r + 1 = r'$), the \emph{repeated measurement detector} for check $j$ compares the outcomes of measuring check $j$ in rounds $r$ and $r'$. It consists of the measurement set $\{\mathtt{phase1}(\langle j, r \rangle), \mathtt{phase1}(\langle j, r' \rangle)\}$ with ideal outcome $0$ for all measurements.
\end{definition}

\begin{definition}[Phase 2 Repeated Detector -- Gauss's Law]
\label{def:FaultTolerantGaugingProcedure.phase2RepeatedDetector_gauss}
\lean{QEC1.FaultTolerantGaugingProcedure.phase2RepeatedDetector_gauss}
\leanok
\uses{def:FaultTolerantGaugingProcedure, def:Detector, def:FTGaugingMeasurement, def:DeformedCodeMeasurement}
For consecutive Phase~2 rounds $r$ and $r'$, the repeated measurement detector for Gauss's law at vertex $v$ compares the outcomes of $\mathtt{gaussLaw}(v, r)$ and $\mathtt{gaussLaw}(v, r')$ with ideal outcome $0$.
\end{definition}

\begin{definition}[Phase 2 Repeated Detector -- Flux]
\label{def:FaultTolerantGaugingProcedure.phase2RepeatedDetector_flux}
\lean{QEC1.FaultTolerantGaugingProcedure.phase2RepeatedDetector_flux}
\leanok
\uses{def:FaultTolerantGaugingProcedure, def:Detector, def:FTGaugingMeasurement, def:DeformedCodeMeasurement}
For consecutive Phase~2 rounds $r$ and $r'$, the repeated measurement detector for flux at cycle $p$ compares the outcomes of $\mathtt{flux}(p, r)$ and $\mathtt{flux}(p, r')$ with ideal outcome $0$.
\end{definition}

\begin{definition}[Phase 2 Repeated Detector -- Deformed Checks]
\label{def:FaultTolerantGaugingProcedure.phase2RepeatedDetector_deformed}
\lean{QEC1.FaultTolerantGaugingProcedure.phase2RepeatedDetector_deformed}
\leanok
\uses{def:FaultTolerantGaugingProcedure, def:Detector, def:FTGaugingMeasurement, def:DeformedCodeMeasurement}
For consecutive Phase~2 rounds $r$ and $r'$, the repeated measurement detector for deformed check $j$ compares the outcomes of $\mathtt{deformed}(j, r)$ and $\mathtt{deformed}(j, r')$ with ideal outcome $0$.
\end{definition}

\begin{definition}[Edge Initialization Detector]
\label{def:FaultTolerantGaugingProcedure.edgeInitDetector}
\lean{QEC1.FaultTolerantGaugingProcedure.edgeInitDetector}
\leanok
\uses{def:FaultTolerantGaugingProcedure, def:Detector, def:FTGaugingMeasurement, def:EdgeInitialization, def:PostDeformationMeasurement}
The \emph{edge initialization detector} for edge $e$ compares the initialization event $\mathtt{edgeInit}(\langle e \rangle)$ with the Phase~3 measurement $\mathtt{phase3}(\mathtt{edgeZ}(e))$, with ideal outcome $0$.
\end{definition}

\begin{definition}[Phase 1 to Phase 2 Boundary Detector]
\label{def:FaultTolerantGaugingProcedure.phase1_to_phase2_detector}
\lean{QEC1.FaultTolerantGaugingProcedure.phase1_to_phase2_detector}
\leanok
\uses{def:FaultTolerantGaugingProcedure, def:Detector, def:FTGaugingMeasurement, def:PreDeformationMeasurement, def:DeformedCodeMeasurement}
The \emph{boundary detector} from Phase~1 to Phase~2 for check $j$ compares the last Phase~1 round measurement $\mathtt{phase1}(\langle j, r_{\text{last}} \rangle)$ with the first Phase~2 deformed check measurement $\mathtt{phase2}(\mathtt{deformed}(j, r_{\text{first}}))$, where $r_{\text{last}} + 1 = d$ and $r_{\text{first}} = 0$, with ideal outcome $0$.
\end{definition}

\begin{definition}[Phase 2 to Phase 3 Boundary Detector]
\label{def:FaultTolerantGaugingProcedure.phase2_to_phase3_detector}
\lean{QEC1.FaultTolerantGaugingProcedure.phase2_to_phase3_detector}
\leanok
\uses{def:FaultTolerantGaugingProcedure, def:Detector, def:FTGaugingMeasurement, def:DeformedCodeMeasurement, def:PostDeformationMeasurement}
The \emph{boundary detector} from Phase~2 to Phase~3 for check $j$ compares the last Phase~2 deformed check measurement $\mathtt{phase2}(\mathtt{deformed}(j, r_{\text{last}}))$ with the first Phase~3 original check measurement $\mathtt{phase3}(\mathtt{originalCheck}(j, r_{\text{first}}))$, where $r_{\text{last}} + 1 = d$ and $r_{\text{first}} = 0$, with ideal outcome $0$.
\end{definition}

\begin{theorem}[Phase 1 Measurement Count]
\label{thm:FaultTolerantGaugingProcedure.phase1_measurement_count}
\lean{QEC1.FaultTolerantGaugingProcedure.phase1_measurement_count}
\leanok
\uses{def:PreDeformationMeasurement}
Phase~1 has $|J| \cdot d$ measurements:
\[
|\text{PreDeformationMeasurement}(J, d)| = |J| \cdot d.
\]
\end{theorem}
\begin{proof}
\leanok
\uses{def:PreDeformationMeasurement}
We establish a bijection between $\text{PreDeformationMeasurement}(J, d)$ and $J \times \operatorname{Fin}(d)$ via the maps $\langle j, r \rangle \mapsto (j, r)$ and $(j, r) \mapsto \langle j, r \rangle$, both of which are inverses by reflexivity. We then rewrite using the cardinality of a product type: $|J \times \operatorname{Fin}(d)| = |J| \cdot |\operatorname{Fin}(d)| = |J| \cdot d$.
\end{proof}

\begin{theorem}[Phase 2 Measurement Count]
\label{thm:FaultTolerantGaugingProcedure.phase2_measurement_count}
\lean{QEC1.FaultTolerantGaugingProcedure.phase2_measurement_count}
\leanok
\uses{def:DeformedCodeMeasurement}
Phase~2 has $(|V| + |C| + |J|) \cdot d$ measurements:
\[
|\text{DeformedCodeMeasurement}(V, C, J, d)| = (|V| + |C| + |J|) \cdot d.
\]
\end{theorem}
\begin{proof}
\leanok
\uses{def:DeformedCodeMeasurement}
We establish a bijection between $\text{DeformedCodeMeasurement}(V, C, J, d)$ and $(V \times \operatorname{Fin}(d)) \oplus (C \times \operatorname{Fin}(d)) \oplus (J \times \operatorname{Fin}(d))$ by mapping each constructor to the corresponding summand and verifying the two maps are inverses (by case analysis, each case holds by reflexivity). We then compute the cardinality using the sum and product type formulas and ring arithmetic to obtain $(|V| + |C| + |J|) \cdot d$.
\end{proof}

\begin{theorem}[Edge Initialization Count]
\label{thm:FaultTolerantGaugingProcedure.edgeInit_count}
\lean{QEC1.FaultTolerantGaugingProcedure.edgeInit_count}
\leanok
\uses{def:EdgeInitialization}
The number of edge initializations equals the number of edges: $|\text{EdgeInitialization}(E)| = |E|$.
\end{theorem}
\begin{proof}
\leanok
\uses{def:EdgeInitialization}
We establish a bijection between $\text{EdgeInitialization}(E)$ and $E$ via $\langle e \rangle \mapsto e$ and $e \mapsto \langle e \rangle$, which are inverses by reflexivity.
\end{proof}

\begin{definition}[Gauging Sign]
\label{def:FaultTolerantGaugingProcedure.gaugingSign}
\lean{QEC1.FaultTolerantGaugingProcedure.gaugingSign}
\leanok
\uses{def:FaultTolerantGaugingProcedure}
The \emph{gauging measurement sign} from Phase~2 Gauss outcomes is
\[
\mathtt{gaugingSign}(\mathit{gaussOutcomes}) = \sum_{v \in V} \mathit{gaussOutcomes}(v) \in \mathbb{Z}/2\mathbb{Z}.
\]
\end{definition}

\begin{lemma}[Gauging Sign is Zero or One]
\label{lem:FaultTolerantGaugingProcedure.gaugingSign_zero_or_one}
\lean{QEC1.FaultTolerantGaugingProcedure.gaugingSign_zero_or_one}
\leanok
\uses{def:FaultTolerantGaugingProcedure.gaugingSign}
For any Gauss outcomes, $\mathtt{gaugingSign}(\mathit{gaussOutcomes}) = 0$ or $\mathtt{gaugingSign}(\mathit{gaussOutcomes}) = 1$.
\end{lemma}
\begin{proof}
\leanok
\uses{def:FaultTolerantGaugingProcedure.gaugingSign}
Every element of $\mathbb{Z}/2\mathbb{Z}$ is either $0$ or $1$, as verified by case analysis on the finite type.
\end{proof}

\begin{theorem}[Gauging Sign Agrees with Def 5 Measurement Sign]
\label{thm:FaultTolerantGaugingProcedure.gaugingSign_eq_measurementSign}
\lean{QEC1.FaultTolerantGaugingProcedure.gaugingSign_eq_measurementSign}
\leanok
\uses{def:FaultTolerantGaugingProcedure.gaugingSign, def:GaugingMeasurement.measurementSign}
For any Gauss outcomes and edge outcomes,
\[
\mathtt{gaugingSign}(\mathit{gaussOutcomes}) = \mathtt{measurementSign}(G, \langle \mathit{gaussOutcomes}, \mathit{edgeOutcomes} \rangle).
\]
\end{theorem}
\begin{proof}
\leanok
\uses{def:FaultTolerantGaugingProcedure.gaugingSign, def:GaugingMeasurement.measurementSign}
Simplifying the definitions of $\mathtt{gaugingSign}$ and $\mathtt{measurementSign}$, both reduce to $\sum_{v \in V} \mathit{gaussOutcomes}(v)$.
\end{proof}

\begin{lemma}[Phase 2 Gauss's Law is Pure X]
\label{lem:FaultTolerantGaugingProcedure.phase2_gaussLaw_pure_X}
\lean{QEC1.FaultTolerantGaugingProcedure.phase2_gaussLaw_pure_X}
\leanok
\uses{def:FaultTolerantGaugingProcedure.phase2Operator}
For any vertex $v \in V$ and round $r$, the Phase~2 Gauss's law operator has zero $Z$-component:
\[
(\mathtt{phase2Operator}(\mathtt{gaussLaw}(v, r))).\mathit{zVec} = 0.
\]
\end{lemma}
\begin{proof}
\leanok
\uses{def:FaultTolerantGaugingProcedure.phase2Operator, lem:GaussFlux.gaussLawOp_zVec}
This follows directly from $\mathtt{gaussLawOp\_zVec}$, which establishes that Gauss's law operators are pure $X$-type.
\end{proof}

\begin{lemma}[Phase 2 Flux is Pure Z]
\label{lem:FaultTolerantGaugingProcedure.phase2_flux_pure_Z}
\lean{QEC1.FaultTolerantGaugingProcedure.phase2_flux_pure_Z}
\leanok
\uses{def:FaultTolerantGaugingProcedure.phase2Operator}
For any cycle $p \in C$ and round $r$, the Phase~2 flux operator has zero $X$-component:
\[
(\mathtt{phase2Operator}(\mathtt{flux}(p, r))).\mathit{xVec} = 0.
\]
\end{lemma}
\begin{proof}
\leanok
\uses{def:FaultTolerantGaugingProcedure.phase2Operator, lem:GaussFlux.fluxOp_xVec}
This follows directly from $\mathtt{fluxOp\_xVec}$, which establishes that flux operators are pure $Z$-type.
\end{proof}

\begin{lemma}[Phase 2 Deformed Checks Have No X-Support on Edges]
\label{lem:FaultTolerantGaugingProcedure.phase2_deformed_noXOnEdges}
\lean{QEC1.FaultTolerantGaugingProcedure.phase2_deformed_noXOnEdges}
\leanok
\uses{def:FaultTolerantGaugingProcedure.phase2Operator}
For any check $j \in J$, round $r$, and edge $e \in E(G)$, the Phase~2 deformed check operator has zero $X$-component on edge qubit $e$:
\[
(\mathtt{phase2Operator}(\mathtt{deformed}(j, r))).\mathit{xVec}(\operatorname{inr}(e)) = 0.
\]
\end{lemma}
\begin{proof}
\leanok
\uses{def:FaultTolerantGaugingProcedure.phase2Operator}
This follows directly from $\mathtt{deformedOpExt\_xVec\_edge}$, which establishes that the deformed extension of an operator has no $X$-support on edge qubits.
\end{proof}

%--- Lem_4: SpacetimeCodeDetectors ---
\chapter{Lem 4: Spacetime Code Detectors}

This chapter establishes that the local detectors in the fault-tolerant gauging measurement procedure form a complete generating set. For the original code phases ($t < t_i$ and $t > t_o$), detectors arise from repeated measurements of checks $s_j$ at consecutive rounds. During the deformed code phase ($t_i < t < t_o$), detectors come from repeated measurements of deformed code checks $A_v$, $B_p$, and $\tilde{s}_j$. At the gauging boundary $t = t_i$, detectors combine edge qubit initializations with first measurements; at the ungauging boundary $t = t_o$, detectors combine last measurements with $Z_e$ readouts.

\begin{definition}[$\mathbb{Z}_2$ Generation]
\label{def:IsGeneratedBy}
\lean{QEC1.IsGeneratedBy}
\leanok

Let $\alpha$ be a type with decidable equality and let $\{\text{generators}_i\}_{i \in \iota}$ be a family of finsets over $\alpha$. A finset $S$ is \emph{$\mathbb{Z}_2$-generated} by the generators if $S$ can be obtained from $\emptyset$ by iterated symmetric differences with generators. Formally, this is defined inductively:
\begin{itemize}
  \item $\emptyset$ is generated.
  \item If $S$ is generated, then $S \mathbin{\triangle} \text{generators}_i$ is generated for any $i \in \iota$.
\end{itemize}
\end{definition}

\begin{lemma}[Generator Membership]
\label{lem:IsGeneratedBy.generator}
\lean{QEC1.IsGeneratedBy.generator}
\leanok
\uses{def:IsGeneratedBy}
Each generator $\text{generators}_i$ is in the $\mathbb{Z}_2$ span.
\end{lemma}

\begin{proof}
\leanok
\uses{def:IsGeneratedBy}
We have $\text{generators}_i = \emptyset \mathbin{\triangle} \text{generators}_i$. Since $\emptyset$ is generated (by the base case), the symmetric difference rule gives that $\emptyset \mathbin{\triangle} \text{generators}_i$ is generated. By the identity $\bot \mathbin{\triangle} x = x$ (where $\bot = \emptyset$), this equals $\text{generators}_i$.
\end{proof}

\begin{lemma}[$\mathbb{Z}_2$ Span Closure under Symmetric Difference]
\label{lem:IsGeneratedBy.symmDiff_closure}
\lean{QEC1.IsGeneratedBy.symmDiff_closure}
\leanok
\uses{def:IsGeneratedBy}
If $S$ and $T$ are both $\mathbb{Z}_2$-generated by the generators, then $S \mathbin{\triangle} T$ is also generated.
\end{lemma}

\begin{proof}
\leanok
\uses{def:IsGeneratedBy}
We proceed by induction on the derivation that $T$ is generated, generalizing over $S$.

\textbf{Base case} ($T = \emptyset$): We have $S \mathbin{\triangle} \emptyset = S \mathbin{\triangle} \bot = S$, which is generated by assumption.

\textbf{Inductive step} ($T = U \mathbin{\triangle} \text{generators}_i$ where $U$ is generated): By the induction hypothesis applied to $S$, we know $S \mathbin{\triangle} U$ is generated. By associativity of symmetric difference, $S \mathbin{\triangle} (U \mathbin{\triangle} \text{generators}_i) = (S \mathbin{\triangle} U) \mathbin{\triangle} \text{generators}_i$. Since $S \mathbin{\triangle} U$ is generated, the symmetric difference rule gives the result.
\end{proof}

\begin{definition}[Cycle Edges]
\label{def:FaultTolerantGaugingProcedure.cycleEdges}
\lean{QEC1.FaultTolerantGaugingProcedure.cycleEdges}
\leanok
\uses{def:FaultTolerantGaugingProcedure}
Given a fault-tolerant gauging procedure and a cycle $p \in C$, the \emph{cycle edges} of $p$ is the finset of edges $e \in G.\text{edgeSet}$ such that $e \in \text{cycles}(p)$.
\end{definition}

\begin{definition}[Edge Path Edges]
\label{def:FaultTolerantGaugingProcedure.edgePathEdges}
\lean{QEC1.FaultTolerantGaugingProcedure.edgePathEdges}
\leanok
\uses{def:FaultTolerantGaugingProcedure}
Given a fault-tolerant gauging procedure and a check index $j \in J$, the \emph{edge-path edges} of $j$ is the finset of edges $e \in G.\text{edgeSet}$ such that $\text{edgePath}(j, e) \neq 0$ in $\mathbb{Z}/2\mathbb{Z}$.
\end{definition}

\begin{definition}[Phase 1 Repeated Detector (Parametric)]
\label{def:FaultTolerantGaugingProcedure.phase1RepeatedDetector_parametric}
\lean{QEC1.FaultTolerantGaugingProcedure.phase1RepeatedDetector_parametric}
\leanok
\uses{def:FaultTolerantGaugingProcedure, def:FTGaugingMeasurement, def:Detector}
Given a fault-tolerant gauging procedure, a check index $j \in J$, consecutive rounds $r, r' \in \operatorname{Fin}(d)$ with $r + 1 = r'$, and an outcome $\sigma \in \mathbb{Z}/2\mathbb{Z}$, the \emph{Phase~1 repeated detector} is the detector with:
\begin{itemize}
  \item Measurements: $\{\texttt{phase1}(j, r),\; \texttt{phase1}(j, r')\}$.
  \item Ideal outcome: $\sigma$ on both measurements, $0$ otherwise.
  \item Constraint: $\sigma + \sigma = 0$ in $\mathbb{Z}/2\mathbb{Z}$ (by characteristic 2).
\end{itemize}
The outcome $\sigma$ represents the unknown but equal eigenvalue of both measurements of the self-inverse check $s_j$.
\end{definition}

\begin{definition}[Phase 3 Repeated Detector]
\label{def:FaultTolerantGaugingProcedure.phase3RepeatedDetector}
\lean{QEC1.FaultTolerantGaugingProcedure.phase3RepeatedDetector}
\leanok
\uses{def:FaultTolerantGaugingProcedure, def:FTGaugingMeasurement, def:PostDeformationMeasurement, def:Detector}
Given a fault-tolerant gauging procedure, a check index $j \in J$, consecutive rounds $r, r' \in \operatorname{Fin}(d)$ with $r + 1 = r'$, and an outcome $\sigma \in \mathbb{Z}/2\mathbb{Z}$, the \emph{Phase~3 repeated detector} is the detector with:
\begin{itemize}
  \item Measurements: $\{\texttt{phase3}(\texttt{originalCheck}(j, r)),\; \texttt{phase3}(\texttt{originalCheck}(j, r'))\}$.
  \item Ideal outcome: $\sigma$ on both measurements, $0$ otherwise.
  \item Constraint: $\sigma + \sigma = 0$ in $\mathbb{Z}/2\mathbb{Z}$.
\end{itemize}
\end{definition}

\begin{definition}[Flux Initialization Detector]
\label{def:FaultTolerantGaugingProcedure.fluxInitDetector}
\lean{QEC1.FaultTolerantGaugingProcedure.fluxInitDetector}
\leanok
\uses{def:FaultTolerantGaugingProcedure, def:FTGaugingMeasurement, def:Detector}
Given a fault-tolerant gauging procedure, a cycle $p \in C$, and the first round $r_{\text{first}}$ with $r_{\text{first}} = 0$, the \emph{flux initialization detector} $B_p^{t_i}$ is the detector with:
\begin{itemize}
  \item Measurements: $\{\texttt{edgeInit}(e) \mid e \in \text{cycles}(p)\} \cup \{\texttt{phase2}(\texttt{flux}(p, r_{\text{first}}))\}$.
  \item Ideal outcome: $0$ for all measurements.
  \item Constraint: $\sum 0 = 0$.
\end{itemize}
Edge qubits initialized in $|0\rangle$ are $Z$-eigenstates with eigenvalue $+1$ (i.e., $0$ in $\mathbb{Z}/2\mathbb{Z}$). Since $B_p = \prod_{e \in p} Z_e$ is pure $Z$-type, $B_p$ on these $|0\rangle$ states gives outcome $\sum 0 = 0$.
\end{definition}

\begin{definition}[Deformed Initialization Detector]
\label{def:FaultTolerantGaugingProcedure.deformedInitDetector}
\lean{QEC1.FaultTolerantGaugingProcedure.deformedInitDetector}
\leanok
\uses{def:FaultTolerantGaugingProcedure, def:FTGaugingMeasurement, def:Detector}
Given a fault-tolerant gauging procedure, a check index $j \in J$, the last Phase~1 round $r_{\text{last}}$ with $r_{\text{last}} + 1 = d$, the first Phase~2 round $r_{\text{first}} = 0$, and an outcome $\sigma \in \mathbb{Z}/2\mathbb{Z}$, the \emph{deformed initialization detector} $\tilde{s}_j^{t_i}$ is the detector with:
\begin{itemize}
  \item Measurements: $\{\texttt{phase1}(j, r_{\text{last}}),\; \texttt{phase2}(\texttt{deformed}(j, r_{\text{first}}))\} \cup \{\texttt{edgeInit}(e) \mid e \in \gamma_j\}$.
  \item Ideal outcome: $\sigma$ for the Phase~1 and Phase~2 deformed measurements, $0$ for edge initializations.
  \item Constraint: $\sigma + \sigma + \sum 0 = 0$ in $\mathbb{Z}/2\mathbb{Z}$.
\end{itemize}
Since $\tilde{s}_j = s_j \cdot \prod_{e \in \gamma_j} Z_e$ and edge qubits are initialized in $|0\rangle$ (so $Z_e \to +1$), the outcomes of $s_j$ and $\tilde{s}_j$ agree. The parameter $\sigma$ encodes this unknown shared eigenvalue.
\end{definition}

\begin{definition}[Flux Ungauging Detector]
\label{def:FaultTolerantGaugingProcedure.fluxUngaugeDetector}
\lean{QEC1.FaultTolerantGaugingProcedure.fluxUngaugeDetector}
\leanok
\uses{def:FaultTolerantGaugingProcedure, def:FTGaugingMeasurement, def:PostDeformationMeasurement, def:Detector}
Given a fault-tolerant gauging procedure, a cycle $p \in C$, and the last round $r_{\text{last}}$ with $r_{\text{last}} + 1 = d$, the \emph{flux ungauging detector} $B_p^{t_o}$ is the detector with:
\begin{itemize}
  \item Measurements: $\{\texttt{phase2}(\texttt{flux}(p, r_{\text{last}}))\} \cup \{\texttt{phase3}(\texttt{edgeZ}(e)) \mid e \in \text{cycles}(p)\}$.
  \item Ideal outcome: $0$ for all measurements.
  \item Constraint: $\beta_p + \sum z_e = 0$ in $\mathbb{Z}/2\mathbb{Z}$, since $B_p = \prod_{e \in p} Z_e$ means $\beta_p = \sum z_e$.
\end{itemize}
\end{definition}

\begin{definition}[Deformed Ungauging Detector]
\label{def:FaultTolerantGaugingProcedure.deformedUngaugeDetector}
\lean{QEC1.FaultTolerantGaugingProcedure.deformedUngaugeDetector}
\leanok
\uses{def:FaultTolerantGaugingProcedure, def:FTGaugingMeasurement, def:PostDeformationMeasurement, def:Detector}
Given a fault-tolerant gauging procedure, a check index $j \in J$, the last Phase~2 round $r_{\text{last}}$ with $r_{\text{last}} + 1 = d$, and the first Phase~3 round $r_{\text{first}} = 0$, the \emph{deformed ungauging detector} $\tilde{s}_j^{t_o}$ is the detector with:
\begin{itemize}
  \item Measurements: $\{\texttt{phase2}(\texttt{deformed}(j, r_{\text{last}})),\; \texttt{phase3}(\texttt{originalCheck}(j, r_{\text{first}}))\} \cup \{\texttt{phase3}(\texttt{edgeZ}(e)) \mid e \in \gamma_j\}$.
  \item Ideal outcome: $0$ for all measurements.
  \item Constraint: $\tilde{\sigma}_j + \sum z_e + \sigma_j = 0$ in $\mathbb{Z}/2\mathbb{Z}$, since $\tilde{s}_j = s_j \cdot \prod_{e \in \gamma_j} Z_e$ implies $\tilde{\sigma}_j = \sigma_j + \sum z_e$.
\end{itemize}
\end{definition}

\begin{theorem}[Phase 2 Gauss Self-Inverse]
\label{thm:FaultTolerantGaugingProcedure.phase2_gauss_selfInverse}
\lean{QEC1.FaultTolerantGaugingProcedure.phase2_gauss_selfInverse}
\leanok
\uses{def:FaultTolerantGaugingProcedure}
For any vertex $v \in V$ and round $r \in \operatorname{Fin}(d)$, the Phase~2 Gauss operator satisfies $A_v \cdot A_v = 1$. This ensures consecutive measurements of $A_v$ yield the same eigenvalue.
\end{theorem}

\begin{proof}
\leanok
\uses{thm:FaultTolerantGaugingProcedure.phase2Operators_selfInverse}
This follows directly from the general self-inverse property of Phase~2 operators applied to the Gauss law measurement $(\texttt{gaussLaw}\; v\; r)$.
\end{proof}

\begin{theorem}[Phase 2 Flux Self-Inverse]
\label{thm:FaultTolerantGaugingProcedure.phase2_flux_selfInverse}
\lean{QEC1.FaultTolerantGaugingProcedure.phase2_flux_selfInverse}
\leanok
\uses{def:FaultTolerantGaugingProcedure}
For any cycle $p \in C$ and round $r \in \operatorname{Fin}(d)$, the Phase~2 flux operator satisfies $B_p \cdot B_p = 1$.
\end{theorem}

\begin{proof}
\leanok
\uses{thm:FaultTolerantGaugingProcedure.phase2Operators_selfInverse}
This follows directly from the general self-inverse property of Phase~2 operators applied to the flux measurement $(\texttt{flux}\; p\; r)$.
\end{proof}

\begin{theorem}[Phase 2 Deformed Self-Inverse]
\label{thm:FaultTolerantGaugingProcedure.phase2_deformed_selfInverse}
\lean{QEC1.FaultTolerantGaugingProcedure.phase2_deformed_selfInverse}
\leanok
\uses{def:FaultTolerantGaugingProcedure}
For any check index $j \in J$ and round $r \in \operatorname{Fin}(d)$, the Phase~2 deformed operator satisfies $\tilde{s}_j \cdot \tilde{s}_j = 1$.
\end{theorem}

\begin{proof}
\leanok
\uses{thm:FaultTolerantGaugingProcedure.phase2Operators_selfInverse}
This follows directly from the general self-inverse property of Phase~2 operators applied to the deformed measurement $(\texttt{deformed}\; j\; r)$.
\end{proof}

\begin{theorem}[Flux is Pure $Z$-Type on Edges]
\label{thm:FaultTolerantGaugingProcedure.flux_pure_Z_on_edges}
\lean{QEC1.FaultTolerantGaugingProcedure.flux_pure_Z_on_edges}
\leanok
\uses{def:FaultTolerantGaugingProcedure}
For any cycle $p \in C$ and round $r \in \operatorname{Fin}(d)$, the Phase~2 flux operator $B_p = \prod_{e \in p} Z_e$ satisfies $\text{xVec}(B_p) = 0$. This underlies the flux boundary detector validity.
\end{theorem}

\begin{proof}
\leanok
\uses{lem:FaultTolerantGaugingProcedure.phase2_flux_pure_Z}
This follows directly from the Phase~2 flux pure-$Z$ property.
\end{proof}

\begin{theorem}[Deformed Operator Has No $X$-Support on Edges]
\label{thm:FaultTolerantGaugingProcedure.deformed_noXOnEdges}
\lean{QEC1.FaultTolerantGaugingProcedure.deformed_noXOnEdges}
\leanok
\uses{def:FaultTolerantGaugingProcedure}
For any check index $j \in J$, round $r \in \operatorname{Fin}(d)$, and edge $e \in G.\text{edgeSet}$, the Phase~2 deformed operator $\tilde{s}_j$ satisfies $\text{xVec}(\tilde{s}_j)(\operatorname{inr}(e)) = 0$. This means $\tilde{s}_j = s_j \cdot \prod_{e \in \gamma_j} Z_e$ has no $X$-support on edge qubits.
\end{theorem}

\begin{proof}
\leanok
\uses{lem:FaultTolerantGaugingProcedure.phase2_deformed_noXOnEdges}
This follows directly from the Phase~2 deformed no-$X$-on-edges property.
\end{proof}

\begin{theorem}[Phase 2 Gauss Round Independence]
\label{thm:FaultTolerantGaugingProcedure.phase2_gauss_round_independent}
\lean{QEC1.FaultTolerantGaugingProcedure.phase2_gauss_round_independent}
\leanok
\uses{def:FaultTolerantGaugingProcedure}
For any vertex $v \in V$ and rounds $r, r' \in \operatorname{Fin}(d)$, the Phase~2 Gauss operator is round-independent: the same $A_v$ is measured each round.
\end{theorem}

\begin{proof}
\leanok
\uses{def:FaultTolerantGaugingProcedure}
This holds by reflexivity: the operator definition does not depend on the round index.
\end{proof}

\begin{theorem}[Phase 2 Flux Round Independence]
\label{thm:FaultTolerantGaugingProcedure.phase2_flux_round_independent}
\lean{QEC1.FaultTolerantGaugingProcedure.phase2_flux_round_independent}
\leanok
\uses{def:FaultTolerantGaugingProcedure}
For any cycle $p \in C$ and rounds $r, r' \in \operatorname{Fin}(d)$, the Phase~2 flux operator is round-independent: the same $B_p$ is measured each round.
\end{theorem}

\begin{proof}
\leanok
\uses{def:FaultTolerantGaugingProcedure}
This holds by reflexivity: the operator definition does not depend on the round index.
\end{proof}

\begin{theorem}[Phase 2 Deformed Round Independence]
\label{thm:FaultTolerantGaugingProcedure.phase2_deformed_round_independent}
\lean{QEC1.FaultTolerantGaugingProcedure.phase2_deformed_round_independent}
\leanok
\uses{def:FaultTolerantGaugingProcedure}
For any check index $j \in J$ and rounds $r, r' \in \operatorname{Fin}(d)$, the Phase~2 deformed operator is round-independent: the same $\tilde{s}_j$ is measured each round.
\end{theorem}

\begin{proof}
\leanok
\uses{def:FaultTolerantGaugingProcedure}
This holds by reflexivity: the operator definition does not depend on the round index.
\end{proof}

\begin{definition}[Detector Index]
\label{def:FaultTolerantGaugingProcedure.DetectorIndex}
\lean{QEC1.FaultTolerantGaugingProcedure.DetectorIndex}
\leanok
\uses{def:FaultTolerantGaugingProcedure}
The \emph{detector index type} enumerates all detector generators in the spacetime code. Its constructors are:
\begin{itemize}
  \item $\texttt{phase1Repeated}(j, r, r', h_r)$: repeated measurement of original check $s_j$ in Phase~1, at consecutive rounds $r, r'$ with $r + 1 = r'$.
  \item $\texttt{phase2GaussRepeated}(v, r, r', h_r)$: repeated measurement of Gauss check $A_v$ in Phase~2.
  \item $\texttt{phase2FluxRepeated}(p, r, r', h_r)$: repeated measurement of flux check $B_p$ in Phase~2.
  \item $\texttt{phase2DeformedRepeated}(j, r, r', h_r)$: repeated measurement of deformed check $\tilde{s}_j$ in Phase~2.
  \item $\texttt{phase3Repeated}(j, r, r', h_r)$: repeated measurement of original check $s_j$ in Phase~3.
  \item $\texttt{fluxInit}(p)$: boundary detector $B_p^{t_i}$ at gauging step.
  \item $\texttt{deformedInit}(j)$: boundary detector $\tilde{s}_j^{t_i}$ at gauging step.
  \item $\texttt{fluxUngauge}(p)$: boundary detector $B_p^{t_o}$ at ungauging step.
  \item $\texttt{deformedUngauge}(j)$: boundary detector $\tilde{s}_j^{t_o}$ at ungauging step.
\end{itemize}
No standalone edge-initialization detectors are needed: edge initialization and readout events are covered by the composite boundary detectors.
\end{definition}

\begin{definition}[Detector of Index]
\label{def:FaultTolerantGaugingProcedure.detectorOfIndex}
\lean{QEC1.FaultTolerantGaugingProcedure.detectorOfIndex}
\leanok
\uses{def:FaultTolerantGaugingProcedure, def:FaultTolerantGaugingProcedure.DetectorIndex, def:FaultTolerantGaugingProcedure.phase1RepeatedDetector_parametric, def:FaultTolerantGaugingProcedure.phase3RepeatedDetector, def:FaultTolerantGaugingProcedure.fluxInitDetector, def:FaultTolerantGaugingProcedure.deformedInitDetector, def:FaultTolerantGaugingProcedure.fluxUngaugeDetector, def:FaultTolerantGaugingProcedure.deformedUngaugeDetector, def:FaultTolerantGaugingProcedure.phase2RepeatedDetector_gauss, def:FaultTolerantGaugingProcedure.phase2RepeatedDetector_flux, def:FaultTolerantGaugingProcedure.phase2RepeatedDetector_deformed, def:Detector}
The map from detector indices to actual detectors. Repeated measurement detectors take outcome $0$ as a canonical choice (the constraint proof works for any outcome). The boundary detectors use the first/last rounds $r = 0$ and $r = d - 1$.
\end{definition}

\begin{theorem}[Every Generator is Valid]
\label{thm:FaultTolerantGaugingProcedure.detectorOfIndex_no_fault}
\lean{QEC1.FaultTolerantGaugingProcedure.detectorOfIndex_no_fault}
\leanok
\uses{def:FaultTolerantGaugingProcedure.detectorOfIndex, def:Detector.isViolated}
For every detector index, the corresponding detector is not violated without faults:
$\neg (\text{detectorOfIndex}(\text{idx})).\text{isViolated}(\emptyset)$.
\end{theorem}

\begin{proof}
\leanok
\uses{thm:Detector.not_isViolated_no_faults}
By case analysis on the detector index; in each case, the result follows directly from the general theorem that no detector is violated when there are no faults.
\end{proof}

\begin{definition}[Generator Measurements]
\label{def:FaultTolerantGaugingProcedure.generatorMeasurements}
\lean{QEC1.FaultTolerantGaugingProcedure.generatorMeasurements}
\leanok
\uses{def:FaultTolerantGaugingProcedure.detectorOfIndex}
For each detector index, the \emph{generator measurement set} is the measurement set of the corresponding detector.
\end{definition}

\begin{lemma}[Flux Init Detector Contains Flux Measurement]
\label{lem:FaultTolerantGaugingProcedure.fluxInitDetector_has_flux}
\lean{QEC1.FaultTolerantGaugingProcedure.fluxInitDetector_has_flux}
\leanok
\uses{def:FaultTolerantGaugingProcedure.fluxInitDetector, def:FTGaugingMeasurement}
The flux initialization detector for cycle $p$ contains the measurement $\texttt{phase2}(\texttt{flux}(p, r_{\text{first}}))$.
\end{lemma}

\begin{proof}
\leanok
\uses{def:FaultTolerantGaugingProcedure.fluxInitDetector}
The flux measurement is the singleton component of the union, so membership follows from $\texttt{Finset.mem\_union\_right}$ and $\texttt{Finset.mem\_singleton.mpr}$ applied to reflexivity.
\end{proof}

\begin{lemma}[Flux Init Detector Contains Edge Inits]
\label{lem:FaultTolerantGaugingProcedure.fluxInitDetector_has_edgeInit}
\lean{QEC1.FaultTolerantGaugingProcedure.fluxInitDetector_has_edgeInit}
\leanok
\uses{def:FaultTolerantGaugingProcedure.fluxInitDetector, def:FTGaugingMeasurement}
For any edge $e$ in the cycle $p$, the flux initialization detector contains the edge initialization measurement $\texttt{edgeInit}(e)$.
\end{lemma}

\begin{proof}
\leanok
\uses{def:FaultTolerantGaugingProcedure.fluxInitDetector}
The edge initialization is in the image part of the union. Since $e \in \text{cycles}(p)$, we have $e$ in the filtered finset, and its image under the $\texttt{edgeInit}$ map gives the desired membership.
\end{proof}

\begin{lemma}[Flux Ungauge Detector Contains Flux Measurement]
\label{lem:FaultTolerantGaugingProcedure.fluxUngaugeDetector_has_flux}
\lean{QEC1.FaultTolerantGaugingProcedure.fluxUngaugeDetector_has_flux}
\leanok
\uses{def:FaultTolerantGaugingProcedure.fluxUngaugeDetector, def:FTGaugingMeasurement}
The flux ungauging detector for cycle $p$ contains the measurement $\texttt{phase2}(\texttt{flux}(p, r_{\text{last}}))$.
\end{lemma}

\begin{proof}
\leanok
\uses{def:FaultTolerantGaugingProcedure.fluxUngaugeDetector}
The flux measurement is the singleton in the left component of the union, so membership follows directly.
\end{proof}

\begin{lemma}[Flux Ungauge Detector Contains Edge $Z$ Measurements]
\label{lem:FaultTolerantGaugingProcedure.fluxUngaugeDetector_has_edgeZ}
\lean{QEC1.FaultTolerantGaugingProcedure.fluxUngaugeDetector_has_edgeZ}
\leanok
\uses{def:FaultTolerantGaugingProcedure.fluxUngaugeDetector, def:FTGaugingMeasurement, def:PostDeformationMeasurement}
For any edge $e$ in the cycle $p$, the flux ungauging detector contains $\texttt{phase3}(\texttt{edgeZ}(e))$.
\end{lemma}

\begin{proof}
\leanok
\uses{def:FaultTolerantGaugingProcedure.fluxUngaugeDetector}
Since $e \in \text{cycles}(p)$, the edge is in the filtered finset, and its image under the $\texttt{phase3} \circ \texttt{edgeZ}$ map gives membership in the right component of the union.
\end{proof}

\begin{lemma}[Deformed Ungauge Detector Contains Deformed Measurement]
\label{lem:FaultTolerantGaugingProcedure.deformedUngaugeDetector_has_deformed}
\lean{QEC1.FaultTolerantGaugingProcedure.deformedUngaugeDetector_has_deformed}
\leanok
\uses{def:FaultTolerantGaugingProcedure.deformedUngaugeDetector, def:FTGaugingMeasurement}
The deformed ungauging detector for check $j$ contains $\texttt{phase2}(\texttt{deformed}(j, r_{\text{last}}))$.
\end{lemma}

\begin{proof}
\leanok
\uses{def:FaultTolerantGaugingProcedure.deformedUngaugeDetector}
This is the first element of the insert pair in the left component of the union.
\end{proof}

\begin{lemma}[Deformed Ungauge Detector Contains Original Check]
\label{lem:FaultTolerantGaugingProcedure.deformedUngaugeDetector_has_originalCheck}
\lean{QEC1.FaultTolerantGaugingProcedure.deformedUngaugeDetector_has_originalCheck}
\leanok
\uses{def:FaultTolerantGaugingProcedure.deformedUngaugeDetector, def:FTGaugingMeasurement, def:PostDeformationMeasurement}
The deformed ungauging detector for check $j$ contains $\texttt{phase3}(\texttt{originalCheck}(j, r_{\text{first}}))$.
\end{lemma}

\begin{proof}
\leanok
\uses{def:FaultTolerantGaugingProcedure.deformedUngaugeDetector}
This is the second element (the singleton) in the insert pair of the left component of the union.
\end{proof}

\begin{lemma}[Deformed Ungauge Detector Contains Edge $Z$]
\label{lem:FaultTolerantGaugingProcedure.deformedUngaugeDetector_has_edgeZ}
\lean{QEC1.FaultTolerantGaugingProcedure.deformedUngaugeDetector_has_edgeZ}
\leanok
\uses{def:FaultTolerantGaugingProcedure.deformedUngaugeDetector, def:FTGaugingMeasurement, def:PostDeformationMeasurement}
For any edge $e$ with $\text{edgePath}(j, e) \neq 0$, the deformed ungauging detector for check $j$ contains $\texttt{phase3}(\texttt{edgeZ}(e))$.
\end{lemma}

\begin{proof}
\leanok
\uses{def:FaultTolerantGaugingProcedure.deformedUngaugeDetector}
Since $\text{edgePath}(j, e) \neq 0$, the edge $e$ is in the filtered finset $\text{edgePathEdges}(j)$, and its image under $\texttt{phase3} \circ \texttt{edgeZ}$ lies in the right component of the union.
\end{proof}

\begin{lemma}[Deformed Init Detector Contains Phase 1]
\label{lem:FaultTolerantGaugingProcedure.deformedInitDetector_has_phase1}
\lean{QEC1.FaultTolerantGaugingProcedure.deformedInitDetector_has_phase1}
\leanok
\uses{def:FaultTolerantGaugingProcedure.deformedInitDetector, def:FTGaugingMeasurement}
The deformed initialization detector for check $j$ contains $\texttt{phase1}(j, r_{\text{last}})$.
\end{lemma}

\begin{proof}
\leanok
\uses{def:FaultTolerantGaugingProcedure.deformedInitDetector}
This is the first element of the insert pair in the left component of the union.
\end{proof}

\begin{lemma}[Deformed Init Detector Contains Deformed Measurement]
\label{lem:FaultTolerantGaugingProcedure.deformedInitDetector_has_deformed}
\lean{QEC1.FaultTolerantGaugingProcedure.deformedInitDetector_has_deformed}
\leanok
\uses{def:FaultTolerantGaugingProcedure.deformedInitDetector, def:FTGaugingMeasurement}
The deformed initialization detector for check $j$ contains $\texttt{phase2}(\texttt{deformed}(j, r_{\text{first}}))$.
\end{lemma}

\begin{proof}
\leanok
\uses{def:FaultTolerantGaugingProcedure.deformedInitDetector}
This is the second element (the singleton) of the insert pair in the left component of the union.
\end{proof}

\begin{lemma}[Deformed Init Detector Contains Edge Inits]
\label{lem:FaultTolerantGaugingProcedure.deformedInitDetector_has_edgeInit}
\lean{QEC1.FaultTolerantGaugingProcedure.deformedInitDetector_has_edgeInit}
\leanok
\uses{def:FaultTolerantGaugingProcedure.deformedInitDetector, def:FTGaugingMeasurement}
For any edge $e$ with $\text{edgePath}(j, e) \neq 0$, the deformed initialization detector contains $\texttt{edgeInit}(e)$.
\end{lemma}

\begin{proof}
\leanok
\uses{def:FaultTolerantGaugingProcedure.deformedInitDetector}
Since $\text{edgePath}(j, e) \neq 0$, the edge $e$ is in the filtered finset $\text{edgePathEdges}(j)$, and its image under $\texttt{edgeInit}$ lies in the right component of the union.
\end{proof}

\begin{lemma}[Phase 1 Coverage]
\label{lem:FaultTolerantGaugingProcedure.phase1_coverage}
\lean{QEC1.FaultTolerantGaugingProcedure.phase1_coverage}
\leanok
\uses{def:FaultTolerantGaugingProcedure.phase1RepeatedDetector_parametric, def:FaultTolerantGaugingProcedure.deformedInitDetector, def:FTGaugingMeasurement}
Phase~1 measurements are covered by repeated or boundary detectors:
\begin{itemize}
  \item If $r + 1 < d$, then $\texttt{phase1}(j, r)$ is in the Phase~1 repeated detector for $(j, r, r+1)$.
  \item If $r + 1 = d$, then $\texttt{phase1}(j, r)$ is in the deformed initialization detector for $j$.
\end{itemize}
\end{lemma}

\begin{proof}
\leanok
\uses{def:FaultTolerantGaugingProcedure.phase1RepeatedDetector_parametric, def:FaultTolerantGaugingProcedure.deformedInitDetector}
In the first case, $\texttt{phase1}(j, r)$ is the first element inserted into the pair. In the second case, it is the first element of the insert pair in the left component of the union of the deformed init detector.
\end{proof}

\begin{lemma}[Phase 2 Gauss Coverage]
\label{lem:FaultTolerantGaugingProcedure.phase2_gauss_coverage}
\lean{QEC1.FaultTolerantGaugingProcedure.phase2_gauss_coverage}
\leanok
\uses{def:FaultTolerantGaugingProcedure, def:FaultTolerantGaugingProcedure.phase2RepeatedDetector_gauss, def:FTGaugingMeasurement}
Phase~2 Gauss measurements are covered by Gauss repeated detectors (assuming $d \geq 2$):
\begin{itemize}
  \item If $r + 1 < d$, then $\texttt{phase2}(\texttt{gaussLaw}(v, r))$ is in the forward repeated detector.
  \item If $0 < r$, then it is in the backward repeated detector pairing round $r-1$ with $r$.
\end{itemize}
\end{lemma}

\begin{proof}
\leanok
\uses{def:FaultTolerantGaugingProcedure.phase2RepeatedDetector_gauss}
In the first case, the measurement is the first element inserted into the pair. In the second case, it is the second element (the singleton) of the pair.
\end{proof}

\begin{lemma}[Phase 2 Flux Coverage]
\label{lem:FaultTolerantGaugingProcedure.phase2_flux_coverage}
\lean{QEC1.FaultTolerantGaugingProcedure.phase2_flux_coverage}
\leanok
\uses{def:FaultTolerantGaugingProcedure.fluxInitDetector, def:FaultTolerantGaugingProcedure.fluxUngaugeDetector, def:FaultTolerantGaugingProcedure.phase2RepeatedDetector_flux, def:FTGaugingMeasurement}
Phase~2 flux measurements are covered by init, ungauge, or repeated detectors:
\begin{itemize}
  \item If $r = 0$, the measurement is in the flux init detector.
  \item If $r + 1 = d$, the measurement is in the flux ungauge detector.
  \item If $r + 1 < d$, the measurement is in a flux repeated detector.
\end{itemize}
\end{lemma}

\begin{proof}
\leanok
\uses{def:FaultTolerantGaugingProcedure.fluxInitDetector, def:FaultTolerantGaugingProcedure.fluxUngaugeDetector, def:FaultTolerantGaugingProcedure.phase2RepeatedDetector_flux}
In the first case, the flux measurement is in the right component (singleton) of the flux init union. In the second case, it is in the left component (singleton) of the flux ungauge union. In the third case, it is the first element of the insert pair in the repeated detector.
\end{proof}

\begin{lemma}[Phase 2 Deformed Coverage]
\label{lem:FaultTolerantGaugingProcedure.phase2_deformed_coverage}
\lean{QEC1.FaultTolerantGaugingProcedure.phase2_deformed_coverage}
\leanok
\uses{def:FaultTolerantGaugingProcedure.deformedInitDetector, def:FaultTolerantGaugingProcedure.deformedUngaugeDetector, def:FaultTolerantGaugingProcedure.phase2RepeatedDetector_deformed, def:FTGaugingMeasurement}
Phase~2 deformed measurements are covered by init, ungauge, or repeated detectors:
\begin{itemize}
  \item If $r = 0$, the measurement is in the deformed init detector.
  \item If $r + 1 = d$, the measurement is in the deformed ungauge detector.
  \item If $r + 1 < d$, the measurement is in a deformed repeated detector.
\end{itemize}
\end{lemma}

\begin{proof}
\leanok
\uses{def:FaultTolerantGaugingProcedure.deformedInitDetector, def:FaultTolerantGaugingProcedure.deformedUngaugeDetector, def:FaultTolerantGaugingProcedure.phase2RepeatedDetector_deformed}
We prove each case separately. For $r = 0$: the deformed measurement is the second element of the insert pair in the deformed init detector's left union component. For $r + 1 = d$: it is the first element of the insert pair in the deformed ungauge detector. For $r + 1 < d$: it is the first element of the insert pair in the repeated detector.
\end{proof}

\begin{lemma}[Phase 3 Original Check Coverage]
\label{lem:FaultTolerantGaugingProcedure.phase3_originalCheck_coverage}
\lean{QEC1.FaultTolerantGaugingProcedure.phase3_originalCheck_coverage}
\leanok
\uses{def:FaultTolerantGaugingProcedure.deformedUngaugeDetector, def:FaultTolerantGaugingProcedure.phase3RepeatedDetector, def:FTGaugingMeasurement, def:PostDeformationMeasurement}
Phase~3 original check measurements are covered:
\begin{itemize}
  \item If $r = 0$, the measurement is in the deformed ungauge detector.
  \item If $0 < r$, the measurement is in a Phase~3 repeated detector pairing round $r-1$ with $r$.
\end{itemize}
\end{lemma}

\begin{proof}
\leanok
\uses{def:FaultTolerantGaugingProcedure.deformedUngaugeDetector, def:FaultTolerantGaugingProcedure.phase3RepeatedDetector}
For $r = 0$: the original check measurement is the second element of the insert pair in the deformed ungauge detector's left union component. For $0 < r$: it is the second element (singleton) of the pair in the Phase~3 repeated detector.
\end{proof}

\begin{lemma}[Phase 3 Edge $Z$ Coverage via Flux]
\label{lem:FaultTolerantGaugingProcedure.phase3_edgeZ_coverage_flux}
\lean{QEC1.FaultTolerantGaugingProcedure.phase3_edgeZ_coverage_flux}
\leanok
\uses{def:FaultTolerantGaugingProcedure.fluxUngaugeDetector, def:FTGaugingMeasurement, def:PostDeformationMeasurement}
For any cycle $p$ containing edge $e$, the measurement $\texttt{phase3}(\texttt{edgeZ}(e))$ is in the flux ungauging detector for $p$.
\end{lemma}

\begin{proof}
\leanok
\uses{lem:FaultTolerantGaugingProcedure.fluxUngaugeDetector_has_edgeZ}
This follows directly from the membership lemma for the flux ungauge detector.
\end{proof}

\begin{lemma}[Phase 3 Edge $Z$ Coverage via Deformed]
\label{lem:FaultTolerantGaugingProcedure.phase3_edgeZ_coverage_deformed}
\lean{QEC1.FaultTolerantGaugingProcedure.phase3_edgeZ_coverage_deformed}
\leanok
\uses{def:FaultTolerantGaugingProcedure.deformedUngaugeDetector, def:FTGaugingMeasurement, def:PostDeformationMeasurement}
For any check $j$ with $\text{edgePath}(j, e) \neq 0$, the measurement $\texttt{phase3}(\texttt{edgeZ}(e))$ is in the deformed ungauging detector for $j$.
\end{lemma}

\begin{proof}
\leanok
\uses{lem:FaultTolerantGaugingProcedure.deformedUngaugeDetector_has_edgeZ}
This follows directly from the membership lemma for the deformed ungauge detector.
\end{proof}

\begin{lemma}[Edge Init Coverage via Flux]
\label{lem:FaultTolerantGaugingProcedure.edgeInit_coverage_flux}
\lean{QEC1.FaultTolerantGaugingProcedure.edgeInit_coverage_flux}
\leanok
\uses{def:FaultTolerantGaugingProcedure.fluxInitDetector, def:FTGaugingMeasurement}
For any cycle $p$ containing edge $e$, the edge initialization measurement $\texttt{edgeInit}(e)$ is in the flux initialization detector for $p$.
\end{lemma}

\begin{proof}
\leanok
\uses{lem:FaultTolerantGaugingProcedure.fluxInitDetector_has_edgeInit}
This follows directly from the membership lemma for the flux init detector.
\end{proof}

\begin{lemma}[Edge Init Coverage via Deformed]
\label{lem:FaultTolerantGaugingProcedure.edgeInit_coverage_deformed}
\lean{QEC1.FaultTolerantGaugingProcedure.edgeInit_coverage_deformed}
\leanok
\uses{def:FaultTolerantGaugingProcedure.deformedInitDetector, def:FTGaugingMeasurement}
For any check $j$ with $\text{edgePath}(j, e) \neq 0$, the edge initialization measurement $\texttt{edgeInit}(e)$ is in the deformed initialization detector for $j$.
\end{lemma}

\begin{proof}
\leanok
\uses{lem:FaultTolerantGaugingProcedure.deformedInitDetector_has_edgeInit}
This follows directly from the membership lemma for the deformed init detector.
\end{proof}

\begin{theorem}[Every Measurement Covered]
\label{thm:FaultTolerantGaugingProcedure.every_measurement_covered}
\lean{QEC1.FaultTolerantGaugingProcedure.every_measurement_covered}
\leanok
\uses{def:FaultTolerantGaugingProcedure, def:FaultTolerantGaugingProcedure.DetectorIndex, def:FaultTolerantGaugingProcedure.detectorOfIndex, def:FTGaugingMeasurement}
Assuming $d \geq 2$ and that every edge qubit is in at least one cycle, every measurement in the procedure participates in at least one of the listed detector generators. Formally: for every measurement label $m$, there exists a detector index $\text{idx}$ such that $m \in (\text{detectorOfIndex}(\text{idx})).\text{measurements}$.
\end{theorem}

\begin{proof}
\leanok
\uses{lem:FaultTolerantGaugingProcedure.phase1_coverage, lem:FaultTolerantGaugingProcedure.phase2_gauss_coverage, lem:FaultTolerantGaugingProcedure.phase2_flux_coverage, lem:FaultTolerantGaugingProcedure.phase2_deformed_coverage, lem:FaultTolerantGaugingProcedure.phase3_originalCheck_coverage, lem:FaultTolerantGaugingProcedure.phase3_edgeZ_coverage_flux, lem:FaultTolerantGaugingProcedure.edgeInit_coverage_flux, lem:FaultTolerantGaugingProcedure.deformedInitDetector_has_phase1}
We proceed by case analysis on the measurement label $m$.

\textbf{Case $m = \texttt{phase1}(j, r)$:} If $r + 1 < d$, we use the Phase~1 repeated detector pairing rounds $r$ and $r + 1$, which covers $m$ by the Phase~1 coverage lemma. If $r + 1 = d$ (so $r = d - 1$), we use the deformed init detector for $j$, and the membership follows from the Phase~1 membership lemma.

\textbf{Case $m = \texttt{edgeInit}(e)$:} By the edge coverage hypothesis, there exists a cycle $p$ with $e \in \text{cycles}(p)$. We use the flux init detector for $p$.

\textbf{Case $m = \texttt{phase2}(\texttt{gaussLaw}(v, r))$:} If $r + 1 < d$, we use the forward Gauss repeated detector. Otherwise $0 < r$ (since $d \geq 2$), and we use the backward Gauss repeated detector pairing $r - 1$ with $r$.

\textbf{Case $m = \texttt{phase2}(\texttt{flux}(p, r))$:} If $r = 0$, we use the flux init detector. If $r + 1 = d$, we use the flux ungauge detector. Otherwise $r + 1 < d$, and we use the flux repeated detector.

\textbf{Case $m = \texttt{phase2}(\texttt{deformed}(j, r))$:} If $r = 0$, we use the deformed init detector. If $r + 1 = d$, we use the deformed ungauge detector. Otherwise $r + 1 < d$, and we use the deformed repeated detector.

\textbf{Case $m = \texttt{phase3}(\texttt{edgeZ}(e))$:} By the edge coverage hypothesis, there exists a cycle $p$ with $e \in \text{cycles}(p)$. We use the flux ungauge detector for $p$.

\textbf{Case $m = \texttt{phase3}(\texttt{originalCheck}(j, r))$:} If $r = 0$, we use the deformed ungauge detector for $j$. If $0 < r$, we use the Phase~3 repeated detector pairing $r - 1$ with $r$.
\end{proof}

\begin{theorem}[Axiom: Generators Span All Detectors]
\label{thm:FaultTolerantGaugingProcedure.generators_span_all_detectors}
\lean{QEC1.generators_span_all_detectors}
\leanok
\uses{def:FaultTolerantGaugingProcedure, def:FaultTolerantGaugingProcedure.generatorMeasurements, def:Detector, def:Detector.isViolated, def:IsGeneratedBy}

\textbf{\color{red}[UNPROVEN AXIOM]}

For any fault-tolerant gauging procedure with $d \geq 2$, any detector $D$ satisfying $\neg D.\text{isViolated}(\emptyset)$ has its measurement set $\mathbb{Z}_2$-generated by the generator measurement sets. That is, $D.\text{measurements}$ can be expressed as a symmetric difference of generator measurement sets.

\textit{Note: This is stated as an axiom because the structural argument --- that within each phase, the only sources of deterministic measurement constraints are repeated measurements of self-inverse stabilizers and boundary initialization/readout relations --- relies on physical reasoning about eigenvalue outcomes that was not fully formalized.}
\end{theorem}

\begin{theorem}[Completeness of Spacetime Code Detectors]
\label{thm:FaultTolerantGaugingProcedure.completeness}
\lean{QEC1.FaultTolerantGaugingProcedure.completeness}
\leanok
\uses{def:FaultTolerantGaugingProcedure, def:FaultTolerantGaugingProcedure.DetectorIndex, def:FaultTolerantGaugingProcedure.detectorOfIndex, def:FaultTolerantGaugingProcedure.generatorMeasurements, def:Detector, def:Detector.isViolated, def:IsGeneratedBy}
Assuming $d \geq 2$ and that every edge is in some cycle, the listed detectors form a complete generating set for the fault-tolerant gauging procedure. Specifically:
\begin{enumerate}
  \item \textbf{Validity}: Every generator is a valid detector (not violated without faults).
  \item \textbf{Coverage}: Every measurement participates in at least one generator.
  \item \textbf{$\mathbb{Z}_2$ closure}: The $\mathbb{Z}_2$ span is closed under symmetric difference.
  \item \textbf{Generation}: Every valid detector decomposes as a $\mathbb{Z}_2$ combination of the generators.
\end{enumerate}
\end{theorem}

\begin{proof}
\leanok
\uses{thm:FaultTolerantGaugingProcedure.detectorOfIndex_no_fault, thm:FaultTolerantGaugingProcedure.every_measurement_covered, lem:IsGeneratedBy.symmDiff_closure, thm:FaultTolerantGaugingProcedure.generators_span_all_detectors}

\textbf{\color{orange}[Uses unproven axiom: thm:FaultTolerantGaugingProcedure.generators\_span\_all\_detectors]}

The four parts are established as follows:
\begin{enumerate}
  \item For validity, every generator detector is not violated without faults, by the theorem \texttt{detectorOfIndex\_no\_fault}.
  \item For coverage, every measurement participates in some generator, by the theorem \texttt{every\_measurement\_covered} applied with the hypotheses $d \geq 2$ and edge coverage.
  \item For $\mathbb{Z}_2$ closure, given $S$ and $T$ in the $\mathbb{Z}_2$ span, $S \mathbin{\triangle} T$ is also in the span by the symmetric difference closure lemma.
  \item For generation, any valid detector $D$ with $\neg D.\text{isViolated}(\emptyset)$ has its measurement set generated by the generators, by the axiom \texttt{generators\_span\_all\_detectors}.
\end{enumerate}
\end{proof}

%--- Def_11: SpacetimeLogicalFault ---
\chapter{Def 11: Spacetime Logical Fault}

A \textbf{spacetime logical fault} is a collection of space-faults and time-faults (Def~7) that:
\begin{enumerate}
\item Does not violate any detector (i.e., the syndrome (Def~9) is empty), AND
\item Changes the outcome of the fault-tolerant gauging measurement procedure (Def~10).
\end{enumerate}
A \textbf{spacetime stabilizer} is a collection of space-faults and time-faults that:
\begin{enumerate}
\item Does not violate any detector, AND
\item Does NOT change the outcome of the procedure.
\end{enumerate}
Every syndrome-free fault collection is either a spacetime logical fault or a spacetime stabilizer.

\begin{definition}[Syndrome-Free Fault]
\label{def:SpacetimeLogicalFault.IsSyndromeFree}
\lean{QEC1.SpacetimeLogicalFault.IsSyndromeFree}
\leanok
\uses{def:syndromeFault, def:SpacetimeFault}
A spacetime fault $F$ is \textbf{syndrome-free} with respect to a collection of detectors $(\delta_i)_{i \in I}$ if the syndrome is empty, i.e.,
\[
\operatorname{syndromeFault}(\delta, F) = \emptyset.
\]
Equivalently, no detector is violated by the time-faults of $F$.
\end{definition}

\begin{theorem}[Syndrome-Free iff All Detectors Not Violated]
\label{thm:SpacetimeLogicalFault.isSyndromeFree_iff}
\lean{QEC1.SpacetimeLogicalFault.isSyndromeFree_iff}
\leanok
\uses{def:SpacetimeLogicalFault.IsSyndromeFree, def:syndromeFault, def:SpacetimeFault}
A fault $F$ is syndrome-free if and only if for every detector index $i \in I$, the detector $\delta_i$ is not violated by $F$'s time-faults:
\[
\operatorname{IsSyndromeFree}(\delta, F) \iff \forall\, i \in I,\;\neg\, \delta_i.\!\operatorname{isViolated}(F.\!\operatorname{timeFaults}).
\]
\end{theorem}

\begin{proof}
\leanok
\uses{def:SpacetimeLogicalFault.IsSyndromeFree, def:syndromeFault}
Unfolding the definitions of $\operatorname{IsSyndromeFree}$ and $\operatorname{syndromeFault}$, the condition that the filter of violated detectors equals the empty set is equivalent, by simplification using the universal membership and true-implies rules, to the statement that no detector is violated.
\end{proof}

\begin{theorem}[Syndrome-Free iff Syndrome Vector is Zero]
\label{thm:SpacetimeLogicalFault.isSyndromeFree_iff_syndromeVec}
\lean{QEC1.SpacetimeLogicalFault.isSyndromeFree_iff_syndromeVec}
\leanok
\uses{def:SpacetimeLogicalFault.IsSyndromeFree, def:syndromeVec, def:SpacetimeFault}
A fault $F$ is syndrome-free if and only if the syndrome vector is identically zero:
\[
\operatorname{IsSyndromeFree}(\delta, F) \iff \operatorname{syndromeVec}(\delta, F) = 0.
\]
\end{theorem}

\begin{proof}
\leanok
\uses{thm:SpacetimeLogicalFault.isSyndromeFree_iff, def:syndromeVec}
We rewrite using the characterization that syndrome-free means every detector is not violated. For the forward direction, assume every detector is not violated. By extensionality, for each index $i$, the syndrome vector at $i$ is defined by an if-then-else on whether $\delta_i$ is violated; since it is not, the value is $0$, so the entire vector is $0$. For the reverse direction, assume the syndrome vector equals $0$. For any index $i$, we extract the component at $i$, which equals $0$. Suppose for contradiction that $\delta_i$ is violated; then the if-branch gives value $1$, contradicting $1 \neq 0$.
\end{proof}

\begin{definition}[Outcome-Preserving Fault]
\label{def:SpacetimeLogicalFault.IsOutcomePreserving}
\lean{QEC1.SpacetimeLogicalFault.IsOutcomePreserving}
\leanok
\uses{def:SpacetimeFault}
A spacetime fault $F$ is \textbf{outcome-preserving} with respect to a predicate $\operatorname{outcomePreserving}$ if $\operatorname{outcomePreserving}(F)$ holds. This means the net effect of $F$ on the procedure is trivial: it preserves both the measurement sign and the logical information in the post-measurement state.
\end{definition}

\begin{definition}[Outcome-Changing Fault]
\label{def:SpacetimeLogicalFault.IsOutcomeChanging}
\lean{QEC1.SpacetimeLogicalFault.IsOutcomeChanging}
\leanok
\uses{def:SpacetimeFault}
A spacetime fault $F$ \textbf{changes the outcome} if it is not outcome-preserving:
\[
\operatorname{IsOutcomeChanging}(F) \;\coloneqq\; \neg\,\operatorname{outcomePreserving}(F).
\]
\end{definition}

\begin{definition}[Spacetime Logical Fault]
\label{def:SpacetimeLogicalFault.IsSpacetimeLogicalFault}
\lean{QEC1.SpacetimeLogicalFault.IsSpacetimeLogicalFault}
\leanok
\uses{def:SpacetimeLogicalFault.IsSyndromeFree, def:SpacetimeLogicalFault.IsOutcomeChanging, def:SpacetimeFault}
A \textbf{spacetime logical fault} is a spacetime fault $F$ satisfying:
\begin{enumerate}
\item $\operatorname{IsSyndromeFree}(\delta, F)$: No detector is violated (syndrome is empty).
\item $\operatorname{IsOutcomeChanging}(F)$: The fault changes the measurement result or applies a non-trivial logical operator to the post-measurement state.
\end{enumerate}
\end{definition}

\begin{definition}[Spacetime Stabilizer]
\label{def:SpacetimeLogicalFault.IsSpacetimeStabilizer}
\lean{QEC1.SpacetimeLogicalFault.IsSpacetimeStabilizer}
\leanok
\uses{def:SpacetimeLogicalFault.IsSyndromeFree, def:SpacetimeLogicalFault.IsOutcomePreserving, def:SpacetimeFault}
A \textbf{spacetime stabilizer} is a spacetime fault $F$ satisfying:
\begin{enumerate}
\item $\operatorname{IsSyndromeFree}(\delta, F)$: No detector is violated (syndrome is empty).
\item $\operatorname{IsOutcomePreserving}(F)$: The fault preserves both the measurement result and the logical information in the post-measurement state.
\end{enumerate}
\end{definition}

\begin{theorem}[Syndrome-Free Dichotomy]
\label{thm:SpacetimeLogicalFault.syndromeFree_dichotomy}
\lean{QEC1.SpacetimeLogicalFault.syndromeFree_dichotomy}
\leanok
\uses{def:SpacetimeLogicalFault.IsSyndromeFree, def:SpacetimeLogicalFault.IsSpacetimeLogicalFault, def:SpacetimeLogicalFault.IsSpacetimeStabilizer}
Every syndrome-free fault is either a spacetime logical fault or a spacetime stabilizer: if $\operatorname{IsSyndromeFree}(\delta, F)$ holds, then
\[
\operatorname{IsSpacetimeLogicalFault}(\delta, F) \;\lor\; \operatorname{IsSpacetimeStabilizer}(\delta, F).
\]
\end{theorem}

\begin{proof}
\leanok
\uses{def:SpacetimeLogicalFault.IsSpacetimeLogicalFault, def:SpacetimeLogicalFault.IsSpacetimeStabilizer}
We consider two cases on whether $\operatorname{outcomePreserving}(F)$ holds. If it does, then $F$ is a spacetime stabilizer (the right disjunct), constructed from the syndrome-free hypothesis and the outcome-preserving property. If it does not, then $F$ is a spacetime logical fault (the left disjunct), constructed from the syndrome-free hypothesis and the negation of outcome-preserving.
\end{proof}

\begin{theorem}[Logical Fault Not Stabilizer]
\label{thm:SpacetimeLogicalFault.logicalFault_not_stabilizer}
\lean{QEC1.SpacetimeLogicalFault.logicalFault_not_stabilizer}
\leanok
\uses{def:SpacetimeLogicalFault.IsSpacetimeLogicalFault, def:SpacetimeLogicalFault.IsSpacetimeStabilizer}
The dichotomy is exclusive: if $F$ is a spacetime logical fault, then it is not a spacetime stabilizer.
\end{theorem}

\begin{proof}
\leanok
\uses{def:SpacetimeLogicalFault.IsSpacetimeLogicalFault, def:SpacetimeLogicalFault.IsSpacetimeStabilizer}
Assume $F$ is a spacetime logical fault and suppose for contradiction it is also a stabilizer. Then $F$ is outcome-preserving (from the stabilizer condition), but $F$ is also outcome-changing (from the logical fault condition). The outcome-changing condition is the negation of outcome-preserving, giving a contradiction.
\end{proof}

\begin{theorem}[Stabilizer Not Logical Fault]
\label{thm:SpacetimeLogicalFault.stabilizer_not_logicalFault}
\lean{QEC1.SpacetimeLogicalFault.stabilizer_not_logicalFault}
\leanok
\uses{def:SpacetimeLogicalFault.IsSpacetimeLogicalFault, def:SpacetimeLogicalFault.IsSpacetimeStabilizer}
Conversely, if $F$ is a spacetime stabilizer, then it is not a spacetime logical fault.
\end{theorem}

\begin{proof}
\leanok
\uses{def:SpacetimeLogicalFault.IsSpacetimeLogicalFault, def:SpacetimeLogicalFault.IsSpacetimeStabilizer}
Assume $F$ is a stabilizer and suppose for contradiction it is also a logical fault. Then the logical fault's outcome-changing condition contradicts the stabilizer's outcome-preserving condition.
\end{proof}

\begin{lemma}[Logical Fault is Syndrome-Free]
\label{lem:SpacetimeLogicalFault.IsSpacetimeLogicalFault.syndromeFree}
\lean{QEC1.SpacetimeLogicalFault.IsSpacetimeLogicalFault.syndromeFree}
\leanok
\uses{def:SpacetimeLogicalFault.IsSpacetimeLogicalFault, def:SpacetimeLogicalFault.IsSyndromeFree}
A spacetime logical fault is syndrome-free.
\end{lemma}

\begin{proof}
\leanok
\uses{def:SpacetimeLogicalFault.IsSpacetimeLogicalFault}
This is the first component of the conjunction defining a spacetime logical fault.
\end{proof}

\begin{lemma}[Logical Fault Changes Outcome]
\label{lem:SpacetimeLogicalFault.IsSpacetimeLogicalFault.changesOutcome}
\lean{QEC1.SpacetimeLogicalFault.IsSpacetimeLogicalFault.changesOutcome}
\leanok
\uses{def:SpacetimeLogicalFault.IsSpacetimeLogicalFault, def:SpacetimeLogicalFault.IsOutcomeChanging}
A spacetime logical fault changes the outcome.
\end{lemma}

\begin{proof}
\leanok
\uses{def:SpacetimeLogicalFault.IsSpacetimeLogicalFault}
This is the second component of the conjunction defining a spacetime logical fault.
\end{proof}

\begin{lemma}[Stabilizer is Syndrome-Free]
\label{lem:SpacetimeLogicalFault.IsSpacetimeStabilizer.syndromeFree}
\lean{QEC1.SpacetimeLogicalFault.IsSpacetimeStabilizer.syndromeFree}
\leanok
\uses{def:SpacetimeLogicalFault.IsSpacetimeStabilizer, def:SpacetimeLogicalFault.IsSyndromeFree}
A spacetime stabilizer is syndrome-free.
\end{lemma}

\begin{proof}
\leanok
\uses{def:SpacetimeLogicalFault.IsSpacetimeStabilizer}
This is the first component of the conjunction defining a spacetime stabilizer.
\end{proof}

\begin{lemma}[Stabilizer Preserves Outcome]
\label{lem:SpacetimeLogicalFault.IsSpacetimeStabilizer.preservesOutcome}
\lean{QEC1.SpacetimeLogicalFault.IsSpacetimeStabilizer.preservesOutcome}
\leanok
\uses{def:SpacetimeLogicalFault.IsSpacetimeStabilizer, def:SpacetimeLogicalFault.IsOutcomePreserving}
A spacetime stabilizer preserves the outcome.
\end{lemma}

\begin{proof}
\leanok
\uses{def:SpacetimeLogicalFault.IsSpacetimeStabilizer}
This is the second component of the conjunction defining a spacetime stabilizer.
\end{proof}

\begin{theorem}[Empty Fault is Syndrome-Free]
\label{thm:SpacetimeLogicalFault.empty_isSyndromeFree}
\lean{QEC1.SpacetimeLogicalFault.empty_isSyndromeFree}
\leanok
\uses{def:SpacetimeLogicalFault.IsSyndromeFree, def:SpacetimeFault.empty, thm:syndromeFault_empty}
The empty spacetime fault has empty syndrome: no faults means no detector violations.
\end{theorem}

\begin{proof}
\leanok
\uses{def:SpacetimeLogicalFault.IsSyndromeFree, thm:syndromeFault_empty}
Unfolding the definition of $\operatorname{IsSyndromeFree}$, we rewrite using the fact that the syndrome of the empty fault is empty ($\operatorname{syndromeFault\_empty}$).
\end{proof}

\begin{theorem}[Empty Fault is a Spacetime Stabilizer]
\label{thm:SpacetimeLogicalFault.empty_isSpacetimeStabilizer}
\lean{QEC1.SpacetimeLogicalFault.empty_isSpacetimeStabilizer}
\leanok
\uses{def:SpacetimeLogicalFault.IsSpacetimeStabilizer, def:SpacetimeFault.empty, thm:SpacetimeLogicalFault.empty_isSyndromeFree}
The empty fault is a spacetime stabilizer, provided it is outcome-preserving (the natural requirement that no fault means no change).
\end{theorem}

\begin{proof}
\leanok
\uses{thm:SpacetimeLogicalFault.empty_isSyndromeFree}
The proof constructs the pair: the first component is $\operatorname{empty\_isSyndromeFree}$ applied to the detectors, and the second component is the hypothesis that the empty fault is outcome-preserving.
\end{proof}

\begin{theorem}[Syndrome-Free Fault of Weight Zero is Empty]
\label{thm:SpacetimeLogicalFault.syndromeFree_weight_zero_eq_empty}
\lean{QEC1.SpacetimeLogicalFault.syndromeFree_weight_zero_eq_empty}
\leanok
\uses{def:SpacetimeLogicalFault.IsSyndromeFree, def:SpacetimeFault.empty, def:SpacetimeFault.weight}
A syndrome-free fault of weight $0$ equals the empty fault.
\end{theorem}

\begin{proof}
\leanok
\uses{lem:SpacetimeFault.weight_zero_iff_empty}
Rewriting the hypothesis $F.\!\operatorname{weight} = 0$ using the lemma $\operatorname{weight\_zero\_iff\_empty}$, the result follows directly.
\end{proof}

\begin{theorem}[Spacetime Logical Fault Has Positive Weight]
\label{thm:SpacetimeLogicalFault.IsSpacetimeLogicalFault.weight_pos}
\lean{QEC1.SpacetimeLogicalFault.IsSpacetimeLogicalFault.weight_pos}
\leanok
\uses{def:SpacetimeLogicalFault.IsSpacetimeLogicalFault, def:SpacetimeFault.weight, def:SpacetimeFault.empty}
A spacetime logical fault must have positive weight: $0 < F.\!\operatorname{weight}$. It cannot be the empty fault.
\end{theorem}

\begin{proof}
\leanok
\uses{thm:SpacetimeLogicalFault.syndromeFree_weight_zero_eq_empty, def:SpacetimeLogicalFault.IsSpacetimeLogicalFault}
Suppose for contradiction that $F.\!\operatorname{weight} \leq 0$. Pushing the negation, we get $F.\!\operatorname{weight} = 0$ (since weight is a natural number). By the theorem $\operatorname{syndromeFree\_weight\_zero\_eq\_empty}$ applied to $F$'s syndrome-free property, we conclude $F = \operatorname{empty}$. Substituting, the logical fault condition requires $\neg\,\operatorname{outcomePreserving}(\operatorname{empty})$, which contradicts the hypothesis that the empty fault is outcome-preserving.
\end{proof}

\begin{theorem}[Logical Fault Has Zero Syndrome Vector]
\label{thm:SpacetimeLogicalFault.IsSpacetimeLogicalFault.syndromeVec_eq_zero}
\lean{QEC1.SpacetimeLogicalFault.IsSpacetimeLogicalFault.syndromeVec_eq_zero}
\leanok
\uses{def:SpacetimeLogicalFault.IsSpacetimeLogicalFault, def:syndromeVec}
A spacetime logical fault has zero syndrome vector: $\operatorname{syndromeVec}(\delta, F) = 0$.
\end{theorem}

\begin{proof}
\leanok
\uses{thm:SpacetimeLogicalFault.isSyndromeFree_iff_syndromeVec}
This follows by applying the forward direction of $\operatorname{isSyndromeFree\_iff\_syndromeVec}$ to the syndrome-free component of the logical fault hypothesis.
\end{proof}

\begin{theorem}[Stabilizer Has Zero Syndrome Vector]
\label{thm:SpacetimeLogicalFault.IsSpacetimeStabilizer.syndromeVec_eq_zero}
\lean{QEC1.SpacetimeLogicalFault.IsSpacetimeStabilizer.syndromeVec_eq_zero}
\leanok
\uses{def:SpacetimeLogicalFault.IsSpacetimeStabilizer, def:syndromeVec}
A spacetime stabilizer has zero syndrome vector: $\operatorname{syndromeVec}(\delta, F) = 0$.
\end{theorem}

\begin{proof}
\leanok
\uses{thm:SpacetimeLogicalFault.isSyndromeFree_iff_syndromeVec}
This follows by applying the forward direction of $\operatorname{isSyndromeFree\_iff\_syndromeVec}$ to the syndrome-free component of the stabilizer hypothesis.
\end{proof}

\begin{theorem}[Syndrome-Free iff Logical Fault or Stabilizer]
\label{thm:SpacetimeLogicalFault.syndromeFree_iff_logicalFault_or_stabilizer}
\lean{QEC1.SpacetimeLogicalFault.syndromeFree_iff_logicalFault_or_stabilizer}
\leanok
\uses{def:SpacetimeLogicalFault.IsSyndromeFree, def:SpacetimeLogicalFault.IsSpacetimeLogicalFault, def:SpacetimeLogicalFault.IsSpacetimeStabilizer}
The set of all syndrome-free faults partitions into logical faults and stabilizers:
\[
\operatorname{IsSyndromeFree}(\delta, F) \iff \operatorname{IsSpacetimeLogicalFault}(\delta, F) \lor \operatorname{IsSpacetimeStabilizer}(\delta, F).
\]
\end{theorem}

\begin{proof}
\leanok
\uses{thm:SpacetimeLogicalFault.syndromeFree_dichotomy}
For the forward direction, we apply $\operatorname{syndromeFree\_dichotomy}$. For the reverse direction, we case-split on whether $F$ is a logical fault or a stabilizer; in either case, the first component of the conjunction gives syndrome-freeness.
\end{proof}

\begin{theorem}[Syndrome-Free: Logical Fault iff Not Stabilizer]
\label{thm:SpacetimeLogicalFault.syndromeFree_logicalFault_iff_not_stabilizer}
\lean{QEC1.SpacetimeLogicalFault.syndromeFree_logicalFault_iff_not_stabilizer}
\leanok
\uses{def:SpacetimeLogicalFault.IsSyndromeFree, def:SpacetimeLogicalFault.IsSpacetimeLogicalFault, def:SpacetimeLogicalFault.IsSpacetimeStabilizer}
A syndrome-free fault is a logical fault if and only if it is not a stabilizer:
\[
\operatorname{IsSyndromeFree}(\delta, F) \;\Longrightarrow\; \bigl(\operatorname{IsSpacetimeLogicalFault}(\delta, F) \iff \neg\,\operatorname{IsSpacetimeStabilizer}(\delta, F)\bigr).
\]
\end{theorem}

\begin{proof}
\leanok
\uses{thm:SpacetimeLogicalFault.logicalFault_not_stabilizer}
For the forward direction, we apply $\operatorname{logicalFault\_not\_stabilizer}$. For the reverse direction, assume $F$ is not a stabilizer. We construct the logical fault: the first component is the syndrome-free hypothesis. For the second component (outcome-changing), assume for contradiction that $F$ is outcome-preserving. Then $F$ would be a stabilizer (pairing syndrome-free with outcome-preserving), contradicting the hypothesis.
\end{proof}

\begin{theorem}[Syndrome-Free: Stabilizer iff Not Logical Fault]
\label{thm:SpacetimeLogicalFault.syndromeFree_stabilizer_iff_not_logicalFault}
\lean{QEC1.SpacetimeLogicalFault.syndromeFree_stabilizer_iff_not_logicalFault}
\leanok
\uses{def:SpacetimeLogicalFault.IsSyndromeFree, def:SpacetimeLogicalFault.IsSpacetimeLogicalFault, def:SpacetimeLogicalFault.IsSpacetimeStabilizer}
A syndrome-free fault is a stabilizer if and only if it is not a logical fault:
\[
\operatorname{IsSyndromeFree}(\delta, F) \;\Longrightarrow\; \bigl(\operatorname{IsSpacetimeStabilizer}(\delta, F) \iff \neg\,\operatorname{IsSpacetimeLogicalFault}(\delta, F)\bigr).
\]
\end{theorem}

\begin{proof}
\leanok
\uses{thm:SpacetimeLogicalFault.logicalFault_not_stabilizer}
For the forward direction, assume $F$ is a stabilizer and suppose $F$ is also a logical fault; we apply $\operatorname{logicalFault\_not\_stabilizer}$ to derive a contradiction. For the reverse direction, assume $F$ is not a logical fault. We construct the stabilizer: the first component is syndrome-freeness. For the second (outcome-preserving), suppose for contradiction that $F$ is not outcome-preserving; then $F$ would be a logical fault, contradicting the hypothesis.
\end{proof}

\begin{definition}[Gauss Sign Flip Indicator]
\label{def:SpacetimeLogicalFault.gaussSignFlip}
\lean{QEC1.SpacetimeLogicalFault.gaussSignFlip}
\leanok
\uses{def:FaultTolerantGaugingProcedure, def:SpacetimeFault, def:FTGaugingMeasurement, def:DeformedCodeMeasurement}
The \textbf{sign flip indicator} for the fault-tolerant gauging procedure is defined as the parity of time-faults that affect Gauss's law measurements:
\[
\operatorname{gaussSignFlip}(F) \;=\; \sum_{v \in V} \sum_{r=0}^{d-1} \begin{cases} 1 & \text{if the time-fault for Gauss's law measurement $(v, r)$ in phase~2 is in } F.\!\operatorname{timeFaults}, \\ 0 & \text{otherwise,} \end{cases}
\]
computed in $\mathbb{Z}/2\mathbb{Z}$. The gauging sign $\sigma = \sum_v \varepsilon_v$, and a time-fault on a Gauss measurement flips one $\varepsilon_v$.
\end{definition}

\begin{definition}[Flips Gauging Sign]
\label{def:SpacetimeLogicalFault.FlipsGaugingSign}
\lean{QEC1.SpacetimeLogicalFault.FlipsGaugingSign}
\leanok
\uses{def:SpacetimeLogicalFault.gaussSignFlip, def:FaultTolerantGaugingProcedure}
A fault \textbf{flips the gauging sign} if the sign flip indicator equals $1$:
\[
\operatorname{FlipsGaugingSign}(F) \;\coloneqq\; \operatorname{gaussSignFlip}(F) = 1.
\]
\end{definition}

\begin{definition}[Preserves Gauging Sign]
\label{def:SpacetimeLogicalFault.PreservesGaugingSign}
\lean{QEC1.SpacetimeLogicalFault.PreservesGaugingSign}
\leanok
\uses{def:SpacetimeLogicalFault.gaussSignFlip, def:FaultTolerantGaugingProcedure}
A fault \textbf{preserves the gauging sign} if the sign flip indicator equals $0$:
\[
\operatorname{PreservesGaugingSign}(F) \;\coloneqq\; \operatorname{gaussSignFlip}(F) = 0.
\]
\end{definition}

\begin{theorem}[Sign Flip is Zero or One]
\label{thm:SpacetimeLogicalFault.gaussSignFlip_zero_or_one}
\lean{QEC1.SpacetimeLogicalFault.gaussSignFlip_zero_or_one}
\leanok
\uses{def:SpacetimeLogicalFault.gaussSignFlip, def:FaultTolerantGaugingProcedure}
The sign flip indicator is always $0$ or $1$ in $\mathbb{Z}/2\mathbb{Z}$:
\[
\operatorname{gaussSignFlip}(F) = 0 \;\lor\; \operatorname{gaussSignFlip}(F) = 1.
\]
\end{theorem}

\begin{proof}
\leanok
\uses{def:SpacetimeLogicalFault.gaussSignFlip}
Every element $x \in \mathbb{Z}/2\mathbb{Z}$ satisfies $x = 0 \lor x = 1$. This is verified by case analysis on the two elements of $\mathbb{Z}/2\mathbb{Z}$ followed by simplification. The result then follows by applying this fact to $\operatorname{gaussSignFlip}(F)$.
\end{proof}

\begin{theorem}[Flips or Preserves Dichotomy]
\label{thm:SpacetimeLogicalFault.flipsOrPreserves}
\lean{QEC1.SpacetimeLogicalFault.flipsOrPreserves}
\leanok
\uses{def:SpacetimeLogicalFault.FlipsGaugingSign, def:SpacetimeLogicalFault.PreservesGaugingSign, def:FaultTolerantGaugingProcedure}
A fault either flips or preserves the gauging sign:
\[
\operatorname{FlipsGaugingSign}(F) \;\lor\; \operatorname{PreservesGaugingSign}(F).
\]
\end{theorem}

\begin{proof}
\leanok
\uses{thm:SpacetimeLogicalFault.gaussSignFlip_zero_or_one}
We case-split on whether $\operatorname{gaussSignFlip}(F)$ is $0$ or $1$ using $\operatorname{gaussSignFlip\_zero\_or\_one}$. If it is $0$, then $F$ preserves the gauging sign (the right disjunct). If it is $1$, then $F$ flips the gauging sign (the left disjunct).
\end{proof}

\begin{theorem}[Flipping Excludes Preserving]
\label{thm:SpacetimeLogicalFault.flipsGaugingSign_not_preserves}
\lean{QEC1.SpacetimeLogicalFault.flipsGaugingSign_not_preserves}
\leanok
\uses{def:SpacetimeLogicalFault.FlipsGaugingSign, def:SpacetimeLogicalFault.PreservesGaugingSign, def:FaultTolerantGaugingProcedure}
If a fault flips the gauging sign, then it does not preserve it.
\end{theorem}

\begin{proof}
\leanok
\uses{def:SpacetimeLogicalFault.FlipsGaugingSign, def:SpacetimeLogicalFault.PreservesGaugingSign}
Assume $F$ flips the gauging sign, so $\operatorname{gaussSignFlip}(F) = 1$. Suppose for contradiction that $F$ also preserves the gauging sign, so $\operatorname{gaussSignFlip}(F) = 0$. Rewriting, we get $0 = 1$, which contradicts $0 \neq 1$.
\end{proof}

\begin{theorem}[Preserving Excludes Flipping]
\label{thm:SpacetimeLogicalFault.preservesGaugingSign_not_flips}
\lean{QEC1.SpacetimeLogicalFault.preservesGaugingSign_not_flips}
\leanok
\uses{def:SpacetimeLogicalFault.FlipsGaugingSign, def:SpacetimeLogicalFault.PreservesGaugingSign, def:FaultTolerantGaugingProcedure}
If a fault preserves the gauging sign, then it does not flip it.
\end{theorem}

\begin{proof}
\leanok
\uses{thm:SpacetimeLogicalFault.flipsGaugingSign_not_preserves}
Assume $F$ preserves the gauging sign and suppose for contradiction that $F$ flips it. By $\operatorname{flipsGaugingSign\_not\_preserves}$, flipping implies not preserving, contradicting our assumption.
\end{proof}

\begin{definition}[Syndrome-Free in Gauging Procedure]
\label{def:SpacetimeLogicalFault.IsSyndromeFreeGauging}
\lean{QEC1.SpacetimeLogicalFault.IsSyndromeFreeGauging}
\leanok
\uses{def:FaultTolerantGaugingProcedure, def:FaultTolerantGaugingProcedure.DetectorIndex, def:SpacetimeFault}
A spacetime fault $F$ is \textbf{syndrome-free in the gauging procedure} if no detector from the spacetime code is violated:
\[
\forall\, \operatorname{idx} : \operatorname{DetectorIndex},\; \neg\,(\delta_{\operatorname{idx}}).\!\operatorname{isViolated}(F.\!\operatorname{timeFaults}).
\]
Here the detector index type is the specialized $\operatorname{DetectorIndex}\;V\;C\;J\;G.\!\operatorname{edgeSet}\;d$ from the fault-tolerant gauging procedure.
\end{definition}

\begin{definition}[Gauging Logical Fault]
\label{def:SpacetimeLogicalFault.IsGaugingLogicalFault}
\lean{QEC1.SpacetimeLogicalFault.IsGaugingLogicalFault}
\leanok
\uses{def:SpacetimeLogicalFault.IsSyndromeFreeGauging, def:FaultTolerantGaugingProcedure, def:SpacetimeFault}
A \textbf{gauging logical fault} is a spacetime fault in the gauging procedure that is syndrome-free AND changes the outcome:
\[
\operatorname{IsGaugingLogicalFault}(F) \;\coloneqq\; \operatorname{IsSyndromeFreeGauging}(F) \;\land\; \neg\,\operatorname{outcomePreserving}(F).
\]
\end{definition}

\begin{definition}[Gauging Stabilizer]
\label{def:SpacetimeLogicalFault.IsGaugingStabilizer}
\lean{QEC1.SpacetimeLogicalFault.IsGaugingStabilizer}
\leanok
\uses{def:SpacetimeLogicalFault.IsSyndromeFreeGauging, def:FaultTolerantGaugingProcedure, def:SpacetimeFault}
A \textbf{gauging stabilizer} is a spacetime fault in the gauging procedure that is syndrome-free AND preserves the outcome:
\[
\operatorname{IsGaugingStabilizer}(F) \;\coloneqq\; \operatorname{IsSyndromeFreeGauging}(F) \;\land\; \operatorname{outcomePreserving}(F).
\]
\end{definition}

\begin{theorem}[Gauging Syndrome-Free Dichotomy]
\label{thm:SpacetimeLogicalFault.gaugingSyndromeFree_dichotomy}
\lean{QEC1.SpacetimeLogicalFault.gaugingSyndromeFree_dichotomy}
\leanok
\uses{def:SpacetimeLogicalFault.IsSyndromeFreeGauging, def:SpacetimeLogicalFault.IsGaugingLogicalFault, def:SpacetimeLogicalFault.IsGaugingStabilizer, def:FaultTolerantGaugingProcedure}
Every syndrome-free fault in the gauging procedure is either a gauging logical fault or a gauging stabilizer.
\end{theorem}

\begin{proof}
\leanok
\uses{def:SpacetimeLogicalFault.IsGaugingLogicalFault, def:SpacetimeLogicalFault.IsGaugingStabilizer}
We case-split on whether $\operatorname{outcomePreserving}(F)$ holds. If yes, $F$ is a gauging stabilizer (right disjunct). If no, $F$ is a gauging logical fault (left disjunct).
\end{proof}

\begin{theorem}[Gauging Logical Fault Not Stabilizer]
\label{thm:SpacetimeLogicalFault.gaugingLogicalFault_not_stabilizer}
\lean{QEC1.SpacetimeLogicalFault.gaugingLogicalFault_not_stabilizer}
\leanok
\uses{def:SpacetimeLogicalFault.IsGaugingLogicalFault, def:SpacetimeLogicalFault.IsGaugingStabilizer, def:FaultTolerantGaugingProcedure}
The gauging dichotomy is exclusive: a gauging logical fault is not a gauging stabilizer.
\end{theorem}

\begin{proof}
\leanok
\uses{def:SpacetimeLogicalFault.IsGaugingLogicalFault, def:SpacetimeLogicalFault.IsGaugingStabilizer}
Assume $F$ is a gauging logical fault and suppose it is also a stabilizer. The logical fault condition gives $\neg\,\operatorname{outcomePreserving}(F)$, while the stabilizer condition gives $\operatorname{outcomePreserving}(F)$, a contradiction.
\end{proof}

\begin{theorem}[Gauging Stabilizer Not Logical Fault]
\label{thm:SpacetimeLogicalFault.gaugingStabilizer_not_logicalFault}
\lean{QEC1.SpacetimeLogicalFault.gaugingStabilizer_not_logicalFault}
\leanok
\uses{def:SpacetimeLogicalFault.IsGaugingLogicalFault, def:SpacetimeLogicalFault.IsGaugingStabilizer, def:FaultTolerantGaugingProcedure}
A gauging stabilizer is not a gauging logical fault.
\end{theorem}

\begin{proof}
\leanok
\uses{def:SpacetimeLogicalFault.IsGaugingLogicalFault, def:SpacetimeLogicalFault.IsGaugingStabilizer}
Assume $F$ is a gauging stabilizer and suppose it is also a logical fault. The logical fault's outcome-changing condition contradicts the stabilizer's outcome-preserving condition.
\end{proof}

\begin{theorem}[Gauging: Logical Fault iff Not Stabilizer]
\label{thm:SpacetimeLogicalFault.gaugingSyndromeFree_logicalFault_iff}
\lean{QEC1.SpacetimeLogicalFault.gaugingSyndromeFree_logicalFault_iff}
\leanok
\uses{def:SpacetimeLogicalFault.IsSyndromeFreeGauging, def:SpacetimeLogicalFault.IsGaugingLogicalFault, def:SpacetimeLogicalFault.IsGaugingStabilizer, def:FaultTolerantGaugingProcedure}
A syndrome-free gauging fault is a logical fault if and only if it is not a stabilizer.
\end{theorem}

\begin{proof}
\leanok
\uses{thm:SpacetimeLogicalFault.gaugingLogicalFault_not_stabilizer}
For the forward direction, apply $\operatorname{gaugingLogicalFault\_not\_stabilizer}$. For the reverse direction, assume $F$ is syndrome-free but not a stabilizer. We construct the logical fault: the first component is syndrome-freeness. For the second, assume for contradiction that $F$ is outcome-preserving; then pairing with syndrome-freeness gives a stabilizer, contradicting the hypothesis.
\end{proof}

\begin{theorem}[Gauging: Stabilizer iff Not Logical Fault]
\label{thm:SpacetimeLogicalFault.gaugingSyndromeFree_stabilizer_iff}
\lean{QEC1.SpacetimeLogicalFault.gaugingSyndromeFree_stabilizer_iff}
\leanok
\uses{def:SpacetimeLogicalFault.IsSyndromeFreeGauging, def:SpacetimeLogicalFault.IsGaugingLogicalFault, def:SpacetimeLogicalFault.IsGaugingStabilizer, def:FaultTolerantGaugingProcedure}
A syndrome-free gauging fault is a stabilizer if and only if it is not a logical fault.
\end{theorem}

\begin{proof}
\leanok
\uses{thm:SpacetimeLogicalFault.gaugingLogicalFault_not_stabilizer}
For the forward direction, assume $F$ is a stabilizer and suppose it is also a logical fault; we apply $\operatorname{gaugingLogicalFault\_not\_stabilizer}$ to derive a contradiction. For the reverse direction, assume $F$ is syndrome-free but not a logical fault. We construct the stabilizer: the first component is syndrome-freeness. For the second, suppose for contradiction that $F$ is not outcome-preserving; then pairing with syndrome-freeness gives a logical fault, contradicting the hypothesis.
\end{proof}

%--- Def_12: SpacetimeFaultDistance ---
\chapter{Def 12: Spacetime Fault-Distance}

The \textbf{spacetime fault-distance} of the fault-tolerant gauging measurement procedure is the minimum weight of a spacetime logical fault. The weight $|F|$ of a spacetime fault $F$ is the total number of elementary fault events: each single-qubit Pauli error (space-fault) counts as weight 1, each measurement error (time-fault) counts as weight 1, and each initialization error (time-fault) counts as weight 1. Formally,
\[
d_{\mathrm{spacetime}} = \min \{ |F| : F \text{ is a spacetime logical fault} \}.
\]
The distance is defined to be $0$ if no spacetime logical faults exist.

\begin{definition}[Logical Fault Weights]
\label{def:SpacetimeFaultDistance.logicalFaultWeights}
\lean{QEC1.SpacetimeFaultDistance.logicalFaultWeights}
\leanok
\uses{def:SpacetimeFault, def:SpacetimeLogicalFault.IsSpacetimeLogicalFault, def:SpacetimeFault.weight}
Given a collection of detectors indexed by $I$ and an outcome-preserving predicate, the \emph{set of logical fault weights} is
\[
\{ w \in \mathbb{N} \mid \exists\, F,\; F \text{ is a spacetime logical fault} \land |F| = w \}.
\]
\end{definition}

\begin{definition}[Spacetime Fault-Distance]
\label{def:SpacetimeFaultDistance.spacetimeFaultDistance}
\lean{QEC1.SpacetimeFaultDistance.spacetimeFaultDistance}
\leanok
\uses{def:SpacetimeFaultDistance.logicalFaultWeights}
The \emph{spacetime fault-distance} is
\[
d_{\mathrm{spacetime}} = \inf\, \bigl(\text{logicalFaultWeights}(\text{detectors}, \text{outcomePreserving})\bigr),
\]
where $\inf$ is the infimum over natural numbers (returning $0$ for the empty set).
\end{definition}

\begin{theorem}[Distance Upper Bound by Logical Fault Weight]
\label{thm:SpacetimeFaultDistance.spacetimeFaultDistance_le_of_logicalFault}
\lean{QEC1.SpacetimeFaultDistance.spacetimeFaultDistance_le_of_logicalFault}
\leanok
\uses{def:SpacetimeFaultDistance.spacetimeFaultDistance, def:SpacetimeLogicalFault.IsSpacetimeLogicalFault, def:SpacetimeFault.weight}
If $F$ is a spacetime logical fault, then $d_{\mathrm{spacetime}} \le |F|$.
\end{theorem}

\begin{proof}
\leanok
\uses{def:SpacetimeFaultDistance.spacetimeFaultDistance, def:SpacetimeFaultDistance.logicalFaultWeights}
We apply $\operatorname{Nat.sInf\_le}$: since $F$ is a spacetime logical fault, we have $|F| \in \text{logicalFaultWeights}$ (witnessed by $\langle F, h_{\mathrm{log}}, \operatorname{rfl}\rangle$), and therefore $\inf(\text{logicalFaultWeights}) \le |F|$.
\end{proof}

\begin{theorem}[Zero Distance When No Logical Faults Exist]
\label{thm:SpacetimeFaultDistance.spacetimeFaultDistance_eq_zero_of_no_logicalFaults}
\lean{QEC1.SpacetimeFaultDistance.spacetimeFaultDistance_eq_zero_of_no_logicalFaults}
\leanok
\uses{def:SpacetimeFaultDistance.spacetimeFaultDistance, def:SpacetimeLogicalFault.IsSpacetimeLogicalFault, def:SpacetimeFaultDistance.logicalFaultWeights}
If no spacetime logical faults exist, then $d_{\mathrm{spacetime}} = 0$.
\end{theorem}

\begin{proof}
\leanok
\uses{def:SpacetimeFaultDistance.spacetimeFaultDistance, def:SpacetimeFaultDistance.logicalFaultWeights}
We show the set of logical fault weights is empty. By extensionality, for any $w$, we simplify the membership condition and note that if $\langle F, h_{\mathrm{log}}, \_\rangle$ existed, it would contradict the hypothesis that no logical faults exist. Hence $\text{logicalFaultWeights} = \emptyset$. Unfolding the definition of $d_{\mathrm{spacetime}}$ and rewriting with this, we conclude by $\operatorname{Nat.sInf\_empty} = 0$.
\end{proof}

\begin{theorem}[Positive Distance When Logical Faults Exist]
\label{thm:SpacetimeFaultDistance.spacetimeFaultDistance_pos}
\lean{QEC1.SpacetimeFaultDistance.spacetimeFaultDistance_pos}
\leanok
\uses{def:SpacetimeFaultDistance.spacetimeFaultDistance, def:SpacetimeFaultDistance.logicalFaultWeights, def:SpacetimeLogicalFault.IsSpacetimeLogicalFault, def:SpacetimeFault.empty, def:SpacetimeFault.weight}
If the empty fault is outcome-preserving and there exists a spacetime logical fault, then $d_{\mathrm{spacetime}} > 0$.
\end{theorem}

\begin{proof}
\leanok
\uses{def:SpacetimeFaultDistance.spacetimeFaultDistance, def:SpacetimeFaultDistance.logicalFaultWeights, def:SpacetimeFault.empty}
First, we show $0 \notin \text{logicalFaultWeights}$: if $\langle F, h_{\mathrm{log}}, h_w\rangle$ witnessed $0 \in \text{logicalFaultWeights}$, then by the positivity of the weight of logical faults (since the empty fault is outcome-preserving), we get a contradiction by integer arithmetic. From the existence hypothesis, we obtain $F$ with $h_{\mathrm{log}}$, so $\text{logicalFaultWeights}$ is nonempty (witnessed by $|F|$). Rewriting the goal as $d_{\mathrm{spacetime}} \ne 0$, we assume for contradiction that $d_{\mathrm{spacetime}} = 0$. By $\operatorname{Nat.sInf\_eq\_zero}$, either $0 \in \text{logicalFaultWeights}$ (contradicting the above) or $\text{logicalFaultWeights} = \emptyset$ (contradicting nonemptiness).
\end{proof}

\begin{theorem}[Faults Below Distance Are Not Logical]
\label{thm:SpacetimeFaultDistance.not_logicalFault_of_weight_lt}
\lean{QEC1.SpacetimeFaultDistance.not_logicalFault_of_weight_lt}
\leanok
\uses{def:SpacetimeFaultDistance.spacetimeFaultDistance, def:SpacetimeLogicalFault.IsSpacetimeLogicalFault, def:SpacetimeFault.weight}
If $|F| < d_{\mathrm{spacetime}}$, then $F$ is not a spacetime logical fault.
\end{theorem}

\begin{proof}
\leanok
\uses{thm:SpacetimeFaultDistance.spacetimeFaultDistance_le_of_logicalFault}
Assume for contradiction that $F$ is a spacetime logical fault. Then by the upper bound theorem, $d_{\mathrm{spacetime}} \le |F|$, contradicting $|F| < d_{\mathrm{spacetime}}$ by integer arithmetic.
\end{proof}

\begin{theorem}[Syndrome-Free Below Distance Implies Stabilizer]
\label{thm:SpacetimeFaultDistance.syndromeFree_weight_lt_is_stabilizer}
\lean{QEC1.SpacetimeFaultDistance.syndromeFree_weight_lt_is_stabilizer}
\leanok
\uses{def:SpacetimeFaultDistance.spacetimeFaultDistance, def:SpacetimeLogicalFault.IsSyndromeFree, def:SpacetimeLogicalFault.IsSpacetimeStabilizer, thm:SpacetimeLogicalFault.syndromeFree_stabilizer_iff_not_logicalFault}
If $F$ is syndrome-free and $|F| < d_{\mathrm{spacetime}}$, then $F$ is a spacetime stabilizer.
\end{theorem}

\begin{proof}
\leanok
\uses{thm:SpacetimeFaultDistance.not_logicalFault_of_weight_lt, thm:SpacetimeLogicalFault.syndromeFree_stabilizer_iff_not_logicalFault}
Since $|F| < d_{\mathrm{spacetime}}$, we have by the previous theorem that $F$ is not a logical fault. Since $F$ is syndrome-free, we apply the characterization that a syndrome-free fault is a stabilizer if and only if it is not a logical fault, concluding that $F$ is a spacetime stabilizer.
\end{proof}

\begin{theorem}[Distance Bounded by Space Plus Time Weight]
\label{thm:SpacetimeFaultDistance.spacetimeFaultDistance_le_space_plus_time}
\lean{QEC1.SpacetimeFaultDistance.spacetimeFaultDistance_le_space_plus_time}
\leanok
\uses{def:SpacetimeFaultDistance.spacetimeFaultDistance, def:SpacetimeLogicalFault.IsSpacetimeLogicalFault, def:SpacetimeFault.spaceWeight, def:SpacetimeFault.timeWeight}
For any spacetime logical fault $F$,
\[
d_{\mathrm{spacetime}} \le |F|_{\mathrm{space}} + |F|_{\mathrm{time}}.
\]
\end{theorem}

\begin{proof}
\leanok
\uses{thm:SpacetimeFaultDistance.spacetimeFaultDistance_le_of_logicalFault}
From the upper bound theorem, $d_{\mathrm{spacetime}} \le |F|$. By simplification using the definitions of weight, space weight, and time weight (which gives $|F| = |F|_{\mathrm{space}} + |F|_{\mathrm{time}}$), the result follows directly.
\end{proof}

\begin{definition}[Gauging Logical Fault Weights]
\label{def:SpacetimeFaultDistance.gaugingLogicalFaultWeights}
\lean{QEC1.SpacetimeFaultDistance.gaugingLogicalFaultWeights}
\leanok
\uses{def:FaultTolerantGaugingProcedure, def:FaultTolerantGaugingProcedure.DetectorIndex, def:SpacetimeFault, def:SpacetimeLogicalFault.IsGaugingLogicalFault, def:SpacetimeFault.weight}
Given a fault-tolerant gauging procedure $\mathrm{proc}$, a collection of detectors indexed by $\operatorname{DetectorIndex}$, and an outcome-preserving predicate, the \emph{set of gauging logical fault weights} is
\[
\{ w \in \mathbb{N} \mid \exists\, F,\; F \text{ is a gauging logical fault for } \mathrm{proc} \land |F| = w \}.
\]
\end{definition}

\begin{definition}[Gauging Spacetime Fault-Distance]
\label{def:SpacetimeFaultDistance.gaugingSpacetimeFaultDistance}
\lean{QEC1.SpacetimeFaultDistance.gaugingSpacetimeFaultDistance}
\leanok
\uses{def:SpacetimeFaultDistance.gaugingLogicalFaultWeights, def:FaultTolerantGaugingProcedure}
The \emph{gauging spacetime fault-distance} is
\[
d_{\mathrm{spacetime}}^{\mathrm{gauging}} = \inf\, \bigl(\text{gaugingLogicalFaultWeights}(\mathrm{proc}, \text{detectors}, \text{outcomePreserving})\bigr).
\]
\end{definition}

\begin{theorem}[Gauging Distance Upper Bound]
\label{thm:SpacetimeFaultDistance.gaugingSpacetimeFaultDistance_le}
\lean{QEC1.SpacetimeFaultDistance.gaugingSpacetimeFaultDistance_le}
\leanok
\uses{def:SpacetimeFaultDistance.gaugingSpacetimeFaultDistance, def:SpacetimeLogicalFault.IsGaugingLogicalFault, def:FaultTolerantGaugingProcedure}
If $F$ is a gauging logical fault, then $d_{\mathrm{spacetime}}^{\mathrm{gauging}} \le |F|$.
\end{theorem}

\begin{proof}
\leanok
\uses{def:SpacetimeFaultDistance.gaugingSpacetimeFaultDistance, def:SpacetimeFaultDistance.gaugingLogicalFaultWeights}
We apply $\operatorname{Nat.sInf\_le}$: since $F$ is a gauging logical fault, $|F| \in \text{gaugingLogicalFaultWeights}$ (witnessed by $\langle F, h_{\mathrm{log}}, \operatorname{rfl}\rangle$), and therefore $\inf(\text{gaugingLogicalFaultWeights}) \le |F|$.
\end{proof}

\begin{theorem}[Gauging Distance Zero When No Faults]
\label{thm:SpacetimeFaultDistance.gaugingSpacetimeFaultDistance_eq_zero_of_no_faults}
\lean{QEC1.SpacetimeFaultDistance.gaugingSpacetimeFaultDistance_eq_zero_of_no_faults}
\leanok
\uses{def:SpacetimeFaultDistance.gaugingSpacetimeFaultDistance, def:SpacetimeLogicalFault.IsGaugingLogicalFault, def:SpacetimeFaultDistance.gaugingLogicalFaultWeights, def:FaultTolerantGaugingProcedure}
If no gauging logical faults exist, then $d_{\mathrm{spacetime}}^{\mathrm{gauging}} = 0$.
\end{theorem}

\begin{proof}
\leanok
\uses{def:SpacetimeFaultDistance.gaugingSpacetimeFaultDistance, def:SpacetimeFaultDistance.gaugingLogicalFaultWeights}
We show the set of gauging logical fault weights is empty. By extensionality, for any $w$, we simplify the membership condition and note that any witness $\langle F, h_{\mathrm{log}}, \_\rangle$ contradicts the hypothesis that no gauging logical faults exist. Hence $\text{gaugingLogicalFaultWeights} = \emptyset$. Unfolding $d_{\mathrm{spacetime}}^{\mathrm{gauging}}$ and rewriting, we conclude by $\operatorname{Nat.sInf\_empty} = 0$.
\end{proof}

\begin{theorem}[Positive Gauging Distance]
\label{thm:SpacetimeFaultDistance.gaugingSpacetimeFaultDistance_pos}
\lean{QEC1.SpacetimeFaultDistance.gaugingSpacetimeFaultDistance_pos}
\leanok
\uses{def:SpacetimeFaultDistance.gaugingSpacetimeFaultDistance, def:SpacetimeFaultDistance.gaugingLogicalFaultWeights, def:SpacetimeLogicalFault.IsGaugingLogicalFault, def:SpacetimeFault.empty, def:FaultTolerantGaugingProcedure}
If the empty fault is outcome-preserving and there exists a gauging logical fault, then $d_{\mathrm{spacetime}}^{\mathrm{gauging}} > 0$.
\end{theorem}

\begin{proof}
\leanok
\uses{def:SpacetimeFaultDistance.gaugingSpacetimeFaultDistance, def:SpacetimeFaultDistance.gaugingLogicalFaultWeights, def:SpacetimeFault.empty, def:SpacetimeFault.weight}
First, we show $0 \notin \text{gaugingLogicalFaultWeights}$. If $\langle F, h_{\mathrm{log}}, h_w\rangle$ witnessed $0$ in this set, then unfolding the definition of weight, $|F.\mathrm{spaceFaults}| + |F.\mathrm{timeFaults}| = 0$, so $F.\mathrm{spaceFaults} = \emptyset$ and $F.\mathrm{timeFaults} = \emptyset$ (by $\operatorname{Finset.card\_eq\_zero}$ and integer arithmetic). By cases on $F$, simplification yields $F = \text{empty}$, and rewriting in $h_{\mathrm{log}}$ contradicts the assumption that the empty fault is outcome-preserving (since a logical fault must not be outcome-preserving).

From the existence hypothesis, we obtain $F$ with $h_{\mathrm{log}}$, giving nonemptiness of $\text{gaugingLogicalFaultWeights}$ (witnessed by $|F|$). Rewriting the goal as $d_{\mathrm{spacetime}}^{\mathrm{gauging}} \ne 0$, we assume for contradiction that $d_{\mathrm{spacetime}}^{\mathrm{gauging}} = 0$. By $\operatorname{Nat.sInf\_eq\_zero}$, either $0$ is in the set (contradicted above) or the set is empty (contradicted by nonemptiness).
\end{proof}

\begin{theorem}[Gauging Faults Below Distance Are Not Logical]
\label{thm:SpacetimeFaultDistance.not_gaugingLogicalFault_of_weight_lt}
\lean{QEC1.SpacetimeFaultDistance.not_gaugingLogicalFault_of_weight_lt}
\leanok
\uses{def:SpacetimeFaultDistance.gaugingSpacetimeFaultDistance, def:SpacetimeLogicalFault.IsGaugingLogicalFault, def:FaultTolerantGaugingProcedure}
If $|F| < d_{\mathrm{spacetime}}^{\mathrm{gauging}}$, then $F$ is not a gauging logical fault.
\end{theorem}

\begin{proof}
\leanok
\uses{thm:SpacetimeFaultDistance.gaugingSpacetimeFaultDistance_le}
Assume for contradiction that $F$ is a gauging logical fault. Then $d_{\mathrm{spacetime}}^{\mathrm{gauging}} \le |F|$ by the upper bound theorem, contradicting $|F| < d_{\mathrm{spacetime}}^{\mathrm{gauging}}$ by integer arithmetic.
\end{proof}

\begin{theorem}[Syndrome-Free Gauging Below Distance Is Stabilizer]
\label{thm:SpacetimeFaultDistance.syndromeFree_gauging_weight_lt_is_stabilizer}
\lean{QEC1.SpacetimeFaultDistance.syndromeFree_gauging_weight_lt_is_stabilizer}
\leanok
\uses{def:SpacetimeFaultDistance.gaugingSpacetimeFaultDistance, def:SpacetimeLogicalFault.IsSyndromeFreeGauging, def:SpacetimeLogicalFault.IsGaugingStabilizer, def:FaultTolerantGaugingProcedure, def:FaultTolerantGaugingProcedure.DetectorIndex}
If $F$ is syndrome-free for the gauging procedure and $|F| < d_{\mathrm{spacetime}}^{\mathrm{gauging}}$, then $F$ is a gauging stabilizer.
\end{theorem}

\begin{proof}
\leanok
\uses{thm:SpacetimeFaultDistance.not_gaugingLogicalFault_of_weight_lt}
Since $|F| < d_{\mathrm{spacetime}}^{\mathrm{gauging}}$, the previous theorem gives that $F$ is not a gauging logical fault. We construct the gauging stabilizer proof as the pair of syndrome-freeness and the fact that $F$ is outcome-preserving: the latter follows by contradiction, since if $F$ were not outcome-preserving, combined with syndrome-freeness it would constitute a gauging logical fault, contradicting what we just established.
\end{proof}

\begin{theorem}[Gauging Distance Bounded by Weight (Explicit)]
\label{thm:SpacetimeFaultDistance.gaugingSpacetimeFaultDistance_le_weight}
\lean{QEC1.SpacetimeFaultDistance.gaugingSpacetimeFaultDistance_le_weight}
\leanok
\uses{def:SpacetimeFaultDistance.gaugingSpacetimeFaultDistance, def:SpacetimeLogicalFault.IsGaugingLogicalFault, def:FaultTolerantGaugingProcedure}
For any specific gauging logical fault $F$, $d_{\mathrm{spacetime}}^{\mathrm{gauging}} \le |F|$.
\end{theorem}

\begin{proof}
\leanok
\uses{thm:SpacetimeFaultDistance.gaugingSpacetimeFaultDistance_le}
This follows directly from the gauging distance upper bound theorem applied to $F$.
\end{proof}

\begin{theorem}[Gauging Distance Bounded by Space Plus Time Weight]
\label{thm:SpacetimeFaultDistance.gaugingSpacetimeFaultDistance_le_space_plus_time}
\lean{QEC1.SpacetimeFaultDistance.gaugingSpacetimeFaultDistance_le_space_plus_time}
\leanok
\uses{def:SpacetimeFaultDistance.gaugingSpacetimeFaultDistance, def:SpacetimeLogicalFault.IsGaugingLogicalFault, def:SpacetimeFault.spaceWeight, def:SpacetimeFault.timeWeight, def:FaultTolerantGaugingProcedure}
For any gauging logical fault $F$,
\[
d_{\mathrm{spacetime}}^{\mathrm{gauging}} \le |F|_{\mathrm{space}} + |F|_{\mathrm{time}}.
\]
\end{theorem}

\begin{proof}
\leanok
\uses{thm:SpacetimeFaultDistance.gaugingSpacetimeFaultDistance_le}
From the gauging distance upper bound, $d_{\mathrm{spacetime}}^{\mathrm{gauging}} \le |F|$. By simplification using the definitions of weight, space weight, and time weight (which yields $|F| = |F|_{\mathrm{space}} + |F|_{\mathrm{time}}$), the result follows directly.
\end{proof}

%--- Lem_5: SpacetimeStabilizers ---

I'll start by reading the Lean file to understand its contents.Now I have a thorough understanding of the file. Let me produce the LaTeX translation.

\chapter{Lem 5: Spacetime Stabilizers}

The local spacetime stabilizers of the fault-tolerant gauging measurement procedure (Definition~10) are generated by specific fault patterns, organized by time phase. A spacetime stabilizer is a syndrome-free fault pattern that does not affect the logical outcome (Definition~11). Each generator consists of space-faults (Pauli errors at specific times) together with measurement faults on anticommuting checks that cancel the resulting detector violations.

\section*{Part I: Generator Predicates}

Generators are classified by their fault structure: which space-faults (Pauli errors) and which time-faults (measurement errors) they contain. Space stabilizers have no measurement faults; time-propagating and boundary generators include measurement faults on anticommuting checks that cancel detector violations.

\begin{definition}[Space Stabilizer Generator]
\label{def:SpacetimeStabilizers.IsSpaceStabilizerGen}
\lean{QEC1.SpacetimeStabilizers.IsSpaceStabilizerGen}
\leanok
\uses{def:SpacetimeFault, def:FaultTolerantGaugingProcedure, def:PauliOp, def:GaussFlux.ExtQubit}
A \emph{space-only stabilizer generator} is a structure $(\mathsf{IsSpaceStabilizerGen}\ \mathrm{proc}\ P\ t\ F)$ asserting that a spacetime fault $F$ consists of a single check operator $P$ applied as an error at time $t$, with no measurement faults. Formally:
\begin{enumerate}
  \item $F.\mathrm{timeFaults} = \emptyset$,
  \item $F.\mathrm{pauliErrorAt}(t) = P$,
  \item For all $t' \neq t$, $F.\mathrm{spaceFaultsAt}(t') = \emptyset$.
\end{enumerate}
This is used for: original checks $s_j$, deformed checks $\tilde{s}_j / A_v / B_p$, and $Z_e$ at initialization or readout times.
\end{definition}

\begin{definition}[Time-Propagating Generator]
\label{def:SpacetimeStabilizers.IsTimePropagatingGen}
\lean{QEC1.SpacetimeStabilizers.IsTimePropagatingGen}
\leanok
\uses{def:SpacetimeFault, def:FaultTolerantGaugingProcedure, def:SpacetimeLogicalFault.IsSyndromeFreeGauging, def:PauliOp, def:GaussFlux.ExtQubit}
A \emph{time-propagating generator} is a structure $(\mathsf{IsTimePropagatingGen}\ \mathrm{proc}\ P\ t\ F)$ asserting that $F$ has Pauli error $P$ at times $t$ and $t+1$, together with measurement faults on all checks that anticommute with $P$ at time $t + \tfrac{1}{2}$. Formally:
\begin{enumerate}
  \item $F.\mathrm{pauliErrorAt}(t) = P$,
  \item $F.\mathrm{pauliErrorAt}(t+1) = P$,
  \item For all $t' \neq t$ and $t' \neq t+1$, $F.\mathrm{spaceFaultsAt}(t') = \emptyset$,
  \item $F$ is syndrome-free for the gauging procedure.
\end{enumerate}
The measurement faults ensure syndrome-freeness: each detector spanning time $t + \tfrac{1}{2}$ has two violations (one from the Pauli error, one from the measurement fault) that cancel via $(-1) \times (-1) = +1$. The net Pauli effect is $P \cdot P = I$, so the logical outcome is preserved.
\end{definition}

\begin{definition}[Initialization $X_e$ Generator]
\label{def:SpacetimeStabilizers.IsInitXeGen}
\lean{QEC1.SpacetimeStabilizers.IsInitXeGen}
\leanok
\uses{def:SpacetimeFault, def:FaultTolerantGaugingProcedure, def:FTGaugingMeasurement, def:PauliOp, def:GaussFlux.ExtQubit}
An \emph{initialization fault $+ X_e$ generator} is a structure $(\mathsf{IsInitXeGen}\ \mathrm{proc}\ e\ F)$ for an edge $e$ and spacetime fault $F$, consisting of a $|0\rangle_e$ initialization fault at time $t_i - \tfrac{1}{2}$ paired with an $X_e$ Pauli fault at the gauging start time $t_i$. Formally:
\begin{enumerate}
  \item $F.\mathrm{timeFaults} = \{\langle \mathrm{edgeInit}\ e \rangle\}$,
  \item $F.\mathrm{pauliErrorAt}(\mathrm{phase2Start}) = X_e$,
  \item For all $t' \neq \mathrm{phase2Start}$, $F.\mathrm{spaceFaultsAt}(t') = \emptyset$.
\end{enumerate}
Together, the initialization fault and $X_e$ cancel in every detector involving edge $e$.
\end{definition}

\begin{definition}[$Z_e + A_v$ Measurement Fault Generator]
\label{def:SpacetimeStabilizers.IsZeAvMeasGen}
\lean{QEC1.SpacetimeStabilizers.IsZeAvMeasGen}
\leanok
\uses{def:SpacetimeFault, def:FaultTolerantGaugingProcedure, def:FTGaugingMeasurement, def:PauliOp, def:GaussFlux.ExtQubit}
A \emph{$Z_e + A_v$ measurement fault generator} is a structure $(\mathsf{IsZeAvMeasGen}\ \mathrm{proc}\ e\ t\ r\ F)$ for an edge $e$, time $t$, and round $r$. It consists of Pauli $Z_e$ at time $t$ together with $A_v$ measurement faults for both endpoints $v \in e$ at measurement round $r$. Formally:
\begin{enumerate}
  \item $F.\mathrm{pauliErrorAt}(t) = Z_e$,
  \item For all $t' \neq t$, $F.\mathrm{spaceFaultsAt}(t') = \emptyset$,
  \item $F.\mathrm{timeFaults}$ equals the image of the set of vertices $v \in e$ under the map $v \mapsto \langle \mathrm{phase2}\ (\mathrm{gaussLaw}\ v\ r) \rangle$.
\end{enumerate}
Since $Z_e$ anticommutes with $A_v$ for exactly the two endpoints $v \in e$, the measurement faults cancel the two detector violations.
\end{definition}

\begin{definition}[Readout $X_e$ Generator]
\label{def:SpacetimeStabilizers.IsReadoutXeGen}
\lean{QEC1.SpacetimeStabilizers.IsReadoutXeGen}
\leanok
\uses{def:SpacetimeFault, def:FaultTolerantGaugingProcedure, def:FTGaugingMeasurement, def:PauliOp, def:GaussFlux.ExtQubit}
A \emph{readout $X_e$ generator} is a structure $(\mathsf{IsReadoutXeGen}\ \mathrm{proc}\ e\ F)$ for an edge $e$ and spacetime fault $F$, consisting of Pauli $X_e$ at ungauging time $t_o$ paired with a $Z_e$ readout measurement fault at $t_o + \tfrac{1}{2}$. Formally:
\begin{enumerate}
  \item $F.\mathrm{timeFaults} = \{\langle \mathrm{phase3}\ (\mathrm{edgeZ}\ e) \rangle\}$,
  \item $F.\mathrm{pauliErrorAt}(\mathrm{phase3Start}) = X_e$,
  \item For all $t' \neq \mathrm{phase3Start}$, $F.\mathrm{spaceFaultsAt}(t') = \emptyset$.
\end{enumerate}
The $X_e$ flips the $Z_e$ eigenvalue, and the $Z_e$ readout measurement fault compensates, so the correct $Z_e$ value is effectively recorded.
\end{definition}

\section*{Part II: Model-Theoretic Foundation}

Any fault with empty time-faults is trivially syndrome-free and preserves the gauging sign.

\begin{lemma}[Syndrome-Free from Empty Time Faults]
\label{lem:SpacetimeStabilizers.syndromeFree_of_empty_timeFaults}
\lean{QEC1.SpacetimeStabilizers.syndromeFree_of_empty_timeFaults}
\leanok
\uses{def:SpacetimeLogicalFault.IsSyndromeFreeGauging, def:FaultTolerantGaugingProcedure, def:SpacetimeFault}
Any spacetime fault $F$ with $F.\mathrm{timeFaults} = \emptyset$ is syndrome-free for the gauging procedure.
\end{lemma}
\begin{proof}
\leanok
\uses{def:SpacetimeLogicalFault.IsSyndromeFreeGauging, def:FaultTolerantGaugingProcedure}
We unfold the definition of $\mathsf{IsSyndromeFreeGauging}$. Let $\mathrm{idx}$ be an arbitrary detector index. Rewriting using the hypothesis that $F.\mathrm{timeFaults} = \emptyset$, the result follows directly from the fact that no detector is violated when there are no faults.
\end{proof}

\begin{lemma}[Preserves Sign from Empty Time Faults]
\label{lem:SpacetimeStabilizers.preservesSign_of_empty_timeFaults}
\lean{QEC1.SpacetimeStabilizers.preservesSign_of_empty_timeFaults}
\leanok
\uses{def:SpacetimeLogicalFault.PreservesGaugingSign, def:SpacetimeLogicalFault.gaussSignFlip, def:FaultTolerantGaugingProcedure, def:SpacetimeFault}
Any spacetime fault $F$ with $F.\mathrm{timeFaults} = \emptyset$ preserves the gauging sign.
\end{lemma}
\begin{proof}
\leanok
\uses{def:SpacetimeLogicalFault.PreservesGaugingSign, def:SpacetimeLogicalFault.gaussSignFlip}
We unfold the definitions of $\mathsf{PreservesGaugingSign}$ and $\mathsf{gaussSignFlip}$. The outer sum over vertices $v$ equals zero because for each $v$, the inner sum over rounds $r$ equals zero. Each term in the inner sum vanishes by simplification using the hypothesis $F.\mathrm{timeFaults} = \emptyset$.
\end{proof}

\begin{lemma}[Gauging Stabilizer from Empty Time Faults]
\label{lem:SpacetimeStabilizers.isGaugingStabilizer_of_empty_timeFaults}
\lean{QEC1.SpacetimeStabilizers.isGaugingStabilizer_of_empty_timeFaults}
\leanok
\uses{def:SpacetimeLogicalFault.IsGaugingStabilizer, def:SpacetimeLogicalFault.IsSyndromeFreeGauging, def:SpacetimeLogicalFault.PreservesGaugingSign, def:FaultTolerantGaugingProcedure, def:SpacetimeFault}
Any spacetime fault $F$ with $F.\mathrm{timeFaults} = \emptyset$ is a gauging stabilizer.
\end{lemma}
\begin{proof}
\leanok
\uses{lem:SpacetimeStabilizers.syndromeFree_of_empty_timeFaults, lem:SpacetimeStabilizers.preservesSign_of_empty_timeFaults}
The gauging stabilizer condition is the conjunction of syndrome-freeness and gauging sign preservation. Both follow from the preceding lemmas applied to the hypothesis $F.\mathrm{timeFaults} = \emptyset$.
\end{proof}

\section*{Part III: Algebraic Classification}

These are the key algebraic ingredients determining which measurement faults must accompany each space-fault to maintain syndrome-freeness.

\begin{lemma}[Deformed Check Self-Inverse]
\label{lem:SpacetimeStabilizers.deformedCheck_selfInverse}
\lean{QEC1.SpacetimeStabilizers.deformedCheck_selfInverse}
\leanok
\uses{def:FaultTolerantGaugingProcedure, def:DeformedCode.allChecks}
All deformed code checks are self-inverse: for any check index $\mathrm{ci}$,
\[
  \mathrm{allChecks}(\mathrm{ci}) \cdot \mathrm{allChecks}(\mathrm{ci}) = 1.
\]
\end{lemma}
\begin{proof}
\leanok
\uses{def:DeformedCode.allChecks}
This follows directly from the $\mathsf{allChecks\_self\_inverse}$ property of deformed code checks.
\end{proof}

\begin{lemma}[Original Check Self-Inverse]
\label{lem:SpacetimeStabilizers.originalCheck_selfInverse}
\lean{QEC1.SpacetimeStabilizers.originalCheck_selfInverse}
\leanok
\uses{def:PauliOp}
For any original check index $j$, the check operator satisfies $\mathrm{checks}(j) \cdot \mathrm{checks}(j) = 1$.
\end{lemma}
\begin{proof}
\leanok
\uses{lem:PauliOp.mul_self}
This follows directly from the Pauli operator self-inverse property $P \cdot P = 1$.
\end{proof}

\begin{lemma}[Pauli Self-Inverse]
\label{lem:SpacetimeStabilizers.pauli_selfInverse}
\lean{QEC1.SpacetimeStabilizers.pauli_selfInverse}
\leanok
\uses{def:PauliOp}
Any Pauli operator $P$ is self-inverse: $P \cdot P = 1$.
\end{lemma}
\begin{proof}
\leanok
\uses{lem:PauliOp.mul_self}
This follows directly from $\mathsf{PauliOp.mul\_self}$.
\end{proof}

\begin{lemma}[Deformed Checks Pairwise Commute]
\label{lem:SpacetimeStabilizers.deformedChecks_pairwiseCommute}
\lean{QEC1.SpacetimeStabilizers.deformedChecks_pairwiseCommute}
\leanok
\uses{def:FaultTolerantGaugingProcedure, def:DeformedCode.allChecks, def:PauliOp.PauliCommute}
All deformed code checks pairwise commute: for any check indices $\mathrm{ci}$ and $\mathrm{cj}$,
\[
  \mathrm{PauliCommute}(\mathrm{allChecks}(\mathrm{ci}),\; \mathrm{allChecks}(\mathrm{cj})).
\]
\end{lemma}
\begin{proof}
\leanok
\uses{def:DeformedCode.allChecks, def:FaultTolerantGaugingProcedure}
This follows from the $\mathsf{allChecks\_commute}$ property, which uses the cycle parity data and the commutativity of the original checks.
\end{proof}

\begin{lemma}[Original Checks Pairwise Commute]
\label{lem:SpacetimeStabilizers.originalChecks_pairwiseCommute}
\lean{QEC1.SpacetimeStabilizers.originalChecks_pairwiseCommute}
\leanok
\uses{def:FaultTolerantGaugingProcedure, def:PauliOp.PauliCommute}
For any original check indices $i$ and $j$, $\mathrm{PauliCommute}(\mathrm{checks}(i),\; \mathrm{checks}(j))$.
\end{lemma}
\begin{proof}
\leanok
\uses{def:FaultTolerantGaugingProcedure}
This follows directly from the stabilizer code assumption $\mathrm{proc.checks\_commute}$.
\end{proof}

\subsection*{$Z_e$ Commutation Properties}

\begin{lemma}[$Z_e$ Commutes with Flux Checks]
\label{lem:SpacetimeStabilizers.pauliZ_edge_commutes_flux}
\lean{QEC1.SpacetimeStabilizers.pauliZ_edge_commutes_flux}
\leanok
\uses{def:PauliOp.PauliCommute, def:PauliOp.symplecticInner, def:GaussFlux.fluxOp, def:DeformedCode.fluxChecks, def:PauliOp.pauliZ, def:GaussFlux.ExtQubit}
For any edge $e$ and cycle $p$, $Z_e$ commutes with the flux check $B_p$. Both operators are pure $Z$-type.
\end{lemma}
\begin{proof}
\leanok
\uses{def:PauliOp.PauliCommute, def:PauliOp.symplecticInner}
We unfold $\mathsf{PauliCommute}$ and $\mathsf{symplecticInner}$ and show the sum equals zero. For each qubit $q$, we case-split: if $q = \mathrm{inl}(w)$ for some vertex $w$, the term vanishes by properties of $\mathsf{pauliZ}$; if $q = \mathrm{inr}(e')$ for some edge $e'$, the term vanishes because the $x$-component of $\mathsf{pauliZ}$ is zero and the $x$-component of the flux operator is zero, so both products are zero.
\end{proof}

\begin{lemma}[$Z_e$ Commutes with Deformed Checks]
\label{lem:SpacetimeStabilizers.pauliZ_edge_commutes_deformed}
\lean{QEC1.SpacetimeStabilizers.pauliZ_edge_commutes_deformed}
\leanok
\uses{def:PauliOp.PauliCommute, def:PauliOp.symplecticInner, def:DeformedCode.deformedOriginalChecks, def:FaultTolerantGaugingProcedure, def:PauliOp.pauliZ, def:GaussFlux.ExtQubit}
For any edge $e$ and original check index $j$, $Z_e$ commutes with the deformed original check $\tilde{s}_j$. This holds because deformed checks have no $X$-support on edges.
\end{lemma}
\begin{proof}
\leanok
\uses{def:PauliOp.PauliCommute, def:PauliOp.symplecticInner, def:DeformedCode.deformedOriginalChecks}
We unfold $\mathsf{PauliCommute}$ and $\mathsf{symplecticInner}$ and show the sum equals zero. For each qubit $q$, we case-split: if $q = \mathrm{inl}(w)$, the term vanishes by properties of $\mathsf{pauliZ}$. If $q = \mathrm{inr}(e')$, we use the fact that the deformed original check has no $X$-support on edges, i.e., $(\tilde{s}_j).\mathrm{xVec}(\mathrm{inr}(e')) = 0$. Simplifying with $\mathsf{pauliZ}$ having zero $x$-component and applying the single-apply rule, the term vanishes in both sub-cases (whether $e' = e$ or not).
\end{proof}

\begin{lemma}[$Z_e$ Commutes with $A_v$ when $v \notin e$]
\label{lem:SpacetimeStabilizers.pauliZ_edge_commutes_gaussLaw_of_not_mem}
\lean{QEC1.SpacetimeStabilizers.pauliZ_edge_commutes_gaussLaw_of_not_mem}
\leanok
\uses{def:PauliOp.PauliCommute, def:PauliOp.symplecticInner, def:DeformedCode.gaussLawChecks, def:GaussFlux.gaussLawOp, def:PauliOp.pauliZ, def:GaussFlux.ExtQubit}
For an edge $e$ and vertex $v$ with $v \notin e$, $Z_e$ commutes with the Gauss law check $A_v$.
\end{lemma}
\begin{proof}
\leanok
\uses{def:PauliOp.PauliCommute, def:PauliOp.symplecticInner}
We unfold $\mathsf{PauliCommute}$ and $\mathsf{symplecticInner}$ and show the sum equals zero. For each qubit $q$: if $q = \mathrm{inl}(w)$, the term vanishes by $\mathsf{pauliZ}$ properties. If $q = \mathrm{inr}(e')$, the $x$-component of $\mathsf{pauliZ}$ is zero, so we simplify and apply the single-apply rule. In the case $e' = e$, we use the hypothesis $v \notin e$ to show the Gauss law operator's $x$-component at edge $e$ is zero. Otherwise the indicator function vanishes. In both cases the term is zero.
\end{proof}

\begin{lemma}[$Z_e$ Anticommutes with $A_v$ when $v \in e$]
\label{lem:SpacetimeStabilizers.pauliZ_edge_anticommutes_gaussLaw}
\lean{QEC1.SpacetimeStabilizers.pauliZ_edge_anticommutes_gaussLaw}
\leanok
\uses{def:PauliOp.PauliCommute, def:PauliOp.symplecticInner, def:DeformedCode.gaussLawChecks, def:GaussFlux.gaussLawOp, def:PauliOp.pauliZ, def:GaussFlux.ExtQubit}
For an edge $e$ and vertex $v$ with $v \in e$, $Z_e$ anticommutes with the Gauss law check $A_v$:
\[
  \neg\, \mathrm{PauliCommute}(Z_e,\; A_v).
\]
\end{lemma}
\begin{proof}
\leanok
\uses{def:PauliOp.PauliCommute, def:PauliOp.symplecticInner}
We unfold $\mathsf{PauliCommute}$ and $\mathsf{symplecticInner}$ and assume for contradiction that the symplectic inner product is $0$. We show the key identity: for each qubit $q$, the symplectic inner product term equals $(\mathrm{gaussLawOp}\ G\ v).\mathrm{xVec}(\mathrm{inr}(e))$ if $q = \mathrm{inr}(e)$, and $0$ otherwise. This is established by case-splitting on $q$: vertex qubits contribute zero by $\mathsf{pauliZ}$ properties, and for edge qubits $\mathrm{inr}(e')$, we simplify the $\mathsf{pauliZ}$ terms and apply the indicator function, which is nonzero only when $e' = e$.

Summing using $\mathsf{Finset.sum\_ite\_eq'}$, we obtain the total as $(\mathrm{gaussLawOp}\ G\ v).\mathrm{xVec}(\mathrm{inr}(e))$. Since $v \in e$, this equals $1$. Substituting into the commutativity hypothesis and rewriting, we obtain $1 = 0$, a contradiction.
\end{proof}

\begin{lemma}[$Z_e$--$A_v$ Commutation Characterization]
\label{lem:SpacetimeStabilizers.Ze_gaussLaw_commutation}
\lean{QEC1.SpacetimeStabilizers.Ze_gaussLaw_commutation}
\leanok
\uses{def:PauliOp.PauliCommute, def:DeformedCode.gaussLawChecks, def:PauliOp.pauliZ, def:GaussFlux.ExtQubit}
For any edge $e$ and vertex $v$:
\[
  \mathrm{PauliCommute}(Z_e,\; A_v) \iff v \notin e.
\]
\end{lemma}
\begin{proof}
\leanok
\uses{lem:SpacetimeStabilizers.pauliZ_edge_anticommutes_gaussLaw, lem:SpacetimeStabilizers.pauliZ_edge_commutes_gaussLaw_of_not_mem}
We prove both directions. For the forward direction, we assume $\mathrm{PauliCommute}(Z_e, A_v)$ and $v \in e$, and derive a contradiction using the anticommutativity lemma. For the reverse direction, we assume $v \notin e$ and apply the commutativity lemma for non-member vertices.
\end{proof}

\subsection*{$X_e$ Commutation Properties}

\begin{lemma}[$X_e$ Commutes with Gauss Law Checks]
\label{lem:SpacetimeStabilizers.pauliX_edge_commutes_gaussLaw}
\lean{QEC1.SpacetimeStabilizers.pauliX_edge_commutes_gaussLaw}
\leanok
\uses{def:PauliOp.PauliCommute, def:PauliOp.symplecticInner, def:DeformedCode.gaussLawChecks, def:GaussFlux.gaussLawOp, def:PauliOp.pauliX, def:GaussFlux.ExtQubit}
For any edge $e$ and vertex $v$, $X_e$ commutes with the Gauss law check $A_v$. Both are pure $X$-type.
\end{lemma}
\begin{proof}
\leanok
\uses{def:PauliOp.PauliCommute, def:PauliOp.symplecticInner}
We unfold $\mathsf{PauliCommute}$ and $\mathsf{symplecticInner}$ and show the sum equals zero. For each qubit $q$, the term vanishes by simplification using the properties of $\mathsf{pauliX}$ and $\mathsf{gaussLawOp}$, case-splitting on whether $q$ is a vertex or edge qubit.
\end{proof}

\begin{lemma}[$X_e$ Anticommutes with $Z_e$]
\label{lem:SpacetimeStabilizers.pauliX_anticommutes_pauliZ_edge}
\lean{QEC1.SpacetimeStabilizers.pauliX_anticommutes_pauliZ_edge}
\leanok
\uses{def:PauliOp.PauliCommute, def:PauliOp.symplecticInner, def:PauliOp.pauliX, def:PauliOp.pauliZ, def:GaussFlux.ExtQubit}
For any edge $e$, $X_e$ anticommutes with $Z_e$:
\[
  \neg\, \mathrm{PauliCommute}(X_e,\; Z_e).
\]
\end{lemma}
\begin{proof}
\leanok
\uses{def:PauliOp.PauliCommute, def:PauliOp.symplecticInner}
We unfold $\mathsf{PauliCommute}$ and $\mathsf{symplecticInner}$ and assume for contradiction that the symplectic inner product is $0$. For each qubit $q$, we show the symplectic inner product term equals $1$ if $q = \mathrm{inr}(e)$ and $0$ otherwise, by simplifying the $\mathsf{pauliX}$ and $\mathsf{pauliZ}$ components and applying the indicator function. Summing using $\mathsf{Finset.sum\_ite\_eq'}$, we obtain the total as $1 \in \mathbb{Z}/2\mathbb{Z}$. Rewriting in the commutativity hypothesis, we get $1 = 0$, a contradiction.
\end{proof}

\subsection*{Vertex Pauli Commutation Properties}

\begin{lemma}[$X_v$ Commutes with Gauss Law Checks]
\label{lem:SpacetimeStabilizers.pauliX_vertex_commutes_gaussLaw}
\lean{QEC1.SpacetimeStabilizers.pauliX_vertex_commutes_gaussLaw}
\leanok
\uses{def:PauliOp.PauliCommute, def:PauliOp.symplecticInner, def:DeformedCode.gaussLawChecks, def:GaussFlux.gaussLawOp, def:PauliOp.pauliX, def:GaussFlux.ExtQubit}
For any vertices $v$ and $w$, $X_v$ commutes with the Gauss law check $A_w$. Both are pure $X$-type.
\end{lemma}
\begin{proof}
\leanok
\uses{def:PauliOp.PauliCommute, def:PauliOp.symplecticInner}
We unfold $\mathsf{PauliCommute}$ and $\mathsf{symplecticInner}$ and show the sum equals zero. Each term vanishes by simplification using the properties of $\mathsf{pauliX}$ and $\mathsf{gaussLawOp}$.
\end{proof}

\begin{lemma}[$X_v$ Commutes with Flux Checks]
\label{lem:SpacetimeStabilizers.pauliX_vertex_commutes_flux}
\lean{QEC1.SpacetimeStabilizers.pauliX_vertex_commutes_flux}
\leanok
\uses{def:PauliOp.PauliCommute, def:PauliOp.symplecticInner, def:DeformedCode.fluxChecks, def:GaussFlux.fluxOp, def:PauliOp.pauliX, def:GaussFlux.ExtQubit}
For any vertex $v$ and cycle $p$, $X_v$ commutes with the flux check $B_p$.
\end{lemma}
\begin{proof}
\leanok
\uses{def:PauliOp.PauliCommute, def:PauliOp.symplecticInner}
We unfold $\mathsf{PauliCommute}$ and $\mathsf{symplecticInner}$ and show the sum equals zero. For each qubit $q$, we simplify using $\mathsf{pauliX}$ and $\mathsf{fluxOp}$ properties, case-splitting on vertex vs.\ edge qubits. Each term vanishes.
\end{proof}

\begin{lemma}[$Z_v$ Anticommutes with $A_v$]
\label{lem:SpacetimeStabilizers.pauliZ_vertex_anticommutes_gaussLaw}
\lean{QEC1.SpacetimeStabilizers.pauliZ_vertex_anticommutes_gaussLaw}
\leanok
\uses{def:PauliOp.PauliCommute, def:PauliOp.symplecticInner, def:DeformedCode.gaussLawChecks, def:GaussFlux.gaussLawOp, def:PauliOp.pauliZ, def:GaussFlux.ExtQubit}
For any vertex $v$, $Z_v$ anticommutes with the Gauss law check $A_v$:
\[
  \neg\, \mathrm{PauliCommute}(Z_v,\; A_v).
\]
\end{lemma}
\begin{proof}
\leanok
\uses{def:PauliOp.PauliCommute, def:PauliOp.symplecticInner}
We unfold $\mathsf{PauliCommute}$ and $\mathsf{symplecticInner}$ and assume for contradiction that the symplectic inner product is $0$. For each qubit $q$, we show the term equals $1$ if $q = \mathrm{inl}(v)$ and $0$ otherwise. For vertex qubits $\mathrm{inl}(w)$, we simplify $\mathsf{pauliZ}$ and apply the indicator function: when $w = v$, the Gauss law operator's $x$-component at vertex $v$ is $1$. For edge qubits, the term vanishes. Summing via $\mathsf{Finset.sum\_ite\_eq'}$, the total is $1$. Substituting into the hypothesis gives $1 = 0$, a contradiction.
\end{proof}

\begin{lemma}[$Z_v$ Commutes with $A_w$ for $w \neq v$]
\label{lem:SpacetimeStabilizers.pauliZ_vertex_commutes_gaussLaw_ne}
\lean{QEC1.SpacetimeStabilizers.pauliZ_vertex_commutes_gaussLaw_ne}
\leanok
\uses{def:PauliOp.PauliCommute, def:PauliOp.symplecticInner, def:DeformedCode.gaussLawChecks, def:GaussFlux.gaussLawOp, def:PauliOp.pauliZ, def:GaussFlux.ExtQubit}
For vertices $v \neq w$, $Z_v$ commutes with the Gauss law check $A_w$.
\end{lemma}
\begin{proof}
\leanok
\uses{def:PauliOp.PauliCommute, def:PauliOp.symplecticInner}
We unfold $\mathsf{PauliCommute}$ and $\mathsf{symplecticInner}$ and show the sum equals zero. For each qubit $q$: if $q = \mathrm{inl}(u)$ for some vertex $u$, we simplify $\mathsf{pauliZ}$ and apply the indicator function. When $u = v$, the Gauss law operator at $w$ evaluated at vertex $v$ gives $0$ because $v \neq w$. Otherwise the indicator vanishes. If $q = \mathrm{inr}(\cdot)$, the term vanishes by $\mathsf{pauliZ}$ and $\mathsf{gaussLawOp}$ properties.
\end{proof}

\begin{lemma}[$Z_v$ Commutes with Flux Checks]
\label{lem:SpacetimeStabilizers.pauliZ_vertex_commutes_flux}
\lean{QEC1.SpacetimeStabilizers.pauliZ_vertex_commutes_flux}
\leanok
\uses{def:PauliOp.PauliCommute, def:PauliOp.symplecticInner, def:DeformedCode.fluxChecks, def:GaussFlux.fluxOp, def:PauliOp.pauliZ, def:GaussFlux.ExtQubit}
For any vertex $v$ and cycle $p$, $Z_v$ commutes with the flux check $B_p$.
\end{lemma}
\begin{proof}
\leanok
\uses{def:PauliOp.PauliCommute, def:PauliOp.symplecticInner}
We unfold $\mathsf{PauliCommute}$ and $\mathsf{symplecticInner}$ and show the sum equals zero. For each qubit $q$, we simplify using $\mathsf{pauliZ}$ and $\mathsf{fluxOp}$ properties, case-splitting on vertex vs.\ edge qubits. Each term vanishes.
\end{proof}

\subsection*{Edge Endpoints}

\begin{lemma}[Edge Has Two Endpoints]
\label{lem:SpacetimeStabilizers.edge_has_two_endpoints}
\lean{QEC1.SpacetimeStabilizers.edge_has_two_endpoints}
\leanok
\uses{def:GaussFlux.ExtQubit}
For any edge $e$ in the graph $G$, the set of vertices $v$ with $v \in e$ has cardinality exactly $2$:
\[
  |\{v \in V \mid v \in e\}| = 2.
\]
\end{lemma}
\begin{proof}
\leanok

We destructure the edge $e$ and proceed by induction on its $\mathrm{Sym2}$ representation $\{a, b\}$. Since $G$ is loopless, $a \neq b$. The filter set $\{v \mid v = a \lor v = b\}$ equals the pair $\{a, b\}$ by extensionality. The cardinality of a two-element set with distinct elements is $2$.
\end{proof}

\begin{lemma}[$Z_e$--$A_v$ Anticommutation Count]
\label{lem:SpacetimeStabilizers.ZeAv_anticommutation_count}
\lean{QEC1.SpacetimeStabilizers.ZeAv_anticommutation_count}
\leanok
\uses{def:PauliOp.PauliCommute, def:DeformedCode.gaussLawChecks, def:PauliOp.pauliZ, def:GaussFlux.ExtQubit}
$Z_e$ anticommutes with exactly $2$ Gauss law checks (those at the endpoints of $e$):
\[
  |\{v \in V \mid \neg\, \mathrm{PauliCommute}(Z_e, A_v)\}| = 2.
\]
\end{lemma}
\begin{proof}
\leanok
\uses{lem:SpacetimeStabilizers.Ze_gaussLaw_commutation, lem:SpacetimeStabilizers.edge_has_two_endpoints}
We show that the set of vertices $v$ such that $\neg\, \mathrm{PauliCommute}(Z_e, A_v)$ equals the set of vertices $v \in e$. This follows by extensionality using the $Z_e$--$A_v$ commutation characterization: $\mathrm{PauliCommute}(Z_e, A_v) \iff v \notin e$, so $\neg\, \mathrm{PauliCommute}(Z_e, A_v) \iff v \in e$. Rewriting, the cardinality equals $2$ by the edge endpoint count lemma.
\end{proof}

\subsection*{$Z_e$ Non-Gauss Commutation}

\begin{lemma}[$Z_e$ Commutes with All Non-Gauss Deformed Checks]
\label{lem:SpacetimeStabilizers.Ze_commutes_with_all_nonGauss}
\lean{QEC1.SpacetimeStabilizers.Ze_commutes_with_all_nonGauss}
\leanok
\uses{def:PauliOp.PauliCommute, def:DeformedCode.allChecks, def:FaultTolerantGaugingProcedure, def:PauliOp.pauliZ, def:GaussFlux.ExtQubit}
For any edge $e$ and check index $\mathrm{ci}$ that is not a Gauss law check (i.e., $\mathrm{ci} \neq \mathrm{gaussLaw}(v)$ for all $v$), $Z_e$ commutes with $\mathrm{allChecks}(\mathrm{ci})$.
\end{lemma}
\begin{proof}
\leanok
\uses{lem:SpacetimeStabilizers.pauliZ_edge_commutes_flux, lem:SpacetimeStabilizers.pauliZ_edge_commutes_deformed}
We case-split on the check index $\mathrm{ci}$:
\begin{itemize}
  \item If $\mathrm{ci} = \mathrm{gaussLaw}(v)$ for some $v$, this contradicts the hypothesis $\mathrm{ci} \neq \mathrm{gaussLaw}(v)$.
  \item If $\mathrm{ci} = \mathrm{flux}(p)$, the check is a flux check $B_p$, and $Z_e$ commutes with it by the $Z_e$--flux commutation lemma.
  \item If $\mathrm{ci} = \mathrm{deformed}(j)$, the check is a deformed original check $\tilde{s}_j$, and $Z_e$ commutes with it by the $Z_e$--deformed commutation lemma.
\end{itemize}
\end{proof}

\section*{Part IV: Generator Stabilizer Proofs}

Space stabilizers (with empty time-faults) are proved directly. Generators with non-empty time-faults require showing that measurement faults cancel detector violations; this is axiomatized since the detector cancellation argument requires reasoning about specific detector measurement membership.

\begin{lemma}[Space Stabilizer Generators are Gauging Stabilizers]
\label{lem:SpacetimeStabilizers.spaceStabilizer_isGaugingStabilizer}
\lean{QEC1.SpacetimeStabilizers.spaceStabilizer_isGaugingStabilizer}
\leanok
\uses{def:SpacetimeLogicalFault.IsGaugingStabilizer, def:SpacetimeLogicalFault.PreservesGaugingSign, def:FaultTolerantGaugingProcedure, def:SpacetimeFault}
If $F$ is a space stabilizer generator (i.e., $\mathsf{IsSpaceStabilizerGen}\ \mathrm{proc}\ P\ t\ F$), then $F$ is a gauging stabilizer.
\end{lemma}
\begin{proof}
\leanok
\uses{lem:SpacetimeStabilizers.isGaugingStabilizer_of_empty_timeFaults}
This follows directly from the gauging stabilizer property for faults with empty time-faults, since $\mathsf{IsSpaceStabilizerGen}$ implies $F.\mathrm{timeFaults} = \emptyset$.
\end{proof}

\begin{theorem}[Axiom: Time-Propagating Generators are Gauging Stabilizers]
\label{thm:SpacetimeStabilizers.timePropagating_isGaugingStabilizer}
\lean{QEC1.SpacetimeStabilizers.timePropagating_isGaugingStabilizer}
\leanok
\uses{def:SpacetimeLogicalFault.IsGaugingStabilizer, def:SpacetimeLogicalFault.PreservesGaugingSign, def:FaultTolerantGaugingProcedure, def:SpacetimeFault}

\textbf{\color{red}[UNPROVEN AXIOM]}

If $F$ is a time-propagating generator $(\mathsf{IsTimePropagatingGen}\ \mathrm{proc}\ P\ t\ F)$, then $F$ is a gauging stabilizer. The proof requires showing that the measurement faults on anticommuting checks cancel all detector violations via the $(-1) \times (-1) = +1$ argument, and that the net Pauli effect $P \cdot P = I$ preserves the logical outcome.

\textit{Note: This is stated as an axiom because the full proof was not completed in the formalization.}
\end{theorem}

\begin{theorem}[Axiom: Init $X_e$ Generators are Gauging Stabilizers]
\label{thm:SpacetimeStabilizers.initXe_isGaugingStabilizer}
\lean{QEC1.SpacetimeStabilizers.initXe_isGaugingStabilizer}
\leanok
\uses{def:SpacetimeLogicalFault.IsGaugingStabilizer, def:SpacetimeLogicalFault.PreservesGaugingSign, def:FaultTolerantGaugingProcedure, def:SpacetimeFault}

\textbf{\color{red}[UNPROVEN AXIOM]}

If $F$ is an initialization $X_e$ generator $(\mathsf{IsInitXeGen}\ \mathrm{proc}\ e\ F)$, then $F$ is a gauging stabilizer. The $|0\rangle_e$ initialization fault flips the init detector for edge $e$; the $X_e$ Pauli at $t_i$ flips the same detector via the check measurement. These cancel: $(-1) \times (-1) = +1$.

\textit{Note: This is stated as an axiom because the full proof was not completed in the formalization.}
\end{theorem}

\begin{theorem}[Axiom: $Z_e + A_v$ Measurement Fault Generators are Gauging Stabilizers]
\label{thm:SpacetimeStabilizers.ZeAvMeas_isGaugingStabilizer}
\lean{QEC1.SpacetimeStabilizers.ZeAvMeas_isGaugingStabilizer}
\leanok
\uses{def:SpacetimeLogicalFault.IsGaugingStabilizer, def:SpacetimeLogicalFault.PreservesGaugingSign, def:FaultTolerantGaugingProcedure, def:SpacetimeFault}

\textbf{\color{red}[UNPROVEN AXIOM]}

If $F$ is a $Z_e + A_v$ measurement fault generator $(\mathsf{IsZeAvMeasGen}\ \mathrm{proc}\ e\ t\ r\ F)$, then $F$ is a gauging stabilizer. $Z_e$ anticommutes with $A_v$ for both endpoints $v \in e$; each $A_v$ measurement fault cancels the corresponding detector violation: $(-1) \times (-1) = +1$ for each endpoint.

\textit{Note: This is stated as an axiom because the full proof was not completed in the formalization.}
\end{theorem}

\begin{theorem}[Axiom: Readout $X_e$ Generators are Gauging Stabilizers]
\label{thm:SpacetimeStabilizers.readoutXe_isGaugingStabilizer}
\lean{QEC1.SpacetimeStabilizers.readoutXe_isGaugingStabilizer}
\leanok
\uses{def:SpacetimeLogicalFault.IsGaugingStabilizer, def:SpacetimeLogicalFault.PreservesGaugingSign, def:FaultTolerantGaugingProcedure, def:SpacetimeFault}

\textbf{\color{red}[UNPROVEN AXIOM]}

If $F$ is a readout $X_e$ generator $(\mathsf{IsReadoutXeGen}\ \mathrm{proc}\ e\ F)$, then $F$ is a gauging stabilizer. $X_e$ flips the $Z_e$ eigenvalue; the $Z_e$ readout fault compensates.

\textit{Note: This is stated as an axiom because the full proof was not completed in the formalization.}
\end{theorem}

\section*{Part V: Listed Generator Classification}

\begin{definition}[Listed Generator]
\label{def:SpacetimeStabilizers.IsListedGenerator}
\lean{QEC1.SpacetimeStabilizers.IsListedGenerator}
\leanok
\uses{def:FaultTolerantGaugingProcedure, def:SpacetimeFault, def:SpacetimeStabilizers.IsSpaceStabilizerGen, def:SpacetimeStabilizers.IsTimePropagatingGen, def:SpacetimeStabilizers.IsInitXeGen, def:SpacetimeStabilizers.IsZeAvMeasGen, def:SpacetimeStabilizers.IsReadoutXeGen, def:DeformedCode.allChecks, def:DeformedCode.deformedOriginalChecks, def:GaussFlux.ExtQubit}
A spacetime fault $F$ is a \emph{listed generator} $(\mathsf{IsListedGenerator}\ \mathrm{proc}\ F)$ if it falls into one of the following categories, organized by time phase:
\begin{enumerate}
  \item \textbf{Phase 1 \& 3 -- Original check:} $F$ is a space stabilizer generator for an original check $\tilde{s}_j$ at time $t$ with $t < t_i$ or $t \geq t_o$.
  \item \textbf{Phase 1 \& 3 -- Time-propagating:} $F$ is a time-propagating generator with Pauli $P$ at time $t$ with $t < t_i$ or $t \geq t_o$.
  \item \textbf{Phase 2 -- Deformed check:} $F$ is a space stabilizer generator for a deformed check $\mathrm{allChecks}(\mathrm{ci})$ at time $t$ with $t_i < t < t_o$.
  \item \textbf{Phase 2 -- Time-propagating $X_v$:} at time $t$ with $t_i \leq t < t_o$.
  \item \textbf{Phase 2 -- Time-propagating $Z_v$:} at time $t$ with $t_i \leq t < t_o$.
  \item \textbf{Phase 2 -- Time-propagating $X_e$:} at time $t$ with $t_i \leq t < t_o$.
  \item \textbf{Phase 2 -- Time-propagating $Z_e$:} at time $t$ with $t_i \leq t < t_o$.
  \item \textbf{Gauging ($t = t_i$) -- Original check $s_j$:} space stabilizer at $t_i$.
  \item \textbf{Gauging ($t = t_i$) -- $Z_e$:} space stabilizer at $t_i$.
  \item \textbf{Gauging ($t = t_i$) -- Init $+ X_e$:} initialization fault paired with $X_e$.
  \item \textbf{Gauging ($t = t_i$) -- $Z_e + A_v$:} $Z_e$ at $t_i + 1$ with $A_v$ measurement faults.
  \item \textbf{Ungauging ($t = t_o$) -- Deformed check:} space stabilizer at $t_o$.
  \item \textbf{Ungauging ($t = t_o$) -- Readout $X_e$:} $X_e$ paired with $Z_e$ readout fault.
  \item \textbf{Ungauging ($t = t_o$) -- Bare $Z_e$:} space stabilizer at $t_o$.
  \item \textbf{Ungauging ($t = t_o$) -- $Z_e + A_v$:} $Z_e$ at $t_o - 1$ with $A_v$ measurement faults.
  \item \textbf{Ungauging ($t = t_o$) -- Time-propagating:} at the $t_o$ boundary.
\end{enumerate}
\end{definition}

\section*{Part VI: Main Theorem}

\begin{theorem}[Lemma 5: Every Listed Generator is a Gauging Stabilizer]
\label{thm:SpacetimeStabilizers.listedGenerator_isGaugingStabilizer}
\lean{QEC1.SpacetimeStabilizers.listedGenerator_isGaugingStabilizer}
\leanok
\uses{def:SpacetimeStabilizers.IsListedGenerator, def:SpacetimeLogicalFault.IsGaugingStabilizer, def:SpacetimeLogicalFault.PreservesGaugingSign, def:FaultTolerantGaugingProcedure, def:SpacetimeFault}
Every listed generator fault pattern $F$ is a gauging stabilizer: it has empty syndrome and does not affect the logical outcome.
\end{theorem}
\begin{proof}
\leanok
\uses{lem:SpacetimeStabilizers.spaceStabilizer_isGaugingStabilizer, thm:SpacetimeStabilizers.timePropagating_isGaugingStabilizer, thm:SpacetimeStabilizers.initXe_isGaugingStabilizer, thm:SpacetimeStabilizers.ZeAvMeas_isGaugingStabilizer, thm:SpacetimeStabilizers.readoutXe_isGaugingStabilizer}
We case-split on the type of listed generator:
\begin{itemize}
  \item \textbf{origCheck:} $F$ is a space stabilizer generator for an original check. By the space stabilizer lemma, $F$ is a gauging stabilizer.
  \item \textbf{origTimeProp:} $F$ is a time-propagating generator in Phase 1 or 3. By the time-propagating axiom, $F$ is a gauging stabilizer.
  \item \textbf{deformedCheck:} $F$ is a space stabilizer generator for a deformed check. By the space stabilizer lemma, $F$ is a gauging stabilizer.
  \item \textbf{deformedTimePropXv, deformedTimePropZv, deformedTimePropXe, deformedTimePropZe:} Each is a time-propagating generator in Phase 2. By the time-propagating axiom, each is a gauging stabilizer.
  \item \textbf{gaugingSj, gaugingZe:} Space stabilizer generators at the gauging time $t_i$. By the space stabilizer lemma, each is a gauging stabilizer.
  \item \textbf{gaugingInitXe:} An init $X_e$ generator. By the init $X_e$ axiom, $F$ is a gauging stabilizer.
  \item \textbf{gaugingZeAv:} A $Z_e + A_v$ measurement fault generator. By the $Z_e + A_v$ axiom, $F$ is a gauging stabilizer.
  \item \textbf{ungaugingCheck:} A space stabilizer generator at $t_o$. By the space stabilizer lemma, $F$ is a gauging stabilizer.
  \item \textbf{ungaugingReadoutXe:} A readout $X_e$ generator. By the readout $X_e$ axiom, $F$ is a gauging stabilizer.
  \item \textbf{ungaugingBareZe:} A space stabilizer generator at $t_o$. By the space stabilizer lemma, $F$ is a gauging stabilizer.
  \item \textbf{ungaugingZeAv:} A $Z_e + A_v$ generator at $t_o - 1$. By the $Z_e + A_v$ axiom, $F$ is a gauging stabilizer.
  \item \textbf{ungaugingTimeProp:} A time-propagating generator at the $t_o$ boundary. By the time-propagating axiom, $F$ is a gauging stabilizer.
\end{itemize}
\end{proof}

\section*{Part VII: Algebraic Justifications}

\begin{theorem}[Spacetime Stabilizer Algebraic Facts]
\label{thm:SpacetimeStabilizers.spacetimeStabilizer_algebraicFacts}
\lean{QEC1.SpacetimeStabilizers.spacetimeStabilizer_algebraicFacts}
\leanok
\uses{def:FaultTolerantGaugingProcedure, def:DeformedCode.allChecks, def:PauliOp.PauliCommute, def:DeformedCode.gaussLawChecks, def:PauliOp.pauliX, def:PauliOp.pauliZ, def:PauliOp, def:SpacetimeFault, def:GaussFlux.ExtQubit}
The algebraic structure underlying the spacetime stabilizer classification consists of the following eleven properties:
\begin{enumerate}
  \item Original checks are self-inverse: $\forall j,\; \mathrm{checks}(j) \cdot \mathrm{checks}(j) = 1$.
  \item Original checks pairwise commute: $\forall i,j,\; \mathrm{PauliCommute}(\mathrm{checks}(i), \mathrm{checks}(j))$.
  \item Deformed checks are self-inverse: $\forall \mathrm{ci},\; \mathrm{allChecks}(\mathrm{ci})^2 = 1$.
  \item Deformed checks pairwise commute: $\forall \mathrm{ci}, \mathrm{cj},\; \mathrm{PauliCommute}(\mathrm{allChecks}(\mathrm{ci}), \mathrm{allChecks}(\mathrm{cj}))$.
  \item $Z_e$--$A_v$ commutation characterization: $\mathrm{PauliCommute}(Z_e, A_v) \iff v \notin e$.
  \item $Z_e$ anticommutes with exactly $2$ Gauss law checks: $|\{v \mid v \in e\}| = 2$.
  \item $Z_e$ commutes with all non-Gauss deformed checks.
  \item $X_e$ anticommutes with $Z_e$ (init/readout generator structure).
  \item $X_e$ commutes with all Gauss law checks.
  \item Any Pauli is self-inverse: $P \cdot P = 1$.
  \item Composition uses symmetric difference on time-faults ($\mathbb{Z}_2$ group closure).
\end{enumerate}
\end{theorem}
\begin{proof}
\leanok
\uses{lem:PauliOp.mul_self, lem:SpacetimeStabilizers.Ze_gaussLaw_commutation, lem:SpacetimeStabilizers.edge_has_two_endpoints, lem:SpacetimeStabilizers.Ze_commutes_with_all_nonGauss, lem:SpacetimeStabilizers.pauliX_anticommutes_pauliZ_edge, lem:SpacetimeStabilizers.pauliX_edge_commutes_gaussLaw}
The proof constructs the eleven-fold conjunction directly:
\begin{enumerate}
  \item By $\mathsf{PauliOp.mul\_self}$.
  \item By $\mathrm{proc.checks\_commute}$.
  \item By $\mathsf{allChecks\_self\_inverse}$.
  \item By $\mathsf{allChecks\_commute}$ using cycle parity and original check commutativity.
  \item By the $Z_e$--$A_v$ commutation characterization lemma.
  \item By the edge endpoint count lemma.
  \item By the $Z_e$ non-Gauss commutation lemma.
  \item By the $X_e$--$Z_e$ anticommutativity lemma.
  \item By the $X_e$--Gauss law commutativity lemma.
  \item By $\mathsf{PauliOp.mul\_self}$.
  \item By reflexivity (the composition is definitionally symmetric difference on time-faults).
\end{enumerate}
\end{proof}

\section*{Part VIII: Completeness}

\begin{theorem}[Axiom: Completeness of Spacetime Stabilizer Generators]
\label{thm:SpacetimeStabilizers.spacetimeStabilizer_completeness}
\lean{QEC1.SpacetimeStabilizers.spacetimeStabilizer_completeness}
\leanok
\uses{def:SpacetimeLogicalFault.IsGaugingStabilizer, def:SpacetimeLogicalFault.PreservesGaugingSign, def:FaultTolerantGaugingProcedure, def:SpacetimeFault, def:SpacetimeStabilizers.IsListedGenerator}

\textbf{\color{red}[UNPROVEN AXIOM]}

Every gauging stabilizer decomposes as a $\mathbb{Z}_2$ composition (symmetric difference) of the listed generators. Formally, if $F$ is a gauging stabilizer, then there exists a list $\mathrm{gens}$ of spacetime faults such that:
\begin{enumerate}
  \item Every $g \in \mathrm{gens}$ satisfies $\mathsf{IsListedGenerator}(\mathrm{proc}, g)$, and
  \item $F = \mathrm{gens.foldl}(\mathsf{compose},\; \mathsf{empty})$.
\end{enumerate}
The proof uses time-ordered decomposition: peel off generators at the earliest active time, proceed inductively, until only measurement faults remain, which must decompose into detector measurement sets.

\textit{Note: This is stated as an axiom because the full proof was not completed in the formalization.}
\end{theorem}

\section*{Part IX: Corollaries}

\begin{lemma}[$Z_e$ Commutes with Original Checks on $V \oplus E$]
\label{lem:SpacetimeStabilizers.Ze_commutes_with_original_check}
\lean{QEC1.SpacetimeStabilizers.Ze_commutes_with_original_check}
\leanok
\uses{def:PauliOp.PauliCommute, def:DeformedCode.deformedOriginalChecks, def:FaultTolerantGaugingProcedure, def:PauliOp.pauliZ, def:GaussFlux.ExtQubit}
For any edge $e$ and original check index $j$, $Z_e$ commutes with the original check $\tilde{s}_j$ lifted to $V \oplus E$.
\end{lemma}
\begin{proof}
\leanok
\uses{lem:SpacetimeStabilizers.pauliZ_edge_commutes_deformed}
This follows directly from the $Z_e$--deformed check commutation lemma.
\end{proof}

% COMMENTED OUT: Isolated declaration (not connected to dependency graph)
% \begin{lemma}[Double Violation Cancellation]
% \label{lem:SpacetimeStabilizers.double_violation_cancels}
% \lean{QEC1.SpacetimeStabilizers.double_violation_cancels}
% \leanok
%
% In $\mathbb{Z}/2\mathbb{Z}$, for any $x$: $x + x = 0$.
% \end{lemma}
% \begin{proof}
% \leanok
%
% This follows from $\mathsf{CharTwo.add\_self\_eq\_zero}$, the characteristic-two property.
% \end{proof}

\section*{Part X: Space-Fault Cleaning and Centralizer Properties}

\begin{theorem}[Axiom: Space-Fault Cleaning]
\label{thm:SpacetimeStabilizers.space_fault_cleaning}
\lean{QEC1.SpacetimeStabilizers.space_fault_cleaning}
\leanok
\uses{def:SpacetimeLogicalFault.IsSyndromeFreeGauging, def:SpacetimeLogicalFault.PreservesGaugingSign, def:FaultTolerantGaugingProcedure, def:SpacetimeFault, def:DeformedCodeChecks.deformedStabilizerCode, def:StabilizerCode.stabilizerGroup}

\textbf{\color{red}[UNPROVEN AXIOM]}

For any syndrome-free spacetime fault $F$, there exists a stabilizer $S_1$ such that:
\begin{enumerate}
  \item $S_1$ is syndrome-free for the gauging procedure,
  \item $S_1$ preserves the gauging sign,
  \item $S_1.\mathrm{pauliErrorAt}(t_i)$ is in the stabilizer group of the deformed code, and
  \item For all $t \neq t_i$, $(F \circ S_1).\mathrm{spaceFaultsAt}(t) = \emptyset$.
\end{enumerate}
The construction uses time-propagating generators to move all space-faults to the gauging time $t_i$. Boundary initialization/readout faults are absorbed using init-$X_e$ and readout-$X_e$ generators. The composed generators cancel intermediate Paulis ($P \cdot P = 1$), leaving only the net effect at $t_i$, which is a product of check operators and hence in the stabilizer group.

\textit{Note: This is stated as an axiom because the full proof was not completed in the formalization.}
\end{theorem}

\begin{theorem}[Axiom: Centralizer Membership for Cleaned Pure-Space Faults]
\label{thm:SpacetimeStabilizers.syndromeFree_pureSpace_inCentralizer}
\lean{QEC1.SpacetimeStabilizers.syndromeFree_pureSpace_inCentralizer}
\leanok
\uses{def:SpacetimeLogicalFault.IsSyndromeFreeGauging, def:FaultTolerantGaugingProcedure, def:SpacetimeFault, def:DeformedCodeChecks.deformedStabilizerCode, def:StabilizerCode.inCentralizer}

\textbf{\color{red}[UNPROVEN AXIOM]}

A pure-space fault $F$ concentrated at the gauging time $t_i$ (i.e., $F.\mathrm{spaceFaultsAt}(t) = \emptyset$ for all $t \neq t_i$ and $F.\mathrm{isPureSpace}$), if syndrome-free for the gauging procedure, has its Pauli error $F.\mathrm{pauliErrorAt}(t_i)$ in the centralizer of the deformed code.

This encodes the fact that during Phase~2, the active checks are the deformed code checks $(A_v, B_p, \tilde{s}_j)$, and the time-propagating generators commute with all active checks away from their support. The cleaning process preserves the commutation structure, so the concentrated Pauli at $t_i$ commutes with every deformed code check.

\textit{Note: This is stated as an axiom because the full proof was not completed in the formalization.}
\end{theorem}

%--- Lem_6: TimeFaultDistance ---
\chapter{Lem 6: Time Fault-Distance}

This chapter proves that the time fault-distance---the minimum weight of a pure-time gauging logical fault in the fault-tolerant gauging measurement procedure---equals $d$, the number of rounds in the deformed code phase.

%% Part I: Pure-Time Logical Fault Definition

\begin{definition}[Pure-Time Fault]
\label{def:TimeFaultDistance.IsPureTimeFault}
\lean{QEC1.TimeFaultDistance.IsPureTimeFault}
\leanok
\uses{def:FaultTolerantGaugingProcedure, def:SpacetimeFault}
A spacetime fault $F$ is a \emph{pure-time fault} if it has no space-faults, i.e., $F$ satisfies $F.\mathrm{isPureTime}$.
\end{definition}

\begin{definition}[Pure-Time Logical Fault]
\label{def:TimeFaultDistance.IsPureTimeLogicalFault}
\lean{QEC1.TimeFaultDistance.IsPureTimeLogicalFault}
\leanok
\uses{def:TimeFaultDistance.IsPureTimeFault, def:SpacetimeLogicalFault.IsGaugingLogicalFault, def:FaultTolerantGaugingProcedure, def:FaultTolerantGaugingProcedure.DetectorIndex}
A spacetime fault $F$ is a \emph{pure-time gauging logical fault} if it is both a pure-time fault and a gauging logical fault:
\[
\operatorname{IsPureTimeLogicalFault}(F) \;\Leftrightarrow\; \operatorname{IsPureTimeFault}(F) \;\land\; \operatorname{IsGaugingLogicalFault}(F).
\]
\end{definition}

\begin{definition}[Pure-Time Logical Fault Weights]
\label{def:TimeFaultDistance.pureTimeLogicalFaultWeights}
\lean{QEC1.TimeFaultDistance.pureTimeLogicalFaultWeights}
\leanok
\uses{def:TimeFaultDistance.IsPureTimeLogicalFault, def:FaultTolerantGaugingProcedure, def:FaultTolerantGaugingProcedure.DetectorIndex}
The set of weights of pure-time gauging logical faults:
\[
\{ w \in \mathbb{N} \mid \exists\, F,\; \operatorname{IsPureTimeLogicalFault}(F) \;\land\; \operatorname{weight}(F) = w \}.
\]
\end{definition}

\begin{definition}[Time Fault-Distance]
\label{def:TimeFaultDistance.timeFaultDistance}
\lean{QEC1.TimeFaultDistance.timeFaultDistance}
\leanok
\uses{def:TimeFaultDistance.pureTimeLogicalFaultWeights, def:FaultTolerantGaugingProcedure, def:FaultTolerantGaugingProcedure.DetectorIndex}
The \emph{time fault-distance} is the infimum of the set of pure-time logical fault weights:
\[
d_{\mathrm{time}} \;=\; \inf\, \operatorname{pureTimeLogicalFaultWeights}.
\]
\end{definition}

%% Part II: The A_v Measurement String (Upper Bound Witness)

\begin{definition}[Gauss Measurement Faults]
\label{def:TimeFaultDistance.gaussMeasFaults}
\lean{QEC1.TimeFaultDistance.gaussMeasFaults}
\leanok
\uses{def:FaultTolerantGaugingProcedure, def:FTGaugingMeasurement, def:DeformedCodeMeasurement}
For a vertex $v \in V$, the \emph{Gauss measurement faults} $\operatorname{gaussMeasFaults}(v)$ is the set of time-faults corresponding to the $A_v$ measurement at all $d$ rounds of Phase~2:
\[
\operatorname{gaussMeasFaults}(v) = \bigl\{ \langle \operatorname{phase2}(\operatorname{gaussLaw}(v, r)) \rangle \mid r \in \operatorname{Fin}(d) \bigr\}.
\]
\end{definition}

\begin{definition}[Gauss String Fault]
\label{def:TimeFaultDistance.gaussStringFault}
\lean{QEC1.TimeFaultDistance.gaussStringFault}
\leanok
\uses{def:TimeFaultDistance.gaussMeasFaults, def:SpacetimeFault, def:FaultTolerantGaugingProcedure}
The \emph{Gauss string fault} for vertex $v$ is the spacetime fault $(\emptyset, \operatorname{gaussMeasFaults}(v))$ with no space-faults and time-faults given by the $A_v$ measurement string.
\end{definition}

\begin{lemma}[Gauss String Fault is Pure-Time]
\label{lem:TimeFaultDistance.gaussStringFault_isPureTime}
\lean{QEC1.TimeFaultDistance.gaussStringFault_isPureTime}
\leanok
\uses{def:TimeFaultDistance.gaussStringFault, def:TimeFaultDistance.IsPureTimeFault}
The Gauss string fault for any vertex $v$ is a pure-time fault.
\end{lemma}

\begin{proof}
\leanok
\uses{def:TimeFaultDistance.gaussStringFault}
By definition, $\operatorname{gaussStringFault}(v)$ has empty space-faults, so $\operatorname{isPureTime}$ holds by simplification.
\end{proof}

\begin{lemma}[Gauss Measurement Faults Map is Injective]
\label{lem:TimeFaultDistance.gaussMeasFaults_injective}
\lean{QEC1.TimeFaultDistance.gaussMeasFaults_injective}
\leanok
\uses{def:TimeFaultDistance.gaussMeasFaults, def:FTGaugingMeasurement, def:DeformedCodeMeasurement}
The map $r \mapsto \langle \operatorname{phase2}(\operatorname{gaussLaw}(v, r)) \rangle$ from $\operatorname{Fin}(d)$ to time-faults is injective.
\end{lemma}

\begin{proof}
\leanok
\uses{def:FTGaugingMeasurement, def:DeformedCodeMeasurement}
Let $r_1, r_2 \in \operatorname{Fin}(d)$ and suppose the images are equal. By injectivity of the $\operatorname{phase2}$ and $\operatorname{gaussLaw}$ constructors, we extract $r_1 = r_2$.
\end{proof}

\begin{lemma}[Gauss Measurement Faults Cardinality]
\label{lem:TimeFaultDistance.gaussMeasFaults_card}
\lean{QEC1.TimeFaultDistance.gaussMeasFaults_card}
\leanok
\uses{def:TimeFaultDistance.gaussMeasFaults, lem:TimeFaultDistance.gaussMeasFaults_injective}
$|\operatorname{gaussMeasFaults}(v)| = d$.
\end{lemma}

\begin{proof}
\leanok
\uses{lem:TimeFaultDistance.gaussMeasFaults_injective}
Rewriting using the definition of $\operatorname{gaussMeasFaults}$, the cardinality of the image of an injective function equals the cardinality of the domain, which is $|\operatorname{Fin}(d)| = d$.
\end{proof}

\begin{lemma}[Gauss String Fault Weight Equals $d$]
\label{lem:TimeFaultDistance.gaussStringFault_weight_eq_d}
\lean{QEC1.TimeFaultDistance.gaussStringFault_weight_eq_d}
\leanok
\uses{def:TimeFaultDistance.gaussStringFault, lem:TimeFaultDistance.gaussMeasFaults_card}
$\operatorname{weight}(\operatorname{gaussStringFault}(v)) = d$.
\end{lemma}

\begin{proof}
\leanok
\uses{def:TimeFaultDistance.gaussStringFault, lem:TimeFaultDistance.gaussMeasFaults_card}
By simplification using the definitions of weight and $\operatorname{gaussStringFault}$, the weight equals $|\emptyset| + |\operatorname{gaussMeasFaults}(v)| = 0 + d = d$.
\end{proof}

\begin{lemma}[Membership in Gauss Measurement Faults]
\label{lem:TimeFaultDistance.mem_gaussMeasFaults}
\lean{QEC1.TimeFaultDistance.mem_gaussMeasFaults}
\leanok
\uses{def:TimeFaultDistance.gaussMeasFaults, def:FTGaugingMeasurement, def:DeformedCodeMeasurement}
A time-fault $\mathit{tf}$ belongs to $\operatorname{gaussMeasFaults}(v)$ if and only if there exists $r \in \operatorname{Fin}(d)$ such that $\mathit{tf} = \langle \operatorname{phase2}(\operatorname{gaussLaw}(v, r)) \rangle$.
\end{lemma}

\begin{proof}
\leanok
\uses{def:TimeFaultDistance.gaussMeasFaults}
By simplification of the membership condition in the image finset. Both directions follow by exhibiting the witness $r$.
\end{proof}

\begin{lemma}[Gauss String Fault Time-Faults]
\label{lem:TimeFaultDistance.gaussStringFault_timeFaults_eq}
\lean{QEC1.TimeFaultDistance.gaussStringFault_timeFaults_eq}
\leanok
\uses{def:TimeFaultDistance.gaussStringFault, def:TimeFaultDistance.gaussMeasFaults}
$(\operatorname{gaussStringFault}(v)).\mathrm{timeFaults} = \operatorname{gaussMeasFaults}(v)$.
\end{lemma}

\begin{proof}
\leanok
\uses{def:TimeFaultDistance.gaussStringFault}
This holds by reflexivity (definitional equality).
\end{proof}

\begin{lemma}[Gauss Measurement at Round $r$ is in String]
\label{lem:TimeFaultDistance.gaussStringFault_mem_gauss}
\lean{QEC1.TimeFaultDistance.gaussStringFault_mem_gauss}
\leanok
\uses{def:TimeFaultDistance.gaussStringFault, lem:TimeFaultDistance.gaussStringFault_timeFaults_eq, lem:TimeFaultDistance.mem_gaussMeasFaults}
For any round $r \in \operatorname{Fin}(d)$, the time-fault $\langle \operatorname{phase2}(\operatorname{gaussLaw}(v, r)) \rangle$ belongs to the time-faults of $\operatorname{gaussStringFault}(v)$.
\end{lemma}

\begin{proof}
\leanok
\uses{lem:TimeFaultDistance.gaussStringFault_timeFaults_eq, lem:TimeFaultDistance.mem_gaussMeasFaults}
Rewriting using $\operatorname{gaussStringFault\_timeFaults\_eq}$ and $\operatorname{mem\_gaussMeasFaults}$, the witness $r$ itself suffices.
\end{proof}

\begin{lemma}[No Other Gauss Measurements in String]
\label{lem:TimeFaultDistance.gaussStringFault_no_other_gauss}
\lean{QEC1.TimeFaultDistance.gaussStringFault_no_other_gauss}
\leanok
\uses{def:TimeFaultDistance.gaussStringFault, lem:TimeFaultDistance.gaussStringFault_timeFaults_eq, lem:TimeFaultDistance.mem_gaussMeasFaults}
For $v \neq w$ and any round $r$, the time-fault $\langle \operatorname{phase2}(\operatorname{gaussLaw}(w, r)) \rangle$ does not belong to the time-faults of $\operatorname{gaussStringFault}(v)$.
\end{lemma}

\begin{proof}
\leanok
\uses{lem:TimeFaultDistance.gaussStringFault_timeFaults_eq, lem:TimeFaultDistance.mem_gaussMeasFaults}
Rewriting using $\operatorname{gaussStringFault\_timeFaults\_eq}$ and $\operatorname{mem\_gaussMeasFaults}$, we push the negation inward. For any candidate round $r'$, the equality hypothesis forces $v = w$ by injectivity of the $\operatorname{gaussLaw}$ constructor, contradicting $v \neq w$.
\end{proof}

%% The A_v string flips the gauging sign

\begin{lemma}[Gauss String Fault Sign Flip]
\label{lem:TimeFaultDistance.gaussStringFault_signFlip}
\lean{QEC1.TimeFaultDistance.gaussStringFault_signFlip}
\leanok
\uses{def:TimeFaultDistance.gaussStringFault, def:SpacetimeLogicalFault.gaussSignFlip, def:FaultTolerantGaugingProcedure, lem:TimeFaultDistance.gaussStringFault_mem_gauss, lem:TimeFaultDistance.gaussStringFault_no_other_gauss}
$\operatorname{gaussSignFlip}(\operatorname{gaussStringFault}(v)) = d \pmod{2}$.
\end{lemma}

\begin{proof}
\leanok
\uses{def:SpacetimeLogicalFault.gaussSignFlip, lem:TimeFaultDistance.gaussStringFault_mem_gauss, lem:TimeFaultDistance.gaussStringFault_no_other_gauss}
Unfolding $\operatorname{gaussSignFlip}$, we establish that for each vertex $w$, the inner sum $\sum_{r \in \operatorname{Fin}(d)} [\langle \operatorname{phase2}(\operatorname{gaussLaw}(w,r))\rangle \in F.\mathrm{timeFaults}]$ equals $d \pmod{2}$ if $w = v$ and $0$ otherwise. For $w = v$: by $\operatorname{gaussStringFault\_mem\_gauss}$, every indicator is $1$, and the constant sum gives $d \cdot 1 = d$. For $w \neq v$: by $\operatorname{gaussStringFault\_no\_other\_gauss}$, every indicator is $0$. Substituting these values, the outer sum over $V$ reduces to a single term at $v$ by $\sum_{w \in V} [\![w = v]\!] \cdot (d \bmod 2) = d \pmod{2}$.
\end{proof}

\begin{lemma}[Gauss String Flips Sign When $d$ is Odd]
\label{lem:TimeFaultDistance.gaussStringFault_flipsSign_of_odd}
\lean{QEC1.TimeFaultDistance.gaussStringFault_flipsSign_of_odd}
\leanok
\uses{def:SpacetimeLogicalFault.FlipsGaugingSign, lem:TimeFaultDistance.gaussStringFault_signFlip, def:FaultTolerantGaugingProcedure}
When $d$ is odd, $\operatorname{gaussStringFault}(v)$ flips the gauging sign.
\end{lemma}

\begin{proof}
\leanok
\uses{def:SpacetimeLogicalFault.FlipsGaugingSign, lem:TimeFaultDistance.gaussStringFault_signFlip}
Rewriting $\operatorname{FlipsGaugingSign}$ and applying $\operatorname{gaussStringFault\_signFlip}$, the goal becomes $d \pmod{2} = 1$, which holds since $d$ is odd.
\end{proof}

%% Syndrome-freeness of the A_v string

\begin{theorem}[Gauss String Fault is Syndrome-Free]
\label{thm:TimeFaultDistance.gaussStringFault_syndromeFree}
\lean{QEC1.TimeFaultDistance.gaussStringFault_syndromeFree}
\leanok
\uses{def:TimeFaultDistance.gaussStringFault, def:SpacetimeLogicalFault.IsSyndromeFreeGauging, def:FaultTolerantGaugingProcedure.detectorOfIndex, def:FaultTolerantGaugingProcedure.DetectorIndex, def:FaultTolerantGaugingProcedure.phase1RepeatedDetector_parametric, def:FaultTolerantGaugingProcedure.phase2RepeatedDetector_gauss, def:FaultTolerantGaugingProcedure.phase2RepeatedDetector_flux, def:FaultTolerantGaugingProcedure.phase2RepeatedDetector_deformed, def:FaultTolerantGaugingProcedure.phase3RepeatedDetector, def:FaultTolerantGaugingProcedure.fluxInitDetector, def:FaultTolerantGaugingProcedure.deformedInitDetector, def:FaultTolerantGaugingProcedure.fluxUngaugeDetector, def:FaultTolerantGaugingProcedure.deformedUngaugeDetector, lem:TimeFaultDistance.gaussStringFault_mem_gauss, lem:TimeFaultDistance.gaussStringFault_no_other_gauss}
The $A_v$ string fault is syndrome-free with respect to all detectors from Lemma~4.
\end{theorem}

\begin{proof}
\leanok
\uses{def:FaultTolerantGaugingProcedure.detectorOfIndex, def:FaultTolerantGaugingProcedure.phase1RepeatedDetector_parametric, def:FaultTolerantGaugingProcedure.phase2RepeatedDetector_gauss, def:FaultTolerantGaugingProcedure.phase2RepeatedDetector_flux, def:FaultTolerantGaugingProcedure.phase2RepeatedDetector_deformed, def:FaultTolerantGaugingProcedure.phase3RepeatedDetector, def:FaultTolerantGaugingProcedure.fluxInitDetector, def:FaultTolerantGaugingProcedure.deformedInitDetector, def:FaultTolerantGaugingProcedure.fluxUngaugeDetector, def:FaultTolerantGaugingProcedure.deformedUngaugeDetector, lem:TimeFaultDistance.gaussStringFault_mem_gauss, lem:TimeFaultDistance.gaussStringFault_no_other_gauss}
We verify that no detector is violated by case analysis on the detector index type:

\textbf{Case $\operatorname{phase1Repeated}(j, r, r', hr)$:} The Phase~1 repeated detector involves Phase~1 measurements, which are disjoint from the Gauss Phase~2 measurements in the $A_v$ string. We apply $\operatorname{not\_isViolated\_disjoint}$: for each measurement $m$ in the detector, it cannot be a Gauss $A_v$ measurement at any round, established by simplification of the constructors.

\textbf{Case $\operatorname{phase2GaussRepeated}(w, r, r', hr)$:} This detector compares $A_w$ at consecutive rounds $r$ and $r'$. Rewriting using $\operatorname{isViolated\_iff\_flipParity}$ and unfolding the detector, the flip parity is a sum over a pair of measurements. If $v = w$: both measurements are in the string by $\operatorname{gaussStringFault\_mem\_gauss}$, so both indicators are $1$ and the sum is $1 + 1 = 0$ in $\mathbb{Z}/2\mathbb{Z}$ (by the characteristic-two identity). If $v \neq w$: both measurements are absent by $\operatorname{gaussStringFault\_no\_other\_gauss}$, so both indicators are $0$ and the sum is $0$.

\textbf{Case $\operatorname{phase2FluxRepeated}(p, r, r', hr)$:} The flux repeated detector involves flux measurements $B_p$, which are disjoint from Gauss $A_v$ measurements. We apply $\operatorname{not\_isViolated\_disjoint}$ with the observation that no flux measurement equals a Gauss measurement.

\textbf{Case $\operatorname{phase2DeformedRepeated}(j, r, r', hr)$:} The deformed repeated detector involves deformed check measurements $\tilde{s}_j$, which are disjoint from Gauss $A_v$ measurements. We apply $\operatorname{not\_isViolated\_disjoint}$ similarly.

\textbf{Case $\operatorname{phase3Repeated}(j, r, r', hr)$:} The Phase~3 repeated detector involves Phase~3 measurements, disjoint from Phase~2 Gauss measurements. We apply $\operatorname{not\_isViolated\_disjoint}$.

\textbf{Case $\operatorname{fluxInit}(p)$:} The flux initialization detector involves edge initialization and flux measurements, both disjoint from Gauss $A_v$ measurements. We apply $\operatorname{not\_isViolated\_disjoint}$.

\textbf{Case $\operatorname{deformedInit}(j)$:} The deformed initialization detector involves edge initialization, Phase~1, and deformed measurements, all disjoint from Gauss $A_v$ measurements. We apply $\operatorname{not\_isViolated\_disjoint}$.

\textbf{Case $\operatorname{fluxUngauge}(p)$:} The flux ungauge detector involves Phase~3 and flux measurements, disjoint from Gauss measurements. We apply $\operatorname{not\_isViolated\_disjoint}$.

\textbf{Case $\operatorname{deformedUngauge}(j)$:} The deformed ungauge detector involves Phase~3, deformed, and edge-$Z$ measurements, all disjoint from Gauss measurements. We apply $\operatorname{not\_isViolated\_disjoint}$.
\end{proof}

%% Part III: Upper Bound

\begin{theorem}[Gauss String Fault is a Logical Fault]
\label{thm:TimeFaultDistance.gaussStringFault_isLogicalFault}
\lean{QEC1.TimeFaultDistance.gaussStringFault_isLogicalFault}
\leanok
\uses{def:SpacetimeLogicalFault.IsGaugingLogicalFault, def:SpacetimeLogicalFault.PreservesGaugingSign, def:TimeFaultDistance.gaussStringFault, thm:TimeFaultDistance.gaussStringFault_syndromeFree, lem:TimeFaultDistance.gaussStringFault_signFlip, def:FaultTolerantGaugingProcedure}
When $d$ is odd, the $A_v$ string fault is a gauging logical fault.
\end{theorem}

\begin{proof}
\leanok
\uses{thm:TimeFaultDistance.gaussStringFault_syndromeFree, lem:TimeFaultDistance.gaussStringFault_signFlip, def:SpacetimeLogicalFault.PreservesGaugingSign}
The fault is syndrome-free by $\operatorname{gaussStringFault\_syndromeFree}$. It does not preserve the gauging sign because, rewriting $\operatorname{PreservesGaugingSign}$ and applying $\operatorname{gaussStringFault\_signFlip}$, we get $\operatorname{gaussSignFlip} = d \pmod{2} = 1 \neq 0$ since $d$ is odd.
\end{proof}

\begin{theorem}[Upper Bound: $d_{\mathrm{time}} \leq d$]
\label{thm:TimeFaultDistance.timeFaultDistance_le_d}
\lean{QEC1.TimeFaultDistance.timeFaultDistance_le_d}
\leanok
\uses{def:TimeFaultDistance.timeFaultDistance, def:TimeFaultDistance.gaussStringFault, lem:TimeFaultDistance.gaussStringFault_isPureTime, thm:TimeFaultDistance.gaussStringFault_isLogicalFault, lem:TimeFaultDistance.gaussStringFault_weight_eq_d, def:FaultTolerantGaugingProcedure}
When $d$ is odd and $V$ is nonempty (witnessed by $v \in V$), the time fault-distance satisfies $d_{\mathrm{time}} \leq d$.
\end{theorem}

\begin{proof}
\leanok
\uses{def:TimeFaultDistance.timeFaultDistance, def:TimeFaultDistance.gaussStringFault, lem:TimeFaultDistance.gaussStringFault_isPureTime, thm:TimeFaultDistance.gaussStringFault_isLogicalFault, lem:TimeFaultDistance.gaussStringFault_weight_eq_d}
Unfolding the definition of $d_{\mathrm{time}}$ as an infimum, we apply $\operatorname{Nat.sInf\_le}$ with the witness $\operatorname{gaussStringFault}(v)$. This fault is pure-time (by $\operatorname{gaussStringFault\_isPureTime}$), is a gauging logical fault (by $\operatorname{gaussStringFault\_isLogicalFault}$), and has weight $d$ (by $\operatorname{gaussStringFault\_weight\_eq\_d}$).
\end{proof}

%% Part IV: Lower Bound

\begin{lemma}[Pure-Time Fault Weight Equals Time-Fault Count]
\label{lem:TimeFaultDistance.pureTime_weight_eq_card}
\lean{QEC1.TimeFaultDistance.pureTime_weight_eq_card}
\leanok
\uses{def:SpacetimeFault, def:FaultTolerantGaugingProcedure}
For a pure-time fault $F$, $\operatorname{weight}(F) = |F.\mathrm{timeFaults}|$.
\end{lemma}

\begin{proof}
\leanok
\uses{def:SpacetimeFault}
By $\operatorname{isPureTime}$, the space-faults are empty: $F.\mathrm{spaceFaults} = \emptyset$. Thus $\operatorname{weight}(F) = |\emptyset| + |F.\mathrm{timeFaults}| = |F.\mathrm{timeFaults}|$.
\end{proof}

\begin{definition}[Gauss Fault Count]
\label{def:TimeFaultDistance.gaussFaultCount}
\lean{QEC1.TimeFaultDistance.gaussFaultCount}
\leanok
\uses{def:SpacetimeFault, def:FaultTolerantGaugingProcedure, def:FTGaugingMeasurement, def:DeformedCodeMeasurement}
The \emph{Gauss fault count} for vertex $v$ in a spacetime fault $F$ is the number of rounds $r \in \operatorname{Fin}(d)$ at which the $A_v$ measurement at round $r$ is a time-fault:
\[
\operatorname{gaussFaultCount}(v, F) = \bigl|\bigl\{ r \in \operatorname{Fin}(d) \mid \langle \operatorname{phase2}(\operatorname{gaussLaw}(v, r)) \rangle \in F.\mathrm{timeFaults} \bigr\}\bigr|.
\]
\end{definition}

\begin{lemma}[Gauss Sign Flip Equals Sum of Parities]
\label{lem:TimeFaultDistance.gaussSignFlip_eq_sum_parities}
\lean{QEC1.TimeFaultDistance.gaussSignFlip_eq_sum_parities}
\leanok
\uses{def:SpacetimeLogicalFault.gaussSignFlip, def:TimeFaultDistance.gaussFaultCount, def:FaultTolerantGaugingProcedure}
$\operatorname{gaussSignFlip}(F) = \sum_{v \in V} \operatorname{gaussFaultCount}(v, F) \pmod{2}$.
\end{lemma}

\begin{proof}
\leanok
\uses{def:SpacetimeLogicalFault.gaussSignFlip, def:TimeFaultDistance.gaussFaultCount}
Unfolding both definitions, we apply congruence. For each vertex $v$, rewriting the cardinality of the filter using $\operatorname{Finset.card\_filter}$ and simplifying with $\operatorname{Finset.sum\_boole}$ gives the equality.
\end{proof}

\begin{lemma}[Consecutive Gauss Rounds Have Same Fault Status]
\label{lem:TimeFaultDistance.syndromeFree_gauss_consecutive}
\lean{QEC1.TimeFaultDistance.syndromeFree_gauss_consecutive}
\leanok
\uses{def:SpacetimeLogicalFault.IsSyndromeFreeGauging, def:FaultTolerantGaugingProcedure.detectorOfIndex, def:FaultTolerantGaugingProcedure.phase2RepeatedDetector_gauss, def:FaultTolerantGaugingProcedure, def:SpacetimeFault, def:FTGaugingMeasurement, def:DeformedCodeMeasurement}
For a syndrome-free fault $F$ and consecutive rounds $r, r'$ with $r + 1 = r'$, the $A_v$ measurement at round $r$ is faulted if and only if the $A_v$ measurement at round $r'$ is faulted.
\end{lemma}

\begin{proof}
\leanok
\uses{def:SpacetimeLogicalFault.IsSyndromeFreeGauging, def:FaultTolerantGaugingProcedure.detectorOfIndex, def:FaultTolerantGaugingProcedure.phase2RepeatedDetector_gauss}
The syndrome-freeness hypothesis applied to the detector $\operatorname{phase2GaussRepeated}(v, r, r', hr)$ gives that this detector is not violated. Rewriting using $\operatorname{isViolated\_iff\_flipParity}$ and unfolding the Gauss repeated detector, the flip parity is a sum over the pair $\{m_r, m_{r'}\}$. Since $r \neq r'$ (as $r + 1 = r'$), this is a two-element sum. We perform case analysis on whether each measurement is in $F.\mathrm{timeFaults}$: if both are present or both absent, the sum is $0$ (consistent with non-violation); if exactly one is present, the sum is $1$ (contradicting non-violation). This establishes the biconditional.
\end{proof}

\begin{lemma}[All-or-None Property for Gauss Faults]
\label{lem:TimeFaultDistance.syndromeFree_gauss_all_or_none}
\lean{QEC1.TimeFaultDistance.syndromeFree_gauss_all_or_none}
\leanok
\uses{def:SpacetimeLogicalFault.IsSyndromeFreeGauging, def:FaultTolerantGaugingProcedure.detectorOfIndex, lem:TimeFaultDistance.syndromeFree_gauss_consecutive, def:FaultTolerantGaugingProcedure, def:SpacetimeFault}
For a syndrome-free fault $F$ and any two rounds $r, r' \in \operatorname{Fin}(d)$, the $A_v$ measurement at round $r$ is faulted if and only if the $A_v$ measurement at round $r'$ is faulted.
\end{lemma}

\begin{proof}
\leanok
\uses{lem:TimeFaultDistance.syndromeFree_gauss_consecutive}
It suffices to show the claim for $r \leq r'$ (the other direction follows by symmetry of the biconditional). We proceed by induction on the difference $k = r' - r$.

\textbf{Base case} ($k = 0$): Then $r = r'$ by $\operatorname{Fin.ext}$, so the biconditional holds by reflexivity.

\textbf{Inductive step} ($k = n + 1$): We have $r' = r + n + 1$. By the induction hypothesis, the fault status at round $r$ equals that at round $r + n$. By $\operatorname{syndromeFree\_gauss\_consecutive}$, the fault status at round $r + n$ equals that at round $r + n + 1 = r'$. Composing these two biconditionals via transitivity gives the result.
\end{proof}

\begin{lemma}[Gauss Fault Count is $0$ or $d$]
\label{lem:TimeFaultDistance.gaussFaultCount_zero_or_d}
\lean{QEC1.TimeFaultDistance.gaussFaultCount_zero_or_d}
\leanok
\uses{def:TimeFaultDistance.gaussFaultCount, def:SpacetimeLogicalFault.IsSyndromeFreeGauging, def:FaultTolerantGaugingProcedure.detectorOfIndex, lem:TimeFaultDistance.syndromeFree_gauss_all_or_none, def:FaultTolerantGaugingProcedure}
For a syndrome-free fault, $\operatorname{gaussFaultCount}(v, F) = 0$ or $\operatorname{gaussFaultCount}(v, F) = d$.
\end{lemma}

\begin{proof}
\leanok
\uses{def:TimeFaultDistance.gaussFaultCount, lem:TimeFaultDistance.syndromeFree_gauss_all_or_none}
Unfolding $\operatorname{gaussFaultCount}$, if $d = 0$ then the filter over $\operatorname{Fin}(0)$ is empty, giving count $0$. Otherwise, $d > 0$ and we let $r_0 = 0 \in \operatorname{Fin}(d)$. By case analysis on whether $A_v$ at round $r_0$ is faulted:

If faulted: by $\operatorname{syndromeFree\_gauss\_all\_or\_none}$, every round $r$ is faulted, so the filter equals $\operatorname{Fin}(d)$ and the count is $d$.

If not faulted: by $\operatorname{syndromeFree\_gauss\_all\_or\_none}$, no round $r$ is faulted (any faulted round would imply $r_0$ is faulted), so the filter is empty and the count is $0$.
\end{proof}

\begin{lemma}[Gauss Fault Parity is $0$ or $d \pmod{2}$]
\label{lem:TimeFaultDistance.gaussFaultParity_zero_or_d}
\lean{QEC1.TimeFaultDistance.gaussFaultParity_zero_or_d}
\leanok
\uses{def:TimeFaultDistance.gaussFaultCount, def:SpacetimeLogicalFault.IsSyndromeFreeGauging, lem:TimeFaultDistance.gaussFaultCount_zero_or_d, def:FaultTolerantGaugingProcedure}
For a syndrome-free fault, $\operatorname{gaussFaultCount}(v, F) \equiv 0 \pmod{2}$ or $\operatorname{gaussFaultCount}(v, F) \equiv d \pmod{2}$.
\end{lemma}

\begin{proof}
\leanok
\uses{lem:TimeFaultDistance.gaussFaultCount_zero_or_d}
By $\operatorname{gaussFaultCount\_zero\_or\_d}$, the count is $0$ or $d$. In either case, the corresponding congruence modulo $2$ follows by simplification.
\end{proof}

\begin{lemma}[Sign Flip Implies Odd Number of Full Strings]
\label{lem:TimeFaultDistance.sign_flip_implies_odd_full_strings}
\lean{QEC1.TimeFaultDistance.sign_flip_implies_odd_full_strings}
\leanok
\uses{def:SpacetimeLogicalFault.FlipsGaugingSign, def:SpacetimeLogicalFault.IsSyndromeFreeGauging, def:TimeFaultDistance.gaussFaultCount, lem:TimeFaultDistance.gaussSignFlip_eq_sum_parities, lem:TimeFaultDistance.gaussFaultParity_zero_or_d, lem:TimeFaultDistance.gaussFaultCount_zero_or_d, def:FaultTolerantGaugingProcedure}
For a syndrome-free fault that flips the gauging sign when $d$ is odd, the number of vertices $v$ with $\operatorname{gaussFaultCount}(v, F) = d$ is odd.
\end{lemma}

\begin{proof}
\leanok
\uses{def:SpacetimeLogicalFault.FlipsGaugingSign, lem:TimeFaultDistance.gaussSignFlip_eq_sum_parities, lem:TimeFaultDistance.gaussFaultParity_zero_or_d, lem:TimeFaultDistance.gaussFaultCount_zero_or_d}
Rewriting $\operatorname{FlipsGaugingSign}$ and applying $\operatorname{gaussSignFlip\_eq\_sum\_parities}$, we have $\sum_{v \in V} \operatorname{gaussFaultCount}(v, F) \equiv 1 \pmod{2}$. Since $d$ is odd, $d \equiv 1 \pmod{2}$. By $\operatorname{gaussFaultParity\_zero\_or\_d}$, each summand is either $0$ or $1$ in $\mathbb{Z}/2\mathbb{Z}$. The sum therefore equals the cardinality (mod 2) of the set $\{v \mid \operatorname{gaussFaultCount}(v,F) \equiv 1 \pmod{2}\}$. We verify this filter equals $\{v \mid \operatorname{gaussFaultCount}(v,F) = d\}$ using $\operatorname{gaussFaultCount\_zero\_or\_d}$: count $= 0$ gives parity $0$, and count $= d$ gives parity $1$ (since $d$ is odd). The cardinality being $1 \pmod{2}$ means the set has odd cardinality, as verified by $\operatorname{ZMod.natCast\_eq\_one\_iff\_odd}$.
\end{proof}

\begin{lemma}[Total Gauss Fault Count $\leq$ Weight]
\label{lem:TimeFaultDistance.pureTime_weight_ge_gauss_sum}
\lean{QEC1.TimeFaultDistance.pureTime_weight_ge_gauss_sum}
\leanok
\uses{def:TimeFaultDistance.gaussFaultCount, lem:TimeFaultDistance.pureTime_weight_eq_card, def:SpacetimeFault, def:FaultTolerantGaugingProcedure}
For a pure-time fault $F$, $\sum_{v \in V} \operatorname{gaussFaultCount}(v, F) \leq \operatorname{weight}(F)$.
\end{lemma}

\begin{proof}
\leanok
\uses{lem:TimeFaultDistance.pureTime_weight_eq_card, def:TimeFaultDistance.gaussFaultCount}
Rewriting the weight as $|F.\mathrm{timeFaults}|$ using $\operatorname{pureTime\_weight\_eq\_card}$, we show that the biUnion of the mapped Gauss faults (over all vertices) is a subset of $F.\mathrm{timeFaults}$: each element of the biUnion is obtained from some $\langle \operatorname{phase2}(\operatorname{gaussLaw}(v,r)) \rangle$ that is in $F.\mathrm{timeFaults}$ by the filter condition. The sum of cardinalities of the per-vertex image sets equals the cardinality of their disjoint union (since different vertices produce distinct measurement labels by injectivity of the $\operatorname{gaussLaw}$ constructor). Thus $\sum_v \operatorname{gaussFaultCount}(v,F) = |\text{biUnion}| \leq |F.\mathrm{timeFaults}|$ by monotonicity of cardinality.
\end{proof}

\begin{lemma}[Pure-Time Fault with Full String has Weight $\geq d$]
\label{lem:TimeFaultDistance.pureTime_weight_ge_d_of_full_string}
\lean{QEC1.TimeFaultDistance.pureTime_weight_ge_d_of_full_string}
\leanok
\uses{def:TimeFaultDistance.gaussFaultCount, lem:TimeFaultDistance.pureTime_weight_ge_gauss_sum, def:SpacetimeFault, def:FaultTolerantGaugingProcedure}
For a pure-time fault $F$ with $\operatorname{gaussFaultCount}(v, F) = d$ for some $v$, we have $d \leq \operatorname{weight}(F)$.
\end{lemma}

\begin{proof}
\leanok
\uses{lem:TimeFaultDistance.pureTime_weight_ge_gauss_sum}
By a chain of inequalities:
\[
d = \operatorname{gaussFaultCount}(v, F) \leq \sum_{w \in V} \operatorname{gaussFaultCount}(w, F) \leq \operatorname{weight}(F),
\]
where the first inequality uses that a single term of a nonneg sum is at most the total sum ($\operatorname{Finset.single\_le\_sum}$), and the second is $\operatorname{pureTime\_weight\_ge\_gauss\_sum}$.
\end{proof}

\begin{theorem}[Lower Bound: Any Pure-Time Logical Fault has Weight $\geq d$]
\label{thm:TimeFaultDistance.pureTime_logicalFault_weight_ge_d}
\lean{QEC1.TimeFaultDistance.pureTime_logicalFault_weight_ge_d}
\leanok
\uses{def:TimeFaultDistance.IsPureTimeFault, def:SpacetimeLogicalFault.IsSyndromeFreeGauging, def:SpacetimeLogicalFault.FlipsGaugingSign, lem:TimeFaultDistance.sign_flip_implies_odd_full_strings, lem:TimeFaultDistance.pureTime_weight_ge_d_of_full_string, def:TimeFaultDistance.gaussFaultCount, def:FaultTolerantGaugingProcedure}
If $F$ is a pure-time fault that is syndrome-free, flips the gauging sign, and $d$ is odd, then $d \leq \operatorname{weight}(F)$.
\end{theorem}

\begin{proof}
\leanok
\uses{lem:TimeFaultDistance.sign_flip_implies_odd_full_strings, lem:TimeFaultDistance.pureTime_weight_ge_d_of_full_string, def:TimeFaultDistance.gaussFaultCount}
By $\operatorname{sign\_flip\_implies\_odd\_full\_strings}$, the set $\{v \mid \operatorname{gaussFaultCount}(v, F) = d\}$ has odd cardinality, hence is nonempty. Let $v$ be an element of this set. Then $\operatorname{gaussFaultCount}(v, F) = d$, and by $\operatorname{pureTime\_weight\_ge\_d\_of\_full\_string}$, we conclude $d \leq \operatorname{weight}(F)$.
\end{proof}

%% Part V: Main Theorem

\begin{theorem}[Lower Bound: $d_{\mathrm{time}} \geq d$]
\label{thm:TimeFaultDistance.timeFaultDistance_ge_d}
\lean{QEC1.TimeFaultDistance.timeFaultDistance_ge_d}
\leanok
\uses{def:TimeFaultDistance.timeFaultDistance, def:TimeFaultDistance.pureTimeLogicalFaultWeights, def:SpacetimeLogicalFault.PreservesGaugingSign, thm:TimeFaultDistance.pureTime_logicalFault_weight_ge_d, thm:SpacetimeLogicalFault.gaussSignFlip_zero_or_one, def:TimeFaultDistance.gaussStringFault, lem:TimeFaultDistance.gaussStringFault_isPureTime, thm:TimeFaultDistance.gaussStringFault_isLogicalFault, lem:TimeFaultDistance.gaussStringFault_weight_eq_d, def:FaultTolerantGaugingProcedure, def:FaultTolerantGaugingProcedure.detectorOfIndex}
When $d$ is odd and $V$ is nonempty, $d \leq d_{\mathrm{time}}$.
\end{theorem}

\begin{proof}
\leanok
\uses{def:TimeFaultDistance.timeFaultDistance, def:TimeFaultDistance.pureTimeLogicalFaultWeights, def:SpacetimeLogicalFault.PreservesGaugingSign, thm:TimeFaultDistance.pureTime_logicalFault_weight_ge_d, thm:SpacetimeLogicalFault.gaussSignFlip_zero_or_one, def:TimeFaultDistance.gaussStringFault, lem:TimeFaultDistance.gaussStringFault_isPureTime, thm:TimeFaultDistance.gaussStringFault_isLogicalFault, lem:TimeFaultDistance.gaussStringFault_weight_eq_d}
Unfolding $d_{\mathrm{time}}$ as an infimum, we perform case analysis on whether the set of pure-time logical fault weights is nonempty.

If nonempty: we apply $\operatorname{le\_csInf}$ and take an arbitrary element $w$ with witness $F$ satisfying $\operatorname{IsPureTimeFault}(F)$, $\operatorname{IsGaugingLogicalFault}(F)$, and $\operatorname{weight}(F) = w$. The gauging logical fault condition gives syndrome-freeness and non-preservation of the sign. By $\operatorname{gaussSignFlip\_zero\_or\_one}$, the sign flip is either $0$ or $1$; since it is not $0$ (non-preservation), it is $1$, meaning the fault flips the sign. By $\operatorname{pureTime\_logicalFault\_weight\_ge\_d}$, $d \leq \operatorname{weight}(F) = w$.

If empty: we derive a contradiction, since $\operatorname{gaussStringFault}(v)$ (for $v \in V$ from nonemptiness) is a pure-time logical fault of weight $d$, so $d$ belongs to the weight set, contradicting emptiness.
\end{proof}

\begin{theorem}[Lemma 6: Time Fault-Distance Equals $d$]
\label{thm:TimeFaultDistance.timeFaultDistance_eq_d}
\lean{QEC1.TimeFaultDistance.timeFaultDistance_eq_d}
\leanok
\uses{def:TimeFaultDistance.timeFaultDistance, thm:TimeFaultDistance.timeFaultDistance_le_d, thm:TimeFaultDistance.timeFaultDistance_ge_d, def:FaultTolerantGaugingProcedure, def:FaultTolerantGaugingProcedure.detectorOfIndex, def:SpacetimeLogicalFault.PreservesGaugingSign}
When $d$ is odd and $V$ is nonempty,
\[
d_{\mathrm{time}} = d.
\]
\end{theorem}

\begin{proof}
\leanok
\uses{thm:TimeFaultDistance.timeFaultDistance_le_d, thm:TimeFaultDistance.timeFaultDistance_ge_d}
By antisymmetry: $d_{\mathrm{time}} \leq d$ from $\operatorname{timeFaultDistance\_le\_d}$ and $d \leq d_{\mathrm{time}}$ from $\operatorname{timeFaultDistance\_ge\_d}$.
\end{proof}

%% Part V-b: Justification of the Outcome Predicate

\begin{theorem}[Preserving Sign Decomposes into Generators]
\label{thm:TimeFaultDistance.pureTime_preservesSign_decomposes_into_generators}
\lean{QEC1.TimeFaultDistance.pureTime_preservesSign_decomposes_into_generators}
\leanok
\uses{def:TimeFaultDistance.IsPureTimeFault, def:SpacetimeLogicalFault.IsSyndromeFreeGauging, def:SpacetimeLogicalFault.PreservesGaugingSign, def:SpacetimeStabilizers.IsListedGenerator, def:SpacetimeFault, def:FaultTolerantGaugingProcedure, def:FaultTolerantGaugingProcedure.detectorOfIndex}
Every syndrome-free pure-time fault that preserves the Gauss sign decomposes into the listed spacetime stabilizer generators from Lemma~5. That is, there exists a list of generators $\mathit{gens}$ such that every $g \in \mathit{gens}$ satisfies $\operatorname{IsListedGenerator}(g)$ and $F = \operatorname{foldl}(\operatorname{compose}, \operatorname{empty}, \mathit{gens})$.
\end{theorem}

\begin{proof}
\leanok
\uses{def:SpacetimeStabilizers.IsListedGenerator, thm:SpacetimeStabilizers.spacetimeStabilizer_completeness}
This follows directly from $\operatorname{spacetimeStabilizer\_completeness}$ applied to $F$ with the pair of hypotheses (syndrome-freeness and sign preservation).
\end{proof}

\begin{theorem}[Pure-Time Syndrome-Free Dichotomy]
\label{thm:TimeFaultDistance.pureTime_syndromeFree_logical_or_stabilizer_generators}
\lean{QEC1.TimeFaultDistance.pureTime_syndromeFree_logical_or_stabilizer_generators}
\leanok
\uses{def:TimeFaultDistance.IsPureTimeFault, def:SpacetimeLogicalFault.IsSyndromeFreeGauging, def:SpacetimeLogicalFault.FlipsGaugingSign, def:SpacetimeLogicalFault.PreservesGaugingSign, def:SpacetimeStabilizers.IsListedGenerator, thm:SpacetimeLogicalFault.flipsOrPreserves, thm:TimeFaultDistance.pureTime_preservesSign_decomposes_into_generators, def:SpacetimeFault, def:FaultTolerantGaugingProcedure, def:FaultTolerantGaugingProcedure.detectorOfIndex}
For a syndrome-free pure-time fault $F$, exactly one of the following holds:
\begin{enumerate}
\item $F$ flips the gauging sign (i.e., is a logical fault), or
\item $F$ preserves the gauging sign and decomposes into the listed stabilizer generators (i.e., is a trivial fault).
\end{enumerate}
\end{theorem}

\begin{proof}
\leanok
\uses{thm:SpacetimeLogicalFault.flipsOrPreserves, thm:TimeFaultDistance.pureTime_preservesSign_decomposes_into_generators}
By $\operatorname{flipsOrPreserves}$, either $F$ flips or preserves the gauging sign. In the first case, the left disjunct holds. In the second case, we combine the preservation hypothesis with $\operatorname{pureTime\_preservesSign\_decomposes\_into\_generators}$ to obtain the generator decomposition.
\end{proof}

%% Part VI: Corollaries

\begin{corollary}[Pure-Time Fault Below Distance is Stabilizer]
\label{cor:TimeFaultDistance.pureTime_weight_lt_d_is_stabilizer}
\lean{QEC1.TimeFaultDistance.pureTime_weight_lt_d_is_stabilizer}
\leanok
\uses{def:TimeFaultDistance.IsPureTimeFault, def:SpacetimeLogicalFault.IsSyndromeFreeGauging, def:SpacetimeLogicalFault.IsGaugingStabilizer, def:SpacetimeLogicalFault.PreservesGaugingSign, thm:TimeFaultDistance.pureTime_logicalFault_weight_ge_d, thm:SpacetimeLogicalFault.gaussSignFlip_zero_or_one, def:FaultTolerantGaugingProcedure, def:FaultTolerantGaugingProcedure.detectorOfIndex}
Any syndrome-free pure-time fault of weight $< d$ (when $d$ is odd) is a gauging stabilizer.
\end{corollary}

\begin{proof}
\leanok
\uses{def:SpacetimeLogicalFault.PreservesGaugingSign, thm:TimeFaultDistance.pureTime_logicalFault_weight_ge_d, thm:SpacetimeLogicalFault.gaussSignFlip_zero_or_one}
The fault is syndrome-free by hypothesis. Suppose for contradiction that the sign is not preserved. By $\operatorname{gaussSignFlip\_zero\_or\_one}$, the sign flip is $0$ or $1$; since it is not $0$, it is $1$, meaning the fault flips the sign. By $\operatorname{pureTime\_logicalFault\_weight\_ge\_d}$, $d \leq \operatorname{weight}(F)$, contradicting $\operatorname{weight}(F) < d$.
\end{proof}

\begin{corollary}[Gauss Fault Count Dichotomy]
\label{cor:TimeFaultDistance.gaussFaultCount_dichotomy}
\lean{QEC1.TimeFaultDistance.gaussFaultCount_dichotomy}
\leanok
\uses{def:TimeFaultDistance.gaussFaultCount, def:SpacetimeLogicalFault.IsSyndromeFreeGauging, lem:TimeFaultDistance.gaussFaultCount_zero_or_d, def:FaultTolerantGaugingProcedure, def:FaultTolerantGaugingProcedure.detectorOfIndex}
For any syndrome-free fault and any vertex $v$, $\operatorname{gaussFaultCount}(v, F) = 0$ or $\operatorname{gaussFaultCount}(v, F) = d$.
\end{corollary}

\begin{proof}
\leanok
\uses{lem:TimeFaultDistance.gaussFaultCount_zero_or_d}
This is exactly $\operatorname{gaussFaultCount\_zero\_or\_d}$.
\end{proof}

\begin{corollary}[Time Fault-Distance Equals Phase 2 Duration]
\label{cor:TimeFaultDistance.timeFaultDistance_eq_phase2_duration}
\lean{QEC1.TimeFaultDistance.timeFaultDistance_eq_phase2_duration}
\leanok
\uses{def:TimeFaultDistance.timeFaultDistance, thm:TimeFaultDistance.timeFaultDistance_eq_d, def:FaultTolerantGaugingProcedure, def:FaultTolerantGaugingProcedure.detectorOfIndex, def:SpacetimeLogicalFault.PreservesGaugingSign}
When $d$ is odd and $V$ is nonempty,
\[
d_{\mathrm{time}} = t_o - t_i,
\]
where $t_o - t_i$ is the Phase~2 duration.
\end{corollary}

\begin{proof}
\leanok
\uses{thm:TimeFaultDistance.timeFaultDistance_eq_d}
Rewriting using $\operatorname{timeFaultDistance\_eq\_d}$ and $\operatorname{phase2\_duration}$.
\end{proof}

\begin{corollary}[Minimum-Weight Logical Fault is a Single String]
\label{cor:TimeFaultDistance.minimum_weight_is_single_string}
\lean{QEC1.TimeFaultDistance.minimum_weight_is_single_string}
\leanok
\uses{def:TimeFaultDistance.IsPureTimeFault, def:SpacetimeLogicalFault.IsSyndromeFreeGauging, def:SpacetimeLogicalFault.FlipsGaugingSign, def:TimeFaultDistance.gaussFaultCount, lem:TimeFaultDistance.sign_flip_implies_odd_full_strings, lem:TimeFaultDistance.pureTime_weight_ge_gauss_sum, def:FaultTolerantGaugingProcedure}
Any pure-time fault that is syndrome-free, flips the gauging sign, has weight exactly $d$, and $d$ is odd, has exactly one vertex $v$ with $\operatorname{gaussFaultCount}(v, F) = d$.
\end{corollary}

\begin{proof}
\leanok
\uses{lem:TimeFaultDistance.sign_flip_implies_odd_full_strings, lem:TimeFaultDistance.pureTime_weight_ge_gauss_sum}
By $\operatorname{sign\_flip\_implies\_odd\_full\_strings}$, the set $S = \{v \mid \operatorname{gaussFaultCount}(v, F) = d\}$ has odd cardinality, hence $|S| \geq 1$. We show $|S| \leq 1$ by contradiction: if $|S| \geq 2$, there exist distinct $v_1, v_2 \in S$ with $\operatorname{gaussFaultCount}(v_i, F) = d$. Then
\[
2d = \operatorname{gaussFaultCount}(v_1, F) + \operatorname{gaussFaultCount}(v_2, F) \leq \sum_{w \in V} \operatorname{gaussFaultCount}(w, F) \leq \operatorname{weight}(F) = d,
\]
where the second inequality is $\operatorname{pureTime\_weight\_ge\_gauss\_sum}$, giving $2d \leq d$. Since $d > 0$ (as $d$ is odd), this is a contradiction. Therefore $|S| = 1$.
\end{proof}

%--- Lem_7: SpaceTimeDecoupling ---

I'll start by reading the Lean file to understand its contents.Now I have a thorough understanding of the file. Let me produce the complete LaTeX translation.

\chapter{Lem 7: Space-Time Decoupling}

\begin{definition}[The Deformed Code]
\label{def:SpaceTimeDecoupling.theDeformedCode}
\lean{QEC1.SpaceTimeDecoupling.theDeformedCode}
\leanok
\uses{def:DeformedCodeMeasurement, def:FaultTolerantGaugingProcedure}
The deformed stabilizer code at the gauging time $t_i$, constructed from the fault-tolerant gauging procedure's data. Given a procedure $\mathrm{proc}$, the deformed code is defined as the deformed stabilizer code built from the procedure's deformed data and cycle parity, using the commutativity of the original checks.
\end{definition}

\begin{definition}[Full Outcome Preserving]
\label{def:SpaceTimeDecoupling.FullOutcomePreserving}
\lean{QEC1.SpaceTimeDecoupling.FullOutcomePreserving}
\leanok
\uses{def:FaultTolerantGaugingProcedure, def:SpacetimeFault, def:SpacetimeLogicalFault.PreservesGaugingSign, def:SpacetimeLogicalFault.gaussSignFlip}
The full outcome-preserving predicate for the fault-tolerant gauging procedure.
A spacetime fault $F$ preserves the outcome if \emph{both}:
\begin{enumerate}
\item The gauging sign $\sigma$ is preserved (no time-logical effect), i.e., $\operatorname{PreservesGaugingSign}(\mathrm{proc}, F)$, AND
\item The net Pauli error at $t_i$ is in the deformed stabilizer group (no space-logical effect), i.e., $F.\operatorname{pauliErrorAt}(\mathrm{proc}.\mathrm{phase2Start}) \in \mathcal{S}(\text{deformedCode})$.
\end{enumerate}
This captures the paper's Def~11: a spacetime stabilizer leaves both the classical measurement record and the quantum code state unchanged.
\end{definition}

\begin{definition}[Full Gauging Logical Fault]
\label{def:SpaceTimeDecoupling.IsFullGaugingLogicalFault}
\lean{QEC1.SpaceTimeDecoupling.IsFullGaugingLogicalFault}
\leanok
\uses{def:SpaceTimeDecoupling.FullOutcomePreserving, def:SpacetimeLogicalFault.IsGaugingLogicalFault, def:FaultTolerantGaugingProcedure}
A spacetime logical fault under the full outcome predicate: the fault is syndrome-free and changes the outcome (either flips the gauging sign or applies a nontrivial logical operator to the code state).
Formally, $\operatorname{IsFullGaugingLogicalFault}(\mathrm{proc}, F) := \operatorname{IsGaugingLogicalFault}(\mathrm{proc}, \mathrm{proc}.\mathrm{detectorOfIndex}, \operatorname{FullOutcomePreserving}(\mathrm{proc}), F)$.
\end{definition}

\begin{definition}[Full Gauging Stabilizer]
\label{def:SpaceTimeDecoupling.IsFullGaugingStabilizer}
\lean{QEC1.SpaceTimeDecoupling.IsFullGaugingStabilizer}
\leanok
\uses{def:SpaceTimeDecoupling.FullOutcomePreserving, def:SpacetimeLogicalFault.IsGaugingStabilizer, def:FaultTolerantGaugingProcedure}
A spacetime stabilizer under the full outcome predicate: the fault is syndrome-free and preserves both the gauging sign and the code state.
Formally, $\operatorname{IsFullGaugingStabilizer}(\mathrm{proc}, F) := \operatorname{IsGaugingStabilizer}(\mathrm{proc}, \mathrm{proc}.\mathrm{detectorOfIndex}, \operatorname{FullOutcomePreserving}(\mathrm{proc}), F)$.
\end{definition}

\begin{theorem}[Logical Fault Outcome Change]
\label{thm:SpaceTimeDecoupling.logicalFault_outcome_change}
\lean{QEC1.SpaceTimeDecoupling.logicalFault_outcome_change}
\leanok
\uses{def:SpaceTimeDecoupling.IsFullGaugingLogicalFault, def:SpaceTimeDecoupling.FullOutcomePreserving, def:SpacetimeLogicalFault.FlipsGaugingSign, thm:SpacetimeLogicalFault.flipsOrPreserves}
A logical fault changes the outcome: $\neg(\text{sign preserved} \land \text{Pauli} \in \mathcal{S})$.
Equivalently: either the sign is flipped, or the Pauli error is a nontrivial logical.
Formally, if $F$ is a full gauging logical fault, then
\[
\operatorname{FlipsGaugingSign}(\mathrm{proc}, F) \;\lor\; F.\operatorname{pauliErrorAt}(t_i) \notin \mathcal{S}(\text{deformedCode}).
\]
\end{theorem}

\begin{proof}
\leanok
\uses{thm:SpacetimeLogicalFault.flipsOrPreserves, def:SpaceTimeDecoupling.FullOutcomePreserving, def:SpacetimeLogicalFault.FlipsGaugingSign}
From the logical fault hypothesis $\mathrm{hlog}$, we extract $\mathrm{hnotpres} := \mathrm{hlog}.2$, which states that the full outcome is \emph{not} preserved. Unfolding the definition of $\operatorname{FullOutcomePreserving}$ and pushing the negation inward, we obtain that either the sign is not preserved or the Pauli error is not in the stabilizer group. By the dichotomy $\operatorname{flipsOrPreserves}(\mathrm{proc}, F)$, the fault either flips or preserves the gauging sign. In the first case, we conclude $\operatorname{FlipsGaugingSign}(\mathrm{proc}, F)$ directly. In the second case, since the sign is preserved, the negated conjunction forces $F.\operatorname{pauliErrorAt}(t_i) \notin \mathcal{S}$.
\end{proof}

\begin{definition}[Space Logical Fault]
\label{def:SpaceTimeDecoupling.IsSpaceLogicalFault}
\lean{QEC1.SpaceTimeDecoupling.IsSpaceLogicalFault}
\leanok
\uses{def:SpacetimeFault, def:FaultTolerantGaugingProcedure, def:SpacetimeFault.isPureSpace}
A space-only logical fault: a spacetime fault $F$ such that
\begin{enumerate}
\item $F$ is pure-space (no time-faults),
\item all space-faults are concentrated at time $t_i$: for all $t \neq \mathrm{proc}.\mathrm{phase2Start}$, $F.\operatorname{spaceFaultsAt}(t) = \emptyset$,
\item the composite Pauli error at $t_i$ is a nontrivial logical of the deformed code: $\operatorname{isLogicalOp}(\text{deformedCode}, F.\operatorname{pauliErrorAt}(t_i))$.
\end{enumerate}
\end{definition}

\begin{definition}[Time Logical Fault]
\label{def:SpaceTimeDecoupling.IsTimeLogicalFault}
\lean{QEC1.SpaceTimeDecoupling.IsTimeLogicalFault}
\leanok
\uses{def:SpacetimeFault, def:FaultTolerantGaugingProcedure, def:SpacetimeLogicalFault.IsSyndromeFreeGauging, def:SpacetimeLogicalFault.FlipsGaugingSign}
A time-only logical fault: a spacetime fault $F$ such that
\begin{enumerate}
\item $F$ is pure-time (no space-faults),
\item $F$ is syndrome-free in the gauging procedure,
\item $F$ flips the gauging sign $\sigma$.
\end{enumerate}
\end{definition}

\begin{theorem}[$\mathbb{Z}_2$-Additivity of gaussSignFlip Under Composition]
\label{thm:SpaceTimeDecoupling.gaussSignFlip_compose_additive}
\lean{QEC1.SpaceTimeDecoupling.gaussSignFlip_compose_additive}
\leanok
\uses{def:SpacetimeLogicalFault.gaussSignFlip, def:SpacetimeFault, def:FaultTolerantGaugingProcedure}
The $\operatorname{gaussSignFlip}$ is $\mathbb{Z}_2$-additive under composition:
\[
\operatorname{gaussSignFlip}(\mathrm{proc}, F_1 \cdot F_2) = \operatorname{gaussSignFlip}(\mathrm{proc}, F_1) + \operatorname{gaussSignFlip}(\mathrm{proc}, F_2)
\]
in $\mathbb{Z}/2\mathbb{Z}$. This follows from the fact that the time-faults of the composition are the symmetric difference, and the indicator function is additive over $\mathbb{Z}_2$.
\end{theorem}

\begin{proof}
\leanok
\uses{def:SpacetimeLogicalFault.gaussSignFlip, def:SpacetimeFault.compose}
Expanding the definition of $\operatorname{gaussSignFlip}$ and of $\operatorname{compose}$ (which takes the symmetric difference of time-faults), we rewrite the sum using $\sum$-distributivity over addition. For each vertex $v$, we apply the $\mathbb{Z}_2$ indicator additivity lemma for symmetric differences, which states that $\sum_{b \in S \mathbin{\triangle} T} \mathbf{1}[g(b)] = \sum_{b \in S} \mathbf{1}[g(b)] + \sum_{b \in T} \mathbf{1}[g(b)]$ in $\mathbb{Z}/2\mathbb{Z}$. This gives the result by congruence.
\end{proof}

\begin{theorem}[Multiplicativity of Pauli Error Under Composition]
\label{thm:SpaceTimeDecoupling.pauliErrorAt_compose_mul}
\lean{QEC1.SpaceTimeDecoupling.pauliErrorAt_compose_mul}
\leanok
\uses{def:SpacetimeFault, def:SpaceFault, def:FaultTolerantGaugingProcedure}
The net Pauli error at time $t$ is multiplicative under composition:
\[
(F_1 \cdot F_2).\operatorname{pauliErrorAt}(t) = F_1.\operatorname{pauliErrorAt}(t) \cdot F_2.\operatorname{pauliErrorAt}(t).
\]
This follows because the space-faults of the composition are the symmetric difference, the filter distributes over symmetric difference, and $\mathbb{Z}_2$ sums are additive (corresponding to the Pauli product).
\end{theorem}

\begin{proof}
\leanok
\uses{def:SpaceFault, def:SpacetimeFault.compose}
By extensionality, we verify both the $x$-vector and $z$-vector components separately. For the $x$-vector component: expanding the definitions of $\operatorname{pauliErrorAt}$, $\operatorname{spaceFaultsAt}$, and $\operatorname{compose}$, and using the fact that filtering distributes over symmetric difference ($\operatorname{filter}_p(S \mathbin{\triangle} T) = \operatorname{filter}_p(S) \mathbin{\triangle} \operatorname{filter}_p(T)$), we apply the $\mathbb{Z}_2$ sum additivity over symmetric differences with the function $f \mapsto \text{if } f.\mathrm{qubit} = q \text{ then } f.\mathrm{xComponent} \text{ else } 0$. The $z$-vector component follows identically with $\mathrm{zComponent}$ replacing $\mathrm{xComponent}$.
\end{proof}

\begin{theorem}[Composition Preserves Syndrome-Freeness]
\label{thm:SpaceTimeDecoupling.compose_syndromeFree_syndromeFree}
\lean{QEC1.SpaceTimeDecoupling.compose_syndromeFree_syndromeFree}
\leanok
\uses{def:SpacetimeLogicalFault.IsSyndromeFreeGauging, def:SpacetimeFault, def:FaultTolerantGaugingProcedure, def:Detector.flipParity}
Composing two syndrome-free faults preserves syndrome-freeness. If both $F$ and $S$ are syndrome-free in the gauging procedure, then $F \cdot S$ is also syndrome-free.
\end{theorem}

\begin{proof}
\leanok
\uses{def:SpacetimeLogicalFault.IsSyndromeFreeGauging, def:Detector.flipParity}
Let $\mathrm{idx}$ be an arbitrary detector index. We need to show that the detector is not violated by $(F \cdot S).\mathrm{timeFaults} = F.\mathrm{timeFaults} \mathbin{\triangle} S.\mathrm{timeFaults}$. By the characterization of violation via $\operatorname{flipParity}$, we rewrite the goal using the $\mathbb{Z}_2$-additivity of $\operatorname{flipParity}$ over symmetric differences. Since $F$ is syndrome-free, $\operatorname{flipParity}(D, F.\mathrm{timeFaults}) \neq 1$, and since every element of $\mathbb{Z}/2\mathbb{Z}$ that is not $1$ equals $0$, we get $\operatorname{flipParity}(D, F.\mathrm{timeFaults}) = 0$. Similarly $\operatorname{flipParity}(D, S.\mathrm{timeFaults}) = 0$. Therefore $\operatorname{flipParity}(D, F.\mathrm{timeFaults} \mathbin{\triangle} S.\mathrm{timeFaults}) = 0 + 0 = 0 \neq 1$.
\end{proof}

\begin{theorem}[Composition With Stabilizer Preserves Syndrome-Freeness]
\label{thm:SpaceTimeDecoupling.compose_preserves_syndromeFree}
\lean{QEC1.SpaceTimeDecoupling.compose_preserves_syndromeFree}
\leanok
\uses{def:SpaceTimeDecoupling.IsFullGaugingStabilizer, def:SpacetimeLogicalFault.IsSyndromeFreeGauging, def:SpacetimeFault, def:FaultTolerantGaugingProcedure}
Composing a syndrome-free fault $F$ with a full gauging stabilizer $S$ preserves syndrome-freeness.
\end{theorem}

\begin{proof}
\leanok
\uses{thm:SpaceTimeDecoupling.compose_syndromeFree_syndromeFree}
This follows directly from the fact that composing two syndrome-free faults preserves syndrome-freeness, since a full gauging stabilizer is in particular syndrome-free.
\end{proof}

\begin{theorem}[Composition of Full Gauging Stabilizers]
\label{thm:SpaceTimeDecoupling.compose_fullGaugingStabilizer}
\lean{QEC1.SpaceTimeDecoupling.compose_fullGaugingStabilizer}
\leanok
\uses{def:SpaceTimeDecoupling.IsFullGaugingStabilizer, def:SpacetimeLogicalFault.PreservesGaugingSign, def:SpacetimeFault, def:FaultTolerantGaugingProcedure}
The composition of two full gauging stabilizers is a full gauging stabilizer. Syndrome-freeness and outcome preservation are both additive over $\mathbb{Z}_2$.
\end{theorem}

\begin{proof}
\leanok
\uses{thm:SpaceTimeDecoupling.compose_preserves_syndromeFree, thm:SpaceTimeDecoupling.gaussSignFlip_compose_additive, thm:SpaceTimeDecoupling.pauliErrorAt_compose_mul}
We verify each condition:
\begin{enumerate}
\item \textbf{Syndrome-free:} This follows from $\operatorname{compose\_preserves\_syndromeFree}$, since $F_1$ is syndrome-free and $F_2$ is a full gauging stabilizer.
\item \textbf{Preserves sign:} By $\mathbb{Z}_2$-additivity, $\operatorname{gaussSignFlip}(\mathrm{proc}, F_1 \cdot F_2) = \operatorname{gaussSignFlip}(\mathrm{proc}, F_1) + \operatorname{gaussSignFlip}(\mathrm{proc}, F_2) = 0 + 0 = 0$.
\item \textbf{Pauli in stabilizer group:} By multiplicativity of $\operatorname{pauliErrorAt}$, the Pauli error of the composition is the product of two stabilizer group elements, which is in the stabilizer group by closure under multiplication.
\end{enumerate}
\end{proof}

\begin{theorem}[Empty Fault is a Full Gauging Stabilizer]
\label{thm:SpaceTimeDecoupling.empty_isFullGaugingStabilizer}
\lean{QEC1.SpaceTimeDecoupling.empty_isFullGaugingStabilizer}
\leanok
\uses{def:SpaceTimeDecoupling.IsFullGaugingStabilizer, def:SpacetimeFault, def:FaultTolerantGaugingProcedure}
The empty spacetime fault (no space-faults, no time-faults) is a full gauging stabilizer.
\end{theorem}

\begin{proof}
\leanok
\uses{def:SpacetimeFault.empty, def:SpacetimeLogicalFault.gaussSignFlip, def:SpacetimeLogicalFault.PreservesGaugingSign}
We verify each condition:
\begin{enumerate}
\item \textbf{Syndrome-free:} With no time-faults, no detector can be violated, so this follows from the fact that a detector with an empty fault set is not violated.
\item \textbf{Preserves sign:} With no time-faults, the $\operatorname{gaussSignFlip}$ reduces to a sum over the empty set, which is $0$.
\item \textbf{Pauli error is $\mathbf{1}$:} The Pauli error at any time of the empty fault is the identity $1$, which is in the stabilizer group.
\end{enumerate}
\end{proof}

\begin{theorem}[Boundary Faults Trivially Absorbed]
\label{thm:SpaceTimeDecoupling.boundary_faults_trivially_absorbed}
\lean{QEC1.SpaceTimeDecoupling.boundary_faults_trivially_absorbed}
\leanok
\uses{def:SpaceTimeDecoupling.IsFullGaugingStabilizer, def:SpacetimeLogicalFault.IsSyndromeFreeGauging, def:SpacetimeFault, def:FaultTolerantGaugingProcedure}
If a syndrome-free fault $F'$ already has its space-faults concentrated at time $t_i$ (i.e., $F'.\operatorname{spaceFaultsAt}(t) = \emptyset$ for all $t \neq t_i$), then there exists a full gauging stabilizer $S_2$ such that $(F' \cdot S_2).\operatorname{spaceFaultsAt}(t) = \emptyset$ for all $t \neq t_i$. In fact, the empty fault serves as such a stabilizer.
\end{theorem}

\begin{proof}
\leanok
\uses{thm:SpaceTimeDecoupling.empty_isFullGaugingStabilizer}
We take $S_2 = \operatorname{empty}$, the empty spacetime fault. By the previous theorem, $\operatorname{empty}$ is a full gauging stabilizer. For the concentration property, let $t \neq t_i$. Then $(F' \cdot \operatorname{empty}).\operatorname{spaceFaultsAt}(t) = F'.\operatorname{spaceFaultsAt}(t) = \emptyset$ by the hypothesis, using the fact that composing with the empty fault on the right is the identity.
\end{proof}

\begin{theorem}[Space-Fault Cleaning with Full Stabilizer]
\label{thm:SpaceTimeDecoupling.space_fault_cleaning_fullStabilizer}
\lean{QEC1.SpaceTimeDecoupling.space_fault_cleaning_fullStabilizer}
\leanok
\uses{def:SpaceTimeDecoupling.IsFullGaugingStabilizer, def:SpacetimeLogicalFault.IsSyndromeFreeGauging, def:SpacetimeFault, def:FaultTolerantGaugingProcedure}
Any syndrome-free spacetime fault $F$ can be composed with a full gauging stabilizer $S_1$ to concentrate all space-faults at $t_i$. That is, there exists $S_1$ with $\operatorname{IsFullGaugingStabilizer}(\mathrm{proc}, S_1)$ such that for all $t \neq \mathrm{proc}.\mathrm{phase2Start}$, $(F \cdot S_1).\operatorname{spaceFaultsAt}(t) = \emptyset$.
This uses the time-propagating and boundary generators from Lem~5.
\end{theorem}

\begin{proof}
\leanok
\uses{def:SpaceTimeDecoupling.IsFullGaugingStabilizer, thm:SpacetimeStabilizers.space_fault_cleaning}
From the $\operatorname{space\_fault\_cleaning}$ axiom (Lem~5), applied to $F$ and the syndrome-freeness hypothesis, we obtain $S_1$ together with witnesses that $S_1$ is syndrome-free, preserves the gauging sign, has its Pauli error in the stabilizer group, and cleans the space-faults. Packaging the first three properties gives $\operatorname{IsFullGaugingStabilizer}(\mathrm{proc}, S_1)$, and the fourth gives the concentration property.
\end{proof}

\begin{theorem}[Centralizer Membership of Cleaned Pure-Space Fault]
\label{thm:SpaceTimeDecoupling.centralizer_of_syndromeFree_pureSpace}
\lean{QEC1.SpaceTimeDecoupling.centralizer_of_syndromeFree_pureSpace}
\leanok
\uses{def:SpacetimeLogicalFault.IsSyndromeFreeGauging, def:SpacetimeFault, def:FaultTolerantGaugingProcedure}
A syndrome-free pure-space fault concentrated at $t_i$ has its Pauli error in the centralizer of the deformed code. Formally, if $F$ is syndrome-free, $F.\operatorname{spaceFaultsAt}(t) = \emptyset$ for all $t \neq t_i$, and $F$ is pure-space, then $\operatorname{inCentralizer}(\text{deformedCode}, F.\operatorname{pauliErrorAt}(t_i))$.
\end{theorem}

\begin{proof}
\leanok
\uses{def:SpacetimeLogicalFault.IsSyndromeFreeGauging, thm:SpacetimeStabilizers.syndromeFree_pureSpace_inCentralizer}
This follows directly from the $\operatorname{syndromeFree\_pureSpace\_inCentralizer}$ result from Lem~5, which encodes the quantum-mechanical fact that the deformed code checks are the active checks at $t_i$ and the cleaning process preserves commutation.
\end{proof}

\begin{theorem}[Time-Faults Dichotomy]
\label{thm:SpaceTimeDecoupling.time_faults_dichotomy}
\lean{QEC1.SpaceTimeDecoupling.time_faults_dichotomy}
\leanok
\uses{def:TimeFaultDistance.IsPureTimeFault, def:SpacetimeLogicalFault.IsSyndromeFreeGauging, def:SpacetimeLogicalFault.FlipsGaugingSign, def:SpacetimeLogicalFault.IsGaugingStabilizer, def:SpacetimeLogicalFault.PreservesGaugingSign, thm:TimeFaultDistance.pureTime_syndromeFree_logical_or_stabilizer_generators}
The cleaned time-faults of a syndrome-free pure-time fault $F_T$ either flip the gauging sign (i.e., $F_T$ is a time-only logical fault corresponding to an $A_v$ measurement string) or decompose into stabilizer generators (i.e., $F_T$ is a gauging stabilizer).
\end{theorem}

\begin{proof}
\leanok
\uses{thm:TimeFaultDistance.pureTime_syndromeFree_logical_or_stabilizer_generators}
By $\operatorname{pureTime\_syndromeFree\_logical\_or\_stabilizer\_generators}$ from Lem~6, applied to $F_T$, $\mathrm{hpure}$, and $\mathrm{hfree}$, we obtain a disjunction. In the first case, $F_T$ flips the gauging sign directly. In the second case, we extract that $F_T$ preserves the sign and combine with syndrome-freeness to form the gauging stabilizer witness.
\end{proof}

\begin{theorem}[gaussSignFlip Depends Only on Time-Faults]
\label{thm:SpaceTimeDecoupling.gaussSignFlip_depends_only_on_timeFaults}
\lean{QEC1.SpaceTimeDecoupling.gaussSignFlip_depends_only_on_timeFaults}
\leanok
\uses{def:SpacetimeLogicalFault.gaussSignFlip, def:SpacetimeFault, def:FaultTolerantGaugingProcedure}
The gauging sign depends only on time-faults (measurement errors on Gauss checks). If two spacetime faults $F_1$ and $F_2$ have equal time-faults ($F_1.\mathrm{timeFaults} = F_2.\mathrm{timeFaults}$), then $\operatorname{gaussSignFlip}(\mathrm{proc}, F_1) = \operatorname{gaussSignFlip}(\mathrm{proc}, F_2)$.
\end{theorem}

\begin{proof}
\leanok
\uses{def:SpacetimeLogicalFault.gaussSignFlip}
Expanding the definition of $\operatorname{gaussSignFlip}$ as a double sum over vertices $v$ and rounds $r$, the result follows by congruence: for each $v$ and $r$, the summand depends only on $F.\mathrm{timeFaults}$, which is equal for $F_1$ and $F_2$ by hypothesis.
\end{proof}

\begin{theorem}[Pure-Space Fault Preserves Gauging Sign]
\label{thm:SpaceTimeDecoupling.pureSpace_preservesSign}
\lean{QEC1.SpaceTimeDecoupling.pureSpace_preservesSign}
\leanok
\uses{def:SpacetimeLogicalFault.PreservesGaugingSign, def:SpacetimeLogicalFault.gaussSignFlip, def:SpacetimeFault}
A pure-space fault preserves the gauging sign: if $F$ has no time-faults ($F.\mathrm{timeFaults} = \emptyset$), then $\operatorname{gaussSignFlip}(\mathrm{proc}, F) = 0$.
\end{theorem}

\begin{proof}
\leanok
\uses{def:SpacetimeLogicalFault.gaussSignFlip, def:SpacetimeLogicalFault.PreservesGaugingSign}
Unfolding $\operatorname{PreservesGaugingSign}$ and $\operatorname{gaussSignFlip}$, and using $F.\mathrm{isPureSpace}$ to replace $F.\mathrm{timeFaults}$ with $\emptyset$, every indicator $\mathbf{1}[f \in \emptyset]$ evaluates to $0$. The entire double sum therefore equals $0$.
\end{proof}

\begin{theorem}[Pauli Error Depends Only on Space-Faults]
\label{thm:SpaceTimeDecoupling.pauliErrorAt_depends_only_on_spaceFaults}
\lean{QEC1.SpaceTimeDecoupling.pauliErrorAt_depends_only_on_spaceFaults}
\leanok
\uses{def:SpacetimeFault, def:SpaceFault}
The Pauli error at any time $t$ depends only on space-faults. If two spacetime faults $F_1$ and $F_2$ have equal space-faults ($F_1.\mathrm{spaceFaults} = F_2.\mathrm{spaceFaults}$), then $F_1.\operatorname{pauliErrorAt}(t) = F_2.\operatorname{pauliErrorAt}(t)$ for all $t$.
\end{theorem}

\begin{proof}
\leanok
\uses{def:SpacetimeFault}
By extensionality, we verify both the $x$-vector and $z$-vector components. Each component of $\operatorname{pauliErrorAt}$ is defined as a sum over $\operatorname{spaceFaultsAt}(t)$, which is a filter of $\mathrm{spaceFaults}$. Since $F_1.\mathrm{spaceFaults} = F_2.\mathrm{spaceFaults}$, simplification gives the result in both components.
\end{proof}

\begin{theorem}[Pure-Time Fault Has Trivial Pauli Error]
\label{thm:SpaceTimeDecoupling.pureTime_pauliError_trivial}
\lean{QEC1.SpaceTimeDecoupling.pureTime_pauliError_trivial}
\leanok
\uses{def:SpacetimeFault}
A pure-time fault does not change the Pauli error at any time: if $F$ has no space-faults ($F.\mathrm{spaceFaults} = \emptyset$), then $F.\operatorname{pauliErrorAt}(t) = 1$ for all $t$.
\end{theorem}

\begin{proof}
\leanok
\uses{def:SpacetimeFault}
Unfolding $F.\mathrm{isPureTime}$ gives $F.\mathrm{spaceFaults} = \emptyset$. For both the $x$-vector and $z$-vector components, $\operatorname{spaceFaultsAt}(t)$ is a filter of the empty set, hence empty, and the sum over the empty set is $0$, which matches the identity Pauli operator's components.
\end{proof}

\begin{theorem}[Independence of Space and Time Effects]
\label{thm:SpaceTimeDecoupling.space_time_independent_effects}
\lean{QEC1.SpaceTimeDecoupling.space_time_independent_effects}
\leanok
\uses{def:SpacetimeLogicalFault.gaussSignFlip, def:SpacetimeFault, def:FaultTolerantGaugingProcedure}
The space and time components of a fault affect different aspects of the procedure outcome independently:
\begin{enumerate}
\item The gauging sign $\sigma$ depends only on time-faults: $\operatorname{gaussSignFlip}(\mathrm{proc}, F) = \operatorname{gaussSignFlip}(\mathrm{proc}, F.\mathrm{timeComponent})$.
\item The Pauli error on the code state depends only on space-faults: for all $t$, $F.\operatorname{pauliErrorAt}(t) = F.\mathrm{spaceComponent}.\operatorname{pauliErrorAt}(t)$.
\end{enumerate}
\end{theorem}

\begin{proof}
\leanok
\uses{thm:SpaceTimeDecoupling.gaussSignFlip_depends_only_on_timeFaults, thm:SpaceTimeDecoupling.pauliErrorAt_depends_only_on_spaceFaults}
We prove both parts. For part (1), we apply $\operatorname{gaussSignFlip\_depends\_only\_on\_timeFaults}$, noting that $F.\mathrm{timeComponent}.\mathrm{timeFaults} = F.\mathrm{timeFaults}$ by definition of the time component. For part (2), for each $t$, we apply $\operatorname{pauliErrorAt\_depends\_only\_on\_spaceFaults}$, noting that $F.\mathrm{spaceComponent}.\mathrm{spaceFaults} = F.\mathrm{spaceFaults}$ by definition of the space component.
\end{proof}

\begin{theorem}[gaussSignFlip of Composition With Pure-Space Fault]
\label{thm:SpaceTimeDecoupling.gaussSignFlip_compose_pureSpace}
\lean{QEC1.SpaceTimeDecoupling.gaussSignFlip_compose_pureSpace}
\leanok
\uses{def:SpacetimeLogicalFault.gaussSignFlip, def:SpacetimeFault, def:FaultTolerantGaugingProcedure}
The sign flip of $F_S \cdot F_T$ equals the sign flip of $F_T$ when $F_S$ is pure-space:
$\operatorname{gaussSignFlip}(\mathrm{proc}, F_S \cdot F_T) = \operatorname{gaussSignFlip}(\mathrm{proc}, F_T)$.
\end{theorem}

\begin{proof}
\leanok
\uses{thm:SpaceTimeDecoupling.gaussSignFlip_depends_only_on_timeFaults}
We apply $\operatorname{gaussSignFlip\_depends\_only\_on\_timeFaults}$. Since $F_S$ is pure-space, $F_S.\mathrm{timeFaults} = \emptyset$, so $(F_S \cdot F_T).\mathrm{timeFaults} = \emptyset \mathbin{\triangle} F_T.\mathrm{timeFaults} = F_T.\mathrm{timeFaults}$.
\end{proof}

\begin{theorem}[Main Theorem: Space-Time Decomposition (Lemma 7)]
\label{thm:SpaceTimeDecoupling.spaceTime_decomposition}
\lean{QEC1.SpaceTimeDecoupling.spaceTime_decomposition}
\leanok
\uses{def:SpaceTimeDecoupling.IsFullGaugingLogicalFault, def:SpaceTimeDecoupling.IsFullGaugingStabilizer, def:SpaceTimeDecoupling.FullOutcomePreserving, def:SpacetimeLogicalFault.FlipsGaugingSign, def:SpacetimeFault, def:FaultTolerantGaugingProcedure, thm:SpacetimeLogicalFault.flipsOrPreserves}
Any spacetime logical fault $F$ in the fault-tolerant gauging measurement procedure is equivalent, up to multiplication by spacetime stabilizers, to the product of a space-only fault $F_S$ and a time-only fault $F_T$.

Specifically: there exist $F_S$, $F_T$, and $S$ such that:
\begin{itemize}
\item $F = (F_S \cdot F_T) \cdot S$ (composition via symmetric difference),
\item $F_S$ is pure-space (no time-faults) and concentrated at time $t_i$,
\item $F_T$ is pure-time (no space-faults) and syndrome-free,
\item $S$ is a full gauging stabilizer (syndrome-free, outcome-preserving),
\item At least one of $F_S$, $F_T$ is nontrivial: either $\operatorname{FlipsGaugingSign}(\mathrm{proc}, F_T)$ or $\operatorname{isLogicalOp}(\text{deformedCode}, F_S.\operatorname{pauliErrorAt}(t_i))$.
\end{itemize}
\end{theorem}

\begin{proof}
\leanok
\uses{thm:SpaceTimeDecoupling.space_fault_cleaning_fullStabilizer, thm:SpaceTimeDecoupling.compose_preserves_syndromeFree, thm:SpaceTimeDecoupling.boundary_faults_trivially_absorbed, thm:SpaceTimeDecoupling.compose_fullGaugingStabilizer, thm:SpaceTimeDecoupling.gaussSignFlip_compose_additive, thm:SpaceTimeDecoupling.gaussSignFlip_depends_only_on_timeFaults, thm:SpaceTimeDecoupling.pauliErrorAt_compose_mul, thm:SpaceTimeDecoupling.pauliErrorAt_depends_only_on_spaceFaults, thm:SpaceTimeDecoupling.centralizer_of_syndromeFree_pureSpace, thm:SpacetimeLogicalFault.flipsOrPreserves}

\textbf{Step 1:} Clean space-faults to $t_i$ using time-propagating stabilizers (Lem~5). By $\operatorname{space\_fault\_cleaning\_fullStabilizer}$, applied to $F$ and its syndrome-freeness, we obtain a stabilizer $S_1$ with $\operatorname{IsFullGaugingStabilizer}(\mathrm{proc}, S_1)$ and the property that for all $t \neq t_i$, $(F \cdot S_1).\operatorname{spaceFaultsAt}(t) = \emptyset$. Set $F' := F \cdot S_1$. Then $F'$ is syndrome-free by $\operatorname{compose\_preserves\_syndromeFree}$.

\textbf{Step 2:} Absorb boundary faults. By $\operatorname{boundary\_faults\_trivially\_absorbed}$,applied to $F'$, we obtain $S_2$ with $\operatorname{IsFullGaugingStabilizer}(\mathrm{proc}, S_2)$ such that $F'' := F' \cdot S_2$ has space-faults concentrated at $t_i$. Again $F''$ is syndrome-free.

\textbf{Step 3:} Decompose $F''$ into space and time components. Let $F_S := F''.\mathrm{spaceComponent}$, $F_T := F''.\mathrm{timeComponent}$, and $S := S_1 \cdot S_2$. Then $S$ is a full gauging stabilizer by $\operatorname{compose\_fullGaugingStabilizer}$.

We now verify each claimed property:
\begin{itemize}
\item \textbf{$F = (F_S \cdot F_T) \cdot S$:} By the space-time decomposition $F'' = F_S \cdot F_T$, and $F'' = F \cdot S_1 \cdot S_2 = F \cdot S$. So $F = F'' \cdot S$ (since composition is involutive: $F \cdot S \cdot S = F$), hence $F = (F_S \cdot F_T) \cdot S$ by associativity and self-cancellation.
\item \textbf{$F_S$ is pure-space:} By $F''.\operatorname{spaceComponent\_isPureSpace}$.
\item \textbf{$F_S$ concentrated at $t_i$:} The space-faults of $F_S$ equal those of $F''$ (by definition of spaceComponent), which are concentrated at $t_i$.
\item \textbf{$F_T$ is pure-time:} By $F''.\operatorname{timeComponent\_isPureTime}$.
\item \textbf{$F_T$ is syndrome-free:} Since $F_T.\mathrm{timeFaults} = F''.\mathrm{timeFaults}$, detector violations are the same, so syndrome-freeness of $F''$ transfers.
\item \textbf{$S$ is a full gauging stabilizer:} Already established.
\item \textbf{Nontriviality:} Since $F$ is a logical fault, $\neg\operatorname{FullOutcomePreserving}(\mathrm{proc}, F)$. By $\operatorname{flipsOrPreserves}$, either $F$ flips the sign or preserves it.

If $F$ flips the sign, then $\operatorname{gaussSignFlip}(\mathrm{proc}, F_T) = \operatorname{gaussSignFlip}(\mathrm{proc}, F'') = \operatorname{gaussSignFlip}(\mathrm{proc}, F') + \operatorname{gaussSignFlip}(\mathrm{proc}, S_2) = \operatorname{gaussSignFlip}(\mathrm{proc}, F) + 0 + 0 = 1$ by $\mathbb{Z}_2$-additivity and the fact that $S_1, S_2$ preserve the sign.

If $F$ preserves the sign, then since the outcome is not preserved, the Pauli error $F.\operatorname{pauliErrorAt}(t_i) \notin \mathcal{S}$. We show $F_S.\operatorname{pauliErrorAt}(t_i) = F''.\operatorname{pauliErrorAt}(t_i)$ (same space-faults). By multiplicativity, $F''.\operatorname{pauliErrorAt}(t_i) = F.\operatorname{pauliErrorAt}(t_i) \cdot (S_1.\operatorname{pauliErrorAt}(t_i) \cdot S_2.\operatorname{pauliErrorAt}(t_i))$, with $S_1, S_2$ having Pauli errors in $\mathcal{S}$. Since $F_S$ is syndrome-free, concentrated at $t_i$, and pure-space, it lies in the centralizer by $\operatorname{centralizer\_of\_syndromeFree\_pureSpace}$. Moreover, $F_S.\operatorname{pauliErrorAt}(t_i) \notin \mathcal{S}$ (otherwise $F.\operatorname{pauliErrorAt}(t_i) \in \mathcal{S}$ by group closure and the inverse, contradicting the hypothesis). Finally $F_S.\operatorname{pauliErrorAt}(t_i) \neq 1$ (otherwise it would be in $\mathcal{S}$). Combining centralizer membership, non-membership in $\mathcal{S}$, and non-identity gives $\operatorname{isLogicalOp}$.
\end{itemize}
\end{proof}

\begin{theorem}[$A_v$ String is Canonical Time Logical Fault]
\label{thm:SpaceTimeDecoupling.gaussString_is_time_logical}
\lean{QEC1.SpaceTimeDecoupling.gaussString_is_time_logical}
\leanok
\uses{def:SpaceTimeDecoupling.IsTimeLogicalFault, def:TimeFaultDistance.gaussStringFault, def:FaultTolerantGaugingProcedure, thm:TimeFaultDistance.gaussStringFault_syndromeFree}
The $A_v$ string is the canonical time-only logical fault. For any vertex $v$ and odd $d$, $\operatorname{gaussStringFault}(\mathrm{proc}, v)$ satisfies $\operatorname{IsTimeLogicalFault}(\mathrm{proc}, \cdot)$: it is pure-time, syndrome-free, and flips the gauging sign.
\end{theorem}

\begin{proof}
\leanok
\uses{def:TimeFaultDistance.gaussStringFault, thm:TimeFaultDistance.gaussStringFault_syndromeFree}
The three conditions are established directly:
\begin{enumerate}
\item Pure-time: by $\operatorname{gaussStringFault\_isPureTime}$.
\item Syndrome-free: by $\operatorname{gaussStringFault\_syndromeFree}$.
\item Flips the gauging sign when $d$ is odd: by $\operatorname{gaussStringFault\_flipsSign\_of\_odd}$.
\end{enumerate}
\end{proof}

\begin{theorem}[Time-Only Logical Faults Have Weight $\geq d$]
\label{thm:SpaceTimeDecoupling.time_logical_weight_ge_d}
\lean{QEC1.SpaceTimeDecoupling.time_logical_weight_ge_d}
\leanok
\uses{def:SpaceTimeDecoupling.IsTimeLogicalFault, def:FaultTolerantGaugingProcedure, thm:TimeFaultDistance.pureTime_logicalFault_weight_ge_d}
Time-only logical faults have weight at least $d$ (when $d$ is odd). If $F_T$ is a time logical fault and $d$ is odd, then $d \leq \operatorname{weight}(F_T)$.
\end{theorem}

\begin{proof}
\leanok
\uses{thm:TimeFaultDistance.pureTime_logicalFault_weight_ge_d}
This follows directly from $\operatorname{pureTime\_logicalFault\_weight\_ge\_d}$ from Lem~6, applied to $F_T$ with the pure-time, syndrome-free, and sign-flipping hypotheses extracted from $\operatorname{IsTimeLogicalFault}$.
\end{proof}

\begin{theorem}[$A_v$ String Flips Sign]
\label{thm:SpaceTimeDecoupling.gaussString_flipsSign}
\lean{QEC1.SpaceTimeDecoupling.gaussString_flipsSign}
\leanok
\uses{def:SpacetimeLogicalFault.FlipsGaugingSign, def:TimeFaultDistance.gaussStringFault, def:FaultTolerantGaugingProcedure}
For any vertex $v$ and odd $d$, $\operatorname{FlipsGaugingSign}(\mathrm{proc}, \operatorname{gaussStringFault}(\mathrm{proc}, v))$.
\end{theorem}

\begin{proof}
\leanok
\uses{def:TimeFaultDistance.gaussStringFault}
This follows directly from $\operatorname{gaussStringFault\_flipsSign\_of\_odd}$.
\end{proof}

\begin{theorem}[gaussSignFlip of Pure-Space Fault is Zero]
\label{thm:SpaceTimeDecoupling.gaussSignFlip_pureSpace}
\lean{QEC1.SpaceTimeDecoupling.gaussSignFlip_pureSpace}
\leanok
\uses{def:SpacetimeLogicalFault.gaussSignFlip, def:SpacetimeFault}
A pure-space fault has zero sign flip: $\operatorname{gaussSignFlip}(\mathrm{proc}, F) = 0$.
\end{theorem}

\begin{proof}
\leanok
\uses{thm:SpaceTimeDecoupling.pureSpace_preservesSign}
This follows directly from $\operatorname{pureSpace\_preservesSign}$.
\end{proof}

\begin{theorem}[Fault Space-Time Decomposition]
\label{thm:SpaceTimeDecoupling.fault_space_time_decomposition}
\lean{QEC1.SpaceTimeDecoupling.fault_space_time_decomposition}
\leanok
\uses{def:SpacetimeFault}
Any spacetime fault $F$ decomposes into its space and time components:
$F = F.\mathrm{spaceComponent} \cdot F.\mathrm{timeComponent}$.
\end{theorem}

\begin{proof}
\leanok
\uses{def:SpacetimeFault}
This is the symmetric form of $\operatorname{SpacetimeFault.decompose\_space\_time}$.
\end{proof}

\begin{theorem}[Weight Decomposition]
\label{thm:SpaceTimeDecoupling.weight_decomposition}
\lean{QEC1.SpaceTimeDecoupling.weight_decomposition}
\leanok
\uses{def:SpacetimeFault}
The weight of a fault is the sum of its space and time weights:
$\operatorname{weight}(F) = \operatorname{spaceWeight}(F) + \operatorname{timeWeight}(F)$.
\end{theorem}

\begin{proof}
\leanok
\uses{def:SpacetimeFault}
This follows directly from $F.\operatorname{weight\_eq\_space\_plus\_time}$.
\end{proof}

\begin{theorem}[Compose Pure-Space and Pure-Time Preserves Space Component]
\label{thm:SpaceTimeDecoupling.compose_pureSpace_pureTime_spaceComponent}
\lean{QEC1.SpaceTimeDecoupling.compose_pureSpace_pureTime_spaceComponent}
\leanok
\uses{def:SpacetimeFault}
The composition of a pure-space fault $F_S$ and a pure-time fault $F_T$ preserves the space component: $(F_S \cdot F_T).\mathrm{spaceComponent} = F_S$.
\end{theorem}

\begin{proof}
\leanok
\uses{def:SpacetimeFault}
Unfolding $\mathrm{isPureSpace}$ and $\mathrm{isPureTime}$ gives $F_S.\mathrm{timeFaults} = \emptyset$ and $F_T.\mathrm{spaceFaults} = \emptyset$. By extensionality, for the space-faults: $(F_S \cdot F_T).\mathrm{spaceComponent}.\mathrm{spaceFaults} = (F_S.\mathrm{spaceFaults} \mathbin{\triangle} \emptyset) = F_S.\mathrm{spaceFaults}$. For the time-faults: the space component has time-faults from the symmetric difference of the original time-fault sets, but since $F_S.\mathrm{timeFaults} = \emptyset$, the time-faults of the space component are empty.
\end{proof}

\begin{theorem}[Compose Pure-Space and Pure-Time Preserves Time Component]
\label{thm:SpaceTimeDecoupling.compose_pureSpace_pureTime_timeComponent}
\lean{QEC1.SpaceTimeDecoupling.compose_pureSpace_pureTime_timeComponent}
\leanok
\uses{def:SpacetimeFault}
The composition of a pure-space fault $F_S$ and a pure-time fault $F_T$ preserves the time component: $(F_S \cdot F_T).\mathrm{timeComponent} = F_T$.
\end{theorem}

\begin{proof}
\leanok
\uses{def:SpacetimeFault}
Unfolding $\mathrm{isPureSpace}$ and $\mathrm{isPureTime}$ gives $F_S.\mathrm{timeFaults} = \emptyset$ and $F_T.\mathrm{spaceFaults} = \emptyset$. By extensionality, for the space-faults: the time component has space-faults from the composition, but since $F_T.\mathrm{spaceFaults} = \emptyset$, the result is $\emptyset$. For the time-faults: $(F_S \cdot F_T).\mathrm{timeComponent}.\mathrm{timeFaults} = (\emptyset \mathbin{\triangle} F_T.\mathrm{timeFaults}) = F_T.\mathrm{timeFaults}$.
\end{proof}

\begin{theorem}[Weight Additivity for Pure-Space and Pure-Time Composition]
\label{thm:SpaceTimeDecoupling.compose_pureSpace_pureTime_weight}
\lean{QEC1.SpaceTimeDecoupling.compose_pureSpace_pureTime_weight}
\leanok
\uses{def:SpacetimeFault}
When $F_S$ is pure-space and $F_T$ is pure-time, their composition has weight equal to the sum of their weights:
$\operatorname{weight}(F_S \cdot F_T) = \operatorname{weight}(F_S) + \operatorname{weight}(F_T)$.
This holds because their fault sets are disjoint.
\end{theorem}

\begin{proof}
\leanok
\uses{def:SpacetimeFault}
Expanding the definition of weight as $|\mathrm{spaceFaults}| + |\mathrm{timeFaults}|$ and using $F_S.\mathrm{timeFaults} = \emptyset$ and $F_T.\mathrm{spaceFaults} = \emptyset$, the symmetric differences simplify: $|F_S.\mathrm{spaceFaults} \mathbin{\triangle} \emptyset| + |\emptyset \mathbin{\triangle} F_T.\mathrm{timeFaults}| = |F_S.\mathrm{spaceFaults}| + |F_T.\mathrm{timeFaults}|$, using $S \mathbin{\triangle} \bot = S$ and $\bot \mathbin{\triangle} S = S$.
\end{proof}

\begin{theorem}[Sign-Flipping Logical Has $A_v$ String]
\label{thm:SpaceTimeDecoupling.sign_flipping_logical_has_Av_string}
\lean{QEC1.SpaceTimeDecoupling.sign_flipping_logical_has_Av_string}
\leanok
\uses{def:SpacetimeLogicalFault.IsSyndromeFreeGauging, def:SpacetimeLogicalFault.FlipsGaugingSign, def:FTGaugingMeasurement, def:DeformedCodeMeasurement, def:TimeFaultDistance.gaussFaultCount, def:FaultTolerantGaugingProcedure}
If $F$ is a syndrome-free fault that flips the gauging sign and $d$ is odd, then there exists a vertex $v \in V$ such that for all rounds $r \in \operatorname{Fin}(d)$, the time-fault corresponding to the Gauss law measurement of $v$ at round $r$ belongs to $F.\mathrm{timeFaults}$. That is, $F$ contains at least one full $A_v$ measurement string.
\end{theorem}

\begin{proof}
\leanok
\uses{def:TimeFaultDistance.gaussFaultCount, def:FTGaugingMeasurement, def:DeformedCodeMeasurement}
By $\operatorname{sign\_flip\_implies\_odd\_full\_strings}$, the number of vertices $v$ with $\operatorname{gaussFaultCount}(\mathrm{proc}, v, F) = d$ is odd. In particular, the filter of such vertices is nonempty (an odd number cannot be zero). Let $v$ be an element of this nonempty set, so $\operatorname{gaussFaultCount}(\mathrm{proc}, v, F) = d$.

For an arbitrary round $r$, we need to show the corresponding time-fault is in $F.\mathrm{timeFaults}$. Consider round $r_0 = 0$ (which exists since $d > 0$ by oddness). By $\operatorname{syndromeFree\_gauss\_all\_or\_none}$, for vertex $v$, either all Gauss measurements at all rounds are in $F.\mathrm{timeFaults}$, or none are. If the round-$0$ measurement were absent, then by $\operatorname{gaussFaultCount\_zero\_or\_d}$, the count would be either $0$ or $d$. Since round $0$ is absent, the count cannot be $d$ (as the filter would have strictly fewer than $|\operatorname{Fin}(d)|$ elements), so it would be $0$, contradicting $\operatorname{gaussFaultCount} = d > 0$. Therefore the round-$0$ measurement is present, and by the all-or-none property, all rounds including $r$ are present.
\end{proof}

\begin{theorem}[Space-Time Decoupling Summary]
\label{thm:SpaceTimeDecoupling.spaceTime_decoupling_summary}
\lean{QEC1.SpaceTimeDecoupling.spaceTime_decoupling_summary}
\leanok
\uses{def:SpaceTimeDecoupling.IsFullGaugingLogicalFault, def:SpaceTimeDecoupling.IsFullGaugingStabilizer, def:SpacetimeLogicalFault.gaussSignFlip, def:SpacetimeFault, def:FaultTolerantGaugingProcedure, def:SpacetimeLogicalFault.IsSyndromeFreeGauging}
For any spacetime logical fault $F$ of the fault-tolerant gauging measurement procedure:
\begin{enumerate}
\item $F$ decomposes as $F = (F_S \cdot F_T) \cdot S$ where $F_S$ is pure-space and concentrated at $t_i$, $F_T$ is pure-time and syndrome-free, and $S$ is a full gauging stabilizer.
\item The gauging sign $\sigma$ is determined entirely by $F_T$: $\operatorname{gaussSignFlip}(F) = \operatorname{gaussSignFlip}(F.\mathrm{timeComponent})$.
\item The Pauli error on the code state is determined entirely by $F_S$: $\operatorname{pauliErrorAt}(F, t) = \operatorname{pauliErrorAt}(F.\mathrm{spaceComponent}, t)$ for all $t$.
\end{enumerate}
\end{theorem}

\begin{proof}
\leanok
\uses{thm:SpaceTimeDecoupling.spaceTime_decomposition, thm:SpaceTimeDecoupling.space_time_independent_effects}
We proveeach part separately.
\begin{enumerate}
\item The decomposition is obtained from $\operatorname{spaceTime\_decomposition}$, extracting $F_S$, $F_T$, $S$ and their properties (dropping the nontriviality clause which is not needed here).
\item The sign independence follows from the first component of $\operatorname{space\_time\_independent\_effects}(\mathrm{proc}, F)$.
\item The Pauli error independence follows from the second component of $\operatorname{space\_time\_independent\_effects}(\mathrm{proc}, F)$.
\end{enumerate}
\end{proof}

\begin{theorem}[Decoupling Weight Bound]
\label{thm:SpaceTimeDecoupling.decoupling_weight_bound}
\lean{QEC1.SpaceTimeDecoupling.decoupling_weight_bound}
\leanok
\uses{def:SpacetimeFault}
For any pair of pure-space and pure-time faults, their composition has weight at most the sum of their weights:
$\operatorname{weight}(F_S \cdot F_T) \leq \operatorname{weight}(F_S) + \operatorname{weight}(F_T)$.
In fact, equality holds.
\end{theorem}

\begin{proof}
\leanok
\uses{thm:SpaceTimeDecoupling.compose_pureSpace_pureTime_weight}
By $\operatorname{compose\_pureSpace\_pureTime\_weight}$, $\operatorname{weight}(F_S \cdot F_T) = \operatorname{weight}(F_S) + \operatorname{weight}(F_T)$, and rewriting gives the inequality (which is in fact an equality).
\end{proof}

%--- Thm_2: FaultTolerantGaugingDistance ---

I'll start by reading the Lean file to understand its contents.Now I have a thorough understanding of the file. Let me produce the LaTeX translation.

\chapter{Thm 2: Fault-Tolerant Gauging Distance}

The spacetime fault-distance of the fault-tolerant gauging measurement procedure equals $d$, the distance of the original $[\![n,k,d]\!]$ stabilizer code, provided that:
\begin{enumerate}
\item The graph $G$ has Cheeger constant $h(G) \geq 1$, and
\item The number of deformed code rounds satisfies $t_o - t_i \geq d$.
\end{enumerate}
Formally: $d_{\text{spacetime}} = d$.

\section{Space-Fault Witness Construction}

The upper bound uses a minimum-weight logical operator $L_0$ of the original code, placed as a space-fault at time $t_i$. For each qubit $q$ in the support of an operator $P$, we create a space fault at qubit $q$, time $t_i$, with the appropriate $X$ and $Z$ components from $P$. The resulting spacetime fault has no time-faults (pure-space), so it is syndrome-free.

\begin{definition}[Space Fault from Qubit]
\label{def:FaultTolerantGaugingDistance.mkSpaceFault}
\lean{QEC1.FaultTolerantGaugingDistance.mkSpaceFault}
\leanok
\uses{def:SpacetimeFault, def:PauliOp, def:PauliOp.support}
Given a Pauli operator $P$ on extended qubits, a time $t$, and a qubit $q \in \operatorname{support}(P)$, we define the space fault
\[
\operatorname{mkSpaceFault}(P, t, q) := (q,\; t,\; P.\operatorname{xVec}(q),\; P.\operatorname{zVec}(q)).
\]
\end{definition}

\begin{definition}[Space Faults of a Pauli Operator]
\label{def:FaultTolerantGaugingDistance.spaceFaultsOfPauliOp}
\lean{QEC1.FaultTolerantGaugingDistance.spaceFaultsOfPauliOp}
\leanok
\uses{def:FaultTolerantGaugingDistance.mkSpaceFault, def:PauliOp.support}
Given a Pauli operator $P$ and a time $t$, the finset of space-faults corresponding to $P$ at time $t$ is
\[
\operatorname{spaceFaultsOfPauliOp}(P, t) := \{ \operatorname{mkSpaceFault}(P, t, q) \mid q \in \operatorname{support}(P) \}.
\]
\end{definition}

\begin{theorem}[Injectivity of mkSpaceFault]
\label{thm:FaultTolerantGaugingDistance.mkSpaceFault_injective}
\lean{QEC1.FaultTolerantGaugingDistance.mkSpaceFault_injective}
\leanok
\uses{def:FaultTolerantGaugingDistance.mkSpaceFault}
The map from $\operatorname{support}(P)$ to space faults given by $q \mapsto \operatorname{mkSpaceFault}(P, t, q)$ is injective: different qubits produce different space faults.
\end{theorem}

\begin{proof}
\leanok
\uses{def:FaultTolerantGaugingDistance.mkSpaceFault}
Let $(q_1, h_{q_1})$ and $(q_2, h_{q_2})$ be elements of the support with $\operatorname{mkSpaceFault}(P, t, q_1) = \operatorname{mkSpaceFault}(P, t, q_2)$. By simplification using the definition of $\operatorname{mkSpaceFault}$ and the injectivity equation for the space fault constructor, we extract that $q_1 = q_2$, hence the subtypes are equal by extensionality.
\end{proof}

\begin{theorem}[Cardinality of Space Faults]
\label{thm:FaultTolerantGaugingDistance.spaceFaultsOfPauliOp_card}
\lean{QEC1.FaultTolerantGaugingDistance.spaceFaultsOfPauliOp_card}
\leanok
\uses{def:FaultTolerantGaugingDistance.spaceFaultsOfPauliOp, thm:FaultTolerantGaugingDistance.mkSpaceFault_injective}
The space-fault finset has the same cardinality as the support of $P$:
\[
|\operatorname{spaceFaultsOfPauliOp}(P, t)| = |\operatorname{support}(P)|.
\]
\end{theorem}

\begin{proof}
\leanok
\uses{thm:FaultTolerantGaugingDistance.mkSpaceFault_injective}
Unfolding the definition of $\operatorname{spaceFaultsOfPauliOp}$, we rewrite using the fact that the image of an injective function preserves cardinality (applying $\operatorname{mkSpaceFault\_injective}$), and then the cardinality of the attached finset equals the cardinality of the support.
\end{proof}

\begin{definition}[Spacetime Fault of Deformed Logical]
\label{def:FaultTolerantGaugingDistance.spacetimeFaultOfDeformedLogical}
\lean{QEC1.FaultTolerantGaugingDistance.spacetimeFaultOfDeformedLogical}
\leanok
\uses{def:FaultTolerantGaugingDistance.spaceFaultsOfPauliOp, def:SpaceTimeDecoupling.theDeformedCode, def:SpacetimeFault, def:StabilizerCode.isLogicalOp, def:FaultTolerantGaugingProcedure}
Given a logical operator $P$ of the deformed code, placing it as a collection of space-faults at time $t_i$ (the start of Phase~2) produces a spacetime fault:
\[
\operatorname{spacetimeFaultOfDeformedLogical}(P) := \bigl(\operatorname{spaceFaultsOfPauliOp}(P, t_i),\; \emptyset\bigr).
\]
This has no time-faults (pure-space).
\end{definition}

\begin{theorem}[Weight of Space-Fault Witness]
\label{thm:FaultTolerantGaugingDistance.spacetimeFaultOfDeformedLogical_weight}
\lean{QEC1.FaultTolerantGaugingDistance.spacetimeFaultOfDeformedLogical_weight}
\leanok
\uses{def:FaultTolerantGaugingDistance.spacetimeFaultOfDeformedLogical, def:SpacetimeFault.weight, def:PauliOp.weight, thm:FaultTolerantGaugingDistance.spaceFaultsOfPauliOp_card}
The space-fault witness has the same weight as the Pauli operator:
\[
\operatorname{weight}(\operatorname{spacetimeFaultOfDeformedLogical}(P)) = \operatorname{weight}(P).
\]
\end{theorem}

\begin{proof}
\leanok
\uses{thm:FaultTolerantGaugingDistance.spaceFaultsOfPauliOp_card}
Unfolding the definitions of $\operatorname{spacetimeFaultOfDeformedLogical}$, spacetime fault weight, and Pauli operator weight, then simplifying using the cardinality result $\operatorname{spaceFaultsOfPauliOp\_card}$, we obtain the equality.
\end{proof}

\begin{theorem}[Space-Fault Witness is Pure-Space]
\label{thm:FaultTolerantGaugingDistance.spacetimeFaultOfDeformedLogical_isPureSpace}
\lean{QEC1.FaultTolerantGaugingDistance.spacetimeFaultOfDeformedLogical_isPureSpace}
\leanok
\uses{def:FaultTolerantGaugingDistance.spacetimeFaultOfDeformedLogical, def:SpacetimeFault.isPureSpace}
The space-fault witness is pure-space (has no time-faults).
\end{theorem}

\begin{proof}
\leanok
\uses{def:FaultTolerantGaugingDistance.spacetimeFaultOfDeformedLogical}
Unfolding the definitions of $\operatorname{spacetimeFaultOfDeformedLogical}$ and $\operatorname{isPureSpace}$, the time-faults component is $\emptyset$ by construction, so this holds by reflexivity.
\end{proof}

\begin{theorem}[Space-Fault Witness is Syndrome-Free]
\label{thm:FaultTolerantGaugingDistance.spacetimeFaultOfDeformedLogical_syndromeFree}
\lean{QEC1.FaultTolerantGaugingDistance.spacetimeFaultOfDeformedLogical_syndromeFree}
\leanok
\uses{def:FaultTolerantGaugingDistance.spacetimeFaultOfDeformedLogical, def:SpacetimeLogicalFault.IsSyndromeFreeGauging, def:FaultTolerantGaugingProcedure}
The space-fault witness is syndrome-free for the gauging procedure: every detector checks against an empty set of time-faults, which is never violated.
\end{theorem}

\begin{proof}
\leanok
\uses{def:FaultTolerantGaugingDistance.spacetimeFaultOfDeformedLogical}
Let $\operatorname{idx}$ be an arbitrary detector index. The time-faults of the space-fault witness are $\emptyset$ by definition. Rewriting with this fact, the result follows from the theorem that no detector is violated when there are no faults.
\end{proof}

\begin{theorem}[Pauli Error of Space-Fault Witness]
\label{thm:FaultTolerantGaugingDistance.spacetimeFaultOfDeformedLogical_pauliErrorAt}
\lean{QEC1.FaultTolerantGaugingDistance.spacetimeFaultOfDeformedLogical_pauliErrorAt}
\leanok
\uses{def:FaultTolerantGaugingDistance.spacetimeFaultOfDeformedLogical, def:SpacetimeFault.pauliErrorAt, def:FaultTolerantGaugingDistance.spaceFaultsOfPauliOp}
The Pauli error at time $t_i$ of the space-fault witness equals $P$:
\[
\operatorname{pauliErrorAt}(\operatorname{spacetimeFaultOfDeformedLogical}(P),\; t_i) = P.
\]
\end{theorem}

\begin{proof}
\leanok
\uses{def:FaultTolerantGaugingDistance.spacetimeFaultOfDeformedLogical, def:FaultTolerantGaugingDistance.spaceFaultsOfPauliOp, def:FaultTolerantGaugingDistance.mkSpaceFault}
By extensionality, it suffices to show equality for each qubit $q$ in both the $\operatorname{xVec}$ and $\operatorname{zVec}$ components.

For the $\operatorname{xVec}$ component: we simplify the definitions of $\operatorname{pauliErrorAt}$, $\operatorname{spaceFaultsAt}$, and $\operatorname{spacetimeFaultOfDeformedLogical}$. Since all space faults have time $t_i$, the filter selecting faults at time $t_i$ returns the full set $\operatorname{spaceFaultsOfPauliOp}(P, t_i)$. We then perform case analysis on whether $q \in \operatorname{support}(P)$:
\begin{itemize}
\item If $q \in \operatorname{support}(P)$: there is exactly one fault at qubit $q$ (namely $\operatorname{mkSpaceFault}(P, t_i, q)$). Using $\operatorname{sum\_eq\_single\_of\_mem}$, the sum reduces to the single term, which by definition of $\operatorname{mkSpaceFault}$ gives $P.\operatorname{xVec}(q)$. All other faults $f$ with $f.\operatorname{qubit} \neq q$ contribute zero, as shown by the uniqueness property of the space faults.
\item If $q \notin \operatorname{support}(P)$: then $P.\operatorname{xVec}(q) = 0$ (from the support membership characterization), and every fault $f$ in the finset has $f.\operatorname{qubit} \neq q$ (otherwise $q$ would be in the support), so each term in the sum is zero.
\end{itemize}

The $\operatorname{zVec}$ component follows by a symmetric argument.
\end{proof}

\begin{theorem}[Space-Fault Witness is a Full Logical Fault]
\label{thm:FaultTolerantGaugingDistance.spaceFaultWitness_isFullLogicalFault}
\lean{QEC1.FaultTolerantGaugingDistance.spaceFaultWitness_isFullLogicalFault}
\leanok
\uses{def:FaultTolerantGaugingDistance.spacetimeFaultOfDeformedLogical, def:SpaceTimeDecoupling.IsFullGaugingLogicalFault, thm:FaultTolerantGaugingDistance.spacetimeFaultOfDeformedLogical_syndromeFree, thm:FaultTolerantGaugingDistance.spacetimeFaultOfDeformedLogical_pauliErrorAt}
The space-fault witness is a full gauging logical fault: it is syndrome-free and \emph{not} outcome-preserving. The sign is preserved (pure-space implies $\operatorname{gaussSignFlip} = 0$), but the Pauli error at $t_i$ is $P$, which is not in the stabilizer group (since $P$ is a logical operator).
\end{theorem}

\begin{proof}
\leanok
\uses{thm:FaultTolerantGaugingDistance.spacetimeFaultOfDeformedLogical_syndromeFree, thm:FaultTolerantGaugingDistance.spacetimeFaultOfDeformedLogical_pauliErrorAt}
We verify the two conditions of $\operatorname{IsFullGaugingLogicalFault}$. The syndrome-free condition follows from $\operatorname{spacetimeFaultOfDeformedLogical\_syndromeFree}$. For the second condition, suppose for contradiction that the fault is outcome-preserving, i.e., that $\operatorname{pauliErrorAt}$ at $t_i$ is in the stabilizer group. Rewriting with $\operatorname{spacetimeFaultOfDeformedLogical\_pauliErrorAt}$, this would mean $P$ is in the stabilizer group, contradicting the fact that $P$ is a logical operator (which by definition is not in the stabilizer group).
\end{proof}

\section{Upper Bound: $d_{\text{spacetime}} \leq d$}

\begin{theorem}[Upper Bound on Spacetime Fault-Distance]
\label{thm:FaultTolerantGaugingDistance.spacetimeFaultDistance_le_d}
\lean{QEC1.FaultTolerantGaugingDistance.spacetimeFaultDistance_le_d}
\leanok
\uses{def:SpacetimeFaultDistance.gaugingSpacetimeFaultDistance, def:SpaceTimeDecoupling.FullOutcomePreserving, def:SpaceTimeDecoupling.theDeformedCode, def:OptimalCheegerConstant.liftToExtended, def:StabilizerCode, thm:OptimalCheegerConstant.liftToExtended_isLogical, thm:SpacetimeFaultDistance.gaugingSpacetimeFaultDistance_le, thm:FaultTolerantGaugingDistance.spaceFaultWitness_isFullLogicalFault, thm:FaultTolerantGaugingDistance.spacetimeFaultOfDeformedLogical_weight}
Place a minimum-weight deformed code logical at time $t_i$ as a collection of space-faults. This has weight $d^* \leq d$ and is a full gauging logical fault. Therefore:
\[
d_{\text{spacetime}} \leq d.
\]
\end{theorem}

\begin{proof}
\leanok
\uses{thm:OptimalCheegerConstant.liftToExtended_isLogical, thm:SpacetimeFaultDistance.gaugingSpacetimeFaultDistance_le, thm:FaultTolerantGaugingDistance.spaceFaultWitness_isFullLogicalFault, thm:FaultTolerantGaugingDistance.spacetimeFaultOfDeformedLogical_weight}
We establish that the deformed code equals the constructed deformed stabilizer code by definition. The lift of $P$ (a pure-$X$ logical of the original code) to the extended qubit space is a logical of the deformed code, by $\operatorname{liftToExtended\_isLogical}$.

Let $F := \operatorname{spacetimeFaultOfDeformedLogical}(\operatorname{liftToExtended}(P))$ be the space-fault witness. By $\operatorname{spaceFaultWitness\_isFullLogicalFault}$, $F$ is a full gauging logical fault. Then:
\[
d_{\text{spacetime}} \leq \operatorname{weight}(F) = \operatorname{weight}(\operatorname{liftToExtended}(P)) = \operatorname{weight}(P) = d_{\text{orig}} = d,
\]
where the first inequality uses $\operatorname{gaugingSpacetimeFaultDistance\_le}$, the first equality uses $\operatorname{spacetimeFaultOfDeformedLogical\_weight}$, the second uses $\operatorname{liftToExtended\_weight}$, the third uses the hypothesis $\operatorname{weight}(P) = d_{\text{orig}}$, and the last uses $d_{\text{orig}} = d$.
\end{proof}

\section{Lower Bound: $d_{\text{spacetime}} \geq d$}

\begin{theorem}[Even $d$ Prevents Time Sign Flip]
\label{thm:FaultTolerantGaugingDistance.even_d_no_time_sign_flip}
\lean{QEC1.FaultTolerantGaugingDistance.even_d_no_time_sign_flip}
\leanok
\uses{def:SpacetimeLogicalFault.PreservesGaugingSign, def:SpacetimeLogicalFault.IsSyndromeFreeGauging, def:SpacetimeFault.isPureTime, def:FaultTolerantGaugingProcedure}
When $d$ is even, no syndrome-free pure-time fault can flip the gauging sign $\sigma$. This is because the all-or-none property forces each vertex's $A_v$ fault count to be $0$ or $d$. When $d$ is even, the total $\operatorname{gaussSignFlip} = \sum (0 \text{ or } d) = k \cdot d \equiv 0 \pmod{2}$ for all $k$.
\end{theorem}

\begin{proof}
\leanok
\uses{cor:TimeFaultDistance.gaussFaultCount_dichotomy}
We rewrite $\operatorname{PreservesGaugingSign}$ and express $\operatorname{gaussSignFlip}$ as a sum of parities over vertices. We show this sum is zero by proving each summand is zero. For each vertex $v$, by the dichotomy $\operatorname{gaussFaultCount\_zero\_or\_d}$, the fault count at $v$ is either $0$ or $d$. If it is $0$, the term vanishes by simplification. If it is $d$, we rewrite and use the fact that $d$ is even implies $d \equiv 0 \pmod{2}$, so the natural number cast to $\mathbb{Z}/2\mathbb{Z}$ is zero.
\end{proof}

\begin{theorem}[Centralizer Closed Under Stabilizer Multiplication]
\label{thm:FaultTolerantGaugingDistance.inCentralizer_mul_stabilizerGroup}
\lean{QEC1.FaultTolerantGaugingDistance.inCentralizer_mul_stabilizerGroup}
\leanok
\uses{def:StabilizerCode.inCentralizer, def:StabilizerCode.stabilizerGroup, def:PauliOp.symplecticInner}
If $P$ is in the centralizer of a stabilizer code and $s$ is in the stabilizer group, then $P \cdot s$ is in the centralizer.
\end{theorem}

\begin{proof}
\leanok
\uses{def:StabilizerCode.inCentralizer, def:PauliOp.PauliCommute}
Let $i$ be an arbitrary check index. We need to show $\operatorname{PauliCommute}(\operatorname{check}(i), P \cdot s)$. Rewriting $\operatorname{PauliCommute}$ in terms of the symplectic inner product, we use linearity: $\langle \operatorname{check}(i), P \cdot s \rangle = \langle \operatorname{check}(i), P \rangle + \langle \operatorname{check}(i), s \rangle$. Since $P$ is in the centralizer, the first term is $0$. Since $s$ is in the stabilizer group (hence in the centralizer), the second term is also $0$. Thus the sum is $0 + 0 = 0$.
\end{proof}

\begin{theorem}[Distance Bound for Logical Operators]
\label{thm:FaultTolerantGaugingDistance.distance_le_weight_of_isLogicalOp}
\lean{QEC1.FaultTolerantGaugingDistance.distance_le_weight_of_isLogicalOp}
\leanok
\uses{def:StabilizerCode.distance, def:StabilizerCode.isLogicalOp, def:PauliOp.weight}
For any logical operator $P$ of a stabilizer code, $d \leq \operatorname{weight}(P)$.
\end{theorem}

\begin{proof}
\leanok
\uses{def:StabilizerCode.distance}
Unfolding the definition of distance as $\inf\{ \operatorname{weight}(P) \mid P \text{ is logical} \}$, we apply $\operatorname{Nat.sInf\_le}$ with the witness $\langle P, h_{\text{log}}, \operatorname{rfl} \rangle$.
\end{proof}

\begin{theorem}[Pauli Error Support Subset]
\label{thm:FaultTolerantGaugingDistance.pauliErrorAt_support_subset_image}
\lean{QEC1.FaultTolerantGaugingDistance.pauliErrorAt_support_subset_image}
\leanok
\uses{def:SpacetimeFault.pauliErrorAt, def:SpacetimeFault.spaceFaultsAt, def:PauliOp.support}
The support of $\operatorname{pauliErrorAt}(F, t)$ is a subset of the qubit image of $\operatorname{spaceFaultsAt}(F, t)$:
\[
\operatorname{support}(\operatorname{pauliErrorAt}(F, t)) \subseteq \operatorname{image}(\operatorname{qubit}, \operatorname{spaceFaultsAt}(F, t)).
\]
\end{theorem}

\begin{proof}
\leanok
\uses{def:SpacetimeFault.pauliErrorAt}
Let $q \in \operatorname{support}(\operatorname{pauliErrorAt}(F, t))$, so $\operatorname{xVec}(q) \neq 0$ or $\operatorname{zVec}(q) \neq 0$. Suppose for contradiction that $q$ is not in the image, i.e., no fault $f$ in $\operatorname{spaceFaultsAt}(F, t)$ has $f.\operatorname{qubit} = q$. Then both the $\operatorname{xVec}$ and $\operatorname{zVec}$ sums at $q$ are zero (each summand vanishes because the conditional on $f.\operatorname{qubit} = q$ is false). This contradicts $q$ being in the support.
\end{proof}

\begin{theorem}[Pauli Error Weight Bounded by Space Weight]
\label{thm:FaultTolerantGaugingDistance.pauliErrorAt_weight_le_spaceWeight}
\lean{QEC1.FaultTolerantGaugingDistance.pauliErrorAt_weight_le_spaceWeight}
\leanok
\uses{def:PauliOp.weight, def:SpacetimeFault.spaceWeight, thm:FaultTolerantGaugingDistance.pauliErrorAt_support_subset_image}
The weight of the Pauli error at any time $t$ is at most the total space weight:
\[
\operatorname{weight}(\operatorname{pauliErrorAt}(F, t)) \leq \operatorname{spaceWeight}(F).
\]
\end{theorem}

\begin{proof}
\leanok
\uses{thm:FaultTolerantGaugingDistance.pauliErrorAt_support_subset_image}
Unfolding the definitions of weight and space weight, we compute:
\[
|\operatorname{support}(\operatorname{pauliErrorAt}(F, t))| \leq |(\operatorname{spaceFaultsAt}(F, t)).\operatorname{image}(\operatorname{qubit})| \leq |\operatorname{spaceFaultsAt}(F, t)| \leq |\operatorname{spaceFaults}(F)|,
\]
where the first inequality uses $\operatorname{pauliErrorAt\_support\_subset\_image}$ (subset implies cardinality inequality), the second uses $\operatorname{card\_image\_le}$ (image cardinality $\leq$ domain cardinality), and the third uses $\operatorname{card\_le\_card}$ since $\operatorname{spaceFaultsAt}$ is a filter of $\operatorname{spaceFaults}$.
\end{proof}

\begin{theorem}[Full Logical Fault Weight $\geq d$]
\label{thm:FaultTolerantGaugingDistance.fullLogicalFault_weight_ge_d}
\lean{QEC1.FaultTolerantGaugingDistance.fullLogicalFault_weight_ge_d}
\leanok
\uses{def:SpaceTimeDecoupling.IsFullGaugingLogicalFault, def:SpaceTimeDecoupling.theDeformedCode, def:SpacetimeFaultDistance.gaugingSpacetimeFaultDistance, def:StabilizerCode, thm:SpaceTimeDecoupling.spaceTime_decomposition, thm:SpaceTimeDecoupling.gaussSignFlip_compose_additive, thm:SpaceTimeDecoupling.gaussSignFlip_pureSpace, thm:SpaceTimeDecoupling.pauliErrorAt_compose_mul, thm:SpaceTimeDecoupling.pureTime_pauliError_trivial, thm:SpaceTimeDecoupling.space_time_independent_effects, thm:TimeFaultDistance.pureTime_logicalFault_weight_ge_d, thm:OptimalCheegerConstant.sufficient_expansion_gives_dstar_ge_d, thm:FaultTolerantGaugingDistance.even_d_no_time_sign_flip, thm:FaultTolerantGaugingDistance.inCentralizer_mul_stabilizerGroup, thm:FaultTolerantGaugingDistance.distance_le_weight_of_isLogicalOp, thm:FaultTolerantGaugingDistance.pauliErrorAt_weight_le_spaceWeight}
Any full gauging logical fault $F$ has weight $\geq d$:
\[
d \leq \operatorname{weight}(F).
\]
\end{theorem}

\begin{proof}
\leanok
\uses{thm:SpaceTimeDecoupling.spaceTime_decomposition, thm:SpaceTimeDecoupling.gaussSignFlip_compose_additive, thm:SpaceTimeDecoupling.gaussSignFlip_pureSpace, thm:SpaceTimeDecoupling.pauliErrorAt_compose_mul, thm:SpaceTimeDecoupling.pureTime_pauliError_trivial, thm:SpaceTimeDecoupling.space_time_independent_effects, thm:TimeFaultDistance.pureTime_logicalFault_weight_ge_d, thm:OptimalCheegerConstant.sufficient_expansion_gives_dstar_ge_d, thm:FaultTolerantGaugingDistance.even_d_no_time_sign_flip, thm:FaultTolerantGaugingDistance.inCentralizer_mul_stabilizerGroup, thm:FaultTolerantGaugingDistance.distance_le_weight_of_isLogicalOp, thm:FaultTolerantGaugingDistance.pauliErrorAt_weight_le_spaceWeight}
We decompose $F = F_S \cdot F_T \cdot S$ via the space-time decomposition (Lem~7), where $F_S$ is pure-space, $F_T$ is pure-time and syndrome-free, and $S$ is a full gauging stabilizer. We perform case analysis on whether $F_T$ flips the gauging sign $\sigma$ or $F_S$ is a nontrivial deformed code logical.

\textbf{Case (a): $F_T$ flips $\sigma$.} We first establish that $F$ itself flips the gauging sign. Using the additivity of $\operatorname{gaussSignFlip}$ under composition:
\[
\operatorname{gaussSignFlip}(F) = \operatorname{gaussSignFlip}(F_S \cdot F_T) + \operatorname{gaussSignFlip}(S) = \operatorname{gaussSignFlip}(F_S) + \operatorname{gaussSignFlip}(F_T) + 0 = 0 + 1 + 0 = 1,
\]
where $\operatorname{gaussSignFlip}(F_S) = 0$ by $\operatorname{gaussSignFlip\_pureSpace}$ and $\operatorname{gaussSignFlip}(S) = 0$ since $S$ preserves the sign.

The time component $F^{\text{time}}$ is pure-time, syndrome-free (same time-faults as $F$), and flips the sign (by $\operatorname{space\_time\_independent\_effects}$, the sign flip depends only on time-faults). We then show $d$ must be odd: if $d$ were even, then by $\operatorname{even\_d\_no\_time\_sign\_flip}$, $F_T$ would preserve the sign, contradicting that it flips. Applying $\operatorname{pureTime\_logicalFault\_weight\_ge\_d}$:
\[
d \leq \operatorname{weight}(F^{\text{time}}) = \operatorname{timeWeight}(F) \leq \operatorname{weight}(F).
\]

\textbf{Case (b): $F_S$ yields a nontrivial deformed code logical.} The deformed code distance satisfies $d^* \geq d$ by $\operatorname{sufficient\_expansion\_gives\_dstar\_ge\_d}$ (using $h(G) \geq 1$).

We compute $\operatorname{pauliErrorAt}(F, t_i)$ via the decomposition. Using $\operatorname{pauliErrorAt\_compose\_mul}$:
\[
\operatorname{pauliErrorAt}(F, t_i) = \operatorname{pauliErrorAt}(F_S, t_i) \cdot (\operatorname{pauliErrorAt}(F_T, t_i) \cdot \operatorname{pauliErrorAt}(S, t_i)).
\]
Since $F_T$ is pure-time, $\operatorname{pauliErrorAt}(F_T, t_i) = 1$ by $\operatorname{pureTime\_pauliError\_trivial}$. Thus $\operatorname{pauliErrorAt}(F, t_i) = \operatorname{pauliErrorAt}(F_S, t_i) \cdot \operatorname{pauliErrorAt}(S, t_i)$.

We show $\operatorname{pauliErrorAt}(F, t_i)$ is a logical of the deformed code. It is in the centralizer by $\operatorname{inCentralizer\_mul\_stabilizerGroup}$ (product of centralizer element and stabilizer element). It is not in the stabilizer group: if $P \cdot s \in \operatorname{Stab}$, then $P = (P \cdot s) \cdot s^{-1} \in \operatorname{Stab}$, contradicting that $F_S$'s Pauli error is not a stabilizer. Similarly, it is not the identity: if $P \cdot s = 1$, then $P = s^{-1} \in \operatorname{Stab}$, again a contradiction.

Therefore:
\[
d_{\text{orig}} \leq d^* \leq \operatorname{weight}(\operatorname{pauliErrorAt}(F, t_i)) \leq \operatorname{spaceWeight}(F) \leq \operatorname{weight}(F),
\]
where the inequalities use $\operatorname{sufficient\_expansion\_gives\_dstar\_ge\_d}$, $\operatorname{distance\_le\_weight\_of\_isLogicalOp}$, $\operatorname{pauliErrorAt\_weight\_le\_spaceWeight}$, and $\operatorname{spaceWeight\_le\_weight}$ respectively.
\end{proof}

\section{Main Theorem}

\begin{theorem}[Lower Bound: $d_{\text{spacetime}} \geq d$]
\label{thm:FaultTolerantGaugingDistance.spacetimeFaultDistance_ge_d}
\lean{QEC1.FaultTolerantGaugingDistance.spacetimeFaultDistance_ge_d}
\leanok
\uses{def:SpacetimeFaultDistance.gaugingSpacetimeFaultDistance, def:SpacetimeFaultDistance.gaugingLogicalFaultWeights, def:SpaceTimeDecoupling.FullOutcomePreserving, def:SpaceTimeDecoupling.theDeformedCode, def:OptimalCheegerConstant.liftToExtended, thm:FaultTolerantGaugingDistance.fullLogicalFault_weight_ge_d, thm:FaultTolerantGaugingDistance.spaceFaultWitness_isFullLogicalFault, thm:OptimalCheegerConstant.liftToExtended_isLogical}
The gauging spacetime fault-distance satisfies $d_{\text{spacetime}} \geq d$.
\end{theorem}

\begin{proof}
\leanok
\uses{thm:FaultTolerantGaugingDistance.fullLogicalFault_weight_ge_d, thm:FaultTolerantGaugingDistance.spaceFaultWitness_isFullLogicalFault, thm:OptimalCheegerConstant.liftToExtended_isLogical}
We first verify that the deformed code has logical operators by constructing the lift: $\operatorname{liftToExtended}(P)$ is a logical of the deformed code by $\operatorname{liftToExtended\_isLogical}$. This establishes that the set of gauging logical fault weights is nonempty (the space-fault witness provides a member via $\operatorname{spaceFaultWitness\_isFullLogicalFault}$).

We then apply $\operatorname{le\_csInf}$ to the nonempty set: for every $w$ in the set of logical fault weights, there exists a fault $F$ with $\operatorname{weight}(F) = w$ that is a full gauging logical fault. By $\operatorname{fullLogicalFault\_weight\_ge\_d}$, $d \leq \operatorname{weight}(F) = w$.
\end{proof}

\begin{theorem}[Theorem 2: Fault-Tolerant Gauging Distance]
\label{thm:FaultTolerantGaugingDistance.spacetimeFaultDistance_eq_d}
\lean{QEC1.FaultTolerantGaugingDistance.spacetimeFaultDistance_eq_d}
\leanok
\uses{def:SpacetimeFaultDistance.gaugingSpacetimeFaultDistance, def:SpaceTimeDecoupling.FullOutcomePreserving, def:SpaceTimeDecoupling.theDeformedCode, def:OptimalCheegerConstant.liftToExtended, def:StabilizerCode, thm:FaultTolerantGaugingDistance.spacetimeFaultDistance_le_d, thm:FaultTolerantGaugingDistance.spacetimeFaultDistance_ge_d}
The spacetime fault-distance of the fault-tolerant gauging measurement procedure equals $d$, the distance of the original code, when $h(G) \geq 1$:
\[
d_{\text{spacetime}} = d.
\]
No parity assumption on $d$ is needed: when $d$ is even, the lower bound still holds because case~(a) (time-logical) is vacuous, and case~(b) (space-logical) provides $|F| \geq d^* \geq d$.
\end{theorem}

\begin{proof}
\leanok
\uses{thm:FaultTolerantGaugingDistance.spacetimeFaultDistance_le_d, thm:FaultTolerantGaugingDistance.spacetimeFaultDistance_ge_d}
We apply antisymmetry of $\leq$. The upper bound $d_{\text{spacetime}} \leq d$ follows from $\operatorname{spacetimeFaultDistance\_le\_d}$, and the lower bound $d \leq d_{\text{spacetime}}$ follows from $\operatorname{spacetimeFaultDistance\_ge\_d}$.
\end{proof}

\section{Corollaries}

\begin{corollary}[Spacetime Fault-Distance is Positive]
\label{cor:FaultTolerantGaugingDistance.spacetimeFaultDistance_pos}
\lean{QEC1.FaultTolerantGaugingDistance.spacetimeFaultDistance_pos}
\leanok
\uses{thm:FaultTolerantGaugingDistance.spacetimeFaultDistance_eq_d, def:SpacetimeFaultDistance.gaugingSpacetimeFaultDistance, def:SpaceTimeDecoupling.FullOutcomePreserving}
The spacetime fault-distance is positive:
\[
0 < d_{\text{spacetime}}.
\]
\end{corollary}

\begin{proof}
\leanok
\uses{thm:FaultTolerantGaugingDistance.spacetimeFaultDistance_eq_d}
Rewriting with $\operatorname{spacetimeFaultDistance\_eq\_d}$, the goal becomes $0 < d$, which holds by the positivity assumption $d > 0$ from the procedure.
\end{proof}

\begin{corollary}[Weight $< d$ Implies Gauging Stabilizer]
\label{cor:FaultTolerantGaugingDistance.weight_lt_d_is_stabilizer}
\lean{QEC1.FaultTolerantGaugingDistance.weight_lt_d_is_stabilizer}
\leanok
\uses{def:SpaceTimeDecoupling.IsFullGaugingStabilizer, def:SpacetimeFaultDistance.gaugingSpacetimeFaultDistance, thm:FaultTolerantGaugingDistance.spacetimeFaultDistance_eq_d, thm:SpacetimeFaultDistance.syndromeFree_gauging_weight_lt_is_stabilizer}
Any syndrome-free full gauging fault of weight $< d$ is a gauging stabilizer.
\end{corollary}

\begin{proof}
\leanok
\uses{thm:FaultTolerantGaugingDistance.spacetimeFaultDistance_eq_d, thm:SpacetimeFaultDistance.syndromeFree_gauging_weight_lt_is_stabilizer}
We first establish that $\operatorname{weight}(F) < d_{\text{spacetime}}$ by rewriting $d_{\text{spacetime}} = d$ using $\operatorname{spacetimeFaultDistance\_eq\_d}$, combined with the hypothesis $\operatorname{weight}(F) < d$. The result then follows directly from $\operatorname{syndromeFree\_gauging\_weight\_lt\_is\_stabilizer}$, which states that any syndrome-free fault of weight below the spacetime fault-distance is a stabilizer.
\end{proof}

\begin{corollary}[Spacetime Fault-Distance Equals Phase 2 Duration]
\label{cor:FaultTolerantGaugingDistance.spacetimeFaultDistance_eq_phase2_duration}
\lean{QEC1.FaultTolerantGaugingDistance.spacetimeFaultDistance_eq_phase2_duration}
\leanok
\uses{def:SpacetimeFaultDistance.gaugingSpacetimeFaultDistance, def:SpaceTimeDecoupling.FullOutcomePreserving, def:FaultTolerantGaugingProcedure, thm:FaultTolerantGaugingDistance.spacetimeFaultDistance_eq_d}
The spacetime fault-distance equals the Phase~2 duration:
\[
d_{\text{spacetime}} = t_{\text{phase3}} - t_{\text{phase2}}.
\]
\end{corollary}

\begin{proof}
\leanok
\uses{thm:FaultTolerantGaugingDistance.spacetimeFaultDistance_eq_d}
Rewriting with $\operatorname{spacetimeFaultDistance\_eq\_d}$, the goal becomes $d = t_{\text{phase3}} - t_{\text{phase2}}$, which holds by the symmetry of $\operatorname{phase2\_duration}$.
\end{proof}

\begin{corollary}[Deformed Distance $\geq d$]
\label{cor:FaultTolerantGaugingDistance.deformed_distance_ge_d}
\lean{QEC1.FaultTolerantGaugingDistance.deformed_distance_ge_d}
\leanok
\uses{def:SpaceTimeDecoupling.theDeformedCode, def:DeformedCodeChecks.deformedStabilizerCode, def:StabilizerCode.distance, thm:OptimalCheegerConstant.sufficient_expansion_gives_dstar_ge_d}
When $h(G) \geq 1$, the space code distance satisfies $d^* \geq d$:
\[
d_{\text{orig}} \leq d^*.
\]
\end{corollary}

\begin{proof}
\leanok
\uses{thm:OptimalCheegerConstant.sufficient_expansion_gives_dstar_ge_d}
This follows directly from $\operatorname{sufficient\_expansion\_gives\_dstar\_ge\_d}$, applied with the deformed code data.
\end{proof}

%--- Rem_15: FluxCheckMeasurementFrequency ---
\chapter{Rem 15: Flux Check Measurement Frequency}

This remark explains why the fault-tolerance proof (Theorem~2) holds even if the flux checks $B_p$ are measured much less frequently than every round, or even not directly measured at all. The key observations are:
\begin{enumerate}
\item $B_p$ can be inferred from initialization ($|0\rangle_e$ at $t_i$) and final readout ($Z_e$ at $t_o$) without direct measurement, since $B_p = \prod_{e \in p} Z_e$ and $|0\rangle_e$ is a $+1$ eigenstate of $Z_e$.
\item The distance bound from Lemma~3 does not require $B_p$ measurements; it only requires $B_p$ to be stabilizers of the code.
\item The time-fault analysis (Lemma~6) shows $A_v$ measurement faults are the bottleneck for time-fault distance, not $B_p$.
\end{enumerate}

\textbf{Caveats:}
\begin{enumerate}
\item Without frequent $B_p$ measurements, detector cells become large (spanning $t_i$ to $t_o$).
\item Large detectors prevent achieving a threshold against random noise.
\item For small fixed code instances, this approach may be practical.
\end{enumerate}

\begin{theorem}[Flux Inferred from Init Readout]
\label{thm:FluxCheckMeasurementFrequency.flux_inferred_from_init_readout}
\lean{QEC1.FluxCheckMeasurementFrequency.flux_inferred_from_init_readout}
\leanok
\uses{def:FaultTolerantGaugingProcedure, def:FaultTolerantGaugingProcedure.fluxInitDetector}
The flux init detector ($B_p^{t_i}$) captures the relationship between edge initialization and the first flux measurement. For every edge $e \in p$ belonging to the cycle of plaquette $p$, the edge initialization event for $e$ is contained in the measurements of the flux init detector. This means $B_p = \prod_{e \in p} Z_e$ is encoded via $|0\rangle$ initialization.
\end{theorem}

\begin{proof}
\leanok
\uses{lem:FaultTolerantGaugingProcedure.fluxInitDetector_has_edgeInit}
Let $e$ be an edge with $e \in \mathrm{cycles}(p)$. This follows directly from the lemma \texttt{fluxInitDetector\_has\_edgeInit}, which states that the flux init detector for plaquette $p$ contains the edge initialization measurement for each edge $e$ in the cycle.
\end{proof}

\begin{theorem}[Flux Inferred from Readout]
\label{thm:FluxCheckMeasurementFrequency.flux_inferred_from_readout}
\lean{QEC1.FluxCheckMeasurementFrequency.flux_inferred_from_readout}
\leanok
\uses{def:FaultTolerantGaugingProcedure, def:FaultTolerantGaugingProcedure.fluxUngaugeDetector}
The flux ungauge detector ($B_p^{t_o}$) captures the relationship between the last flux measurement and individual $Z_e$ readouts for $e \in p$. For every edge $e$ in the cycle of plaquette $p$, the Phase~3 edge-$Z$ readout measurement for $e$ is contained in the measurements of the flux ungauge detector.
\end{theorem}

\begin{proof}
\leanok
\uses{lem:FaultTolerantGaugingProcedure.fluxUngaugeDetector_has_edgeZ}
Let $e$ be an edge with $e \in \mathrm{cycles}(p)$. This follows directly from the lemma \texttt{fluxUngaugeDetector\_has\_edgeZ}, which states that the flux ungauge detector for plaquette $p$ at the last round contains the Phase~3 edge-$Z$ measurement for each edge in the cycle.
\end{proof}

\begin{theorem}[Flux Init Contains First Measurement]
\label{thm:FluxCheckMeasurementFrequency.flux_init_contains_first_measurement}
\lean{QEC1.FluxCheckMeasurementFrequency.flux_init_contains_first_measurement}
\leanok
\uses{def:FaultTolerantGaugingProcedure, def:FaultTolerantGaugingProcedure.fluxInitDetector}
The flux init detector also contains the first $B_p$ measurement in Phase~2. This single measurement, together with the edge initializations, suffices to capture $B_p$ information --- no repeated $B_p$ measurements are needed.
\end{theorem}

\begin{proof}
\leanok
\uses{lem:FaultTolerantGaugingProcedure.fluxInitDetector_has_flux}
This follows directly from \texttt{fluxInitDetector\_has\_flux}, which states that the flux init detector for plaquette $p$ at round $0$ contains the Phase~2 flux measurement at round~$0$.
\end{proof}

\begin{theorem}[Flux Ungauge Contains Last Measurement]
\label{thm:FluxCheckMeasurementFrequency.flux_ungauge_contains_last_measurement}
\lean{QEC1.FluxCheckMeasurementFrequency.flux_ungauge_contains_last_measurement}
\leanok
\uses{def:FaultTolerantGaugingProcedure, def:FaultTolerantGaugingProcedure.fluxUngaugeDetector}
The flux ungauge detector contains the last $B_p$ measurement in Phase~2.
\end{theorem}

\begin{proof}
\leanok
\uses{lem:FaultTolerantGaugingProcedure.fluxUngaugeDetector_has_flux}
This follows directly from \texttt{fluxUngaugeDetector\_has\_flux}, which states that the flux ungauge detector for plaquette $p$ at the last round ($d-1$) contains the Phase~2 flux measurement at that round.
\end{proof}

\begin{theorem}[Space Distance Independent of Flux Measurements]
\label{thm:FluxCheckMeasurementFrequency.space_distance_independent_of_flux_measurements}
\lean{QEC1.FluxCheckMeasurementFrequency.space_distance_independent_of_flux_measurements}
\leanok
\uses{def:StabilizerCode, def:StabilizerCode.distance, def:DeformedCode.DeformedCodeData, def:DeformedCodeChecks.deformedStabilizerCode, def:DesiderataForGraphG.SufficientExpansion}
Lemma~3's distance bound requires only that $B_p$ operators are elements of the deformed stabilizer group, which is an algebraic fact. No measurement of $B_p$ is needed for $d^* \geq d$. Formally, given a stabilizer code $C$ with checks, a deformed code data, connectivity, the exactness conditions, and sufficient expansion $h(G) \geq 1$, we have
\[
d(C) \leq d(\text{deformed code}).
\]
\end{theorem}

\begin{proof}
\leanok
\uses{thm:SpaceDistance.deformed_distance_ge_d_sufficient_expansion}
This follows directly from the space distance theorem \texttt{deformed\_distance\_ge\_d\_sufficient\_expansion}, which establishes that when the graph $G$ has sufficient expansion, the deformed code distance is at least the original code distance. The proof uses only algebraic properties of flux operators as stabilizers, connectivity, coboundary exactness, and the Cheeger expansion bound --- no flux measurements are required.
\end{proof}

\begin{theorem}[Time Fault Bottleneck Is Gauss]
\label{thm:FluxCheckMeasurementFrequency.time_fault_bottleneck_is_gauss}
\lean{QEC1.FluxCheckMeasurementFrequency.time_fault_bottleneck_is_gauss}
\leanok
\uses{def:TimeFaultDistance.timeFaultDistance, def:TimeFaultDistance.gaussStringFault}
The time-fault distance is determined by $A_v$ measurement strings, not $B_p$. The canonical minimum-weight pure-time logical fault is a single $A_v$ string of weight $d$. For any vertex $v$ and odd $d$:
\[
d_{\mathrm{time}} \leq d.
\]
\end{theorem}

\begin{proof}
\leanok
\uses{thm:TimeFaultDistance.timeFaultDistance_le_d}
This follows directly from \texttt{timeFaultDistance\_le\_d}, which provides the upper bound by exhibiting the Gauss string fault for vertex $v$ as a syndrome-free, sign-flipping pure-time fault of weight $d$.
\end{proof}

\begin{theorem}[Gauss String Weight]
\label{thm:FluxCheckMeasurementFrequency.gauss_string_weight}
\lean{QEC1.FluxCheckMeasurementFrequency.gauss_string_weight}
\leanok
\uses{def:TimeFaultDistance.gaussStringFault}
The $A_v$ measurement string has weight exactly $d$:
\[
w(\mathrm{gaussStringFault}(v)) = d.
\]
\end{theorem}

\begin{proof}
\leanok
\uses{lem:TimeFaultDistance.gaussStringFault_weight_eq_d}
This follows directly from \texttt{gaussStringFault\_weight\_eq\_d}.
\end{proof}

\begin{theorem}[Gauss String Is Syndrome Free]
\label{thm:FluxCheckMeasurementFrequency.gauss_string_is_syndrome_free}
\lean{QEC1.FluxCheckMeasurementFrequency.gauss_string_is_syndrome_free}
\leanok
\uses{def:TimeFaultDistance.gaussStringFault, def:SpacetimeLogicalFault.IsSyndromeFreeGauging}
The $A_v$ string is syndrome-free: consecutive repeated Gauss detectors see paired flips that cancel. This does not involve any flux detectors.
\end{theorem}

\begin{proof}
\leanok
\uses{thm:TimeFaultDistance.gaussStringFault_syndromeFree}
This follows directly from \texttt{gaussStringFault\_syndromeFree}.
\end{proof}

\begin{theorem}[Gauss String Flips Sign]
\label{thm:FluxCheckMeasurementFrequency.gauss_string_flips_sign}
\lean{QEC1.FluxCheckMeasurementFrequency.gauss_string_flips_sign}
\leanok
\uses{def:TimeFaultDistance.gaussStringFault, def:SpacetimeLogicalFault.FlipsGaugingSign}
The $A_v$ string flips the gauging sign when $d$ is odd.
\end{theorem}

\begin{proof}
\leanok
\uses{lem:TimeFaultDistance.gaussStringFault_flipsSign_of_odd}
This follows directly from \texttt{gaussStringFault\_flipsSign\_of\_odd}, applied to vertex $v$ and the hypothesis that $d$ is odd.
\end{proof}

\begin{theorem}[Flux Fault Preserves Sign]
\label{thm:FluxCheckMeasurementFrequency.flux_fault_preserves_sign}
\lean{QEC1.FluxCheckMeasurementFrequency.flux_fault_preserves_sign}
\leanok
\uses{def:SpacetimeFault, def:SpacetimeLogicalFault.PreservesGaugingSign, def:SpacetimeLogicalFault.gaussSignFlip}
$B_p$ measurement faults cannot flip the gauging sign $\sigma$. The sign $\sigma = \sum_v \varepsilon_v$ is defined as a sum over Gauss measurement outcomes only. A fault $F$ whose time faults contain only flux measurements contributes $0$ to this sum: no $A_v$ measurement is faulted. Formally, if every time fault $\mathrm{tf} \in F.\mathrm{timeFaults}$ is of the form $\mathrm{phase2}(\mathrm{flux}\; p\; r)$ for some plaquette $p$ and round $r$, then $F$ preserves the gauging sign.
\end{theorem}

\begin{proof}
\leanok
\uses{def:SpacetimeLogicalFault.gaussSignFlip}
We unfold the definitions of \texttt{PreservesGaugingSign} and \texttt{gaussSignFlip}. The gauging sign flip is a double sum $\sum_{v \in V} \sum_{r \in \mathrm{Fin}\, d}$ of indicators. We apply \texttt{Finset.sum\_eq\_zero} twice: for each vertex $v$ and each round $r$, it suffices to show the indicator is zero.

We split on the conditional. In the case where the indicator is $1$, we have the hypothesis $h$ that the Gauss measurement $\langle \mathrm{phase2}(\mathrm{gaussLaw}\; v\; r) \rangle$ is in $F.\mathrm{timeFaults}$. But by assumption, all time faults are flux measurements. We obtain a plaquette $p$ and round $r'$ such that $\langle \mathrm{phase2}(\mathrm{gaussLaw}\; v\; r) \rangle = \langle \mathrm{phase2}(\mathrm{flux}\; p\; r') \rangle$. Extracting the measurement field equality, we get $\mathrm{phase2}(\mathrm{gaussLaw}\; v\; r) = \mathrm{phase2}(\mathrm{flux}\; p\; r')$. Since \texttt{phase2} is injective, this gives $\mathrm{gaussLaw}\; v\; r = \mathrm{flux}\; p\; r'$, which is impossible by constructor disjointness. In the other case, the result holds by reflexivity.
\end{proof}

\begin{theorem}[Gauss Sign Flip Ignores Flux]
\label{thm:FluxCheckMeasurementFrequency.gaussSignFlip_ignores_flux}
\lean{QEC1.FluxCheckMeasurementFrequency.gaussSignFlip_ignores_flux}
\leanok
\uses{def:TimeFaultDistance.gaussStringFault, def:SpacetimeLogicalFault.gaussSignFlip}
The gauging sign is determined entirely by $A_v$ measurement faults. The \texttt{gaussSignFlip} for the Gauss string fault at vertex $v$ equals
\[
\sum_{w \in V} \sum_{r \in \mathrm{Fin}\, d} \begin{cases} 1 & \text{if } \langle \mathrm{phase2}(\mathrm{gaussLaw}\; w\; r) \rangle \in (\mathrm{gaussStringFault}\; v).\mathrm{timeFaults} \\ 0 & \text{otherwise} \end{cases}
\]
which sums indicators of $A_v$ membership in time faults only, ignoring all non-Gauss faults.
\end{theorem}

\begin{proof}
\leanok
\uses{def:TimeFaultDistance.gaussStringFault, def:SpacetimeLogicalFault.gaussSignFlip}
This holds by reflexivity (definitional unfolding).
\end{proof}

\begin{theorem}[Time Fault Distance Equals $d$]
\label{thm:FluxCheckMeasurementFrequency.time_fault_distance_eq_d}
\lean{QEC1.FluxCheckMeasurementFrequency.time_fault_distance_eq_d}
\leanok
\uses{def:TimeFaultDistance.timeFaultDistance, def:TimeFaultDistance.gaussStringFault}
The full time-fault distance result from Lemma~6: for any vertex $v$, odd $d$, and nonempty vertex set $V$,
\[
d_{\mathrm{time}} = d.
\]
\end{theorem}

\begin{proof}
\leanok
\uses{thm:TimeFaultDistance.timeFaultDistance_eq_d}
This follows directly from \texttt{timeFaultDistance\_eq\_d}, which combines the upper bound (from the Gauss string fault of weight $d$) with the lower bound (any syndrome-free sign-flipping pure-time fault has weight $\geq d$).
\end{proof}

\begin{theorem}[Repeated vs Boundary Detector Weight]
\label{thm:FluxCheckMeasurementFrequency.repeated_vs_boundary_detector_weight}
\lean{QEC1.FluxCheckMeasurementFrequency.repeated_vs_boundary_detector_weight}
\leanok
\uses{def:FaultTolerantGaugingProcedure, def:FaultTolerantGaugingProcedure.phase2RepeatedDetector_flux}
Repeated flux detectors have weight $2$ (two consecutive $B_p$ measurements). Formally, for any plaquette $p$ and consecutive rounds $r, r'$ with $r + 1 = r'$:
\[
|\mathrm{phase2RepeatedDetector\_flux}(p, r, r').\mathrm{measurements}| = 2.
\]
\end{theorem}

\begin{proof}
\leanok
\uses{def:FaultTolerantGaugingProcedure.phase2RepeatedDetector_flux}
We simplify using the definition of \texttt{phase2RepeatedDetector\_flux} and compute the cardinality of the pair. To show the two measurements are distinct, we assume for contradiction that they are equal. Since \texttt{phase2} is injective and \texttt{flux} is injective, this would imply $r = r'$. Substituting, we obtain $r + 1 = r$ from the hypothesis $r + 1 = r'$, which is a contradiction by integer arithmetic.
\end{proof}

\begin{theorem}[Flux Boundary Detector Spans Full Duration]
\label{thm:FluxCheckMeasurementFrequency.flux_boundary_detector_spans_full_duration}
\lean{QEC1.FluxCheckMeasurementFrequency.flux_boundary_detector_spans_full_duration}
\leanok
\uses{def:FaultTolerantGaugingProcedure, def:FaultTolerantGaugingProcedure.phase2Start, def:FaultTolerantGaugingProcedure.phase3Start}
Without repeated flux detectors, the boundary detectors span the full Phase~2 duration $d$. Formally:
\[
\mathrm{phase2Start} \leq \mathrm{phase3Start} \quad \text{and} \quad \mathrm{phase3Start} - \mathrm{phase2Start} = d.
\]
\end{theorem}

\begin{proof}
\leanok
\uses{lem:FaultTolerantGaugingProcedure.phase2Start_le_phase3Start, lem:FaultTolerantGaugingProcedure.phase2_duration}
The first conjunct follows from \texttt{phase2Start\_le\_phase3Start} and the second from \texttt{phase2\_duration}.
\end{proof}

\begin{theorem}[Boundary Larger Than Repeated]
\label{thm:FluxCheckMeasurementFrequency.boundary_larger_than_repeated}
\lean{QEC1.FluxCheckMeasurementFrequency.boundary_larger_than_repeated}
\leanok
\uses{def:FaultTolerantGaugingProcedure, def:FaultTolerantGaugingProcedure.phase2RepeatedDetector_flux}
Large detector cells: the number of measurements in a boundary detector ($|p| + 1$ for flux init) is strictly greater than a repeated detector ($2$), as soon as the cycle has at least $2$ edges. Formally, if $|\{e \in G.\mathrm{edgeSet} \mid e \in \mathrm{cycles}(p)\}| \geq 2$, then
\[
|\mathrm{phase2RepeatedDetector\_flux}(p, r, r').\mathrm{measurements}| < |\{e \mid e \in \mathrm{cycles}(p)\}| + 1.
\]
\end{theorem}

\begin{proof}
\leanok
\uses{thm:FluxCheckMeasurementFrequency.repeated_vs_boundary_detector_weight}
We rewrite using the fact that the repeated detector has cardinality $2$ (from \texttt{repeated\_vs\_boundary\_detector\_weight}). Then the inequality $2 < |p| + 1$ follows by integer arithmetic from the hypothesis $2 \leq |p|$.
\end{proof}

\begin{theorem}[Space Distance from Stabilizer Algebra]
\label{thm:FluxCheckMeasurementFrequency.space_distance_from_stabilizer_algebra}
\lean{QEC1.FluxCheckMeasurementFrequency.space_distance_from_stabilizer_algebra}
\leanok
\uses{def:StabilizerCode, def:StabilizerCode.stabilizerGroup, def:DeformedCode.DeformedCodeData, def:DeformedCodeChecks.deformedStabilizerCode, def:DeformedCode.fluxChecks}
The space-distance bound $d^* \geq d$ (Lemma~3) requires $h(G) \geq 1$ and exactness. No measurement of flux checks $B_p$ is needed --- only their algebraic presence as stabilizers of the deformed code. Formally, for any plaquette $p$:
\[
\mathrm{fluxChecks}(G, \mathrm{cycles}, p) \in \mathrm{stabilizerGroup}(\text{deformed code}).
\]
\end{theorem}

\begin{proof}
\leanok
\uses{thm:DeformedCodeChecks.flux_mem_stabilizerGroup}
This follows directly from \texttt{flux\_mem\_stabilizerGroup}, which establishes algebraically that every flux operator belongs to the stabilizer group of the deformed code.
\end{proof}

\begin{theorem}[Flux in Centralizer Algebraically]
\label{thm:FluxCheckMeasurementFrequency.flux_in_centralizer_algebraically}
\lean{QEC1.FluxCheckMeasurementFrequency.flux_in_centralizer_algebraically}
\leanok
\uses{def:StabilizerCode.inCentralizer, def:DeformedCodeChecks.deformedStabilizerCode, def:DeformedCode.fluxChecks}
Flux operators are also in the centralizer (commute with all checks), which is the property actually used in Lemma~3. For any plaquette $p$:
\[
\mathrm{inCentralizer}(\text{deformed code}, \mathrm{fluxChecks}(G, \mathrm{cycles}, p)).
\]
\end{theorem}

\begin{proof}
\leanok
\uses{thm:DeformedCodeChecks.flux_inCentralizer}
This follows directly from \texttt{flux\_inCentralizer}, which establishes that flux checks commute with all checks of the deformed code.
\end{proof}

%--- Rem_16: BoundaryRoundsOverkill ---

\chapter{Rem 16: Boundary Rounds Overkill}

The $d$ rounds of error correction in the original code before and after the gauging measurement (Phases 1 and 3 in Def 10) are conservative and often unnecessary in practice.

\textbf{Justification for $d$ rounds:} The $d$ rounds ensure that any error process involving both the gauging measurement and the initial/final boundary has weight greater than $d$, where $d$ is the distance of the original $[\![n,k,d]\!]$ stabilizer code. This facilitates a clean proof of the fault-distance bound (Thm 2).

\textbf{In practice:}
\begin{itemize}
\item When the gauging measurement is part of a larger fault-tolerant computation, the surrounding operations provide natural boundaries for the spacetime fault analysis.
\item A constant number of rounds before and after may suffice.
\item The optimal number depends on the specific computation and affects the effective distance and threshold.
\end{itemize}

\section*{Point 1: The All-or-None Property is Purely Phase 2}

The critical all-or-none property states that for a syndrome-free fault, each vertex $v$ has either ALL $d$ Gauss $A_v$ measurements faulted or NONE of them. This is enforced by the Phase 2 repeated Gauss detectors, which pair consecutive $A_v$ measurements within Phase 2. Crucially, this property does NOT use Phase 1 or Phase 3 detectors at all.

\begin{theorem}[All-or-None is Phase 2 Only]
\label{thm:BoundaryRoundsOverkill.allOrNone_is_phase2_only}
\lean{QEC1.BoundaryRoundsOverkill.allOrNone_is_phase2_only}
\leanok
\uses{def:TimeFaultDistance.gaussFaultCount, def:FaultTolerantGaugingProcedure.DetectorIndex}
The all-or-none property (each vertex has $0$ or $d$ Gauss faults) is enforced by Phase 2 repeated Gauss detectors. Given any syndrome-free fault $F$, the Gauss fault count for each vertex $v$ satisfies
\[
\operatorname{gaussFaultCount}(v, F) = 0 \quad \lor \quad \operatorname{gaussFaultCount}(v, F) = d.
\]
This uses only Phase 2's repeated Gauss detectors, not Phases 1 or 3.
\end{theorem}

\begin{proof}
\leanok
\uses{def:TimeFaultDistance.gaussFaultCount}
This follows directly from \texttt{gaussFaultCount\_zero\_or\_d}, which establishes the all-or-none dichotomy for any syndrome-free gauging fault using only the Phase 2 repeated Gauss detector structure.
\end{proof}

\begin{theorem}[Time-Fault Lower Bound Independent of Boundary]
\label{thm:BoundaryRoundsOverkill.time_fault_lower_bound_independent_of_boundary}
\lean{QEC1.BoundaryRoundsOverkill.time_fault_lower_bound_independent_of_boundary}
\leanok
\uses{def:TimeFaultDistance.IsPureTimeFault, thm:TimeFaultDistance.pureTime_logicalFault_weight_ge_d}
The time-fault distance lower bound (pure-time logical faults have weight $\geq d$) follows from the all-or-none property and sign-flip analysis, both of which are Phase 2 phenomena. Specifically, if $F$ is a pure-time fault that is syndrome-free and flips the gauging sign, and $d$ is odd, then
\[
d \leq \operatorname{weight}(F).
\]
This does not depend on the number of Phase 1 or Phase 3 rounds.
\end{theorem}

\begin{proof}
\leanok
\uses{thm:TimeFaultDistance.pureTime_logicalFault_weight_ge_d}
This follows directly from \texttt{pureTime\_logicalFault\_weight\_ge\_d}, which establishes that any pure-time logical fault (syndrome-free, sign-flipping) must have weight at least $d$, using only the Phase 2 all-or-none structure.
\end{proof}

\section*{Point 2: Boundary Detectors Connect Phases}

The boundary detectors bridge Phase $1 \leftrightarrow 2$ and Phase $2 \leftrightarrow 3$ respectively. These detectors involve the LAST measurement of Phase 1 (or Phase 2), the FIRST measurement of Phase 2 (or Phase 3), and edge initialization or readout events. Only one round from each boundary phase participates in these detectors.

\begin{theorem}[Boundary Detector at $t_i$ Uses One Phase 1 Round]
\label{thm:BoundaryRoundsOverkill.boundary_detector_at_ti_uses_one_phase1_round}
\lean{QEC1.BoundaryRoundsOverkill.boundary_detector_at_ti_uses_one_phase1_round}
\leanok
\uses{def:FaultTolerantGaugingProcedure.DetectorIndex}
The boundary detector at $t_i$ connects the last Phase 1 round to the first Phase 2 round for deformed checks. It includes exactly one measurement from Phase 1 (the last round $r = d-1$) and one from Phase 2 (the first round $r = 0$), plus edge initialization events. Specifically, for $d \geq 2$ and any check index $j$, the Phase 1 measurement at round $r_{\mathrm{last}} = d-1$ belongs to the measurements of the deformed initialization detector.
\end{theorem}

\begin{proof}
\leanok
\uses{def:FaultTolerantGaugingProcedure.DetectorIndex}
This follows directly from \texttt{deformedInitDetector\_has\_phase1}, which witnesses that the Phase 1 measurement at the last round is contained in the boundary detector's measurement set.
\end{proof}

\begin{theorem}[Boundary Detector at $t_o$ Uses One Phase 3 Round]
\label{thm:BoundaryRoundsOverkill.boundary_detector_at_to_uses_one_phase3_round}
\lean{QEC1.BoundaryRoundsOverkill.boundary_detector_at_to_uses_one_phase3_round}
\leanok
\uses{def:FaultTolerantGaugingProcedure.DetectorIndex}
The boundary detector at $t_o$ connects the last Phase 2 round to the first Phase 3 round for deformed checks. It includes exactly one measurement from Phase 2 (the last round $r = d-1$) and one from Phase 3 (the first round $r = 0$), plus edge $Z$ readout events. Specifically, for $d \geq 2$ and any check index $j$, the Phase 3 measurement at round $r_{\mathrm{first}} = 0$ belongs to the measurements of the deformed ungauge detector.
\end{theorem}

\begin{proof}
\leanok
\uses{def:FaultTolerantGaugingProcedure.DetectorIndex}
This follows directly from \texttt{deformedUngaugeDetector\_has\_originalCheck}, which witnesses that the Phase 3 measurement at the first round is contained in the boundary detector's measurement set.
\end{proof}

\section*{Point 3: Only One Boundary Round Needed for Boundary Coverage}

The boundary detectors only use ONE measurement from Phase 1 (the last round $r = d-1$) and ONE from Phase 3 (the first round $r = 0$). The remaining $d-1$ rounds in each phase form repeated-measurement detectors pairing consecutive rounds of the same check. A single boundary round ($d_{\mathrm{boundary}} = 1$) would suffice for the boundary detector coverage.

\begin{theorem}[Phase 1 Repeated Detectors are Internal]
\label{thm:BoundaryRoundsOverkill.phase1_repeated_detectors_internal}
\lean{QEC1.BoundaryRoundsOverkill.phase1_repeated_detectors_internal}
\leanok
\uses{def:FaultTolerantGaugingProcedure.DetectorIndex}
Within Phase 1, the repeated detectors pair consecutive rounds of the same check. These are independent of the Phase $1 \leftrightarrow 2$ boundary detector and provide only standard error correction. For any check index $j$ and consecutive rounds $r, r'$ with $r + 1 = r'$, both Phase 1 measurements at rounds $r$ and $r'$ belong to the parametric repeated detector's measurement set.
\end{theorem}

\begin{proof}
\leanok
\uses{def:FaultTolerantGaugingProcedure.DetectorIndex}
We verify both membership conditions separately. For the first measurement at round $r$: it is the left element of the insert, so membership follows from $\operatorname{Or.inl}(\operatorname{rfl})$. For the second measurement at round $r'$: it is the singleton element, so membership follows from $\operatorname{Or.inr}(\operatorname{mem\_singleton.mpr}(\operatorname{rfl}))$.
\end{proof}

\begin{theorem}[Coverage Requires $d \geq 2$]
\label{thm:BoundaryRoundsOverkill.coverage_requires_d_ge_2}
\lean{QEC1.BoundaryRoundsOverkill.coverage_requires_d_ge_2}
\leanok
\uses{def:FaultTolerantGaugingProcedure.DetectorIndex, thm:FaultTolerantGaugingProcedure.every_measurement_covered}
The full measurement coverage theorem requires $d \geq 2$. With $d = 1$, there are no repeated detectors within Phase 1 or Phase 3 (since there is only one round each), but the boundary detectors still cover the single boundary measurement. The condition $d \geq 2$ comes from needing at least 2 rounds for the repeated Gauss detectors in Phase 2. Formally, for $d \geq 2$ and assuming every edge is covered by some cycle, every measurement label $m$ belongs to some detector:
\[
\forall m,\ \exists\, \mathit{idx} : \operatorname{DetectorIndex},\quad m \in (\operatorname{detectorOfIndex}(\mathit{idx})).\mathit{measurements}.
\]
\end{theorem}

\begin{proof}
\leanok
\uses{thm:FaultTolerantGaugingProcedure.every_measurement_covered}
This follows directly from \texttt{every\_measurement\_covered}, which establishes the complete measurement coverage property given $d \geq 2$ and the edge coverage hypothesis.
\end{proof}

\section*{Point 4: $d$ Rounds Used in Theorem 2's Clean Proof}

The full $d$ rounds in Phases 1 and 3 are used in Theorem 2 to ensure that ANY spacetime fault (not just pure-time faults) has weight $\geq d$. The argument uses the space-time decoupling: $F$ decomposes as $F = F_S \cdot F_T \cdot S$. For time-faults, the lower bound uses the all-or-none property (Phase 2 only). For space-faults, the lower bound uses the space distance $d^* \geq d$. The $d$ boundary rounds ensure that the space-time decoupling can move all space-faults to time $t_i$ using time-propagating generators that span all phases.

\begin{theorem}[$d$ Rounds Used in Theorem 2]
\label{thm:BoundaryRoundsOverkill.d_rounds_used_in_thm2}
\lean{QEC1.BoundaryRoundsOverkill.d_rounds_used_in_thm2}
\leanok
\uses{def:FaultTolerantGaugingProcedure.DetectorIndex, def:TimeFaultDistance.IsPureTimeFault, def:TimeFaultDistance.gaussFaultCount}
Theorem 2 uses the full $d$-round structure to establish $d_{\mathrm{spacetime}} = d$. Under the hypotheses that the original code has distance $d$, the graph $G$ is connected with $|V| \geq 2$, the cycle/boundary exactness conditions hold, the graph has sufficient expansion, the deformed code has logical operators, and a minimum-weight pure-$X$ logical operator exists that does not become a deformed stabilizer, the gauging spacetime fault distance equals $d$:
\[
d_{\mathrm{spacetime}} = d.
\]
The $d$ rounds in Phases 1 and 3 are conservative but yield a clean proof.
\end{theorem}

\begin{proof}
\leanok
\uses{def:FaultTolerantGaugingProcedure.DetectorIndex}
This follows directly from \texttt{FaultTolerantGaugingDistance.spacetimeFaultDistance\_eq\_d}, which combines the space distance bound ($d^* \geq d$ from sufficient expansion and Cheeger constant), the time distance bound (from the all-or-none property), and the space-time decoupling to establish the exact equality $d_{\mathrm{spacetime}} = d$.
\end{proof}

\section*{Point 5: Space-Faults at $t_i$ Caught by Phase 2}

Space-faults concentrated at time $t_i$ (the gauging time) are detected by the deformed code checks measured during Phase 2. The syndrome-free condition forces the Pauli error to be in the centralizer of the deformed code. This mechanism is purely Phase 2 and does not require Phases 1 or 3.

\begin{theorem}[Space-Fault at $t_i$ Caught by Phase 2]
\label{thm:BoundaryRoundsOverkill.space_fault_at_ti_caught_by_phase2}
\lean{QEC1.BoundaryRoundsOverkill.space_fault_at_ti_caught_by_phase2}
\leanok
\uses{def:FaultTolerantGaugingProcedure.DetectorIndex}
A pure-space fault with no time-faults is syndrome-free: space-faults do not flip measurement outcomes, so no detector is violated. Detection of space-faults relies on their Pauli-group effect on the code state, not on measurement faults. This is purely Phase 2 structure (deformed code checks at $t_i$). Formally, if $F$ is a pure-space fault (i.e., $F.\mathit{timeFaults} = \emptyset$), then $F$ is syndrome-free with respect to all gauging detectors.
\end{theorem}

\begin{proof}
\leanok
\uses{def:FaultTolerantGaugingProcedure.DetectorIndex}
Let $\mathit{idx}$ be an arbitrary detector index. Since $F$ is a pure-space fault, we have $F.\mathit{timeFaults} = \emptyset$. Rewriting with this fact, the detector applied to an empty set of time-faults is not violated, which follows from \texttt{Detector.not\_isViolated\_no\_faults}.
\end{proof}

\begin{theorem}[Space-Fault Preserves Sign]
\label{thm:BoundaryRoundsOverkill.space_fault_preserves_sign}
\lean{QEC1.BoundaryRoundsOverkill.space_fault_preserves_sign}
\leanok
\uses{def:FaultTolerantGaugingProcedure.DetectorIndex}
Pure-space faults preserve the gauging sign $\sigma$: the sign is determined entirely by time-faults (measurement errors on Gauss's law checks), so space-faults at $t_i$ have no effect on $\sigma$. Formally, if $F$ is a pure-space fault, then $F$ preserves the gauging sign.
\end{theorem}

\begin{proof}
\leanok
\uses{def:FaultTolerantGaugingProcedure.DetectorIndex}
Unfolding the definitions of $\operatorname{PreservesGaugingSign}$ and $\operatorname{gaussSignFlip}$, since $F$ is a pure-space fault we have $F.\mathit{timeFaults} = \emptyset$. Rewriting with this, the sum over the empty set is zero by $\operatorname{Finset.sum\_empty}$, and simplification completes the proof.
\end{proof}

\section*{Summary}

\begin{theorem}[Boundary Rounds Overkill Summary]
\label{thm:BoundaryRoundsOverkill.boundary_rounds_overkill_summary}
\lean{QEC1.BoundaryRoundsOverkill.boundary_rounds_overkill_summary}
\leanok
\uses{def:FaultTolerantGaugingProcedure.DetectorIndex, def:TimeFaultDistance.gaussFaultCount}
The key fault-distance mechanisms and their phase dependence are summarized as follows. For any vertex $v$, check index $j$, and syndrome-free fault $F$:
\begin{enumerate}
\item \textbf{All-or-none is Phase 2 only:} $\operatorname{gaussFaultCount}(v, F) = 0 \;\lor\; \operatorname{gaussFaultCount}(v, F) = d$.
\item \textbf{Boundary detector uses 1 Phase 1 measurement:} The Phase 1 measurement at round $r = d-1$ belongs to the deformed initialization detector's measurements.
\item \textbf{Boundary detector uses 1 Phase 3 measurement:} The Phase 3 measurement at round $r = 0$ belongs to the deformed ungauge detector's measurements.
\end{enumerate}
The $d$ boundary rounds are conservative: a constant number of boundary rounds would preserve the essential detector coverage structure.
\end{theorem}

\begin{proof}
\leanok
\uses{def:TimeFaultDistance.gaussFaultCount, def:FaultTolerantGaugingProcedure.DetectorIndex}
We prove all three conjuncts using $\operatorname{refine}$. For part (1), the all-or-none dichotomy follows directly from \texttt{gaussFaultCount\_zero\_or\_d} applied to the syndrome-free hypothesis. For part (2), the Phase 1 membership follows from \texttt{deformedInitDetector\_has\_phase1} applied to check $j$ with the last round $r = d-1$, the first round $r' = 0$, and the cancellation proof $d - 1 + 1 = d$. For part (3), the Phase 3 membership follows from \texttt{deformedUngaugeDetector\_has\_originalCheck} applied to check $j$ with the corresponding round indices and cancellation proof.
\end{proof}

%--- Cor_1: WorstCaseOverhead ---
\chapter{Cor 1: Worst-Case Qubit Overhead}

The worst-case qubit overhead for the fault-tolerant gauging measurement of a Pauli logical operator $L$ of weight $W$ in an $[\![n,k,d]\!]$ qLDPC stabilizer code is $O(W \log^2 W)$, where the implicit constant depends on the code parameters (the check weight bound $w$ and the qubit participation bound $c$) but not on $W$ or $n$.

More precisely: there exists a choice of graph $G$ such that the total number of additional qubits (edge qubits) introduced by the fault-tolerant gauging measurement procedure is $O(W \log^2 W)$, and the resulting deformed code satisfies all desiderata (bounded-weight checks, preserved distance $d^* \geq d$, and LDPC property).

\section*{Edge Overhead from Sparsified Graph Construction}

\begin{theorem}[Edge Overhead from Degree]
\label{thm:WorstCaseOverhead.edge_overhead_from_degree}
\lean{QEC1.WorstCaseOverhead.edge_overhead_from_degree}
\leanok
\uses{thm:WorstCaseGraphConstruction.construction_satisfies_all_desiderata}
Let $N, \Delta$ be natural numbers such that $2N \leq \Delta \cdot N$. Then $N \leq \Delta \cdot N / 2$.
\end{theorem}

\begin{proof}
\leanok
\uses{thm:WorstCaseGraphConstruction.construction_satisfies_all_desiderata}
This follows directly by integer arithmetic from the hypothesis $2N \leq \Delta \cdot N$.
\end{proof}

\begin{theorem}[Edge Overhead is $O(W \log^2 W)$]
\label{thm:WorstCaseOverhead.edge_overhead_Wlog2W}
\lean{QEC1.WorstCaseOverhead.edge_overhead_Wlog2W}
\leanok
\uses{thm:WorstCaseGraphConstruction.construction_satisfies_all_desiderata}
Let $W, \Delta, K, R$ be natural numbers with $R \leq K \cdot (\log_2 W)^2$ and $\Delta > 0$. Then
\[
2 \cdot (R+1) \cdot W \leq 2 \cdot (K \cdot (\log_2 W)^2 + 1) \cdot W.
\]
\end{theorem}

\begin{proof}
\leanok
\uses{thm:WorstCaseGraphConstruction.construction_satisfies_all_desiderata}
We apply monotonicity of multiplication on the left by $2$ and on the right by $W$. It suffices to show $R + 1 \leq K \cdot (\log_2 W)^2 + 1$, which follows by integer arithmetic from the hypothesis $R \leq K \cdot (\log_2 W)^2$.
\end{proof}

\begin{theorem}[Per-Vertex Overhead]
\label{thm:WorstCaseOverhead.per_vertex_overhead}
\lean{QEC1.WorstCaseOverhead.per_vertex_overhead}
\leanok
\uses{thm:WorstCaseGraphConstruction.construction_satisfies_all_desiderata}
Let $W, K, R$ be natural numbers with $R \leq K \cdot (\log_2 W)^2$ and $W > 0$. Then
\[
\frac{(R+1) \cdot W}{W} \leq K \cdot (\log_2 W)^2 + 1.
\]
\end{theorem}

\begin{proof}
\leanok
\uses{thm:WorstCaseGraphConstruction.construction_satisfies_all_desiderata}
Rewriting $(R+1) \cdot W / W = R + 1$ using the cancellation lemma (since $W > 0$), the result follows by integer arithmetic from $R \leq K \cdot (\log_2 W)^2$.
\end{proof}

\section*{Desiderata Satisfied with $O(W \log^2 W)$ Overhead}

\begin{theorem}[All Desiderata with Overhead Bound]
\label{thm:WorstCaseOverhead.all_desiderata_with_overhead_bound}
\lean{QEC1.WorstCaseOverhead.all_desiderata_with_overhead_bound}
\leanok
\uses{thm:WorstCaseGraphConstruction.construction_satisfies_all_desiderata, thm:SpaceDistance.deformed_distance_ge_d_sufficient_expansion}
Let $G$ be a constant-degree $\Delta$ connected graph on $V$ with $|V| \geq 2$, sufficient expansion $h(G) \geq 1$, $Z$-support matching edges, and a low-weight cycle basis with cycles of weight $\leq 4$. Given the cycle rank property $|C| + |V| = |E| + 1$, a layer cycle degree bound with $M$ layers satisfying $M \leq K \cdot (\log_2 |V|)^2$, and a per-layer parameter $c > 0$, the following all hold:
\begin{enumerate}
\item All desiderata are satisfied: $\operatorname{AllDesiderataSatisfied}(G, \text{cycles}, \text{checks}, 1, 4)$.
\item There exist $R$ layers and a coloring $f$ with $\operatorname{LayerCycleDegreeBound}(\text{cycles}, f, c)$ and $R \leq K \cdot (\log_2 |V|)^2 / c$.
\item The edge overhead is linear: $2|E| \leq \Delta \cdot |V|$.
\item The expansion gives a boundary bound: for all nonempty $S \subseteq V$ with $2|S| \leq |V|$, we have $|S| \leq |\partial S|$.
\end{enumerate}
\end{theorem}

\begin{proof}
\leanok
\uses{thm:WorstCaseGraphConstruction.construction_satisfies_all_desiderata, thm:SpaceDistance.deformed_distance_ge_d_sufficient_expansion}
The four conclusions are established as a conjunction:
\begin{enumerate}
\item All desiderata follow directly from \texttt{construction\_satisfies\_all\_desiderata} applied to $G$, the cycles, checks, $\Delta$, the $Z$-support adjacency hypothesis, the sufficient expansion, the low-weight cycle basis, and the constant degree bound.
\item The $O(\log^2 W)$ layer count follows from \texttt{decongestion\_log\_squared} applied to $G$, $\Delta$, the constant degree, connectivity, vertex count bound, the positivity of the expansion constant, the cycles, $c$, the cycle rank property, and the K\"onig-Freedman-Hastings bound.
\item The edge overhead bound $2|E| \leq \Delta \cdot |V|$ follows from \texttt{constant\_degree\_bounds\_edges}.
\item The expansion-based boundary bound follows from \texttt{expansion\_gives\_boundary\_bound} applied to each nonempty subset $S$ with $2|S| \leq |V|$.
\end{enumerate}
\end{proof}

\begin{theorem}[Check Weights Bounded]
\label{thm:WorstCaseOverhead.check_weights_bounded}
\lean{QEC1.WorstCaseOverhead.check_weights_bounded}
\leanok
\uses{thm:WorstCaseGraphConstruction.construction_satisfies_all_desiderata}
Let $G$ be a constant-degree $\Delta$ graph with a low-weight cycle basis of weight $\leq 4$. Then:
\begin{enumerate}
\item For every vertex $v \in V$, the Gauss law check $A_v$ has weight at most $1 + \Delta$.
\item For every cycle $p \in C$, the flux check $B_p$ has weight at most $4$.
\end{enumerate}
\end{theorem}

\begin{proof}
\leanok
\uses{thm:WorstCaseGraphConstruction.construction_satisfies_all_desiderata}
The bound on Gauss law check weights follows from \texttt{gaussLaw\_checks\_weight\_bounded} applied to $G$, $\Delta$, and the constant degree hypothesis. The bound on flux check weights follows from \texttt{flux\_checks\_weight\_bounded} applied to $G$, the cycles, $4$, and the low-weight cycle basis hypothesis.
\end{proof}

\section*{Distance Preservation}

\begin{theorem}[Distance Preserved with Overhead]
\label{thm:WorstCaseOverhead.distance_preserved_with_overhead}
\lean{QEC1.WorstCaseOverhead.distance_preserved_with_overhead}
\leanok
\uses{thm:SpaceDistance.deformed_distance_ge_d_sufficient_expansion, thm:WorstCaseGraphConstruction.construction_satisfies_all_desiderata}
Let $G$ be a connected constant-degree $\Delta$ graph on $V$ with $|V| \geq 2$ and sufficient expansion, equipped with a cycle basis satisfying the exactness conditions. Let $\mathcal{C}$ be the original stabilizer code with logical operator $L$ and let the deformed code have at least one logical operator. Given $R \leq K \cdot (\log_2 |V|)^2$ layers, we have:
\begin{enumerate}
\item The distance is preserved: $d \leq d^*$, where $d$ is the distance of the original code and $d^*$ is the distance of the deformed code.
\item The edge overhead is bounded: $2|E| \leq \Delta \cdot |V|$.
\end{enumerate}
\end{theorem}

\begin{proof}
\leanok
\uses{thm:SpaceDistance.deformed_distance_ge_d_sufficient_expansion, thm:WorstCaseGraphConstruction.construction_satisfies_all_desiderata}
The distance bound $d \leq d^*$ follows directly from \texttt{deformed\_distance\_ge\_d\_sufficient\_expansion} applied to $G$, the cycles, checks, the original code, the identification of check indices, the deformed code data, the cycle parity condition, the commutativity of checks, connectivity, the vertex count bound, the exactness conditions, the sufficient expansion, the logical operator hypothesis, and the existence of deformed logicals. The edge overhead bound follows from \texttt{constant\_degree\_bounds\_edges}.
\end{proof}

\section*{Main Corollary}

\begin{theorem}[Corollary 1: Worst-Case Overhead]
\label{thm:WorstCaseOverhead.worst_case_overhead}
\lean{QEC1.WorstCaseOverhead.worst_case_overhead}
\leanok
\uses{thm:SpaceDistance.deformed_distance_ge_d_sufficient_expansion, thm:WorstCaseGraphConstruction.construction_satisfies_all_desiderata}
For a qLDPC stabilizer code with a Pauli logical operator $L$ of weight $W$, let $G$ be a constant-degree $\Delta$ connected graph on $V$ with $|V| = W$ (support of $L$), sufficient expansion $h(G) \geq 1$, $Z$-support matching edges, a low-weight cycle basis with cycles of weight $\leq 4$, and the cycle rank property $|C| + |V| = |E| + 1$. Given a K\"onig coloring with $M$ layers and the Freedman--Hastings bound $M \leq K \cdot (\log_2 W)^2$, and a per-layer parameter $c > 0$, the following all hold:
\begin{enumerate}
\item All desiderata are satisfied: $\operatorname{AllDesiderataSatisfied}(G, \text{cycles}, \text{checks}, 1, 4)$.
\item There exist $R$ layers and a coloring $f$ with layer cycle degree bound $c$ and $R \leq K \cdot (\log_2 W)^2 / c$.
\item The edge overhead is $O(W)$: $2|E| \leq \Delta \cdot W$.
\item The distance is preserved: $d \leq d^*$.
\item Gauss law check weights are bounded: $\operatorname{wt}(A_v) \leq 1 + \Delta$ for all $v$.
\item Flux check weights are bounded: $\operatorname{wt}(B_p) \leq 4$ for all $p$.
\item The vertex overhead is bounded: for all $R \leq K \cdot (\log_2 W)^2/c$, we have $(R+1) \cdot W \leq (K \cdot (\log_2 W)^2/c + 1) \cdot W$.
\end{enumerate}
\end{theorem}

\begin{proof}
\leanok
\uses{thm:SpaceDistance.deformed_distance_ge_d_sufficient_expansion, thm:WorstCaseGraphConstruction.construction_satisfies_all_desiderata}
We prove each of the seven conclusions:
\begin{enumerate}
\item All desiderata follow from \texttt{construction\_satisfies\_all\_desiderata} applied to $G$, the cycles, checks, $\Delta$, the $Z$-support adjacency, the sufficient expansion, the low-weight cycle basis, and the constant degree.
\item The layer count bound follows from \texttt{decongestion\_log\_squared} applied to $G$, $\Delta$, the constant degree, connectivity, vertex count bound, positivity of the expansion constant, the cycles, $c$, the cycle rank property, and the K\"onig-Freedman-Hastings bound inputs.
\item The edge overhead bound $2|E| \leq \Delta \cdot |V|$ follows from \texttt{constant\_degree\_bounds\_edges}.
\item The distance preservation $d \leq d^*$ follows from \texttt{deformed\_distance\_ge\_d\_sufficient\_expansion} applied with the original code, the identification of check indices, the deformed code data, the cycle parity condition, the commutativity of checks, connectivity, the vertex count bound, the exactness conditions, the sufficient expansion, the logical operator hypothesis, and the existence of deformed logicals.
\item The Gauss law check weight bound follows from \texttt{gaussLaw\_checks\_weight\_bounded} for each vertex $v$.
\item The flux check weight bound follows from \texttt{flux\_checks\_weight\_bounded} for each cycle $p$.
\item For the vertex overhead, let $R \leq K \cdot (\log_2 W)^2/c$. We apply monotonicity of multiplication on the right by $W$, and the bound $R + 1 \leq K \cdot (\log_2 W)^2/c + 1$ follows by integer arithmetic.
\end{enumerate}
\end{proof}

\begin{theorem}[Qubit Overhead is $O(W \log^2 W)$]
\label{thm:WorstCaseOverhead.qubit_overhead_is_Wlog2W}
\lean{QEC1.WorstCaseOverhead.qubit_overhead_is_Wlog2W}
\leanok
\uses{thm:WorstCaseGraphConstruction.construction_satisfies_all_desiderata}
Let $W, \Delta, K, c > 0$ be natural numbers, and let $R \leq K \cdot (\log_2 W)^2 / c$. If $2|E| \leq \Delta \cdot (R+1) \cdot W$, then
\[
|E| \leq \frac{\Delta \cdot (K \cdot (\log_2 W)^2/c + 1) \cdot W}{2}.
\]
\end{theorem}

\begin{proof}
\leanok
\uses{thm:WorstCaseGraphConstruction.construction_satisfies_all_desiderata}
We first establish that $R + 1 \leq K \cdot (\log_2 W)^2/c + 1$ by integer arithmetic from $R \leq K \cdot (\log_2 W)^2/c$. Then $\Delta \cdot (R+1) \cdot W \leq \Delta \cdot (K \cdot (\log_2 W)^2/c + 1) \cdot W$ by monotonicity of multiplication on the left by $\Delta$ and on the right by $W$. The result follows by integer arithmetic combining with the hypothesis $2|E| \leq \Delta \cdot (R+1) \cdot W$.
\end{proof}

\begin{theorem}[Codespace Dimension with Overhead]
\label{thm:WorstCaseOverhead.codespace_dimension_with_overhead}
\lean{QEC1.WorstCaseOverhead.codespace_dimension_with_overhead}
\leanok
\uses{thm:WorstCaseGraphConstruction.construction_satisfies_all_desiderata}
Let $G$ be a constant-degree $\Delta$ graph with cycle rank property, and let the original code have parameters $[\![n,k,d]\!]$ with $k \geq 1$. Given $R \leq K \cdot (\log_2 n)^2$ layers, the deformed code satisfies:
\begin{enumerate}
\item The number of encoded qubits decreases by one: $n' - m' = k - 1$, where $n'$ is the number of qubits and $m'$ is the number of checks of the deformed code.
\item The edge overhead is bounded: $2|E| \leq \Delta \cdot |V|$.
\end{enumerate}
\end{theorem}

\begin{proof}
\leanok
\uses{thm:WorstCaseGraphConstruction.construction_satisfies_all_desiderata}
The codespace dimension change follows directly from \texttt{codespace\_dimension\_change\_after\_gauging} applied to $G$, the cycles, checks, the deformed code data, the cycle parity condition, the commutativity of checks, $n$, $k$, the cardinality identification, the dimension formula, $k \geq 1$, and the cycle rank property. The edge overhead bound follows from \texttt{constant\_degree\_bounds\_edges}.
\end{proof}

\section*{Concrete Overhead Characterization}

\begin{theorem}[Overhead Ratio is $O(\log^2 W)$]
\label{thm:WorstCaseOverhead.overhead_ratio_is_log2W}
\lean{QEC1.WorstCaseOverhead.overhead_ratio_is_log2W}
\leanok
\uses{thm:WorstCaseGraphConstruction.construction_satisfies_all_desiderata}
Let $W, K, c > 0$ be natural numbers and $R \leq K \cdot (\log_2 W)^2/c$. Then the overhead ratio satisfies $R + 1 \leq K \cdot (\log_2 W)^2/c + 1$.
\end{theorem}

\begin{proof}
\leanok
\uses{thm:WorstCaseGraphConstruction.construction_satisfies_all_desiderata}
This follows by integer arithmetic from $R \leq K \cdot (\log_2 W)^2/c$.
\end{proof}

\begin{theorem}[Total Qubit Count]
\label{thm:WorstCaseOverhead.total_qubit_count}
\lean{QEC1.WorstCaseOverhead.total_qubit_count}
\leanok
\uses{thm:WorstCaseGraphConstruction.construction_satisfies_all_desiderata}
Let $G$ be a constant-degree $\Delta$ graph on $V$. Then the total qubit count satisfies
\[
2 \cdot (|V| + |E|) \leq (2 + \Delta) \cdot |V|.
\]
\end{theorem}

\begin{proof}
\leanok
\uses{thm:WorstCaseGraphConstruction.construction_satisfies_all_desiderata}
From \texttt{constant\_degree\_bounds\_edges}, we have $2|E| \leq \Delta \cdot |V|$. Then $2(|V| + |E|) = 2|V| + 2|E| \leq 2|V| + \Delta \cdot |V| = (2 + \Delta) \cdot |V|$, which follows by nonlinear arithmetic using the distributivity $(2 + \Delta) \cdot |V| = 2 \cdot |V| + \Delta \cdot |V|$.
\end{proof}

\begin{theorem}[Overhead Independent of $n$]
\label{thm:WorstCaseOverhead.overhead_independent_of_n}
\lean{QEC1.WorstCaseOverhead.overhead_independent_of_n}
\leanok
\uses{thm:WorstCaseGraphConstruction.construction_satisfies_all_desiderata}
Let $W, \Delta, K, c > 0$ be natural numbers and $R \leq K \cdot (\log_2 W)^2/c$. Then the overhead bound
\[
(R+1) \cdot W \leq (K \cdot (\log_2 W)^2/c + 1) \cdot W
\]
depends on $W$, $\Delta$, $K$, $c$ but \emph{not} on the code size $n$.
\end{theorem}

\begin{proof}
\leanok
\uses{thm:WorstCaseGraphConstruction.construction_satisfies_all_desiderata}
We apply monotonicity of multiplication on the right by $W$. The bound $R + 1 \leq K \cdot (\log_2 W)^2/c + 1$ follows by integer arithmetic from the hypothesis $R \leq K \cdot (\log_2 W)^2/c$.
\end{proof}

\begin{theorem}[Constant Depends on Code Parameters]
\label{thm:WorstCaseOverhead.constant_depends_on_code_params}
\lean{QEC1.WorstCaseOverhead.constant_depends_on_code_params}
\leanok
\uses{thm:WorstCaseGraphConstruction.construction_satisfies_all_desiderata}
Let $W, \Delta, K, c > 0$ be natural numbers and $R \leq K \cdot (\log_2 W)^2/c$. Then the multiplier satisfies
\[
\Delta \cdot (R+1) \cdot W \leq \Delta \cdot (K \cdot (\log_2 W)^2/c + 1) \cdot W.
\]
The constant $\Delta \cdot (K/c + 1)$ depends on the code parameters $\Delta$ (maximum degree of $G$, a function of check weight $w$ and qubit degree), $K$ (the Freedman--Hastings constant), and $c$ (per-layer cycle-degree bound), but is independent of $W$ and $n$.
\end{theorem}

\begin{proof}
\leanok
\uses{thm:WorstCaseGraphConstruction.construction_satisfies_all_desiderata}
We apply monotonicity of multiplication on the left by $\Delta$ and then on the right by $W$. The inner bound $R + 1 \leq K \cdot (\log_2 W)^2/c + 1$ follows by integer arithmetic from the hypothesis.
\end{proof}

\begin{theorem}[Deformed Code Total Qubits]
\label{thm:WorstCaseOverhead.deformed_code_total_qubits}
\lean{QEC1.WorstCaseOverhead.deformed_code_total_qubits}
\leanok
\uses{thm:WorstCaseGraphConstruction.construction_satisfies_all_desiderata}
Let $G$ be a constant-degree $\Delta$ graph on $V$ with cycles and checks, and let the deformed stabilizer code be constructed from this data. Then the number of qubits of the deformed code satisfies
\[
n_{\text{deformed}} \leq |V| + \frac{\Delta \cdot |V|}{2}.
\]
\end{theorem}

\begin{proof}
\leanok
\uses{thm:WorstCaseGraphConstruction.construction_satisfies_all_desiderata}
Rewriting the number of qubits of the deformed code using \texttt{deformedStabilizerCode\_numQubits}, which gives $n_{\text{deformed}} = |V| + |E|$, and applying the edge overhead bound $2|E| \leq \Delta \cdot |V|$ from \texttt{constant\_degree\_bounds\_edges}, we obtain $|E| \leq \Delta \cdot |V|/2$ and thus $n_{\text{deformed}} \leq |V| + \Delta \cdot |V|/2$ by integer arithmetic.
\end{proof}

%--- Rem_17: HypergraphGeneralization ---
\chapter{Rem 17: Hypergraph Generalization}

This remark shows that the gauging measurement procedure can be generalized by replacing the graph $G$ with a hypergraph $H = (V, E)$, where each hyperedge $e \in E$ is a nonempty subset of vertices. One introduces one auxiliary qubit per hyperedge, defines Gauss's law operators $A_v = X_v \prod_{e \ni v} X_e$, and a boundary map $\partial : \mathbb{Z}_2^E \to \mathbb{Z}_2^V$ with $(\partial \gamma)_v = \sum_{e \ni v} \gamma_e$. The key result is that $\gamma \in \ker(\partial)$ if and only if the corresponding pure-$Z$ hyperedge operator commutes with all $A_v$.

\begin{definition}[Hyper-Incident Edges]
\label{def:HypergraphGeneralization.hyperIncidentEdges}
\lean{QEC1.HypergraphGeneralization.hyperIncidentEdges}
\leanok
\uses{def:PauliOp}
Given a hypergraph incidence relation $\operatorname{incident} : V \to E \to \operatorname{Prop}$ and a vertex $v \in V$, the set of hyperedges incident to $v$ is
\[
  \operatorname{hyperIncidentEdges}(v) = \{ e \in E \mid \operatorname{incident}(v, e) \}.
\]
\end{definition}

\begin{definition}[Hyperedge Vertices]
\label{def:HypergraphGeneralization.hyperedgeVertices}
\lean{QEC1.HypergraphGeneralization.hyperedgeVertices}
\leanok
\uses{def:PauliOp}
Given a hyperedge $e \in E$, the set of vertices contained in $e$ is
\[
  \operatorname{hyperedgeVertices}(e) = \{ v \in V \mid \operatorname{incident}(v, e) \}.
\]
\end{definition}

\begin{definition}[Hypergraph Boundary Map]
\label{def:HypergraphGeneralization.hyperBoundaryMap}
\lean{QEC1.HypergraphGeneralization.hyperBoundaryMap}
\leanok
\uses{def:HypergraphGeneralization.hyperIncidentEdges}
The hypergraph boundary map $\partial : \mathbb{Z}_2^E \to \mathbb{Z}_2^V$ is the $\mathbb{Z}_2$-linear map defined by
\[
  (\partial \gamma)_v = \sum_{e \in E} \begin{cases} \gamma_e & \text{if } \operatorname{incident}(v, e), \\ 0 & \text{otherwise,} \end{cases}
\]
for $\gamma \in \mathbb{Z}_2^E$ and $v \in V$. This generalizes the graph boundary map: for graphs, each edge is incident to exactly two vertices; for hypergraphs, a hyperedge can be incident to any nonempty set of vertices.
\end{definition}

\begin{definition}[Hypergraph Coboundary Map]
\label{def:HypergraphGeneralization.hyperCoboundaryMap}
\lean{QEC1.HypergraphGeneralization.hyperCoboundaryMap}
\leanok
\uses{def:HypergraphGeneralization.hyperBoundaryMap}
The hypergraph coboundary map $\delta : \mathbb{Z}_2^V \to \mathbb{Z}_2^E$ is the transpose of $\partial$. For $f \in \mathbb{Z}_2^V$,
\[
  (\delta f)_e = \sum_{v \in V} \begin{cases} f_v & \text{if } \operatorname{incident}(v, e), \\ 0 & \text{otherwise.} \end{cases}
\]
\end{definition}

\begin{theorem}[Coboundary Is Transpose of Boundary]
\label{thm:HypergraphGeneralization.hyperCoboundaryMap_eq_transpose}
\lean{QEC1.HypergraphGeneralization.hyperCoboundaryMap_eq_transpose}
\leanok
\uses{def:HypergraphGeneralization.hyperCoboundaryMap, def:HypergraphGeneralization.hyperBoundaryMap}
For all $f \in \mathbb{Z}_2^V$ and $\gamma \in \mathbb{Z}_2^E$,
\[
  \sum_{e \in E} (\delta f)_e \cdot \gamma_e = \sum_{v \in V} f_v \cdot (\partial \gamma)_v.
\]
\end{theorem}

\begin{proof}
\leanok
\uses{def:HypergraphGeneralization.hyperCoboundaryMap, def:HypergraphGeneralization.hyperBoundaryMap}
Expanding the definitions, the left-hand side becomes $\sum_e \bigl(\sum_v [\operatorname{incident}(v,e)] \cdot f_v\bigr) \cdot \gamma_e$ and the right-hand side becomes $\sum_v f_v \cdot \bigl(\sum_e [\operatorname{incident}(v,e)] \cdot \gamma_e\bigr)$. Distributing the sums, both sides equal $\sum_v \sum_e [\operatorname{incident}(v,e)] \cdot f_v \cdot \gamma_e$ after exchanging the order of summation and applying commutativity of multiplication in $\mathbb{Z}_2$. The terms where $\operatorname{incident}(v,e)$ is false contribute zero, so the sums agree.
\end{proof}

\begin{definition}[Hypergraph Gauss's Law Operator]
\label{def:HypergraphGeneralization.hyperGaussLawOp}
\lean{QEC1.HypergraphGeneralization.hyperGaussLawOp}
\leanok
\uses{def:PauliOp}
The hypergraph Gauss's law operator $A_v$ on the extended system $V \oplus E$ is defined as the Pauli operator with
\[
  A_v = X_v \prod_{e \ni v} X_e,
\]
i.e., its $X$-vector is $(\operatorname{xVec})_q = \begin{cases} 1 & \text{if } q = \operatorname{inl}(v), \\ 1 & \text{if } q = \operatorname{inr}(e) \text{ and } \operatorname{incident}(v,e), \\ 0 & \text{otherwise,} \end{cases}$ and its $Z$-vector is identically zero.
\end{definition}

\begin{theorem}[Hypergraph Gauss's Law Operators Are Pure $X$]
\label{thm:HypergraphGeneralization.hyperGaussLawOp_pure_X}
\lean{QEC1.HypergraphGeneralization.hyperGaussLawOp_pure_X}
\leanok
\uses{def:HypergraphGeneralization.hyperGaussLawOp, def:PauliOp}
For every vertex $v \in V$, the operator $A_v$ is pure $X$-type: its $Z$-vector is identically zero.
\end{theorem}

\begin{proof}
\leanok
\uses{def:HypergraphGeneralization.hyperGaussLawOp}
By definition, the $Z$-vector of $A_v$ is set to $0$, so for any qubit $q \in V \oplus E$, $(\operatorname{zVec}(A_v))_q = 0$. This holds by simplification.
\end{proof}

\begin{theorem}[Hypergraph Gauss's Law Operators Commute]
\label{thm:HypergraphGeneralization.hyperGaussLaw_commute}
\lean{QEC1.HypergraphGeneralization.hyperGaussLaw_commute}
\leanok
\uses{def:HypergraphGeneralization.hyperGaussLawOp, def:PauliOp.PauliCommute, def:PauliOp.symplecticInner}
For all vertices $v, w \in V$, the operators $A_v$ and $A_w$ commute:
\[
  [A_v, A_w] = 0.
\]
\end{theorem}

\begin{proof}
\leanok
\uses{def:HypergraphGeneralization.hyperGaussLawOp, def:PauliOp.PauliCommute, def:PauliOp.symplecticInner}
Commutation of Pauli operators is determined by the symplectic inner product $\langle A_v, A_w \rangle = \sum_q (\operatorname{xVec}(A_v)_q \cdot \operatorname{zVec}(A_w)_q + \operatorname{zVec}(A_v)_q \cdot \operatorname{xVec}(A_w)_q)$. Since both $A_v$ and $A_w$ have $\operatorname{zVec} = 0$, every term in the sum is zero. Hence $\langle A_v, A_w \rangle = 0$, which means they commute.
\end{proof}

\begin{definition}[$X$-Operator from Boundary]
\label{def:HypergraphGeneralization.xOpFromBoundary}
\lean{QEC1.HypergraphGeneralization.xOpFromBoundary}
\leanok
\uses{def:PauliOp, def:HypergraphGeneralization.hyperBoundaryMap}
Given an edge vector $\gamma \in \mathbb{Z}_2^E$, the corresponding $X$-type operator on vertex qubits $V$ is
\[
  \prod_{v \in V} X_v^{(\partial \gamma)_v},
\]
defined as the Pauli operator with $\operatorname{xVec} = \partial \gamma$ and $\operatorname{zVec} = 0$.
\end{definition}

\begin{theorem}[Gauss's Law Product on Vertex Qubits]
\label{thm:HypergraphGeneralization.hyperGaussLaw_product_xVec_vertex}
\lean{QEC1.HypergraphGeneralization.hyperGaussLaw_product_xVec_vertex}
\leanok
\uses{def:HypergraphGeneralization.hyperGaussLawOp}
On vertex qubits, the sum of $X$-vectors of all Gauss operators gives the all-ones vector: for each vertex $w$,
\[
  \sum_{v \in V} (\operatorname{xVec}(A_v))_{\operatorname{inl}(w)} = 1.
\]
\end{theorem}

\begin{proof}
\leanok
\uses{def:HypergraphGeneralization.hyperGaussLawOp}
The $X$-vector of $A_v$ at position $\operatorname{inl}(w)$ equals $1$ if $w = v$ and $0$ otherwise. Therefore the sum over all $v \in V$ has exactly one nonzero term (when $v = w$), giving $1$. This is verified using the singleton sum lemma.
\end{proof}

\begin{theorem}[Gauss's Law Product on Edge Qubits]
\label{thm:HypergraphGeneralization.hyperGaussLaw_product_xVec_edge}
\lean{QEC1.HypergraphGeneralization.hyperGaussLaw_product_xVec_edge}
\leanok
\uses{def:HypergraphGeneralization.hyperGaussLawOp, def:HypergraphGeneralization.hyperedgeVertices}
On edge qubits, the sum of $X$-vectors of all Gauss operators gives the cardinality of the hyperedge modulo 2: for each hyperedge $e$,
\[
  \sum_{v \in V} (\operatorname{xVec}(A_v))_{\operatorname{inr}(e)} = |\operatorname{hyperedgeVertices}(e)| \pmod{2}.
\]
\end{theorem}

\begin{proof}
\leanok
\uses{def:HypergraphGeneralization.hyperGaussLawOp, def:HypergraphGeneralization.hyperedgeVertices}
The $X$-vector of $A_v$ at position $\operatorname{inr}(e)$ equals $1$ if $\operatorname{incident}(v,e)$ and $0$ otherwise. The sum $\sum_v [\operatorname{incident}(v,e)]$ counts the number of vertices in hyperedge $e$, which equals $|\operatorname{hyperedgeVertices}(e)|$ in $\mathbb{Z}_2$. This follows by rewriting the sum as a Boolean indicator sum and the definition of $\operatorname{hyperedgeVertices}$.
\end{proof}

\begin{theorem}[Kernel of Boundary Map Characterization]
\label{thm:HypergraphGeneralization.mem_ker_hyperBoundary_iff}
\lean{QEC1.HypergraphGeneralization.mem_ker_hyperBoundary_iff}
\leanok
\uses{def:HypergraphGeneralization.hyperBoundaryMap}
An edge vector $\gamma \in \mathbb{Z}_2^E$ lies in $\ker(\partial)$ if and only if $(\partial \gamma)_v = 0$ for all $v \in V$.
\end{theorem}

\begin{proof}
\leanok
\uses{def:HypergraphGeneralization.hyperBoundaryMap}
By definition, $\gamma \in \ker(\partial)$ means $\partial \gamma = 0$ as a function $V \to \mathbb{Z}_2$. By function extensionality, this is equivalent to $(\partial \gamma)_v = 0$ for all $v$.
\end{proof}

\begin{theorem}[Kernel Element Gives Identity on $V$]
\label{thm:HypergraphGeneralization.ker_boundary_gives_identity_on_V}
\lean{QEC1.HypergraphGeneralization.ker_boundary_gives_identity_on_V}
\leanok
\uses{def:HypergraphGeneralization.xOpFromBoundary, def:HypergraphGeneralization.hyperBoundaryMap}
If $\gamma \in \ker(\partial)$, then the $X$-type operator on $V$ from $\gamma$ is the identity: $\operatorname{xOpFromBoundary}(\gamma) = \mathbf{1}$.
\end{theorem}

\begin{proof}
\leanok
\uses{def:HypergraphGeneralization.xOpFromBoundary, def:HypergraphGeneralization.hyperBoundaryMap}
Since $\gamma \in \ker(\partial)$, we have $\partial \gamma = 0$. The $X$-vector of $\operatorname{xOpFromBoundary}(\gamma)$ is $\partial \gamma = 0$, and the $Z$-vector is $0$ by definition. By extensionality, this equals the identity operator $\mathbf{1}$.
\end{proof}

\begin{definition}[Hypergraph Gauss Subset Product]
\label{def:HypergraphGeneralization.hyperGaussSubsetProduct}
\lean{QEC1.HypergraphGeneralization.hyperGaussSubsetProduct}
\leanok
\uses{def:PauliOp, def:HypergraphGeneralization.hyperGaussLawOp}
The Gauss subset product for a vector $c \in \mathbb{Z}_2^V$ is the product $\prod_{v : c_v = 1} A_v$, defined as the Pauli operator on $V \oplus E$ with
\[
  \operatorname{xVec}(q) = \begin{cases} c_v & \text{if } q = \operatorname{inl}(v), \\ \sum_{v \in V} [\operatorname{incident}(v,e)] \cdot c_v & \text{if } q = \operatorname{inr}(e), \end{cases}
\]
and $\operatorname{zVec} = 0$. On vertex qubits the $X$-vector is $c$; on edge qubits it equals the coboundary $\delta c$.
\end{definition}

\begin{theorem}[Gauss Subset Product Is Additive]
\label{thm:HypergraphGeneralization.hyperGaussSubsetProduct_add}
\lean{QEC1.HypergraphGeneralization.hyperGaussSubsetProduct_add}
\leanok
\uses{def:HypergraphGeneralization.hyperGaussSubsetProduct, def:PauliOp.mul}
For all $c_1, c_2 \in \mathbb{Z}_2^V$,
\[
  \operatorname{hyperGaussSubsetProduct}(c_1 + c_2) = \operatorname{hyperGaussSubsetProduct}(c_1) \cdot \operatorname{hyperGaussSubsetProduct}(c_2).
\]
\end{theorem}

\begin{proof}
\leanok
\uses{def:HypergraphGeneralization.hyperGaussSubsetProduct, def:PauliOp.mul}
By extensionality on each qubit $q$. For vertex qubits $q = \operatorname{inl}(v)$: both sides give $(c_1 + c_2)_v = (c_1)_v + (c_2)_v$ by the definition of Pauli multiplication (component-wise addition of $X$-vectors in $\mathbb{Z}_2$). For edge qubits $q = \operatorname{inr}(e)$: the left side gives $\sum_v [\operatorname{incident}(v,e)] \cdot (c_1 + c_2)_v$, which distributes as $\sum_v [\operatorname{incident}(v,e)] \cdot (c_1)_v + \sum_v [\operatorname{incident}(v,e)] \cdot (c_2)_v$ by linearity of summation; this matches the right side. For the $Z$-vector, both sides are zero since $0 + 0 = 0$.
\end{proof}

\begin{theorem}[Gauss Subset Product Edge Equals Coboundary]
\label{thm:HypergraphGeneralization.hyperGaussSubsetProduct_edge_eq_coboundary}
\lean{QEC1.HypergraphGeneralization.hyperGaussSubsetProduct_edge_eq_coboundary}
\leanok
\uses{def:HypergraphGeneralization.hyperGaussSubsetProduct, def:HypergraphGeneralization.hyperCoboundaryMap}
For all $c \in \mathbb{Z}_2^V$ and $e \in E$,
\[
  (\operatorname{xVec}(\operatorname{hyperGaussSubsetProduct}(c)))_{\operatorname{inr}(e)} = (\delta c)_e.
\]
\end{theorem}

\begin{proof}
\leanok
\uses{def:HypergraphGeneralization.hyperGaussSubsetProduct, def:HypergraphGeneralization.hyperCoboundaryMap}
By simplification: both sides equal $\sum_{v \in V} [\operatorname{incident}(v,e)] \cdot c_v$, which is the definition of the coboundary map applied to $c$ at $e$.
\end{proof}

\begin{definition}[Pure-$Z$ Hyperedge Operator]
\label{def:HypergraphGeneralization.pureZHyperedgeOp}
\lean{QEC1.HypergraphGeneralization.pureZHyperedgeOp}
\leanok
\uses{def:PauliOp}
Given an edge vector $\gamma \in \mathbb{Z}_2^E$, the pure-$Z$ hyperedge operator on $V \oplus E$ acts as $Z$ on hyperedge qubit $e$ if and only if $\gamma_e = 1$. Formally, $\operatorname{xVec} = 0$ and
\[
  \operatorname{zVec}(q) = \begin{cases} 0 & \text{if } q = \operatorname{inl}(v), \\ \gamma_e & \text{if } q = \operatorname{inr}(e). \end{cases}
\]
\end{definition}

\begin{theorem}[Symplectic Inner Product of Pure-$Z$ Operator with Gauss's Law]
\label{thm:HypergraphGeneralization.symplecticInner_pureZ_gaussLaw}
\lean{QEC1.HypergraphGeneralization.symplecticInner_pureZ_gaussLaw}
\leanok
\uses{def:HypergraphGeneralization.pureZHyperedgeOp, def:HypergraphGeneralization.hyperGaussLawOp, def:PauliOp.symplecticInner, def:HypergraphGeneralization.hyperBoundaryMap}
For all $\gamma \in \mathbb{Z}_2^E$ and $v \in V$,
\[
  \langle Z(\gamma),\, A_v \rangle = (\partial \gamma)_v,
\]
where $Z(\gamma)$ denotes the pure-$Z$ hyperedge operator and $\langle \cdot, \cdot \rangle$ is the symplectic inner product.
\end{theorem}

\begin{proof}
\leanok
\uses{def:HypergraphGeneralization.pureZHyperedgeOp, def:HypergraphGeneralization.hyperGaussLawOp, def:PauliOp.symplecticInner, def:HypergraphGeneralization.hyperBoundaryMap}
We expand the symplectic inner product $\langle Z(\gamma), A_v \rangle = \sum_{q \in V \oplus E} (\operatorname{xVec}(Z(\gamma))_q \cdot \operatorname{zVec}(A_v)_q + \operatorname{zVec}(Z(\gamma))_q \cdot \operatorname{xVec}(A_v)_q)$ and split the sum over $V$ and $E$ using $\operatorname{Fintype.sum\_sum\_type}$. For the sum over vertex qubits: $\operatorname{xVec}(Z(\gamma)) = 0$ and $\operatorname{zVec}(Z(\gamma))_{\operatorname{inl}(w)} = 0$ for all $w$, so every term vanishes and the vertex sum is $0$. For the sum over edge qubits: $\operatorname{xVec}(Z(\gamma)) = 0$, so the first product in each term is zero. The second product gives $\gamma_e \cdot [\operatorname{incident}(v,e)]$, which equals $[\operatorname{incident}(v,e)] \cdot \gamma_e$. Summing over all $e$ yields $\sum_e [\operatorname{incident}(v,e)] \cdot \gamma_e = (\partial \gamma)_v$.
\end{proof}

\begin{theorem}[Pure-$Z$ Operator Commutes with $A_v$ iff Boundary Vanishes]
\label{thm:HypergraphGeneralization.pureZHyperedgeOp_commutes_gaussLaw_iff}
\lean{QEC1.HypergraphGeneralization.pureZHyperedgeOp_commutes_gaussLaw_iff}
\leanok
\uses{def:HypergraphGeneralization.pureZHyperedgeOp, def:HypergraphGeneralization.hyperGaussLawOp, def:PauliOp.PauliCommute, def:HypergraphGeneralization.hyperBoundaryMap, def:PauliOp.symplecticInner}
For all $\gamma \in \mathbb{Z}_2^E$ and $v \in V$,
\[
  [Z(\gamma), A_v] = 0 \quad \Longleftrightarrow \quad (\partial \gamma)_v = 0.
\]
\end{theorem}

\begin{proof}
\leanok
\uses{thm:HypergraphGeneralization.symplecticInner_pureZ_gaussLaw, def:PauliOp.PauliCommute}
Commutation of Pauli operators is equivalent to their symplectic inner product being zero. By the previous theorem, $\langle Z(\gamma), A_v \rangle = (\partial \gamma)_v$, so $\operatorname{PauliCommute}(Z(\gamma), A_v) \Leftrightarrow (\partial \gamma)_v = 0$.
\end{proof}

\begin{theorem}[Kernel of $\partial$ Characterizes Commutation with All Gauss Operators]
\label{thm:HypergraphGeneralization.ker_boundary_iff_commutes_all_gaussLaw}
\lean{QEC1.HypergraphGeneralization.ker_boundary_iff_commutes_all_gaussLaw}
\leanok
\uses{def:HypergraphGeneralization.pureZHyperedgeOp, def:HypergraphGeneralization.hyperGaussLawOp, def:PauliOp.PauliCommute, def:HypergraphGeneralization.hyperBoundaryMap}
For all $\gamma \in \mathbb{Z}_2^E$,
\[
  \gamma \in \ker(\partial) \quad \Longleftrightarrow \quad \forall\, v \in V,\; [Z(\gamma), A_v] = 0.
\]
\end{theorem}

\begin{proof}
\leanok
\uses{thm:HypergraphGeneralization.mem_ker_hyperBoundary_iff, thm:HypergraphGeneralization.pureZHyperedgeOp_commutes_gaussLaw_iff}
We rewrite $\gamma \in \ker(\partial)$ as $(\partial \gamma)_v = 0$ for all $v$ using the kernel characterization. For each direction: if $(\partial \gamma)_v = 0$ for all $v$, then the commutation condition $(\partial \gamma)_v = 0$ holds for each $v$, so $Z(\gamma)$ commutes with each $A_v$. Conversely, if $Z(\gamma)$ commutes with each $A_v$, then $(\partial \gamma)_v = 0$ for each $v$ by the pointwise equivalence.
\end{proof}

\begin{theorem}[Commuting Pure-$Z$ Operators Correspond to Kernel Elements]
\label{thm:HypergraphGeneralization.commuting_pureZ_operators_eq_ker_image}
\lean{QEC1.HypergraphGeneralization.commuting_pureZ_operators_eq_ker_image}
\leanok
\uses{def:HypergraphGeneralization.pureZHyperedgeOp, def:HypergraphGeneralization.hyperGaussLawOp, def:PauliOp.PauliCommute, def:HypergraphGeneralization.hyperBoundaryMap, def:PauliOp}
If $P$ is a Pauli operator on $V \oplus E$ with $\operatorname{xVec}(P) = 0$, $\operatorname{zVec}(P)_{\operatorname{inl}(v)} = 0$ for all $v \in V$, and $[P, A_v] = 0$ for all $v \in V$, then the edge $Z$-vector $\gamma_e := \operatorname{zVec}(P)_{\operatorname{inr}(e)}$ lies in $\ker(\partial)$.
\end{theorem}

\begin{proof}
\leanok
\uses{thm:HypergraphGeneralization.ker_boundary_iff_commutes_all_gaussLaw, def:HypergraphGeneralization.pureZHyperedgeOp}
We first rewrite the goal using the characterization that $\gamma \in \ker(\partial)$ iff $Z(\gamma)$ commutes with all $A_v$. Let $v \in V$ be arbitrary. We show that $P = \operatorname{pureZHyperedgeOp}(\gamma)$ by extensionality: for $\operatorname{xVec}$, both sides are zero (on vertex qubits by hypothesis $\operatorname{xVec}(P) = 0$, and on edge qubits similarly); for $\operatorname{zVec}$, on vertex qubits both are zero (by hypothesis for $P$, by definition for $Z(\gamma)$), and on edge qubits both equal $\gamma_e$ by construction. Since $P = Z(\gamma)$, the commutation hypothesis $[P, A_v] = 0$ gives $[Z(\gamma), A_v] = 0$.
\end{proof}

\begin{definition}[Graph-Like Hypergraph]
\label{def:HypergraphGeneralization.IsGraphLike}
\lean{QEC1.HypergraphGeneralization.IsGraphLike}
\leanok
\uses{def:HypergraphGeneralization.hyperedgeVertices}
A hypergraph is \emph{graph-like} if every hyperedge has exactly $2$ vertices:
\[
  \operatorname{IsGraphLike} \;\Longleftrightarrow\; \forall\, e \in E,\; |\operatorname{hyperedgeVertices}(e)| = 2.
\]
\end{definition}

\begin{theorem}[Graph-Like Boundary of Single Edge Sums to Zero]
\label{thm:HypergraphGeneralization.graphLike_boundary_single_sum}
\lean{QEC1.HypergraphGeneralization.graphLike_boundary_single_sum}
\leanok
\uses{def:HypergraphGeneralization.IsGraphLike, def:HypergraphGeneralization.hyperBoundaryMap, def:HypergraphGeneralization.hyperedgeVertices}
For a graph-like hypergraph, the boundary of a single edge indicator sums to zero over all vertices: for any $e_0 \in E$,
\[
  \sum_{v \in V} (\partial \mathbf{1}_{e_0})_v = 0.
\]
This holds because each edge has exactly $2$ endpoints and $2 \equiv 0 \pmod{2}$.
\end{theorem}

\begin{proof}
\leanok
\uses{def:HypergraphGeneralization.IsGraphLike, def:HypergraphGeneralization.hyperBoundaryMap, def:HypergraphGeneralization.hyperedgeVertices}
By the single-edge boundary formula, $(\partial \mathbf{1}_{e_0})_v = [\operatorname{incident}(v, e_0)]$, so $\sum_v (\partial \mathbf{1}_{e_0})_v = \sum_v [\operatorname{incident}(v, e_0)]$. This sum counts the number of vertices in hyperedge $e_0$, which is $|\operatorname{hyperedgeVertices}(e_0)|$. Rewriting the sum as a Boolean indicator sum and using the filter characterization, we get $|\operatorname{univ}.\operatorname{filter}(\operatorname{incident}(\cdot, e_0))| = |\operatorname{hyperedgeVertices}(e_0)| = 2$ by the graph-like hypothesis. Since $2 = 0$ in $\mathbb{Z}_2$ (verified by computation), the sum equals $0$.
\end{proof}

\begin{theorem}[CSS Initialization as Special Case]
\label{thm:HypergraphGeneralization.cssInit_boundary_eq}
\lean{QEC1.HypergraphGeneralization.cssInit_boundary_eq}
\leanok
\uses{def:HypergraphGeneralization.hyperBoundaryMap}
For CSS code initialization with physical qubits $Q$, $X$-type check index set $I$, and check supports $\operatorname{xCheckSupport} : I \to \operatorname{Finset}(Q)$, the hypergraph boundary map with incidence $\operatorname{incident}(q, i) \Leftrightarrow q \in \operatorname{xCheckSupport}(i)$ satisfies
\[
  (\partial \gamma)_q = \sum_{i \in I} [\,q \in \operatorname{xCheckSupport}(i)\,] \cdot \gamma_i
\]
for all $\gamma \in \mathbb{Z}_2^I$ and $q \in Q$.
\end{theorem}

\begin{proof}
\leanok
\uses{def:HypergraphGeneralization.hyperBoundaryMap}
This follows directly by simplification from the definition of $\operatorname{hyperBoundaryMap}$: substituting the incidence relation $\operatorname{incident}(v, i) := v \in \operatorname{xCheckSupport}(i)$ into the formula $(\partial \gamma)_q = \sum_i [\operatorname{incident}(q, i)] \cdot \gamma_i$ yields the stated identity.
\end{proof}

\begin{theorem}[Hypergraph Generalization Summary]
\label{thm:HypergraphGeneralization.hypergraph_generalization_summary}
\lean{QEC1.HypergraphGeneralization.hypergraph_generalization_summary}
\leanok
\uses{def:HypergraphGeneralization.hyperGaussLawOp, def:HypergraphGeneralization.hyperBoundaryMap, def:HypergraphGeneralization.hyperCoboundaryMap, def:HypergraphGeneralization.hyperGaussSubsetProduct, def:HypergraphGeneralization.pureZHyperedgeOp, def:PauliOp.PauliCommute, def:PauliOp.symplecticInner}
The hypergraph generalization of the gauging framework satisfies the following four properties simultaneously:
\begin{enumerate}
  \item All Gauss operators mutually commute: $\forall\, v, w \in V$, $[A_v, A_w] = 0$.
  \item Pure-$Z$ edge operators commute with all $A_v$ iff $\gamma \in \ker(\partial)$: $\forall\, \gamma \in \mathbb{Z}_2^E$, $\gamma \in \ker(\partial) \Leftrightarrow \forall\, v \in V,\; [Z(\gamma), A_v] = 0$.
  \item The coboundary $\delta$ is the transpose of $\partial$: $\forall\, f \in \mathbb{Z}_2^V,\, \gamma \in \mathbb{Z}_2^E$, $\sum_e (\delta f)_e \gamma_e = \sum_v f_v (\partial \gamma)_v$.
  \item The Gauss subset product for the zero vector gives the identity: $\operatorname{hyperGaussSubsetProduct}(0) = \mathbf{1}$.
\end{enumerate}
\end{theorem}

\begin{proof}
\leanok
\uses{thm:HypergraphGeneralization.hyperGaussLaw_commute, thm:HypergraphGeneralization.ker_boundary_iff_commutes_all_gaussLaw, thm:HypergraphGeneralization.hyperCoboundaryMap_eq_transpose}
This is the conjunction of the four results proved above:
\begin{enumerate}
  \item follows from \texttt{hyperGaussLaw\_commute},
  \item follows from \texttt{ker\_boundary\_iff\_commutes\_all\_gaussLaw},
  \item follows from \texttt{hyperCoboundaryMap\_eq\_transpose},
  \item follows from \texttt{hyperGaussSubsetProduct\_zero}.
\end{enumerate}
\end{proof}

%--- Rem_18: RelationToLatticeSurgery ---

\chapter{Rem 18: Relation to Lattice Surgery}

The gauging measurement procedure generalizes surface code lattice surgery.
We formalize three scenarios: standard lattice surgery via a ladder graph,
long-range lattice surgery via a dummy grid graph, and LDPC generalization
via a bridge graph.

%% ============================================================
%% Part 1: Ladder Graph Construction
%% ============================================================

\begin{definition}[Ladder Adjacency]
\label{def:RelationToLatticeSurgery.ladderAdj}
\lean{QEC1.RelationToLatticeSurgery.ladderAdj}
\leanok
\uses{def:PauliOp}
The adjacency relation for the ladder graph on $2n$ vertices.
The vertex set is $\mathrm{Bool} \times \mathrm{Fin}\, n$.
Two vertices $p = (b_1, i)$ and $q = (b_2, j)$ are adjacent if $p \neq q$ and either:
\begin{enumerate}
  \item \textbf{Rung}: $b_1 \neq b_2$ and $i = j$ (same position, different boundary), or
  \item \textbf{Rail}: $b_1 = b_2$ and $|i - j| = 1$ (same boundary, consecutive positions).
\end{enumerate}
\end{definition}

\begin{lemma}[Ladder Adjacency is Symmetric]
\label{lem:RelationToLatticeSurgery.ladderAdj_symm}
\lean{QEC1.RelationToLatticeSurgery.ladderAdj_symm}
\leanok
\uses{def:RelationToLatticeSurgery.ladderAdj}
For all $p, q \in \mathrm{Bool} \times \mathrm{Fin}\, n$, if $\operatorname{ladderAdj}(p, q)$ then $\operatorname{ladderAdj}(q, p)$.
\end{lemma}

\begin{proof}
\leanok
\uses{def:RelationToLatticeSurgery.ladderAdj}
Let $\langle h_{\neq}, h_{\lor}\rangle$ be the hypothesis. We have $q \neq p$ by symmetry of $\neq$. We case-split on $h_{\lor}$:
\begin{itemize}
  \item \textbf{Rung case}: We have $b_1 \neq b_2$ and $i = j$. By symmetry, $b_2 \neq b_1$ and $j = i$.
  \item \textbf{Rail case}: We have $b_1 = b_2$ and ($i + 1 = j$ or $j + 1 = i$). By symmetry, $b_2 = b_1$ and the disjunction is swapped.
\end{itemize}
\end{proof}

\begin{lemma}[Ladder Adjacency is Irreflexive]
\label{lem:RelationToLatticeSurgery.ladderAdj_irrefl}
\lean{QEC1.RelationToLatticeSurgery.ladderAdj_irrefl}
\leanok
\uses{def:RelationToLatticeSurgery.ladderAdj}
For all $p \in \mathrm{Bool} \times \mathrm{Fin}\, n$, $\neg\operatorname{ladderAdj}(p, p)$.
\end{lemma}

\begin{proof}
\leanok
\uses{def:RelationToLatticeSurgery.ladderAdj}
Suppose $\operatorname{ladderAdj}(p, p)$. Then $p \neq p$, which is a contradiction.
\end{proof}

\begin{definition}[Ladder Graph]
\label{def:RelationToLatticeSurgery.LadderGraph}
\lean{QEC1.RelationToLatticeSurgery.LadderGraph}
\leanok
\uses{def:RelationToLatticeSurgery.ladderAdj}
The ladder graph connecting two boundary qubit sets of size $n$.
It is a simple graph on the vertex set $\mathrm{Bool} \times \mathrm{Fin}\, n$ ($2n$ vertices total),
with adjacency given by $\operatorname{ladderAdj}$: edges consist of rungs (across boundaries) and rails (within each boundary).
\end{definition}

\begin{lemma}[Ladder Graph Has $2n$ Vertices]
\label{lem:RelationToLatticeSurgery.ladderGraph_card}
\lean{QEC1.RelationToLatticeSurgery.ladderGraph_card}
\leanok
\uses{def:RelationToLatticeSurgery.LadderGraph}
$|\mathrm{Bool} \times \mathrm{Fin}\, n| = 2n$.
\end{lemma}

\begin{proof}
\leanok
\uses{def:RelationToLatticeSurgery.LadderGraph}
By simplification using the cardinality of product types, $|\mathrm{Bool}| = 2$ and $|\mathrm{Fin}\, n| = n$.
\end{proof}

\begin{lemma}[Rungs Are Edges]
\label{lem:RelationToLatticeSurgery.rung_is_edge}
\lean{QEC1.RelationToLatticeSurgery.rung_is_edge}
\leanok
\uses{def:RelationToLatticeSurgery.LadderGraph}
For each $i \in \mathrm{Fin}\, n$, the pair $(\mathrm{false}, i)$ and $(\mathrm{true}, i)$ are adjacent in the ladder graph.
\end{lemma}

\begin{proof}
\leanok
\uses{def:RelationToLatticeSurgery.LadderGraph, def:RelationToLatticeSurgery.ladderAdj}
Unfolding the definitions, the pair satisfies $p \neq q$ (since $\mathrm{false} \neq \mathrm{true}$) and the rung condition: different boundaries, same position.
\end{proof}

\begin{lemma}[Rails Are Edges]
\label{lem:RelationToLatticeSurgery.rail_is_edge}
\lean{QEC1.RelationToLatticeSurgery.rail_is_edge}
\leanok
\uses{def:RelationToLatticeSurgery.LadderGraph}
For boundary $b$ and positions $i, j \in \mathrm{Fin}\, n$ with $i + 1 = j$, the pair $(b, i)$ and $(b, j)$ are adjacent in the ladder graph.
\end{lemma}

\begin{proof}
\leanok
\uses{def:RelationToLatticeSurgery.LadderGraph, def:RelationToLatticeSurgery.ladderAdj}
Unfolding the definitions, the pair satisfies the rail condition: same boundary, consecutive positions. The distinctness $p \neq q$ follows because $i \neq j$ (since $i + 1 = j$ implies $i < j$), which is verified by integer arithmetic.
\end{proof}

\begin{theorem}[Ladder Graph is Connected]
\label{thm:RelationToLatticeSurgery.ladderGraph_connected}
\lean{QEC1.RelationToLatticeSurgery.ladderGraph_connected}
\leanok
\uses{def:RelationToLatticeSurgery.LadderGraph}
For $n \geq 1$, the ladder graph $\operatorname{LadderGraph}(n)$ is connected.
\end{theorem}

\begin{proof}
\leanok
\uses{def:RelationToLatticeSurgery.LadderGraph, lem:RelationToLatticeSurgery.rung_is_edge, lem:RelationToLatticeSurgery.rail_is_edge}
We show preconnectedness: for any two vertices $u$ and $v$, there is a path from $u$ to $v$. It suffices to show that every vertex $w = (b, k)$ is reachable from $(\mathrm{false}, 0)$. We proceed by induction on $k$:
\begin{itemize}
  \item \textbf{Base case} ($k = 0$): $(\mathrm{false}, 0)$ is reachable from itself by reflexivity.
  \item \textbf{Inductive step}: By the induction hypothesis, $(\mathrm{false}, m)$ is reachable. Since $m + 1 < n$, the rail edge from $(\mathrm{false}, m)$ to $(\mathrm{false}, m+1)$ extends the path.
\end{itemize}
This shows $(\mathrm{false}, k)$ is reachable. For $b = \mathrm{true}$, we use the rung edge from $(\mathrm{false}, k)$ to $(\mathrm{true}, k)$. Nonemptiness holds since $(\mathrm{false}, 0)$ exists when $n \geq 1$.
\end{proof}

\begin{theorem}[Ladder Graph Degree $\leq 3$]
\label{thm:RelationToLatticeSurgery.ladderGraph_degree_le_three}
\lean{QEC1.RelationToLatticeSurgery.ladderGraph_degree_le_three}
\leanok
\uses{def:RelationToLatticeSurgery.LadderGraph}
Each vertex of the ladder graph has degree at most $3$. The rung contributes $1$ neighbor, and each rail direction contributes at most $1$.
\end{theorem}

\begin{proof}
\leanok
\uses{def:RelationToLatticeSurgery.LadderGraph}
For any vertex $v$, the neighbor set is contained in the set $\{(\neg v_1, v_2),\, (v_1, \min(v_2+1, n-1)),\, (v_1, v_2-1)\}$, which has at most $3$ elements. Any neighbor $w$ is either a rung partner (different boundary, same position) or a rail neighbor (same boundary, position differing by $1$), and each falls into one of these three candidates. Hence the degree is at most $3$.
\end{proof}

\begin{theorem}[Ladder Graph Edge Count $\leq 3n$]
\label{thm:RelationToLatticeSurgery.ladderGraph_edge_count_le}
\lean{QEC1.RelationToLatticeSurgery.ladderGraph_edge_count_le}
\leanok
\uses{def:RelationToLatticeSurgery.LadderGraph}
The edge count of the ladder graph is at most $3n$.
\end{theorem}

\begin{proof}
\leanok
\uses{def:RelationToLatticeSurgery.LadderGraph, thm:RelationToLatticeSurgery.ladderGraph_degree_le_three, lem:RelationToLatticeSurgery.ladderGraph_card}
By the handshaking lemma, $2|E| = \sum_v \deg(v)$. Since each vertex has degree $\leq 3$ and there are $2n$ vertices, we have $\sum_v \deg(v) \leq 2n \cdot 3 = 6n$. Therefore $|E| \leq 3n$.
\end{proof}

%% ============================================================
%% Part 2: Lattice Surgery as a Special Case of Gauging
%% ============================================================

\begin{theorem}[Deformed Code on Ladder Graph is a Valid Stabilizer Code]
\label{thm:RelationToLatticeSurgery.ladder_deformed_code_is_stabilizer}
\lean{QEC1.RelationToLatticeSurgery.ladder_deformed_code_is_stabilizer}
\leanok
\uses{def:DeformedCode.CheckIndex, def:DeformedCode.DeformedCodeData, def:DeformedCode.allChecks, def:PauliOp, thm:DeformedCode.allChecks_commute}
For a graph $G$ with vertex set $V$, cycle set $C$, check operators $\{s_j\}_{j \in J}$, deformed code data, a cycle condition (each cycle has even incidence with every vertex), and pairwise commuting original checks, all checks of the deformed code pairwise commute:
\[
  \forall\, i, j \in \operatorname{CheckIndex}(V, C, J),\quad \operatorname{PauliCommute}\!\bigl(\operatorname{allChecks}(i),\, \operatorname{allChecks}(j)\bigr).
\]
For the ladder graph, this means the deformed code is a surface code on the union of the two patches: Gauss checks $A_v$ become vertex stabilizers, flux checks $B_p$ become face stabilizers, and deformed checks $\tilde{s}_j$ become boundary stabilizers.
\end{theorem}

\begin{proof}
\leanok
\uses{thm:DeformedCode.allChecks_commute}
This follows directly from the general pairwise commutation theorem $\operatorname{allChecks\_commute}$.
\end{proof}

\begin{theorem}[Gauss Check is Pure $X$-Type]
\label{thm:RelationToLatticeSurgery.gauss_check_is_pure_X}
\lean{QEC1.RelationToLatticeSurgery.gauss_check_is_pure_X}
\leanok
\uses{def:GaussFlux.gaussLawOp}
For any vertex $v \in V$, the Gauss law operator $A_v = \operatorname{gaussLawOp}(G, v)$ is pure $X$-type:
\[
  A_v.\mathrm{zVec} = 0.
\]
On the ladder graph, $A_v$ corresponds to the surface code vertex stabilizer: a star of $X$ operators centered at $v$.
\end{theorem}

\begin{proof}
\leanok
\uses{def:GaussFlux.gaussLawOp}
By extensionality, for each qubit $q$, the $Z$-component of $\operatorname{gaussLawOp}(G, v)$ is zero by simplification of the definition.
\end{proof}

\begin{theorem}[Logical is Product of Gauss Operators]
\label{thm:RelationToLatticeSurgery.logical_is_gauss_product}
\lean{QEC1.RelationToLatticeSurgery.logical_is_gauss_product}
\leanok
\uses{def:GaussFlux.gaussLawOp, def:GaussFlux.logicalOp, thm:GaussFlux.gaussLaw_product}
The logical operator equals the product of all Gauss operators:
\[
  L = \operatorname{logicalOp}(G) = \prod_{v \in V} A_v.
\]
On the ladder graph where $V = \operatorname{supp}(\bar{X}_1) \cup \operatorname{supp}(\bar{X}_2)$, this gives $L = \bar{X}_1 \otimes \bar{X}_2$.
\end{theorem}

\begin{proof}
\leanok
\uses{thm:GaussFlux.gaussLaw_product}
This is the symmetric form of the Gauss law product identity $\prod_v A_v = L$.
\end{proof}

\begin{theorem}[Phase 3 Splitting Step]
\label{thm:RelationToLatticeSurgery.ladder_phase3_disentangles}
\lean{QEC1.RelationToLatticeSurgery.ladder_phase3_disentangles}
\leanok
\uses{def:GaussFlux.gaussLawOp, def:PauliOp}
Phase 3 of the gauging procedure measures $Z_e$ on each edge qubit $e \in E(G)$. For each edge $e$, the measured operator $Z_e^{\mathrm{meas}}$ satisfies three properties:
\begin{enumerate}
  \item $Z_e^{\mathrm{meas}}.\mathrm{xVec} = 0$ (pure $Z$-type, does not disturb vertex qubits),
  \item $\forall\, e',\; \operatorname{PauliCommute}(Z_e^{\mathrm{meas}}, Z_{e'}^{\mathrm{meas}})$ (all mutually commute, can be measured simultaneously),
  \item $Z_e^{\mathrm{meas}} \cdot Z_e^{\mathrm{meas}} = 1$ (self-inverse, valid projective measurement).
\end{enumerate}
After measuring all $Z_e$, the edge qubits are projected onto $Z$ eigenstates, disentangling them from the vertex qubits and recovering the two separate surface code patches.
\end{theorem}

\begin{proof}
\leanok
\uses{def:GaussFlux.gaussLawOp, def:PauliOp}
The three components follow directly from:
\begin{enumerate}
  \item $\operatorname{edgeZOperatorMeasured\_xVec}$ (the edge $Z$ operator has no $X$ support),
  \item $\operatorname{edgeZ\_measured\_commute}$ (edge $Z$ operators pairwise commute),
  \item $\operatorname{edgeZOperatorMeasured\_mul\_self}$ (each edge $Z$ operator is self-inverse).
\end{enumerate}
\end{proof}

\begin{theorem}[Logical Operator is Pure $X$-Type]
\label{thm:RelationToLatticeSurgery.logical_is_pure_X}
\lean{QEC1.RelationToLatticeSurgery.logical_is_pure_X}
\leanok
\uses{def:GaussFlux.logicalOp}
The logical operator $L = \operatorname{logicalOp}(G)$ has no $Z$ support:
\[
  L.\mathrm{zVec} = 0.
\]
\end{theorem}

\begin{proof}
\leanok
\uses{def:GaussFlux.logicalOp}
By extensionality, for each qubit $q$, the $Z$-component of $\operatorname{logicalOp}(G)$ is zero by simplification of the definition.
\end{proof}

\begin{theorem}[Logical Operator is Self-Inverse]
\label{thm:RelationToLatticeSurgery.logical_self_inverse}
\lean{QEC1.RelationToLatticeSurgery.logical_self_inverse}
\leanok
\uses{def:GaussFlux.logicalOp, def:PauliOp}
The logical operator satisfies $L \cdot L = 1$.
\end{theorem}

\begin{proof}
\leanok
\uses{def:GaussFlux.logicalOp, def:PauliOp}
By extensionality on both the $X$ and $Z$ vector components, each component squared is zero in $\mathbb{F}_2$, giving the identity.
\end{proof}

\begin{lemma}[Ladder Vertex Decomposition]
\label{lem:RelationToLatticeSurgery.ladder_vertex_decomposition}
\lean{QEC1.RelationToLatticeSurgery.ladder_vertex_decomposition}
\leanok
\uses{def:RelationToLatticeSurgery.LadderGraph}
$|\mathrm{Bool} \times \mathrm{Fin}\, n| = n + n$.
\end{lemma}

\begin{proof}
\leanok
\uses{def:RelationToLatticeSurgery.LadderGraph}
By simplification using cardinalities of product types: $|\mathrm{Bool}| \cdot |\mathrm{Fin}\, n| = 2n = n + n$.
\end{proof}

\begin{theorem}[Lattice Surgery is a Special Case of Gauging]
\label{thm:RelationToLatticeSurgery.lattice_surgery_is_gauging}
\lean{QEC1.RelationToLatticeSurgery.lattice_surgery_is_gauging}
\leanok
\uses{def:RelationToLatticeSurgery.LadderGraph}
For a ladder graph $G$ connecting two boundary qubit sets of size $n \geq 1$:
\begin{enumerate}
  \item $G$ is connected,
  \item $|\mathrm{Bool} \times \mathrm{Fin}\, n| = 2n$,
  \item every vertex has degree $\leq 3$ (constant overhead per qubit),
  \item the edge count is at most $3n$.
\end{enumerate}
\end{theorem}

\begin{proof}
\leanok
\uses{thm:RelationToLatticeSurgery.ladderGraph_connected, lem:RelationToLatticeSurgery.ladderGraph_card, thm:RelationToLatticeSurgery.ladderGraph_degree_le_three, thm:RelationToLatticeSurgery.ladderGraph_edge_count_le}
The four components follow directly from the previously established results: connectedness of the ladder graph, vertex count $2n$, degree bound $\leq 3$, and edge count bound $\leq 3n$.
\end{proof}

%% ============================================================
%% Part 3: Dummy Grid Graph for Non-Adjacent Surface Codes
%% ============================================================

\begin{definition}[Grid Adjacency]
\label{def:RelationToLatticeSurgery.gridAdj}
\lean{QEC1.RelationToLatticeSurgery.gridAdj}
\leanok
\uses{def:PauliOp}
The adjacency relation for the grid graph on $n \times (D+2)$ vertices.
Vertices are $\mathrm{Fin}\, n \times \mathrm{Fin}\,(D+2)$. Two vertices $p = (r_1, c_1)$ and $q = (r_2, c_2)$ are adjacent if $p \neq q$ and either:
\begin{enumerate}
  \item \textbf{Horizontal}: $r_1 = r_2$ and $|c_1 - c_2| = 1$ (same row, adjacent columns), or
  \item \textbf{Vertical}: $c_1 = c_2$ and $|r_1 - r_2| = 1$ (same column, adjacent rows).
\end{enumerate}
\end{definition}

\begin{definition}[Dummy Grid Graph]
\label{def:RelationToLatticeSurgery.DummyGridGraph}
\lean{QEC1.RelationToLatticeSurgery.DummyGridGraph}
\leanok
\uses{def:RelationToLatticeSurgery.gridAdj}
The grid graph connecting two boundary edges with $D$ columns of dummy ancillas.
Column $0$ is boundary $1$, column $D+1$ is boundary $2$, and columns $1, \ldots, D$ are dummy ancillas.
This is a simple graph on $\mathrm{Fin}\, n \times \mathrm{Fin}\,(D+2)$.
\end{definition}

\begin{lemma}[Grid Graph Vertex Count]
\label{lem:RelationToLatticeSurgery.dummyGridGraph_card}
\lean{QEC1.RelationToLatticeSurgery.dummyGridGraph_card}
\leanok
\uses{def:RelationToLatticeSurgery.DummyGridGraph}
$|\mathrm{Fin}\, n \times \mathrm{Fin}\,(D+2)| = n \cdot (D + 2)$.
\end{lemma}

\begin{proof}
\leanok
\uses{def:RelationToLatticeSurgery.DummyGridGraph}
By simplification using cardinalities of product and $\mathrm{Fin}$ types.
\end{proof}

\begin{lemma}[Horizontal Edges in Grid]
\label{lem:RelationToLatticeSurgery.grid_horizontal_edge}
\lean{QEC1.RelationToLatticeSurgery.grid_horizontal_edge}
\leanok
\uses{def:RelationToLatticeSurgery.DummyGridGraph}
For $i \in \mathrm{Fin}\, n$ and $j, j' \in \mathrm{Fin}\,(D+2)$ with $j + 1 = j'$, the pair $(i, j)$ and $(i, j')$ are adjacent in the grid graph.
\end{lemma}

\begin{proof}
\leanok
\uses{def:RelationToLatticeSurgery.DummyGridGraph, def:RelationToLatticeSurgery.gridAdj}
The pair satisfies the horizontal adjacency condition: same row, adjacent columns. Distinctness follows because $j \neq j'$ (since $j + 1 = j'$ implies $j < j'$), verified by integer arithmetic.
\end{proof}

\begin{lemma}[Vertical Edges in Grid]
\label{lem:RelationToLatticeSurgery.grid_vertical_edge}
\lean{QEC1.RelationToLatticeSurgery.grid_vertical_edge}
\leanok
\uses{def:RelationToLatticeSurgery.DummyGridGraph}
For $i, i' \in \mathrm{Fin}\, n$ and $j \in \mathrm{Fin}\,(D+2)$ with $i + 1 = i'$, the pair $(i, j)$ and $(i', j)$ are adjacent in the grid graph.
\end{lemma}

\begin{proof}
\leanok
\uses{def:RelationToLatticeSurgery.DummyGridGraph, def:RelationToLatticeSurgery.gridAdj}
The pair satisfies the vertical adjacency condition: same column, adjacent rows. Distinctness follows because $i \neq i'$ (since $i + 1 = i'$), verified by integer arithmetic.
\end{proof}

\begin{theorem}[Grid Graph is Connected]
\label{thm:RelationToLatticeSurgery.dummyGridGraph_connected}
\lean{QEC1.RelationToLatticeSurgery.dummyGridGraph_connected}
\leanok
\uses{def:RelationToLatticeSurgery.DummyGridGraph}
For $n \geq 1$, the dummy grid graph $\operatorname{DummyGridGraph}(n, D)$ is connected.
\end{theorem}

\begin{proof}
\leanok
\uses{def:RelationToLatticeSurgery.DummyGridGraph, lem:RelationToLatticeSurgery.grid_horizontal_edge, lem:RelationToLatticeSurgery.grid_vertical_edge}
We show preconnectedness. It suffices to show every vertex $(r, c)$ is reachable from $(0, 0)$.
First, we reach $(0, c)$ from $(0, 0)$ by induction on $c$, using horizontal edges at each step.
Then we reach $(r, c)$ from $(0, c)$ by induction on $r$, using vertical edges at each step.
Composing these two paths gives reachability of any vertex from $(0, 0)$.
For any two vertices $u, v$, we have paths from $(0,0)$ to each, giving a path between them by transitivity and symmetry.
Nonemptiness holds since $(0, 0)$ exists when $n \geq 1$.
\end{proof}

\begin{theorem}[Grid Graph Degree $\leq 4$]
\label{thm:RelationToLatticeSurgery.dummyGridGraph_degree_le_four}
\lean{QEC1.RelationToLatticeSurgery.dummyGridGraph_degree_le_four}
\leanok
\uses{def:RelationToLatticeSurgery.DummyGridGraph}
Each vertex of the grid graph has degree at most $4$ (up, down, left, right).
\end{theorem}

\begin{proof}
\leanok
\uses{def:RelationToLatticeSurgery.DummyGridGraph}
For any vertex $v = (r, c)$, the neighbor set is contained in the set of at most four candidates: $(r, \min(c+1, D+1))$, $(r, c-1)$, $(\min(r+1, n-1), c)$, and $(r-1, c)$. This set has at most $4$ elements. Any neighbor $w$ must be horizontally or vertically adjacent, and in either case falls into one of these four candidates. Hence the degree is at most $4$.
\end{proof}

\begin{theorem}[Constant Overhead Per Unit Distance]
\label{thm:RelationToLatticeSurgery.dummy_overhead_per_unit_distance}
\lean{QEC1.RelationToLatticeSurgery.dummy_overhead_per_unit_distance}
\leanok
\uses{def:RelationToLatticeSurgery.DummyGridGraph}
For boundary qubit sets of size $n$ separated by distance $D$:
\[
  n \cdot D + 2n = n \cdot (D + 2).
\]
The overhead is $n$ qubits per column, independent of $D$.
\end{theorem}

\begin{proof}
\leanok
\uses{def:RelationToLatticeSurgery.DummyGridGraph}
This follows by ring arithmetic.
\end{proof}

\begin{theorem}[Overhead Decomposition]
\label{thm:RelationToLatticeSurgery.dummy_overhead_decomposition}
\lean{QEC1.RelationToLatticeSurgery.dummy_overhead_decomposition}
\leanok
\uses{def:RelationToLatticeSurgery.DummyGridGraph}
The total qubit count decomposes into boundary plus ancilla:
\[
  n \cdot (D + 2) = 2n + n \cdot D.
\]
\end{theorem}

\begin{proof}
\leanok
\uses{def:RelationToLatticeSurgery.DummyGridGraph}
This follows by ring arithmetic.
\end{proof}

\begin{theorem}[Grid Graph Edge Count]
\label{thm:RelationToLatticeSurgery.dummyGridGraph_edge_count_le}
\lean{QEC1.RelationToLatticeSurgery.dummyGridGraph_edge_count_le}
\leanok
\uses{def:RelationToLatticeSurgery.DummyGridGraph}
The edge count of the grid graph satisfies
\[
  |E(\operatorname{DummyGridGraph}(n, D))| \leq 2 \cdot n \cdot (D + 2).
\]
\end{theorem}

\begin{proof}
\leanok
\uses{def:RelationToLatticeSurgery.DummyGridGraph, thm:RelationToLatticeSurgery.dummyGridGraph_degree_le_four, lem:RelationToLatticeSurgery.dummyGridGraph_card}
By the handshaking lemma, $2|E| = \sum_v \deg(v)$. Since each vertex has degree $\leq 4$ and there are $n(D+2)$ vertices, we have $\sum_v \deg(v) \leq n(D+2) \cdot 4$. Therefore $|E| \leq 2 \cdot n \cdot (D+2)$.
\end{proof}

%% ============================================================
%% Part 4: Bridge Graph for LDPC Generalization
%% ============================================================

\begin{definition}[Bridge Adjacency]
\label{def:RelationToLatticeSurgery.bridgeAdj}
\lean{QEC1.RelationToLatticeSurgery.bridgeAdj}
\leanok
\uses{def:PauliOp}
The adjacency relation for the bridge graph on $V \oplus V$.
Two vertices $p, q \in V \oplus V$ are adjacent if:
\begin{itemize}
  \item Both in copy 1 ($\operatorname{inl}(v)$, $\operatorname{inl}(w)$): $G.\operatorname{Adj}(v, w)$,
  \item Both in copy 2 ($\operatorname{inr}(v)$, $\operatorname{inr}(w)$): $G.\operatorname{Adj}(v, w)$,
  \item Across copies ($\operatorname{inl}(v)$, $\operatorname{inr}(w)$ or vice versa): $v = w$ and $v \in \operatorname{bridges}$.
\end{itemize}
\end{definition}

\begin{lemma}[Bridge Adjacency is Symmetric]
\label{lem:RelationToLatticeSurgery.bridgeAdj_symm}
\lean{QEC1.RelationToLatticeSurgery.bridgeAdj_symm}
\leanok
\uses{def:RelationToLatticeSurgery.bridgeAdj}
For all $p, q \in V \oplus V$, if $\operatorname{bridgeAdj}(G, \operatorname{bridges}, p, q)$ then $\operatorname{bridgeAdj}(G, \operatorname{bridges}, q, p)$.
\end{lemma}

\begin{proof}
\leanok
\uses{def:RelationToLatticeSurgery.bridgeAdj}
We match on $p$ and $q$. If both are in the same copy, symmetry follows from $G$'s symmetry. If they are in different copies, we have $v = w$ and $v \in \operatorname{bridges}$; swapping gives $w = v$ and $w \in \operatorname{bridges}$ (after substitution).
\end{proof}

\begin{lemma}[Bridge Adjacency is Irreflexive]
\label{lem:RelationToLatticeSurgery.bridgeAdj_irrefl}
\lean{QEC1.RelationToLatticeSurgery.bridgeAdj_irrefl}
\leanok
\uses{def:RelationToLatticeSurgery.bridgeAdj}
For all $p \in V \oplus V$, $\neg\operatorname{bridgeAdj}(G, \operatorname{bridges}, p, p)$.
\end{lemma}

\begin{proof}
\leanok
\uses{def:RelationToLatticeSurgery.bridgeAdj}
Matching on $p$: if $p = \operatorname{inl}(v)$ or $p = \operatorname{inr}(v)$, then $\operatorname{bridgeAdj}(p, p)$ would require $G.\operatorname{Adj}(v, v)$, which contradicts the loopless property of $G$.
\end{proof}

\begin{definition}[Bridge Graph]
\label{def:RelationToLatticeSurgery.BridgeGraph}
\lean{QEC1.RelationToLatticeSurgery.BridgeGraph}
\leanok
\uses{def:RelationToLatticeSurgery.bridgeAdj}
The bridge graph: two copies of $G$ on vertex set $V \oplus V$, connected by bridge edges at specified vertices in $\operatorname{bridges} \subseteq V$.
\end{definition}

\begin{lemma}[Copy 1 Edges are Bridge Graph Edges]
\label{lem:RelationToLatticeSurgery.bridgeGraph_copy1_adj}
\lean{QEC1.RelationToLatticeSurgery.bridgeGraph_copy1_adj}
\leanok
\uses{def:RelationToLatticeSurgery.BridgeGraph}
If $G.\operatorname{Adj}(v, w)$, then $(\operatorname{BridgeGraph}\, G\, \operatorname{bridges}).\operatorname{Adj}(\operatorname{inl}(v), \operatorname{inl}(w))$.
\end{lemma}

\begin{proof}
\leanok
\uses{def:RelationToLatticeSurgery.BridgeGraph}
This holds directly by the definition of bridge adjacency for same-copy vertices.
\end{proof}

\begin{lemma}[Copy 2 Edges are Bridge Graph Edges]
\label{lem:RelationToLatticeSurgery.bridgeGraph_copy2_adj}
\lean{QEC1.RelationToLatticeSurgery.bridgeGraph_copy2_adj}
\leanok
\uses{def:RelationToLatticeSurgery.BridgeGraph}
If $G.\operatorname{Adj}(v, w)$, then $(\operatorname{BridgeGraph}\, G\, \operatorname{bridges}).\operatorname{Adj}(\operatorname{inr}(v), \operatorname{inr}(w))$.
\end{lemma}

\begin{proof}
\leanok
\uses{def:RelationToLatticeSurgery.BridgeGraph}
This holds directly by the definition of bridge adjacency for same-copy vertices.
\end{proof}

\begin{lemma}[Bridge Edges Connect Copies]
\label{lem:RelationToLatticeSurgery.bridgeGraph_bridge_adj}
\lean{QEC1.RelationToLatticeSurgery.bridgeGraph_bridge_adj}
\leanok
\uses{def:RelationToLatticeSurgery.BridgeGraph}
For $v \in \operatorname{bridges}$, the bridge graph has an edge $\operatorname{inl}(v) \sim \operatorname{inr}(v)$.
\end{lemma}

\begin{proof}
\leanok
\uses{def:RelationToLatticeSurgery.BridgeGraph, def:RelationToLatticeSurgery.bridgeAdj}
Unfolding the definition of bridge adjacency for cross-copy vertices, the condition $v = v \land v \in \operatorname{bridges}$ holds by reflexivity and the hypothesis.
\end{proof}

\begin{lemma}[Bridge Graph Vertex Count]
\label{lem:RelationToLatticeSurgery.bridgeGraph_card}
\lean{QEC1.RelationToLatticeSurgery.bridgeGraph_card}
\leanok
\uses{def:RelationToLatticeSurgery.BridgeGraph}
$|V \oplus V| = 2|V|$.
\end{lemma}

\begin{proof}
\leanok
\uses{def:RelationToLatticeSurgery.BridgeGraph}
By simplification using the cardinality of sum types and ring arithmetic.
\end{proof}

\begin{definition}[Copy 1 Embedding]
\label{def:RelationToLatticeSurgery.bridgeGraph_copy1_embedding}
\lean{QEC1.RelationToLatticeSurgery.bridgeGraph_copy1_embedding}
\leanok
\uses{def:RelationToLatticeSurgery.BridgeGraph}
The graph homomorphism $G \to \operatorname{BridgeGraph}(G, \operatorname{bridges})$ embedding $G$ as copy 1, via $v \mapsto \operatorname{inl}(v)$.
Paths in $G$ correspond to paths in the bridge graph within copy 1.
\end{definition}

\begin{definition}[Copy 2 Embedding]
\label{def:RelationToLatticeSurgery.bridgeGraph_copy2_embedding}
\lean{QEC1.RelationToLatticeSurgery.bridgeGraph_copy2_embedding}
\leanok
\uses{def:RelationToLatticeSurgery.BridgeGraph}
The graph homomorphism $G \to \operatorname{BridgeGraph}(G, \operatorname{bridges})$ embedding $G$ as copy 2, via $v \mapsto \operatorname{inr}(v)$.
\end{definition}

\begin{theorem}[Bridge Graph is Connected]
\label{thm:RelationToLatticeSurgery.bridgeGraph_connected}
\lean{QEC1.RelationToLatticeSurgery.bridgeGraph_connected}
\leanok
\uses{def:RelationToLatticeSurgery.BridgeGraph}
If $G$ is connected and $\operatorname{bridges}$ is nonempty, then $\operatorname{BridgeGraph}(G, \operatorname{bridges})$ is connected.
\end{theorem}

\begin{proof}
\leanok
\uses{def:RelationToLatticeSurgery.BridgeGraph, def:RelationToLatticeSurgery.bridgeGraph_copy1_embedding, def:RelationToLatticeSurgery.bridgeGraph_copy2_embedding, lem:RelationToLatticeSurgery.bridgeGraph_bridge_adj}
Let $b \in \operatorname{bridges}$. It suffices to show every vertex $w \in V \oplus V$ is reachable from $\operatorname{inl}(b)$.
\begin{itemize}
  \item If $w = \operatorname{inl}(v)$: Since $G$ is connected, $b$ and $v$ are connected in $G$. The copy 1 embedding maps this path to a path in the bridge graph from $\operatorname{inl}(b)$ to $\operatorname{inl}(v)$.
  \item If $w = \operatorname{inr}(v)$: First cross the bridge from $\operatorname{inl}(b)$ to $\operatorname{inr}(b)$ (using $b \in \operatorname{bridges}$). Then, since $G$ is connected, $b$ and $v$ are connected in $G$. The copy 2 embedding maps this to a path from $\operatorname{inr}(b)$ to $\operatorname{inr}(v)$.
\end{itemize}
Nonemptiness holds since $\operatorname{inl}(b)$ exists.
\end{proof}

\begin{theorem}[Bridge Graph Distance Bound]
\label{thm:RelationToLatticeSurgery.bridgeGraph_dist_le_copy1}
\lean{QEC1.RelationToLatticeSurgery.bridgeGraph_dist_le_copy1}
\leanok
\uses{def:RelationToLatticeSurgery.BridgeGraph, def:RelationToLatticeSurgery.bridgeGraph_copy1_embedding}
If $v$ and $w$ are reachable in $G$, then:
\[
  \operatorname{dist}_{\operatorname{BridgeGraph}}(\operatorname{inl}(v), \operatorname{inl}(w)) \leq \operatorname{dist}_G(v, w).
\]
Paths within a copy of $G$ are paths in the bridge graph.
\end{theorem}

\begin{proof}
\leanok
\uses{def:RelationToLatticeSurgery.bridgeGraph_copy1_embedding}
From the reachability hypothesis, we obtain a walk $p$ in $G$ from $v$ to $w$ with $|p| = \operatorname{dist}_G(v, w)$. Mapping $p$ via the copy 1 embedding gives a walk in the bridge graph of the same length. Since the distance is at most the length of any walk, we conclude $\operatorname{dist}_{\operatorname{BridgeGraph}}(\operatorname{inl}(v), \operatorname{inl}(w)) \leq |p| = \operatorname{dist}_G(v, w)$.
\end{proof}

%% ============================================================
%% Part 5: LDPC Desiderata on the Bridge Graph
%% ============================================================

\begin{theorem}[Short Paths Transfer Within Copy]
\label{thm:RelationToLatticeSurgery.bridge_short_paths_within_copy}
\lean{QEC1.RelationToLatticeSurgery.bridge_short_paths_within_copy}
\leanok
\uses{def:DesiderataForGraphG.ShortPathsForDeformation, def:RelationToLatticeSurgery.BridgeGraph, def:PauliOp}
If $G$ has the short paths property with bound $D$ (i.e., for each check $s_j$, any two vertices $u, v$ in the $Z$-support of $s_j$ satisfy $\operatorname{dist}_G(u, v) \leq D$), and $G$ is connected, then for any bridge set, vertices in the same copy of the bridge graph also have short paths:
\[
  \operatorname{dist}_{\operatorname{BridgeGraph}}(\operatorname{inl}(u), \operatorname{inl}(v)) \leq D.
\]
\end{theorem}

\begin{proof}
\leanok
\uses{def:DesiderataForGraphG.ShortPathsForDeformation, thm:RelationToLatticeSurgery.bridgeGraph_dist_le_copy1}
We compute:
\[
  \operatorname{dist}_{\operatorname{BridgeGraph}}(\operatorname{inl}(u), \operatorname{inl}(v)) \leq \operatorname{dist}_G(u, v) \leq D,
\]
where the first inequality is the bridge graph distance bound (since $u$ and $v$ are reachable in the connected graph $G$), and the second follows from the short paths property.
\end{proof}

\begin{theorem}[Low-Weight Cycle Basis in Copy]
\label{thm:RelationToLatticeSurgery.bridge_low_weight_cycles_in_copy}
\lean{QEC1.RelationToLatticeSurgery.bridge_low_weight_cycles_in_copy}
\leanok
\uses{def:DesiderataForGraphG.LowWeightCycleBasis}
When $G$ has a cycle basis where each cycle has weight $\leq W$, the bridge graph inherits this bound for cycles within each copy:
\[
  \forall\, c \in C,\quad |\{e \in E(G) : e \in \operatorname{cycles}(c)\}| \leq W.
\]
\end{theorem}

\begin{proof}
\leanok
\uses{def:DesiderataForGraphG.LowWeightCycleBasis}
This follows directly from the low-weight cycle basis property of $G$.
\end{proof}

\begin{theorem}[Cheeger Expansion Acceptable with Desiderata]
\label{thm:RelationToLatticeSurgery.cheeger_expansion_acceptable_with_desiderata}
\lean{QEC1.RelationToLatticeSurgery.cheeger_expansion_acceptable_with_desiderata}
\leanok
\uses{def:DesiderataForGraphG.ShortPathsForDeformation, def:DesiderataForGraphG.LowWeightCycleBasis, def:RelationToLatticeSurgery.BridgeGraph, def:PauliOp}
If $G$ satisfies short paths with bound $D$ and low-weight cycles with bound $W$, and $G$ is connected with a nonempty bridge set, then the bridge graph provides a valid gauging setup regardless of its Cheeger constant:
\begin{enumerate}
  \item $\operatorname{BridgeGraph}(G, \operatorname{bridges})$ is connected,
  \item Short paths within each copy: $\operatorname{dist}_{\operatorname{BridgeGraph}}(\operatorname{inl}(u), \operatorname{inl}(v)) \leq D$ for $u, v$ in the $Z$-support of any check,
  \item Low-weight cycles: each cycle in the basis has weight $\leq W$.
\end{enumerate}
\end{theorem}

\begin{proof}
\leanok
\uses{thm:RelationToLatticeSurgery.bridgeGraph_connected, thm:RelationToLatticeSurgery.bridge_short_paths_within_copy, thm:RelationToLatticeSurgery.bridge_low_weight_cycles_in_copy}
The three components follow from the bridge graph connectivity theorem, the short paths transfer theorem, and the low-weight cycles inheritance, respectively.
\end{proof}

\begin{theorem}[LDPC Lattice Surgery Generalization]
\label{thm:RelationToLatticeSurgery.ldpc_lattice_surgery_generalization}
\lean{QEC1.RelationToLatticeSurgery.ldpc_lattice_surgery_generalization}
\leanok
\uses{def:DesiderataForGraphG.ShortPathsForDeformation, def:DesiderataForGraphG.LowWeightCycleBasis, def:RelationToLatticeSurgery.BridgeGraph, def:PauliOp}
For two code blocks with weight-$W$ logical $X$ operators, given a connected graph $G$ satisfying short paths (bound $D$) and low-weight cycle basis (bound $W$), with a nonempty bridge set:
\begin{enumerate}
  \item $\operatorname{BridgeGraph}(G, \operatorname{bridges})$ is connected,
  \item $|V \oplus V| = 2|V|$,
  \item Short paths within each copy are preserved (bound $D$),
  \item Low-weight cycles within each copy are preserved (bound $W$).
\end{enumerate}
This generalizes lattice surgery to arbitrary LDPC codes.
\end{theorem}

\begin{proof}
\leanok
\uses{thm:RelationToLatticeSurgery.bridgeGraph_connected, lem:RelationToLatticeSurgery.bridgeGraph_card, thm:RelationToLatticeSurgery.bridge_short_paths_within_copy, thm:RelationToLatticeSurgery.bridge_low_weight_cycles_in_copy}
The four components follow from bridge graph connectivity, the vertex count identity $|V \oplus V| = 2|V|$, the short paths transfer theorem, and the low-weight cycles inheritance, respectively.
\end{proof}

%% ============================================================
%% Part 6: Long-Range Lattice Surgery
%% ============================================================

\begin{theorem}[Long-Range Surgery Overhead]
\label{thm:RelationToLatticeSurgery.long_range_surgery_overhead}
\lean{QEC1.RelationToLatticeSurgery.long_range_surgery_overhead}
\leanok
\uses{def:RelationToLatticeSurgery.DummyGridGraph}
For boundary qubit sets of size $n \geq 1$ separated by distance $D$:
\begin{enumerate}
  \item The grid graph $\operatorname{DummyGridGraph}(n, D)$ is connected,
  \item $|\mathrm{Fin}\, n \times \mathrm{Fin}\,(D+2)| = n(D+2)$,
  \item $n(D+2) = 2n + nD$ (the total decomposes into $2n$ boundary qubits and $nD$ ancillas).
\end{enumerate}
\end{theorem}

\begin{proof}
\leanok
\uses{thm:RelationToLatticeSurgery.dummyGridGraph_connected, lem:RelationToLatticeSurgery.dummyGridGraph_card, thm:RelationToLatticeSurgery.dummy_overhead_decomposition}
The three components follow from the grid graph connectivity theorem, the vertex count formula, and the overhead decomposition identity.
\end{proof}

%% ============================================================
%% Summary
%% ============================================================

\begin{theorem}[Gauging Generalizes Lattice Surgery]
\label{thm:RelationToLatticeSurgery.gauging_generalizes_lattice_surgery}
\lean{QEC1.RelationToLatticeSurgery.gauging_generalizes_lattice_surgery}
\leanok
\uses{def:RelationToLatticeSurgery.LadderGraph, def:RelationToLatticeSurgery.DummyGridGraph, def:RelationToLatticeSurgery.BridgeGraph}
The gauging measurement procedure generalizes lattice surgery in three settings:
\begin{enumerate}
  \item \textbf{Standard lattice surgery (ladder graph):} For all $n \geq 1$, the ladder graph is connected, has $2n$ vertices, degree $\leq 3$, and at most $3n$ edges.
  \item \textbf{Long-range lattice surgery (grid graph):} For all $n \geq 1$ and $D \geq 0$, the grid graph is connected and $n(D+2) = 2n + nD$.
  \item \textbf{LDPC generalization (bridge graph):} For any connected graph $G$ with nonempty bridge set, the bridge graph is connected with $2|V|$ vertices.
\end{enumerate}
\end{theorem}

\begin{proof}
\leanok
\uses{thm:RelationToLatticeSurgery.ladderGraph_connected, lem:RelationToLatticeSurgery.ladderGraph_card, thm:RelationToLatticeSurgery.ladderGraph_degree_le_three, thm:RelationToLatticeSurgery.ladderGraph_edge_count_le, thm:RelationToLatticeSurgery.dummyGridGraph_connected, thm:RelationToLatticeSurgery.dummy_overhead_decomposition, thm:RelationToLatticeSurgery.bridgeGraph_connected, lem:RelationToLatticeSurgery.bridgeGraph_card}
We prove each part separately:
\begin{enumerate}
  \item For the ladder graph with $n \geq 1$: connectivity, vertex count $2n$, degree bound $\leq 3$, and edge bound $\leq 3n$ follow from the established results.
  \item For the grid graph with $n \geq 1$: connectivity follows from the grid connectivity theorem, and the decomposition $n(D+2) = 2n + nD$ follows from the overhead decomposition.
  \item For the bridge graph: connectivity follows from the bridge graph connectivity theorem (using $G$ connected and bridges nonempty), and the vertex count $|V \oplus V| = 2|V|$ follows from the cardinality identity.
\end{enumerate}
\end{proof}

%--- Rem_19: BivariateBicycleCodeNotation ---
\chapter{Rem 19: Bivariate Bicycle Code Notation}

The Bivariate Bicycle (BB) code construction uses cyclic permutation matrices and their
tensor products to define CSS codes on $2\ell m$ qubits. This remark introduces the group algebra
$\mathbb{F}_2[x,y]/(x^\ell - 1, y^m - 1)$, defines the X-type and Z-type parity checks,
and constructs the BB code as a stabilizer code under the CSS condition $AB^T = BA^T$.

\begin{definition}[BB Monomial Group]
\label{def:BivariateBicycle.BBMonomial}
\lean{QEC1.BivariateBicycle.BBMonomial}
\leanok

The \emph{monomial group} $M = \mathbb{Z}_\ell \times \mathbb{Z}_m$ represents monomials $x^a y^b$
with $a \in \{0, \ldots, \ell-1\}$ and $b \in \{0, \ldots, m-1\}$. We use the additive group structure
of $\mathbb{Z}_\ell \times \mathbb{Z}_m$ (addition corresponds to monomial multiplication).
\end{definition}

\begin{definition}[BB Group Algebra]
\label{def:BivariateBicycle.BBGroupAlgebra}
\lean{QEC1.BivariateBicycle.BBGroupAlgebra}
\leanok
\uses{def:BivariateBicycle.BBMonomial}
The \emph{group algebra} $\mathbb{F}_2[x,y]/(x^\ell - 1, y^m - 1)$. An element is a function
$M \to \mathbb{Z}/2\mathbb{Z}$, where the value at $(a,b)$ is the coefficient of $x^a y^b$.
\end{definition}

\begin{definition}[BB Monomial Element]
\label{def:BivariateBicycle.bbMonomial}
\lean{QEC1.BivariateBicycle.bbMonomial}
\leanok
\uses{def:BivariateBicycle.BBGroupAlgebra}
The monomial $x^a y^b$ as an element of the group algebra: the indicator function of the single element $(a, b)$.
That is, $\operatorname{bbMonomial}(a, b)(\gamma) = 1$ if $\gamma = (a, b)$ and $0$ otherwise.
\end{definition}

\begin{definition}[BB Zero Polynomial]
\label{def:BivariateBicycle.bbZero}
\lean{QEC1.BivariateBicycle.bbZero}
\leanok
\uses{def:BivariateBicycle.BBGroupAlgebra}
The zero polynomial in the group algebra, mapping every monomial to $0$.
\end{definition}

\begin{definition}[BB Unit Polynomial]
\label{def:BivariateBicycle.bbOne}
\lean{QEC1.BivariateBicycle.bbOne}
\leanok
\uses{def:BivariateBicycle.bbMonomial}
The unit polynomial $1 = x^0 y^0$ in the group algebra, defined as $\operatorname{bbMonomial}(0, 0)$.
\end{definition}

\begin{definition}[BB Addition]
\label{def:BivariateBicycle.bbAdd}
\lean{QEC1.BivariateBicycle.bbAdd}
\leanok
\uses{def:BivariateBicycle.BBGroupAlgebra}
Addition in the group algebra, given by pointwise XOR (since coefficients are in $\mathbb{Z}/2\mathbb{Z}$).
For polynomials $p, q$, $(p + q)(\gamma) = p(\gamma) + q(\gamma)$.
\end{definition}

\begin{definition}[BB Convolution]
\label{def:BivariateBicycle.bbConvolve}
\lean{QEC1.BivariateBicycle.bbConvolve}
\leanok
\uses{def:BivariateBicycle.BBGroupAlgebra, def:BivariateBicycle.BBMonomial}
Convolution (polynomial multiplication) in $\mathbb{F}_2[x,y]/(x^\ell - 1, y^m - 1)$.
For polynomials $p, q$:
\[
(p * q)(\gamma) = \sum_{\alpha \in M} p(\alpha) \cdot q(\gamma - \alpha),
\]
where subtraction is taken in $\mathbb{Z}_\ell \times \mathbb{Z}_m$.
This corresponds to matrix multiplication of the associated circulant-like matrices.
\end{definition}

\begin{definition}[BB Support (Set)]
\label{def:BivariateBicycle.bbSupport}
\lean{QEC1.BivariateBicycle.bbSupport}
\leanok
\uses{def:BivariateBicycle.BBGroupAlgebra, def:BivariateBicycle.BBMonomial}
The support of a polynomial $p$: the set of monomials with nonzero coefficient,
$\operatorname{supp}(p) = \{ \alpha \in M \mid p(\alpha) \neq 0 \}$.
\end{definition}

\begin{definition}[BB Support (Finset)]
\label{def:BivariateBicycle.bbSupportFinset}
\lean{QEC1.BivariateBicycle.bbSupportFinset}
\leanok
\uses{def:BivariateBicycle.BBGroupAlgebra, def:BivariateBicycle.BBMonomial}
The support of a polynomial as a finite set (decidable version), obtained by filtering
the universe of $M$ for monomials $\alpha$ with $p(\alpha) \neq 0$.
\end{definition}

\begin{definition}[BB Transpose]
\label{def:BivariateBicycle.bbTranspose}
\lean{QEC1.BivariateBicycle.bbTranspose}
\leanok
\uses{def:BivariateBicycle.BBGroupAlgebra}
The transpose operation on the group algebra: $A^T(x,y) = A(x^{-1}, y^{-1})$.
Since $x^{-1} = x^{\ell-1}$ and $y^{-1} = y^{m-1}$ in the quotient ring,
this maps the coefficient of $x^a y^b$ to $x^{-a} y^{-b}$. Formally,
$\operatorname{bbTranspose}(p)(\alpha) = p(-\alpha)$.
\end{definition}

\begin{theorem}[Transpose is Involutive]
\label{thm:BivariateBicycle.bbTranspose_involutive}
\lean{QEC1.BivariateBicycle.bbTranspose_involutive}
\leanok
\uses{def:BivariateBicycle.bbTranspose}
For any polynomial $p$ in the group algebra,
$\operatorname{bbTranspose}(\operatorname{bbTranspose}(p)) = p$.
\end{theorem}

\begin{proof}
\leanok
\uses{def:BivariateBicycle.bbTranspose}
By extensionality, it suffices to show equality for an arbitrary $\alpha$.
By definition of transpose, $\operatorname{bbTranspose}(\operatorname{bbTranspose}(p))(\alpha) = \operatorname{bbTranspose}(p)(-\alpha) = p(-(-\alpha)) = p(\alpha)$,
using $-(-\alpha) = \alpha$.
\end{proof}

\begin{theorem}[Transpose of Monomial]
\label{thm:BivariateBicycle.bbTranspose_bbMonomial}
\lean{QEC1.BivariateBicycle.bbTranspose_bbMonomial}
\leanok
\uses{def:BivariateBicycle.bbTranspose, def:BivariateBicycle.bbMonomial}
For $a \in \mathbb{Z}_\ell$ and $b \in \mathbb{Z}_m$,
$\operatorname{bbTranspose}(\operatorname{bbMonomial}(a,b)) = \operatorname{bbMonomial}(-a, -b)$.
\end{theorem}

\begin{proof}
\leanok
\uses{def:BivariateBicycle.bbTranspose, def:BivariateBicycle.bbMonomial}
By extensionality on pairs $(c, d)$. We unfold the definitions of transpose and monomial.
The condition $-\alpha = (a, b)$ is equivalent to $\alpha = (-a, -b)$ by applying $\operatorname{neg\_eq\_iff\_eq\_neg}$.
\end{proof}

\begin{theorem}[Transpose Distributes over Addition]
\label{thm:BivariateBicycle.bbTranspose_add}
\lean{QEC1.BivariateBicycle.bbTranspose_add}
\leanok
\uses{def:BivariateBicycle.bbTranspose}
For polynomials $p, q$ in the group algebra,
$\operatorname{bbTranspose}(p + q) = \operatorname{bbTranspose}(p) + \operatorname{bbTranspose}(q)$.
\end{theorem}

\begin{proof}
\leanok
\uses{def:BivariateBicycle.bbTranspose}
By extensionality on $\alpha$. We compute
$\operatorname{bbTranspose}(p + q)(\alpha) = (p + q)(-\alpha) = p(-\alpha) + q(-\alpha)
= \operatorname{bbTranspose}(p)(\alpha) + \operatorname{bbTranspose}(q)(\alpha)$,
using the pointwise addition of functions.
\end{proof}

\begin{theorem}[Convolution is Commutative]
\label{thm:BivariateBicycle.bbConvolve_comm}
\lean{QEC1.BivariateBicycle.bbConvolve_comm}
\leanok
\uses{def:BivariateBicycle.bbConvolve}
For polynomials $p, q$ in the group algebra,
$p * q = q * p$
(convolution is commutative because the monomial group is abelian).
\end{theorem}

\begin{proof}
\leanok
\uses{def:BivariateBicycle.bbConvolve}
By extensionality on $\gamma$. We unfold the definition of convolution.
We reindex the left-hand side sum with $\beta = \gamma - \alpha$ (so $\alpha = \gamma - \beta$)
using the equivalence $\operatorname{Equiv.subLeft}(\gamma)$, obtaining
$\sum_\alpha p(\alpha) \cdot q(\gamma - \alpha) = \sum_\beta p(\gamma - \beta) \cdot q(\beta)$.
We then apply commutativity of multiplication in $\mathbb{Z}/2\mathbb{Z}$ to each term,
giving $\sum_\beta q(\beta) \cdot p(\gamma - \beta) = (q * p)(\gamma)$.
\end{proof}

\begin{definition}[BB Qubit Type]
\label{def:BivariateBicycle.BBQubit}
\lean{QEC1.BivariateBicycle.BBQubit}
\leanok
\uses{def:BivariateBicycle.BBMonomial}
The qubit type for a BB code: Left (L) qubits and Right (R) qubits,
each indexed by the monomial group $M = \mathbb{Z}_\ell \times \mathbb{Z}_m$.
Formally, $\operatorname{BBQubit}(\ell, m) = M \oplus M$.
\end{definition}

\begin{theorem}[BB Code Number of Qubits]
\label{thm:BivariateBicycle.bbCode_numQubits}
\lean{QEC1.BivariateBicycle.bbCode_numQubits}
\leanok
\uses{def:BivariateBicycle.BBQubit}
The number of physical qubits in a BB code is $|\operatorname{BBQubit}(\ell, m)| = 2\ell m$.
\end{theorem}

\begin{proof}
\leanok
\uses{def:BivariateBicycle.BBQubit, def:BivariateBicycle.BBMonomial}
By simplification using the cardinality of sum types, product types, and $|\mathbb{Z}_n| = n$.
We have $|\operatorname{BBQubit}(\ell, m)| = |M| + |M| = 2|M| = 2(\ell \cdot m)$.
\end{proof}

\begin{definition}[Pauli X on BB Code]
\label{def:BivariateBicycle.pauliXBB}
\lean{QEC1.BivariateBicycle.pauliXBB}
\leanok
\uses{def:PauliOp, def:BivariateBicycle.BBGroupAlgebra, def:BivariateBicycle.BBQubit}
The X-type Pauli operator $X(p, q)$ on a BB code.
It acts with $X$ on L qubit $\gamma$ iff $p(\gamma) = 1$, and on R qubit $\delta$ iff $q(\delta) = 1$.
This is a pure X-type operator ($z$-vector $= 0$). Formally:
$\operatorname{xVec} = \operatorname{Sum.elim}(p, q)$ and $\operatorname{zVec} = 0$.
\end{definition}

\begin{definition}[Pauli Z on BB Code]
\label{def:BivariateBicycle.pauliZBB}
\lean{QEC1.BivariateBicycle.pauliZBB}
\leanok
\uses{def:PauliOp, def:BivariateBicycle.BBGroupAlgebra, def:BivariateBicycle.BBQubit}
The Z-type Pauli operator $Z(p, q)$ on a BB code.
It acts with $Z$ on L qubit $\gamma$ iff $p(\gamma) = 1$, and on R qubit $\delta$ iff $q(\delta) = 1$.
This is a pure Z-type operator ($x$-vector $= 0$). Formally:
$\operatorname{xVec} = 0$ and $\operatorname{zVec} = \operatorname{Sum.elim}(p, q)$.
\end{definition}

\begin{definition}[BB Shift]
\label{def:BivariateBicycle.bbShift}
\lean{QEC1.BivariateBicycle.bbShift}
\leanok
\uses{def:BivariateBicycle.BBGroupAlgebra, def:BivariateBicycle.BBMonomial}
Left-shift a polynomial by monomial $\alpha$: $(\operatorname{shift}_\alpha\, p)(\gamma) = p(\gamma - \alpha)$.
This corresponds to multiplication by $x^a y^b$ in the group algebra.
\end{definition}

\begin{theorem}[Shift by Zero]
\label{thm:BivariateBicycle.bbShift_zero}
\lean{QEC1.BivariateBicycle.bbShift_zero}
\leanok
\uses{def:BivariateBicycle.bbShift}
For any polynomial $p$, $\operatorname{shift}_0\, p = p$.
\end{theorem}

\begin{proof}
\leanok
\uses{def:BivariateBicycle.bbShift}
By extensionality on $\gamma$. We have $\operatorname{shift}_0(p)(\gamma) = p(\gamma - 0) = p(\gamma)$
by simplification.
\end{proof}

\begin{theorem}[Shift Composition]
\label{thm:BivariateBicycle.bbShift_add}
\lean{QEC1.BivariateBicycle.bbShift_add}
\leanok
\uses{def:BivariateBicycle.bbShift}
For monomials $\alpha, \beta$ and polynomial $p$,
$\operatorname{shift}_\alpha(\operatorname{shift}_\beta\, p) = \operatorname{shift}_{\alpha + \beta}\, p$.
\end{theorem}

\begin{proof}
\leanok
\uses{def:BivariateBicycle.bbShift}
By extensionality on $\gamma$. Unfolding the definition, the left-hand side gives
$p((\gamma - \alpha) - \beta)$ and the right-hand side gives $p(\gamma - (\alpha + \beta))$.
These are equal since $(\gamma - \alpha) - \beta = \gamma - (\alpha + \beta)$ by the associativity of subtraction.
\end{proof}

\begin{definition}[BB X-Check]
\label{def:BivariateBicycle.bbCheckX}
\lean{QEC1.BivariateBicycle.bbCheckX}
\leanok
\uses{def:BivariateBicycle.pauliXBB, def:BivariateBicycle.bbShift, def:BivariateBicycle.BBGroupAlgebra, def:BivariateBicycle.BBMonomial, def:PauliOp}
The X-type check indexed by $\alpha \in M$ for polynomials $A, B$. Its X-support on L qubits is
the support of $\operatorname{shift}_\alpha A$, and on R qubits is the support of $\operatorname{shift}_\alpha B$:
\[
\operatorname{check}(\alpha, X) = X(\operatorname{shift}_\alpha A,\, \operatorname{shift}_\alpha B).
\]
\end{definition}

\begin{definition}[BB Z-Check]
\label{def:BivariateBicycle.bbCheckZ}
\lean{QEC1.BivariateBicycle.bbCheckZ}
\leanok
\uses{def:BivariateBicycle.pauliZBB, def:BivariateBicycle.bbShift, def:BivariateBicycle.bbTranspose, def:BivariateBicycle.BBGroupAlgebra, def:BivariateBicycle.BBMonomial, def:PauliOp}
The Z-type check indexed by $\beta \in M$ for polynomials $A, B$. Its Z-support on L qubits is
the support of $\operatorname{shift}_\beta B^T$, and on R qubits is the support of $\operatorname{shift}_\beta A^T$:
\[
\operatorname{check}(\beta, Z) = Z(\operatorname{shift}_\beta B^T,\, \operatorname{shift}_\beta A^T).
\]
\end{definition}

\begin{theorem}[X Checks are Pure X-Type]
\label{thm:BivariateBicycle.bbCheckX_pure_X}
\lean{QEC1.BivariateBicycle.bbCheckX_pure_X}
\leanok
\uses{def:BivariateBicycle.bbCheckX}
For any polynomials $A, B$ and index $\alpha \in M$, the X check $\operatorname{bbCheckX}(A, B, \alpha)$
has $z$-vector equal to $0$.
\end{theorem}

\begin{proof}
\leanok
\uses{def:BivariateBicycle.bbCheckX, def:BivariateBicycle.pauliXBB}
This holds by definitional equality (reflexivity), since the X check is constructed as
$\operatorname{pauliXBB}$ which has $\operatorname{zVec} = 0$.
\end{proof}

\begin{theorem}[Z Checks are Pure Z-Type]
\label{thm:BivariateBicycle.bbCheckZ_pure_Z}
\lean{QEC1.BivariateBicycle.bbCheckZ_pure_Z}
\leanok
\uses{def:BivariateBicycle.bbCheckZ}
For any polynomials $A, B$ and index $\beta \in M$, the Z check $\operatorname{bbCheckZ}(A, B, \beta)$
has $x$-vector equal to $0$.
\end{theorem}

\begin{proof}
\leanok
\uses{def:BivariateBicycle.bbCheckZ, def:BivariateBicycle.pauliZBB}
This holds by definitional equality (reflexivity), since the Z check is constructed as
$\operatorname{pauliZBB}$ which has $\operatorname{xVec} = 0$.
\end{proof}

\begin{theorem}[X Checks Commute]
\label{thm:BivariateBicycle.bbCheckX_commute}
\lean{QEC1.BivariateBicycle.bbCheckX_commute}
\leanok
\uses{def:BivariateBicycle.bbCheckX, def:PauliOp.PauliCommute, def:PauliOp.symplecticInner}
For any polynomials $A, B$ and indices $\alpha_1, \alpha_2 \in M$,
$\operatorname{bbCheckX}(A, B, \alpha_1)$ and $\operatorname{bbCheckX}(A, B, \alpha_2)$ commute.
\end{theorem}

\begin{proof}
\leanok
\uses{def:BivariateBicycle.bbCheckX, def:PauliOp.PauliCommute, def:PauliOp.symplecticInner}
Both checks are pure X-type, so their $z$-vectors are zero. The symplectic inner product
$\langle P_1, P_2 \rangle = \sum_i (x_1(i) \cdot z_2(i) + z_1(i) \cdot x_2(i))$
vanishes since all $z$-components are zero. By simplification using the definition of
$\operatorname{PauliCommute}$ and $\operatorname{symplecticInner}$ with $\operatorname{zVec} = 0$,
the result follows.
\end{proof}

\begin{theorem}[Z Checks Commute]
\label{thm:BivariateBicycle.bbCheckZ_commute}
\lean{QEC1.BivariateBicycle.bbCheckZ_commute}
\leanok
\uses{def:BivariateBicycle.bbCheckZ, def:PauliOp.PauliCommute, def:PauliOp.symplecticInner}
For any polynomials $A, B$ and indices $\beta_1, \beta_2 \in M$,
$\operatorname{bbCheckZ}(A, B, \beta_1)$ and $\operatorname{bbCheckZ}(A, B, \beta_2)$ commute.
\end{theorem}

\begin{proof}
\leanok
\uses{def:BivariateBicycle.bbCheckZ, def:PauliOp.PauliCommute, def:PauliOp.symplecticInner}
Both checks are pure Z-type, so their $x$-vectors are zero. The symplectic inner product
vanishes since all $x$-components are zero. By simplification using the definition of
$\operatorname{PauliCommute}$ and $\operatorname{symplecticInner}$ with $\operatorname{xVec} = 0$,
the result follows.
\end{proof}

\begin{theorem}[XZ Commutation Condition]
\label{thm:BivariateBicycle.bbCheckXZ_commute_iff}
\lean{QEC1.BivariateBicycle.bbCheckXZ_commute_iff}
\leanok
\uses{def:BivariateBicycle.bbCheckX, def:BivariateBicycle.bbCheckZ, def:PauliOp.PauliCommute}
The X check at $\alpha$ commutes with the Z check at $\beta$ if and only if
\[
\sum_{\gamma \in M} A(\gamma - \alpha) \cdot B(-(\gamma - \beta)) + \sum_{\delta \in M} B(\delta - \alpha) \cdot A(-(\delta - \beta)) = 0.
\]
This is the condition $H_X \cdot H_Z^T = 0$, equivalently $AB^T = BA^T$.
\end{theorem}

\begin{proof}
\leanok
\uses{def:BivariateBicycle.bbCheckX, def:BivariateBicycle.bbCheckZ, def:PauliOp.PauliCommute, def:PauliOp.symplecticInner}
We unfold the definitions of $\operatorname{PauliCommute}$ and $\operatorname{symplecticInner}$.
For both directions we convert using the sum decomposition over the sum type (L and R qubits),
and simplify the components using the definitions of $\operatorname{bbCheckX}$, $\operatorname{bbCheckZ}$,
$\operatorname{pauliXBB}$, $\operatorname{pauliZBB}$, $\operatorname{bbShift}$, and $\operatorname{bbTranspose}$.
\end{proof}

\begin{definition}[BB CSS Condition]
\label{def:BivariateBicycle.BBCSSCondition}
\lean{QEC1.BivariateBicycle.BBCSSCondition}
\leanok
\uses{def:BivariateBicycle.bbConvolve, def:BivariateBicycle.bbTranspose, def:BivariateBicycle.BBGroupAlgebra}
The CSS condition for a BB code states that $A * B^T = B * A^T$ as convolutions in the group algebra.
This is the necessary and sufficient condition for all X and Z checks to commute, ensuring the
parity check matrices satisfy $H_X \cdot H_Z^T = 0$.
\end{definition}

\begin{theorem}[XZ Checks Commute under CSS Condition]
\label{thm:BivariateBicycle.bbChecksXZ_commute_of_css}
\lean{QEC1.BivariateBicycle.bbChecksXZ_commute_of_css}
\leanok
\uses{def:BivariateBicycle.BBCSSCondition, def:BivariateBicycle.bbCheckX, def:BivariateBicycle.bbCheckZ, def:PauliOp.PauliCommute}
Under the CSS condition $AB^T = BA^T$, for all $\alpha, \beta \in M$,
the X check at $\alpha$ commutes with the Z check at $\beta$.
In fact, for BB codes over abelian groups the commutation is automatic:
each sum equals the convolution evaluated at $\beta - \alpha$, and by
commutativity of convolution their sum vanishes in characteristic $2$.
\end{theorem}

\begin{proof}
\leanok
\uses{def:BivariateBicycle.BBCSSCondition, def:BivariateBicycle.bbConvolve, thm:BivariateBicycle.bbCheckXZ_commute_iff, thm:BivariateBicycle.bbConvolve_comm}
We rewrite using the XZ commutation characterization. We need to show
$\sum_\gamma A(\gamma - \alpha) \cdot B(-(\gamma - \beta)) + \sum_\delta B(\delta - \alpha) \cdot A(-(\delta - \beta)) = 0$.
We establish that the first sum equals $\operatorname{bbConvolve}(A, B)(\beta - \alpha)$
by reindexing with $\delta = \gamma - \alpha$ using $\operatorname{Equiv.addRight}(\alpha)$
and simplifying $B(-(\gamma - \beta)) = B((\beta - \alpha) - \delta)$ by abelian group arithmetic.
Similarly, the second sum equals $\operatorname{bbConvolve}(B, A)(\beta - \alpha)$.
Rewriting using commutativity of convolution ($\operatorname{bbConvolve\_comm}$),
both sums are equal, so their sum in $\mathbb{Z}/2\mathbb{Z}$ is $0$ by $\operatorname{CharTwo.add\_self\_eq\_zero}$.
\end{proof}

\begin{definition}[BB Check Index]
\label{def:BivariateBicycle.BBCheckIndex}
\lean{QEC1.BivariateBicycle.BBCheckIndex}
\leanok
\uses{def:BivariateBicycle.BBMonomial}
The check index type for a BB code: X checks indexed by $M$ (left summand) and Z checks indexed by $M$ (right summand).
Formally, $\operatorname{BBCheckIndex}(\ell, m) = M \oplus M$.
\end{definition}

\begin{theorem}[BB Code Number of Checks]
\label{thm:BivariateBicycle.bbCode_numChecks}
\lean{QEC1.BivariateBicycle.bbCode_numChecks}
\leanok
\uses{def:BivariateBicycle.BBCheckIndex}
The number of checks in a BB code is $|\operatorname{BBCheckIndex}(\ell, m)| = 2\ell m$.
\end{theorem}

\begin{proof}
\leanok
\uses{def:BivariateBicycle.BBCheckIndex, def:BivariateBicycle.BBMonomial}
By simplification using the cardinality of sum types, product types, and $|\mathbb{Z}_n| = n$.
We have $|\operatorname{BBCheckIndex}(\ell, m)| = |M| + |M| = 2|M| = 2(\ell \cdot m)$.
\end{proof}

\begin{definition}[BB Check Map]
\label{def:BivariateBicycle.bbCheck}
\lean{QEC1.BivariateBicycle.bbCheck}
\leanok
\uses{def:BivariateBicycle.BBCheckIndex, def:BivariateBicycle.bbCheckX, def:BivariateBicycle.bbCheckZ, def:PauliOp, def:BivariateBicycle.BBQubit}
The check map for a BB code: given polynomials $A, B$, it maps X check indices (left summand) to
$\operatorname{bbCheckX}(A, B, \alpha)$ and Z check indices (right summand) to $\operatorname{bbCheckZ}(A, B, \beta)$.
\end{definition}

\begin{theorem}[All BB Checks Commute]
\label{thm:BivariateBicycle.bbChecks_commute}
\lean{QEC1.BivariateBicycle.bbChecks_commute}
\leanok
\uses{def:BivariateBicycle.bbCheck, def:BivariateBicycle.BBCSSCondition, def:PauliOp.PauliCommute}
Under the CSS condition $AB^T = BA^T$, all checks in a BB code pairwise commute.
That is, for all check indices $i, j$,
$\operatorname{PauliCommute}(\operatorname{bbCheck}(A, B, i),\, \operatorname{bbCheck}(A, B, j))$.
\end{theorem}

\begin{proof}
\leanok
\uses{thm:BivariateBicycle.bbCheckX_commute, thm:BivariateBicycle.bbCheckZ_commute, thm:BivariateBicycle.bbChecksXZ_commute_of_css, def:BivariateBicycle.bbCheck}
We case-split on the check indices $i$ and $j$.
\textbf{Case} $i = (\alpha_1, X)$, $j = (\alpha_2, X)$: This follows from $\operatorname{bbCheckX\_commute}$.
\textbf{Case} $i = (\alpha, X)$, $j = (\beta, Z)$: This follows from $\operatorname{bbChecksXZ\_commute\_of\_css}$.
\textbf{Case} $i = (\beta, Z)$, $j = (\alpha, X)$: We apply commutativity of $\operatorname{PauliCommute}$
and then use $\operatorname{bbChecksXZ\_commute\_of\_css}$.
\textbf{Case} $i = (\beta_1, Z)$, $j = (\beta_2, Z)$: This follows from $\operatorname{bbCheckZ\_commute}$.
\end{proof}

\begin{definition}[Bivariate Bicycle Code]
\label{def:BivariateBicycle.BBCode}
\lean{QEC1.BivariateBicycle.BBCode}
\leanok
\uses{def:StabilizerCode, def:BivariateBicycle.BBCheckIndex, def:BivariateBicycle.bbCheck, thm:BivariateBicycle.bbChecks_commute, def:BivariateBicycle.BBCSSCondition}
A BB code forms a valid stabilizer code under the CSS condition $AB^T = BA^T$.
Given polynomials $A, B$ in the group algebra satisfying the CSS condition, the stabilizer code has:
\begin{itemize}
\item Qubit type: $\operatorname{BBQubit}(\ell, m) = M \oplus M$,
\item Check index: $\operatorname{BBCheckIndex}(\ell, m) = M \oplus M$,
\item Check map: $\operatorname{bbCheck}(A, B)$,
\item Commutation: guaranteed by $\operatorname{bbChecks\_commute}$.
\end{itemize}
\end{definition}

\begin{definition}[X-Shift Monomial]
\label{def:BivariateBicycle.xShift}
\lean{QEC1.BivariateBicycle.xShift}
\leanok
\uses{def:BivariateBicycle.BBMonomial}
The $x$-shift operator: the monomial $x = x^1 y^0$ in the monomial group, represented as $(1, 0) \in \mathbb{Z}_\ell \times \mathbb{Z}_m$.
Shifting by this monomial maps the coefficient at $(a, b)$ to $(a-1, b)$.
\end{definition}

\begin{definition}[Y-Shift Monomial]
\label{def:BivariateBicycle.yShift}
\lean{QEC1.BivariateBicycle.yShift}
\leanok
\uses{def:BivariateBicycle.BBMonomial}
The $y$-shift operator: the monomial $y = x^0 y^1$ in the monomial group, represented as $(0, 1) \in \mathbb{Z}_\ell \times \mathbb{Z}_m$.
Shifting by this monomial maps the coefficient at $(a, b)$ to $(a, b-1)$.
\end{definition}

\begin{theorem}[$x^\ell = 1$]
\label{thm:BivariateBicycle.xShift_pow_ell}
\lean{QEC1.BivariateBicycle.xShift_pow_ℓ}
\leanok
\uses{def:BivariateBicycle.xShift}
Shifting by $\ell$ in the first coordinate is the identity modulo $\ell$:
$\ell \cdot \operatorname{xShift} = 0$ in the monomial group.
\end{theorem}

\begin{proof}
\leanok
\uses{def:BivariateBicycle.xShift}
By simplification. We compute $\ell \cdot (1, 0) = (\ell, 0) = (0, 0)$ in $\mathbb{Z}_\ell \times \mathbb{Z}_m$,
using $\ell \equiv 0 \pmod{\ell}$.
\end{proof}

\begin{theorem}[$y^m = 1$]
\label{thm:BivariateBicycle.yShift_pow_m}
\lean{QEC1.BivariateBicycle.yShift_pow_m}
\leanok
\uses{def:BivariateBicycle.yShift}
Shifting by $m$ in the second coordinate is the identity modulo $m$:
$m \cdot \operatorname{yShift} = 0$ in the monomial group.
\end{theorem}

\begin{proof}
\leanok
\uses{def:BivariateBicycle.yShift}
By simplification. We compute $m \cdot (0, 1) = (0, m) = (0, 0)$ in $\mathbb{Z}_\ell \times \mathbb{Z}_m$,
using $m \equiv 0 \pmod{m}$.
\end{proof}

\begin{theorem}[$x$ and $y$ Commute]
\label{thm:BivariateBicycle.xShift_yShift_comm}
\lean{QEC1.BivariateBicycle.xShift_yShift_comm}
\leanok
\uses{def:BivariateBicycle.xShift, def:BivariateBicycle.yShift}
The monomial group is abelian: $\operatorname{xShift} + \operatorname{yShift} = \operatorname{yShift} + \operatorname{xShift}$.
\end{theorem}

\begin{proof}
\leanok
\uses{def:BivariateBicycle.xShift, def:BivariateBicycle.yShift}
By simplification. We unfold the definitions to get $(1, 0) + (0, 1) = (0, 1) + (1, 0)$,
which holds by commutativity of addition in each component.
\end{proof}

\begin{theorem}[Transpose as Negation]
\label{thm:BivariateBicycle.bbTranspose_eq_neg}
\lean{QEC1.BivariateBicycle.bbTranspose_eq_neg}
\leanok
\uses{def:BivariateBicycle.bbTranspose}
The transpose satisfies $A^T(\gamma) = A(-\gamma)$, which corresponds to
substituting $x \to x^{-1} = x^{\ell-1}$ and $y \to y^{-1} = y^{m-1}$.
\end{theorem}

\begin{proof}
\leanok
\uses{def:BivariateBicycle.bbTranspose}
This holds by definitional equality (reflexivity).
\end{proof}

\begin{theorem}[Transpose of Monomial via Negation]
\label{thm:BivariateBicycle.bbTranspose_monomial_neg}
\lean{QEC1.BivariateBicycle.bbTranspose_monomial_neg}
\leanok
\uses{def:BivariateBicycle.bbTranspose, def:BivariateBicycle.bbMonomial}
Transpose of a monomial: $(x^a y^b)^T = x^{-a} y^{-b}$. Formally, for all $\gamma$,
$\operatorname{bbTranspose}(\operatorname{bbMonomial}(a, b))(\gamma) = \operatorname{bbMonomial}(-a, -b)(\gamma)$.
\end{theorem}

\begin{proof}
\leanok
\uses{def:BivariateBicycle.bbTranspose, def:BivariateBicycle.bbMonomial}
We unfold the definitions of transpose and monomial and split on the conditional.
There are four cases depending on whether $-\gamma = (a, b)$ and whether $\gamma = (-a, -b)$.
When $-\gamma = (a, b)$ but $\gamma \neq (-a, -b)$, we derive a contradiction from
$\gamma = -(-\gamma) = -(a,b) = (-a, -b)$ by double negation.
The symmetric case is handled by applying negation to the pair equality.
The two remaining cases (both true or both false) are immediate.
\end{proof}

\begin{theorem}[Convolution Distributes over Transposition]
\label{thm:BivariateBicycle.bbConvolve_transpose}
\lean{QEC1.BivariateBicycle.bbConvolve_transpose}
\leanok
\uses{def:BivariateBicycle.bbConvolve, def:BivariateBicycle.bbTranspose}
For polynomials $p, q$ in the group algebra,
$\operatorname{bbTranspose}(p * q) = q^T * p^T$.
\end{theorem}

\begin{proof}
\leanok
\uses{def:BivariateBicycle.bbConvolve, def:BivariateBicycle.bbTranspose}
By extensionality on $\gamma$. The left-hand side is
$(p * q)(-\gamma) = \sum_\alpha p(\alpha) \cdot q(-\gamma - \alpha)$.
We reindex with $\beta = -\gamma - \alpha$ (so $\alpha = -\gamma - \beta$) using $\operatorname{Equiv.subLeft}(-\gamma)$,
obtaining $\sum_\beta p(-\gamma - \beta) \cdot q(\beta)$.
The right-hand side is $\sum_\alpha q(-\alpha) \cdot p(-(\gamma - \alpha))$.
We reindex with $\beta = -\alpha$ (so $\alpha = -\beta$) using $\operatorname{Equiv.neg}$,
obtaining $\sum_\beta q(\beta) \cdot p(-\gamma - \beta)$.
The two expressions are equal by commutativity of multiplication in $\mathbb{Z}/2\mathbb{Z}$.
\end{proof}

\begin{definition}[Left Qubit]
\label{def:BivariateBicycle.leftQubit}
\lean{QEC1.BivariateBicycle.leftQubit}
\leanok
\uses{def:BivariateBicycle.BBQubit, def:BivariateBicycle.BBMonomial}
Left qubit labeled by $\gamma \in M$, defined as $\operatorname{Sum.inl}(\gamma)$ in the qubit type $M \oplus M$.
\end{definition}

\begin{definition}[Right Qubit]
\label{def:BivariateBicycle.rightQubit}
\lean{QEC1.BivariateBicycle.rightQubit}
\leanok
\uses{def:BivariateBicycle.BBQubit, def:BivariateBicycle.BBMonomial}
Right qubit labeled by $\delta \in M$, defined as $\operatorname{Sum.inr}(\delta)$ in the qubit type $M \oplus M$.
\end{definition}

\begin{definition}[X Check Index]
\label{def:BivariateBicycle.xCheckIndex}
\lean{QEC1.BivariateBicycle.xCheckIndex}
\leanok
\uses{def:BivariateBicycle.BBCheckIndex, def:BivariateBicycle.BBMonomial}
X check labeled by $\alpha \in M$, defined as $\operatorname{Sum.inl}(\alpha)$ in the check index type $M \oplus M$.
\end{definition}

\begin{definition}[Z Check Index]
\label{def:BivariateBicycle.zCheckIndex}
\lean{QEC1.BivariateBicycle.zCheckIndex}
\leanok
\uses{def:BivariateBicycle.BBCheckIndex, def:BivariateBicycle.BBMonomial}
Z check labeled by $\beta \in M$, defined as $\operatorname{Sum.inr}(\beta)$ in the check index type $M \oplus M$.
\end{definition}

\begin{theorem}[CSS Condition Characterization]
\label{thm:BivariateBicycle.bbCSSCondition_iff}
\lean{QEC1.BivariateBicycle.bbCSSCondition_iff}
\leanok
\uses{def:BivariateBicycle.BBCSSCondition, def:BivariateBicycle.bbConvolve, def:BivariateBicycle.bbTranspose}
The CSS condition $AB^T = BA^T$ is equivalent to: for every $\gamma \in M$,
\[
\sum_{\alpha \in M} A(\alpha) \cdot B(\alpha - \gamma) = \sum_{\alpha \in M} B(\alpha) \cdot A(\alpha - \gamma).
\]
This ensures the parity check matrices $H_X$ and $H_Z$ satisfy $H_X \cdot H_Z^T = 0$.
\end{theorem}

\begin{proof}
\leanok
\uses{def:BivariateBicycle.BBCSSCondition, def:BivariateBicycle.bbConvolve, def:BivariateBicycle.bbTranspose}
We unfold the definitions of $\operatorname{BBCSSCondition}$, $\operatorname{bbConvolve}$, and $\operatorname{bbTranspose}$,
and use function extensionality.
For both directions, the key step is converting between the convolution form
$\sum_\alpha p(\alpha) \cdot q(-(\gamma - \alpha))$ and the stated form $\sum_\alpha p(\alpha) \cdot q(\alpha - \gamma)$
by the abelian group identity $-(\gamma - \alpha) = \alpha - \gamma$.
\end{proof}

\begin{theorem}[$X(p,q)$ is Pure X-Type]
\label{thm:BivariateBicycle.pauliXBB_pure_X}
\lean{QEC1.BivariateBicycle.pauliXBB_pure_X}
\leanok
\uses{def:BivariateBicycle.pauliXBB}
For any polynomials $p, q$, the operator $X(p, q)$ satisfies $\operatorname{zVec} = 0$.
\end{theorem}

\begin{proof}
\leanok
\uses{def:BivariateBicycle.pauliXBB}
This holds by definitional equality (reflexivity).
\end{proof}

\begin{theorem}[$Z(p,q)$ is Pure Z-Type]
\label{thm:BivariateBicycle.pauliZBB_pure_Z}
\lean{QEC1.BivariateBicycle.pauliZBB_pure_Z}
\leanok
\uses{def:BivariateBicycle.pauliZBB}
For any polynomials $p, q$, the operator $Z(p, q)$ satisfies $\operatorname{xVec} = 0$.
\end{theorem}

\begin{proof}
\leanok
\uses{def:BivariateBicycle.pauliZBB}
This holds by definitional equality (reflexivity).
\end{proof}

\begin{theorem}[$X(0,0) = I$]
\label{thm:BivariateBicycle.pauliXBB_zero}
\lean{QEC1.BivariateBicycle.pauliXBB_zero}
\leanok
\uses{def:BivariateBicycle.pauliXBB, def:PauliOp}
The operator $X(0, 0)$ is the identity Pauli operator.
\end{theorem}

\begin{proof}
\leanok
\uses{def:BivariateBicycle.pauliXBB}
By extensionality on qubit indices. For the $x$-vector, we case-split on left and right qubits:
in both cases, $\operatorname{Sum.elim}(0, 0)$ evaluates to $0$, matching the identity.
For the $z$-vector, $\operatorname{pauliXBB}$ has $\operatorname{zVec} = 0$ by definition, matching the identity.
\end{proof}

\begin{theorem}[$Z(0,0) = I$]
\label{thm:BivariateBicycle.pauliZBB_zero}
\lean{QEC1.BivariateBicycle.pauliZBB_zero}
\leanok
\uses{def:BivariateBicycle.pauliZBB, def:PauliOp}
The operator $Z(0, 0)$ is the identity Pauli operator.
\end{theorem}

\begin{proof}
\leanok
\uses{def:BivariateBicycle.pauliZBB}
By extensionality on qubit indices. For the $x$-vector, $\operatorname{pauliZBB}$ has $\operatorname{xVec} = 0$ by definition.
For the $z$-vector, we case-split on left and right qubits:
in both cases, $\operatorname{Sum.elim}(0, 0)$ evaluates to $0$, matching the identity.
\end{proof}

\begin{theorem}[Multiplication of X-Type Operators]
\label{thm:BivariateBicycle.pauliXBB_mul}
\lean{QEC1.BivariateBicycle.pauliXBB_mul}
\leanok
\uses{def:BivariateBicycle.pauliXBB, def:PauliOp.mul}
$X(p_1, q_1) \cdot X(p_2, q_2) = X(p_1 + p_2, q_1 + q_2)$
as Pauli operators (X-type operators multiply by adding their binary vectors).
\end{theorem}

\begin{proof}
\leanok
\uses{def:BivariateBicycle.pauliXBB, def:PauliOp.mul}
By extensionality on qubit indices $v$. For the $x$-vector, we case-split:
for left qubits $\operatorname{Sum.inl}(\gamma)$, both sides give $p_1(\gamma) + p_2(\gamma)$;
for right qubits $\operatorname{Sum.inr}(\delta)$, both sides give $q_1(\delta) + q_2(\delta)$.
For the $z$-vector, we case-split similarly: both sides give $0 + 0 = 0$ for each qubit, since both operators are pure X-type.
\end{proof}

\begin{theorem}[Multiplication of Z-Type Operators]
\label{thm:BivariateBicycle.pauliZBB_mul}
\lean{QEC1.BivariateBicycle.pauliZBB_mul}
\leanok
\uses{def:BivariateBicycle.pauliZBB, def:PauliOp.mul}
$Z(p_1, q_1) \cdot Z(p_2, q_2) = Z(p_1 + p_2, q_1 + q_2)$.
\end{theorem}

\begin{proof}
\leanok
\uses{def:BivariateBicycle.pauliZBB, def:PauliOp.mul}
By extensionality on qubit indices $v$. For the $x$-vector, we case-split:
both sides give $0 + 0 = 0$ for each qubit, since both operators are pure Z-type.
For the $z$-vector, we case-split:
for left qubits $\operatorname{Sum.inl}(\gamma)$, both sides give $p_1(\gamma) + p_2(\gamma)$;
for right qubits $\operatorname{Sum.inr}(\delta)$, both sides give $q_1(\delta) + q_2(\delta)$.
\end{proof}

\begin{theorem}[X-Type Operators Commute]
\label{thm:BivariateBicycle.pauliXBB_pauliXBB_commute}
\lean{QEC1.BivariateBicycle.pauliXBB_pauliXBB_commute}
\leanok
\uses{def:BivariateBicycle.pauliXBB, def:PauliOp.PauliCommute, def:PauliOp.symplecticInner}
Two X-type Pauli operators always commute: $\operatorname{PauliCommute}(X(p_1, q_1), X(p_2, q_2))$.
\end{theorem}

\begin{proof}
\leanok
\uses{def:BivariateBicycle.pauliXBB, def:PauliOp.PauliCommute, def:PauliOp.symplecticInner}
By simplification. Both operators have $\operatorname{zVec} = 0$, so the symplectic inner product
$\sum_i (x_1(i) \cdot z_2(i) + z_1(i) \cdot x_2(i)) = \sum_i (x_1(i) \cdot 0 + 0 \cdot x_2(i)) = 0$.
\end{proof}

\begin{theorem}[Z-Type Operators Commute]
\label{thm:BivariateBicycle.pauliZBB_pauliZBB_commute}
\lean{QEC1.BivariateBicycle.pauliZBB_pauliZBB_commute}
\leanok
\uses{def:BivariateBicycle.pauliZBB, def:PauliOp.PauliCommute, def:PauliOp.symplecticInner}
Two Z-type Pauli operators always commute: $\operatorname{PauliCommute}(Z(p_1, q_1), Z(p_2, q_2))$.
\end{theorem}

\begin{proof}
\leanok
\uses{def:BivariateBicycle.pauliZBB, def:PauliOp.PauliCommute, def:PauliOp.symplecticInner}
By simplification. Both operators have $\operatorname{xVec} = 0$, so the symplectic inner product
$\sum_i (x_1(i) \cdot z_2(i) + z_1(i) \cdot x_2(i)) = \sum_i (0 \cdot z_2(i) + z_1(i) \cdot 0) = 0$.
\end{proof}

\begin{theorem}[Symplectic Inner Product of X and Z Operators]
\label{thm:BivariateBicycle.symplecticInner_pauliXBB_pauliZBB}
\lean{QEC1.BivariateBicycle.symplecticInner_pauliXBB_pauliZBB}
\leanok
\uses{def:BivariateBicycle.pauliXBB, def:BivariateBicycle.pauliZBB, def:PauliOp.symplecticInner}
The symplectic inner product of $X(p_1, q_1)$ and $Z(p_2, q_2)$ equals
\[
\sum_{\gamma \in M} p_1(\gamma) \cdot p_2(\gamma) + \sum_{\delta \in M} q_1(\delta) \cdot q_2(\delta),
\]
which is $\langle p_1, p_2 \rangle + \langle q_1, q_2 \rangle$ where $\langle \cdot, \cdot \rangle$ is the standard inner product in $\mathbb{Z}/2\mathbb{Z}$.
\end{theorem}

\begin{proof}
\leanok
\uses{def:BivariateBicycle.pauliXBB, def:BivariateBicycle.pauliZBB, def:PauliOp.symplecticInner}
We unfold the definitions of $\operatorname{symplecticInner}$, $\operatorname{pauliXBB}$, and $\operatorname{pauliZBB}$.
Since $X(p_1, q_1)$ has $\operatorname{zVec} = 0$ and $Z(p_2, q_2)$ has $\operatorname{xVec} = 0$,
the cross terms vanish. The remaining sum over $\operatorname{BBQubit}$ decomposes as a sum over
left qubits plus a sum over right qubits (using $\operatorname{Fintype.sum\_sum\_type}$).
Simplifying the $\operatorname{Sum.elim}$ applications gives the stated formula.
\end{proof}

\begin{theorem}[Monomial Group Cardinality]
\label{thm:BivariateBicycle.bbMonomial_card}
\lean{QEC1.BivariateBicycle.bbMonomial_card}
\leanok
\uses{def:BivariateBicycle.BBMonomial}
The monomial group has $\ell m$ elements: $|M| = \ell \cdot m$.
\end{theorem}

\begin{proof}
\leanok
\uses{def:BivariateBicycle.BBMonomial}
By simplification using the cardinality of product types and $|\mathbb{Z}_n| = n$.
\end{proof}

\begin{theorem}[X Check Acts on Left Qubit]
\label{thm:BivariateBicycle.bbCheckX_acts_on_left}
\lean{QEC1.BivariateBicycle.bbCheckX_acts_on_left}
\leanok
\uses{def:BivariateBicycle.bbCheckX}
The X check $(\alpha, X)$ acts on L qubit $\gamma$ with coefficient $A(\gamma - \alpha)$,
i.e., $\gamma$ is in the support of the shifted polynomial $\operatorname{shift}_\alpha A$.
\end{theorem}

\begin{proof}
\leanok
\uses{def:BivariateBicycle.bbCheckX}
This holds by definitional equality (reflexivity).
\end{proof}

\begin{theorem}[X Check Acts on Right Qubit]
\label{thm:BivariateBicycle.bbCheckX_acts_on_right}
\lean{QEC1.BivariateBicycle.bbCheckX_acts_on_right}
\leanok
\uses{def:BivariateBicycle.bbCheckX}
The X check $(\alpha, X)$ acts on R qubit $\delta$ with coefficient $B(\delta - \alpha)$,
i.e., $\delta$ is in the support of the shifted polynomial $\operatorname{shift}_\alpha B$.
\end{theorem}

\begin{proof}
\leanok
\uses{def:BivariateBicycle.bbCheckX}
This holds by definitional equality (reflexivity).
\end{proof}

\begin{theorem}[Z Check Acts on Left Qubit]
\label{thm:BivariateBicycle.bbCheckZ_acts_on_left}
\lean{QEC1.BivariateBicycle.bbCheckZ_acts_on_left}
\leanok
\uses{def:BivariateBicycle.bbCheckZ}
The Z check $(\beta, Z)$ acts on L qubit $\gamma$ with coefficient $B(\beta - \gamma)$,
i.e., $\gamma$ is in the support of the shifted polynomial $\operatorname{shift}_\beta B^T$.
\end{theorem}

\begin{proof}
\leanok
\uses{def:BivariateBicycle.bbCheckZ, def:BivariateBicycle.pauliZBB, def:BivariateBicycle.bbShift, def:BivariateBicycle.bbTranspose}
By simplification using the definitions of $\operatorname{bbCheckZ}$, $\operatorname{pauliZBB}$,
$\operatorname{bbShift}$, and $\operatorname{bbTranspose}$, together with the identity $-(\gamma - \beta) = \beta - \gamma$.
\end{proof}

\begin{theorem}[Z Check Acts on Right Qubit]
\label{thm:BivariateBicycle.bbCheckZ_acts_on_right}
\lean{QEC1.BivariateBicycle.bbCheckZ_acts_on_right}
\leanok
\uses{def:BivariateBicycle.bbCheckZ}
The Z check $(\beta, Z)$ acts on R qubit $\delta$ with coefficient $A(\beta - \delta)$,
i.e., $\delta$ is in the support of the shifted polynomial $\operatorname{shift}_\beta A^T$.
\end{theorem}

\begin{proof}
\leanok
\uses{def:BivariateBicycle.bbCheckZ, def:BivariateBicycle.pauliZBB, def:BivariateBicycle.bbShift, def:BivariateBicycle.bbTranspose}
By simplification using the definitions of $\operatorname{bbCheckZ}$, $\operatorname{pauliZBB}$,
$\operatorname{bbShift}$, and $\operatorname{bbTranspose}$, together with the identity $-(\delta - \beta) = \beta - \delta$.
\end{proof}

\begin{theorem}[X Checks are Self-Inverse]
\label{thm:BivariateBicycle.bbCheckX_mul_self}
\lean{QEC1.BivariateBicycle.bbCheckX_mul_self}
\leanok
\uses{def:BivariateBicycle.bbCheckX}
For any polynomials $A, B$ and index $\alpha \in M$,
$\operatorname{bbCheckX}(A, B, \alpha) \cdot \operatorname{bbCheckX}(A, B, \alpha) = I$.
\end{theorem}

\begin{proof}
\leanok
\uses{def:BivariateBicycle.bbCheckX, thm:BivariateBicycle.pauliXBB_mul, def:BivariateBicycle.pauliXBB}
We rewrite the X check as $\operatorname{pauliXBB}(\operatorname{shift}_\alpha A, \operatorname{shift}_\alpha B)$
and apply the multiplication rule for X-type operators to get
$X(\operatorname{shift}_\alpha A + \operatorname{shift}_\alpha A, \operatorname{shift}_\alpha B + \operatorname{shift}_\alpha B)$.
By extensionality on qubit indices: for the $x$-vector, on each qubit (left or right),
$f(\gamma) + f(\gamma) = 0$ in $\mathbb{Z}/2\mathbb{Z}$ by $\operatorname{CharTwo.add\_self\_eq\_zero}$,
matching the identity operator. For the $z$-vector, both sides are $0$.
\end{proof}

\begin{theorem}[Z Checks are Self-Inverse]
\label{thm:BivariateBicycle.bbCheckZ_mul_self}
\lean{QEC1.BivariateBicycle.bbCheckZ_mul_self}
\leanok
\uses{def:BivariateBicycle.bbCheckZ}
For any polynomials $A, B$ and index $\beta \in M$,
$\operatorname{bbCheckZ}(A, B, \beta) \cdot \operatorname{bbCheckZ}(A, B, \beta) = I$.
\end{theorem}

\begin{proof}
\leanok
\uses{def:BivariateBicycle.bbCheckZ, thm:BivariateBicycle.pauliZBB_mul, def:BivariateBicycle.pauliZBB}
We rewrite the Z check as $\operatorname{pauliZBB}(\operatorname{shift}_\beta B^T, \operatorname{shift}_\beta A^T)$
and apply the multiplication rule for Z-type operators to get
$Z(\operatorname{shift}_\beta B^T + \operatorname{shift}_\beta B^T, \operatorname{shift}_\beta A^T + \operatorname{shift}_\beta A^T)$.
By extensionality on qubit indices: for the $x$-vector, both sides are $0$.
For the $z$-vector, on each qubit (left or right),
$f(\gamma) + f(\gamma) = 0$ in $\mathbb{Z}/2\mathbb{Z}$ by $\operatorname{CharTwo.add\_self\_eq\_zero}$,
matching the identity operator.
\end{proof}

%--- Rem_20: GrossCodeDefinition ---
\chapter{Rem 20: The Gross Code Definition}

The \textbf{Gross code} is a $[\![144, 12, 12]\!]$ bivariate bicycle (BB) code with parameters $\ell = 12$, $m = 6$. The polynomials defining the code are $A = x^3 + y^2 + y$ and $B = y^3 + x^2 + x$ in $\mathbb{F}_2[x,y]/(x^{12}-1, y^6-1)$. This section defines the Gross code, its logical operators, and establishes key properties including weight, type purity, and centralizer membership.

\begin{definition}[Gross Monomial Group]
\label{def:GrossCode.GrossMonomial}
\lean{QEC1.BivariateBicycle.GrossCode.GrossMonomial}
\leanok
\uses{def:BivariateBicycle.BBMonomial}
The monomial group for the Gross code is $\operatorname{BBMonomial}(12, 6) = \mathbb{Z}_{12} \times \mathbb{Z}_6$.
\end{definition}

\begin{definition}[Gross Algebra]
\label{def:GrossCode.GrossAlgebra}
\lean{QEC1.BivariateBicycle.GrossCode.GrossAlgebra}
\leanok
\uses{def:BivariateBicycle.BBGroupAlgebra}
The group algebra for the Gross code is $\operatorname{BBGroupAlgebra}(12, 6) = \mathbb{F}_2[x,y]/(x^{12}-1, y^6-1)$.
\end{definition}

\begin{definition}[Gross Qubit Type]
\label{def:GrossCode.GrossQubit}
\lean{QEC1.BivariateBicycle.GrossCode.GrossQubit}
\leanok
\uses{def:BivariateBicycle.BBQubit}
The qubit type for the Gross code is $\operatorname{BBQubit}(12, 6)$, consisting of left ($L$) and right ($R$) qubits indexed by elements of $\mathbb{Z}_{12} \times \mathbb{Z}_6$.
\end{definition}

\begin{definition}[Polynomial $A$]
\label{def:GrossCode.grossA}
\lean{QEC1.BivariateBicycle.GrossCode.grossA}
\leanok
\uses{def:BivariateBicycle.bbMonomial, def:GrossCode.GrossAlgebra}
The polynomial $A \in \mathbb{F}_2[x,y]/(x^{12}-1, y^6-1)$ is defined as
\[
  A = x^3 + y^2 + y.
\]
\end{definition}

\begin{definition}[Polynomial $B$]
\label{def:GrossCode.grossB}
\lean{QEC1.BivariateBicycle.GrossCode.grossB}
\leanok
\uses{def:BivariateBicycle.bbMonomial, def:GrossCode.GrossAlgebra}
The polynomial $B \in \mathbb{F}_2[x,y]/(x^{12}-1, y^6-1)$ is defined as
\[
  B = y^3 + x^2 + x.
\]
\end{definition}

\begin{theorem}[X and Z Checks Commute]
\label{thm:GrossCode.grossChecks_XZ_commute}
\lean{QEC1.BivariateBicycle.GrossCode.grossChecks_XZ_commute}
\leanok
\uses{def:GrossCode.grossA, def:GrossCode.grossB, def:BivariateBicycle.bbCheckX, def:BivariateBicycle.bbCheckZ, thm:BivariateBicycle.bbCheckXZ_commute_iff, def:PauliOp.PauliCommute}
For all $\alpha, \beta \in \mathbb{Z}_{12} \times \mathbb{Z}_6$, the X-check $\operatorname{bbCheckX}(A, B, \alpha)$ and the Z-check $\operatorname{bbCheckZ}(A, B, \beta)$ commute as Pauli operators.
\end{theorem}

\begin{proof}
\leanok
\uses{thm:BivariateBicycle.bbCheckXZ_commute_iff}
We rewrite using the characterization that X and Z checks commute if and only if a certain convolution sum vanishes. This convolution condition is then verified by native computation.
\end{proof}

\begin{theorem}[All Gross Checks Commute]
\label{thm:GrossCode.grossChecks_commute}
\lean{QEC1.BivariateBicycle.GrossCode.grossChecks_commute}
\leanok
\uses{def:GrossCode.grossA, def:GrossCode.grossB, def:BivariateBicycle.bbCheck, def:BivariateBicycle.BBCheckIndex, def:PauliOp.PauliCommute, thm:BivariateBicycle.bbCheckX_commute, thm:BivariateBicycle.bbCheckZ_commute, thm:GrossCode.grossChecks_XZ_commute}
For all check indices $i, j$, the checks $\operatorname{bbCheck}(A, B, i)$ and $\operatorname{bbCheck}(A, B, j)$ commute.
\end{theorem}

\begin{proof}
\leanok
\uses{thm:BivariateBicycle.bbCheckX_commute, thm:BivariateBicycle.bbCheckZ_commute, thm:GrossCode.grossChecks_XZ_commute}
We case-split on whether each check index is an X-check ($\operatorname{inl}$) or a Z-check ($\operatorname{inr}$):
\begin{itemize}
\item Case $(\operatorname{inl}~\alpha_1, \operatorname{inl}~\alpha_2)$: Both are X-checks, so they commute by the general BB X-check commutativity theorem.
\item Case $(\operatorname{inl}~\alpha, \operatorname{inr}~\beta)$: This follows from the XZ commutation theorem above.
\item Case $(\operatorname{inr}~\beta, \operatorname{inl}~\alpha)$: By commutativity of PauliCommute, this reduces to the XZ case.
\item Case $(\operatorname{inr}~\beta_1, \operatorname{inr}~\beta_2)$: Both are Z-checks, so they commute by the general BB Z-check commutativity theorem.
\end{itemize}
\end{proof}

\begin{definition}[The Gross Code]
\label{def:GrossCode.grossCode}
\lean{QEC1.BivariateBicycle.GrossCode.grossCode}
\leanok
\uses{def:StabilizerCode, def:BivariateBicycle.BBCheckIndex, def:BivariateBicycle.bbCheck, def:GrossCode.grossA, def:GrossCode.grossB, thm:GrossCode.grossChecks_commute}
The Gross code is a stabilizer code with check index set $\operatorname{BBCheckIndex}(12,6)$, check map $\operatorname{bbCheck}(A, B)$, and the proof that all checks pairwise commute.
\end{definition}

\begin{theorem}[Gross Code Has 144 Qubits]
\label{thm:GrossCode.grossCode_numQubits}
\lean{QEC1.BivariateBicycle.GrossCode.grossCode_numQubits}
\leanok
\uses{def:GrossCode.grossCode, def:StabilizerCode.numQubits}
The Gross code has $n = 144$ physical qubits: $\operatorname{grossCode.numQubits} = 144$.
\end{theorem}

\begin{proof}
\leanok
\uses{def:GrossCode.grossCode, def:StabilizerCode.numQubits}
By simplification using the definition of $\operatorname{numQubits}$ as the cardinality of the qubit type $\operatorname{BBQubit}(12,6) = (\mathbb{Z}_{12} \times \mathbb{Z}_6) \oplus (\mathbb{Z}_{12} \times \mathbb{Z}_6)$, we compute $|\mathbb{Z}_{12}| \cdot |\mathbb{Z}_6| \cdot 2 = 12 \cdot 6 \cdot 2 = 144$ by numerical computation.
\end{proof}

\begin{theorem}[Gross Code Has 144 Checks]
\label{thm:GrossCode.grossCode_numChecks}
\lean{QEC1.BivariateBicycle.GrossCode.grossCode_numChecks}
\leanok
\uses{def:GrossCode.grossCode, def:StabilizerCode.numChecks}
The Gross code has $144$ checks: $\operatorname{grossCode.numChecks} = 144$.
\end{theorem}

\begin{proof}
\leanok
\uses{def:GrossCode.grossCode, def:StabilizerCode.numChecks}
By simplification using the definition of $\operatorname{numChecks}$ as the cardinality of $\operatorname{BBCheckIndex}(12,6) = (\mathbb{Z}_{12} \times \mathbb{Z}_6) \oplus (\mathbb{Z}_{12} \times \mathbb{Z}_6)$, we compute $12 \cdot 6 \cdot 2 = 144$ by numerical computation.
\end{proof}

\begin{definition}[Polynomial $f$]
\label{def:GrossCode.grossF}
\lean{QEC1.BivariateBicycle.GrossCode.grossF}
\leanok
\uses{def:BivariateBicycle.bbMonomial, def:GrossCode.GrossAlgebra}
The logical operator polynomial $f \in \mathbb{F}_2[x,y]/(x^{12}-1, y^6-1)$ is defined as
\[
  f = 1 + x + x^2 + x^3 + x^6 + x^7 + x^8 + x^9 + (x + x^5 + x^7 + x^{11})y^3.
\]
\end{definition}

\begin{definition}[Polynomial $g$]
\label{def:GrossCode.grossG}
\lean{QEC1.BivariateBicycle.GrossCode.grossG}
\leanok
\uses{def:BivariateBicycle.bbMonomial, def:GrossCode.GrossAlgebra}
The logical operator polynomial $g \in \mathbb{F}_2[x,y]/(x^{12}-1, y^6-1)$ is defined as
\[
  g = x + x^2 y + (1+x)y^2 + x^2 y^3 + y^4.
\]
\end{definition}

\begin{definition}[Polynomial $h$]
\label{def:GrossCode.grossH}
\lean{QEC1.BivariateBicycle.GrossCode.grossH}
\leanok
\uses{def:BivariateBicycle.bbMonomial, def:GrossCode.GrossAlgebra}
The logical operator polynomial $h \in \mathbb{F}_2[x,y]/(x^{12}-1, y^6-1)$ is defined as
\[
  h = 1 + (1+x)y + y^2 + (1+x)y^3.
\]
\end{definition}

\begin{definition}[Logical $\bar{X}_\alpha$]
\label{def:GrossCode.logicalXBar}
\lean{QEC1.BivariateBicycle.GrossCode.logicalXBar}
\leanok
\uses{def:BivariateBicycle.pauliXBB, def:BivariateBicycle.bbShift, def:GrossCode.grossF, def:GrossCode.GrossMonomial, def:GrossCode.GrossQubit, def:PauliOp}
For $\alpha \in \mathbb{Z}_{12} \times \mathbb{Z}_6$, the logical X operator is
\[
  \bar{X}_\alpha = X(\alpha \cdot f,\; 0),
\]
where $\alpha \cdot f$ denotes the shift of $f$ by $\alpha$.
\end{definition}

\begin{definition}[Logical $\bar{X}'_\beta$]
\label{def:GrossCode.logicalXBar'}
\lean{QEC1.BivariateBicycle.GrossCode.logicalXBar'}
\leanok
\uses{def:BivariateBicycle.pauliXBB, def:BivariateBicycle.bbShift, def:GrossCode.grossG, def:GrossCode.grossH, def:GrossCode.GrossMonomial, def:GrossCode.GrossQubit, def:PauliOp}
For $\beta \in \mathbb{Z}_{12} \times \mathbb{Z}_6$, the logical X$'$ operator is
\[
  \bar{X}'_\beta = X(\beta \cdot g,\; \beta \cdot h).
\]
\end{definition}

\begin{definition}[Logical $\bar{Z}_\beta$]
\label{def:GrossCode.logicalZBar}
\lean{QEC1.BivariateBicycle.GrossCode.logicalZBar}
\leanok
\uses{def:BivariateBicycle.pauliZBB, def:BivariateBicycle.bbShift, def:BivariateBicycle.bbTranspose, def:GrossCode.grossH, def:GrossCode.grossG, def:GrossCode.GrossMonomial, def:GrossCode.GrossQubit, def:PauliOp}
For $\beta \in \mathbb{Z}_{12} \times \mathbb{Z}_6$, the logical Z operator is
\[
  \bar{Z}_\beta = Z(\beta \cdot h^T,\; \beta \cdot g^T),
\]
where $h^T$ and $g^T$ denote the transposes (negation of exponents) of $h$ and $g$.
\end{definition}

\begin{definition}[Logical $\bar{Z}'_\alpha$]
\label{def:GrossCode.logicalZBar'}
\lean{QEC1.BivariateBicycle.GrossCode.logicalZBar'}
\leanok
\uses{def:BivariateBicycle.pauliZBB, def:BivariateBicycle.bbShift, def:BivariateBicycle.bbTranspose, def:GrossCode.grossF, def:GrossCode.GrossMonomial, def:GrossCode.GrossQubit, def:PauliOp}
For $\alpha \in \mathbb{Z}_{12} \times \mathbb{Z}_6$, the logical Z$'$ operator is
\[
  \bar{Z}'_\alpha = Z(0,\; \alpha \cdot f^T).
\]
\end{definition}

\begin{theorem}[$\bar{X}_\alpha$ is Pure X-Type]
\label{thm:GrossCode.logicalXBar_pure_X}
\lean{QEC1.BivariateBicycle.GrossCode.logicalXBar_pure_X}
\leanok
\uses{def:GrossCode.logicalXBar, def:BivariateBicycle.pauliXBB}
For all $\alpha$, $\bar{X}_\alpha$ has no Z-support: $(\bar{X}_\alpha).\operatorname{zVec} = 0$.
\end{theorem}

\begin{proof}
\leanok
\uses{def:GrossCode.logicalXBar, def:BivariateBicycle.pauliXBB}
By simplification using the definitions of $\operatorname{logicalXBar}$ and $\operatorname{pauliXBB}$, the Z-component is identically zero.
\end{proof}

\begin{theorem}[$\bar{X}'_\beta$ is Pure X-Type]
\label{thm:GrossCode.logicalXBar'_pure_X}
\lean{QEC1.BivariateBicycle.GrossCode.logicalXBar'_pure_X}
\leanok
\uses{def:GrossCode.logicalXBar', def:BivariateBicycle.pauliXBB}
For all $\beta$, $\bar{X}'_\beta$ has no Z-support: $(\bar{X}'_\beta).\operatorname{zVec} = 0$.
\end{theorem}

\begin{proof}
\leanok
\uses{def:GrossCode.logicalXBar', def:BivariateBicycle.pauliXBB}
By simplification using the definitions of $\operatorname{logicalXBar}'$ and $\operatorname{pauliXBB}$, the Z-component is identically zero.
\end{proof}

\begin{theorem}[$\bar{Z}_\beta$ is Pure Z-Type]
\label{thm:GrossCode.logicalZBar_pure_Z}
\lean{QEC1.BivariateBicycle.GrossCode.logicalZBar_pure_Z}
\leanok
\uses{def:GrossCode.logicalZBar, def:BivariateBicycle.pauliZBB}
For all $\beta$, $\bar{Z}_\beta$ has no X-support: $(\bar{Z}_\beta).\operatorname{xVec} = 0$.
\end{theorem}

\begin{proof}
\leanok
\uses{def:GrossCode.logicalZBar, def:BivariateBicycle.pauliZBB}
By simplification using the definitions of $\operatorname{logicalZBar}$ and $\operatorname{pauliZBB}$, the X-component is identically zero.
\end{proof}

\begin{theorem}[$\bar{Z}'_\alpha$ is Pure Z-Type]
\label{thm:GrossCode.logicalZBar'_pure_Z}
\lean{QEC1.BivariateBicycle.GrossCode.logicalZBar'_pure_Z}
\leanok
\uses{def:GrossCode.logicalZBar', def:BivariateBicycle.pauliZBB}
For all $\alpha$, $\bar{Z}'_\alpha$ has no X-support: $(\bar{Z}'_\alpha).\operatorname{xVec} = 0$.
\end{theorem}

\begin{proof}
\leanok
\uses{def:GrossCode.logicalZBar', def:BivariateBicycle.pauliZBB}
By simplification using the definitions of $\operatorname{logicalZBar}'$ and $\operatorname{pauliZBB}$, the X-component is identically zero.
\end{proof}

\begin{theorem}[$\bar{X}_\alpha$ Acts Only on $L$ Qubits]
\label{thm:GrossCode.logicalXBar_acts_only_on_L}
\lean{QEC1.BivariateBicycle.GrossCode.logicalXBar_acts_only_on_L}
\leanok
\uses{def:GrossCode.logicalXBar, def:BivariateBicycle.pauliXBB}
For all $\alpha$ and all $\delta \in \mathbb{Z}_{12} \times \mathbb{Z}_6$, the X-support of $\bar{X}_\alpha$ on the right ($R$) qubit indexed by $\delta$ is zero:
\[
  (\bar{X}_\alpha).\operatorname{xVec}(\operatorname{inr}~\delta) = 0.
\]
\end{theorem}

\begin{proof}
\leanok
\uses{def:GrossCode.logicalXBar, def:BivariateBicycle.pauliXBB}
By simplification using the definitions of $\operatorname{logicalXBar}$ and $\operatorname{pauliXBB}$, the right component of the X-vector is zero since the second argument to $\operatorname{pauliXBB}$ is $0$.
\end{proof}

\begin{theorem}[$\bar{Z}'_\alpha$ Acts Only on $R$ Qubits]
\label{thm:GrossCode.logicalZBar'_acts_only_on_R}
\lean{QEC1.BivariateBicycle.GrossCode.logicalZBar'_acts_only_on_R}
\leanok
\uses{def:GrossCode.logicalZBar', def:BivariateBicycle.pauliZBB}
For all $\alpha$ and all $\gamma \in \mathbb{Z}_{12} \times \mathbb{Z}_6$, the Z-support of $\bar{Z}'_\alpha$ on the left ($L$) qubit indexed by $\gamma$ is zero:
\[
  (\bar{Z}'_\alpha).\operatorname{zVec}(\operatorname{inl}~\gamma) = 0.
\]
\end{theorem}

\begin{proof}
\leanok
\uses{def:GrossCode.logicalZBar', def:BivariateBicycle.pauliZBB}
By simplification using the definitions of $\operatorname{logicalZBar}'$ and $\operatorname{pauliZBB}$, the left component of the Z-vector is zero since the first argument to $\operatorname{pauliZBB}$ is $0$.
\end{proof}

\begin{theorem}[$f$ Has Weight 12]
\label{thm:GrossCode.grossF_weight_eq_12}
\lean{QEC1.BivariateBicycle.GrossCode.grossF_weight_eq_12}
\leanok
\uses{def:GrossCode.grossF, def:GrossCode.GrossMonomial}
The polynomial $f$ has exactly 12 nonzero coefficients:
\[
  |\{\gamma \in \mathbb{Z}_{12} \times \mathbb{Z}_6 : f(\gamma) \neq 0\}| = 12.
\]
\end{theorem}

\begin{proof}
\leanok
\uses{def:GrossCode.grossF}
This is verified by native computation.
\end{proof}

\begin{theorem}[$A$ Has Support of Size 3]
\label{thm:GrossCode.grossA_support_card}
\lean{QEC1.BivariateBicycle.GrossCode.grossA_support_card}
\leanok
\uses{def:GrossCode.grossA, def:GrossCode.GrossMonomial}
The polynomial $A$ has exactly 3 nonzero coefficients:
\[
  |\{\gamma \in \mathbb{Z}_{12} \times \mathbb{Z}_6 : A(\gamma) \neq 0\}| = 3.
\]
\end{theorem}

\begin{proof}
\leanok
\uses{def:GrossCode.grossA}
This is verified by native computation.
\end{proof}

\begin{theorem}[$B$ Has Support of Size 3]
\label{thm:GrossCode.grossB_support_card}
\lean{QEC1.BivariateBicycle.GrossCode.grossB_support_card}
\leanok
\uses{def:GrossCode.grossB, def:GrossCode.GrossMonomial}
The polynomial $B$ has exactly 3 nonzero coefficients:
\[
  |\{\gamma \in \mathbb{Z}_{12} \times \mathbb{Z}_6 : B(\gamma) \neq 0\}| = 3.
\]
\end{theorem}

\begin{proof}
\leanok
\uses{def:GrossCode.grossB}
This is verified by native computation.
\end{proof}

\begin{theorem}[$\bar{X}_\alpha$ Has Weight 12]
\label{thm:GrossCode.logicalXBar_weight}
\lean{QEC1.BivariateBicycle.GrossCode.logicalXBar_weight}
\leanok
\uses{def:GrossCode.logicalXBar, def:PauliOp.weight, def:PauliOp.support, def:GrossCode.grossF}
For all $\alpha \in \mathbb{Z}_{12} \times \mathbb{Z}_6$, the logical operator $\bar{X}_\alpha$ has weight 12:
\[
  \operatorname{weight}(\bar{X}_\alpha) = 12.
\]
\end{theorem}

\begin{proof}
\leanok
\uses{def:GrossCode.logicalXBar, def:GrossCode.grossF}
By simplification using the definitions of weight and support, the support cardinality of $\operatorname{pauliXBB}(\alpha \cdot f, 0)$ is shift-invariant. Rewriting using this shift-invariance reduces the computation to the case $\alpha = 0$, which is verified by native computation to equal 12.
\end{proof}

\begin{theorem}[$\bar{X}_\alpha$ is Self-Inverse]
\label{thm:GrossCode.logicalXBar_mul_self}
\lean{QEC1.BivariateBicycle.GrossCode.logicalXBar_mul_self}
\leanok
\uses{def:GrossCode.logicalXBar, def:BivariateBicycle.pauliXBB, thm:BivariateBicycle.pauliXBB_mul}
For all $\alpha$, $\bar{X}_\alpha \cdot \bar{X}_\alpha = \mathbf{1}$.
\end{theorem}

\begin{proof}
\leanok
\uses{def:GrossCode.logicalXBar, def:BivariateBicycle.pauliXBB, thm:BivariateBicycle.pauliXBB_mul}
Expanding $\operatorname{logicalXBar}$ and applying the $\operatorname{pauliXBB}$ multiplication rule, we verify equality by extensionality. For the X-component: on left qubits $(\operatorname{inl}~\gamma)$, each coefficient is $f(\gamma - \alpha) + f(\gamma - \alpha) = 0$ by characteristic 2; on right qubits $(\operatorname{inr}~\delta)$, both sides are $0$. For the Z-component, both sides are $0$ by the pure X-type property.
\end{proof}

\begin{theorem}[$\bar{X}'_\beta$ is Self-Inverse]
\label{thm:GrossCode.logicalXBar'_mul_self}
\lean{QEC1.BivariateBicycle.GrossCode.logicalXBar'_mul_self}
\leanok
\uses{def:GrossCode.logicalXBar', def:BivariateBicycle.pauliXBB, thm:BivariateBicycle.pauliXBB_mul}
For all $\beta$, $\bar{X}'_\beta \cdot \bar{X}'_\beta = \mathbf{1}$.
\end{theorem}

\begin{proof}
\leanok
\uses{def:GrossCode.logicalXBar', def:BivariateBicycle.pauliXBB, thm:BivariateBicycle.pauliXBB_mul}
Expanding and applying the $\operatorname{pauliXBB}$ multiplication rule, we verify by extensionality. For the X-component on left qubits, each coefficient is $g(\gamma - \beta) + g(\gamma - \beta) = 0$; on right qubits, $h(\delta - \beta) + h(\delta - \beta) = 0$, both by characteristic 2. The Z-component is zero on both sides.
\end{proof}

\begin{theorem}[$\bar{Z}_\beta$ is Self-Inverse]
\label{thm:GrossCode.logicalZBar_mul_self}
\lean{QEC1.BivariateBicycle.GrossCode.logicalZBar_mul_self}
\leanok
\uses{def:GrossCode.logicalZBar, def:BivariateBicycle.pauliZBB, thm:BivariateBicycle.pauliZBB_mul}
For all $\beta$, $\bar{Z}_\beta \cdot \bar{Z}_\beta = \mathbf{1}$.
\end{theorem}

\begin{proof}
\leanok
\uses{def:GrossCode.logicalZBar, def:BivariateBicycle.pauliZBB, thm:BivariateBicycle.pauliZBB_mul}
Expanding and applying the $\operatorname{pauliZBB}$ multiplication rule, we verify by extensionality. The X-component is zero on both sides. For the Z-component on left qubits, $h^T(\gamma - \beta) + h^T(\gamma - \beta) = 0$; on right qubits, $g^T(\delta - \beta) + g^T(\delta - \beta) = 0$, both by characteristic 2.
\end{proof}

\begin{theorem}[$\bar{Z}'_\alpha$ is Self-Inverse]
\label{thm:GrossCode.logicalZBar'_mul_self}
\lean{QEC1.BivariateBicycle.GrossCode.logicalZBar'_mul_self}
\leanok
\uses{def:GrossCode.logicalZBar', def:BivariateBicycle.pauliZBB, thm:BivariateBicycle.pauliZBB_mul}
For all $\alpha$, $\bar{Z}'_\alpha \cdot \bar{Z}'_\alpha = \mathbf{1}$.
\end{theorem}

\begin{proof}
\leanok
\uses{def:GrossCode.logicalZBar', def:BivariateBicycle.pauliZBB, thm:BivariateBicycle.pauliZBB_mul}
Expanding and applying the $\operatorname{pauliZBB}$ multiplication rule, we verify by extensionality. The X-component is zero on both sides. For the Z-component: on left qubits $(\operatorname{inl}~\gamma)$, both sides are $0$ since the first argument is $0$; on right qubits $(\operatorname{inr}~\delta)$, $f^T(\delta - \alpha) + f^T(\delta - \alpha) = 0$ by characteristic 2.
\end{proof}

\begin{theorem}[$\bar{X}_\alpha$ Commutes with X-Checks]
\label{thm:GrossCode.logicalXBar_commute_xCheck}
\lean{QEC1.BivariateBicycle.GrossCode.logicalXBar_commute_xCheck}
\leanok
\uses{def:GrossCode.logicalXBar, def:BivariateBicycle.bbCheckX, def:GrossCode.grossA, def:GrossCode.grossB, def:PauliOp.PauliCommute, def:PauliOp.symplecticInner, def:BivariateBicycle.pauliXBB}
For all $\alpha, \beta$, $\bar{X}_\alpha$ commutes with the X-check at $\beta$.
\end{theorem}

\begin{proof}
\leanok
\uses{def:GrossCode.logicalXBar, def:BivariateBicycle.bbCheckX, def:BivariateBicycle.pauliXBB}
By simplification: both operators are pure X-type, so the symplectic inner product vanishes identically.
\end{proof}

\begin{theorem}[$\bar{X}_\alpha$ Commutes with Z-Checks]
\label{thm:GrossCode.logicalXBar_commute_zCheck}
\lean{QEC1.BivariateBicycle.GrossCode.logicalXBar_commute_zCheck}
\leanok
\uses{def:GrossCode.logicalXBar, def:BivariateBicycle.bbCheckZ, def:GrossCode.grossA, def:GrossCode.grossB, def:GrossCode.grossF, def:PauliOp.PauliCommute, def:PauliOp.symplecticInner, def:BivariateBicycle.pauliXBB, def:BivariateBicycle.pauliZBB, def:BivariateBicycle.bbShift, def:BivariateBicycle.bbTranspose}
For all $\alpha, \beta$, $\bar{X}_\alpha$ commutes with the Z-check at $\beta$.
\end{theorem}

\begin{proof}
\leanok
\uses{def:GrossCode.logicalXBar, def:BivariateBicycle.bbCheckZ, def:BivariateBicycle.pauliXBB, def:BivariateBicycle.pauliZBB, def:GrossCode.grossF, def:GrossCode.grossB, def:BivariateBicycle.bbShift, def:BivariateBicycle.bbTranspose}
Rewriting the Pauli commutation condition into the symplectic inner product, we expand the definitions. After simplification, the sum splits over $L$ and $R$ qubits, and the $R$ contribution vanishes. The remaining $L$ contribution reduces by a change of variables $\delta \mapsto \delta - \alpha$ to the convolution $\sum_\delta f(\delta) \cdot B^T(\delta - (\beta - \alpha))$, which equals zero by the kernel condition $fB^T = 0$ verified by native computation.
\end{proof}

\begin{theorem}[$\bar{X}_\alpha$ Commutes with All Checks]
\label{thm:GrossCode.logicalXBar_commute_allChecks}
\lean{QEC1.BivariateBicycle.GrossCode.logicalXBar_commute_allChecks}
\leanok
\uses{def:GrossCode.logicalXBar, def:BivariateBicycle.bbCheck, def:GrossCode.grossA, def:GrossCode.grossB, thm:GrossCode.logicalXBar_commute_xCheck, thm:GrossCode.logicalXBar_commute_zCheck}
For all $\alpha$ and all check indices $i$, $\bar{X}_\alpha$ commutes with the check at $i$.
\end{theorem}

\begin{proof}
\leanok
\uses{thm:GrossCode.logicalXBar_commute_xCheck, thm:GrossCode.logicalXBar_commute_zCheck}
We case-split on whether $i$ is an X-check or Z-check and apply the respective commutation theorems.
\end{proof}

\begin{theorem}[$\bar{X}'_\beta$ Commutes with X-Checks]
\label{thm:GrossCode.logicalXBar'_commute_xCheck}
\lean{QEC1.BivariateBicycle.GrossCode.logicalXBar'_commute_xCheck}
\leanok
\uses{def:GrossCode.logicalXBar', def:BivariateBicycle.bbCheckX, def:GrossCode.grossA, def:GrossCode.grossB, def:PauliOp.PauliCommute, def:PauliOp.symplecticInner, def:BivariateBicycle.pauliXBB}
For all $\alpha, \beta$, $\bar{X}'_\alpha$ commutes with the X-check at $\beta$.
\end{theorem}

\begin{proof}
\leanok
\uses{def:GrossCode.logicalXBar', def:BivariateBicycle.bbCheckX, def:BivariateBicycle.pauliXBB}
Both operators are pure X-type, so the symplectic inner product vanishes identically.
\end{proof}

\begin{theorem}[$\bar{X}'_\beta$ Commutes with Z-Checks]
\label{thm:GrossCode.logicalXBar'_commute_zCheck}
\lean{QEC1.BivariateBicycle.GrossCode.logicalXBar'_commute_zCheck}
\leanok
\uses{def:GrossCode.logicalXBar', def:BivariateBicycle.bbCheckZ, def:GrossCode.grossA, def:GrossCode.grossB, def:GrossCode.grossG, def:GrossCode.grossH, def:PauliOp.PauliCommute, def:PauliOp.symplecticInner, def:BivariateBicycle.pauliXBB, def:BivariateBicycle.pauliZBB, def:BivariateBicycle.bbShift, def:BivariateBicycle.bbTranspose}
For all $\alpha, \beta$, $\bar{X}'_\alpha$ commutes with the Z-check at $\beta$.
\end{theorem}

\begin{proof}
\leanok
\uses{def:GrossCode.logicalXBar', def:BivariateBicycle.bbCheckZ, def:BivariateBicycle.pauliXBB, def:BivariateBicycle.pauliZBB, def:GrossCode.grossG, def:GrossCode.grossH, def:GrossCode.grossA, def:GrossCode.grossB, def:BivariateBicycle.bbShift, def:BivariateBicycle.bbTranspose}
Rewriting the Pauli commutation into the symplectic inner product, we expand definitions and split over qubit types. After the change of variables $\delta \mapsto \delta - \alpha$, the sum reduces to $\sum_\delta g(\delta) B^T(\delta - (\beta - \alpha)) + \sum_\delta h(\delta) A^T(\delta - (\beta - \alpha))$, which equals zero by the kernel condition $gB^T + hA^T = 0$ verified by native computation.
\end{proof}

\begin{theorem}[$\bar{X}'_\beta$ Commutes with All Checks]
\label{thm:GrossCode.logicalXBar'_commute_allChecks}
\lean{QEC1.BivariateBicycle.GrossCode.logicalXBar'_commute_allChecks}
\leanok
\uses{def:GrossCode.logicalXBar', def:BivariateBicycle.bbCheck, def:GrossCode.grossA, def:GrossCode.grossB, thm:GrossCode.logicalXBar'_commute_xCheck, thm:GrossCode.logicalXBar'_commute_zCheck}
For all $\beta$ and all check indices $i$, $\bar{X}'_\beta$ commutes with the check at $i$.
\end{theorem}

\begin{proof}
\leanok
\uses{thm:GrossCode.logicalXBar'_commute_xCheck, thm:GrossCode.logicalXBar'_commute_zCheck}
We case-split on whether $i$ is an X-check or Z-check and apply the respective commutation theorems.
\end{proof}

\begin{theorem}[$\bar{Z}_\beta$ Commutes with Z-Checks]
\label{thm:GrossCode.logicalZBar_commute_zCheck}
\lean{QEC1.BivariateBicycle.GrossCode.logicalZBar_commute_zCheck}
\leanok
\uses{def:GrossCode.logicalZBar, def:BivariateBicycle.bbCheckZ, def:GrossCode.grossA, def:GrossCode.grossB, def:PauliOp.PauliCommute, def:PauliOp.symplecticInner, def:BivariateBicycle.pauliZBB}
For all $\alpha, \beta$, $\bar{Z}_\alpha$ commutes with the Z-check at $\beta$.
\end{theorem}

\begin{proof}
\leanok
\uses{def:GrossCode.logicalZBar, def:BivariateBicycle.bbCheckZ, def:BivariateBicycle.pauliZBB}
Both operators are pure Z-type, so the symplectic inner product vanishes identically.
\end{proof}

\begin{theorem}[$\bar{Z}_\beta$ Commutes with X-Checks]
\label{thm:GrossCode.logicalZBar_commute_xCheck}
\lean{QEC1.BivariateBicycle.GrossCode.logicalZBar_commute_xCheck}
\leanok
\uses{def:GrossCode.logicalZBar, def:BivariateBicycle.bbCheckX, def:GrossCode.grossA, def:GrossCode.grossB, def:GrossCode.grossH, def:GrossCode.grossG, def:PauliOp.PauliCommute, def:PauliOp.symplecticInner, def:BivariateBicycle.pauliXBB, def:BivariateBicycle.pauliZBB, def:BivariateBicycle.bbShift, def:BivariateBicycle.bbTranspose}
For all $\alpha, \beta$, $\bar{Z}_\alpha$ commutes with the X-check at $\beta$.
\end{theorem}

\begin{proof}
\leanok
\uses{def:GrossCode.logicalZBar, def:BivariateBicycle.bbCheckX, def:BivariateBicycle.pauliXBB, def:BivariateBicycle.pauliZBB, def:GrossCode.grossH, def:GrossCode.grossG, def:GrossCode.grossA, def:GrossCode.grossB, def:BivariateBicycle.bbShift, def:BivariateBicycle.bbTranspose}
Using commutativity of PauliCommute, we rewrite the symplectic inner product from the X-check's perspective. Expanding definitions and splitting over qubit types, after the change of variables $\delta \mapsto \delta - \beta$, the sum reduces to $\sum_\delta A(\delta) h^T(\delta - (\alpha - \beta)) + \sum_\delta B(\delta) g^T(\delta - (\alpha - \beta))$, which equals zero by the kernel condition $Ah^T + Bg^T = 0$ verified by native computation.
\end{proof}

\begin{theorem}[$\bar{Z}_\beta$ Commutes with All Checks]
\label{thm:GrossCode.logicalZBar_commute_allChecks}
\lean{QEC1.BivariateBicycle.GrossCode.logicalZBar_commute_allChecks}
\leanok
\uses{def:GrossCode.logicalZBar, def:BivariateBicycle.bbCheck, def:GrossCode.grossA, def:GrossCode.grossB, thm:GrossCode.logicalZBar_commute_xCheck, thm:GrossCode.logicalZBar_commute_zCheck}
For all $\beta$ and all check indices $i$, $\bar{Z}_\beta$ commutes with the check at $i$.
\end{theorem}

\begin{proof}
\leanok
\uses{thm:GrossCode.logicalZBar_commute_xCheck, thm:GrossCode.logicalZBar_commute_zCheck}
We case-split on whether $i$ is an X-check or Z-check and apply the respective commutation theorems.
\end{proof}

\begin{theorem}[$\bar{Z}'_\alpha$ Commutes with Z-Checks]
\label{thm:GrossCode.logicalZBar'_commute_zCheck}
\lean{QEC1.BivariateBicycle.GrossCode.logicalZBar'_commute_zCheck}
\leanok
\uses{def:GrossCode.logicalZBar', def:BivariateBicycle.bbCheckZ, def:GrossCode.grossA, def:GrossCode.grossB, def:PauliOp.PauliCommute, def:PauliOp.symplecticInner, def:BivariateBicycle.pauliZBB}
For all $\alpha, \beta$, $\bar{Z}'_\alpha$ commutes with the Z-check at $\beta$.
\end{theorem}

\begin{proof}
\leanok
\uses{def:GrossCode.logicalZBar', def:BivariateBicycle.bbCheckZ, def:BivariateBicycle.pauliZBB}
Both operators are pure Z-type, so the symplectic inner product vanishes identically.
\end{proof}

\begin{theorem}[$\bar{Z}'_\alpha$ Commutes with X-Checks]
\label{thm:GrossCode.logicalZBar'_commute_xCheck}
\lean{QEC1.BivariateBicycle.GrossCode.logicalZBar'_commute_xCheck}
\leanok
\uses{def:GrossCode.logicalZBar', def:BivariateBicycle.bbCheckX, def:GrossCode.grossA, def:GrossCode.grossB, def:GrossCode.grossF, def:PauliOp.PauliCommute, def:PauliOp.symplecticInner, def:BivariateBicycle.pauliXBB, def:BivariateBicycle.pauliZBB, def:BivariateBicycle.bbShift, def:BivariateBicycle.bbTranspose}
For all $\alpha, \beta$, $\bar{Z}'_\alpha$ commutes with the X-check at $\beta$.
\end{theorem}

\begin{proof}
\leanok
\uses{def:GrossCode.logicalZBar', def:BivariateBicycle.bbCheckX, def:BivariateBicycle.pauliXBB, def:BivariateBicycle.pauliZBB, def:GrossCode.grossF, def:GrossCode.grossB, def:BivariateBicycle.bbShift, def:BivariateBicycle.bbTranspose}
Using commutativity of PauliCommute, we rewrite the symplectic inner product. Expanding definitions and splitting over qubit types, the $L$ contribution vanishes since the first argument to $\operatorname{pauliZBB}$ is $0$. After the change of variables $\delta \mapsto \delta - \beta$, the remaining $R$ contribution is $\sum_\delta B(\delta) f^T(\delta - (\alpha - \beta))$, which equals zero by the kernel condition $Bf^T = 0$ verified by native computation.
\end{proof}

\begin{theorem}[$\bar{Z}'_\alpha$ Commutes with All Checks]
\label{thm:GrossCode.logicalZBar'_commute_allChecks}
\lean{QEC1.BivariateBicycle.GrossCode.logicalZBar'_commute_allChecks}
\leanok
\uses{def:GrossCode.logicalZBar', def:BivariateBicycle.bbCheck, def:GrossCode.grossA, def:GrossCode.grossB, thm:GrossCode.logicalZBar'_commute_xCheck, thm:GrossCode.logicalZBar'_commute_zCheck}
For all $\alpha$ and all check indices $i$, $\bar{Z}'_\alpha$ commutes with the check at $i$.
\end{theorem}

\begin{proof}
\leanok
\uses{thm:GrossCode.logicalZBar'_commute_xCheck, thm:GrossCode.logicalZBar'_commute_zCheck}
We case-split on whether $i$ is an X-check or Z-check and apply the respective commutation theorems.
\end{proof}

\begin{theorem}[$\bar{X}_\alpha$ is in the Centralizer]
\label{thm:GrossCode.logicalXBar_inCentralizer}
\lean{QEC1.BivariateBicycle.GrossCode.logicalXBar_inCentralizer}
\leanok
\uses{def:GrossCode.grossCode, def:GrossCode.logicalXBar, def:StabilizerCode.inCentralizer, thm:GrossCode.logicalXBar_commute_allChecks}
For all $\alpha$, $\bar{X}_\alpha$ is in the centralizer of the Gross code.
\end{theorem}

\begin{proof}
\leanok
\uses{thm:GrossCode.logicalXBar_commute_allChecks}
Let $i$ be an arbitrary check index. The result follows directly from the fact that $\bar{X}_\alpha$ commutes with all checks.
\end{proof}

\begin{theorem}[$\bar{X}'_\beta$ is in the Centralizer]
\label{thm:GrossCode.logicalXBar'_inCentralizer}
\lean{QEC1.BivariateBicycle.GrossCode.logicalXBar'_inCentralizer}
\leanok
\uses{def:GrossCode.grossCode, def:GrossCode.logicalXBar', def:StabilizerCode.inCentralizer, thm:GrossCode.logicalXBar'_commute_allChecks}
For all $\beta$, $\bar{X}'_\beta$ is in the centralizer of the Gross code.
\end{theorem}

\begin{proof}
\leanok
\uses{thm:GrossCode.logicalXBar'_commute_allChecks}
Let $i$ be an arbitrary check index. The result follows directly from the fact that $\bar{X}'_\beta$ commutes with all checks.
\end{proof}

\begin{theorem}[$\bar{Z}_\beta$ is in the Centralizer]
\label{thm:GrossCode.logicalZBar_inCentralizer}
\lean{QEC1.BivariateBicycle.GrossCode.logicalZBar_inCentralizer}
\leanok
\uses{def:GrossCode.grossCode, def:GrossCode.logicalZBar, def:StabilizerCode.inCentralizer, thm:GrossCode.logicalZBar_commute_allChecks}
For all $\beta$, $\bar{Z}_\beta$ is in the centralizer of the Gross code.
\end{theorem}

\begin{proof}
\leanok
\uses{thm:GrossCode.logicalZBar_commute_allChecks}
Let $i$ be an arbitrary check index. The result follows directly from the fact that $\bar{Z}_\beta$ commutes with all checks.
\end{proof}

\begin{theorem}[$\bar{Z}'_\alpha$ is in the Centralizer]
\label{thm:GrossCode.logicalZBar'_inCentralizer}
\lean{QEC1.BivariateBicycle.GrossCode.logicalZBar'_inCentralizer}
\leanok
\uses{def:GrossCode.grossCode, def:GrossCode.logicalZBar', def:StabilizerCode.inCentralizer, thm:GrossCode.logicalZBar'_commute_allChecks}
For all $\alpha$, $\bar{Z}'_\alpha$ is in the centralizer of the Gross code.
\end{theorem}

\begin{proof}
\leanok
\uses{thm:GrossCode.logicalZBar'_commute_allChecks}
Let $i$ be an arbitrary check index. The result follows directly from the fact that $\bar{Z}'_\alpha$ commutes with all checks.
\end{proof}

\begin{theorem}[X-Check Weight is 6]
\label{thm:GrossCode.xCheck_weight}
\lean{QEC1.BivariateBicycle.GrossCode.xCheck_weight}
\leanok
\uses{def:BivariateBicycle.bbCheckX, def:GrossCode.grossA, def:GrossCode.grossB, def:PauliOp.weight, def:PauliOp.support}
Each X-check of the Gross code has weight 6: for all $\alpha$,
\[
  \operatorname{weight}(\operatorname{bbCheckX}(A, B, \alpha)) = 6.
\]
\end{theorem}

\begin{proof}
\leanok
\uses{def:BivariateBicycle.bbCheckX, def:GrossCode.grossA, def:GrossCode.grossB}
By simplification using the definitions of weight and support, the support cardinality of the X-check is shift-invariant. Rewriting reduces the computation to the case $\alpha = 0$, which is verified by native computation to equal 6.
\end{proof}

\begin{theorem}[Insufficient Tanner Graph Expansion]
\label{thm:GrossCode.logicalXBar_tanner_expansion_insufficient}
\lean{QEC1.BivariateBicycle.GrossCode.logicalXBar_tanner_expansion_insufficient}
\leanok
\uses{def:GrossCode.logicalXBar, def:GrossCode.grossA, thm:GrossCode.logicalXBar_weight, thm:GrossCode.logicalXBar_pure_X, thm:GrossCode.logicalXBar_acts_only_on_L, thm:GrossCode.grossA_support_card}
For all $\alpha \in \mathbb{Z}_{12} \times \mathbb{Z}_6$, the following four properties hold simultaneously:
\begin{enumerate}
\item $\operatorname{weight}(\bar{X}_\alpha) = 12$,
\item $(\bar{X}_\alpha).\operatorname{zVec} = 0$ (pure X-type),
\item $\forall \delta,\; (\bar{X}_\alpha).\operatorname{xVec}(\operatorname{inr}~\delta) = 0$ (acts only on $L$ qubits),
\item $|\operatorname{supp}(A)| = 3$.
\end{enumerate}
Since $\bar{X}_\alpha$ acts only on $L$ qubits and each $L$ qubit participates in at most $|\operatorname{supp}(A)| = 3$ X-type checks, the Tanner subgraph on the support of $\bar{X}_\alpha$ has limited expansion.
\end{theorem}

\begin{proof}
\leanok
\uses{thm:GrossCode.logicalXBar_weight, thm:GrossCode.logicalXBar_pure_X, thm:GrossCode.logicalXBar_acts_only_on_L, thm:GrossCode.grossA_support_card}
This is the conjunction of the four previously established results: the weight theorem, the pure X-type theorem, the $L$-qubit-only theorem, and the support cardinality of $A$.
\end{proof}

\begin{theorem}[Gross Code Parameters]
\label{thm:GrossCode.grossCode_parameters}
\lean{QEC1.BivariateBicycle.GrossCode.grossCode_parameters}
\leanok
\uses{def:GrossCode.grossCode, thm:GrossCode.grossCode_numQubits, thm:GrossCode.grossCode_numChecks}
The Gross code has 144 qubits and 144 checks:
\[
  \operatorname{grossCode.numQubits} = 144 \quad \text{and} \quad \operatorname{grossCode.numChecks} = 144.
\]
\end{theorem}

\begin{proof}
\leanok
\uses{thm:GrossCode.grossCode_numQubits, thm:GrossCode.grossCode_numChecks}
This is the conjunction of the two previously established parameter theorems.
\end{proof}

%--- Rem_21: GrossCodeGaugingMeasurement ---
\chapter{Rem 21: Gross Code Gauging Measurement Construction}

This chapter presents the explicit construction of the gauging graph $G$ for measuring $\bar{X}_\alpha = X(\alpha f, 0)$ in the Gross code, with degree bounds on the Tanner graph of the deformed code. The construction specifies the 12 vertices (support of $f$), 22 edges (18 matching + 4 expansion), and 7 independent cycles, yielding a total overhead of 41 additional checks and qubits.

\begin{definition}[Gauging Vertices]
\label{def:GrossCode.gaugingVertices}
\lean{QEC1.GrossCode.gaugingVertices}
\leanok
\uses{def:GrossCode.GrossMonomial, def:GrossCode.grossF}
The \emph{gauging vertices} are the support of $f$: the set of monomials $\gamma \in \mathbb{Z}_{12} \times \mathbb{Z}_6$ such that $f(\gamma) \neq 0$:
\[
V(G) = \{\gamma \in \operatorname{GrossMonomial} \mid f(\gamma) \neq 0\}.
\]
\end{definition}

\begin{theorem}[Gauging Vertices Cardinality]
\label{thm:GrossCode.gaugingVertices_card}
\lean{QEC1.GrossCode.gaugingVertices_card}
\leanok
\uses{def:GrossCode.gaugingVertices}
The gauging graph has exactly 12 vertices: $|V(G)| = 12$.
\end{theorem}

\begin{proof}
\leanok
\uses{def:GrossCode.gaugingVertices}
This is verified by computation (native\_decide).
\end{proof}

\begin{definition}[Matching Difference Predicate]
\label{def:GrossCode.isMatchingDiff}
\lean{QEC1.GrossCode.isMatchingDiff}
\leanok
\uses{def:GrossCode.GrossMonomial}
A monomial difference $d$ is a \emph{matching difference} if $d = -B_i + B_j$ for some distinct terms $B_i, B_j$ of $B = y^3 + x^2 + x$, where the terms of $B$ are $(0,3)$, $(2,0)$, and $(1,0)$ in $\mathbb{Z}_{12} \times \mathbb{Z}_6$, and $d \neq 0$.
\end{definition}

\begin{definition}[Matching Edge Predicate]
\label{def:GrossCode.isMatchingEdge}
\lean{QEC1.GrossCode.isMatchingEdge}
\leanok
\uses{def:GrossCode.GrossMonomial, def:GrossCode.grossF, def:GrossCode.isMatchingDiff}
Two monomials $\gamma$ and $\delta$ form a \emph{matching edge} if both are in $\operatorname{supp}(f)$ (i.e., $f(\gamma) \neq 0$ and $f(\delta) \neq 0$) and their difference $\gamma - \delta$ is a matching difference. This corresponds to $\gamma$ and $\delta$ sharing a $Z$ check.
\end{definition}

\begin{definition}[Expansion Edges]
\label{def:GrossCode.expansionEdges}
\lean{QEC1.GrossCode.expansionEdges}
\leanok
\uses{def:GrossCode.GrossMonomial}
The 4 expansion edges ensuring the deformed code has distance 12, given as pairs of monomials:
\[
\{((2,0),(5,3)),\; ((2,0),(6,0)),\; ((5,3),(11,3)),\; ((7,3),(11,3))\}.
\]
\end{definition}

\begin{definition}[Expansion Edge Predicate]
\label{def:GrossCode.isExpansionEdge}
\lean{QEC1.GrossCode.isExpansionEdge}
\leanok
\uses{def:GrossCode.GrossMonomial, def:GrossCode.expansionEdges}
A pair $(\gamma, \delta)$ is an \emph{expansion edge} if it appears (in either order) in the list of 4 expansion edges.
\end{definition}

\begin{definition}[Gauging Graph Adjacency]
\label{def:GrossCode.gaugingAdj}
\lean{QEC1.GrossCode.gaugingAdj}
\leanok
\uses{def:GrossCode.GrossMonomial, def:GrossCode.isMatchingEdge, def:GrossCode.isExpansionEdge}
The adjacency relation on the gauging graph: $\gamma$ and $\delta$ are adjacent if $\gamma \neq \delta$ and either $(\gamma, \delta)$ is a matching edge or an expansion edge.
\end{definition}

\begin{definition}[Gauging Graph]
\label{def:GrossCode.gaugingGraph}
\lean{QEC1.GrossCode.gaugingGraph}
\leanok
\uses{def:GrossCode.GrossMonomial, def:GrossCode.gaugingAdj}
The \emph{gauging graph} $G$ for measuring $\bar{X}_\alpha$ in the Gross code is the simple graph on $\operatorname{GrossMonomial} = \mathbb{Z}_{12} \times \mathbb{Z}_6$ with adjacency given by $\operatorname{gaugingAdj}$. The ``active'' vertices are $\operatorname{supp}(f)$.
\end{definition}

\begin{theorem}[Edge Count]
\label{thm:GrossCode.gaugingEdges_card}
\lean{QEC1.GrossCode.gaugingEdges_card}
\leanok
\uses{def:GrossCode.gaugingGraph}
The gauging graph has exactly 22 edges: $|E(G)| = 22$.
\end{theorem}

\begin{proof}
\leanok
\uses{def:GrossCode.gaugingGraph}
This is verified by computation (native\_decide).
\end{proof}

\begin{definition}[Gauging Cycles List]
\label{def:GrossCode.gaugingCyclesList}
\lean{QEC1.GrossCode.gaugingCyclesList}
\leanok
\uses{def:GrossCode.GrossMonomial}
The 7 independent cycles for the flux checks, given as lists of vertices. Each list $[v_1, v_2, \ldots, v_k]$ represents the cycle $v_1 \to v_2 \to \cdots \to v_k \to v_1$:
\begin{enumerate}
\item $[(9,0),\, (7,3),\, (8,0)]$
\item $[(9,0),\, (11,3),\, (7,3)]$
\item $[(6,0),\, (7,0),\, (5,3)]$
\item $[(6,0),\, (2,0),\, (5,3)]$
\item $[(3,0),\, (2,0),\, (5,3)]$
\item $[(6,0),\, (7,0),\, (8,0),\, (7,3)]$
\item $[(1,0),\, (2,0),\, (5,3),\, (11,3)]$
\end{enumerate}
\end{definition}

\begin{theorem}[Cycle Count]
\label{thm:GrossCode.gaugingCycles_count}
\lean{QEC1.GrossCode.gaugingCycles_count}
\leanok
\uses{def:GrossCode.gaugingCyclesList}
There are exactly 7 independent cycles: $|\text{cycles}| = 7$.
\end{theorem}

\begin{proof}
\leanok
\uses{def:GrossCode.gaugingCyclesList}
This holds by reflexivity (the list length is definitionally 7).
\end{proof}

\begin{theorem}[Cycle Vertices in Support]
\label{thm:GrossCode.gaugingCycles_vertices_in_support}
\lean{QEC1.GrossCode.gaugingCycles_vertices_in_support}
\leanok
\uses{def:GrossCode.gaugingCyclesList, def:GrossCode.gaugingVertices}
Every vertex appearing in any of the 7 cycles belongs to the gauging vertices $V(G) = \operatorname{supp}(f)$.
\end{theorem}

\begin{proof}
\leanok
\uses{def:GrossCode.gaugingCyclesList, def:GrossCode.gaugingVertices}
This is verified by computation (native\_decide).
\end{proof}

\begin{theorem}[Cycles Have at Least 3 Vertices]
\label{thm:GrossCode.gaugingCycles_length_ge_3}
\lean{QEC1.GrossCode.gaugingCycles_length_ge_3}
\leanok
\uses{def:GrossCode.gaugingCyclesList}
Each cycle has at least 3 vertices.
\end{theorem}

\begin{proof}
\leanok
\uses{def:GrossCode.gaugingCyclesList}
This is verified by computation (native\_decide).
\end{proof}

\begin{theorem}[Cycle Edges Valid]
\label{thm:GrossCode.gaugingCycles_edges_valid}
\lean{QEC1.GrossCode.gaugingCycles_edges_valid}
\leanok
\uses{def:GrossCode.gaugingCyclesList, def:GrossCode.gaugingGraph}
For each cycle, consecutive vertices (including the wrap-around from the last vertex back to the first) are adjacent in the gauging graph.
\end{theorem}

\begin{proof}
\leanok
\uses{def:GrossCode.gaugingCyclesList, def:GrossCode.gaugingGraph}
This is verified by computation (native\_decide).
\end{proof}

\begin{theorem}[Cycles Are Short]
\label{thm:GrossCode.gaugingCycles_short}
\lean{QEC1.GrossCode.gaugingCycles_short}
\leanok
\uses{def:GrossCode.gaugingCyclesList}
All cycles have length 3 or 4.
\end{theorem}

\begin{proof}
\leanok
\uses{def:GrossCode.gaugingCyclesList}
This is verified by computation (native\_decide).
\end{proof}

\begin{theorem}[Maximum Cycle Length]
\label{thm:GrossCode.gaugingCycles_max_length}
\lean{QEC1.GrossCode.gaugingCycles_max_length}
\leanok
\uses{def:GrossCode.gaugingCyclesList, thm:GrossCode.gaugingCycles_short}
Every cycle has length at most 4.
\end{theorem}

\begin{proof}
\leanok
\uses{thm:GrossCode.gaugingCycles_short}
Let $\mathrm{cyc}$ be any cycle in the list and let $h_{\mathrm{mem}}$ be the proof that it belongs to the list. By the theorem that all cycles have length 3 or 4 (gaugingCycles\_short), we obtain $|\mathrm{cyc}| = 3$ or $|\mathrm{cyc}| = 4$. In either case, $|\mathrm{cyc}| \leq 4$ follows by integer arithmetic.
\end{proof}

\begin{theorem}[Cycle Space Dimension]
\label{thm:GrossCode.cycle_space_dim}
\lean{QEC1.GrossCode.cycle_space_dim}
\leanok
\uses{thm:GrossCode.gaugingEdges_card, thm:GrossCode.gaugingVertices_card}
By Euler's formula, the cycle space dimension satisfies $|E| + 1 = |V| + 11$, i.e., $22 + 1 = 12 + 11$.
\end{theorem}

\begin{proof}
\leanok
\uses{thm:GrossCode.gaugingEdges_card, thm:GrossCode.gaugingVertices_card}
Rewriting using $|E(G)| = 22$ and $|V(G)| = 12$, the equation $22 + 1 = 12 + 11$ holds.
\end{proof}

\begin{theorem}[Additional Gauss Checks]
\label{thm:GrossCode.additional_gauss_checks}
\lean{QEC1.GrossCode.additional_gauss_checks}
\leanok
\uses{def:GrossCode.gaugingVertices, thm:GrossCode.gaugingVertices_card}
The number of additional Gauss's law checks $A_v$ (one per vertex of $G$) is 12.
\end{theorem}

\begin{proof}
\leanok
\uses{thm:GrossCode.gaugingVertices_card}
This follows directly from the vertex count $|V(G)| = 12$.
\end{proof}

\begin{theorem}[Additional Flux Checks]
\label{thm:GrossCode.additional_flux_checks}
\lean{QEC1.GrossCode.additional_flux_checks}
\leanok
\uses{def:GrossCode.gaugingCyclesList, thm:GrossCode.gaugingCycles_count}
The number of additional flux checks $B_p$ (one per independent cycle) is 7.
\end{theorem}

\begin{proof}
\leanok
\uses{thm:GrossCode.gaugingCycles_count}
This follows directly from the cycle count of 7.
\end{proof}

\begin{theorem}[Additional Qubits]
\label{thm:GrossCode.additional_qubits}
\lean{QEC1.GrossCode.additional_qubits}
\leanok
\uses{def:GrossCode.gaugingGraph, thm:GrossCode.gaugingEdges_card}
The number of additional edge qubits (one per edge of $G$) is 22.
\end{theorem}

\begin{proof}
\leanok
\uses{thm:GrossCode.gaugingEdges_card}
This follows directly from the edge count $|E(G)| = 22$.
\end{proof}

\begin{theorem}[Total Overhead]
\label{thm:GrossCode.total_overhead}
\lean{QEC1.GrossCode.total_overhead}
\leanok
\uses{def:GrossCode.gaugingVertices, def:GrossCode.gaugingCyclesList, def:GrossCode.gaugingGraph, thm:GrossCode.gaugingVertices_card, thm:GrossCode.gaugingCycles_count, thm:GrossCode.gaugingEdges_card}
The total overhead is $|V(G)| + |\text{cycles}| + |E(G)| = 12 + 7 + 22 = 41$ additional checks and qubits.
\end{theorem}

\begin{proof}
\leanok
\uses{thm:GrossCode.gaugingVertices_card, thm:GrossCode.gaugingCycles_count, thm:GrossCode.gaugingEdges_card}
Rewriting using $|V(G)| = 12$, $|\text{cycles}| = 7$, and $|E(G)| = 22$, we obtain $12 + 7 + 22 = 41$.
\end{proof}

\begin{theorem}[Expansion Edges Valid]
\label{thm:GrossCode.expansion_edges_valid}
\lean{QEC1.GrossCode.expansion_edges_valid}
\leanok
\uses{def:GrossCode.expansionEdges, def:GrossCode.gaugingGraph}
All 4 expansion edges are edges of the gauging graph: for every pair $(a,b) \in \text{expansionEdges}$, $a$ and $b$ are adjacent in $G$.
\end{theorem}

\begin{proof}
\leanok
\uses{def:GrossCode.expansionEdges, def:GrossCode.gaugingGraph}
This is verified by computation (native\_decide).
\end{proof}

\begin{theorem}[Expansion Edge Endpoints in Support]
\label{thm:GrossCode.expansion_edges_in_support}
\lean{QEC1.GrossCode.expansion_edges_in_support}
\leanok
\uses{def:GrossCode.expansionEdges, def:GrossCode.gaugingVertices}
All expansion edge endpoints belong to $\operatorname{supp}(f)$: for every $(a,b) \in \text{expansionEdges}$, both $a \in V(G)$ and $b \in V(G)$.
\end{theorem}

\begin{proof}
\leanok
\uses{def:GrossCode.expansionEdges, def:GrossCode.gaugingVertices}
This is verified by computation (native\_decide).
\end{proof}

\begin{theorem}[Degree Bound]
\label{thm:GrossCode.gaugingDegree_le_six}
\lean{QEC1.GrossCode.gaugingDegree_le_six}
\leanok
\uses{def:GrossCode.gaugingGraph}
The maximum degree in the gauging graph isat most 6: for every vertex $v$, $\deg(v) \leq 6$.
\end{theorem}

\begin{proof}
\leanok
\uses{def:GrossCode.gaugingGraph}
This is verified by computation (native\_decide).
\end{proof}

\begin{theorem}[Matching Edges Count]
\label{thm:GrossCode.matching_edges_count}
\lean{QEC1.GrossCode.matching_edges_count}
\leanok
\uses{def:GrossCode.gaugingGraph, def:GrossCode.expansionEdges}
The number of matching edges (non-expansion edges) is 18.
\end{theorem}

\begin{proof}
\leanok
\uses{def:GrossCode.gaugingGraph, def:GrossCode.expansionEdges}
This is verified by computation (native\_decide).
\end{proof}

\begin{theorem}[Expansion Edges Count]
\label{thm:GrossCode.expansion_edges_count}
\lean{QEC1.GrossCode.expansion_edges_count}
\leanok
\uses{def:GrossCode.gaugingGraph, def:GrossCode.expansionEdges}
The number of expansion edges is 4.
\end{theorem}

\begin{proof}
\leanok
\uses{def:GrossCode.gaugingGraph, def:GrossCode.expansionEdges}
This is verified by computation (native\_decide).
\end{proof}

\begin{theorem}[Adjacent Z Checks Count]
\label{thm:GrossCode.adjacent_z_checks_count}
\lean{QEC1.GrossCode.adjacent_z_checks_count}
\leanok
\uses{def:GrossCode.gaugingVertices, def:GrossCode.grossB}
The number of $Z$ checks adjacent to the support of $\bar{X}_\alpha$ is 18: the number of monomials $\beta$ such that there exists at least one $\gamma \in V(G)$ with $B(\beta - \gamma) \neq 0$ is 18.
\end{theorem}

\begin{proof}
\leanok
\uses{def:GrossCode.gaugingVertices, def:GrossCode.grossB}
This is verified by computation (native\_decide).
\end{proof}

\begin{theorem}[Gauging Construction Summary]
\label{thm:GrossCode.gauging_construction_summary}
\lean{QEC1.GrossCode.gauging_construction_summary}
\leanok
\uses{def:GrossCode.gaugingVertices, def:GrossCode.gaugingGraph, def:GrossCode.gaugingCyclesList, thm:GrossCode.gaugingVertices_card, thm:GrossCode.gaugingEdges_card, thm:GrossCode.gaugingCycles_count, thm:GrossCode.total_overhead, thm:GrossCode.gaugingDegree_le_six, thm:GrossCode.gaugingCycles_max_length}
The gauging construction for $\bar{X}_\alpha$ in the Gross code satisfies:
\begin{enumerate}
\item $|V(G)| = 12$,
\item $|E(G)| = 22$,
\item There are 7 independent cycles,
\item Total overhead: $12 + 7 + 22 = 41$,
\item Maximum degree $\leq 6$,
\item Maximum cycle length $\leq 4$.
\end{enumerate}
\end{theorem}

\begin{proof}
\leanok
\uses{thm:GrossCode.gaugingVertices_card, thm:GrossCode.gaugingEdges_card, thm:GrossCode.gaugingCycles_count, thm:GrossCode.total_overhead, thm:GrossCode.gaugingDegree_le_six, thm:GrossCode.gaugingCycles_max_length}
This follows directly by combining the previously established results: gaugingVertices\_card, gaugingEdges\_card, gaugingCycles\_count, total\_overhead, gaugingDegree\_le\_six, and gaugingCycles\_max\_length.
\end{proof}

%--- Rem_22: DoubleGrossCodeDefinition ---

\chapter{Rem 22: The Double Gross Code Definition}

The \textbf{Double Gross code} is a $[\![288, 12, 18]\!]$ bivariate bicycle (BB) code with parameters $\ell = 12$, $m = 12$, $A = x^3 + y^7 + y^2$, $B = y^3 + x^2 + x$. The logical operators take the form $\bar{X}_\alpha = X(\alpha f, 0)$ where $f$ has weight 18. The gauging measurement construction involves 18 vertices (the support of $f$), 27 matching edges plus 7 expansion edges (one double edge, for 34 edges counting multiplicity), and 13 independent cycles (out of a cycle rank of 17). The total overhead is $18 + 13 + 34 = 65$.

\section{Code Parameters}

\begin{definition}[Double Gross Monomial Group]
\label{def:DoubleGrossCode.DGMonomial}
\lean{QEC1.BivariateBicycle.DoubleGrossCode.DGMonomial}
\leanok
\uses{def:BivariateBicycle.BBMonomial}
The monomial group for the Double Gross code is $\operatorname{DGMonomial} := \operatorname{BBMonomial}(12, 12) = \mathbb{Z}_{12} \times \mathbb{Z}_{12}$.
\end{definition}

\begin{definition}[Double Gross Group Algebra]
\label{def:DoubleGrossCode.DGAlgebra}
\lean{QEC1.BivariateBicycle.DoubleGrossCode.DGAlgebra}
\leanok
\uses{def:BivariateBicycle.BBGroupAlgebra}
The group algebra for the Double Gross code is $\operatorname{DGAlgebra} := \operatorname{BBGroupAlgebra}(12, 12) = \mathbb{F}_2[x,y]/(x^{12} - 1, y^{12} - 1)$.
\end{definition}

\begin{definition}[Double Gross Qubit Type]
\label{def:DoubleGrossCode.DGQubit}
\lean{QEC1.BivariateBicycle.DoubleGrossCode.DGQubit}
\leanok
\uses{def:BivariateBicycle.BBQubit}
The qubit type for the Double Gross code is $\operatorname{DGQubit} := \operatorname{BBQubit}(12, 12)$.
\end{definition}

\begin{definition}[Polynomial $A$]
\label{def:DoubleGrossCode.doubleGrossA}
\lean{QEC1.BivariateBicycle.DoubleGrossCode.doubleGrossA}
\leanok
\uses{def:BivariateBicycle.bbMonomial}
The polynomial $A = x^3 + y^7 + y^2$ in $\mathbb{F}_2[x,y]/(x^{12} - 1, y^{12} - 1)$, defined as
\[
A = \operatorname{bbMonomial}(3,0) + \operatorname{bbMonomial}(0,7) + \operatorname{bbMonomial}(0,2).
\]
\end{definition}

\begin{definition}[Polynomial $B$]
\label{def:DoubleGrossCode.doubleGrossB}
\lean{QEC1.BivariateBicycle.DoubleGrossCode.doubleGrossB}
\leanok
\uses{def:BivariateBicycle.bbMonomial}
The polynomial $B = y^3 + x^2 + x$ in $\mathbb{F}_2[x,y]/(x^{12} - 1, y^{12} - 1)$, defined as
\[
B = \operatorname{bbMonomial}(0,3) + \operatorname{bbMonomial}(2,0) + \operatorname{bbMonomial}(1,0).
\]
\end{definition}

\section{Check Commutation}

\begin{theorem}[Double Gross X-Z Checks Commute]
\label{thm:DoubleGrossCode.doubleGrossChecks_XZ_commute}
\lean{QEC1.BivariateBicycle.DoubleGrossCode.doubleGrossChecks_XZ_commute}
\leanok
\uses{def:BivariateBicycle.BBMonomial, def:BivariateBicycle.bbCheckX, def:BivariateBicycle.bbCheckZ, def:DoubleGrossCode.doubleGrossA, def:DoubleGrossCode.doubleGrossB, thm:BivariateBicycle.bbCheckXZ_commute_iff}
For all $\alpha, \beta \in \operatorname{DGMonomial}$, the X-type check $\operatorname{bbCheckX}(A, B, \alpha)$ and the Z-type check $\operatorname{bbCheckZ}(A, B, \beta)$ commute as Pauli operators.
\end{theorem}

\begin{proof}
\leanok
\uses{thm:BivariateBicycle.bbCheckXZ_commute_iff, def:DoubleGrossCode.doubleGrossA, def:DoubleGrossCode.doubleGrossB}
We use classical logic and rewrite with the commutation criterion \texttt{bbCheckXZ\_commute\_iff}, which reduces the Pauli commutation condition to a sum-of-products identity over the monomial group. The resulting identity
\[
\sum_{\gamma} A(\gamma - \alpha) \cdot B(-(\gamma - \beta)) + \sum_{\delta} B(\delta - \alpha) \cdot A(-(\delta - \beta)) = 0
\]
for all $\alpha, \beta$ is verified by \texttt{native\_decide}.
\end{proof}

\begin{theorem}[Double Gross All Checks Commute]
\label{thm:DoubleGrossCode.doubleGrossChecks_commute}
\lean{QEC1.BivariateBicycle.DoubleGrossCode.doubleGrossChecks_commute}
\leanok
\uses{def:BivariateBicycle.BBCheckIndex, def:BivariateBicycle.bbCheck, def:BivariateBicycle.bbCheckX, def:BivariateBicycle.bbCheckZ, def:DoubleGrossCode.doubleGrossA, def:DoubleGrossCode.doubleGrossB, thm:BivariateBicycle.bbCheckX_commute, thm:BivariateBicycle.bbCheckZ_commute, thm:DoubleGrossCode.doubleGrossChecks_XZ_commute}
For all check indices $i, j \in \operatorname{BBCheckIndex}(12, 12)$, the checks $\operatorname{bbCheck}(A, B, i)$ and $\operatorname{bbCheck}(A, B, j)$ commute.
\end{theorem}

\begin{proof}
\leanok
\uses{thm:BivariateBicycle.bbCheckX_commute, thm:BivariateBicycle.bbCheckZ_commute, thm:DoubleGrossCode.doubleGrossChecks_XZ_commute}
We case-split on the check indices $i$ and $j$, each of which is either $\operatorname{inl}(\alpha)$ (X-type) or $\operatorname{inr}(\beta)$ (Z-type):
\begin{itemize}
\item \textbf{Case} $(\operatorname{inl}(\alpha_1), \operatorname{inl}(\alpha_2))$: Both are X-type checks, and their commutation follows from \texttt{bbCheckX\_commute}.
\item \textbf{Case} $(\operatorname{inl}(\alpha), \operatorname{inr}(\beta))$: An X-type and Z-type check commute by \texttt{doubleGrossChecks\_XZ\_commute}.
\item \textbf{Case} $(\operatorname{inr}(\beta), \operatorname{inl}(\alpha))$: By commutativity of Pauli commutation and the previous case.
\item \textbf{Case} $(\operatorname{inr}(\beta_1), \operatorname{inr}(\beta_2))$: Both are Z-type checks, and their commutation follows from \texttt{bbCheckZ\_commute}.
\end{itemize}
\end{proof}

\section{The Double Gross Stabilizer Code}

\begin{definition}[Double Gross Code]
\label{def:DoubleGrossCode.doubleGrossCode}
\lean{QEC1.BivariateBicycle.DoubleGrossCode.doubleGrossCode}
\leanok
\uses{def:StabilizerCode, def:BivariateBicycle.BBCheckIndex, def:BivariateBicycle.bbCheck, def:DoubleGrossCode.doubleGrossA, def:DoubleGrossCode.doubleGrossB, thm:DoubleGrossCode.doubleGrossChecks_commute}
The Double Gross code is a stabilizer code on qubits $\operatorname{DGQubit}$, with check index set $I = \operatorname{BBCheckIndex}(12, 12)$, check map $\operatorname{bbCheck}(A, B)$, and commutation proof from the theorem above.
\end{definition}

\begin{theorem}[Double Gross Code Has 288 Qubits]
\label{thm:DoubleGrossCode.doubleGrossCode_numQubits}
\lean{QEC1.BivariateBicycle.DoubleGrossCode.doubleGrossCode_numQubits}
\leanok
\uses{def:DoubleGrossCode.doubleGrossCode, def:StabilizerCode.numQubits}
The Double Gross code has $n = 288 = 2 \times 12 \times 12$ physical qubits.
\end{theorem}

\begin{proof}
\leanok
\uses{def:DoubleGrossCode.doubleGrossCode, def:StabilizerCode.numQubits}
By simplification using the definition of \texttt{numQubits}, the cardinality of the sum type, and the cardinality of the product $\mathbb{Z}_{12} \times \mathbb{Z}_{12}$, we obtain $|\operatorname{Fin}(12) \times \operatorname{Fin}(12)| + |\operatorname{Fin}(12) \times \operatorname{Fin}(12)| = 144 + 144 = 288$. The result follows by numerical computation.
\end{proof}

\begin{theorem}[Double Gross Code Has 288 Checks]
\label{thm:DoubleGrossCode.doubleGrossCode_numChecks}
\lean{QEC1.BivariateBicycle.DoubleGrossCode.doubleGrossCode_numChecks}
\leanok
\uses{def:DoubleGrossCode.doubleGrossCode, def:StabilizerCode.numChecks}
The Double Gross code has 288 checks: 144 X-type and 144 Z-type.
\end{theorem}

\begin{proof}
\leanok
\uses{def:DoubleGrossCode.doubleGrossCode, def:StabilizerCode.numChecks}
By simplification using the definition of \texttt{numChecks} and the cardinality of $\operatorname{BBCheckIndex}(12, 12) = (\mathbb{Z}_{12} \times \mathbb{Z}_{12}) \oplus (\mathbb{Z}_{12} \times \mathbb{Z}_{12})$, we obtain $144 + 144 = 288$ by numerical computation.
\end{proof}

\begin{theorem}[Double Gross Code Encodes $k = 12$ Logical Qubits]
\label{thm:DoubleGrossCode.doubleGrossCode_k}
\lean{QEC1.BivariateBicycle.DoubleGrossCode.doubleGrossCode_k}
\leanok
\uses{def:DoubleGrossCode.doubleGrossCode}
The Double Gross code encodes $k = 12$ logical qubits. This is computed as $k = 2 \times \operatorname{nullity}([A; B])$, where $[A; B]$ is the stacked convolution matrix of size $288 \times 144$ over $\mathbb{F}_2$. The rank of $[A; B]$ is 138, so the nullity is $144 - 138 = 6$, giving $k = 2 \times 6 = 12$. The factor of 2 comes from the CSS structure (6 X-type + 6 Z-type independent logical operators).
\end{theorem}

\begin{proof}
\leanok
\uses{def:DoubleGrossCode.doubleGrossCode}
Rewriting with the computed rank $\operatorname{f2GaussRank}(144, \text{rows}) = 138$ (verified by \texttt{native\_decide} via Gaussian elimination over $\mathbb{F}_2$), we obtain $2 \times (144 - 138) = 2 \times 6 = 12$.
\end{proof}

\section{Logical Operator Polynomial}

\begin{definition}[Polynomial $f$]
\label{def:DoubleGrossCode.doubleGrossF}
\lean{QEC1.BivariateBicycle.DoubleGrossCode.doubleGrossF}
\leanok
\uses{def:BivariateBicycle.bbMonomial}
The logical operator polynomial $f \in \mathbb{F}_2[x,y]/(x^{12} - 1, y^{12} - 1)$ is defined as:
\begin{align*}
f &= (1 + x + x^2 + x^7 + x^8 + x^9 + x^{10} + x^{11}) \\
  &\quad + (1 + x^6 + x^8 + x^{10})y^3 \\
  &\quad + (x^5 + x^6 + x^9 + x^{10})y^6 \\
  &\quad + (x^4 + x^8)y^9.
\end{align*}
\end{definition}

\section{Logical X Operators}

\begin{definition}[Logical X Operator $\bar{X}_\alpha$]
\label{def:DoubleGrossCode.dblLogicalXBar}
\lean{QEC1.BivariateBicycle.DoubleGrossCode.dblLogicalXBar}
\leanok
\uses{def:BivariateBicycle.pauliXBB, def:BivariateBicycle.bbShift, def:DoubleGrossCode.doubleGrossF}
For $\alpha \in \operatorname{DGMonomial}$, the logical X operator is $\bar{X}_\alpha = X(\alpha f, 0) = \operatorname{pauliXBB}(\operatorname{bbShift}(\alpha, f), 0)$.
\end{definition}

\begin{theorem}[$\bar{X}_\alpha$ Is Pure X-Type]
\label{thm:DoubleGrossCode.dblLogicalXBar_pure_X}
\lean{QEC1.BivariateBicycle.DoubleGrossCode.dblLogicalXBar_pure_X}
\leanok
\uses{def:DoubleGrossCode.dblLogicalXBar, def:BivariateBicycle.pauliXBB}
For all $\alpha$, the logical operator $\bar{X}_\alpha$ has no Z-support: $(\bar{X}_\alpha).\operatorname{zVec} = 0$.
\end{theorem}

\begin{proof}
\leanok
\uses{def:DoubleGrossCode.dblLogicalXBar, def:BivariateBicycle.pauliXBB}
By simplification using the definitions of $\operatorname{dblLogicalXBar}$ and $\operatorname{pauliXBB}$, the Z-component is identically zero.
\end{proof}

\begin{theorem}[$\bar{X}_\alpha$ Acts Only on L Qubits]
\label{thm:DoubleGrossCode.dblLogicalXBar_acts_only_on_L}
\lean{QEC1.BivariateBicycle.DoubleGrossCode.dblLogicalXBar_acts_only_on_L}
\leanok
\uses{def:DoubleGrossCode.dblLogicalXBar, def:BivariateBicycle.pauliXBB}
For all $\alpha$ and all R-type qubits $\delta$, $(\bar{X}_\alpha).\operatorname{xVec}(\operatorname{inr}(\delta)) = 0$.
\end{theorem}

\begin{proof}
\leanok
\uses{def:DoubleGrossCode.dblLogicalXBar, def:BivariateBicycle.pauliXBB}
By simplification using the definitions of $\operatorname{dblLogicalXBar}$ and $\operatorname{pauliXBB}$, the X-component on R-type qubits is zero.
\end{proof}

\begin{theorem}[$\bar{X}_\alpha$ Is Self-Inverse]
\label{thm:DoubleGrossCode.dblLogicalXBar_mul_self}
\lean{QEC1.BivariateBicycle.DoubleGrossCode.dblLogicalXBar_mul_self}
\leanok
\uses{def:DoubleGrossCode.dblLogicalXBar, def:BivariateBicycle.pauliXBB, thm:BivariateBicycle.pauliXBB_mul}
For all $\alpha$, $\bar{X}_\alpha \cdot \bar{X}_\alpha = \mathbf{1}$.
\end{theorem}

\begin{proof}
\leanok
\uses{def:DoubleGrossCode.dblLogicalXBar, thm:BivariateBicycle.pauliXBB_mul, def:BivariateBicycle.pauliXBB}
Unfolding$\operatorname{dblLogicalXBar}$ and applying $\operatorname{pauliXBB\_mul}$, we verify equality by extensionality on the X and Z components. For the X-component, we case-split on the qubit type:
\begin{itemize}
\item For L-type qubits ($\operatorname{inl}$): the X-component satisfies $f(\gamma - \alpha) + f(\gamma - \alpha) = 0$ by the characteristic-two identity $a + a = 0$.
\item For R-type qubits ($\operatorname{inr}$): the X-component is $0 + 0 = 0$ by the definition of $\operatorname{pauliXBB}$.
\end{itemize}
For the Z-component, both sides are zero by the definition of $\operatorname{pauliXBB}$.
\end{proof}

\section{Support and Weight}

\begin{theorem}[Polynomial $f$ Has Weight 18]
\label{thm:DoubleGrossCode.doubleGrossF_weight_eq_18}
\lean{QEC1.BivariateBicycle.DoubleGrossCode.doubleGrossF_weight_eq_18}
\leanok
\uses{def:DoubleGrossCode.doubleGrossF}
The polynomial $f$ has 18 nonzero coefficients:
\[
|\{\gamma \in \mathbb{Z}_{12} \times \mathbb{Z}_{12} : f(\gamma) \neq 0\}| = 18.
\]
\end{theorem}

\begin{proof}
\leanok
\uses{def:DoubleGrossCode.doubleGrossF}
This is verified by \texttt{native\_decide}, which evaluates $f$ at all $144$ monomials in $\mathbb{Z}_{12} \times \mathbb{Z}_{12}$ and counts the nonzero values.
\end{proof}

\begin{theorem}[$A$ Has 3 Nonzero Coefficients]
\label{thm:DoubleGrossCode.doubleGrossA_support_card}
\lean{QEC1.BivariateBicycle.DoubleGrossCode.doubleGrossA_support_card}
\leanok
\uses{def:DoubleGrossCode.doubleGrossA}
$|\{\gamma : A(\gamma) \neq 0\}| = 3$.
\end{theorem}

\begin{proof}
\leanok
\uses{def:DoubleGrossCode.doubleGrossA}
Verified by \texttt{native\_decide}.
\end{proof}

\begin{theorem}[$B$ Has 3 Nonzero Coefficients]
\label{thm:DoubleGrossCode.doubleGrossB_support_card}
\lean{QEC1.BivariateBicycle.DoubleGrossCode.doubleGrossB_support_card}
\leanok
\uses{def:DoubleGrossCode.doubleGrossB}
$|\{\gamma : B(\gamma) \neq 0\}| = 3$.
\end{theorem}

\begin{proof}
\leanok
\uses{def:DoubleGrossCode.doubleGrossB}
Verified by \texttt{native\_decide}.
\end{proof}

\begin{theorem}[$\bar{X}_\alpha$ Has Weight 18]
\label{thm:DoubleGrossCode.dblLogicalXBar_weight}
\lean{QEC1.BivariateBicycle.DoubleGrossCode.dblLogicalXBar_weight}
\leanok
\uses{def:DoubleGrossCode.dblLogicalXBar, def:DoubleGrossCode.doubleGrossF, def:BivariateBicycle.pauliXBB, def:BivariateBicycle.bbShift}
For all $\alpha \in \operatorname{DGMonomial}$, $\operatorname{weight}(\bar{X}_\alpha) = 18$.
\end{theorem}

\begin{proof}
\leanok
\uses{def:DoubleGrossCode.dblLogicalXBar, def:DoubleGrossCode.doubleGrossF, def:BivariateBicycle.pauliXBB}
Using classical logic, we unfold the weight as the cardinality of the support. We apply the shift-invariance lemma, which shows that the support cardinality of $\operatorname{pauliXBB}(\operatorname{bbShift}(\alpha, f), 0)$ equals that of $\operatorname{pauliXBB}(f, 0)$ for any $\alpha$ (via a bijection $\gamma \mapsto \gamma - \alpha$ on L-type qubits). The base case $\operatorname{weight}(\operatorname{pauliXBB}(f, 0)) = 18$ is then verified by \texttt{native\_decide}.
\end{proof}

\section{Commutation with Checks}

\begin{theorem}[$\bar{X}_\alpha$ Commutes with X Checks]
\label{thm:DoubleGrossCode.dblLogicalXBar_commute_xCheck}
\lean{QEC1.BivariateBicycle.DoubleGrossCode.dblLogicalXBar_commute_xCheck}
\leanok
\uses{def:DoubleGrossCode.dblLogicalXBar, def:BivariateBicycle.bbCheckX, def:DoubleGrossCode.doubleGrossA, def:DoubleGrossCode.doubleGrossB, def:BivariateBicycle.pauliXBB}
For all $\alpha, \beta$, $\bar{X}_\alpha$ commutes with $\operatorname{bbCheckX}(A, B, \beta)$.
\end{theorem}

\begin{proof}
\leanok
\uses{def:DoubleGrossCode.dblLogicalXBar, def:BivariateBicycle.bbCheckX, def:BivariateBicycle.pauliXBB}
Using classical logic, we simplify the Pauli commutation condition and the symplectic inner product. Since both $\bar{X}_\alpha$ and $\operatorname{bbCheckX}(A, B, \beta)$ are pure X-type operators (having no Z-support), their symplectic inner product is trivially zero.
\end{proof}

\begin{theorem}[$\bar{X}_\alpha$ Commutes with Z Checks]
\label{thm:DoubleGrossCode.dblLogicalXBar_commute_zCheck}
\lean{QEC1.BivariateBicycle.DoubleGrossCode.dblLogicalXBar_commute_zCheck}
\leanok
\uses{def:DoubleGrossCode.dblLogicalXBar, def:BivariateBicycle.bbCheckZ, def:BivariateBicycle.pauliXBB, def:BivariateBicycle.pauliZBB, def:BivariateBicycle.bbShift, def:BivariateBicycle.bbTranspose, def:DoubleGrossCode.doubleGrossA, def:DoubleGrossCode.doubleGrossB, def:DoubleGrossCode.doubleGrossF}
For all $\alpha, \beta$, $\bar{X}_\alpha$ commutes with $\operatorname{bbCheckZ}(A, B, \beta)$.
\end{theorem}

\begin{proof}
\leanok
\uses{def:DoubleGrossCode.dblLogicalXBar, def:BivariateBicycle.bbCheckZ, def:BivariateBicycle.pauliZBB, def:BivariateBicycle.pauliXBB, def:BivariateBicycle.bbShift, def:BivariateBicycle.bbTranspose, def:DoubleGrossCode.doubleGrossF, def:DoubleGrossCode.doubleGrossB}
Using classical logic, we expand the Pauli commutation condition and the symplectic inner product. Unfolding the definitions of $\bar{X}_\alpha$ (pure X-type) and $\operatorname{bbCheckZ}$ (pure Z-type), the cross terms reduce the symplectic inner product to:
\[
\sum_{\gamma \in \mathbb{Z}_{12} \times \mathbb{Z}_{12}} f(\gamma - \alpha) \cdot B^T(\gamma - \beta) = 0.
\]
We split the sum over L and R qubit types using $\operatorname{Fintype.sum\_sum\_type}$; the R-qubit contribution vanishes since $\bar{X}_\alpha$ has no X-support on R qubits. After re-indexing the sum via the substitution $\delta = \gamma - \alpha$, we obtain $\sum_\delta f(\delta) \cdot B(-(\delta - (\beta - \alpha))) = 0$, which follows from the kernel condition $f \in \ker(B^T)$ verified by \texttt{native\_decide}.
\end{proof}

\begin{theorem}[$\bar{X}_\alpha$ Commutes with All Checks]
\label{thm:DoubleGrossCode.dblLogicalXBar_commute_allChecks}
\lean{QEC1.BivariateBicycle.DoubleGrossCode.dblLogicalXBar_commute_allChecks}
\leanok
\uses{def:DoubleGrossCode.dblLogicalXBar, def:BivariateBicycle.BBCheckIndex, def:BivariateBicycle.bbCheck, def:DoubleGrossCode.doubleGrossA, def:DoubleGrossCode.doubleGrossB, thm:DoubleGrossCode.dblLogicalXBar_commute_xCheck, thm:DoubleGrossCode.dblLogicalXBar_commute_zCheck}
For all $\alpha$ and all check indices $i \in \operatorname{BBCheckIndex}(12, 12)$, $\bar{X}_\alpha$ commutes with $\operatorname{bbCheck}(A, B, i)$.
\end{theorem}

\begin{proof}
\leanok
\uses{thm:DoubleGrossCode.dblLogicalXBar_commute_xCheck, thm:DoubleGrossCode.dblLogicalXBar_commute_zCheck}
We case-split on the check index $i$: if $i = \operatorname{inl}(\beta)$ (X-type), the result follows from \texttt{dblLogicalXBar\_commute\_xCheck}; if $i = \operatorname{inr}(\beta)$ (Z-type), the result follows from \texttt{dblLogicalXBar\_commute\_zCheck}.
\end{proof}

\begin{theorem}[$\bar{X}_\alpha$ Is in the Centralizer]
\label{thm:DoubleGrossCode.dblLogicalXBar_inCentralizer}
\lean{QEC1.BivariateBicycle.DoubleGrossCode.dblLogicalXBar_inCentralizer}
\leanok
\uses{def:DoubleGrossCode.doubleGrossCode, def:DoubleGrossCode.dblLogicalXBar, def:StabilizerCode.inCentralizer, thm:DoubleGrossCode.dblLogicalXBar_commute_allChecks}
For all $\alpha$, $\bar{X}_\alpha$ is in the centralizer of the Double Gross code.
\end{theorem}

\begin{proof}
\leanok
\uses{thm:DoubleGrossCode.dblLogicalXBar_commute_allChecks, def:StabilizerCode.inCentralizer}
Let $i$ be an arbitrary check index. The commutation $\bar{X}_\alpha$ with check $i$ follows directly from \texttt{dblLogicalXBar\_commute\_allChecks}.
\end{proof}

\section{The Gauging Graph}

\begin{definition}[Gauging Vertices]
\label{def:DoubleGrossCode.dblGaugingVertices}
\lean{QEC1.BivariateBicycle.DoubleGrossCode.dblGaugingVertices}
\leanok
\uses{def:DoubleGrossCode.doubleGrossF}
The gauging vertices are the support of $f$:
\[
\operatorname{dblGaugingVertices} = \{\gamma \in \mathbb{Z}_{12} \times \mathbb{Z}_{12} : f(\gamma) \neq 0\}.
\]
\end{definition}

\begin{theorem}[18 Gauging Vertices]
\label{thm:DoubleGrossCode.dblGaugingVertices_card}
\lean{QEC1.BivariateBicycle.DoubleGrossCode.dblGaugingVertices_card}
\leanok
\uses{def:DoubleGrossCode.dblGaugingVertices}
$|\operatorname{dblGaugingVertices}| = 18$.
\end{theorem}

\begin{proof}
\leanok
\uses{def:DoubleGrossCode.dblGaugingVertices}
Verified by \texttt{native\_decide}.
\end{proof}

\begin{definition}[Matching Difference Condition]
\label{def:DoubleGrossCode.dblIsMatchingDiff}
\lean{QEC1.BivariateBicycle.DoubleGrossCode.dblIsMatchingDiff}
\leanok
\uses{def:DoubleGrossCode.doubleGrossB}
A monomial difference $d \in \mathbb{Z}_{12} \times \mathbb{Z}_{12}$ satisfies the matching condition if $d = -B_i + B_j$ for some distinct terms $B_i, B_j$ of $B$, where the terms of $B = y^3 + x^2 + x$ are $(0,3)$, $(2,0)$, and $(1,0)$.
\end{definition}

\begin{definition}[Matching Edge Predicate]
\label{def:DoubleGrossCode.dblIsMatchingEdge}
\lean{QEC1.BivariateBicycle.DoubleGrossCode.dblIsMatchingEdge}
\leanok
\uses{def:DoubleGrossCode.doubleGrossF, def:DoubleGrossCode.dblIsMatchingDiff}
Two monomials $\gamma, \delta$ form a matching edge if both lie in $\operatorname{supp}(f)$ and share a Z check, i.e., $f(\gamma) \neq 0$, $f(\delta) \neq 0$, and $\operatorname{dblIsMatchingDiff}(\gamma - \delta)$.
\end{definition}

\begin{definition}[Expansion Edges]
\label{def:DoubleGrossCode.dblExpansionEdges}
\lean{QEC1.BivariateBicycle.DoubleGrossCode.dblExpansionEdges}
\leanok
\uses{def:DoubleGrossCode.doubleGrossF}
The 6 distinct expansion edges (as ordered pairs in $\mathbb{Z}_{12} \times \mathbb{Z}_{12}$):
\begin{align*}
&(x^4 y^9,\; x^9 y^6), \quad (y^3,\; x^{11}), \quad (x^7,\; x^{10} y^6), \\
&(x^8 y^3,\; x^{10} y^6), \quad (1,\; x^8), \quad (x^2,\; x^6 y^3).
\end{align*}
The edge $(x^2, x^6 y^3)$ has multiplicity 2 in the paper's construction (yielding 7 expansion edges counting multiplicity).
\end{definition}

\begin{definition}[Expansion Edge Predicate]
\label{def:DoubleGrossCode.dblIsExpansionEdge}
\lean{QEC1.BivariateBicycle.DoubleGrossCode.dblIsExpansionEdge}
\leanok
\uses{def:DoubleGrossCode.dblExpansionEdges}
$\operatorname{dblIsExpansionEdge}(\gamma, \delta)$ holds iff $(\gamma, \delta)$ or $(\delta, \gamma)$ appears in the expansion edge list.
\end{definition}

\begin{definition}[Gauging Adjacency]
\label{def:DoubleGrossCode.dblGaugingAdj}
\lean{QEC1.BivariateBicycle.DoubleGrossCode.dblGaugingAdj}
\leanok
\uses{def:DoubleGrossCode.dblIsMatchingEdge, def:DoubleGrossCode.dblIsExpansionEdge}
Two monomials $\gamma, \delta$ are adjacent in the gauging graph if $\gamma \neq \delta$ and either they form a matching edge or an expansion edge.
\end{definition}

\begin{definition}[Gauging Graph]
\label{def:DoubleGrossCode.dblGaugingGraph}
\lean{QEC1.BivariateBicycle.DoubleGrossCode.dblGaugingGraph}
\leanok
\uses{def:DoubleGrossCode.dblGaugingAdj}
The gauging graph $G$ for measuring $\bar{X}_\alpha$ in the Double Gross code is the simple graph on $\mathbb{Z}_{12} \times \mathbb{Z}_{12}$ with adjacency $\operatorname{dblGaugingAdj}$. Its symmetry is verified by \texttt{native\_decide}, and irreflexivity follows from the $\gamma \neq \delta$ condition.
\end{definition}

\section{Edge Counts}

\begin{theorem}[27 Matching Edges]
\label{thm:DoubleGrossCode.dblMatching_edges_count}
\lean{QEC1.BivariateBicycle.DoubleGrossCode.dblMatching_edges_count}
\leanok
\uses{def:DoubleGrossCode.dblGaugingGraph, def:DoubleGrossCode.dblExpansionEdges}
The gauging graph has 27 matching edges (edges that are not expansion edges).
\end{theorem}

\begin{proof}
\leanok
\uses{def:DoubleGrossCode.dblGaugingGraph, def:DoubleGrossCode.dblExpansionEdges}
Verified by \texttt{native\_decide}.
\end{proof}

\begin{theorem}[6 Distinct Expansion Edges]
\label{thm:DoubleGrossCode.dblExpansion_edges_count}
\lean{QEC1.BivariateBicycle.DoubleGrossCode.dblExpansion_edges_count}
\leanok
\uses{def:DoubleGrossCode.dblGaugingGraph, def:DoubleGrossCode.dblExpansionEdges}
The gauging graph has 6 distinct expansion edges.
\end{theorem}

\begin{proof}
\leanok
\uses{def:DoubleGrossCode.dblGaugingGraph, def:DoubleGrossCode.dblExpansionEdges}
Verified by \texttt{native\_decide}.
\end{proof}

\begin{theorem}[33 Distinct Edges]
\label{thm:DoubleGrossCode.dblGaugingEdges_card}
\lean{QEC1.BivariateBicycle.DoubleGrossCode.dblGaugingEdges_card}
\leanok
\uses{def:DoubleGrossCode.dblGaugingGraph}
The gauging graph has 33 distinct edges as a simple graph. Counting the one double edge $(x^2, x^6 y^3)$ with multiplicity 2, the total is 34.
\end{theorem}

\begin{proof}
\leanok
\uses{def:DoubleGrossCode.dblGaugingGraph}
Verified by \texttt{native\_decide}.
\end{proof}

\begin{theorem}[34 Edges with Multiplicity]
\label{thm:DoubleGrossCode.dblGaugingEdges_with_multiplicity}
\lean{QEC1.BivariateBicycle.DoubleGrossCode.dblGaugingEdges_with_multiplicity}
\leanok
\uses{def:DoubleGrossCode.dblGaugingGraph, thm:DoubleGrossCode.dblGaugingEdges_card}
$|\operatorname{edgeFinset}(G)| + 1 = 33 + 1 = 34$.
\end{theorem}

\begin{proof}
\leanok
\uses{thm:DoubleGrossCode.dblGaugingEdges_card}
Rewriting with $|\operatorname{edgeFinset}(G)| = 33$.
\end{proof}

\section{Expansion Edge Validity}

\begin{theorem}[Expansion Edges Are Valid]
\label{thm:DoubleGrossCode.dblExpansion_edges_valid}
\lean{QEC1.BivariateBicycle.DoubleGrossCode.dblExpansion_edges_valid}
\leanok
\uses{def:DoubleGrossCode.dblGaugingGraph, def:DoubleGrossCode.dblExpansionEdges}
All 6 distinct expansion edges are edges of the gauging graph.
\end{theorem}

\begin{proof}
\leanok
\uses{def:DoubleGrossCode.dblGaugingGraph, def:DoubleGrossCode.dblExpansionEdges}
Verified by \texttt{native\_decide}.
\end{proof}

\begin{theorem}[Expansion Edge Endpoints in Support]
\label{thm:DoubleGrossCode.dblExpansion_edges_in_support}
\lean{QEC1.BivariateBicycle.DoubleGrossCode.dblExpansion_edges_in_support}
\leanok
\uses{def:DoubleGrossCode.dblExpansionEdges, def:DoubleGrossCode.dblGaugingVertices}
All expansion edge endpoints lie in $\operatorname{supp}(f)$.
\end{theorem}

\begin{proof}
\leanok
\uses{def:DoubleGrossCode.dblExpansionEdges, def:DoubleGrossCode.dblGaugingVertices}
Verified by \texttt{native\_decide}.
\end{proof}

\section{Graph Connectivity and Cycle Rank}

\begin{theorem}[Gauging Graph Connected on Support]
\label{thm:DoubleGrossCode.dblGaugingGraph_connected_on_support}
\lean{QEC1.BivariateBicycle.DoubleGrossCode.dblGaugingGraph_connected_on_support}
\leanok
\uses{def:DoubleGrossCode.dblGaugingGraph, def:DoubleGrossCode.dblGaugingVertices, def:DoubleGrossCode.doubleGrossF}
The gauging graph is connected on the 18-vertex support of $f$. This is verified by a BFS traversal starting from the vertex $(0, 0) \in \operatorname{supp}(f)$, confirming that all 18 vertices are reachable.
\end{theorem}

\begin{proof}
\leanok
\uses{def:DoubleGrossCode.dblGaugingGraph, def:DoubleGrossCode.doubleGrossF}
Verified by \texttt{native\_decide}.
\end{proof}

\begin{theorem}[Simple Cycle Rank]
\label{thm:DoubleGrossCode.dblCycleRank_simple}
\lean{QEC1.BivariateBicycle.DoubleGrossCode.dblCycleRank_simple}
\leanok
\uses{def:DoubleGrossCode.dblGaugingGraph, def:DoubleGrossCode.dblGaugingVertices, thm:DoubleGrossCode.dblGaugingEdges_card, thm:DoubleGrossCode.dblGaugingVertices_card}
By Euler's formula for a connected graph, the cycle rank (first Betti number) of the simple graph is $|E| - |V| + 1 = 33 - 18 + 1 = 16$.
\end{theorem}

\begin{proof}
\leanok
\uses{thm:DoubleGrossCode.dblGaugingEdges_card, thm:DoubleGrossCode.dblGaugingVertices_card}
Rewriting with $|E| = 33$ and $|V| = 18$, we verify $33 + 1 = 18 + 16$.
\end{proof}

\begin{theorem}[Cycle Rank with Multiplicity]
\label{thm:DoubleGrossCode.dblCycleRank_with_multiplicity}
\lean{QEC1.BivariateBicycle.DoubleGrossCode.dblCycleRank_with_multiplicity}
\leanok
\uses{def:DoubleGrossCode.dblGaugingGraph, def:DoubleGrossCode.dblGaugingVertices, thm:DoubleGrossCode.dblGaugingEdges_card, thm:DoubleGrossCode.dblGaugingVertices_card}
With one double edge, the effective cycle rank is $(|E| + 1) - |V| + 1 = 34 - 18 + 1 = 17$.
\end{theorem}

\begin{proof}
\leanok
\uses{thm:DoubleGrossCode.dblGaugingEdges_card, thm:DoubleGrossCode.dblGaugingVertices_card}
Rewriting with $|E| = 33$ and $|V| = 18$, we verify $(33 + 1) + 1 = 18 + 17$.
\end{proof}

\begin{theorem}[Flux Checks Bounded by Cycle Rank]
\label{thm:DoubleGrossCode.dblFluxChecks_le_cycleRank}
\lean{QEC1.BivariateBicycle.DoubleGrossCode.dblFluxChecks_le_cycleRank}
\leanok
\uses{thm:DoubleGrossCode.dblCycleRank_with_multiplicity}
The number of flux checks (13) is bounded by the cycle rank with multiplicity (17): $13 \leq 17$. The paper shows that 13 of the 17 independent cycles suffice because 4 become redundant due to dependencies among Z checks restricted to $\operatorname{supp}(f)$.
\end{theorem}

\begin{proof}
\leanok
\uses{thm:DoubleGrossCode.dblCycleRank_with_multiplicity}
By integer arithmetic, $13 \leq 17$.
\end{proof}

\section{Overhead Calculation}

\begin{theorem}[Additional Gauss's Law Checks]
\label{thm:DoubleGrossCode.dblAdditional_gauss_checks}
\lean{QEC1.BivariateBicycle.DoubleGrossCode.dblAdditional_gauss_checks}
\leanok
\uses{def:DoubleGrossCode.dblGaugingVertices, thm:DoubleGrossCode.dblGaugingVertices_card}
The number of additional Gauss's law checks ($A_v$, one per vertex of $G$) is 18.
\end{theorem}

\begin{proof}
\leanok
\uses{thm:DoubleGrossCode.dblGaugingVertices_card}
This is exactly \texttt{dblGaugingVertices\_card}.
\end{proof}

\begin{theorem}[Restricted Z-Check Rank]
\label{thm:DoubleGrossCode.dblRestrictedZCheckRank_eq}
\lean{QEC1.BivariateBicycle.DoubleGrossCode.dblRestrictedZCheckRank_eq}
\leanok
\uses{def:DoubleGrossCode.dblGaugingVertices, def:DoubleGrossCode.doubleGrossB, def:DoubleGrossCode.doubleGrossF}
The restricted Z-check matrix (with rows indexed by $\beta \in \mathbb{Z}_{12} \times \mathbb{Z}_{12}$ and columns indexed by $\gamma \in \operatorname{supp}(f)$, with entry $B(\beta - \gamma) \neq 0$) has rank 17. For a connected 18-vertex subgraph, the boundary map has rank $|V| - 1 = 17$, which matches.
\end{theorem}

\begin{proof}
\leanok
\uses{def:DoubleGrossCode.doubleGrossB, def:DoubleGrossCode.doubleGrossF}
Verified by \texttt{native\_decide} via Gaussian elimination over $\mathbb{F}_2$.
\end{proof}

\begin{theorem}[Flux Check Upper Bound]
\label{thm:DoubleGrossCode.dblNumFluxChecks_upper_bound}
\lean{QEC1.BivariateBicycle.DoubleGrossCode.dblNumFluxChecks_upper_bound}
\leanok
\uses{def:DoubleGrossCode.dblGaugingGraph, thm:DoubleGrossCode.dblGaugingEdges_card, thm:DoubleGrossCode.dblRestrictedZCheckRank_eq}
The number of independent flux checks is $|E_{\text{mult}}| - \operatorname{restrictedZCheckRank} = 34 - 17 = 17$ for the full cycle space. The paper shows 4 of these are already implied by original code Z checks, leaving 13 independent flux checks.
\end{theorem}

\begin{proof}
\leanok
\uses{thm:DoubleGrossCode.dblGaugingEdges_card, thm:DoubleGrossCode.dblRestrictedZCheckRank_eq}
Rewriting with $|E| = 33$ and $\operatorname{restrictedZCheckRank} = 17$, we verify $(33 + 1) - 17 = 17$.
\end{proof}

\begin{theorem}[34 Additional Qubits]
\label{thm:DoubleGrossCode.dblAdditional_qubits}
\lean{QEC1.BivariateBicycle.DoubleGrossCode.dblAdditional_qubits}
\leanok
\uses{def:DoubleGrossCode.dblGaugingGraph, thm:DoubleGrossCode.dblGaugingEdges_with_multiplicity}
The number of additional qubits (one per edge of $G$, counting multiplicity) is 34.
\end{theorem}

\begin{proof}
\leanok
\uses{thm:DoubleGrossCode.dblGaugingEdges_with_multiplicity}
This is exactly \texttt{dblGaugingEdges\_with\_multiplicity}.
\end{proof}

\begin{theorem}[Total Overhead Is 65]
\label{thm:DoubleGrossCode.dblTotal_overhead}
\lean{QEC1.BivariateBicycle.DoubleGrossCode.dblTotal_overhead}
\leanok
\uses{def:DoubleGrossCode.dblGaugingVertices, def:DoubleGrossCode.dblGaugingGraph, thm:DoubleGrossCode.dblGaugingVertices_card, thm:DoubleGrossCode.dblGaugingEdges_card}
The total overhead is $|\text{vertices}| + |\text{flux checks}| + |\text{edge qubits}| = 18 + 13 + 34 = 65$. The 13 comes from the paper's claim that 13 of the 17 independent cycles (cycle rank with multiplicity) suffice for the flux checks.
\end{theorem}

\begin{proof}
\leanok
\uses{thm:DoubleGrossCode.dblGaugingVertices_card, thm:DoubleGrossCode.dblGaugingEdges_card}
Rewriting with $|\operatorname{dblGaugingVertices}| = 18$ and $|\operatorname{edgeFinset}(G)| = 33$, we verify $18 + 13 + (33 + 1) = 65$.
\end{proof}

\section{Tanner Expansion Property}

\begin{theorem}[Tanner Expansion Insufficient]
\label{thm:DoubleGrossCode.dblLogicalXBar_tanner_expansion_insufficient}
\lean{QEC1.BivariateBicycle.DoubleGrossCode.dblLogicalXBar_tanner_expansion_insufficient}
\leanok
\uses{def:DoubleGrossCode.dblLogicalXBar, def:DoubleGrossCode.doubleGrossA, thm:DoubleGrossCode.dblLogicalXBar_weight, thm:DoubleGrossCode.dblLogicalXBar_pure_X, thm:DoubleGrossCode.dblLogicalXBar_acts_only_on_L, thm:DoubleGrossCode.doubleGrossA_support_card}
For all $\alpha$: (1) $\bar{X}_\alpha$ has weight 18, (2) $\bar{X}_\alpha$ is pure X-type ($\operatorname{zVec} = 0$), (3) $\bar{X}_\alpha$ acts only on L qubits ($\operatorname{xVec}(\operatorname{inr}(\delta)) = 0$ for all $\delta$), and (4) each L qubit participates in at most $|\operatorname{supp}(A)| = 3$ X-type checks. This shows that the Tanner graph expansion is insufficient: a weight-18 operator only triggers $\leq 3 \times 18 = 54$ check violations, which may not suffice for robust error detection.
\end{theorem}

\begin{proof}
\leanok
\uses{thm:DoubleGrossCode.dblLogicalXBar_weight, thm:DoubleGrossCode.dblLogicalXBar_pure_X, thm:DoubleGrossCode.dblLogicalXBar_acts_only_on_L, thm:DoubleGrossCode.doubleGrossA_support_card}
The four properties are assembled from the previously proven theorems: \texttt{dblLogicalXBar\_weight}, \texttt{dblLogicalXBar\_pure\_X}, \texttt{dblLogicalXBar\_acts\_only\_on\_L}, and \texttt{doubleGrossA\_support\_card}.
\end{proof}

\section{Check Weight}

\begin{theorem}[X Check Weight Is 6]
\label{thm:DoubleGrossCode.dblXCheck_weight}
\lean{QEC1.BivariateBicycle.DoubleGrossCode.dblXCheck_weight}
\leanok
\uses{def:BivariateBicycle.bbCheckX, def:DoubleGrossCode.doubleGrossA, def:DoubleGrossCode.doubleGrossB}
Each X-type check of the Double Gross code has weight 6: for all $\alpha \in \operatorname{DGMonomial}$, $\operatorname{weight}(\operatorname{bbCheckX}(A, B, \alpha)) = 6$.
\end{theorem}

\begin{proof}
\leanok
\uses{def:BivariateBicycle.bbCheckX, def:DoubleGrossCode.doubleGrossA, def:DoubleGrossCode.doubleGrossB}
Using classical logic, we unfold the weight as the support cardinality. We apply a shift-invariance argument: the support of $\operatorname{bbCheckX}(A, B, \alpha)$ has the same cardinality as that of $\operatorname{bbCheckX}(A, B, 0)$, via the injective bijection $q \mapsto q - \alpha$ on qubit indices (separately on L and R components). The Z-component of an X-type check is zero, so the support reduces to the X-component support. The base case $\operatorname{weight}(\operatorname{bbCheckX}(A, B, 0)) = 6$ is verified by \texttt{native\_decide}.
\end{proof}

\section{Summary}

\begin{theorem}[Double Gross Code Parameters]
\label{thm:DoubleGrossCode.doubleGrossCode_parameters}
\lean{QEC1.BivariateBicycle.DoubleGrossCode.doubleGrossCode_parameters}
\leanok
\uses{def:DoubleGrossCode.doubleGrossCode, thm:DoubleGrossCode.doubleGrossCode_numQubits, thm:DoubleGrossCode.doubleGrossCode_numChecks}
The Double Gross code has 288 qubits and 288 checks: $n = 288 \land |\text{checks}| = 288$.
\end{theorem}

\begin{proof}
\leanok
\uses{thm:DoubleGrossCode.doubleGrossCode_numQubits, thm:DoubleGrossCode.doubleGrossCode_numChecks}
This follows directly from \texttt{doubleGrossCode\_numQubits} and \texttt{doubleGrossCode\_numChecks}.
\end{proof}

\begin{theorem}[Gauging Construction Summary]
\label{thm:DoubleGrossCode.dblGauging_construction_summary}
\lean{QEC1.BivariateBicycle.DoubleGrossCode.dblGauging_construction_summary}
\leanok
\uses{def:DoubleGrossCode.dblGaugingVertices, def:DoubleGrossCode.dblGaugingGraph, thm:DoubleGrossCode.dblGaugingVertices_card, thm:DoubleGrossCode.dblGaugingEdges_card, thm:DoubleGrossCode.dblGaugingEdges_with_multiplicity, thm:DoubleGrossCode.dblCycleRank_with_multiplicity, thm:DoubleGrossCode.dblFluxChecks_le_cycleRank, thm:DoubleGrossCode.dblTotal_overhead}
Summary of the gauging construction for $\bar{X}_\alpha$ in the Double Gross code:
\begin{enumerate}
\item $|\operatorname{dblGaugingVertices}| = 18$,
\item $|\operatorname{edgeFinset}(G)| = 33$,
\item $|\operatorname{edgeFinset}(G)| + 1 = 34$ (with multiplicity),
\item $(|\operatorname{edgeFinset}(G)| + 1) + 1 = |\operatorname{dblGaugingVertices}| + 17$ (cycle rank),
\item $13 \leq 17$ (flux checks $\leq$ cycle rank),
\item $|\operatorname{dblGaugingVertices}| + 13 + (|\operatorname{edgeFinset}(G)| + 1) = 65$ (total overhead).
\end{enumerate}
\end{theorem}

\begin{proof}
\leanok
\uses{thm:DoubleGrossCode.dblGaugingVertices_card, thm:DoubleGrossCode.dblGaugingEdges_card, thm:DoubleGrossCode.dblGaugingEdges_with_multiplicity, thm:DoubleGrossCode.dblCycleRank_with_multiplicity, thm:DoubleGrossCode.dblFluxChecks_le_cycleRank, thm:DoubleGrossCode.dblTotal_overhead}
The six-tuple is assembled directly from the previously proven theorems: \texttt{dblGaugingVertices\_card}, \texttt{dblGaugingEdges\_card}, \texttt{dblGaugingEdges\_with\_multiplicity}, \texttt{dblCycleRank\_with\_multiplicity}, \texttt{dblFluxChecks\_le\_cycleRank}, and \texttt{dblTotal\_overhead}.
\end{proof}

%--- Rem_23: GeneralizationsBeyondPauli ---
\chapter{Rem 23: Generalizations Beyond Pauli}

This remark establishes that the gauging measurement procedure generalizes beyond Pauli logical operators on qubits in three directions: (1) qudit systems where $\mathbb{Z}_2$ is replaced by $\mathbb{Z}_p$, (2) abelian group charges where local measurements determine global charge, and (3) nonabelian groups where local measurements leave the global charge ambiguous.

\section{Qudit Generalization: Boundary Maps over $\mathbb{Z}_p$}

\begin{definition}[Qudit Boundary Map]
\label{def:GeneralizationsBeyondPauli.quditBoundaryMap}
\lean{QEC1.GeneralizationsBeyondPauli.quditBoundaryMap}
\leanok
\uses{def:GraphMaps.boundaryMap}
The \emph{qudit boundary map} $\partial : \mathbb{Z}_p^E \to \mathbb{Z}_p^V$ generalizes the boundary map from $\mathbb{Z}_2$ to $\mathbb{Z}_p$. For $\gamma \in \mathbb{Z}_p^E$, the value at vertex $v$ is
\[
(\partial \gamma)_v = \sum_{\substack{e \in E \\ v \in e}} \gamma_e \pmod{p}.
\]
This is a $\mathbb{Z}_p$-linear map, with linearity verified by distributing sums and scalar multiplication over the conditional summation.
\end{definition}

\begin{definition}[Qudit Coboundary Map]
\label{def:GeneralizationsBeyondPauli.quditCoboundaryMap}
\lean{QEC1.GeneralizationsBeyondPauli.quditCoboundaryMap}
\leanok
\uses{def:GraphMaps.coboundaryMap}
The \emph{qudit coboundary map} $\delta : \mathbb{Z}_p^V \to \mathbb{Z}_p^E$ generalizes the coboundary map from $\mathbb{Z}_2$ to $\mathbb{Z}_p$. For $f \in \mathbb{Z}_p^V$ and edge $e = \{a,b\}$, the value is
\[
(\delta f)_e = f(a) + f(b) \pmod{p}.
\]
This is a $\mathbb{Z}_p$-linear map.
\end{definition}

\begin{definition}[Qudit Second Boundary Map]
\label{def:GeneralizationsBeyondPauli.quditSecondBoundaryMap}
\lean{QEC1.GeneralizationsBeyondPauli.quditSecondBoundaryMap}
\leanok
\uses{def:GraphMaps.secondBoundaryMap}
The \emph{qudit second boundary map} $\partial_2 : \mathbb{Z}_p^C \to \mathbb{Z}_p^E$ generalizes the second boundary map from $\mathbb{Z}_2$ to $\mathbb{Z}_p$. For $\sigma \in \mathbb{Z}_p^C$, the value at edge $e$ is
\[
(\partial_2 \sigma)_e = \sum_{\substack{c \in C \\ e \in \text{cycles}(c)}} \sigma_c \pmod{p}.
\]
\end{definition}

\begin{definition}[Qudit Second Coboundary Map]
\label{def:GeneralizationsBeyondPauli.quditSecondCoboundaryMap}
\lean{QEC1.GeneralizationsBeyondPauli.quditSecondCoboundaryMap}
\leanok
\uses{def:GraphMaps.secondCoboundaryMap}
The \emph{qudit second coboundary map} $\delta_2 : \mathbb{Z}_p^E \to \mathbb{Z}_p^C$ generalizes the second coboundary map from $\mathbb{Z}_2$ to $\mathbb{Z}_p$. For $\gamma \in \mathbb{Z}_p^E$, the value at cycle $c$ is
\[
(\delta_2 \gamma)_c = \sum_{\substack{e \in E \\ e \in \text{cycles}(c)}} \gamma_e \pmod{p}.
\]
\end{definition}

\section{Transpose Properties}

\begin{theorem}[Qudit Coboundary is Transpose of Boundary]
\label{thm:GeneralizationsBeyondPauli.quditCoboundaryMap_eq_transpose}
\lean{QEC1.GeneralizationsBeyondPauli.quditCoboundaryMap_eq_transpose}
\leanok
\uses{def:GeneralizationsBeyondPauli.quditCoboundaryMap, def:GeneralizationsBeyondPauli.quditBoundaryMap}
The coboundary map $\delta$ is the transpose of the boundary map $\partial$ over $\mathbb{Z}_p$: for all $f \in \mathbb{Z}_p^V$ and $\gamma \in \mathbb{Z}_p^E$,
\[
\sum_{e \in E} (\delta f)_e \cdot \gamma_e = \sum_{v \in V} f_v \cdot (\partial \gamma)_v.
\]
\end{theorem}

\begin{proof}
\leanok
\uses{def:GeneralizationsBeyondPauli.quditCoboundaryMap, def:GeneralizationsBeyondPauli.quditBoundaryMap}
Expanding the definitions, we rewrite each side using the defining sums. We then apply the commutativity of the double summation via \texttt{Finset.sum\_comm}. For each edge $e = \{a, b\}$, we use the fact that the graph is loopless (so $a \neq b$) and decompose $(f(a) + f(b)) \cdot \gamma_e = f(a) \cdot \gamma_e + f(b) \cdot \gamma_e$. We then establish a key identity: for each vertex $x$, $f(x) \cdot (\text{if } x \in \{a,b\} \text{ then } \gamma_e \text{ else } 0)$ equals $f(a) \cdot \gamma_e$ when $x = a$, $f(b) \cdot \gamma_e$ when $x = b$, and $0$ otherwise. Rewriting with this identity and splitting the sum into two parts (one for $a$ and one for $b$), we apply \texttt{Finset.sum\_add\_distrib} and simplify using \texttt{Finset.sum\_ite\_eq'} to conclude.
\end{proof}

\begin{theorem}[Qudit Second Coboundary is Transpose of Second Boundary]
\label{thm:GeneralizationsBeyondPauli.quditSecondCoboundaryMap_eq_transpose}
\lean{QEC1.GeneralizationsBeyondPauli.quditSecondCoboundaryMap_eq_transpose}
\leanok
\uses{def:GeneralizationsBeyondPauli.quditSecondCoboundaryMap, def:GeneralizationsBeyondPauli.quditSecondBoundaryMap}
The second coboundary map $\delta_2$ is the transpose of the second boundary map $\partial_2$ over $\mathbb{Z}_p$: for all $\gamma \in \mathbb{Z}_p^E$ and $\sigma \in \mathbb{Z}_p^C$,
\[
\sum_{c \in C} (\delta_2 \gamma)_c \cdot \sigma_c = \sum_{e \in E} \gamma_e \cdot (\partial_2 \sigma)_e.
\]
\end{theorem}

\begin{proof}
\leanok
\uses{def:GeneralizationsBeyondPauli.quditSecondCoboundaryMap, def:GeneralizationsBeyondPauli.quditSecondBoundaryMap}
Expanding the definitions, we distribute multiplication over the inner sums on both sides. We then exchange the order of summation via \texttt{Finset.sum\_comm}. The result follows by showing that the summands agree: when the membership condition holds, the terms are equal by ring arithmetic; when the condition fails, both sides are zero.
\end{proof}

\section{Chain Complex Property}

\begin{theorem}[Chain Complex Property over $\mathbb{Z}_p$]
\label{thm:GeneralizationsBeyondPauli.quditBoundary_comp_secondBoundary_eq_zero}
\lean{QEC1.GeneralizationsBeyondPauli.quditBoundary_comp_secondBoundary_eq_zero}
\leanok
\uses{def:GeneralizationsBeyondPauli.quditBoundaryMap, def:GeneralizationsBeyondPauli.quditSecondBoundaryMap}
The chain complex property $\partial \circ \partial_2 = 0$ holds over $\mathbb{Z}_p$, provided each cycle $c$ satisfies the condition that for every vertex $v$, $p$ divides the number of edges in $c$ incident to $v$. Over $\mathbb{Z}_2$, this is the standard cycle condition (even incidence). Over $\mathbb{Z}_p$ for odd $p$, this requires $p \mid (\text{incidence count})$.
\end{theorem}

\begin{proof}
\leanok
\uses{def:GeneralizationsBeyondPauli.quditBoundaryMap, def:GeneralizationsBeyondPauli.quditSecondBoundaryMap}
We apply linear map extensionality: let $\sigma$ be arbitrary and show $(\partial \circ \partial_2)(\sigma) = 0$ pointwise at each vertex $v$. Expanding both definitions, we exchange the order of summation (swapping the sum over edges with the sum over cycles). After combining conditionals using the identity $(\text{if } P \text{ then } (\text{if } Q \text{ then } a \text{ else } 0) \text{ else } 0) = (\text{if } P \wedge Q \text{ then } a \text{ else } 0)$, we rewrite each inner sum as $|\{e : e \in \text{cycles}(c) \wedge v \in e\}| \cdot \sigma(c)$. By hypothesis, $p$ divides this cardinality, so its cast to $\mathbb{Z}_p$ is zero, giving $0 \cdot \sigma(c) = 0$. The outer sum of zeros is zero.
\end{proof}

\begin{theorem}[$\operatorname{im}(\partial_2) \leq \ker(\partial)$ over $\mathbb{Z}_p$]
\label{thm:GeneralizationsBeyondPauli.qudit_range_secondBoundary_le_ker_boundary}
\lean{QEC1.GeneralizationsBeyondPauli.qudit_range_secondBoundary_le_ker_boundary}
\leanok
\uses{def:GeneralizationsBeyondPauli.quditBoundaryMap, def:GeneralizationsBeyondPauli.quditSecondBoundaryMap, thm:GeneralizationsBeyondPauli.quditBoundary_comp_secondBoundary_eq_zero}
Under the same cycle incidence hypothesis, $\operatorname{im}(\partial_2) \subseteq \ker(\partial)$ over $\mathbb{Z}_p$.
\end{theorem}

\begin{proof}
\leanok
\uses{thm:GeneralizationsBeyondPauli.quditBoundary_comp_secondBoundary_eq_zero}
Let $\gamma \in \operatorname{im}(\partial_2)$, so $\gamma = \partial_2(\sigma)$ for some $\sigma$. We need $\partial(\gamma) = 0$. By the chain complex property $\partial \circ \partial_2 = 0$, we have $\partial(\partial_2(\sigma)) = 0$, and substituting $\gamma = \partial_2(\sigma)$ gives $\partial(\gamma) = 0$.
\end{proof}

\section{Specialization to $\mathbb{Z}_2$}

\begin{theorem}[Qudit Boundary Specializes to Qubit]
\label{thm:GeneralizationsBeyondPauli.quditBoundaryMap_specializes_to_qubit}
\lean{QEC1.GeneralizationsBeyondPauli.quditBoundaryMap_specializes_to_qubit}
\leanok
\uses{def:GeneralizationsBeyondPauli.quditBoundaryMap, def:GraphMaps.boundaryMap}
The qudit boundary map at $p = 2$ agrees with the standard $\mathbb{Z}_2$ boundary map: for all $\gamma \in \mathbb{Z}_2^E$ and $v \in V$,
\[
(\partial_{\mathbb{Z}_2} \gamma)_v = (\partial_{\text{Def\_1}} \gamma)_v.
\]
\end{theorem}

\begin{proof}
\leanok
\uses{def:GeneralizationsBeyondPauli.quditBoundaryMap, def:GraphMaps.boundaryMap}
This holds by simplification, as both definitions compute the same conditional sum over incident edges.
\end{proof}

\begin{theorem}[Qudit Coboundary Specializes to Qubit]
\label{thm:GeneralizationsBeyondPauli.quditCoboundaryMap_specializes_to_qubit}
\lean{QEC1.GeneralizationsBeyondPauli.quditCoboundaryMap_specializes_to_qubit}
\leanok
\uses{def:GeneralizationsBeyondPauli.quditCoboundaryMap, def:GraphMaps.coboundaryMap}
The qudit coboundary map at $p = 2$ agrees with the standard $\mathbb{Z}_2$ coboundary map.
\end{theorem}

\begin{proof}
\leanok
\uses{def:GeneralizationsBeyondPauli.quditCoboundaryMap, def:GraphMaps.coboundaryMap}
This holds by simplification of the definitions.
\end{proof}

\begin{theorem}[Qudit Second Boundary Specializes to Qubit]
\label{thm:GeneralizationsBeyondPauli.quditSecondBoundaryMap_specializes_to_qubit}
\lean{QEC1.GeneralizationsBeyondPauli.quditSecondBoundaryMap_specializes_to_qubit}
\leanok
\uses{def:GeneralizationsBeyondPauli.quditSecondBoundaryMap, def:GraphMaps.secondBoundaryMap}
The qudit second boundary map at $p = 2$ agrees with the standard $\mathbb{Z}_2$ second boundary map.
\end{theorem}

\begin{proof}
\leanok
\uses{def:GeneralizationsBeyondPauli.quditSecondBoundaryMap, def:GraphMaps.secondBoundaryMap}
This holds by simplification of the definitions.
\end{proof}

\begin{theorem}[Qudit Second Coboundary Specializes to Qubit]
\label{thm:GeneralizationsBeyondPauli.quditSecondCoboundaryMap_specializes_to_qubit}
\lean{QEC1.GeneralizationsBeyondPauli.quditSecondCoboundaryMap_specializes_to_qubit}
\leanok
\uses{def:GeneralizationsBeyondPauli.quditSecondCoboundaryMap, def:GraphMaps.secondCoboundaryMap}
The qudit second coboundary map at $p = 2$ agrees with the standard $\mathbb{Z}_2$ second coboundary map.
\end{theorem}

\begin{proof}
\leanok
\uses{def:GeneralizationsBeyondPauli.quditSecondCoboundaryMap, def:GraphMaps.secondCoboundaryMap}
This holds by simplification of the definitions.
\end{proof}

\section{Abelian Group Charge Determination}

\begin{theorem}[Abelian Charge Sum is Well-Defined]
\label{thm:GeneralizationsBeyondPauli.abelian_charge_sum_well_defined}
\lean{QEC1.GeneralizationsBeyondPauli.abelian_charge_sum_well_defined}
\leanok

For an abelian (additive commutative) monoid, the sum of local charges is independent of the ordering. For any charges $(q_1, \ldots, q_n)$ and any permutation $\pi$ of $\{1, \ldots, n\}$,
\[
\sum_{i=1}^{n} q_{\pi(i)} = \sum_{i=1}^{n} q_i.
\]
This is the core mathematical fact that makes abelian gauging work: measuring individual $A_v$ operators in any order gives the same total charge.
\end{theorem}

\begin{proof}
\leanok

This follows directly from \texttt{Equiv.sum\_comp}, which states that summing a function over a permutation of the index set gives the same result.
\end{proof}

% COMMENTED OUT: Disconnected cluster (abelian sum permutation invariance pair)
% \begin{theorem}[Abelian Sum Permutation Invariance (General)]
% \label{thm:GeneralizationsBeyondPauli.abelian_sum_perm_invariant}
% ... and abelian_local_determines_global
% \end{theorem}

\section{Nonabelian Groups: Product Order Dependence}

\begin{theorem}[Nonabelian Product is Order Dependent]
\label{thm:GeneralizationsBeyondPauli.nonabelian_product_order_dependent}
\lean{QEC1.GeneralizationsBeyondPauli.nonabelian_product_order_dependent}
\leanok

For nonabelian groups, the product of elements depends on the order of multiplication. If $g_1 \cdot g_2 \neq g_2 \cdot g_1$, then the two orderings $[g_1, g_2]$ and $[g_2, g_1]$ give different products:
\[
g_1 \cdot g_2 \neq g_2 \cdot g_1 \implies \prod [g_1, g_2] \neq \prod [g_2, g_1].
\]
This is the fundamental obstruction to measuring nonabelian charges locally.
\end{theorem}

\begin{proof}
\leanok

Simplifying \texttt{List.prod} on two-element lists, we have $\prod [g_1, g_2] = g_1 \cdot g_2 \cdot 1 = g_1 \cdot g_2$ and similarly $\prod [g_2, g_1] = g_2 \cdot g_1$. The conclusion follows directly from the hypothesis $g_1 \cdot g_2 \neq g_2 \cdot g_1$.
\end{proof}

% COMMENTED OUT: Isolated declaration (not connected to dependency graph)
% \begin{theorem}[Abelian Product is Permutation Invariant]
% \label{thm:GeneralizationsBeyondPauli.abelian_product_perm_invariant}
% \lean{QEC1.GeneralizationsBeyondPauli.abelian_product_perm_invariant}
% \leanok
%
% For a commutative (abelian) monoid, the product of any list is invariant under permutation. If $l_1$ is a permutation of $l_2$, then $\prod l_1 = \prod l_2$.
% \end{theorem}
%
% \begin{proof}
% \leanok
%
% This follows directly from \texttt{List.Perm.prod\_eq}, which establishes that permuting a list preserves its product in a commutative monoid.
% \end{proof}

% COMMENTED OUT: Disconnected cluster (abelian product permutation invariance pair)
% \begin{theorem}[Abelian Fintype Product Permutation Invariance]
% \label{thm:GeneralizationsBeyondPauli.abelian_fintype_prod_perm_invariant}
% ... and abelian_all_perms_same_product
% \end{theorem}

\section{Qudit Gauss's Law Operators (Generalized)}

\begin{definition}[Qudit Hypergraph Boundary Map]
\label{def:GeneralizationsBeyondPauli.quditHyperBoundaryMap}
\lean{QEC1.GeneralizationsBeyondPauli.quditHyperBoundaryMap}
\leanok
\uses{def:HypergraphGeneralization.hyperBoundaryMap}
The \emph{qudit hypergraph boundary map} $\partial : \mathbb{Z}_p^E \to \mathbb{Z}_p^V$ generalizes both the graph case and the hypergraph case from $\mathbb{Z}_2$ to $\mathbb{Z}_p$. For $\gamma \in \mathbb{Z}_p^E$,
\[
(\partial \gamma)_v = \sum_{\substack{e \in E \\ v \sim e}} \gamma_e \pmod{p},
\]
where $v \sim e$ means $v$ is incident to hyperedge $e$.
\end{definition}

\begin{definition}[Qudit Hypergraph Coboundary Map]
\label{def:GeneralizationsBeyondPauli.quditHyperCoboundaryMap}
\lean{QEC1.GeneralizationsBeyondPauli.quditHyperCoboundaryMap}
\leanok
\uses{def:HypergraphGeneralization.hyperCoboundaryMap}
The \emph{qudit hypergraph coboundary map} $\delta : \mathbb{Z}_p^V \to \mathbb{Z}_p^E$ generalizes the hypergraph coboundary from $\mathbb{Z}_2$ to $\mathbb{Z}_p$. For $f \in \mathbb{Z}_p^V$,
\[
(\delta f)_e = \sum_{\substack{v \in V \\ v \sim e}} f(v) \pmod{p}.
\]
\end{definition}

\begin{theorem}[Qudit Hypergraph Coboundary is Transpose]
\label{thm:GeneralizationsBeyondPauli.quditHyperCoboundaryMap_eq_transpose}
\lean{QEC1.GeneralizationsBeyondPauli.quditHyperCoboundaryMap_eq_transpose}
\leanok
\uses{def:GeneralizationsBeyondPauli.quditHyperBoundaryMap, def:GeneralizationsBeyondPauli.quditHyperCoboundaryMap}
The qudit hypergraph coboundary is the transpose of the boundary over $\mathbb{Z}_p$:
\[
\sum_{e \in E} (\delta f)_e \cdot \gamma_e = \sum_{v \in V} f_v \cdot (\partial \gamma)_v.
\]
\end{theorem}

\begin{proof}
\leanok
\uses{def:GeneralizationsBeyondPauli.quditHyperBoundaryMap, def:GeneralizationsBeyondPauli.quditHyperCoboundaryMap}
Expanding the definitions, we distribute multiplication over the inner sums on both sides, then exchange the order of summation via \texttt{Finset.sum\_comm}. For each pair $(v, e)$, when the incidence condition holds, the terms agree by ring arithmetic; when it fails, both sides are zero.
\end{proof}

\begin{theorem}[Qudit Hypergraph Boundary Specializes to $\mathbb{Z}_2$]
\label{thm:GeneralizationsBeyondPauli.quditHyperBoundaryMap_specializes_to_Z2}
\lean{QEC1.GeneralizationsBeyondPauli.quditHyperBoundaryMap_specializes_to_Z2}
\leanok
\uses{def:GeneralizationsBeyondPauli.quditHyperBoundaryMap, def:HypergraphGeneralization.hyperBoundaryMap}
The qudit hypergraph boundary map at $p = 2$ agrees with the $\mathbb{Z}_2$ version from Rem~17.
\end{theorem}

\begin{proof}
\leanok
\uses{def:GeneralizationsBeyondPauli.quditHyperBoundaryMap, def:HypergraphGeneralization.hyperBoundaryMap}
This holds by simplification of the definitions.
\end{proof}

\begin{theorem}[Qudit Hypergraph Coboundary Specializes to $\mathbb{Z}_2$]
\label{thm:GeneralizationsBeyondPauli.quditHyperCoboundaryMap_specializes_to_Z2}
\lean{QEC1.GeneralizationsBeyondPauli.quditHyperCoboundaryMap_specializes_to_Z2}
\leanok
\uses{def:GeneralizationsBeyondPauli.quditHyperCoboundaryMap, def:HypergraphGeneralization.hyperCoboundaryMap}
The qudit hypergraph coboundary map at $p = 2$ agrees with the $\mathbb{Z}_2$ version from Rem~17.
\end{theorem}

\begin{proof}
\leanok
\uses{def:GeneralizationsBeyondPauli.quditHyperCoboundaryMap, def:HypergraphGeneralization.hyperCoboundaryMap}
This holds by simplification of the definitions.
\end{proof}

\section{Nonabelian Local vs.\ Global Charge}

\begin{theorem}[Nonabelian Local Underdetermines Global]
\label{thm:GeneralizationsBeyondPauli.nonabelian_local_underdetermines_global}
\lean{QEC1.GeneralizationsBeyondPauli.nonabelian_local_underdetermines_global}
\leanok
\uses{thm:GeneralizationsBeyondPauli.nonabelian_product_order_dependent}
In a nonabelian group, knowing the individual elements $g_v$ does \emph{not} determine their product uniquely, because the product depends on the order of multiplication. Formally, if $g_1 \cdot g_2 \neq g_2 \cdot g_1$, then it is \emph{not} the case that for all lists $l_1, l_2$ with $l_1$ a permutation of $l_2$, $\prod l_1 = \prod l_2$:
\[
g_1 g_2 \neq g_2 g_1 \implies \neg\bigl(\forall l_1, l_2,\; l_1 \sim l_2 \implies \textstyle\prod l_1 = \textstyle\prod l_2\bigr).
\]
\end{theorem}

\begin{proof}
\leanok
\uses{thm:GeneralizationsBeyondPauli.nonabelian_product_order_dependent}
Assume for contradiction that all permutations preserve products. The lists $[g_1, g_2]$ and $[g_2, g_1]$ are permutations of each other (by a swap). By the assumption, $\prod [g_1, g_2] = \prod [g_2, g_1]$. Simplifying, $g_1 \cdot g_2 = g_2 \cdot g_1$, contradicting the hypothesis.
\end{proof}

% COMMENTED OUT: Isolated declarations (not connected to dependency graph)
% \begin{theorem}[Abelian Local Determines Global (Multiplicative)]
% \label{thm:GeneralizationsBeyondPauli.abelian_local_determines_global'}
% \lean{QEC1.GeneralizationsBeyondPauli.abelian_local_determines_global'}
% \leanok
%
% In a commutative group, the product of elements is uniquely determined regardless of ordering. If $l_1$ is a permutation of $l_2$, then $\prod l_1 = \prod l_2$.
% \end{theorem}
%
% \begin{proof}
% \leanok
%
% This follows directly from \texttt{List.Perm.prod\_eq}.
% \end{proof}
%
% \begin{theorem}[Charge Dichotomy Iff Commutativity]
% \label{thm:GeneralizationsBeyondPauli.charge_dichotomy_iff_comm}
% \lean{QEC1.GeneralizationsBeyondPauli.charge_dichotomy_iff_comm}
% \leanok
%
% The abelian-nonabelian dichotomy: a group has the property that all 2-element ordered products are the same if and only if it is commutative:
% \[
% \bigl(\forall g_1, g_2,\; \textstyle\prod [g_1, g_2] = \textstyle\prod [g_2, g_1]\bigr) \iff \bigl(\forall g_1, g_2,\; g_1 g_2 = g_2 g_1\bigr).
% \]
% \end{theorem}
%
% \begin{proof}
% \leanok
%
% Simplifying $\prod [g_1, g_2] = g_1 \cdot g_2 \cdot 1 = g_1 \cdot g_2$ and $\prod [g_2, g_1] = g_2 \cdot g_1$, the equivalence becomes tautological.
% \end{proof}

\section{Summary}

\begin{theorem}[Generalizations Beyond Pauli Summary]
\label{thm:GeneralizationsBeyondPauli.generalizations_beyond_pauli_summary}
\lean{QEC1.GeneralizationsBeyondPauli.generalizations_beyond_pauli_summary}
\leanok
\uses{thm:GeneralizationsBeyondPauli.quditCoboundaryMap_eq_transpose, thm:GeneralizationsBeyondPauli.quditBoundaryMap_specializes_to_qubit, thm:GeneralizationsBeyondPauli.nonabelian_product_order_dependent, thm:GeneralizationsBeyondPauli.abelian_charge_sum_well_defined, def:GeneralizationsBeyondPauli.quditBoundaryMap, def:GeneralizationsBeyondPauli.quditCoboundaryMap, def:GraphMaps.boundaryMap}
The three generalizations beyond Pauli operators on qubits are:
\begin{enumerate}
\item \textbf{Qudit}: Boundary/coboundary maps generalize from $\mathbb{Z}_2$ to $\mathbb{Z}_p$, preserving linearity and the transpose property $\langle \delta f, \gamma \rangle_E = \langle f, \partial \gamma \rangle_V$. The chain complex property $\partial \circ \partial_2 = 0$ still holds.
\item \textbf{Abelian}: For abelian groups, the sum/product of local charges is order-independent, so measuring local charges determines the global charge: $\sum_i q_{\pi(i)} = \sum_i q_i$.
\item \textbf{Nonabelian}: For nonabelian groups, the product of local charges depends on the order, so local measurements do \emph{not} determine a definite global charge: $g_1 g_2 \neq g_2 g_1 \implies \prod [g_1, g_2] \neq \prod [g_2, g_1]$.
\end{enumerate}
\end{theorem}

\begin{proof}
\leanok
\uses{thm:GeneralizationsBeyondPauli.quditCoboundaryMap_eq_transpose, thm:GeneralizationsBeyondPauli.quditBoundaryMap_specializes_to_qubit, thm:GeneralizationsBeyondPauli.nonabelian_product_order_dependent, thm:GeneralizationsBeyondPauli.abelian_charge_sum_well_defined}
The four components are established as follows:
\begin{enumerate}
\item The transpose property of qudit maps follows from \texttt{quditCoboundaryMap\_eq\_transpose}.
\item Specialization to $\mathbb{Z}_2$ follows from \texttt{quditBoundaryMap\_specializes\_to\_qubit}.
\item Abelian permutation invariance follows from \texttt{Equiv.sum\_comp}.
\item Nonabelian order dependence follows from \texttt{nonabelian\_product\_order\_dependent}.
\end{enumerate}
\end{proof}

%--- Rem_24: ShorStyleMeasurement ---
\chapter{Rem 24: Shor-Style Measurement as Gauging}

This remark shows that the gauging measurement procedure recovers Shor-style logical measurement as a special case. Given a Pauli logical operator $L$ with weight $W = |\operatorname{supp}(L)|$, the Shor-style measurement uses an auxiliary GHZ state on $W$ qubits, transversal CX gates, and $X$ measurements on the auxiliary qubits. From the gauging viewpoint, one chooses a graph $G$ with vertex set $\operatorname{supp}(L) \cup \{d_1, \ldots, d_W\}$ where: (1) $W$ dummy vertices $d_i$ (one per qubit in $\operatorname{supp}(L)$); (2) a connected subgraph on the dummy vertices; (3) each $d_i$ connected by a single edge to the corresponding qubit $i$. After measuring the edges of the dummy subgraph first (the ungauging step), the remaining state on dummy qubits is equivalent to a GHZ state entangled with $\operatorname{supp}(L)$, recovering Shor-style measurement. The key advantage of the gauging viewpoint is that the dummy subgraph can be chosen freely (path graph, star graph, expander, etc.) to optimize for hardware constraints.

\section{Shor-Style Graph Construction}

\begin{definition}[Shor Vertex Type]
\label{def:ShorStyleMeasurement.ShorVertex}
\lean{QEC1.ShorStyleMeasurement.ShorVertex}
\leanok
\uses{def:PauliOp}
The vertex type for the Shor-style gauging graph is $\operatorname{Fin}(W) \oplus \operatorname{Fin}(W)$, representing support qubits and dummy qubits respectively. The total vertex count is $2W$.
\end{definition}

\begin{definition}[Shor Graph Adjacency]
\label{def:ShorStyleMeasurement.shorGraphAdj}
\lean{QEC1.ShorStyleMeasurement.shorGraphAdj}
\leanok
\uses{def:ShorStyleMeasurement.ShorVertex}
The adjacency relation for the Shor-style gauging graph. Two vertices $p, q$ are adjacent if $p \neq q$ and one of the following holds:
\begin{enumerate}
\item \textbf{Pairing edge:} there exists $i \in \operatorname{Fin}(W)$ with $p = \operatorname{inl}(i)$ and $q = \operatorname{inr}(i)$ (or vice versa), connecting support qubit $i$ to dummy vertex $d_i$.
\item \textbf{Dummy subgraph edge:} there exist $i, j \in \operatorname{Fin}(W)$ with $p = \operatorname{inr}(i)$, $q = \operatorname{inr}(j)$, and $i \sim j$ in $G_{\mathrm{dummy}}$.
\end{enumerate}
\end{definition}

\begin{lemma}[Shor Graph Adjacency is Symmetric]
\label{lem:ShorStyleMeasurement.shorGraphAdj_symm}
\lean{QEC1.ShorStyleMeasurement.shorGraphAdj_symm}
\leanok
\uses{def:ShorStyleMeasurement.shorGraphAdj}
If $\operatorname{shorGraphAdj}(W, G_{\mathrm{dummy}}, p, q)$ holds, then $\operatorname{shorGraphAdj}(W, G_{\mathrm{dummy}}, q, p)$ holds.
\end{lemma}

\begin{proof}
\leanok
\uses{def:ShorStyleMeasurement.shorGraphAdj}
Assume $p \neq q$ and the adjacency condition holds. Since $p \neq q$ implies $q \neq p$, it remains to check the disjunction. If the edge is a pairing edge $(\operatorname{inl}(i), \operatorname{inr}(i))$, then $(\operatorname{inr}(i), \operatorname{inl}(i))$ is also a pairing edge (by the second disjunct). If the edge is a reverse pairing edge $(\operatorname{inr}(i), \operatorname{inl}(i))$, then $(\operatorname{inl}(i), \operatorname{inr}(i))$ is a pairing edge. If the edge is a dummy subgraph edge with $G_{\mathrm{dummy}}.\operatorname{Adj}(i, j)$, then by symmetry of $G_{\mathrm{dummy}}$ we have $G_{\mathrm{dummy}}.\operatorname{Adj}(j, i)$.
\end{proof}

\begin{lemma}[Shor Graph Adjacency is Irreflexive]
\label{lem:ShorStyleMeasurement.shorGraphAdj_irrefl}
\lean{QEC1.ShorStyleMeasurement.shorGraphAdj_irrefl}
\leanok
\uses{def:ShorStyleMeasurement.shorGraphAdj}
For any vertex $p$, $\neg\operatorname{shorGraphAdj}(W, G_{\mathrm{dummy}}, p, p)$.
\end{lemma}

\begin{proof}
\leanok
\uses{def:ShorStyleMeasurement.shorGraphAdj}
Suppose $\operatorname{shorGraphAdj}(W, G_{\mathrm{dummy}}, p, p)$ holds. By definition, the first condition requires $p \neq p$, which is a contradiction.
\end{proof}

\begin{definition}[Shor Graph]
\label{def:ShorStyleMeasurement.ShorGraph}
\lean{QEC1.ShorStyleMeasurement.ShorGraph}
\leanok
\uses{def:ShorStyleMeasurement.shorGraphAdj, lem:ShorStyleMeasurement.shorGraphAdj_symm, lem:ShorStyleMeasurement.shorGraphAdj_irrefl}
The Shor-style gauging graph on $2W$ vertices. It is a simple graph on $\operatorname{ShorVertex}(W)$ whose adjacency is given by $\operatorname{shorGraphAdj}$, with symmetry established by \texttt{shorGraphAdj\_symm} and irreflexivity by \texttt{shorGraphAdj\_irrefl}. Its edges consist of $W$ pairing edges $\{(\operatorname{inl}(i), \operatorname{inr}(i))\}$ plus the dummy subgraph edges.
\end{definition}

\begin{theorem}[Shor Graph Vertex Count]
\label{thm:ShorStyleMeasurement.shorGraph_card}
\lean{QEC1.ShorStyleMeasurement.shorGraph_card}
\leanok
\uses{def:ShorStyleMeasurement.ShorVertex}
The Shor graph has $2W$ vertices:
\[
|\operatorname{ShorVertex}(W)| = 2W.
\]
\end{theorem}

\begin{proof}
\leanok
\uses{def:ShorStyleMeasurement.ShorVertex}
By the cardinality of a sum type, $|\operatorname{Fin}(W) \oplus \operatorname{Fin}(W)| = |\operatorname{Fin}(W)| + |\operatorname{Fin}(W)| = W + W = 2W$.
\end{proof}

\begin{theorem}[Pairing Edge Exists]
\label{thm:ShorStyleMeasurement.shor_pairing_edge}
\lean{QEC1.ShorStyleMeasurement.shor_pairing_edge}
\leanok
\uses{def:ShorStyleMeasurement.ShorGraph}
For each $i \in \operatorname{Fin}(W)$, $\operatorname{inl}(i)$ and $\operatorname{inr}(i)$ are adjacent in the Shor graph:
\[
(\operatorname{ShorGraph}(W, G_{\mathrm{dummy}})).\operatorname{Adj}(\operatorname{inl}(i), \operatorname{inr}(i)).
\]
\end{theorem}

\begin{proof}
\leanok
\uses{def:ShorStyleMeasurement.ShorGraph}
We have $\operatorname{inl}(i) \neq \operatorname{inr}(i)$ (they are in different summands), and $i$ witnesses the pairing edge condition $p = \operatorname{inl}(i) \land q = \operatorname{inr}(i)$.
\end{proof}

\begin{theorem}[Dummy Subgraph Edges Lift]
\label{thm:ShorStyleMeasurement.shor_dummy_edge}
\lean{QEC1.ShorStyleMeasurement.shor_dummy_edge}
\leanok
\uses{def:ShorStyleMeasurement.ShorGraph}
If $i \sim j$ in $G_{\mathrm{dummy}}$, then $\operatorname{inr}(i) \sim \operatorname{inr}(j)$ in the Shor graph.
\end{theorem}

\begin{proof}
\leanok
\uses{def:ShorStyleMeasurement.ShorGraph}
We must show $\operatorname{inr}(i) \neq \operatorname{inr}(j)$ and the adjacency condition. For the first, if $\operatorname{inr}(i) = \operatorname{inr}(j)$ then $i = j$ by injectivity of $\operatorname{inr}$, but $G_{\mathrm{dummy}}$ is loopless so $i \sim i$ is impossible. For the second, we produce witnesses $i, j$ with $G_{\mathrm{dummy}}.\operatorname{Adj}(i, j)$.
\end{proof}

\begin{definition}[Shor Dummy Homomorphism]
\label{def:ShorStyleMeasurement.shorDummyHom}
\lean{QEC1.ShorStyleMeasurement.shorDummyHom}
\leanok
\uses{def:ShorStyleMeasurement.ShorGraph, thm:ShorStyleMeasurement.shor_dummy_edge}
The graph homomorphism $G_{\mathrm{dummy}} \to \operatorname{ShorGraph}(W, G_{\mathrm{dummy}})$ given by $v \mapsto \operatorname{inr}(v)$, which maps edges of $G_{\mathrm{dummy}}$ to dummy subgraph edges of the Shor graph.
\end{definition}

\begin{theorem}[Shor Graph Connectivity]
\label{thm:ShorStyleMeasurement.shorGraph_connected}
\lean{QEC1.ShorStyleMeasurement.shorGraph_connected}
\leanok
\uses{def:ShorStyleMeasurement.ShorGraph, def:ShorStyleMeasurement.shorDummyHom, thm:ShorStyleMeasurement.shor_pairing_edge, lem:ShorStyleMeasurement.shorGraphAdj_symm}
If $W > 0$ and $G_{\mathrm{dummy}}$ is connected, then the Shor graph is connected.
\end{theorem}

\begin{proof}
\leanok
\uses{def:ShorStyleMeasurement.ShorGraph, def:ShorStyleMeasurement.shorDummyHom, thm:ShorStyleMeasurement.shor_pairing_edge, lem:ShorStyleMeasurement.shorGraphAdj_symm}
Fix a base vertex $\operatorname{inr}(0)$. It suffices to show that every vertex $w$ is reachable from $\operatorname{inr}(0)$.

\textbf{Case $w = \operatorname{inr}(j)$:} Since $G_{\mathrm{dummy}}$ is connected, vertex $0$ is reachable from $j$ in $G_{\mathrm{dummy}}$. Applying the graph homomorphism $\operatorname{shorDummyHom}$ (which maps via $\operatorname{inr}$), $\operatorname{inr}(0)$ is reachable from $\operatorname{inr}(j)$ in the Shor graph.

\textbf{Case $w = \operatorname{inl}(i)$:} By the same argument, $\operatorname{inr}(0)$ can reach $\operatorname{inr}(i)$ in the Shor graph. The pairing edge gives $\operatorname{inr}(i) \sim \operatorname{inl}(i)$, so $\operatorname{inr}(0)$ can reach $\operatorname{inl}(i)$ by transitivity.

The graph is nonempty since $W > 0$ guarantees $\operatorname{inr}(0)$ exists.
\end{proof}

\section{Extended Logical Operator}

\begin{definition}[Shor Logical Operator]
\label{def:ShorStyleMeasurement.shorLogicalOp}
\lean{QEC1.ShorStyleMeasurement.shorLogicalOp}
\leanok
\uses{def:PauliOp, def:ShorStyleMeasurement.ShorVertex}
The logical operator $L'$ on the Shor graph vertex set:
\[
L' = \prod_{v \in V \oplus D} X_v,
\]
defined as the Pauli operator with $\operatorname{xVec}(v) = 1$ for all $v$ and $\operatorname{zVec} = 0$. This is the extended logical from Rem~10 with $V = D = \operatorname{Fin}(W)$.
\end{definition}

\begin{theorem}[Shor Logical is Pure X-Type]
\label{thm:ShorStyleMeasurement.shorLogicalOp_pure_X}
\lean{QEC1.ShorStyleMeasurement.shorLogicalOp_pure_X}
\leanok
\uses{def:ShorStyleMeasurement.shorLogicalOp}
The logical operator $L'$ has no $Z$-support: $(L').\operatorname{zVec} = 0$.
\end{theorem}

\begin{proof}
\leanok
\uses{def:ShorStyleMeasurement.shorLogicalOp}
This holds by definition, since $\operatorname{zVec}$ is defined to be $0$.
\end{proof}

\begin{theorem}[Shor Logical is Self-Inverse]
\label{thm:ShorStyleMeasurement.shorLogicalOp_self_inverse}
\lean{QEC1.ShorStyleMeasurement.shorLogicalOp_self_inverse}
\leanok
\uses{def:ShorStyleMeasurement.shorLogicalOp, def:PauliOp.mul}
$L' \cdot L' = \mathbf{1}$.
\end{theorem}

\begin{proof}
\leanok
\uses{def:ShorStyleMeasurement.shorLogicalOp, def:PauliOp.mul}
This follows from the fact that every Pauli operator is self-inverse: $P \cdot P = \mathbf{1}$.
\end{proof}

\begin{theorem}[Shor Logical Weight]
\label{thm:ShorStyleMeasurement.shorLogicalOp_weight}
\lean{QEC1.ShorStyleMeasurement.shorLogicalOp_weight}
\leanok
\uses{def:ShorStyleMeasurement.shorLogicalOp, def:PauliOp.weight, def:PauliOp.support}
The weight of $L'$ is $2W$:
\[
\operatorname{weight}(L') = 2W.
\]
\end{theorem}

\begin{proof}
\leanok
\uses{def:ShorStyleMeasurement.shorLogicalOp, def:PauliOp.weight, def:PauliOp.support}
Since $\operatorname{xVec}(v) = 1 \neq 0$ for all $v$, every vertex is in the support of $L'$. The support filter returns all of $\operatorname{univ}$, so the weight equals $|\operatorname{Fin}(W) \oplus \operatorname{Fin}(W)| = W + W = 2W$.
\end{proof}

\begin{definition}[Original Logical Operator]
\label{def:ShorStyleMeasurement.shorOriginalLogical}
\lean{QEC1.ShorStyleMeasurement.shorOriginalLogical}
\leanok
\uses{def:PauliOp, def:ShorStyleMeasurement.ShorVertex}
The support-restricted logical operator: $L$ acts as $X$ on all $\operatorname{inl}$ qubits (support qubits) and as $I$ on all $\operatorname{inr}$ qubits (dummy qubits). Formally, $\operatorname{xVec}(\operatorname{inl}(i)) = 1$, $\operatorname{xVec}(\operatorname{inr}(i)) = 0$, and $\operatorname{zVec} = 0$.
\end{definition}

\begin{definition}[Dummy Product]
\label{def:ShorStyleMeasurement.shorDummyProduct}
\lean{QEC1.ShorStyleMeasurement.shorDummyProduct}
\leanok
\uses{def:PauliOp, def:ShorStyleMeasurement.ShorVertex}
The dummy product $\prod_{d \in D} X_d$ on dummy qubits only. Formally, $\operatorname{xVec}(\operatorname{inl}(i)) = 0$, $\operatorname{xVec}(\operatorname{inr}(i)) = 1$, and $\operatorname{zVec} = 0$.
\end{definition}

\begin{theorem}[Logical Factorization]
\label{thm:ShorStyleMeasurement.shorLogicalOp_eq_mul}
\lean{QEC1.ShorStyleMeasurement.shorLogicalOp_eq_mul}
\leanok
\uses{def:ShorStyleMeasurement.shorLogicalOp, def:ShorStyleMeasurement.shorOriginalLogical, def:ShorStyleMeasurement.shorDummyProduct, def:PauliOp.mul}
The full logical factorizes as $L' = L \cdot \prod_d X_d$:
\[
\operatorname{shorLogicalOp}(W) = \operatorname{shorOriginalLogical}(W) \cdot \operatorname{shorDummyProduct}(W).
\]
\end{theorem}

\begin{proof}
\leanok
\uses{def:ShorStyleMeasurement.shorLogicalOp, def:ShorStyleMeasurement.shorOriginalLogical, def:ShorStyleMeasurement.shorDummyProduct, def:PauliOp.mul}
By extensionality, we check each component. For $\operatorname{xVec}$: on $\operatorname{inl}(i)$, $1 = 1 + 0$; on $\operatorname{inr}(i)$, $1 = 0 + 1$. For $\operatorname{zVec}$: on all vertices, $0 = 0 + 0$. Each case follows by simplification of the multiplication definition.
\end{proof}

\begin{theorem}[Original Logical is Pure X-Type]
\label{thm:ShorStyleMeasurement.shorOriginalLogical_pure_X}
\lean{QEC1.ShorStyleMeasurement.shorOriginalLogical_pure_X}
\leanok
\uses{def:ShorStyleMeasurement.shorOriginalLogical}
$(L).\operatorname{zVec} = 0$.
\end{theorem}

\begin{proof}
\leanok
\uses{def:ShorStyleMeasurement.shorOriginalLogical}
This holds by definition.
\end{proof}

\begin{theorem}[Dummy Product is Pure X-Type]
\label{thm:ShorStyleMeasurement.shorDummyProduct_pure_X}
\lean{QEC1.ShorStyleMeasurement.shorDummyProduct_pure_X}
\leanok
\uses{def:ShorStyleMeasurement.shorDummyProduct}
$(\prod_d X_d).\operatorname{zVec} = 0$.
\end{theorem}

\begin{proof}
\leanok
\uses{def:ShorStyleMeasurement.shorDummyProduct}
This holds by definition.
\end{proof}

\begin{theorem}[Original Logical Commutes with Dummy Product]
\label{thm:ShorStyleMeasurement.shorOriginalLogical_comm_dummyProduct}
\lean{QEC1.ShorStyleMeasurement.shorOriginalLogical_comm_dummyProduct}
\leanok
\uses{def:ShorStyleMeasurement.shorOriginalLogical, def:ShorStyleMeasurement.shorDummyProduct, def:PauliOp.mul}
$L \cdot \prod_d X_d = \prod_d X_d \cdot L$.
\end{theorem}

\begin{proof}
\leanok
\uses{def:ShorStyleMeasurement.shorOriginalLogical, def:ShorStyleMeasurement.shorDummyProduct}
Both operators are pure $X$-type, so they commute by the commutativity of Pauli operator multiplication.
\end{proof}

\begin{theorem}[Original Logical is Self-Inverse]
\label{thm:ShorStyleMeasurement.shorOriginalLogical_self_inverse}
\lean{QEC1.ShorStyleMeasurement.shorOriginalLogical_self_inverse}
\leanok
\uses{def:ShorStyleMeasurement.shorOriginalLogical, def:PauliOp.mul}
$L \cdot L = \mathbf{1}$.
\end{theorem}

\begin{proof}
\leanok
\uses{def:ShorStyleMeasurement.shorOriginalLogical}
This follows from the general self-inverse property of Pauli operators.
\end{proof}

\begin{theorem}[Dummy Product is Self-Inverse]
\label{thm:ShorStyleMeasurement.shorDummyProduct_self_inverse}
\lean{QEC1.ShorStyleMeasurement.shorDummyProduct_self_inverse}
\leanok
\uses{def:ShorStyleMeasurement.shorDummyProduct, def:PauliOp.mul}
$\prod_d X_d \cdot \prod_d X_d = \mathbf{1}$.
\end{theorem}

\begin{proof}
\leanok
\uses{def:ShorStyleMeasurement.shorDummyProduct}
This follows from the general self-inverse property of Pauli operators.
\end{proof}

\section{Measurement Sign with Dummies}

\begin{theorem}[Measurement Sign with Dummies]
\label{thm:ShorStyleMeasurement.shor_measurement_sign_dummies}
\lean{QEC1.ShorStyleMeasurement.shor_measurement_sign_dummies}
\leanok
\uses{def:ShorStyleMeasurement.ShorVertex}
If all dummy outcomes are $0$ (i.e., all dummy qubits are in the $+1$ eigenstate of $X$), then the measurement sign over the full vertex set equals the sum over support outcomes only:
\[
\sum_{v \in V \oplus D} \varepsilon_v = \sum_{i \in \operatorname{Fin}(W)} \varepsilon_{\operatorname{inl}(i)}.
\]
\end{theorem}

\begin{proof}
\leanok
\uses{def:ShorStyleMeasurement.ShorVertex}
Decompose the sum over the sum type: $\sum_{v} f(v) = \sum_{i} f(\operatorname{inl}(i)) + \sum_{j} f(\operatorname{inr}(j))$. Since each dummy outcome $\varepsilon_{\operatorname{inr}(j)} = 0$ by hypothesis, the second sum vanishes, leaving only $\sum_i \varepsilon_{\operatorname{inl}(i)}$.
\end{proof}

\section{Gauging Structure on the Shor Graph}

\begin{theorem}[Gauss Operator on Shor Graph is Pure X-Type]
\label{thm:ShorStyleMeasurement.shor_gaussLaw_pure_X}
\lean{QEC1.ShorStyleMeasurement.shor_gaussLaw_pure_X}
\leanok
\uses{def:ShorStyleMeasurement.ShorGraph, def:GaussFlux.gaussLawOp}
For every vertex $v$ of the Shor graph, the Gauss's law operator $A_v$ is pure $X$-type:
\[
(A_v).\operatorname{zVec} = 0.
\]
\end{theorem}

\begin{proof}
\leanok
\uses{def:ShorStyleMeasurement.ShorGraph, def:GaussFlux.gaussLawOp}
By extensionality and simplification of the $\operatorname{gaussLawOp}$ definition, every component of the $\operatorname{zVec}$ is $0$.
\end{proof}

\begin{theorem}[Gauss Product Equals Logical]
\label{thm:ShorStyleMeasurement.shor_gauss_product_eq_logical}
\lean{QEC1.ShorStyleMeasurement.shor_gauss_product_eq_logical}
\leanok
\uses{def:ShorStyleMeasurement.ShorGraph, def:GaussFlux.gaussLawOp, def:GaussFlux.logicalOp, thm:GaussFlux.gaussLaw_product}
The product of all Gauss operators on the Shor graph equals the logical operator:
\[
L' = \prod_{v \in V \oplus D} A_v.
\]
This is an instance of the general $\operatorname{gaussLaw\_product}$ applied to the Shor graph.
\end{theorem}

\begin{proof}
\leanok
\uses{def:ShorStyleMeasurement.ShorGraph, def:GaussFlux.logicalOp, def:GaussFlux.gaussLawOp, thm:GaussFlux.gaussLaw_product}
This follows directly by symmetry from $\operatorname{gaussLaw\_product}(\operatorname{ShorGraph}(W, G_{\mathrm{dummy}}))$.
\end{proof}

\section{Circuit Correspondence}

\begin{theorem}[Circuit Transforms Gauss to Pauli X]
\label{thm:ShorStyleMeasurement.shor_circuit_transforms_gaussLaw}
\lean{QEC1.ShorStyleMeasurement.shor_circuit_transforms_gaussLaw}
\leanok
\uses{def:ShorStyleMeasurement.ShorGraph, def:GaussFlux.gaussLawOp, def:PauliOp.pauliX, def:GaussFlux.ExtQubit}
The entangling circuit transforms $A_v$ to $X_v$ on the Shor graph: for each vertex $v$,
\[
\operatorname{entanglingCircuitAction}(A_v) = X_{\operatorname{inl}(v)}.
\]
\end{theorem}

\begin{proof}
\leanok
\uses{def:ShorStyleMeasurement.ShorGraph, def:GaussFlux.gaussLawOp, def:PauliOp.pauliX}
This is a direct application of the general $\operatorname{entanglingCircuit\_transforms\_gaussLaw}$ theorem to the Shor graph.
\end{proof}

\begin{theorem}[Circuit Transforms Pauli X to Gauss]
\label{thm:ShorStyleMeasurement.shor_circuit_transforms_pauliX}
\lean{QEC1.ShorStyleMeasurement.shor_circuit_transforms_pauliX}
\leanok
\uses{def:ShorStyleMeasurement.ShorGraph, def:GaussFlux.gaussLawOp, def:PauliOp.pauliX, def:GaussFlux.ExtQubit}
The inverse direction: the CX circuit transforms $X_v$ back to $A_v$.
\end{theorem}

\begin{proof}
\leanok
\uses{def:ShorStyleMeasurement.ShorGraph, def:GaussFlux.gaussLawOp, def:PauliOp.pauliX}
This is a direct application of $\operatorname{entanglingCircuit\_transforms\_pauliX\_to\_gaussLaw}$ to the Shor graph.
\end{proof}

\begin{theorem}[Circuit is Involutive]
\label{thm:ShorStyleMeasurement.shor_circuit_involutive}
\lean{QEC1.ShorStyleMeasurement.shor_circuit_involutive}
\leanok
\uses{def:ShorStyleMeasurement.ShorGraph, def:GaussFlux.ExtQubit, def:PauliOp}
Applying the entangling circuit twice gives the identity: for any Pauli operator $P$ on the extended qubit type,
\[
\operatorname{entanglingCircuitAction}(\operatorname{entanglingCircuitAction}(P)) = P.
\]
\end{theorem}

\begin{proof}
\leanok
\uses{def:ShorStyleMeasurement.ShorGraph, def:PauliOp}
This is a direct application of $\operatorname{entanglingCircuitAction\_involutive}$ to the Shor graph.
\end{proof}

\begin{theorem}[Circuit Preserves Commutation]
\label{thm:ShorStyleMeasurement.shor_circuit_preserves_commutation}
\lean{QEC1.ShorStyleMeasurement.shor_circuit_preserves_commutation}
\leanok
\uses{def:ShorStyleMeasurement.ShorGraph, def:PauliOp.PauliCommute, def:PauliOp, def:GaussFlux.ExtQubit}
The circuit preserves Pauli commutation: $[P, Q] = 0$ if and only if $[\operatorname{entanglingCircuitAction}(P), \operatorname{entanglingCircuitAction}(Q)] = 0$.
\end{theorem}

\begin{proof}
\leanok
\uses{def:ShorStyleMeasurement.ShorGraph, def:PauliOp.PauliCommute}
This is the symmetric form of $\operatorname{entanglingCircuit\_preserves\_commutation}$ applied to the Shor graph.
\end{proof}

\section{Edge Structure}

\begin{theorem}[No Support-Support Edges]
\label{thm:ShorStyleMeasurement.shor_no_support_support_edges}
\lean{QEC1.ShorStyleMeasurement.shor_no_support_support_edges}
\leanok
\uses{def:ShorStyleMeasurement.ShorGraph}
No two support qubits are adjacent in the Shor graph: for all $i, j \in \operatorname{Fin}(W)$,
\[
\neg\, (\operatorname{ShorGraph}).\operatorname{Adj}(\operatorname{inl}(i), \operatorname{inl}(j)).
\]
\end{theorem}

\begin{proof}
\leanok
\uses{def:ShorStyleMeasurement.ShorGraph, def:ShorStyleMeasurement.shorGraphAdj}
Suppose $\operatorname{inl}(i) \sim \operatorname{inl}(j)$. Decomposing the adjacency disjunction: a pairing edge would require $\operatorname{inl}(j) = \operatorname{inr}(k)$, which contradicts the type; a reverse pairing edge requires $\operatorname{inl}(i) = \operatorname{inr}(k)$, another contradiction; a dummy subgraph edge requires $\operatorname{inl}(i) = \operatorname{inr}(a)$, again a contradiction. All cases are impossible.
\end{proof}

\begin{theorem}[Support-Dummy Adjacency Characterization]
\label{thm:ShorStyleMeasurement.shor_support_dummy_iff}
\lean{QEC1.ShorStyleMeasurement.shor_support_dummy_iff}
\leanok
\uses{def:ShorStyleMeasurement.ShorGraph, thm:ShorStyleMeasurement.shor_pairing_edge}
$\operatorname{inl}(i) \sim \operatorname{inr}(j)$ in the Shor graph if and only if $i = j$.
\end{theorem}

\begin{proof}
\leanok
\uses{def:ShorStyleMeasurement.ShorGraph, thm:ShorStyleMeasurement.shor_pairing_edge}
($\Rightarrow$) If $\operatorname{inl}(i) \sim \operatorname{inr}(j)$, then among the three disjuncts, only the pairing edge case is consistent with the types: there exists $k$ with $\operatorname{inl}(i) = \operatorname{inl}(k)$ and $\operatorname{inr}(j) = \operatorname{inr}(k)$, giving $i = k = j$.

($\Leftarrow$) If $i = j$, substitute and apply $\operatorname{shor\_pairing\_edge}$.
\end{proof}

\begin{theorem}[Dummy-Dummy Adjacency Characterization]
\label{thm:ShorStyleMeasurement.shor_dummy_dummy_iff}
\lean{QEC1.ShorStyleMeasurement.shor_dummy_dummy_iff}
\leanok
\uses{def:ShorStyleMeasurement.ShorGraph, thm:ShorStyleMeasurement.shor_dummy_edge}
$\operatorname{inr}(i) \sim \operatorname{inr}(j)$ in the Shor graph if and only if $i \sim j$ in $G_{\mathrm{dummy}}$.
\end{theorem}

\begin{proof}
\leanok
\uses{def:ShorStyleMeasurement.ShorGraph, thm:ShorStyleMeasurement.shor_dummy_edge}
($\Rightarrow$) If $\operatorname{inr}(i) \sim \operatorname{inr}(j)$, then among the disjuncts, only the dummy subgraph case is type-consistent: there exist $a, b$ with $\operatorname{inr}(i) = \operatorname{inr}(a)$, $\operatorname{inr}(j) = \operatorname{inr}(b)$, and $G_{\mathrm{dummy}}.\operatorname{Adj}(a, b)$. By injectivity, $i = a$ and $j = b$, so $G_{\mathrm{dummy}}.\operatorname{Adj}(i, j)$.

($\Leftarrow$) Apply $\operatorname{shor\_dummy\_edge}$.
\end{proof}

\begin{theorem}[Support Vertex Neighbor Set]
\label{thm:ShorStyleMeasurement.shor_support_degree}
\lean{QEC1.ShorStyleMeasurement.shor_support_degree}
\leanok
\uses{def:ShorStyleMeasurement.ShorGraph, thm:ShorStyleMeasurement.shor_no_support_support_edges, thm:ShorStyleMeasurement.shor_support_dummy_iff, thm:ShorStyleMeasurement.shor_pairing_edge}
The neighbor set of $\operatorname{inl}(i)$ in the Shor graph is exactly $\{\operatorname{inr}(i)\}$.
\end{theorem}

\begin{proof}
\leanok
\uses{def:ShorStyleMeasurement.ShorGraph, thm:ShorStyleMeasurement.shor_no_support_support_edges, thm:ShorStyleMeasurement.shor_support_dummy_iff, thm:ShorStyleMeasurement.shor_pairing_edge}
By extensionality, a vertex $q$ is in the neighbor set if and only if $q = \operatorname{inr}(i)$. For $q = \operatorname{inl}(j)$, adjacency is impossible by the no-support-support-edges theorem. For $q = \operatorname{inr}(j)$, adjacency holds if and only if $i = j$ by the support-dummy characterization, so $q = \operatorname{inr}(i)$.
\end{proof}

\begin{theorem}[Support Vertex Degree is One]
\label{thm:ShorStyleMeasurement.shor_support_degree_eq_one}
\lean{QEC1.ShorStyleMeasurement.shor_support_degree_eq_one}
\leanok
\uses{def:ShorStyleMeasurement.ShorGraph, thm:ShorStyleMeasurement.shor_support_degree}
Each support vertex has degree exactly $1$:
\[
\deg_{\operatorname{ShorGraph}}(\operatorname{inl}(i)) = 1.
\]
\end{theorem}

\begin{proof}
\leanok
\uses{thm:ShorStyleMeasurement.shor_support_degree}
By the neighbor set characterization, the neighbor set is $\{\operatorname{inr}(i)\}$, which has cardinality $1$.
\end{proof}

\begin{theorem}[Dummy Vertex Neighbor Set]
\label{thm:ShorStyleMeasurement.shor_dummy_neighborFinset}
\lean{QEC1.ShorStyleMeasurement.shor_dummy_neighborFinset}
\leanok
\uses{def:ShorStyleMeasurement.ShorGraph, thm:ShorStyleMeasurement.shor_support_dummy_iff, thm:ShorStyleMeasurement.shor_dummy_dummy_iff, thm:ShorStyleMeasurement.shor_pairing_edge}
The neighbor set of $\operatorname{inr}(i)$ in the Shor graph is
\[
N(\operatorname{inr}(i)) = \{\operatorname{inl}(i)\} \cup \{\operatorname{inr}(j) : j \in N_{G_{\mathrm{dummy}}}(i)\}.
\]
\end{theorem}

\begin{proof}
\leanok
\uses{def:ShorStyleMeasurement.ShorGraph, thm:ShorStyleMeasurement.shor_support_dummy_iff, thm:ShorStyleMeasurement.shor_dummy_dummy_iff, thm:ShorStyleMeasurement.shor_pairing_edge}
By extensionality. For a vertex $q$: if $q = \operatorname{inl}(j)$, then $\operatorname{inr}(i) \sim \operatorname{inl}(j)$ iff $j = i$ (by symmetry and the support-dummy characterization), corresponding to membership in $\{\operatorname{inl}(i)\}$. If $q = \operatorname{inr}(j)$, then $\operatorname{inr}(i) \sim \operatorname{inr}(j)$ iff $G_{\mathrm{dummy}}.\operatorname{Adj}(i, j)$ (by the dummy-dummy characterization), corresponding to $j \in N_{G_{\mathrm{dummy}}}(i)$. The disjointness of the two parts follows because $\operatorname{inl}(i) \neq \operatorname{inr}(j)$ for all $j$.
\end{proof}

\begin{theorem}[Edge Trichotomy]
\label{thm:ShorStyleMeasurement.shor_edge_trichotomy}
\lean{QEC1.ShorStyleMeasurement.shor_edge_trichotomy}
\leanok
\uses{def:ShorStyleMeasurement.ShorGraph}
Every edge of the Shor graph is either a pairing edge or a dummy subgraph edge: if $p \sim q$, then either there exists $i$ with $(p, q)$ or $(q, p) = (\operatorname{inl}(i), \operatorname{inr}(i))$, or there exist $i, j$ with $p = \operatorname{inr}(i)$, $q = \operatorname{inr}(j)$, and $G_{\mathrm{dummy}}.\operatorname{Adj}(i, j)$.
\end{theorem}

\begin{proof}
\leanok
\uses{def:ShorStyleMeasurement.ShorGraph, def:ShorStyleMeasurement.shorGraphAdj}
This follows directly by case analysis on the adjacency disjunction from the definition of $\operatorname{shorGraphAdj}$.
\end{proof}

\section{Structural Correspondence}

\begin{theorem}[Gauss Operators Commute on Shor Graph]
\label{thm:ShorStyleMeasurement.shor_gauss_operators_commute}
\lean{QEC1.ShorStyleMeasurement.shor_gauss_operators_commute}
\leanok
\uses{def:ShorStyleMeasurement.ShorGraph, def:GaussFlux.gaussLawOp, def:PauliOp.PauliCommute, thm:GaussFlux.gaussLaw_commute}
All Gauss operators on the Shor graph mutually commute:
\[
[A_v, A_w] = 0 \quad \text{for all } v, w.
\]
\end{theorem}

\begin{proof}
\leanok
\uses{def:ShorStyleMeasurement.ShorGraph, def:GaussFlux.gaussLawOp, thm:GaussFlux.gaussLaw_commute}
This is a direct application of $\operatorname{gaussLaw\_commute}$ to the Shor graph.
\end{proof}

\begin{theorem}[Gauss Operators are Self-Inverse on Shor Graph]
\label{thm:ShorStyleMeasurement.shor_gauss_self_inverse}
\lean{QEC1.ShorStyleMeasurement.shor_gauss_self_inverse}
\leanok
\uses{def:ShorStyleMeasurement.ShorGraph, def:GaussFlux.gaussLawOp, def:PauliOp.mul}
Each Gauss operator on the Shor graph satisfies $A_v \cdot A_v = \mathbf{1}$.
\end{theorem}

\begin{proof}
\leanok
\uses{def:ShorStyleMeasurement.ShorGraph, def:GaussFlux.gaussLawOp}
This follows from the general self-inverse property of Pauli operators.
\end{proof}

\begin{theorem}[Product Measurement]
\label{thm:ShorStyleMeasurement.shor_product_measurement}
\lean{QEC1.ShorStyleMeasurement.shor_product_measurement}
\leanok
\uses{def:ShorStyleMeasurement.ShorGraph, def:GaussFlux.gaussLawOp, def:GaussFlux.logicalOp, thm:GaussFlux.gaussLaw_product}
The product of all Gauss operators equals the logical operator:
\[
\prod_{v} A_v = L'.
\]
This is the fundamental equivalence that makes the gauging procedure work.
\end{theorem}

\begin{proof}
\leanok
\uses{def:ShorStyleMeasurement.ShorGraph, def:GaussFlux.gaussLawOp, def:GaussFlux.logicalOp, thm:GaussFlux.gaussLaw_product}
This is a direct application of $\operatorname{gaussLaw\_product}$ to the Shor graph.
\end{proof}

\begin{theorem}[Structural Correspondence Summary]
\label{thm:ShorStyleMeasurement.shor_style_structural_correspondence}
\lean{QEC1.ShorStyleMeasurement.shor_style_structural_correspondence}
\leanok
\uses{def:ShorStyleMeasurement.ShorGraph, def:ShorStyleMeasurement.ShorVertex, def:GaussFlux.gaussLawOp, def:GaussFlux.logicalOp, def:PauliOp.PauliCommute, thm:ShorStyleMeasurement.shorGraph_connected, thm:ShorStyleMeasurement.shorGraph_card, thm:ShorStyleMeasurement.shor_gauss_operators_commute, thm:ShorStyleMeasurement.shor_gauss_self_inverse, thm:ShorStyleMeasurement.shor_product_measurement, thm:ShorStyleMeasurement.shor_support_degree_eq_one, thm:GaussFlux.gaussLaw_commute, thm:GaussFlux.gaussLaw_product}
Assuming $W > 0$ and $G_{\mathrm{dummy}}$ is connected, the Shor-style gauging graph satisfies all of the following simultaneously:
\begin{enumerate}
\item The Shor graph is connected.
\item $|\operatorname{ShorVertex}(W)| = 2W$.
\item All Gauss operators mutually commute.
\item Each Gauss operator is self-inverse.
\item The product of all Gauss operators equals the logical $L'$.
\item Each support vertex has degree exactly $1$.
\end{enumerate}
\end{theorem}

\begin{proof}
\leanok
\uses{thm:ShorStyleMeasurement.shorGraph_connected, thm:ShorStyleMeasurement.shorGraph_card, thm:ShorStyleMeasurement.shor_gauss_operators_commute, thm:ShorStyleMeasurement.shor_gauss_self_inverse, thm:ShorStyleMeasurement.shor_product_measurement, thm:ShorStyleMeasurement.shor_support_degree_eq_one}
This is a conjunction of previously proved results: connectivity from $\operatorname{shorGraph\_connected}$, vertex count from $\operatorname{shorGraph\_card}$, commutation from $\operatorname{shor\_gauss\_operators\_commute}$, self-inverseness from $\operatorname{shor\_gauss\_self\_inverse}$, the product formula from $\operatorname{shor\_product\_measurement}$, and degree from $\operatorname{shor\_support\_degree\_eq\_one}$.
\end{proof}

\section{Degree Analysis}

\begin{theorem}[Dummy Vertex Degree]
\label{thm:ShorStyleMeasurement.shor_dummy_degree_eq}
\lean{QEC1.ShorStyleMeasurement.shor_dummy_degree_eq}
\leanok
\uses{def:ShorStyleMeasurement.ShorGraph, thm:ShorStyleMeasurement.shor_dummy_neighborFinset}
The degree of dummy vertex $d_i$ in the Shor graph is $1 + \deg_{G_{\mathrm{dummy}}}(i)$:
\[
\deg_{\operatorname{ShorGraph}}(\operatorname{inr}(i)) = 1 + \deg_{G_{\mathrm{dummy}}}(i).
\]
\end{theorem}

\begin{proof}
\leanok
\uses{thm:ShorStyleMeasurement.shor_dummy_neighborFinset}
By the neighbor set characterization, the neighbor set of $\operatorname{inr}(i)$ is $\{\operatorname{inl}(i)\} \cup \operatorname{inr}(N_{G_{\mathrm{dummy}}}(i))$. These two sets are disjoint (different summands), so the cardinality is $1 + |N_{G_{\mathrm{dummy}}}(i)| = 1 + \deg_{G_{\mathrm{dummy}}}(i)$.
\end{proof}

\begin{theorem}[Support Degree Sum]
\label{thm:ShorStyleMeasurement.shor_support_degree_sum}
\lean{QEC1.ShorStyleMeasurement.shor_support_degree_sum}
\leanok
\uses{def:ShorStyleMeasurement.ShorGraph, thm:ShorStyleMeasurement.shor_support_degree_eq_one}
The sum of degrees of all support vertices is $W$:
\[
\sum_{i \in \operatorname{Fin}(W)} \deg(\operatorname{inl}(i)) = W.
\]
\end{theorem}

\begin{proof}
\leanok
\uses{thm:ShorStyleMeasurement.shor_support_degree_eq_one}
Since each support vertex has degree $1$, the sum is $\sum_{i} 1 = W$.
\end{proof}

\begin{theorem}[Dummy Degree Sum]
\label{thm:ShorStyleMeasurement.shor_dummy_degree_sum}
\lean{QEC1.ShorStyleMeasurement.shor_dummy_degree_sum}
\leanok
\uses{def:ShorStyleMeasurement.ShorGraph, thm:ShorStyleMeasurement.shor_dummy_degree_eq}
The sum of degrees of all dummy vertices is $W + \sum_i \deg_{G_{\mathrm{dummy}}}(i)$:
\[
\sum_{i \in \operatorname{Fin}(W)} \deg(\operatorname{inr}(i)) = W + \sum_{i \in \operatorname{Fin}(W)} \deg_{G_{\mathrm{dummy}}}(i).
\]
\end{theorem}

\begin{proof}
\leanok
\uses{thm:ShorStyleMeasurement.shor_dummy_degree_eq}
Substituting $\deg(\operatorname{inr}(i)) = 1 + \deg_{G_{\mathrm{dummy}}}(i)$ and distributing the sum: $\sum_i (1 + \deg_{G_{\mathrm{dummy}}}(i)) = \sum_i 1 + \sum_i \deg_{G_{\mathrm{dummy}}}(i) = W + \sum_i \deg_{G_{\mathrm{dummy}}}(i)$.
\end{proof}

\begin{theorem}[Total Degree Sum]
\label{thm:ShorStyleMeasurement.shor_total_degree_sum}
\lean{QEC1.ShorStyleMeasurement.shor_total_degree_sum}
\leanok
\uses{def:ShorStyleMeasurement.ShorGraph}
The total degree sum of the Shor graph equals twice its edge count:
\[
\sum_v \deg(v) = 2|E(\operatorname{ShorGraph})|.
\]
\end{theorem}

\begin{proof}
\leanok
\uses{def:ShorStyleMeasurement.ShorGraph}
This is the handshaking lemma applied to the Shor graph.
\end{proof}

\begin{theorem}[Degree Sum Decomposition]
\label{thm:ShorStyleMeasurement.shor_degree_sum_decompose}
\lean{QEC1.ShorStyleMeasurement.shor_degree_sum_decompose}
\leanok
\uses{def:ShorStyleMeasurement.ShorGraph, def:ShorStyleMeasurement.ShorVertex}
The total degree sum decomposes into support and dummy contributions:
\[
\sum_v \deg(v) = \sum_{i} \deg(\operatorname{inl}(i)) + \sum_{i} \deg(\operatorname{inr}(i)).
\]
\end{theorem}

\begin{proof}
\leanok
\uses{def:ShorStyleMeasurement.ShorGraph, def:ShorStyleMeasurement.ShorVertex}
This follows from the decomposition of sums over a sum type: $\sum_{v \in A \oplus B} f(v) = \sum_{a \in A} f(\operatorname{inl}(a)) + \sum_{b \in B} f(\operatorname{inr}(b))$.
\end{proof}

\begin{theorem}[Connected Dummy Edge Bound]
\label{thm:ShorStyleMeasurement.shor_connected_dummy_edge_bound}
\lean{QEC1.ShorStyleMeasurement.shor_connected_dummy_edge_bound}
\leanok
\uses{def:ShorStyleMeasurement.ShorGraph}
If $G_{\mathrm{dummy}}$ is connected on $W$ vertices, then it has at least $W - 1$ edges:
\[
W \le |E(G_{\mathrm{dummy}})| + 1.
\]
\end{theorem}

\begin{proof}
\leanok
\uses{def:ShorStyleMeasurement.ShorGraph}
This is the standard graph-theoretic fact that a connected graph on $n$ vertices has at least $n-1$ edges.
\end{proof}

\section{Optimization Flexibility}

\begin{theorem}[Dummy Vertex Degree Bound]
\label{thm:ShorStyleMeasurement.shor_dummy_degree_bound}
\lean{QEC1.ShorStyleMeasurement.shor_dummy_degree_bound}
\leanok
\uses{def:ShorStyleMeasurement.ShorGraph, thm:ShorStyleMeasurement.shor_dummy_degree_eq}
If every dummy vertex has $\deg_{G_{\mathrm{dummy}}}(i) \le d$, then every dummy vertex has $\deg_{\operatorname{ShorGraph}}(\operatorname{inr}(i)) \le d + 1$.
\end{theorem}

\begin{proof}
\leanok
\uses{thm:ShorStyleMeasurement.shor_dummy_degree_eq}
By the dummy degree formula, $\deg(\operatorname{inr}(i)) = 1 + \deg_{G_{\mathrm{dummy}}}(i) \le 1 + d = d + 1$.
\end{proof}

\begin{theorem}[Max Degree Bound]
\label{thm:ShorStyleMeasurement.shor_max_degree_bound}
\lean{QEC1.ShorStyleMeasurement.shor_max_degree_bound}
\leanok
\uses{def:ShorStyleMeasurement.ShorGraph, thm:ShorStyleMeasurement.shor_support_degree_eq_one, thm:ShorStyleMeasurement.shor_dummy_degree_bound}
If every dummy vertex has $\deg_{G_{\mathrm{dummy}}}(i) \le d$, then every vertex $v$ of the Shor graph has $\deg(v) \le d + 1$.
\end{theorem}

\begin{proof}
\leanok
\uses{thm:ShorStyleMeasurement.shor_support_degree_eq_one, thm:ShorStyleMeasurement.shor_dummy_degree_bound}
We consider two cases. If $v = \operatorname{inl}(i)$, then $\deg(v) = 1 \le d + 1$ (since $d \ge 0$). If $v = \operatorname{inr}(i)$, the bound follows from $\operatorname{shor\_dummy\_degree\_bound}$.
\end{proof}

\begin{theorem}[Edge Count via Handshaking]
\label{thm:ShorStyleMeasurement.shor_edge_count_via_handshaking}
\lean{QEC1.ShorStyleMeasurement.shor_edge_count_via_handshaking}
\leanok
\uses{def:ShorStyleMeasurement.ShorGraph, thm:ShorStyleMeasurement.shor_total_degree_sum, thm:ShorStyleMeasurement.shor_degree_sum_decompose, thm:ShorStyleMeasurement.shor_support_degree_sum, thm:ShorStyleMeasurement.shor_dummy_degree_sum}
The Shor graph edge count satisfies the handshaking identity:
\[
2|E(\operatorname{ShorGraph})| = W + \Bigl(W + \sum_{i} \deg_{G_{\mathrm{dummy}}}(i)\Bigr).
\]
\end{theorem}

\begin{proof}
\leanok
\uses{thm:ShorStyleMeasurement.shor_total_degree_sum, thm:ShorStyleMeasurement.shor_degree_sum_decompose, thm:ShorStyleMeasurement.shor_support_degree_sum, thm:ShorStyleMeasurement.shor_dummy_degree_sum}
Rewriting: $2|E| = \sum_v \deg(v)$ by the handshaking lemma. Then decompose the sum as $\sum_{\operatorname{inl}} + \sum_{\operatorname{inr}} = W + (W + \sum_i \deg_{G_{\mathrm{dummy}}}(i))$ using the previously established support and dummy degree sums.
\end{proof}

\begin{theorem}[Dummy Handshaking]
\label{thm:ShorStyleMeasurement.shor_dummy_handshaking}
\lean{QEC1.ShorStyleMeasurement.shor_dummy_handshaking}
\leanok
\uses{def:ShorStyleMeasurement.ShorGraph}
The handshaking lemma applied to the dummy subgraph:
\[
\sum_{i \in \operatorname{Fin}(W)} \deg_{G_{\mathrm{dummy}}}(i) = 2|E(G_{\mathrm{dummy}})|.
\]
\end{theorem}

\begin{proof}
\leanok
\uses{def:ShorStyleMeasurement.ShorGraph}
This is the standard handshaking lemma for $G_{\mathrm{dummy}}$.
\end{proof}

%--- Rem_25: SteaneStyleMeasurement ---
\chapter{Rem 25: Steane-Style Measurement as Gauging}

This remark shows that the Steane-style measurement procedure for stabilizer codes can be decomposed into a composition of gauging operations. For a CSS ancilla code, state preparation uses hypergraph gauging on Z-type check incidence, entangling uses graph gauging on a perfect matching, and readout uses Z measurements (ungauging).

\section{CSS Stabilizer Codes}

\begin{definition}[CSS Code]
\label{def:SteaneStyleMeasurement.IsCSS}
\lean{QEC1.SteaneStyleMeasurement.IsCSS}
\leanok
\uses{def:StabilizerCode, def:PauliOp}
A stabilizer code $C$ is \emph{CSS} (Calderbank--Shor--Steane) if every check is either purely $X$-type or purely $Z$-type. Formally, for every check index $i$,
\[
(\mathtt{check}(i)).\mathtt{zVec} = 0 \quad \lor \quad (\mathtt{check}(i)).\mathtt{xVec} = 0.
\]
\end{definition}

\begin{definition}[X-Check Indices]
\label{def:SteaneStyleMeasurement.xCheckIndices}
\lean{QEC1.SteaneStyleMeasurement.xCheckIndices}
\leanok
\uses{def:StabilizerCode, def:SteaneStyleMeasurement.IsCSS}
The \emph{X-type check indices} of a stabilizer code $C$ are those indices $i$ such that $(\mathtt{check}(i)).\mathtt{zVec} = 0$ and $(\mathtt{check}(i)).\mathtt{xVec} \neq 0$.
\end{definition}

\begin{definition}[Z-Check Indices]
\label{def:SteaneStyleMeasurement.zCheckIndices}
\lean{QEC1.SteaneStyleMeasurement.zCheckIndices}
\leanok
\uses{def:StabilizerCode, def:SteaneStyleMeasurement.IsCSS}
The \emph{Z-type check indices} of a stabilizer code $C$ are those indices $i$ such that $(\mathtt{check}(i)).\mathtt{xVec} = 0$ and $(\mathtt{check}(i)).\mathtt{zVec} \neq 0$.
\end{definition}

\begin{theorem}[X-Check is Pure X]
\label{thm:SteaneStyleMeasurement.xCheck_is_pureX}
\lean{QEC1.SteaneStyleMeasurement.xCheck_is_pureX}
\leanok
\uses{def:StabilizerCode, def:SteaneStyleMeasurement.xCheckIndices}
If $i \in \operatorname{xCheckIndices}(C)$, then $(\mathtt{check}(i)).\mathtt{zVec} = 0$.
\end{theorem}

\begin{proof}
\leanok
\uses{def:SteaneStyleMeasurement.xCheckIndices}
By the definition of $\operatorname{xCheckIndices}$, membership requires $(\mathtt{check}(i)).\mathtt{zVec} = 0$. Simplifying the filter condition from the membership hypothesis yields the result directly.
\end{proof}

\begin{theorem}[Z-Check is Pure Z]
\label{thm:SteaneStyleMeasurement.zCheck_is_pureZ}
\lean{QEC1.SteaneStyleMeasurement.zCheck_is_pureZ}
\leanok
\uses{def:StabilizerCode, def:SteaneStyleMeasurement.zCheckIndices}
If $i \in \operatorname{zCheckIndices}(C)$, then $(\mathtt{check}(i)).\mathtt{xVec} = 0$.
\end{theorem}

\begin{proof}
\leanok
\uses{def:SteaneStyleMeasurement.zCheckIndices}
By the definition of $\operatorname{zCheckIndices}$, membership requires $(\mathtt{check}(i)).\mathtt{xVec} = 0$. Simplifying the filter condition yields the result.
\end{proof}

\begin{theorem}[X-Checks Commute]
\label{thm:SteaneStyleMeasurement.xChecks_commute}
\lean{QEC1.SteaneStyleMeasurement.xChecks_commute}
\leanok
\uses{def:StabilizerCode, def:PauliOp.PauliCommute, def:SteaneStyleMeasurement.xCheckIndices}
For a stabilizer code $C$ and indices $i, j \in \operatorname{xCheckIndices}(C)$, the checks $\mathtt{check}(i)$ and $\mathtt{check}(j)$ commute.
\end{theorem}

\begin{proof}
\leanok
\uses{def:StabilizerCode}
This follows directly from the fact that all checks of a stabilizer code pairwise commute.
\end{proof}

\begin{theorem}[Z-Checks Commute]
\label{thm:SteaneStyleMeasurement.zChecks_commute}
\lean{QEC1.SteaneStyleMeasurement.zChecks_commute}
\leanok
\uses{def:StabilizerCode, def:PauliOp.PauliCommute, def:SteaneStyleMeasurement.zCheckIndices}
For a stabilizer code $C$ and indices $i, j \in \operatorname{zCheckIndices}(C)$, the checks $\mathtt{check}(i)$ and $\mathtt{check}(j)$ commute.
\end{theorem}

\begin{proof}
\leanok
\uses{def:StabilizerCode}
This follows directly from the pairwise commutation of stabilizer code checks.
\end{proof}

\begin{theorem}[Pure X Operators Commute]
\label{thm:SteaneStyleMeasurement.pureX_pureX_commute}
\lean{QEC1.SteaneStyleMeasurement.pureX_pureX_commute}
\leanok
\uses{def:PauliOp, def:PauliOp.PauliCommute, def:PauliOp.symplecticInner}
If $P$ and $R$ are Pauli operators with $P.\mathtt{zVec} = 0$ and $R.\mathtt{zVec} = 0$, then $P$ and $R$ commute.
\end{theorem}

\begin{proof}
\leanok
\uses{def:PauliOp.PauliCommute, def:PauliOp.symplecticInner}
We expand the definition of $\operatorname{PauliCommute}$ as the symplectic inner product. The symplectic inner product $\langle P, R \rangle = \sum_v (P.\mathtt{xVec}(v) \cdot R.\mathtt{zVec}(v) + P.\mathtt{zVec}(v) \cdot R.\mathtt{xVec}(v))$. Since $P.\mathtt{zVec} = 0$ and $R.\mathtt{zVec} = 0$, each term in the sum vanishes, so the sum equals zero.
\end{proof}

\begin{theorem}[Pure Z Operators Commute]
\label{thm:SteaneStyleMeasurement.pureZ_pureZ_commute}
\lean{QEC1.SteaneStyleMeasurement.pureZ_pureZ_commute}
\leanok
\uses{def:PauliOp, def:PauliOp.PauliCommute, def:PauliOp.symplecticInner}
If $P$ and $R$ are Pauli operators with $P.\mathtt{xVec} = 0$ and $R.\mathtt{xVec} = 0$, then $P$ and $R$ commute.
\end{theorem}

\begin{proof}
\leanok
\uses{def:PauliOp.PauliCommute, def:PauliOp.symplecticInner}
We expand the symplectic inner product. Since $P.\mathtt{xVec} = 0$ and $R.\mathtt{xVec} = 0$, every term $P.\mathtt{xVec}(v) \cdot R.\mathtt{zVec}(v) + P.\mathtt{zVec}(v) \cdot R.\mathtt{xVec}(v)$ vanishes, so the total sum is zero.
\end{proof}

\begin{theorem}[Identity is CSS-Compatible]
\label{thm:SteaneStyleMeasurement.identity_is_css_compatible}
\lean{QEC1.SteaneStyleMeasurement.identity_is_css_compatible}
\leanok
\uses{def:PauliOp}
The identity operator $1 : \operatorname{PauliOp}(Q)$ satisfies $(1).\mathtt{zVec} = 0 \lor (1).\mathtt{xVec} = 0$.
\end{theorem}

\begin{proof}
\leanok
\uses{def:PauliOp}
We choose the left disjunct. By the definition of the identity Pauli operator, $(1).\mathtt{zVec} = 0$.
\end{proof}

\section{Step 1: State Preparation via Hypergraph Gauging}

\begin{definition}[Z-Check Incidence]
\label{def:SteaneStyleMeasurement.zCheckIncident}
\lean{QEC1.SteaneStyleMeasurement.zCheckIncident}
\leanok
\uses{def:PauliOp}
The incidence relation for state preparation: qubit $q$ is incident to Z-check $i$ if and only if the check has Z-action at $q$, i.e., $(\mathtt{checks}(i)).\mathtt{zVec}(q) \neq 0$.
\end{definition}

\begin{theorem}[Z-Check Boundary Equation]
\label{thm:SteaneStyleMeasurement.zCheck_boundary_eq}
\lean{QEC1.SteaneStyleMeasurement.zCheck_boundary_eq}
\leanok
\uses{def:SteaneStyleMeasurement.zCheckIncident, def:HypergraphGeneralization.hyperBoundaryMap}
The hypergraph boundary map for Z-check incidence computes the Z-parity at each qubit from a vector of check activations:
\[
(\partial \gamma)(q) = \sum_{i : I} \begin{cases} \gamma_i & \text{if } (\mathtt{checks}(i)).\mathtt{zVec}(q) \neq 0, \\ 0 & \text{otherwise.} \end{cases}
\]
\end{theorem}

\begin{proof}
\leanok
\uses{def:SteaneStyleMeasurement.zCheckIncident, def:HypergraphGeneralization.hyperBoundaryMap}
By simplification using the definitions of $\operatorname{hyperBoundaryMap}$ and $\operatorname{zCheckIncident}$.
\end{proof}

\begin{theorem}[Step 1 Gauss Operators are Pure X]
\label{thm:SteaneStyleMeasurement.step1_gauss_pure_X}
\lean{QEC1.SteaneStyleMeasurement.step1_gauss_pure_X}
\leanok
\uses{def:SteaneStyleMeasurement.zCheckIncident, def:HypergraphGeneralization.hyperGaussLawOp}
The Gauss's law operators for Z-check hypergraph gauging are all pure X-type: for every vertex $v$, $(\operatorname{hyperGaussLawOp}(v)).\mathtt{zVec} = 0$.
\end{theorem}

\begin{proof}
\leanok
\uses{def:HypergraphGeneralization.hyperGaussLawOp, def:SteaneStyleMeasurement.zCheckIncident}
This follows directly from the fact that hypergraph Gauss's law operators have zero Z-component.
\end{proof}

\begin{theorem}[Step 1 Gauss Operators Commute]
\label{thm:SteaneStyleMeasurement.step1_gauss_commute}
\lean{QEC1.SteaneStyleMeasurement.step1_gauss_commute}
\leanok
\uses{def:SteaneStyleMeasurement.zCheckIncident, def:HypergraphGeneralization.hyperGaussLawOp, def:PauliOp.PauliCommute, thm:HypergraphGeneralization.hyperGaussLaw_commute}
The Gauss's law operators for Z-check hypergraph gauging mutually commute: for all vertices $v, w$, $\operatorname{PauliCommute}(\operatorname{hyperGaussLawOp}(v), \operatorname{hyperGaussLawOp}(w))$.
\end{theorem}

\begin{proof}
\leanok
\uses{thm:HypergraphGeneralization.hyperGaussLaw_commute}
This follows directly from the general result that hypergraph Gauss's law operators commute.
\end{proof}

\begin{theorem}[Step 1: State Preparation via Hypergraph Gauging]
\label{thm:SteaneStyleMeasurement.step1_state_prep_hypergraph}
\lean{QEC1.SteaneStyleMeasurement.step1_state_prep_hypergraph}
\leanok
\uses{def:SteaneStyleMeasurement.zCheckIncident, def:HypergraphGeneralization.hyperGaussLawOp, def:PauliOp.PauliCommute, thm:HypergraphGeneralization.hyperGaussLaw_commute}
The hypergraph whose hyperedges correspond to the Z-type checks of the ancilla CSS code defines a valid hypergraph gauging. Specifically:
\begin{enumerate}
\item The Gauss operators $A_v$ are all pure X-type: $(\operatorname{hyperGaussLawOp}(v)).\mathtt{zVec} = 0$ for all $v$.
\item The Gauss operators mutually commute: $\operatorname{PauliCommute}(A_v, A_w)$ for all $v, w$.
\end{enumerate}
\end{theorem}

\begin{proof}
\leanok
\uses{thm:SteaneStyleMeasurement.step1_gauss_commute, thm:SteaneStyleMeasurement.step1_gauss_pure_X}
The result is the conjunction of the two properties established above: mutual commutation (from \texttt{step1\_gauss\_commute}) and pure X-type (from \texttt{step1\_gauss\_pure\_X}).
\end{proof}

\section{Step 2: Entangling via Perfect Matching Graph}

\begin{definition}[Matching Graph Adjacency]
\label{def:SteaneStyleMeasurement.matchingGraphAdj}
\lean{QEC1.SteaneStyleMeasurement.matchingGraphAdj}
\leanok
\uses{def:PauliOp}
The adjacency relation for the perfect matching graph on vertex set $\operatorname{Fin}(n) \oplus \operatorname{Fin}(n)$: data qubit $\operatorname{inl}(i)$ is adjacent to ancilla qubit $\operatorname{inr}(i)$ (and vice versa), with no other edges. Formally, $p \sim q$ iff there exists $i : \operatorname{Fin}(n)$ such that $(p = \operatorname{inl}(i) \land q = \operatorname{inr}(i))$ or $(p = \operatorname{inr}(i) \land q = \operatorname{inl}(i))$.
\end{definition}

\begin{theorem}[Matching Graph Adjacency is Symmetric]
\label{thm:SteaneStyleMeasurement.matchingGraphAdj_symm}
\lean{QEC1.SteaneStyleMeasurement.matchingGraphAdj_symm}
\leanok
\uses{def:SteaneStyleMeasurement.matchingGraphAdj}
The matching graph adjacency relation is symmetric.
\end{theorem}

\begin{proof}
\leanok
\uses{def:SteaneStyleMeasurement.matchingGraphAdj}
Given $\langle i, h \rangle$ witnessing $p \sim q$, we decompose $h$ into the two cases. In each case, swapping the two components of the conjunction gives the witness for $q \sim p$.
\end{proof}

\begin{theorem}[Matching Graph Adjacency is Irreflexive]
\label{thm:SteaneStyleMeasurement.matchingGraphAdj_irrefl}
\lean{QEC1.SteaneStyleMeasurement.matchingGraphAdj_irrefl}
\leanok
\uses{def:SteaneStyleMeasurement.matchingGraphAdj}
No vertex is adjacent to itself in the matching graph.
\end{theorem}

\begin{proof}
\leanok
\uses{def:SteaneStyleMeasurement.matchingGraphAdj}
Suppose $p \sim p$. Then there exists $i$ such that $p = \operatorname{inl}(i)$ and $p = \operatorname{inr}(i)$ (or vice versa), which contradicts $\operatorname{inl} \neq \operatorname{inr}$.
\end{proof}

\begin{definition}[Matching Graph]
\label{def:SteaneStyleMeasurement.MatchingGraph}
\lean{QEC1.SteaneStyleMeasurement.MatchingGraph}
\leanok
\uses{def:SteaneStyleMeasurement.matchingGraphAdj}
The perfect matching graph $G$ on $\operatorname{Fin}(n) \oplus \operatorname{Fin}(n)$, defined as the simple graph with adjacency relation $\operatorname{matchingGraphAdj}(n)$.
\end{definition}

\begin{theorem}[Matching Graph has $2n$ Vertices]
\label{thm:SteaneStyleMeasurement.matchingGraph_card}
\lean{QEC1.SteaneStyleMeasurement.matchingGraph_card}
\leanok
\uses{def:SteaneStyleMeasurement.MatchingGraph}
The matching graph has $|\operatorname{Fin}(n) \oplus \operatorname{Fin}(n)| = 2n$ vertices.
\end{theorem}

\begin{proof}
\leanok
\uses{def:SteaneStyleMeasurement.MatchingGraph}
The cardinality of a sum type is the sum of the cardinalities: $|\operatorname{Fin}(n)| + |\operatorname{Fin}(n)| = n + n = 2n$.
\end{proof}

\begin{theorem}[Matching Graph Adjacency Characterization]
\label{thm:SteaneStyleMeasurement.matchingGraph_adj_iff}
\lean{QEC1.SteaneStyleMeasurement.matchingGraph_adj_iff}
\leanok
\uses{def:SteaneStyleMeasurement.MatchingGraph, def:SteaneStyleMeasurement.matchingGraphAdj}
$p \sim q$ in the matching graph if and only if there exists $i : \operatorname{Fin}(n)$ such that $(p = \operatorname{inl}(i) \land q = \operatorname{inr}(i))$ or $(p = \operatorname{inr}(i) \land q = \operatorname{inl}(i))$.
\end{theorem}

\begin{proof}
\leanok
\uses{def:SteaneStyleMeasurement.MatchingGraph}
This holds by definitional equality (reflexivity).
\end{proof}

\begin{theorem}[Matching Edge]
\label{thm:SteaneStyleMeasurement.matching_edge}
\lean{QEC1.SteaneStyleMeasurement.matching_edge}
\leanok
\uses{def:SteaneStyleMeasurement.MatchingGraph}
For each $i : \operatorname{Fin}(n)$, $\operatorname{inl}(i) \sim \operatorname{inr}(i)$ in the matching graph.
\end{theorem}

\begin{proof}
\leanok
\uses{def:SteaneStyleMeasurement.MatchingGraph}
We provide the witness $\langle i, \operatorname{Or.inl}\langle \operatorname{rfl}, \operatorname{rfl}\rangle\rangle$.
\end{proof}

\begin{theorem}[Matching Edge Reverse]
\label{thm:SteaneStyleMeasurement.matching_edge_rev}
\lean{QEC1.SteaneStyleMeasurement.matching_edge_rev}
\leanok
\uses{def:SteaneStyleMeasurement.MatchingGraph}
For each $i : \operatorname{Fin}(n)$, $\operatorname{inr}(i) \sim \operatorname{inl}(i)$ in the matching graph.
\end{theorem}

\begin{proof}
\leanok
\uses{def:SteaneStyleMeasurement.MatchingGraph}
We provide the witness $\langle i, \operatorname{Or.inr}\langle \operatorname{rfl}, \operatorname{rfl}\rangle\rangle$.
\end{proof}

\begin{theorem}[No Data--Data Edges]
\label{thm:SteaneStyleMeasurement.no_data_data_edge}
\lean{QEC1.SteaneStyleMeasurement.no_data_data_edge}
\leanok
\uses{def:SteaneStyleMeasurement.MatchingGraph}
No two data qubits are adjacent: $\neg(\operatorname{inl}(i) \sim \operatorname{inl}(j))$ for all $i, j$.
\end{theorem}

\begin{proof}
\leanok
\uses{def:SteaneStyleMeasurement.MatchingGraph}
Suppose $\operatorname{inl}(i) \sim \operatorname{inl}(j)$. Then there exists $k$ with $\operatorname{inl}(i) = \operatorname{inl}(k)$ and $\operatorname{inl}(j) = \operatorname{inr}(k)$ (or vice versa), but $\operatorname{inl}(j) = \operatorname{inr}(k)$ contradicts $\operatorname{inl} \neq \operatorname{inr}$.
\end{proof}

\begin{theorem}[No Ancilla--Ancilla Edges]
\label{thm:SteaneStyleMeasurement.no_ancilla_ancilla_edge}
\lean{QEC1.SteaneStyleMeasurement.no_ancilla_ancilla_edge}
\leanok
\uses{def:SteaneStyleMeasurement.MatchingGraph}
No two ancilla qubits are adjacent: $\neg(\operatorname{inr}(i) \sim \operatorname{inr}(j))$ for all $i, j$.
\end{theorem}

\begin{proof}
\leanok
\uses{def:SteaneStyleMeasurement.MatchingGraph}
Suppose $\operatorname{inr}(i) \sim \operatorname{inr}(j)$. Then there exists $k$ with $\operatorname{inr}(i) = \operatorname{inr}(k)$ and $\operatorname{inr}(j) = \operatorname{inl}(k)$ (or a symmetric case), but $\operatorname{inr} \neq \operatorname{inl}$ gives a contradiction.
\end{proof}

\begin{theorem}[Data Adjacency Characterization]
\label{thm:SteaneStyleMeasurement.data_adj_iff}
\lean{QEC1.SteaneStyleMeasurement.data_adj_iff}
\leanok
\uses{def:SteaneStyleMeasurement.MatchingGraph}
Data qubit $\operatorname{inl}(i)$ is adjacent to $q$ if and only if $q = \operatorname{inr}(i)$.
\end{theorem}

\begin{proof}
\leanok
\uses{def:SteaneStyleMeasurement.MatchingGraph, thm:SteaneStyleMeasurement.matching_edge}
For the forward direction, given $\langle k, h \rangle$ witnessing adjacency, in the first case $\operatorname{inl}(i) = \operatorname{inl}(k)$ gives $i = k$ by injectivity, so $q = \operatorname{inr}(k) = \operatorname{inr}(i)$. The second case gives $\operatorname{inl}(i) = \operatorname{inr}(k)$, contradicting $\operatorname{inl} \neq \operatorname{inr}$. For the reverse direction, if $q = \operatorname{inr}(i)$ then the result follows from the matching edge lemma.
\end{proof}

\begin{theorem}[Ancilla Adjacency Characterization]
\label{thm:SteaneStyleMeasurement.ancilla_adj_iff}
\lean{QEC1.SteaneStyleMeasurement.ancilla_adj_iff}
\leanok
\uses{def:SteaneStyleMeasurement.MatchingGraph}
Ancilla qubit $\operatorname{inr}(i)$ is adjacent to $q$ if and only if $q = \operatorname{inl}(i)$.
\end{theorem}

\begin{proof}
\leanok
\uses{def:SteaneStyleMeasurement.MatchingGraph, thm:SteaneStyleMeasurement.matching_edge_rev}
Analogous to the data adjacency characterization, using the reverse matching edge.
\end{proof}

\begin{theorem}[Matching Pair Reachable]
\label{thm:SteaneStyleMeasurement.matching_pair_reachable}
\lean{QEC1.SteaneStyleMeasurement.matching_pair_reachable}
\leanok
\uses{def:SteaneStyleMeasurement.MatchingGraph}
For each $i : \operatorname{Fin}(n)$, the vertices $\operatorname{inl}(i)$ and $\operatorname{inr}(i)$ are reachable from each other in the matching graph.
\end{theorem}

\begin{proof}
\leanok
\uses{thm:SteaneStyleMeasurement.matching_edge}
Since $\operatorname{inl}(i) \sim \operatorname{inr}(i)$, adjacency implies reachability.
\end{proof}

\begin{theorem}[Matching Components]
\label{thm:SteaneStyleMeasurement.matching_components}
\lean{QEC1.SteaneStyleMeasurement.matching_components}
\leanok
\uses{def:SteaneStyleMeasurement.MatchingGraph}
The matching graph consists of $n$ connected components: if $p$ and $q$ are reachable, then there exists $i : \operatorname{Fin}(n)$ such that $p \in \{\operatorname{inl}(i), \operatorname{inr}(i)\}$ and $q \in \{\operatorname{inl}(i), \operatorname{inr}(i)\}$.
\end{theorem}

\begin{proof}
\leanok
\uses{def:SteaneStyleMeasurement.MatchingGraph}
From the reachability hypothesis, we obtain a walk from $p$ to $q$. We prove by induction on the walk that any walk stays within a single component $\{\operatorname{inl}(i), \operatorname{inr}(i)\}$. For the nil walk, $p = q$ and the component is determined by whether $p$ is $\operatorname{inl}(i)$ or $\operatorname{inr}(i)$. For a cons walk, by the inductive hypothesis the tail of the walk stays in some component $i$, and the adjacency step must connect to a vertex in the same component (since each vertex is only adjacent to its partner).
\end{proof}

\section{Step 2: Entangling Gauss Operators}

\begin{theorem}[Step 2 Gauss Operators are Pure X]
\label{thm:SteaneStyleMeasurement.step2_gauss_pure_X}
\lean{QEC1.SteaneStyleMeasurement.step2_gauss_pure_X}
\leanok
\uses{def:SteaneStyleMeasurement.MatchingGraph, def:GaussFlux.gaussLawOp}
The Gauss's law operators for the matching graph are all pure X-type: $(\operatorname{gaussLawOp}(v)).\mathtt{zVec} = 0$.
\end{theorem}

\begin{proof}
\leanok
\uses{def:GaussFlux.gaussLawOp, def:SteaneStyleMeasurement.MatchingGraph}
This follows from the general result that Gauss's law operators have zero Z-component.
\end{proof}

\begin{theorem}[Step 2 Gauss Operators Commute]
\label{thm:SteaneStyleMeasurement.step2_gauss_commute}
\lean{QEC1.SteaneStyleMeasurement.step2_gauss_commute}
\leanok
\uses{def:SteaneStyleMeasurement.MatchingGraph, def:GaussFlux.gaussLawOp, def:PauliOp.PauliCommute, thm:GaussFlux.gaussLaw_commute}
The Gauss's law operators for the matching graph all commute.
\end{theorem}

\begin{proof}
\leanok
\uses{thm:GaussFlux.gaussLaw_commute}
This is an instance of the general commutation result for Gauss's law operators.
\end{proof}

\begin{theorem}[Step 2 Gauss Product]
\label{thm:SteaneStyleMeasurement.step2_gauss_product}
\lean{QEC1.SteaneStyleMeasurement.step2_gauss_product}
\leanok
\uses{def:SteaneStyleMeasurement.MatchingGraph, def:GaussFlux.gaussLawOp, def:GaussFlux.logicalOp, thm:GaussFlux.gaussLaw_product}
The product of all Gauss operators on the matching graph equals the logical operator $L = \prod_v X_v$ on all vertex qubits:
\[
\prod_{v} A_v = L.
\]
\end{theorem}

\begin{proof}
\leanok
\uses{thm:GaussFlux.gaussLaw_product}
This is an instance of the general Gauss's law product theorem.
\end{proof}

\begin{theorem}[Step 2 Gauss Self-Inverse]
\label{thm:SteaneStyleMeasurement.step2_gauss_self_inverse}
\lean{QEC1.SteaneStyleMeasurement.step2_gauss_self_inverse}
\leanok
\uses{def:SteaneStyleMeasurement.MatchingGraph, def:GaussFlux.gaussLawOp}
Each Gauss operator on the matching graph is self-inverse: $A_v^2 = 1$.
\end{theorem}

\begin{proof}
\leanok
\uses{def:GaussFlux.gaussLawOp}
This follows from the general fact that every Pauli operator is self-inverse ($P \cdot P = 1$).
\end{proof}

\begin{theorem}[Step 2 Logical X-Vec]
\label{thm:SteaneStyleMeasurement.step2_logical_xVec}
\lean{QEC1.SteaneStyleMeasurement.step2_logical_xVec}
\leanok
\uses{def:SteaneStyleMeasurement.MatchingGraph, def:GaussFlux.logicalOp}
The logical operator on the matching graph acts as $X$ on all vertex qubits: $(L).\mathtt{xVec}(\operatorname{inl}(v)) = 1$ for all $v$.
\end{theorem}

\begin{proof}
\leanok
\uses{def:GaussFlux.logicalOp}
By simplification using the definition of the logical operator.
\end{proof}

\begin{theorem}[Step 2 Logical is Pure X]
\label{thm:SteaneStyleMeasurement.step2_logical_pure_X}
\lean{QEC1.SteaneStyleMeasurement.step2_logical_pure_X}
\leanok
\uses{def:SteaneStyleMeasurement.MatchingGraph, def:GaussFlux.logicalOp}
The logical operator on the matching graph is pure X-type: $(L).\mathtt{zVec} = 0$.
\end{theorem}

\begin{proof}
\leanok
\uses{def:GaussFlux.logicalOp}
By simplification using the definition of the logical operator.
\end{proof}

\section{Step 3: Readout via Z Measurements}

\begin{theorem}[Edge Z Operators are Pure Z]
\label{thm:SteaneStyleMeasurement.step3_edgeZ_pure_Z}
\lean{QEC1.SteaneStyleMeasurement.step3_edgeZ_pure_Z}
\leanok
\uses{def:SteaneStyleMeasurement.MatchingGraph, def:PauliOp.pauliZ, def:GaussFlux.ExtQubit}
Edge $Z$ operators on the matching graph have no $X$-support: $(Z_e).\mathtt{xVec} = 0$.
\end{theorem}

\begin{proof}
\leanok
\uses{def:PauliOp.pauliZ}
By extensionality and simplification using the definition of $\operatorname{pauliZ}$.
\end{proof}

\begin{theorem}[Edge Z Operators Commute]
\label{thm:SteaneStyleMeasurement.step3_edgeZ_commute}
\lean{QEC1.SteaneStyleMeasurement.step3_edgeZ_commute}
\leanok
\uses{def:SteaneStyleMeasurement.MatchingGraph, def:PauliOp.pauliZ, def:PauliOp.PauliCommute, def:GaussFlux.ExtQubit}
All edge $Z$ operators on the matching graph commute with each other.
\end{theorem}

\begin{proof}
\leanok
\uses{thm:SteaneStyleMeasurement.pureZ_pureZ_commute, def:PauliOp.pauliZ}
Both operators are pure Z-type (their $\mathtt{xVec}$ is zero), so they commute by the pure Z commutation lemma.
\end{proof}

\begin{theorem}[Edge Z Operators are Self-Inverse]
\label{thm:SteaneStyleMeasurement.step3_edgeZ_self_inverse}
\lean{QEC1.SteaneStyleMeasurement.step3_edgeZ_self_inverse}
\leanok
\uses{def:SteaneStyleMeasurement.MatchingGraph, def:PauliOp.pauliZ, def:GaussFlux.ExtQubit}
Each edge $Z$ operator satisfies $Z_e^2 = 1$.
\end{theorem}

\begin{proof}
\leanok
\uses{def:PauliOp.pauliZ}
This follows from the general self-inverse property of Pauli operators.
\end{proof}

\begin{theorem}[Step 3: Readout is Ungauging]
\label{thm:SteaneStyleMeasurement.step3_readout_is_ungauging}
\lean{QEC1.SteaneStyleMeasurement.step3_readout_is_ungauging}
\leanok
\uses{def:SteaneStyleMeasurement.MatchingGraph, def:PauliOp.pauliZ, def:PauliOp.PauliCommute, def:GaussFlux.ExtQubit}
The readout step consists of measuring $Z$ on each edge qubit. All $Z$ measurements commute, are self-inverse, and are pure Z-type, which is exactly the ungauging step.
\end{theorem}

\begin{proof}
\leanok
\uses{thm:SteaneStyleMeasurement.step3_edgeZ_commute, thm:SteaneStyleMeasurement.step3_edgeZ_self_inverse, thm:SteaneStyleMeasurement.step3_edgeZ_pure_Z}
The result is the conjunction of the three properties: commutation, self-inverse, and pure Z-type.
\end{proof}

\section{Composition: Steane-Style Measurement as Gauging}

\begin{theorem}[CX Circuit Transforms Gauss to Pauli X]
\label{thm:SteaneStyleMeasurement.step2_cx_transforms_gauss}
\lean{QEC1.SteaneStyleMeasurement.step2_cx_transforms_gauss}
\leanok
\uses{def:SteaneStyleMeasurement.MatchingGraph, def:CircuitImplementation.entanglingCircuitAction, def:GaussFlux.gaussLawOp, def:PauliOp.pauliX, thm:CircuitImplementation.entanglingCircuit_transforms_gaussLaw}
The CX circuit transformation maps $A_v$ to $X_v$ on the matching graph:
\[
\operatorname{entanglingCircuitAction}(A_v) = X_{\operatorname{inl}(v)}.
\]
\end{theorem}

\begin{proof}
\leanok
\uses{thm:CircuitImplementation.entanglingCircuit_transforms_gaussLaw}
This is an instance of the general result that the entangling circuit transforms Gauss's law operators to single-qubit $X$ operators.
\end{proof}

\begin{theorem}[CX Circuit Transforms Pauli X to Gauss]
\label{thm:SteaneStyleMeasurement.step2_cx_transforms_pauliX}
\lean{QEC1.SteaneStyleMeasurement.step2_cx_transforms_pauliX}
\leanok
\uses{def:SteaneStyleMeasurement.MatchingGraph, def:CircuitImplementation.entanglingCircuitAction, def:GaussFlux.gaussLawOp, def:PauliOp.pauliX, def:GaussFlux.ExtQubit, thm:CircuitImplementation.entanglingCircuit_transforms_pauliX_to_gaussLaw}
The CX circuit transforms $X_{\operatorname{inl}(v)}$ back to $A_v$:
\[
\operatorname{entanglingCircuitAction}(X_{\operatorname{inl}(v)}) = A_v.
\]
\end{theorem}

\begin{proof}
\leanok
\uses{thm:CircuitImplementation.entanglingCircuit_transforms_pauliX_to_gaussLaw}
This is an instance of the general result that the entangling circuit transforms single-qubit $X$ operators to Gauss's law operators.
\end{proof}

\begin{theorem}[CX Circuit is Involutive]
\label{thm:SteaneStyleMeasurement.step2_cx_involutive}
\lean{QEC1.SteaneStyleMeasurement.step2_cx_involutive}
\leanok
\uses{def:SteaneStyleMeasurement.MatchingGraph, def:CircuitImplementation.entanglingCircuitAction, thm:CircuitImplementation.entanglingCircuitAction_involutive}
The entangling circuit action is involutive on the matching graph: applying it twice returns the original operator.
\end{theorem}

\begin{proof}
\leanok
\uses{thm:CircuitImplementation.entanglingCircuitAction_involutive}
This is an instance of the general involutivity of the entangling circuit action.
\end{proof}

\begin{theorem}[CX Circuit Preserves Commutation]
\label{thm:SteaneStyleMeasurement.step2_cx_preserves_commutation}
\lean{QEC1.SteaneStyleMeasurement.step2_cx_preserves_commutation}
\leanok
\uses{def:SteaneStyleMeasurement.MatchingGraph, def:CircuitImplementation.entanglingCircuitAction, def:PauliOp.PauliCommute, thm:CircuitImplementation.entanglingCircuit_preserves_commutation}
The entangling circuit preserves commutation relations: $P$ and $R$ commute if and only if their images under the circuit action commute.
\end{theorem}

\begin{proof}
\leanok
\uses{thm:CircuitImplementation.entanglingCircuit_preserves_commutation}
This follows from the general result that the entangling circuit preserves symplectic inner products (and hence commutation), applied symmetrically.
\end{proof}

\begin{theorem}[Steane-Style Measurement is Composition of Gauging]
\label{thm:SteaneStyleMeasurement.steane_is_composition_of_gauging}
\lean{QEC1.SteaneStyleMeasurement.steane_is_composition_of_gauging}
\leanok
\uses{def:SteaneStyleMeasurement.MatchingGraph, def:SteaneStyleMeasurement.zCheckIncident, def:HypergraphGeneralization.hyperGaussLawOp, def:GaussFlux.gaussLawOp, def:CircuitImplementation.entanglingCircuitAction, def:PauliOp.pauliX, def:PauliOp.pauliZ, def:PauliOp.PauliCommute, def:GaussFlux.ExtQubit}
The full Steane-style measurement decomposes into three gauging operations:
\begin{enumerate}
\item \textbf{Step 1 (State preparation):} The hypergraph Gauss operators for Z-check incidence mutually commute and are all pure X-type.
\item \textbf{Step 2 (Entangling):} The Gauss operators for the matching graph mutually commute, are pure X-type, and the CX circuit transforms $A_v \mapsto X_{\operatorname{inl}(v)}$.
\item \textbf{Step 3 (Readout):} The edge $Z$ operators mutually commute and are self-inverse.
\end{enumerate}
\end{theorem}

\begin{proof}
\leanok
\uses{thm:SteaneStyleMeasurement.step1_gauss_commute, thm:SteaneStyleMeasurement.step1_gauss_pure_X, thm:SteaneStyleMeasurement.step2_gauss_commute, thm:SteaneStyleMeasurement.step2_gauss_pure_X, thm:SteaneStyleMeasurement.step2_cx_transforms_gauss, thm:SteaneStyleMeasurement.step3_edgeZ_commute, thm:SteaneStyleMeasurement.step3_edgeZ_self_inverse}
The result is assembled from the three steps: Step~1 uses \texttt{step1\_gauss\_commute} and \texttt{step1\_gauss\_pure\_X}; Step~2 uses \texttt{step2\_gauss\_commute}, \texttt{step2\_gauss\_pure\_X}, and \texttt{step2\_cx\_transforms\_gauss}; Step~3 uses \texttt{step3\_edgeZ\_commute} and \texttt{step3\_edgeZ\_self\_inverse}.
\end{proof}

\section{Ancilla Code Properties}

\begin{theorem}[CSS Check Partition]
\label{thm:SteaneStyleMeasurement.css_check_partition}
\lean{QEC1.SteaneStyleMeasurement.css_check_partition}
\leanok
\uses{def:StabilizerCode, def:SteaneStyleMeasurement.IsCSS, def:SteaneStyleMeasurement.xCheckIndices, def:SteaneStyleMeasurement.zCheckIndices}
For a CSS code $C$, every non-identity check belongs to either the X-type check indices or the Z-type check indices: if $\mathtt{check}(i) \neq 1$ then $i \in \operatorname{xCheckIndices}(C) \lor i \in \operatorname{zCheckIndices}(C)$.
\end{theorem}

\begin{proof}
\leanok
\uses{def:SteaneStyleMeasurement.IsCSS, def:SteaneStyleMeasurement.xCheckIndices, def:SteaneStyleMeasurement.zCheckIndices}
From the CSS property, either $(\mathtt{check}(i)).\mathtt{zVec} = 0$ or $(\mathtt{check}(i)).\mathtt{xVec} = 0$. In the first case, if $\mathtt{xVec} = 0$ as well, then the check is the identity, contradicting $\mathtt{check}(i) \neq 1$; otherwise $i \in \operatorname{xCheckIndices}$. The second case is analogous, yielding $i \in \operatorname{zCheckIndices}$.
\end{proof}

\begin{theorem}[Dummy Qubit Count]
\label{thm:SteaneStyleMeasurement.dummy_qubit_count}
\lean{QEC1.SteaneStyleMeasurement.dummy_qubit_count}
\leanok
\uses{def:StabilizerCode, def:SteaneStyleMeasurement.xCheckIndices}
The number of dummy qubits for state preparation equals the number of X-type checks in the ancilla CSS code.
\end{theorem}

\begin{proof}
\leanok
\uses{def:SteaneStyleMeasurement.xCheckIndices}
This holds by definitional equality.
\end{proof}

\begin{theorem}[X-Check and Z-Check Commute]
\label{thm:SteaneStyleMeasurement.xCheck_zCheck_commute}
\lean{QEC1.SteaneStyleMeasurement.xCheck_zCheck_commute}
\leanok
\uses{def:StabilizerCode, def:PauliOp.PauliCommute, def:SteaneStyleMeasurement.xCheckIndices, def:SteaneStyleMeasurement.zCheckIndices}
For a CSS code, X-type checks commute with Z-type checks.
\end{theorem}

\begin{proof}
\leanok
\uses{def:StabilizerCode}
This follows directly from the fact that all checks of a stabilizer code pairwise commute.
\end{proof}

\begin{theorem}[Z-Check Boundary Structure]
\label{thm:SteaneStyleMeasurement.zCheck_boundary_structure}
\lean{QEC1.SteaneStyleMeasurement.zCheck_boundary_structure}
\leanok
\uses{def:StabilizerCode, def:SteaneStyleMeasurement.zCheckIncident, def:SteaneStyleMeasurement.zCheckIndices}
For a Z-type check index $i$ and qubit $q$, the Z-check incidence relation $\operatorname{zCheckIncident}(\mathtt{check}, q, i)$ holds if and only if $(\mathtt{check}(i)).\mathtt{zVec}(q) \neq 0$.
\end{theorem}

\begin{proof}
\leanok
\uses{def:SteaneStyleMeasurement.zCheckIncident}
This holds by definitional equality.
\end{proof}

\section{Degree Analysis}

\begin{theorem}[Data Neighbor Set]
\label{thm:SteaneStyleMeasurement.data_neighborFinset}
\lean{QEC1.SteaneStyleMeasurement.data_neighborFinset}
\leanok
\uses{def:SteaneStyleMeasurement.MatchingGraph}
The neighbor set of data vertex $\operatorname{inl}(i)$ in the matching graph is $\{\operatorname{inr}(i)\}$.
\end{theorem}

\begin{proof}
\leanok
\uses{thm:SteaneStyleMeasurement.data_adj_iff}
By extensionality, membership in the neighbor finset is equivalent to adjacency, which by the data adjacency characterization is equivalent to $q = \operatorname{inr}(i)$, i.e., membership in the singleton $\{\operatorname{inr}(i)\}$.
\end{proof}

\begin{theorem}[Ancilla Neighbor Set]
\label{thm:SteaneStyleMeasurement.ancilla_neighborFinset}
\lean{QEC1.SteaneStyleMeasurement.ancilla_neighborFinset}
\leanok
\uses{def:SteaneStyleMeasurement.MatchingGraph}
The neighbor set of ancilla vertex $\operatorname{inr}(i)$ in the matching graph is $\{\operatorname{inl}(i)\}$.
\end{theorem}

\begin{proof}
\leanok
\uses{thm:SteaneStyleMeasurement.ancilla_adj_iff}
By extensionality, membership in the neighbor finset is equivalent to adjacency, which by the ancilla adjacency characterization is equivalent to $q = \operatorname{inl}(i)$.
\end{proof}

\begin{theorem}[Data Degree is One]
\label{thm:SteaneStyleMeasurement.data_degree}
\lean{QEC1.SteaneStyleMeasurement.data_degree}
\leanok
\uses{def:SteaneStyleMeasurement.MatchingGraph}
Each data vertex has degree exactly 1 in the matching graph.
\end{theorem}

\begin{proof}
\leanok
\uses{thm:SteaneStyleMeasurement.data_neighborFinset}
Rewriting the degree as the cardinality of the neighbor finset and using the data neighbor set result, $|\{\operatorname{inr}(i)\}| = 1$.
\end{proof}

\begin{theorem}[Ancilla Degree is One]
\label{thm:SteaneStyleMeasurement.ancilla_degree}
\lean{QEC1.SteaneStyleMeasurement.ancilla_degree}
\leanok
\uses{def:SteaneStyleMeasurement.MatchingGraph}
Each ancilla vertex has degree exactly 1 in the matching graph.
\end{theorem}

\begin{proof}
\leanok
\uses{thm:SteaneStyleMeasurement.ancilla_neighborFinset}
Rewriting the degree as the cardinality of the neighbor finset and using the ancilla neighbor set result, $|\{\operatorname{inl}(i)\}| = 1$.
\end{proof}

\begin{theorem}[Matching Graph Degree Equals One]
\label{thm:SteaneStyleMeasurement.matchingGraph_degree_eq_one}
\lean{QEC1.SteaneStyleMeasurement.matchingGraph_degree_eq_one}
\leanok
\uses{def:SteaneStyleMeasurement.MatchingGraph}
Every vertex in the matching graph has degree exactly 1.
\end{theorem}

\begin{proof}
\leanok
\uses{thm:SteaneStyleMeasurement.data_degree, thm:SteaneStyleMeasurement.ancilla_degree}
We case-split on whether $v = \operatorname{inl}(i)$ or $v = \operatorname{inr}(i)$. In the first case we apply \texttt{data\_degree}, in the second \texttt{ancilla\_degree}.
\end{proof}

\begin{theorem}[Matching Graph Degree Sum]
\label{thm:SteaneStyleMeasurement.matchingGraph_degree_sum}
\lean{QEC1.SteaneStyleMeasurement.matchingGraph_degree_sum}
\leanok
\uses{def:SteaneStyleMeasurement.MatchingGraph}
The degree sum of the matching graph is $2n$:
\[
\sum_{v} \deg(v) = 2n.
\]
\end{theorem}

\begin{proof}
\leanok
\uses{thm:SteaneStyleMeasurement.data_degree, thm:SteaneStyleMeasurement.ancilla_degree}
We split the sum over the sum type $\operatorname{Fin}(n) \oplus \operatorname{Fin}(n)$. Each data vertex contributes degree 1 and each ancilla vertex contributes degree 1, giving $n \cdot 1 + n \cdot 1 = 2n$.
\end{proof}

\begin{theorem}[Matching Graph Edge Count]
\label{thm:SteaneStyleMeasurement.matchingGraph_edge_count}
\lean{QEC1.SteaneStyleMeasurement.matchingGraph_edge_count}
\leanok
\uses{def:SteaneStyleMeasurement.MatchingGraph}
The matching graph has exactly $n$ edges: $2|E| = 2n$.
\end{theorem}

\begin{proof}
\leanok
\uses{thm:SteaneStyleMeasurement.matchingGraph_degree_sum}
By the handshaking lemma, $2|E| = \sum_v \deg(v) = 2n$.
\end{proof}

\section{Summary}

\begin{theorem}[Steane Measurement Summary]
\label{thm:SteaneStyleMeasurement.steane_measurement_summary}
\lean{QEC1.SteaneStyleMeasurement.steane_measurement_summary}
\leanok
\uses{def:SteaneStyleMeasurement.MatchingGraph, def:SteaneStyleMeasurement.zCheckIncident, def:HypergraphGeneralization.hyperGaussLawOp, def:GaussFlux.gaussLawOp, def:CircuitImplementation.entanglingCircuitAction, def:PauliOp.pauliX, def:PauliOp.pauliZ, def:PauliOp.PauliCommute, def:GaussFlux.ExtQubit}
The complete Steane-style measurement decomposition satisfies:
\begin{enumerate}
\item \textbf{Step 1:} All hypergraph Gauss operators commute.
\item \textbf{Step 2:} All matching graph Gauss operators commute, and the CX circuit maps $A_v \mapsto X_{\operatorname{inl}(v)}$.
\item \textbf{Step 3:} All edge $Z$ operators commute.
\item \textbf{Degree bound:} Every vertex has degree 1.
\item \textbf{Edge count:} $2|E| = 2n$.
\end{enumerate}
\end{theorem}

\begin{proof}
\leanok
\uses{thm:SteaneStyleMeasurement.step1_gauss_commute, thm:SteaneStyleMeasurement.step2_gauss_commute, thm:SteaneStyleMeasurement.step2_cx_transforms_gauss, thm:SteaneStyleMeasurement.step3_edgeZ_commute, thm:SteaneStyleMeasurement.matchingGraph_degree_eq_one, thm:SteaneStyleMeasurement.matchingGraph_edge_count}
The result assembles all the previously established properties: Step~1 commutation, Step~2 commutation and CX transformation, Step~3 commutation, the degree bound, and the edge count.
\end{proof}

%--- Rem_26: CohenEtAlSchemeRecovery ---
Now I have the full file. Let me produce the LaTeX translation.

\chapter{Rem 26: Cohen et al.\ Scheme Recovery}

\begin{definition}[Logical Support]
\label{def:CohenEtAlSchemeRecovery.logicalSupport}
\lean{QEC1.CohenEtAlSchemeRecovery.logicalSupport}
\leanok
\uses{def:PauliOp}
Given a Pauli operator $L$ on qubits $Q$, the \emph{logical support} is defined as the $X$-support of $L$:
\[
\operatorname{logicalSupport}(L) := \operatorname{supp}_X(L).
\]
\end{definition}

\begin{definition}[Z-Type Check]
\label{def:CohenEtAlSchemeRecovery.isZTypeCheck}
\lean{QEC1.CohenEtAlSchemeRecovery.isZTypeCheck}
\leanok
\uses{def:StabilizerCode, def:PauliOp}
A check index $i$ of a stabilizer code $C$ is a \emph{$Z$-type check} if the corresponding check operator has no $X$-support, i.e., the $X$-vector of $C.\operatorname{check}(i)$ is zero.
\end{definition}

\begin{definition}[Relevant Z-Checks]
\label{def:CohenEtAlSchemeRecovery.relevantZChecks}
\lean{QEC1.CohenEtAlSchemeRecovery.relevantZChecks}
\leanok
\uses{def:StabilizerCode, def:CohenEtAlSchemeRecovery.isZTypeCheck, def:CohenEtAlSchemeRecovery.logicalSupport}
The \emph{relevant $Z$-checks} of a stabilizer code $C$ with respect to a logical operator $L$ are the $Z$-type check indices $i$ whose $Z$-support has nonempty intersection with $\operatorname{logicalSupport}(L)$:
\[
E_L := \{ i \in C.I \mid \text{isZTypeCheck}(i) \;\wedge\; \operatorname{supp}_Z(C.\operatorname{check}(i)) \cap \operatorname{logicalSupport}(L) \neq \emptyset \}.
\]
\end{definition}

\begin{definition}[Restricted Z-Check Incidence]
\label{def:CohenEtAlSchemeRecovery.restrictedZCheckIncident}
\lean{QEC1.CohenEtAlSchemeRecovery.restrictedZCheckIncident}
\leanok
\uses{def:StabilizerCode, def:CohenEtAlSchemeRecovery.logicalSupport, def:CohenEtAlSchemeRecovery.isZTypeCheck}
The \emph{restricted incidence relation} for the hypergraph on $\operatorname{logicalSupport}(L)$: a qubit $q$ is incident to check index $i$ if and only if $q \in \operatorname{logicalSupport}(L)$, the $Z$-vector of $C.\operatorname{check}(i)$ is nonzero at $q$, and $i$ is a $Z$-type check.
\end{definition}

\begin{definition}[Restricted Boundary Map]
\label{def:CohenEtAlSchemeRecovery.restrictedBoundaryMap}
\lean{QEC1.CohenEtAlSchemeRecovery.restrictedBoundaryMap}
\leanok
\uses{def:HypergraphGeneralization.hyperBoundaryMap, def:CohenEtAlSchemeRecovery.restrictedZCheckIncident}
The \emph{restricted boundary map} $\partial : \mathbb{Z}_2^{E_L} \to \mathbb{Z}_2^{V_L}$ is the hypergraph boundary map for the restricted incidence relation.
\end{definition}

\begin{definition}[Restricted Coboundary Map]
\label{def:CohenEtAlSchemeRecovery.restrictedCoboundaryMap}
\lean{QEC1.CohenEtAlSchemeRecovery.restrictedCoboundaryMap}
\leanok
\uses{def:HypergraphGeneralization.hyperCoboundaryMap, def:CohenEtAlSchemeRecovery.restrictedZCheckIncident}
The \emph{restricted coboundary map} $\delta : \mathbb{Z}_2^{V_L} \to \mathbb{Z}_2^{E_L}$ is the hypergraph coboundary map (transpose of $\partial$) for the restricted incidence relation.
\end{definition}

\begin{definition}[X-Type Logical Operator]
\label{def:CohenEtAlSchemeRecovery.IsXTypeLogical}
\lean{QEC1.CohenEtAlSchemeRecovery.IsXTypeLogical}
\leanok
\uses{def:StabilizerCode, def:PauliOp, def:StabilizerCode.isLogicalOp}
A Pauli operator $L$ is an \emph{$X$-type logical operator} for a stabilizer code $C$ if $L$ is pure $X$-type (i.e., $L.\operatorname{zVec} = 0$) and $L$ is a logical operator of $C$ (in the centralizer, not a stabilizer, and not the identity).
\end{definition}

\begin{definition}[Irreducible X Logical Operator]
\label{def:CohenEtAlSchemeRecovery.IsIrreducibleXLogical}
\lean{QEC1.CohenEtAlSchemeRecovery.IsIrreducibleXLogical}
\leanok
\uses{def:CohenEtAlSchemeRecovery.IsXTypeLogical, def:PauliOp, def:PauliOp.weight, def:StabilizerCode, def:StabilizerCode.inCentralizer}
An $X$-type logical operator $L$ is \emph{irreducible} if it cannot be written as a product $L = P \cdot R$ of two pure $X$-type centralizer elements where both $P$ and $R$ have weight strictly less than $\operatorname{wt}(L)$ and neither is the identity. Formally, $L$ is an $X$-type logical and for all pure-$X$ centralizer elements $P, R$ with $P \cdot R = L$, we have $\operatorname{wt}(P) \ge \operatorname{wt}(L)$ or $\operatorname{wt}(R) \ge \operatorname{wt}(L)$ or $P = \mathbf{1}$ or $R = \mathbf{1}$.
\end{definition}

\begin{theorem}[Z-Check Restricted Support Even]
\label{thm:CohenEtAlSchemeRecovery.zCheck_restricted_support_even}
\lean{QEC1.CohenEtAlSchemeRecovery.zCheck_restricted_support_even}
\leanok
\uses{def:StabilizerCode, def:PauliOp, def:PauliOp.PauliCommute, def:PauliOp.symplecticInner, def:StabilizerCode.inCentralizer, def:CohenEtAlSchemeRecovery.restrictedZCheckIncident, def:CohenEtAlSchemeRecovery.isZTypeCheck}
If $L$ is a pure $X$-type operator ($L.\operatorname{zVec} = 0$) that commutes with all checks of a stabilizer code $C$ (i.e., $L \in \operatorname{Centralizer}(C)$), then for each $Z$-type check $s_j$, the set of qubits $q$ satisfying the restricted incidence relation with $j$ has even cardinality. That is:
\[
\bigl|\{q \in Q \mid \operatorname{restrictedZCheckIncident}(C, L, q, j)\}\bigr| \text{ is even.}
\]
\end{theorem}

\begin{proof}
\leanok
\uses{def:PauliOp.PauliCommute, def:PauliOp.symplecticInner, def:CohenEtAlSchemeRecovery.restrictedZCheckIncident, def:CohenEtAlSchemeRecovery.logicalSupport, def:CohenEtAlSchemeRecovery.isZTypeCheck}
Let $h_{\mathrm{comm}} := h_{L,\mathrm{comm}}(i)$ be the commutation condition $\langle L, C.\operatorname{check}(i) \rangle_{\mathrm{symp}} = 0$. Unfolding the definitions of $\operatorname{PauliCommute}$ and $\operatorname{symplecticInner}$, and using that $L.\operatorname{zVec} = 0$ and $(C.\operatorname{check}(i)).\operatorname{xVec} = 0$ (since $i$ is $Z$-type), the symplectic inner product simplifies to $\sum_q L.\operatorname{xVec}(q) \cdot (C.\operatorname{check}(i)).\operatorname{zVec}(q) = 0$. We rewrite the cardinality condition using $\operatorname{natCast\_eq\_zero\_iff\_even}$, express the cardinality as a sum of ones over the filter, and push the cast. We then show the sums agree by extensionality: for each $q$, we case-split on whether $L.\operatorname{xVec}(q) = 0$ and whether $(C.\operatorname{check}(i)).\operatorname{zVec}(q) = 0$. Over $\mathbb{Z}_2$, each nonzero element equals $1$, so both sides of the summand agree in all cases.
\end{proof}

\begin{theorem}[prodX Commutes with Z-Check iff Even Overlap]
\label{thm:CohenEtAlSchemeRecovery.prodX_commutes_zCheck_iff_even}
\lean{QEC1.CohenEtAlSchemeRecovery.prodX_commutes_zCheck_iff_even}
\leanok
\uses{def:StabilizerCode, def:PauliOp, def:PauliOp.PauliCommute, def:PauliOp.symplecticInner, def:PauliOp.prodX, def:CohenEtAlSchemeRecovery.isZTypeCheck}
For a CSS code, $\operatorname{prodX}(S)$ commutes with a $Z$-type check $s_j$ if and only if $|S \cap \operatorname{supp}_Z(s_j)|$ is even:
\[
\operatorname{PauliCommute}(\operatorname{prodX}(S), C.\operatorname{check}(j)) \;\Leftrightarrow\; \bigl|\{q \in S \mid (C.\operatorname{check}(j)).\operatorname{zVec}(q) \neq 0\}\bigr| \text{ is even.}
\]
\end{theorem}

\begin{proof}
\leanok
\uses{def:PauliOp.PauliCommute, def:PauliOp.symplecticInner, def:PauliOp.prodX, def:CohenEtAlSchemeRecovery.isZTypeCheck}
Unfolding $\operatorname{PauliCommute}$ and $\operatorname{symplecticInner}$, we apply the helper lemma that computes the symplectic inner product of $\operatorname{prodX}(S)$ with a $Z$-type check as a filtered sum. The result then follows by rewriting the even-cardinality condition via $\operatorname{natCast\_eq\_zero\_iff\_even}$, expressing the cardinality as a sum of ones over the filter, and pushing the cast. The two sides are definitionally equal.
\end{proof}

\begin{theorem}[prodX in Centralizer from Even Overlap]
\label{thm:CohenEtAlSchemeRecovery.prodX_centralizer_of_even_overlap}
\lean{QEC1.CohenEtAlSchemeRecovery.prodX_centralizer_of_even_overlap}
\leanok
\uses{def:StabilizerCode, def:PauliOp.prodX, def:SteaneStyleMeasurement.IsCSS, def:StabilizerCode.inCentralizer, def:CohenEtAlSchemeRecovery.isZTypeCheck, thm:SteaneStyleMeasurement.pureX_pureX_commute}
For a CSS code $C$, $\operatorname{prodX}(S)$ is in the centralizer of $C$ if and only if $S$ has even overlap with every $Z$-type check's $Z$-support. That is, if for all $Z$-type checks $j$, $|\{q \in S \mid (C.\operatorname{check}(j)).\operatorname{zVec}(q) \neq 0\}|$ is even, then $C.\operatorname{inCentralizer}(\operatorname{prodX}(S))$.
\end{theorem}

\begin{proof}
\leanok
\uses{def:SteaneStyleMeasurement.IsCSS, thm:SteaneStyleMeasurement.pureX_pureX_commute, thm:CohenEtAlSchemeRecovery.prodX_commutes_zCheck_iff_even}
Let $i$ be any check index. By the CSS property, check $i$ is either pure $X$-type or pure $Z$-type. In the first case, $\operatorname{prodX}(S)$ commutes with it by the pure-$X$/pure-$X$ commutation lemma (since $\operatorname{prodX}(S)$ has zero $Z$-vector). In the second case, the check is $Z$-type, and commutation follows from the even overlap hypothesis via $\operatorname{prodX\_commutes\_zCheck\_iff\_even}$.
\end{proof}

\begin{theorem}[Coboundary Kernel Gives Centralizer Pair]
\label{thm:CohenEtAlSchemeRecovery.coboundary_ker_gives_centralizer_pair}
\lean{QEC1.CohenEtAlSchemeRecovery.coboundary_ker_gives_centralizer_pair}
\leanok
\uses{def:StabilizerCode, def:PauliOp, def:PauliOp.prodX, def:SteaneStyleMeasurement.IsCSS, def:StabilizerCode.inCentralizer, def:CohenEtAlSchemeRecovery.restrictedCoboundaryMap, def:CohenEtAlSchemeRecovery.logicalSupport, def:HypergraphGeneralization.hyperCoboundaryMap, def:CohenEtAlSchemeRecovery.restrictedZCheckIncident}
Let $C$ be a CSS stabilizer code and $L$ a Pauli operator. If $f : Q \to \mathbb{Z}_2$ lies in the kernel of the restricted coboundary map and is supported on $\operatorname{logicalSupport}(L)$ (i.e., $f(q) = 0$ for $q \notin V_L$), then $\operatorname{prodX}(\{q \mid f(q) \neq 0\})$ is in the centralizer of $C$.
\end{theorem}

\begin{proof}
\leanok
\uses{thm:CohenEtAlSchemeRecovery.prodX_centralizer_of_even_overlap, def:SteaneStyleMeasurement.IsCSS, def:CohenEtAlSchemeRecovery.restrictedCoboundaryMap, def:HypergraphGeneralization.hyperCoboundaryMap, def:CohenEtAlSchemeRecovery.restrictedZCheckIncident}
We apply $\operatorname{prodX\_centralizer\_of\_even\_overlap}$, using the CSS hypothesis. For each $Z$-type check $j$, the kernel condition $f \in \ker(\delta)$ gives $\delta(f)(j) = 0$, which by definition means $\sum_q [\text{incident}(q,j)] \cdot f(q) = 0$. We rewrite the even-cardinality goal using $\operatorname{natCast\_eq\_zero\_iff\_even}$ and express the cardinality as a sum. The two sums are then shown to agree by a summand-by-summand case analysis: for each $q$, we consider whether $f(q) = 0$, whether the restricted incidence holds, and whether $q \in V_L$. Over $\mathbb{Z}_2$, nonzero values equal $1$, so the terms match in all cases. If $q \notin V_L$, the support hypothesis forces $f(q) = 0$.
\end{proof}

\begin{theorem}[$\mathbb{Z}_2$ Rank-Nullity for Incidence Matrices]
\label{thm:CohenEtAlSchemeRecovery.z2_finrank_ker_formula}
\lean{QEC1.CohenEtAlSchemeRecovery.z2_finrank_ker_formula}
\leanok
\uses{def:HypergraphGeneralization.hyperBoundaryMap, def:HypergraphGeneralization.hyperCoboundaryMap}
For an incidence relation between finite types $V_0$ and $E_0$ over $\mathbb{Z}_2$, the boundary map $\partial = M \cdot$ and coboundary map $\delta = M^T \cdot$ satisfy:
\[
\dim(\ker(\delta)) + |E_0| = \dim(\ker(\partial)) + |V_0|.
\]
\end{theorem}

\begin{proof}
\leanok
\uses{def:HypergraphGeneralization.hyperBoundaryMap, def:HypergraphGeneralization.hyperCoboundaryMap}
We rewrite both maps as matrix multiplication linear maps: $\partial = M \cdot$ and $\delta = M^T \cdot$, where $M$ is the $\mathbb{Z}_2$ incidence matrix. Applying the rank-nullity theorem $\operatorname{finrank}(\operatorname{range}(A)) + \operatorname{finrank}(\ker(A)) = \operatorname{finrank}(\text{domain})$ to both $M$ and $M^T$, and using $\operatorname{rank}(M) = \operatorname{rank}(M^T)$ (by $\operatorname{Matrix.rank\_transpose}$), the result follows by linear arithmetic (omega).
\end{proof}

\begin{theorem}[Coboundary Third Element]
\label{thm:CohenEtAlSchemeRecovery.coboundary_third_element}
\lean{QEC1.CohenEtAlSchemeRecovery.coboundary_third_element}
\leanok
\uses{def:HypergraphGeneralization.hyperBoundaryMap, def:HypergraphGeneralization.hyperCoboundaryMap}
Given $\dim(\ker(\partial)) \ge 2$ and $|E_0| \le |V_0|$, and any $w \in \ker(\delta)$, there exists $g \in \ker(\delta)$ with $g \neq 0$ and $g \neq w$.
\end{theorem}

\begin{proof}
\leanok
\uses{thm:CohenEtAlSchemeRecovery.z2_finrank_ker_formula}
From the $\mathbb{Z}_2$ rank-nullity formula and the hypothesis $|E_0| \le |V_0|$, we obtain $\dim(\ker(\delta)) \ge 2$. We then apply the submodule existence lemma: in a $\mathbb{Z}_2$-submodule of $\operatorname{finrank} \ge 2$, given any element $w$, there exists $v \neq 0$ with $v \neq w$.
\end{proof}

\begin{definition}[Core Incidence on Subtypes]
\label{def:CohenEtAlSchemeRecovery.coreIncident}
\lean{QEC1.CohenEtAlSchemeRecovery.coreIncident}
\leanok
\uses{def:StabilizerCode, def:CohenEtAlSchemeRecovery.logicalSupport, def:CohenEtAlSchemeRecovery.relevantZChecks}
The \emph{core incidence} on subtypes: a vertex $q \in V_L$ is incident to a check $i \in E_L$ if and only if $(C.\operatorname{check}(i)).\operatorname{zVec}(q) \neq 0$.
\end{definition}

\begin{definition}[Lift from $V_L$ to $Q$]
\label{def:CohenEtAlSchemeRecovery.liftVL}
\lean{QEC1.CohenEtAlSchemeRecovery.liftVL}
\leanok
\uses{def:CohenEtAlSchemeRecovery.logicalSupport}
Given a function $g : V_L \to \mathbb{Z}_2$, the \emph{lift} $\operatorname{liftVL}(g) : Q \to \mathbb{Z}_2$ extends $g$ by zero outside $V_L$:
\[
\operatorname{liftVL}(g)(q) = \begin{cases} g(q) & \text{if } q \in V_L, \\ 0 & \text{otherwise.} \end{cases}
\]
\end{definition}

\begin{lemma}[Lift is Injective]
\label{lem:CohenEtAlSchemeRecovery.liftVL_injective}
\lean{QEC1.CohenEtAlSchemeRecovery.liftVL_injective}
\leanok
\uses{def:CohenEtAlSchemeRecovery.liftVL, def:CohenEtAlSchemeRecovery.logicalSupport}
The lift map $\operatorname{liftVL}$ is injective.
\end{lemma}

\begin{proof}
\leanok
\uses{def:CohenEtAlSchemeRecovery.liftVL}
Let $g_1, g_2$ be functions on $V_L$ with $\operatorname{liftVL}(g_1) = \operatorname{liftVL}(g_2)$. By extensionality, for any $\langle q, h_q \rangle \in V_L$, evaluating the equality at $q$ and using the definition of $\operatorname{liftVL}$ (which applies $g$ at $q$ when $q \in V_L$) gives $g_1(\langle q, h_q \rangle) = g_2(\langle q, h_q \rangle)$.
\end{proof}

\begin{lemma}[Lift Preserves Coboundary Kernel]
\label{lem:CohenEtAlSchemeRecovery.liftVL_ker_coboundary}
\lean{QEC1.CohenEtAlSchemeRecovery.liftVL_ker_coboundary}
\leanok
\uses{def:CohenEtAlSchemeRecovery.liftVL, def:CohenEtAlSchemeRecovery.coreIncident, def:CohenEtAlSchemeRecovery.restrictedCoboundaryMap, def:HypergraphGeneralization.hyperCoboundaryMap, def:CohenEtAlSchemeRecovery.restrictedZCheckIncident, def:CohenEtAlSchemeRecovery.relevantZChecks, def:CohenEtAlSchemeRecovery.logicalSupport}
If $g \in \ker(\operatorname{hyperCoboundaryMap}(\operatorname{coreIncident}))$, then $\operatorname{liftVL}(g) \in \ker(\operatorname{restrictedCoboundaryMap})$.
\end{lemma}

\begin{proof}
\leanok
\uses{def:CohenEtAlSchemeRecovery.liftVL, def:CohenEtAlSchemeRecovery.coreIncident, def:CohenEtAlSchemeRecovery.restrictedCoboundaryMap, def:HypergraphGeneralization.hyperCoboundaryMap, def:CohenEtAlSchemeRecovery.restrictedZCheckIncident, def:CohenEtAlSchemeRecovery.relevantZChecks}
We show the coboundary applied to $\operatorname{liftVL}(g)$ is zero componentwise. For each check index $j$, we consider two cases. If $j \in E_L$ (the relevant $Z$-checks), we convert between the sum over all qubits $Q$ (with restricted incidence) and the sum over the subtype $V_L$ (with core incidence). The key step normalizes the subtype summand, converts the subtype sum to a finset sum via $\operatorname{sum\_coe\_sort}$, then to a full type sum via $\operatorname{sum\_ite\_mem}$, and finally shows summand-by-summand agreement with the restricted incidence formulation. For each qubit $q$, we case-split on membership in $V_L$ and on whether the $Z$-vector is nonzero, showing agreement in all cases. If $j \notin E_L$, no vertex is incident to $j$ (by the definition of relevant checks), so every summand is zero.
\end{proof}

\begin{theorem}[Coboundary from Boundary Step (Rank-Nullity)]
\label{thm:CohenEtAlSchemeRecovery.coboundary_from_boundary_step}
\lean{QEC1.CohenEtAlSchemeRecovery.coboundary_from_boundary_step}
\leanok
\uses{def:StabilizerCode, def:PauliOp, def:CohenEtAlSchemeRecovery.IsIrreducibleXLogical, def:CohenEtAlSchemeRecovery.restrictedBoundaryMap, def:CohenEtAlSchemeRecovery.restrictedCoboundaryMap, def:CohenEtAlSchemeRecovery.relevantZChecks, def:CohenEtAlSchemeRecovery.logicalSupport, def:HypergraphGeneralization.hyperBoundaryMap, def:CohenEtAlSchemeRecovery.coreIncident, def:CohenEtAlSchemeRecovery.liftVL}
Let $L$ be an irreducible $X$ logical operator. If $\gamma \in \ker(\partial)$ with $\gamma \neq 0$ and $\gamma$ not equal to the all-ones vector on $E_L$, and $|E_L| \le |V_L|$, then there exists $f : Q \to \mathbb{Z}_2$ with:
\begin{enumerate}
\item $f \in \ker(\delta)$ (the restricted coboundary kernel),
\item $f$ is supported on $V_L$,
\item $f \neq 0$,
\item there exists $q \in V_L$ with $f(q) = 0$.
\end{enumerate}
\end{theorem}

\begin{proof}
\leanok
\uses{thm:CohenEtAlSchemeRecovery.coboundary_third_element, thm:CohenEtAlSchemeRecovery.z2_finrank_ker_formula, lem:CohenEtAlSchemeRecovery.liftVL_injective, lem:CohenEtAlSchemeRecovery.liftVL_ker_coboundary, def:CohenEtAlSchemeRecovery.coreIncident, def:HypergraphGeneralization.hyperBoundaryMap, def:HypergraphGeneralization.hyperCoboundaryMap}
We restrict $\gamma$ and the all-ones vector to the core (subtype) boundary map. Letting $\gamma_{\mathrm{core}}(i) = \gamma(i_1)$ for $i \in E_L$ and $\mathbf{1}_{\mathrm{core}}(i) = 1$, we verify both lie in $\ker(\operatorname{hyperBoundaryMap}(\operatorname{coreIncident}))$ by converting between the full $C.I$ sums and subtype sums, matching summands by case analysis on membership and incidence. Since $\gamma_{\mathrm{core}} \neq 0$ (from $\gamma \neq 0$ and the support condition) and $\gamma_{\mathrm{core}} \neq \mathbf{1}_{\mathrm{core}}$ (from $\gamma$ not being all-ones), over $\mathbb{Z}_2$ these two nonzero distinct kernel elements force $\dim(\ker(\partial_{\mathrm{core}})) \ge 2$: if $\dim \le 1$, by $\operatorname{finrank\_le\_one}$ every element is a scalar multiple of a generator, but over $\mathbb{Z}_2$ the only nonzero scalar is $1$, forcing $\gamma_{\mathrm{core}} = \mathbf{1}_{\mathrm{core}}$, a contradiction. The all-ones function on $V_L$ lies in $\ker(\delta_{\mathrm{core}})$ by row evenness. With $|E_L| \le |V_L|$ and $\dim(\ker(\partial_{\mathrm{core}})) \ge 2$, $\operatorname{coboundary\_third\_element}$ yields $g \in \ker(\delta_{\mathrm{core}})$ with $g \neq 0$ and $g \neq \mathbf{1}$. We lift $g$ to $Q$ via $\operatorname{liftVL}$: the lift is in $\ker(\delta)$ by $\operatorname{liftVL\_ker\_coboundary}$, is supported on $V_L$, is nonzero by injectivity of the lift, and since $g \neq \mathbf{1}$, there exists $q \in V_L$ with $g(q) \neq 1$, hence $g(q) = 0$ over $\mathbb{Z}_2$.
\end{proof}

\begin{theorem}[Kernel Restricted Boundary Trivial]
\label{thm:CohenEtAlSchemeRecovery.kernel_restricted_boundary_trivial}
\lean{QEC1.CohenEtAlSchemeRecovery.kernel_restricted_boundary_trivial}
\leanok
\uses{def:StabilizerCode, def:PauliOp, def:PauliOp.prodX, def:PauliOp.weight, def:SteaneStyleMeasurement.IsCSS, def:CohenEtAlSchemeRecovery.IsIrreducibleXLogical, def:CohenEtAlSchemeRecovery.restrictedBoundaryMap, def:CohenEtAlSchemeRecovery.restrictedCoboundaryMap, def:CohenEtAlSchemeRecovery.relevantZChecks, def:CohenEtAlSchemeRecovery.logicalSupport, def:HypergraphGeneralization.hyperBoundaryMap, def:HypergraphGeneralization.hyperCoboundaryMap}
For an irreducible $X$ logical operator $L$ of a CSS stabilizer code $C$, with $|E_L| \le |V_L|$, the kernel of the restricted boundary map $\partial : \mathbb{Z}_2^{E_L} \to \mathbb{Z}_2^{V_L}$ is trivial: for any $\gamma \in \ker(\partial)$ supported on $E_L$, either $\gamma = 0$ or $\gamma$ is the all-ones vector on $E_L$ (i.e., $\gamma(i) = 1$ for all $i \in E_L$).
\end{theorem}

\begin{proof}
\leanok
\uses{thm:CohenEtAlSchemeRecovery.coboundary_from_boundary_step, thm:CohenEtAlSchemeRecovery.coboundary_ker_gives_centralizer_pair, def:CohenEtAlSchemeRecovery.IsIrreducibleXLogical, def:SteaneStyleMeasurement.IsCSS}
Assume $\gamma \neq 0$. We show $\gamma$ is all-ones on $E_L$. Suppose for contradiction that there exists $i_0 \in E_L$ with $\gamma(i_0) \neq 1$, so $\gamma$ is neither zero nor all-ones.

By $\operatorname{coboundary\_from\_boundary\_step}$, there exists $f \in \ker(\delta)$ supported on $V_L$ with $f \neq 0$ and some $q \in V_L$ with $f(q) = 0$. Let $S = \{q \mid f(q) \neq 0\}$ and $T = \{q \in V_L \mid f(q) = 0\}$.

Since $f \in \ker(\delta)$, $\operatorname{coboundary\_ker\_gives\_centralizer\_pair}$ gives $\operatorname{prodX}(S) \in \operatorname{Centralizer}(C)$. The complement function $g(q) = 1 + f(q)$ on $V_L$ (extended by zero) also lies in $\ker(\delta)$: its coboundary equals the sum of the all-ones coboundary (zero by row evenness) and $f$'s coboundary (zero by hypothesis). Applying $\operatorname{coboundary\_ker\_gives\_centralizer\_pair}$ to $g$ gives $\operatorname{prodX}(T) \in \operatorname{Centralizer}(C)$.

Since $S$ and $T$ are disjoint with $S \cup T = V_L = \operatorname{logicalSupport}(L)$, we have $\operatorname{prodX}(S) \cdot \operatorname{prodX}(T) = \operatorname{prodX}(V_L) = L$ (as $L$ is pure $X$-type). Both $S$ and $T$ are nonempty ($S$ because $f \neq 0$; $T$ because some $q \in V_L$ has $f(q) = 0$), so both factors have weight strictly less than $\operatorname{wt}(L)$ and are not the identity. This contradicts the irreducibility of $L$.
\end{proof}

\begin{definition}[Layered Incidence]
\label{def:CohenEtAlSchemeRecovery.layeredIncident}
\lean{QEC1.CohenEtAlSchemeRecovery.layeredIncident}
\leanok
\uses{def:StabilizerCode, def:PauliOp, def:CohenEtAlSchemeRecovery.restrictedZCheckIncident}
The \emph{layered incidence relation} for the Cohen et al.\ construction with $d$ layers. The vertex set is $Q \times \operatorname{Fin}(d)$ and edges are either inter-layer or intra-layer. A vertex $(q, \ell)$ is incident to:
\begin{itemize}
\item An \emph{inter-layer edge} $(q', k)$: if $q = q'$ and $\ell \in \{k, k+1\}$ (path graph connection between consecutive layers).
\item An \emph{intra-layer edge} $(i, \ell')$: if $\ell = \ell'$ and $q$ is incident to check $i$ in the restricted hypergraph.
\end{itemize}
\end{definition}

\begin{theorem}[Cohen Layered All-Ones Measures $L$]
\label{thm:CohenEtAlSchemeRecovery.cohen_layered_all_ones_measures_L}
\lean{QEC1.CohenEtAlSchemeRecovery.cohen_layered_all_ones_measures_L}
\leanok
\uses{def:HypergraphGeneralization.hyperGaussSubsetProduct, def:CohenEtAlSchemeRecovery.layeredIncident, def:StabilizerCode, def:PauliOp}
The all-ones Gauss subset product of the layered incidence with $d$ copies satisfies: for every vertex $v$ of the Cohen construction,
\[
\bigl(\operatorname{hyperGaussSubsetProduct}(\operatorname{layeredIncident}, \mathbf{1})\bigr).\operatorname{xVec}(\operatorname{inl}(v)) = 1,
\]
and the operator is pure $X$-type:
\[
\bigl(\operatorname{hyperGaussSubsetProduct}(\operatorname{layeredIncident}, \mathbf{1})\bigr).\operatorname{zVec} = 0.
\]
\end{theorem}

\begin{proof}
\leanok
\uses{def:HypergraphGeneralization.hyperGaussSubsetProduct, def:CohenEtAlSchemeRecovery.layeredIncident}
The first component follows by simplification using $\operatorname{hyperGaussSubsetProduct\_xVec\_vertex}$. The second component is $\operatorname{hyperGaussSubsetProduct\_zVec}$ applied to the layered incidence.
\end{proof}

\begin{definition}[Path Graph Adjacency]
\label{def:CohenEtAlSchemeRecovery.pathGraphAdj}
\lean{QEC1.CohenEtAlSchemeRecovery.pathGraphAdj}
\leanok

The \emph{path graph adjacency} on $R$ vertices: $a$ and $b$ in $\operatorname{Fin}(R)$ are adjacent if $a + 1 = b$ or $b + 1 = a$.
\end{definition}

\begin{theorem}[Path Graph Adjacency is Symmetric]
\label{thm:CohenEtAlSchemeRecovery.pathGraphAdj_symm}
\lean{QEC1.CohenEtAlSchemeRecovery.pathGraphAdj_symm}
\leanok
\uses{def:CohenEtAlSchemeRecovery.pathGraphAdj}
The path graph adjacency relation is symmetric.
\end{theorem}

\begin{proof}
\leanok
\uses{def:CohenEtAlSchemeRecovery.pathGraphAdj}
Unfolding the definition, $\operatorname{pathGraphAdj}(a, b)$ gives $a + 1 = b \lor b + 1 = a$. Symmetry follows by swapping the disjuncts, which is verified by integer arithmetic (omega).
\end{proof}

\begin{theorem}[Path Graph Adjacency is Irreflexive]
\label{thm:CohenEtAlSchemeRecovery.pathGraphAdj_irrefl}
\lean{QEC1.CohenEtAlSchemeRecovery.pathGraphAdj_irrefl}
\leanok
\uses{def:CohenEtAlSchemeRecovery.pathGraphAdj}
The path graph adjacency relation is irreflexive: $\neg \operatorname{pathGraphAdj}(a, a)$ for all $a$.
\end{theorem}

\begin{proof}
\leanok
\uses{def:CohenEtAlSchemeRecovery.pathGraphAdj}
By simplification of the path graph adjacency definition, $a + 1 = a$ is impossible.
\end{proof}

\begin{definition}[Path Graph]
\label{def:CohenEtAlSchemeRecovery.pathGraph}
\lean{QEC1.CohenEtAlSchemeRecovery.pathGraph}
\leanok
\uses{def:CohenEtAlSchemeRecovery.pathGraphAdj}
The \emph{path graph} on $R$ vertices is the simple graph on $\operatorname{Fin}(R)$ with adjacency given by $\operatorname{pathGraphAdj}$, using the symmetry and irreflexivity proved above.
\end{definition}

\begin{theorem}[Cross et al.\ Gauss Product is Pure $X$ with Vertex Values]
\label{thm:CohenEtAlSchemeRecovery.cross_layered_gauss_product_pure_X_with_vertex_values}
\lean{QEC1.CohenEtAlSchemeRecovery.cross_layered_gauss_product_pure_X_with_vertex_values}
\leanok
\uses{def:HypergraphGeneralization.hyperGaussSubsetProduct, def:CohenEtAlSchemeRecovery.layeredIncident, def:StabilizerCode, def:PauliOp}
The Cross et al.\ scheme with $R$ layers: for any coefficient function $c : Q \times \operatorname{Fin}(R) \to \mathbb{Z}_2$, the Gauss subset product is pure $X$-type and restricts to $c(v)$ on each vertex $v$:
\[
\bigl(\operatorname{hyperGaussSubsetProduct}(\operatorname{layeredIncident}, c)\bigr).\operatorname{zVec} = 0
\]
and
\[
\bigl(\operatorname{hyperGaussSubsetProduct}(\operatorname{layeredIncident}, c)\bigr).\operatorname{xVec}(\operatorname{inl}(v)) = c(v).
\]
\end{theorem}

\begin{proof}
\leanok
\uses{def:HypergraphGeneralization.hyperGaussSubsetProduct, def:CohenEtAlSchemeRecovery.layeredIncident}
The $Z$-vector vanishes by $\operatorname{hyperGaussSubsetProduct\_zVec}$. The $X$-vector restriction follows by simplification using $\operatorname{hyperGaussSubsetProduct\_xVec\_vertex}$.
\end{proof}

\begin{theorem}[Cross et al.\ Expansion Bound]
\label{thm:CohenEtAlSchemeRecovery.cross_expansion_bound}
\lean{QEC1.CohenEtAlSchemeRecovery.cross_expansion_bound}
\leanok
\uses{def:CohenEtAlSchemeRecovery.pathGraph, def:SimpleGraph.edgeBoundary, def:cheegerConstant}
For the Cross et al.\ scheme with $R$ layers, for any valid subset $S \subseteq \operatorname{Fin}(R)$ with $S \neq \emptyset$ and $2|S| \le R$, and any constant $c \le h(P_R)$ where $h(P_R)$ is the Cheeger constant of the path graph:
\[
c \cdot |S| \le |\partial S|,
\]
where $\partial S$ is the edge boundary of $S$ in the path graph.
\end{theorem}

\begin{proof}
\leanok
\uses{def:CohenEtAlSchemeRecovery.pathGraph, def:cheegerConstant, def:SimpleGraph.edgeBoundary}
This follows directly from $\operatorname{edgeBoundary\_card\_ge\_of\_cheeger}$ applied to the path graph.
\end{proof}

\begin{definition}[Joint Incidence]
\label{def:CohenEtAlSchemeRecovery.jointIncident}
\lean{QEC1.CohenEtAlSchemeRecovery.jointIncident}
\leanok

The \emph{joint incidence} relation combines two individual incidence relations with bridge edges. Given two systems $(Q_1, E_1, \operatorname{incident}_1)$ and $(Q_2, E_2, \operatorname{incident}_2)$ with bridge edges $B$ connecting specified pairs, the joint incidence on vertices $Q_1 \oplus Q_2$ and edges $E_1 \oplus E_2 \oplus B$ is:
\begin{itemize}
\item $(\operatorname{inl}(q_1), \operatorname{inl}(e_1))$: $\operatorname{incident}_1(q_1, e_1)$,
\item $(\operatorname{inr}(q_2), \operatorname{inr}(\operatorname{inl}(e_2)))$: $\operatorname{incident}_2(q_2, e_2)$,
\item $(\operatorname{inl}(q_1), \operatorname{inr}(\operatorname{inr}(b)))$: $q_1 = \operatorname{bridgeQ1}(b)$,
\item $(\operatorname{inr}(q_2), \operatorname{inr}(\operatorname{inr}(b)))$: $q_2 = \operatorname{bridgeQ2}(b)$,
\item otherwise: $\bot$.
\end{itemize}
\end{definition}

\begin{theorem}[Joint Gauss Product Measures Product]
\label{thm:CohenEtAlSchemeRecovery.joint_gauss_product_measures_product}
\lean{QEC1.CohenEtAlSchemeRecovery.joint_gauss_product_measures_product}
\leanok
\uses{def:CohenEtAlSchemeRecovery.jointIncident, def:HypergraphGeneralization.hyperGaussSubsetProduct}
Adding bridge edges enables joint measurement: the joint Gauss subset product for coefficient $c = (c_1, c_2)$ on $Q_1 \oplus Q_2$ gives a pure $X$-type operator that restricts to $c_1$ on $Q_1$ vertices and $c_2$ on $Q_2$ vertices. Formally, if $c = \operatorname{Sum.elim}(c_1, c_2)$, then:
\begin{enumerate}
\item For all $q_1 \in Q_1$: $(\operatorname{hyperGaussSubsetProduct}(\operatorname{jointIncident}, c)).\operatorname{xVec}(\operatorname{inl}(\operatorname{inl}(q_1))) = c_1(q_1)$.
\item For all $q_2 \in Q_2$: $(\operatorname{hyperGaussSubsetProduct}(\operatorname{jointIncident}, c)).\operatorname{xVec}(\operatorname{inl}(\operatorname{inr}(q_2))) = c_2(q_2)$.
\item $(\operatorname{hyperGaussSubsetProduct}(\operatorname{jointIncident}, c)).\operatorname{zVec} = 0$.
\end{enumerate}
\end{theorem}

\begin{proof}
\leanok
\uses{def:CohenEtAlSchemeRecovery.jointIncident, def:HypergraphGeneralization.hyperGaussSubsetProduct}
After substituting $c = \operatorname{Sum.elim}(c_1, c_2)$, the first two components follow by simplification using $\operatorname{hyperGaussSubsetProduct\_xVec\_vertex}$, and the third is $\operatorname{hyperGaussSubsetProduct\_zVec}$ applied to the joint incidence.
\end{proof}

